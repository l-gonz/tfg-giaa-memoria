\section{Literature review}
\begin{enumerate}

\item Ref~\cite{Mao2017345} Indoor Follow Me Drone
Tracking through acoustic signals in indoor environments produced by speakers on the drone and received by a mobile device. Discards CV for stability and processing power. Controller on mobile device with three degrees of freedom, pitch and yaw through MPC and roll through PID, with prediction of target's movement.
\begin{itemize}
\item Identification and path following control of an AR.Drone quadrotor \href{https://www.scopus.com/record/display.uri?eid=2-s2.0-84893212045&origin=reflist}{--Link--} \\ 
Path following application based on IMC position controllers, external webcam video stream.

\item Tracking a ground moving target with a quadrotor using switching control: Nonlinear modeling and control \href{https://www.scopus.com/record/display.uri?eid=2-s2.0-84871633622&origin=reflist}{--Link--} \\ 
Tracking of a moving target on ground, embedded camera, use of 2-dimensional images to compute the relative 3-dimensional position and translational velocity of the UAV with respect to the target, switching controllers.

\item A computer vision and control algorithm to follow a human target in a generic environment using a drone \href{https://www.scopus.com/record/display.uri?eid=2-s2.0-84978870802&origin=reflist}{--Link--} \\ 
Tracking and following a generic human target, using the HOG classifier, and on local brightness information, using the optical flow algorithm.

\item Framework for autonomous on-board navigation with the AR.Drone \href{https://www.scopus.com/record/display.uri?eid=2-s2.0-84899426060&origin=reflist}{--Link--} \\ 
All sensing and computations on-board, three systems to autonomously following several trajectory patterns, visually estimate its position and detecting and following a person.

\item A robust real-time embedded vision system on an unmanned rotorcraft for ground target following \href{https://www.scopus.com/record/display.uri?eid=2-s2.0-80054803865&origin=reflist}{--Link--} \\

\item UAV path following for constant line-of-sight \href{https://www.scopus.com/record/display.uri?eid=2-s2.0-85088181184&origin=reflist}{--Link--} \\ 
Flight path guidance and synchronous camera angles to observe a target, analytic expressions are derived for trajectories required for constant line-of-sight orientation relative to the aircraft.

\item Reactive control of autonomous drones \href{https://www.scopus.com/record/display.uri?eid=2-s2.0-84979920485&origin=reflist}{--Link--} \\ 
Reactive control that supersedes the time-triggered approach, control decisions are taken only upon recognizing the need to, based on observed changes in the navigation sensors, rate of execution dynamically adapts to the circumstances.

\item Correlation filter based visual trackers for person pursuit using a low-cost Quadrotor \href{https://www.scopus.com/record/display.uri?eid=2-s2.0-84954554999&origin=reflist}{--Link--} \\ 
Correlation filters for short-term tracking and a redetection strategy based on tracking-learning-detection (TLD), flight experiments in unconstrained environments using human targets and an existing visual servoing controller.

\item The Navigation and Control technology inside the AR.Drone micro UAV \href{https://www.scopus.com/record/display.uri?eid=2-s2.0-84863704626&origin=reflist}{--Link--} \\ 
Navigation and Control technology embedded in a recently commercialized micro Unmanned Aerial Vehicle (UAV), the AR.Drone.

\item Vector field path following for miniature air vehicles \href{https://www.scopus.com/record/display.uri?eid=2-s2.0-34447327237&origin=reflist}{--Link--} \\ 
Method for accurate path following for miniature air vehicles is developed, vector-field path-following control laws are developed for straight-line paths and circular arcs and orbits.

\item Trajectory tracking for autonomous vehicles: An integrated approach to guidance and control \href{https://www.scopus.com/record/display.uri?eid=2-s2.0-0031673631&origin=reflist}{--Link--} \\

\item Trajectory-tracking and path-following of underactuated autonomous vehicles with parametric modeling uncertainty \href{https://www.scopus.com/record/display.uri?eid=2-s2.0-34548237452&origin=reflist}{--Link--} \\

\item Combined trajectory tracking and path following: An application to the coordinated control of autonomous marine craft \href{https://www.scopus.com/record/display.uri?eid=2-s2.0-0035713115&origin=reflist}{--Link--} \\ 
Good trajectory tracking performance while keeping some of the desired properties normally associated with path following

\item Understanding the basis of the kalman filter via a simple and intuitive derivation [lecture notes] \href{https://www.scopus.com/record/display.uri?eid=2-s2.0-85032780920&origin=reflist}{--Link--} \\
\end{itemize}

\item Ref~\cite{Pestana20141886} Computer Vision Based General Object Following for GPS-denied Multirotor Unmanned Vehicles, Ref~\cite{Pestana2013} (6) Vision based GPS-denied Object Tracking and Following for Unmanned Aerial Vehicles
Object tracker + IBVS controllers

\item Ref~\cite{Chakrabarty201625} Autonomous Indoor Object Tracking with the Parrot AR.Drone
Deformable Object
Tracking (CMT) tracker with image-based visual servoing (IBVS), relies directly on image features to compute control values as opposed to position-based servoing

\item Ref~\cite{Bartak201635} Any Object Tracking and Following by a Flying Drone
 Offboard control solution though WiFi on Parrot AR Drone with computer vision (Training-Learning-Detection), with two PID controllers for forward/backward (scale estimation) and yaw (center position).

\end{enumerate}