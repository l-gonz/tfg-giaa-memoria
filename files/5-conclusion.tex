\chapter{Conclusions}
\label{chap:conclusion}

\todo[inline]{Write: short recap with conclusions}


\section{Evaluation of objectives}
\label{sec:consecucion-objetivos}

% Esta sección es la sección espejo de las dos primeras del capítulo de objetivos, donde se planteaba el objetivo general y se elaboraban los específicos.
% Es aquí donde hay que debatir qué se ha conseguido y qué no. 
% Cuando algo no se ha conseguido, se ha de justificar, en términos de qué problemas se han encontrado y qué medidas se han tomado para mitigar esos problemas.
\todo[inline]{Write: Go through objectives in introduction and debate what's been achieved}


\section{Lessons learned}
\label{sec:lessons-learned}

\todo[inline]{Write: Subsection 1 - cosas aprendidas del grado aplicadas al proyecto }
\todo[inline]{Write: Subsection 2 - cosas aprendidas del tfg en general }


\section{Future work}
\label{sec:fut-work}

\todo[inline]{Write: future work}

% Ningún proyecto ni software se termina, así que aquí vienen ideas y funcionalidades que estaría bien tener implementadas en el futuro.

% Es un apartado que sirve para dar ideas de cara a futuros TFGs/TFMs.