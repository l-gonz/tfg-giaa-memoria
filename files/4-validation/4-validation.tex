\chapter{Experiments and validation}
\label{chap:validation}

This chapter outlines the validation process for each essential component required for the proper functioning of the entire system, ranging from initial simulation tests to comprehensive flight tests conducted on the final guidance solution. The validation process follows the order depicted in Figure \ref{fig:validation-chart}.

The first section involves testing the integrity of the different components in the simulated environment. Initially, there is a component testing section where each part is tested individually. Afterwards comes the integration testing, where the SITL simulation of the PX4 flight controller is integrated with the AirSim environment and the DroneVisionControl offboard application to validate the exchange of control commands and images for the detection software.

In the second section, the simulation tools are employed to tune the PID controllers that drive the behaviour of the follow control solution. The purpose of the tuning process is to find good enough values for the control parameters to be able to validate the vision control solution and determine the limits of the simulation tools. During the process, first, the response of each of the two controllers will be analysed individually to select the appropriate gains. Afterwards, the controllers will be integrated together by applying the chosen values to the follow solution to ensure that they work in a satisfactory manner.

In the third section, once the software is proven to adhere to the established safety requirements, the simulation is moved to the HITL mode, where the software runs on dedicated hardware instead of on the simulation computer. This hardware includes the autopilot board Pixhawk 4 and the Raspberry Pi as its companion computer. The main goal is to verify that all the communication channels function as expected and that the devices deliver the required performance.

Lastly, in the fourth and final section, a series of flight tests are carried out, gradually increasing in complexity. These tests span from basic manual control of the vehicle using an RC controller to fully autonomous target-following flight executing in the complete system.

\begin{figure}
  \centering
  \makebox[\textwidth][c]{
  \includegraphics[width=1.2\textwidth, keepaspectratio]{img/validation-chart.jpg}}
  \caption{Summary of the validation process followed in this chapter.}\label{fig:validation-chart}
\end{figure}

\section{PX4 SITL simulation and validation}
\label{sec:test-2-sitl}

The software-in-the-loop simulation mode developed by PX4 is described in Section \ref{sec:devenv}. The advantage of this simulation method is that it enables testing and validating the correct operation of individual software components of the program's architecture and their correct integration into one control flow before adding further complexity with the dedicated hardware.

\subsection{Basic functionality tests with Gazebo}

To facilitate starting the tests with as few components as possible at a time and reduce the amount of configuration needed, the initial validations will be run with Gazebo\footnote{\url{https://gazebosim.org/home}} as the flight mechanics simulator. This simulator comes by default with the PX4's SITL installation. Gazebo works natively on Linux, so it can run in parallel with the simulated flight stack in SITL mode and the project's software on the same machine without having to be concerned about the networking between Windows and WSL. To set up a Linux machine for these tests, PX4 and DroneVisionControl need to be installed as detailed in Appendix \ref{app:install-dev-env}.

The initial tests conducted in this section will ensure that the foundational features of the control algorithms, sending commands to the autopilot and retrieving images for analysis, are reliable.
To do that, the sequence of steps will be as follows:
\begin{enumerate}
    \item Verify SITL simulation. The simulated flight controller (PX4) and the 3D program (Gazebo) connect to each other. Commands on the PX4 console are visible in Gazebo.
    \item Verify pilot module. DroneVisionControl connects to PX4 through the MAVSDK library, and commands are received by the flight controller.
    \item Verify video source module (\texttt{CameraSource}). Images from a camera are read into the program and displayed.
    \item Verify image detection by MediaPipe Hand. Target landmarks are identified on test images by the computer vision detection utility.
\end{enumerate}

%%%%%%%

\subsubsection{Verify SITL simulation}

The expected result is that building and starting the PX4 software in a console also starts the Gazebo program with a drone model that is flown by the flight controller executing on the console. This console can be used to send commands to the vehicle and set configuration parameters for the simulation.

Once the required software is installed, the simulation can be started with the \texttt{make\ px4\_sitl\ gazebo} command or using the \texttt{simulator.sh} script found on the project repository\footnote{\url{https://github.com/l-gonz/tfg-giaa-dronecontrol/blob/main/simulator.sh}}.
The result can be seen in Figure \ref{fig:gazebo}, with the user interface and 3D world of the Gazebo simulator on the left side and the PX4 console on the right side.


\begin{figure}[H]
  \centering
  \includegraphics[width=\textwidth, keepaspectratio]{img/gazebo.png}
  \caption{Gazebo simulator (left) and output from the PX4 console (right) after PX4's software-in-the-loop simulation is started.}
  \label{fig:gazebo}
\end{figure}

The first command to test is takeoff, which is done by sending \texttt{commander takeoff} through the PX4 console.
Figure \ref{fig:gazebo-takeoff} shows the simulator's state after the takeoff command, where the vehicle model has climbed to the default takeoff height of 2.5 meters above the ground.
The command to land the vehicle again is \texttt{commander land}.

\begin{figure}[H]
  \centering
  \includegraphics[width=\textwidth, keepaspectratio]{img/gazebo-takeoff.png}
  \caption{Gazebo simulator (left) and output from the PX4 console (right) after the takeoff command has been executed.}
  \label{fig:gazebo-takeoff}
\end{figure}


\subsubsection{Verify pilot module}

The second test will focus on the pilot module to verify whether the DroneVisionControl application can connect to the simulation and send flight commands. The expected result is that the connection is established successfully, and the keyboard inputs sent to the application are transformed into commands. These commands are interpreted by PX4 and shown visually in the 3D simulation in Gazebo.

The \texttt{test-camera} utility described in Section \ref{subsec:cam-tool} has been developed specifically to test different modules without engaging any of the program's control mechanisms. It can be started through the tools section of the application's command-line interface, using the \texttt{--sim} option to specify that the test target is the connection to the simulator (\texttt{dronevisioncontrol tools test-camera --sim}).
Once the connection to the simulation is established successfully, movement commands can be sent to the vehicle through keyboard input. To make this work, the program reads the input and maps it to a command in the pilot module, which is then queued until any previous commands are finished. When it is time to execute the command, the pilot module communicates it to the connected vehicle through the MAVSDK library. In the PX4 console, the logs should show that the command has been received before the vehicle model in Gazebo shows the effect visually.

For example, pressing the "T" key in the console executing the DroneVisionControl application will trigger takeoff. The result should be the same as sending the \texttt{commander takeoff} command through the PX4 console. This verifies that the MAVSDK library and the pilot module work as expected.

\subsubsection{Verify video source module - \texttt{CameraSource}}

The goal of this test is to verify that the video source module can retrieve and operate on images taken from a camera connected to the computer. This feature can be tested directly on a standalone Linux OS. However, if the PX4 simulation is running in WSL, as will be needed later for the complete simulation environment, it is necessary to change the configuration. The DroneVisionControl application must be executed from the Windows system instead, as WSL cannot access hardware devices or USB ports on the host computer. Appendix \ref{app:install-dev-env} contains the details on configuring the PX4 flight stack simulation to allow connecting to a MAVLink server through a different machine in the local network.

Once the application is installed in the Windows system, the same \texttt{test-camera} utility used before can be run with the \texttt{--camera} option to retrieve images from a connected camera. The expected result is that the program starts and a GUI window is drawn on the screen showing the live images taken from the connected camera. It is possible to save the images from the camera for later analysis in either photo or video format using the spacebar in the keyboard as the trigger. The '<' key changes the capture mode between photo and video.

\todo[inline]{Verify Windows to WSL connection}

\subsubsection{Verify image detection by MediaPipe Hand}

For the last test in this section, the goal is to verify the effectiveness of the image detection mechanisms on the images taken by the camera in the previous step. In this case, the \texttt{test-camera} tool can be used with the \texttt{-f/--file} option to use a video saved in the computer as the source for the \texttt{video-source} module. Additionally, the \texttt{-h/--hand-detection} option can be used to run the hand-detection algorithm provided by the MediaPipe library on the source images. The expected result of running these commands is that the application starts, and a window is displayed with the recorded video, with the landmarks detected in the image drawn over the joints of the hand in the correct positions.

Figure \ref{fig:sitl-hand} shows the image and text output of the program when the \texttt{test-camera} tool is run with the hand-detection feature activated.
On the left side, the detection algorithm tracks the shape of a hand detected in the image, and on the right side, the logged information shows the connection being established and keyboard commands being sent to the simulator.

\begin{figure}[H]
  \centering
  \includegraphics[width=\textwidth, keepaspectratio]{img/sitl-hand.png}
  \caption{Hand detection algorithm running on images taken from the computer's integrated webcam.}\label{fig:sitl-hand}
\end{figure}

After testing the flight stack, the default simulator and the pilot and image modules of the developed application, it is time to add the AirSim simulator to the environment.


\subsection{System integration tests with AirSim}
\label{sec:test-3-airsim}

The end goal for the development environment is to use the AirSim simulator to take advantage of its 3D-rendering and computer vision capabilities.
For this reason, it becomes necessary to validate that the new simulator can run correctly inside Unreal Engine, interacting with PX4 as the default Gazebo simulator did. Additionally, all the necessary detection, tracking and following features should work as expected.

To execute the subsequent tests, the AirSim simulator must be installed in the Windows host. Meanwhile, the PX4 flight controller and the DroneVisionControl application will run in a WSL subsystem as described in Figure \ref{fig:sitl-connections}. The complete installation process for this setup is described in Appendix \ref{app:install-airsim}.
There are specific configuration parameters that have to be set to be able to connect the AirSim simulator in Windows to the PX4 SITL simulation running inside WSL. On the simulator side, AirSim's settings file has to include a line defining the IP address of the network interface to use (the virtual WSL Ethernet adapter). This parameter can be found in Appendix \ref{app:airsim-config}, along with the complete configuration file used in AirSim for SITL testing. On the PX4 side, the flight controller must also be made aware of the network interface to listen to the simulator and started in a specific mode that sets it up to respond to AirSim's attempt to connect. Both of these points are taken care of behind the scenes when starting the PX4 console with the provided \texttt{simulator.sh}\footnote{\url{https://github.com/l-gonz/tfg-giaa-dronecontrol/blob/main/simulator.sh}} script with the \texttt{--airsim} option.


After the installation is complete, the necessary characteristics will be validated in the order below:
\begin{enumerate}
    \item Verify SITL simulation in Airsim. The simulator can start, connect to the PX4 SITL through the WSL virtual network and receive commands from the PX4 console.
    %\item Verify module in Airsim. Same as above, below.
    \item Verify integration with hand solution. The individual modules tested before are integrated together by running the hand-gesture control solution described in Section \ref{sec:hands}.
    \item Verify video source module (\texttt{SimulatorSource}). Images from the virtual camera in the simulator are read into DroneVisionControl and display AirSim's simulated world.
    \item Verify image detection by MediaPipe Pose. Target landmarks are identified on the 3D-model of a person in the simulator by the computer vision detection utility.
    \item Verify integration with follow solution. The follow solution can control the vehicle's velocity directly in PX4's offboard mode, reacting to the position of a detected person.
\end{enumerate}


\subsubsection{Verify SITL simulation in Airsim}

The objective of this test is to confirm that the Unreal Engine test environment containing the AirSim simulator can start and that the PX4 flight controller running in the console finds it and connects to it successfully. The expected result is that, after first pressing play on Unreal and then starting PX4 with the \texttt{simulator.sh --airsim} command, the drone model in AirSim can be controlled from the PX4 console in the same manner as with the Gazebo simulator.

Figure \ref{fig:airsim-sitl} shows the testing environment after the AirSim simulator and the PX4 console have been started successfully.
At this point, it is possible to use the PX4 console to send takeoff and land commands to the simulator and observe the 3D model of the vehicle climb into the air.

\begin{figure}
  \centering
  \includegraphics[width=\textwidth, keepaspectratio]{img/airsim-sitl.png}
  \caption{AirSim environment (right) connected to the PX4 console (left).}
  \label{fig:airsim-sitl}
\end{figure}


\subsubsection{Verify integration with hand solution}

In the second test, the goal is to integrate the individual modules tested in the previous steps: the pilot, the external camera and the hand recognition software. This will be achieved through the proof-of-concept hand control solution by running the DroneVisionControl application with the gesture-based control loop enabled. This mechanism is started with the \texttt{dronecontrol hand} command. The expected result after the command is executed is that the DroneVisionControl application connects to the PX4 console, and a window is displayed with the output from the external camera. When a hand is shown to the camera, the detection software draws the landmarks over the image. At that moment, the developed control software will attempt to interpret the gesture signalled with the hand and map it to its corresponding command for the vehicle, according to the list in Section \ref{sec:hands}.

The complete execution is shown in the video\footnote{\url{https://l-gonz.github.io/tfg-giaa-dronecontrol/videos/test-sitl-hand}} accessible through this \href{https://l-gonz.github.io/tfg-giaa-dronecontrol/videos/test-sitl-hand}{link}. Additionally, a frame extracted from the video can be observed in Figure \ref{fig:sitl-hand-video}.

\begin{figure}
  \centering
  \includegraphics[width=\textwidth, keepaspectratio]{img/video-hand-sitl.png}
  \caption{Single frame extracted from the video of the full execution of the hand-gesture control solution. Gesture detection is shown on the upper left side of the screen. The lower left side shows the DroneVisionControl console that logs the mapping between detected gestures and flight commands. The right side shows the vehicle's movement response inside the simulator.}
  \label{fig:sitl-hand-video}
\end{figure}

\subsubsection{Verify simulator video source and pose detection}

The validation process will now focus on the tools required to execute the pose detection and tracking mechanism. To assess the detection and tracking of human figures from images captured within the simulator, the camera testing tool provided with DroneVisionControl can be employed once again.

Figure \ref{fig:airsim-sitl-pose} demonstrates the output obtained when running the tool with a 3D model of a person positioned in front of the drone within the simulated environment. The following command was executed: 
% \texttt{dronevisioncontrol tools test-camera -{}-wsl -{}-sim -{}-pose-detection}
\begin{minted}{bash}
dronevisioncontrol tools test-camera --wsl --sim --pose-detection
\end{minted}

%%%%% THIS %%%%%
In the image, the computer vision utility successfully detects the key features of the human body, outlining them with a bounding box. Simultaneously, the program's logged output displays two calculated positions in the terminal: the x-coordinate of the midpoint within the bounding box and the percentage of the image height covered by the height of the bounding box.
These two values serve as inputs for the PID controllers that drive the follow solution, as described in Section \ref{sec:follow}. The logs can be employed to calibrate the distance at which the drone should track the person when the control mode is engaged. This is done by setting the simulated vehicle at the target distance from the person model and using the output height percentage as the set point for the forward PID controller.
%%%%%%%%%%%%%%%%%%

\begin{figure}[H]
  \centering
  \includegraphics[width=\textwidth, keepaspectratio]{img/airsim-sitl-pose.png}
  \caption{AirSim, PX4 and DroneVisionControl applications running side-by-side and connecting to each other.}
  \label{fig:airsim-sitl-pose}
\end{figure}


\subsubsection{Verify integration with follow solution}

Before using the testing environment to fine-tune the PID controllers in the follow solution based on the vehicle's response, it is essential to confirm that the controllers can appropriately react to changes in the figure's position. This can be achieved by enabling only the proportional term of the controllers with a suitably low magnitude, ensuring slow and smooth movement. The results show that the vehicle starts moving forward when the person moves backwards and turns to the right when the person moves to the right, mirroring the movement in the opposite directions.

To run the follow control program with specific values for the proportional terms of the yaw and forward controllers (10 and 2, respectively), the following command can be used:

\begin{minted}{bash}
dronevisioncontrol follow --sim --yaw-pid (10, 0, 0) --fwd-pid (2, 0, 0)
\end{minted}
\section{PID controller design}
\label{sec:test-1-pid}

% Process of tuning the PIDs
% Graphs from test-controller
% Analysis of error

The person-following mechanism relies on two PID controllers to obtain velocity outputs from the positions calculated by the image detection. These controllers need to be tuned to perform their function; that is, appropriate coefficients for the particular system need to be selected for $K_P$, $K_I$ and $K_D$. These coefficients will be chosen experimentally for the project. Even though, as discussed in \ref{subsec:pid-tools}, it is not possible to obtain the theoretical optical values with this method, it is still the simplest way of achieving a good enough behaviour from the closed-loop system to validate the follow control solution. For each of the parameters, several values will be tested. The objective is to find a balance between the more aggressive controller (larger gains) that leads to faster control and the more robust control of smaller gains.

The procedure will be as follows. First, the sign of the process gain is selected. For the controller to work as expected, a positive output results in an increase in the input. If the system works in the opposite way, the feedback will lead to unstable behaviour where the process variable grows exponentially. This behaviour can be fixed by inverting the sign of the output of the controller before it feeds back to the process.

The second step is selecting the control parameters defined in Equation \ref{eq:pid}. Initially, the controller will be operated as a pure P-controller, with the I-portion and the D-portion turned off to select the proportional gain ($K_D)$. To reach a good parameter, different values of $K_P$ are tested, starting with a low enough gain that the process variable changes slowly and gradually increasing until there is a clear overshoot that quickly subsides and before there starts to be a noticeable oscillation.

With a P-only controller, there is always a remaining control deviation so that the setpoint will never be hit exactly. To fix this remaining error, it is necessary to add an integral gain that compensates the error over time. In this step, the proportional gain will be set to the value already selected, and the integral gain will be increased gradually until the control deviation is compensated. As before, a too-aggressive setting will lead to unwanted oscillations.

Finally, to improve the controller further, a derivative gain can be added to dampen the initial overshoot in the process variable, following the same gradually increasing procedure. A good value to start is about a tenth of the integral gain \cite{pid-tuning}. 

%------------------------------------------------

\subsubsection{The testing environment}

The method outlined will be applied to the yaw and forward controllers independently by only enabling flight control in one direction at a time.
The custom tool developed for this project and described in Section \ref{subsec:pid-tools} will be used to facilitate testing many values for the controller parameters and comparing them. This tool allows selecting which controller will be enabled for testing (yaw or forward) and which values will be iterated for the parameters. To tune each gain independently, two of the parameters will be set with one fixed value, and the other will have several values that will be applied sequentially.

To obtain the step response of the controller under focus, the vehicle and the person model in the simulation environment are situated in an offset position from the reference position to which the controller is defined, i.e., the position at which the input at the controller matches its setpoint.
In the coordinate system of the simulation environment (AirSim), with the vehicle located at $x=0, y=0$ on the ground plane, the reference position for the person model is $x=600, y=0$. By situating the person at an offset position along either the y or x-axis, a step response according to the chosen parameters will be triggered on the yaw or forward controller, respectively.
Figure \ref{fig:tune-start-pos} shows the reference position for the controllers.

At the end of the process, the results are visualized in several graphs plotted by the program. The available graphs show the controller input over time and output over time, as well as the actual change in position and velocity measured by the autopilot telemetry over time. The exact telemetry variables depicted depend on the specific controller analysed. The yaw controller graphs focus on the heading in degrees and yaw speed. The forward controller graphs focus on the position along the forward axis in meters and the ground speed.
Due to the noise introduced by the image detection mechanisms, it is often more useful to tune the controllers by looking at the measured position than the controller input, as the internal autopilot controller helps to smooth the resulting curves.


%-------------------------------------------------

\begin{figure}[H]
  \centering
  \includegraphics[width=\textwidth, keepaspectratio]{img/pid/tune-ref-pos.jpg}
  \caption{Reference position for the yaw and forward PID controllers. From left to right, the panels show the DroneVisionControl application window, the AirSim simulator world view and the world location of the human model in the simulator. The distance between the vehicle and the person is 600 units in the x direction and 0 units in the y direction.}
  \label{fig:tune-start-pos}
\end{figure}


\subsection{Yaw controller}
The first controller to tune will be the yaw controller. As mentioned, the first step is to select the sign of the process gain. In this case, the process variable or input to the controller is the normalized position of the detected person in the horizontal axis of the camera field of view, with 0 being the person situated on the left edge of the field of view and 1 on the right edge. A positive output on the controller produces a positive yaw velocity, which causes the vehicle to rotate towards the right. In response, the person moves to the left in the camera field of view, resulting in a decreasing input to the controller. Since positive outputs should cause increasing inputs, the sign of the output velocity needs to be inverted to avoid exponentially growing behaviour.

Once the sign is selected and before any parameters are tested, the starting position needs to be decided. In the starting position, the target person model is offset from the reference position shown in Figure \ref{fig:tune-start-pos} to provoke a step response in the controller. For the yaw controller, the offset will be 100 units in the y-axis. With the vehicle situated in the origin in the simulation environment, the person model should be situated at $x=500, y=100$. Figure \ref{fig:tune-ref-pos-yaw} shows the starting position in the simulation.

\begin{figure}[H]
  \centering
  \includegraphics[width=\textwidth, keepaspectratio]{img/pid/tune-ref-pos-yaw.jpg}
  \caption{Starting position of the simulator for tuning the yaw controller. The human model is situated 500 units forward and 100 units to the right of the vehicle model.}
  \label{fig:tune-ref-pos-yaw}
\end{figure}

\subsubsection{Proportional component}

The proportional gain is the first parameter that needs to be tuned. The values chosen to test for $K_D$ range from 25 to 150 in steps of 25. To have a P-only controller during the test, the $K_I$ and $K_D$ components are set to 0. The results of the test are shown in Figure \ref{fig:tune-yaw-prop}.


\begin{figure}[H]
    \begin{minipage}[t]{0.5\linewidth}
        \centering
        \scalebox{0.55}{%% Creator: Matplotlib, PGF backend
%%
%% To include the figure in your LaTeX document, write
%%   \input{<filename>.pgf}
%%
%% Make sure the required packages are loaded in your preamble
%%   \usepackage{pgf}
%%
%% Also ensure that all the required font packages are loaded; for instance,
%% the lmodern package is sometimes necessary when using math font.
%%   \usepackage{lmodern}
%%
%% Figures using additional raster images can only be included by \input if
%% they are in the same directory as the main LaTeX file. For loading figures
%% from other directories you can use the `import` package
%%   \usepackage{import}
%%
%% and then include the figures with
%%   \import{<path to file>}{<filename>.pgf}
%%
%% Matplotlib used the following preamble
%%   \usepackage{fontspec}
%%   \setmainfont{DejaVuSerif.ttf}[Path=\detokenize{/home/lgonz/tfg-aero/tfg-giaa-dronecontrol/venv/lib/python3.8/site-packages/matplotlib/mpl-data/fonts/ttf/}]
%%   \setsansfont{DejaVuSans.ttf}[Path=\detokenize{/home/lgonz/tfg-aero/tfg-giaa-dronecontrol/venv/lib/python3.8/site-packages/matplotlib/mpl-data/fonts/ttf/}]
%%   \setmonofont{DejaVuSansMono.ttf}[Path=\detokenize{/home/lgonz/tfg-aero/tfg-giaa-dronecontrol/venv/lib/python3.8/site-packages/matplotlib/mpl-data/fonts/ttf/}]
%%
\begingroup%
\makeatletter%
\begin{pgfpicture}%
\pgfpathrectangle{\pgfpointorigin}{\pgfqpoint{6.400000in}{4.800000in}}%
\pgfusepath{use as bounding box, clip}%
\begin{pgfscope}%
\pgfsetbuttcap%
\pgfsetmiterjoin%
\definecolor{currentfill}{rgb}{1.000000,1.000000,1.000000}%
\pgfsetfillcolor{currentfill}%
\pgfsetlinewidth{0.000000pt}%
\definecolor{currentstroke}{rgb}{1.000000,1.000000,1.000000}%
\pgfsetstrokecolor{currentstroke}%
\pgfsetdash{}{0pt}%
\pgfpathmoveto{\pgfqpoint{0.000000in}{0.000000in}}%
\pgfpathlineto{\pgfqpoint{6.400000in}{0.000000in}}%
\pgfpathlineto{\pgfqpoint{6.400000in}{4.800000in}}%
\pgfpathlineto{\pgfqpoint{0.000000in}{4.800000in}}%
\pgfpathlineto{\pgfqpoint{0.000000in}{0.000000in}}%
\pgfpathclose%
\pgfusepath{fill}%
\end{pgfscope}%
\begin{pgfscope}%
\pgfsetbuttcap%
\pgfsetmiterjoin%
\definecolor{currentfill}{rgb}{1.000000,1.000000,1.000000}%
\pgfsetfillcolor{currentfill}%
\pgfsetlinewidth{0.000000pt}%
\definecolor{currentstroke}{rgb}{0.000000,0.000000,0.000000}%
\pgfsetstrokecolor{currentstroke}%
\pgfsetstrokeopacity{0.000000}%
\pgfsetdash{}{0pt}%
\pgfpathmoveto{\pgfqpoint{0.800000in}{0.528000in}}%
\pgfpathlineto{\pgfqpoint{5.760000in}{0.528000in}}%
\pgfpathlineto{\pgfqpoint{5.760000in}{4.224000in}}%
\pgfpathlineto{\pgfqpoint{0.800000in}{4.224000in}}%
\pgfpathlineto{\pgfqpoint{0.800000in}{0.528000in}}%
\pgfpathclose%
\pgfusepath{fill}%
\end{pgfscope}%
\begin{pgfscope}%
\pgfpathrectangle{\pgfqpoint{0.800000in}{0.528000in}}{\pgfqpoint{4.960000in}{3.696000in}}%
\pgfusepath{clip}%
\pgfsetrectcap%
\pgfsetroundjoin%
\pgfsetlinewidth{0.803000pt}%
\definecolor{currentstroke}{rgb}{0.690196,0.690196,0.690196}%
\pgfsetstrokecolor{currentstroke}%
\pgfsetdash{}{0pt}%
\pgfpathmoveto{\pgfqpoint{1.025455in}{0.528000in}}%
\pgfpathlineto{\pgfqpoint{1.025455in}{4.224000in}}%
\pgfusepath{stroke}%
\end{pgfscope}%
\begin{pgfscope}%
\pgfsetbuttcap%
\pgfsetroundjoin%
\definecolor{currentfill}{rgb}{0.000000,0.000000,0.000000}%
\pgfsetfillcolor{currentfill}%
\pgfsetlinewidth{0.803000pt}%
\definecolor{currentstroke}{rgb}{0.000000,0.000000,0.000000}%
\pgfsetstrokecolor{currentstroke}%
\pgfsetdash{}{0pt}%
\pgfsys@defobject{currentmarker}{\pgfqpoint{0.000000in}{-0.048611in}}{\pgfqpoint{0.000000in}{0.000000in}}{%
\pgfpathmoveto{\pgfqpoint{0.000000in}{0.000000in}}%
\pgfpathlineto{\pgfqpoint{0.000000in}{-0.048611in}}%
\pgfusepath{stroke,fill}%
}%
\begin{pgfscope}%
\pgfsys@transformshift{1.025455in}{0.528000in}%
\pgfsys@useobject{currentmarker}{}%
\end{pgfscope}%
\end{pgfscope}%
\begin{pgfscope}%
\definecolor{textcolor}{rgb}{0.000000,0.000000,0.000000}%
\pgfsetstrokecolor{textcolor}%
\pgfsetfillcolor{textcolor}%
\pgftext[x=1.025455in,y=0.430778in,,top]{\color{textcolor}\sffamily\fontsize{10.000000}{12.000000}\selectfont 0.0}%
\end{pgfscope}%
\begin{pgfscope}%
\pgfpathrectangle{\pgfqpoint{0.800000in}{0.528000in}}{\pgfqpoint{4.960000in}{3.696000in}}%
\pgfusepath{clip}%
\pgfsetrectcap%
\pgfsetroundjoin%
\pgfsetlinewidth{0.803000pt}%
\definecolor{currentstroke}{rgb}{0.690196,0.690196,0.690196}%
\pgfsetstrokecolor{currentstroke}%
\pgfsetdash{}{0pt}%
\pgfpathmoveto{\pgfqpoint{1.583185in}{0.528000in}}%
\pgfpathlineto{\pgfqpoint{1.583185in}{4.224000in}}%
\pgfusepath{stroke}%
\end{pgfscope}%
\begin{pgfscope}%
\pgfsetbuttcap%
\pgfsetroundjoin%
\definecolor{currentfill}{rgb}{0.000000,0.000000,0.000000}%
\pgfsetfillcolor{currentfill}%
\pgfsetlinewidth{0.803000pt}%
\definecolor{currentstroke}{rgb}{0.000000,0.000000,0.000000}%
\pgfsetstrokecolor{currentstroke}%
\pgfsetdash{}{0pt}%
\pgfsys@defobject{currentmarker}{\pgfqpoint{0.000000in}{-0.048611in}}{\pgfqpoint{0.000000in}{0.000000in}}{%
\pgfpathmoveto{\pgfqpoint{0.000000in}{0.000000in}}%
\pgfpathlineto{\pgfqpoint{0.000000in}{-0.048611in}}%
\pgfusepath{stroke,fill}%
}%
\begin{pgfscope}%
\pgfsys@transformshift{1.583185in}{0.528000in}%
\pgfsys@useobject{currentmarker}{}%
\end{pgfscope}%
\end{pgfscope}%
\begin{pgfscope}%
\definecolor{textcolor}{rgb}{0.000000,0.000000,0.000000}%
\pgfsetstrokecolor{textcolor}%
\pgfsetfillcolor{textcolor}%
\pgftext[x=1.583185in,y=0.430778in,,top]{\color{textcolor}\sffamily\fontsize{10.000000}{12.000000}\selectfont 2.5}%
\end{pgfscope}%
\begin{pgfscope}%
\pgfpathrectangle{\pgfqpoint{0.800000in}{0.528000in}}{\pgfqpoint{4.960000in}{3.696000in}}%
\pgfusepath{clip}%
\pgfsetrectcap%
\pgfsetroundjoin%
\pgfsetlinewidth{0.803000pt}%
\definecolor{currentstroke}{rgb}{0.690196,0.690196,0.690196}%
\pgfsetstrokecolor{currentstroke}%
\pgfsetdash{}{0pt}%
\pgfpathmoveto{\pgfqpoint{2.140916in}{0.528000in}}%
\pgfpathlineto{\pgfqpoint{2.140916in}{4.224000in}}%
\pgfusepath{stroke}%
\end{pgfscope}%
\begin{pgfscope}%
\pgfsetbuttcap%
\pgfsetroundjoin%
\definecolor{currentfill}{rgb}{0.000000,0.000000,0.000000}%
\pgfsetfillcolor{currentfill}%
\pgfsetlinewidth{0.803000pt}%
\definecolor{currentstroke}{rgb}{0.000000,0.000000,0.000000}%
\pgfsetstrokecolor{currentstroke}%
\pgfsetdash{}{0pt}%
\pgfsys@defobject{currentmarker}{\pgfqpoint{0.000000in}{-0.048611in}}{\pgfqpoint{0.000000in}{0.000000in}}{%
\pgfpathmoveto{\pgfqpoint{0.000000in}{0.000000in}}%
\pgfpathlineto{\pgfqpoint{0.000000in}{-0.048611in}}%
\pgfusepath{stroke,fill}%
}%
\begin{pgfscope}%
\pgfsys@transformshift{2.140916in}{0.528000in}%
\pgfsys@useobject{currentmarker}{}%
\end{pgfscope}%
\end{pgfscope}%
\begin{pgfscope}%
\definecolor{textcolor}{rgb}{0.000000,0.000000,0.000000}%
\pgfsetstrokecolor{textcolor}%
\pgfsetfillcolor{textcolor}%
\pgftext[x=2.140916in,y=0.430778in,,top]{\color{textcolor}\sffamily\fontsize{10.000000}{12.000000}\selectfont 5.0}%
\end{pgfscope}%
\begin{pgfscope}%
\pgfpathrectangle{\pgfqpoint{0.800000in}{0.528000in}}{\pgfqpoint{4.960000in}{3.696000in}}%
\pgfusepath{clip}%
\pgfsetrectcap%
\pgfsetroundjoin%
\pgfsetlinewidth{0.803000pt}%
\definecolor{currentstroke}{rgb}{0.690196,0.690196,0.690196}%
\pgfsetstrokecolor{currentstroke}%
\pgfsetdash{}{0pt}%
\pgfpathmoveto{\pgfqpoint{2.698647in}{0.528000in}}%
\pgfpathlineto{\pgfqpoint{2.698647in}{4.224000in}}%
\pgfusepath{stroke}%
\end{pgfscope}%
\begin{pgfscope}%
\pgfsetbuttcap%
\pgfsetroundjoin%
\definecolor{currentfill}{rgb}{0.000000,0.000000,0.000000}%
\pgfsetfillcolor{currentfill}%
\pgfsetlinewidth{0.803000pt}%
\definecolor{currentstroke}{rgb}{0.000000,0.000000,0.000000}%
\pgfsetstrokecolor{currentstroke}%
\pgfsetdash{}{0pt}%
\pgfsys@defobject{currentmarker}{\pgfqpoint{0.000000in}{-0.048611in}}{\pgfqpoint{0.000000in}{0.000000in}}{%
\pgfpathmoveto{\pgfqpoint{0.000000in}{0.000000in}}%
\pgfpathlineto{\pgfqpoint{0.000000in}{-0.048611in}}%
\pgfusepath{stroke,fill}%
}%
\begin{pgfscope}%
\pgfsys@transformshift{2.698647in}{0.528000in}%
\pgfsys@useobject{currentmarker}{}%
\end{pgfscope}%
\end{pgfscope}%
\begin{pgfscope}%
\definecolor{textcolor}{rgb}{0.000000,0.000000,0.000000}%
\pgfsetstrokecolor{textcolor}%
\pgfsetfillcolor{textcolor}%
\pgftext[x=2.698647in,y=0.430778in,,top]{\color{textcolor}\sffamily\fontsize{10.000000}{12.000000}\selectfont 7.5}%
\end{pgfscope}%
\begin{pgfscope}%
\pgfpathrectangle{\pgfqpoint{0.800000in}{0.528000in}}{\pgfqpoint{4.960000in}{3.696000in}}%
\pgfusepath{clip}%
\pgfsetrectcap%
\pgfsetroundjoin%
\pgfsetlinewidth{0.803000pt}%
\definecolor{currentstroke}{rgb}{0.690196,0.690196,0.690196}%
\pgfsetstrokecolor{currentstroke}%
\pgfsetdash{}{0pt}%
\pgfpathmoveto{\pgfqpoint{3.256378in}{0.528000in}}%
\pgfpathlineto{\pgfqpoint{3.256378in}{4.224000in}}%
\pgfusepath{stroke}%
\end{pgfscope}%
\begin{pgfscope}%
\pgfsetbuttcap%
\pgfsetroundjoin%
\definecolor{currentfill}{rgb}{0.000000,0.000000,0.000000}%
\pgfsetfillcolor{currentfill}%
\pgfsetlinewidth{0.803000pt}%
\definecolor{currentstroke}{rgb}{0.000000,0.000000,0.000000}%
\pgfsetstrokecolor{currentstroke}%
\pgfsetdash{}{0pt}%
\pgfsys@defobject{currentmarker}{\pgfqpoint{0.000000in}{-0.048611in}}{\pgfqpoint{0.000000in}{0.000000in}}{%
\pgfpathmoveto{\pgfqpoint{0.000000in}{0.000000in}}%
\pgfpathlineto{\pgfqpoint{0.000000in}{-0.048611in}}%
\pgfusepath{stroke,fill}%
}%
\begin{pgfscope}%
\pgfsys@transformshift{3.256378in}{0.528000in}%
\pgfsys@useobject{currentmarker}{}%
\end{pgfscope}%
\end{pgfscope}%
\begin{pgfscope}%
\definecolor{textcolor}{rgb}{0.000000,0.000000,0.000000}%
\pgfsetstrokecolor{textcolor}%
\pgfsetfillcolor{textcolor}%
\pgftext[x=3.256378in,y=0.430778in,,top]{\color{textcolor}\sffamily\fontsize{10.000000}{12.000000}\selectfont 10.0}%
\end{pgfscope}%
\begin{pgfscope}%
\pgfpathrectangle{\pgfqpoint{0.800000in}{0.528000in}}{\pgfqpoint{4.960000in}{3.696000in}}%
\pgfusepath{clip}%
\pgfsetrectcap%
\pgfsetroundjoin%
\pgfsetlinewidth{0.803000pt}%
\definecolor{currentstroke}{rgb}{0.690196,0.690196,0.690196}%
\pgfsetstrokecolor{currentstroke}%
\pgfsetdash{}{0pt}%
\pgfpathmoveto{\pgfqpoint{3.814109in}{0.528000in}}%
\pgfpathlineto{\pgfqpoint{3.814109in}{4.224000in}}%
\pgfusepath{stroke}%
\end{pgfscope}%
\begin{pgfscope}%
\pgfsetbuttcap%
\pgfsetroundjoin%
\definecolor{currentfill}{rgb}{0.000000,0.000000,0.000000}%
\pgfsetfillcolor{currentfill}%
\pgfsetlinewidth{0.803000pt}%
\definecolor{currentstroke}{rgb}{0.000000,0.000000,0.000000}%
\pgfsetstrokecolor{currentstroke}%
\pgfsetdash{}{0pt}%
\pgfsys@defobject{currentmarker}{\pgfqpoint{0.000000in}{-0.048611in}}{\pgfqpoint{0.000000in}{0.000000in}}{%
\pgfpathmoveto{\pgfqpoint{0.000000in}{0.000000in}}%
\pgfpathlineto{\pgfqpoint{0.000000in}{-0.048611in}}%
\pgfusepath{stroke,fill}%
}%
\begin{pgfscope}%
\pgfsys@transformshift{3.814109in}{0.528000in}%
\pgfsys@useobject{currentmarker}{}%
\end{pgfscope}%
\end{pgfscope}%
\begin{pgfscope}%
\definecolor{textcolor}{rgb}{0.000000,0.000000,0.000000}%
\pgfsetstrokecolor{textcolor}%
\pgfsetfillcolor{textcolor}%
\pgftext[x=3.814109in,y=0.430778in,,top]{\color{textcolor}\sffamily\fontsize{10.000000}{12.000000}\selectfont 12.5}%
\end{pgfscope}%
\begin{pgfscope}%
\pgfpathrectangle{\pgfqpoint{0.800000in}{0.528000in}}{\pgfqpoint{4.960000in}{3.696000in}}%
\pgfusepath{clip}%
\pgfsetrectcap%
\pgfsetroundjoin%
\pgfsetlinewidth{0.803000pt}%
\definecolor{currentstroke}{rgb}{0.690196,0.690196,0.690196}%
\pgfsetstrokecolor{currentstroke}%
\pgfsetdash{}{0pt}%
\pgfpathmoveto{\pgfqpoint{4.371840in}{0.528000in}}%
\pgfpathlineto{\pgfqpoint{4.371840in}{4.224000in}}%
\pgfusepath{stroke}%
\end{pgfscope}%
\begin{pgfscope}%
\pgfsetbuttcap%
\pgfsetroundjoin%
\definecolor{currentfill}{rgb}{0.000000,0.000000,0.000000}%
\pgfsetfillcolor{currentfill}%
\pgfsetlinewidth{0.803000pt}%
\definecolor{currentstroke}{rgb}{0.000000,0.000000,0.000000}%
\pgfsetstrokecolor{currentstroke}%
\pgfsetdash{}{0pt}%
\pgfsys@defobject{currentmarker}{\pgfqpoint{0.000000in}{-0.048611in}}{\pgfqpoint{0.000000in}{0.000000in}}{%
\pgfpathmoveto{\pgfqpoint{0.000000in}{0.000000in}}%
\pgfpathlineto{\pgfqpoint{0.000000in}{-0.048611in}}%
\pgfusepath{stroke,fill}%
}%
\begin{pgfscope}%
\pgfsys@transformshift{4.371840in}{0.528000in}%
\pgfsys@useobject{currentmarker}{}%
\end{pgfscope}%
\end{pgfscope}%
\begin{pgfscope}%
\definecolor{textcolor}{rgb}{0.000000,0.000000,0.000000}%
\pgfsetstrokecolor{textcolor}%
\pgfsetfillcolor{textcolor}%
\pgftext[x=4.371840in,y=0.430778in,,top]{\color{textcolor}\sffamily\fontsize{10.000000}{12.000000}\selectfont 15.0}%
\end{pgfscope}%
\begin{pgfscope}%
\pgfpathrectangle{\pgfqpoint{0.800000in}{0.528000in}}{\pgfqpoint{4.960000in}{3.696000in}}%
\pgfusepath{clip}%
\pgfsetrectcap%
\pgfsetroundjoin%
\pgfsetlinewidth{0.803000pt}%
\definecolor{currentstroke}{rgb}{0.690196,0.690196,0.690196}%
\pgfsetstrokecolor{currentstroke}%
\pgfsetdash{}{0pt}%
\pgfpathmoveto{\pgfqpoint{4.929570in}{0.528000in}}%
\pgfpathlineto{\pgfqpoint{4.929570in}{4.224000in}}%
\pgfusepath{stroke}%
\end{pgfscope}%
\begin{pgfscope}%
\pgfsetbuttcap%
\pgfsetroundjoin%
\definecolor{currentfill}{rgb}{0.000000,0.000000,0.000000}%
\pgfsetfillcolor{currentfill}%
\pgfsetlinewidth{0.803000pt}%
\definecolor{currentstroke}{rgb}{0.000000,0.000000,0.000000}%
\pgfsetstrokecolor{currentstroke}%
\pgfsetdash{}{0pt}%
\pgfsys@defobject{currentmarker}{\pgfqpoint{0.000000in}{-0.048611in}}{\pgfqpoint{0.000000in}{0.000000in}}{%
\pgfpathmoveto{\pgfqpoint{0.000000in}{0.000000in}}%
\pgfpathlineto{\pgfqpoint{0.000000in}{-0.048611in}}%
\pgfusepath{stroke,fill}%
}%
\begin{pgfscope}%
\pgfsys@transformshift{4.929570in}{0.528000in}%
\pgfsys@useobject{currentmarker}{}%
\end{pgfscope}%
\end{pgfscope}%
\begin{pgfscope}%
\definecolor{textcolor}{rgb}{0.000000,0.000000,0.000000}%
\pgfsetstrokecolor{textcolor}%
\pgfsetfillcolor{textcolor}%
\pgftext[x=4.929570in,y=0.430778in,,top]{\color{textcolor}\sffamily\fontsize{10.000000}{12.000000}\selectfont 17.5}%
\end{pgfscope}%
\begin{pgfscope}%
\pgfpathrectangle{\pgfqpoint{0.800000in}{0.528000in}}{\pgfqpoint{4.960000in}{3.696000in}}%
\pgfusepath{clip}%
\pgfsetrectcap%
\pgfsetroundjoin%
\pgfsetlinewidth{0.803000pt}%
\definecolor{currentstroke}{rgb}{0.690196,0.690196,0.690196}%
\pgfsetstrokecolor{currentstroke}%
\pgfsetdash{}{0pt}%
\pgfpathmoveto{\pgfqpoint{5.487301in}{0.528000in}}%
\pgfpathlineto{\pgfqpoint{5.487301in}{4.224000in}}%
\pgfusepath{stroke}%
\end{pgfscope}%
\begin{pgfscope}%
\pgfsetbuttcap%
\pgfsetroundjoin%
\definecolor{currentfill}{rgb}{0.000000,0.000000,0.000000}%
\pgfsetfillcolor{currentfill}%
\pgfsetlinewidth{0.803000pt}%
\definecolor{currentstroke}{rgb}{0.000000,0.000000,0.000000}%
\pgfsetstrokecolor{currentstroke}%
\pgfsetdash{}{0pt}%
\pgfsys@defobject{currentmarker}{\pgfqpoint{0.000000in}{-0.048611in}}{\pgfqpoint{0.000000in}{0.000000in}}{%
\pgfpathmoveto{\pgfqpoint{0.000000in}{0.000000in}}%
\pgfpathlineto{\pgfqpoint{0.000000in}{-0.048611in}}%
\pgfusepath{stroke,fill}%
}%
\begin{pgfscope}%
\pgfsys@transformshift{5.487301in}{0.528000in}%
\pgfsys@useobject{currentmarker}{}%
\end{pgfscope}%
\end{pgfscope}%
\begin{pgfscope}%
\definecolor{textcolor}{rgb}{0.000000,0.000000,0.000000}%
\pgfsetstrokecolor{textcolor}%
\pgfsetfillcolor{textcolor}%
\pgftext[x=5.487301in,y=0.430778in,,top]{\color{textcolor}\sffamily\fontsize{10.000000}{12.000000}\selectfont 20.0}%
\end{pgfscope}%
\begin{pgfscope}%
\definecolor{textcolor}{rgb}{0.000000,0.000000,0.000000}%
\pgfsetstrokecolor{textcolor}%
\pgfsetfillcolor{textcolor}%
\pgftext[x=3.280000in,y=0.240809in,,top]{\color{textcolor}\sffamily\fontsize{10.000000}{12.000000}\selectfont time [s]}%
\end{pgfscope}%
\begin{pgfscope}%
\pgfpathrectangle{\pgfqpoint{0.800000in}{0.528000in}}{\pgfqpoint{4.960000in}{3.696000in}}%
\pgfusepath{clip}%
\pgfsetrectcap%
\pgfsetroundjoin%
\pgfsetlinewidth{0.803000pt}%
\definecolor{currentstroke}{rgb}{0.690196,0.690196,0.690196}%
\pgfsetstrokecolor{currentstroke}%
\pgfsetdash{}{0pt}%
\pgfpathmoveto{\pgfqpoint{0.800000in}{0.569608in}}%
\pgfpathlineto{\pgfqpoint{5.760000in}{0.569608in}}%
\pgfusepath{stroke}%
\end{pgfscope}%
\begin{pgfscope}%
\pgfsetbuttcap%
\pgfsetroundjoin%
\definecolor{currentfill}{rgb}{0.000000,0.000000,0.000000}%
\pgfsetfillcolor{currentfill}%
\pgfsetlinewidth{0.803000pt}%
\definecolor{currentstroke}{rgb}{0.000000,0.000000,0.000000}%
\pgfsetstrokecolor{currentstroke}%
\pgfsetdash{}{0pt}%
\pgfsys@defobject{currentmarker}{\pgfqpoint{-0.048611in}{0.000000in}}{\pgfqpoint{-0.000000in}{0.000000in}}{%
\pgfpathmoveto{\pgfqpoint{-0.000000in}{0.000000in}}%
\pgfpathlineto{\pgfqpoint{-0.048611in}{0.000000in}}%
\pgfusepath{stroke,fill}%
}%
\begin{pgfscope}%
\pgfsys@transformshift{0.800000in}{0.569608in}%
\pgfsys@useobject{currentmarker}{}%
\end{pgfscope}%
\end{pgfscope}%
\begin{pgfscope}%
\definecolor{textcolor}{rgb}{0.000000,0.000000,0.000000}%
\pgfsetstrokecolor{textcolor}%
\pgfsetfillcolor{textcolor}%
\pgftext[x=0.285508in, y=0.516846in, left, base]{\color{textcolor}\sffamily\fontsize{10.000000}{12.000000}\selectfont \ensuremath{-}0.12}%
\end{pgfscope}%
\begin{pgfscope}%
\pgfpathrectangle{\pgfqpoint{0.800000in}{0.528000in}}{\pgfqpoint{4.960000in}{3.696000in}}%
\pgfusepath{clip}%
\pgfsetrectcap%
\pgfsetroundjoin%
\pgfsetlinewidth{0.803000pt}%
\definecolor{currentstroke}{rgb}{0.690196,0.690196,0.690196}%
\pgfsetstrokecolor{currentstroke}%
\pgfsetdash{}{0pt}%
\pgfpathmoveto{\pgfqpoint{0.800000in}{1.085253in}}%
\pgfpathlineto{\pgfqpoint{5.760000in}{1.085253in}}%
\pgfusepath{stroke}%
\end{pgfscope}%
\begin{pgfscope}%
\pgfsetbuttcap%
\pgfsetroundjoin%
\definecolor{currentfill}{rgb}{0.000000,0.000000,0.000000}%
\pgfsetfillcolor{currentfill}%
\pgfsetlinewidth{0.803000pt}%
\definecolor{currentstroke}{rgb}{0.000000,0.000000,0.000000}%
\pgfsetstrokecolor{currentstroke}%
\pgfsetdash{}{0pt}%
\pgfsys@defobject{currentmarker}{\pgfqpoint{-0.048611in}{0.000000in}}{\pgfqpoint{-0.000000in}{0.000000in}}{%
\pgfpathmoveto{\pgfqpoint{-0.000000in}{0.000000in}}%
\pgfpathlineto{\pgfqpoint{-0.048611in}{0.000000in}}%
\pgfusepath{stroke,fill}%
}%
\begin{pgfscope}%
\pgfsys@transformshift{0.800000in}{1.085253in}%
\pgfsys@useobject{currentmarker}{}%
\end{pgfscope}%
\end{pgfscope}%
\begin{pgfscope}%
\definecolor{textcolor}{rgb}{0.000000,0.000000,0.000000}%
\pgfsetstrokecolor{textcolor}%
\pgfsetfillcolor{textcolor}%
\pgftext[x=0.285508in, y=1.032491in, left, base]{\color{textcolor}\sffamily\fontsize{10.000000}{12.000000}\selectfont \ensuremath{-}0.10}%
\end{pgfscope}%
\begin{pgfscope}%
\pgfpathrectangle{\pgfqpoint{0.800000in}{0.528000in}}{\pgfqpoint{4.960000in}{3.696000in}}%
\pgfusepath{clip}%
\pgfsetrectcap%
\pgfsetroundjoin%
\pgfsetlinewidth{0.803000pt}%
\definecolor{currentstroke}{rgb}{0.690196,0.690196,0.690196}%
\pgfsetstrokecolor{currentstroke}%
\pgfsetdash{}{0pt}%
\pgfpathmoveto{\pgfqpoint{0.800000in}{1.600897in}}%
\pgfpathlineto{\pgfqpoint{5.760000in}{1.600897in}}%
\pgfusepath{stroke}%
\end{pgfscope}%
\begin{pgfscope}%
\pgfsetbuttcap%
\pgfsetroundjoin%
\definecolor{currentfill}{rgb}{0.000000,0.000000,0.000000}%
\pgfsetfillcolor{currentfill}%
\pgfsetlinewidth{0.803000pt}%
\definecolor{currentstroke}{rgb}{0.000000,0.000000,0.000000}%
\pgfsetstrokecolor{currentstroke}%
\pgfsetdash{}{0pt}%
\pgfsys@defobject{currentmarker}{\pgfqpoint{-0.048611in}{0.000000in}}{\pgfqpoint{-0.000000in}{0.000000in}}{%
\pgfpathmoveto{\pgfqpoint{-0.000000in}{0.000000in}}%
\pgfpathlineto{\pgfqpoint{-0.048611in}{0.000000in}}%
\pgfusepath{stroke,fill}%
}%
\begin{pgfscope}%
\pgfsys@transformshift{0.800000in}{1.600897in}%
\pgfsys@useobject{currentmarker}{}%
\end{pgfscope}%
\end{pgfscope}%
\begin{pgfscope}%
\definecolor{textcolor}{rgb}{0.000000,0.000000,0.000000}%
\pgfsetstrokecolor{textcolor}%
\pgfsetfillcolor{textcolor}%
\pgftext[x=0.285508in, y=1.548136in, left, base]{\color{textcolor}\sffamily\fontsize{10.000000}{12.000000}\selectfont \ensuremath{-}0.08}%
\end{pgfscope}%
\begin{pgfscope}%
\pgfpathrectangle{\pgfqpoint{0.800000in}{0.528000in}}{\pgfqpoint{4.960000in}{3.696000in}}%
\pgfusepath{clip}%
\pgfsetrectcap%
\pgfsetroundjoin%
\pgfsetlinewidth{0.803000pt}%
\definecolor{currentstroke}{rgb}{0.690196,0.690196,0.690196}%
\pgfsetstrokecolor{currentstroke}%
\pgfsetdash{}{0pt}%
\pgfpathmoveto{\pgfqpoint{0.800000in}{2.116542in}}%
\pgfpathlineto{\pgfqpoint{5.760000in}{2.116542in}}%
\pgfusepath{stroke}%
\end{pgfscope}%
\begin{pgfscope}%
\pgfsetbuttcap%
\pgfsetroundjoin%
\definecolor{currentfill}{rgb}{0.000000,0.000000,0.000000}%
\pgfsetfillcolor{currentfill}%
\pgfsetlinewidth{0.803000pt}%
\definecolor{currentstroke}{rgb}{0.000000,0.000000,0.000000}%
\pgfsetstrokecolor{currentstroke}%
\pgfsetdash{}{0pt}%
\pgfsys@defobject{currentmarker}{\pgfqpoint{-0.048611in}{0.000000in}}{\pgfqpoint{-0.000000in}{0.000000in}}{%
\pgfpathmoveto{\pgfqpoint{-0.000000in}{0.000000in}}%
\pgfpathlineto{\pgfqpoint{-0.048611in}{0.000000in}}%
\pgfusepath{stroke,fill}%
}%
\begin{pgfscope}%
\pgfsys@transformshift{0.800000in}{2.116542in}%
\pgfsys@useobject{currentmarker}{}%
\end{pgfscope}%
\end{pgfscope}%
\begin{pgfscope}%
\definecolor{textcolor}{rgb}{0.000000,0.000000,0.000000}%
\pgfsetstrokecolor{textcolor}%
\pgfsetfillcolor{textcolor}%
\pgftext[x=0.285508in, y=2.063781in, left, base]{\color{textcolor}\sffamily\fontsize{10.000000}{12.000000}\selectfont \ensuremath{-}0.06}%
\end{pgfscope}%
\begin{pgfscope}%
\pgfpathrectangle{\pgfqpoint{0.800000in}{0.528000in}}{\pgfqpoint{4.960000in}{3.696000in}}%
\pgfusepath{clip}%
\pgfsetrectcap%
\pgfsetroundjoin%
\pgfsetlinewidth{0.803000pt}%
\definecolor{currentstroke}{rgb}{0.690196,0.690196,0.690196}%
\pgfsetstrokecolor{currentstroke}%
\pgfsetdash{}{0pt}%
\pgfpathmoveto{\pgfqpoint{0.800000in}{2.632187in}}%
\pgfpathlineto{\pgfqpoint{5.760000in}{2.632187in}}%
\pgfusepath{stroke}%
\end{pgfscope}%
\begin{pgfscope}%
\pgfsetbuttcap%
\pgfsetroundjoin%
\definecolor{currentfill}{rgb}{0.000000,0.000000,0.000000}%
\pgfsetfillcolor{currentfill}%
\pgfsetlinewidth{0.803000pt}%
\definecolor{currentstroke}{rgb}{0.000000,0.000000,0.000000}%
\pgfsetstrokecolor{currentstroke}%
\pgfsetdash{}{0pt}%
\pgfsys@defobject{currentmarker}{\pgfqpoint{-0.048611in}{0.000000in}}{\pgfqpoint{-0.000000in}{0.000000in}}{%
\pgfpathmoveto{\pgfqpoint{-0.000000in}{0.000000in}}%
\pgfpathlineto{\pgfqpoint{-0.048611in}{0.000000in}}%
\pgfusepath{stroke,fill}%
}%
\begin{pgfscope}%
\pgfsys@transformshift{0.800000in}{2.632187in}%
\pgfsys@useobject{currentmarker}{}%
\end{pgfscope}%
\end{pgfscope}%
\begin{pgfscope}%
\definecolor{textcolor}{rgb}{0.000000,0.000000,0.000000}%
\pgfsetstrokecolor{textcolor}%
\pgfsetfillcolor{textcolor}%
\pgftext[x=0.285508in, y=2.579426in, left, base]{\color{textcolor}\sffamily\fontsize{10.000000}{12.000000}\selectfont \ensuremath{-}0.04}%
\end{pgfscope}%
\begin{pgfscope}%
\pgfpathrectangle{\pgfqpoint{0.800000in}{0.528000in}}{\pgfqpoint{4.960000in}{3.696000in}}%
\pgfusepath{clip}%
\pgfsetrectcap%
\pgfsetroundjoin%
\pgfsetlinewidth{0.803000pt}%
\definecolor{currentstroke}{rgb}{0.690196,0.690196,0.690196}%
\pgfsetstrokecolor{currentstroke}%
\pgfsetdash{}{0pt}%
\pgfpathmoveto{\pgfqpoint{0.800000in}{3.147832in}}%
\pgfpathlineto{\pgfqpoint{5.760000in}{3.147832in}}%
\pgfusepath{stroke}%
\end{pgfscope}%
\begin{pgfscope}%
\pgfsetbuttcap%
\pgfsetroundjoin%
\definecolor{currentfill}{rgb}{0.000000,0.000000,0.000000}%
\pgfsetfillcolor{currentfill}%
\pgfsetlinewidth{0.803000pt}%
\definecolor{currentstroke}{rgb}{0.000000,0.000000,0.000000}%
\pgfsetstrokecolor{currentstroke}%
\pgfsetdash{}{0pt}%
\pgfsys@defobject{currentmarker}{\pgfqpoint{-0.048611in}{0.000000in}}{\pgfqpoint{-0.000000in}{0.000000in}}{%
\pgfpathmoveto{\pgfqpoint{-0.000000in}{0.000000in}}%
\pgfpathlineto{\pgfqpoint{-0.048611in}{0.000000in}}%
\pgfusepath{stroke,fill}%
}%
\begin{pgfscope}%
\pgfsys@transformshift{0.800000in}{3.147832in}%
\pgfsys@useobject{currentmarker}{}%
\end{pgfscope}%
\end{pgfscope}%
\begin{pgfscope}%
\definecolor{textcolor}{rgb}{0.000000,0.000000,0.000000}%
\pgfsetstrokecolor{textcolor}%
\pgfsetfillcolor{textcolor}%
\pgftext[x=0.285508in, y=3.095070in, left, base]{\color{textcolor}\sffamily\fontsize{10.000000}{12.000000}\selectfont \ensuremath{-}0.02}%
\end{pgfscope}%
\begin{pgfscope}%
\pgfpathrectangle{\pgfqpoint{0.800000in}{0.528000in}}{\pgfqpoint{4.960000in}{3.696000in}}%
\pgfusepath{clip}%
\pgfsetrectcap%
\pgfsetroundjoin%
\pgfsetlinewidth{0.803000pt}%
\definecolor{currentstroke}{rgb}{0.690196,0.690196,0.690196}%
\pgfsetstrokecolor{currentstroke}%
\pgfsetdash{}{0pt}%
\pgfpathmoveto{\pgfqpoint{0.800000in}{3.663477in}}%
\pgfpathlineto{\pgfqpoint{5.760000in}{3.663477in}}%
\pgfusepath{stroke}%
\end{pgfscope}%
\begin{pgfscope}%
\pgfsetbuttcap%
\pgfsetroundjoin%
\definecolor{currentfill}{rgb}{0.000000,0.000000,0.000000}%
\pgfsetfillcolor{currentfill}%
\pgfsetlinewidth{0.803000pt}%
\definecolor{currentstroke}{rgb}{0.000000,0.000000,0.000000}%
\pgfsetstrokecolor{currentstroke}%
\pgfsetdash{}{0pt}%
\pgfsys@defobject{currentmarker}{\pgfqpoint{-0.048611in}{0.000000in}}{\pgfqpoint{-0.000000in}{0.000000in}}{%
\pgfpathmoveto{\pgfqpoint{-0.000000in}{0.000000in}}%
\pgfpathlineto{\pgfqpoint{-0.048611in}{0.000000in}}%
\pgfusepath{stroke,fill}%
}%
\begin{pgfscope}%
\pgfsys@transformshift{0.800000in}{3.663477in}%
\pgfsys@useobject{currentmarker}{}%
\end{pgfscope}%
\end{pgfscope}%
\begin{pgfscope}%
\definecolor{textcolor}{rgb}{0.000000,0.000000,0.000000}%
\pgfsetstrokecolor{textcolor}%
\pgfsetfillcolor{textcolor}%
\pgftext[x=0.393533in, y=3.610715in, left, base]{\color{textcolor}\sffamily\fontsize{10.000000}{12.000000}\selectfont 0.00}%
\end{pgfscope}%
\begin{pgfscope}%
\pgfpathrectangle{\pgfqpoint{0.800000in}{0.528000in}}{\pgfqpoint{4.960000in}{3.696000in}}%
\pgfusepath{clip}%
\pgfsetrectcap%
\pgfsetroundjoin%
\pgfsetlinewidth{0.803000pt}%
\definecolor{currentstroke}{rgb}{0.690196,0.690196,0.690196}%
\pgfsetstrokecolor{currentstroke}%
\pgfsetdash{}{0pt}%
\pgfpathmoveto{\pgfqpoint{0.800000in}{4.179122in}}%
\pgfpathlineto{\pgfqpoint{5.760000in}{4.179122in}}%
\pgfusepath{stroke}%
\end{pgfscope}%
\begin{pgfscope}%
\pgfsetbuttcap%
\pgfsetroundjoin%
\definecolor{currentfill}{rgb}{0.000000,0.000000,0.000000}%
\pgfsetfillcolor{currentfill}%
\pgfsetlinewidth{0.803000pt}%
\definecolor{currentstroke}{rgb}{0.000000,0.000000,0.000000}%
\pgfsetstrokecolor{currentstroke}%
\pgfsetdash{}{0pt}%
\pgfsys@defobject{currentmarker}{\pgfqpoint{-0.048611in}{0.000000in}}{\pgfqpoint{-0.000000in}{0.000000in}}{%
\pgfpathmoveto{\pgfqpoint{-0.000000in}{0.000000in}}%
\pgfpathlineto{\pgfqpoint{-0.048611in}{0.000000in}}%
\pgfusepath{stroke,fill}%
}%
\begin{pgfscope}%
\pgfsys@transformshift{0.800000in}{4.179122in}%
\pgfsys@useobject{currentmarker}{}%
\end{pgfscope}%
\end{pgfscope}%
\begin{pgfscope}%
\definecolor{textcolor}{rgb}{0.000000,0.000000,0.000000}%
\pgfsetstrokecolor{textcolor}%
\pgfsetfillcolor{textcolor}%
\pgftext[x=0.393533in, y=4.126360in, left, base]{\color{textcolor}\sffamily\fontsize{10.000000}{12.000000}\selectfont 0.02}%
\end{pgfscope}%
\begin{pgfscope}%
\definecolor{textcolor}{rgb}{0.000000,0.000000,0.000000}%
\pgfsetstrokecolor{textcolor}%
\pgfsetfillcolor{textcolor}%
\pgftext[x=0.229952in,y=2.376000in,,bottom,rotate=90.000000]{\color{textcolor}\sffamily\fontsize{10.000000}{12.000000}\selectfont Computed error [-]}%
\end{pgfscope}%
\begin{pgfscope}%
\pgfpathrectangle{\pgfqpoint{0.800000in}{0.528000in}}{\pgfqpoint{4.960000in}{3.696000in}}%
\pgfusepath{clip}%
\pgfsetrectcap%
\pgfsetroundjoin%
\pgfsetlinewidth{1.505625pt}%
\definecolor{currentstroke}{rgb}{0.121569,0.466667,0.705882}%
\pgfsetstrokecolor{currentstroke}%
\pgfsetdash{}{0pt}%
\pgfpathmoveto{\pgfqpoint{1.025455in}{0.712081in}}%
\pgfpathlineto{\pgfqpoint{1.063776in}{0.696000in}}%
\pgfpathlineto{\pgfqpoint{1.138623in}{0.711314in}}%
\pgfpathlineto{\pgfqpoint{1.212806in}{0.842489in}}%
\pgfpathlineto{\pgfqpoint{1.286280in}{1.145714in}}%
\pgfpathlineto{\pgfqpoint{1.360571in}{1.432675in}}%
\pgfpathlineto{\pgfqpoint{1.434952in}{1.732022in}}%
\pgfpathlineto{\pgfqpoint{1.508998in}{1.885539in}}%
\pgfpathlineto{\pgfqpoint{1.584714in}{2.026093in}}%
\pgfpathlineto{\pgfqpoint{1.661929in}{2.188805in}}%
\pgfpathlineto{\pgfqpoint{1.735143in}{2.398230in}}%
\pgfpathlineto{\pgfqpoint{1.808687in}{2.513056in}}%
\pgfpathlineto{\pgfqpoint{1.882936in}{2.615015in}}%
\pgfpathlineto{\pgfqpoint{1.957436in}{2.722138in}}%
\pgfpathlineto{\pgfqpoint{2.032221in}{2.777356in}}%
\pgfpathlineto{\pgfqpoint{2.106580in}{2.859100in}}%
\pgfpathlineto{\pgfqpoint{2.181303in}{2.942273in}}%
\pgfpathlineto{\pgfqpoint{2.255350in}{2.980329in}}%
\pgfpathlineto{\pgfqpoint{2.330461in}{3.026141in}}%
\pgfpathlineto{\pgfqpoint{2.404739in}{3.104836in}}%
\pgfpathlineto{\pgfqpoint{2.479219in}{3.127947in}}%
\pgfpathlineto{\pgfqpoint{2.555322in}{3.157530in}}%
\pgfpathlineto{\pgfqpoint{2.629577in}{3.191829in}}%
\pgfpathlineto{\pgfqpoint{2.703086in}{3.226892in}}%
\pgfpathlineto{\pgfqpoint{2.777270in}{3.264032in}}%
\pgfpathlineto{\pgfqpoint{2.852756in}{3.305920in}}%
\pgfpathlineto{\pgfqpoint{2.927602in}{3.347435in}}%
\pgfpathlineto{\pgfqpoint{3.001582in}{3.470657in}}%
\pgfpathlineto{\pgfqpoint{3.076834in}{3.439839in}}%
\pgfpathlineto{\pgfqpoint{3.150951in}{3.438114in}}%
\pgfpathlineto{\pgfqpoint{3.227261in}{3.451766in}}%
\pgfpathlineto{\pgfqpoint{3.301360in}{3.527610in}}%
\pgfpathlineto{\pgfqpoint{3.375878in}{3.537456in}}%
\pgfpathlineto{\pgfqpoint{3.452246in}{3.572963in}}%
\pgfpathlineto{\pgfqpoint{3.527990in}{3.568798in}}%
\pgfpathlineto{\pgfqpoint{3.599038in}{3.585830in}}%
\pgfpathlineto{\pgfqpoint{3.673314in}{3.600962in}}%
\pgfpathlineto{\pgfqpoint{3.747844in}{3.582396in}}%
\pgfpathlineto{\pgfqpoint{3.822123in}{3.621213in}}%
\pgfpathlineto{\pgfqpoint{3.896442in}{3.633919in}}%
\pgfpathlineto{\pgfqpoint{3.970902in}{3.650584in}}%
\pgfpathlineto{\pgfqpoint{4.045637in}{3.655765in}}%
\pgfpathlineto{\pgfqpoint{4.119856in}{3.655431in}}%
\pgfpathlineto{\pgfqpoint{4.193845in}{3.653380in}}%
\pgfpathlineto{\pgfqpoint{4.268656in}{3.626673in}}%
\pgfpathlineto{\pgfqpoint{4.345033in}{3.667087in}}%
\pgfpathlineto{\pgfqpoint{4.418046in}{3.665640in}}%
\pgfpathlineto{\pgfqpoint{4.492418in}{3.666472in}}%
\pgfpathlineto{\pgfqpoint{4.566377in}{3.692297in}}%
\pgfpathlineto{\pgfqpoint{4.640637in}{3.686526in}}%
\pgfpathlineto{\pgfqpoint{4.715036in}{3.681462in}}%
\pgfpathlineto{\pgfqpoint{4.789632in}{3.685527in}}%
\pgfpathlineto{\pgfqpoint{4.864110in}{3.690495in}}%
\pgfpathlineto{\pgfqpoint{4.938524in}{3.745055in}}%
\pgfpathlineto{\pgfqpoint{5.012461in}{3.724862in}}%
\pgfpathlineto{\pgfqpoint{5.086863in}{3.709051in}}%
\pgfpathlineto{\pgfqpoint{5.161604in}{3.740338in}}%
\pgfpathlineto{\pgfqpoint{5.237893in}{3.668447in}}%
\pgfpathlineto{\pgfqpoint{5.311107in}{3.632056in}}%
\pgfpathlineto{\pgfqpoint{5.386056in}{3.632079in}}%
\pgfpathlineto{\pgfqpoint{5.459939in}{3.762462in}}%
\pgfpathlineto{\pgfqpoint{5.534545in}{3.729021in}}%
\pgfusepath{stroke}%
\end{pgfscope}%
\begin{pgfscope}%
\pgfpathrectangle{\pgfqpoint{0.800000in}{0.528000in}}{\pgfqpoint{4.960000in}{3.696000in}}%
\pgfusepath{clip}%
\pgfsetrectcap%
\pgfsetroundjoin%
\pgfsetlinewidth{1.505625pt}%
\definecolor{currentstroke}{rgb}{1.000000,0.498039,0.054902}%
\pgfsetstrokecolor{currentstroke}%
\pgfsetdash{}{0pt}%
\pgfpathmoveto{\pgfqpoint{1.025455in}{0.750514in}}%
\pgfpathlineto{\pgfqpoint{1.099065in}{0.757257in}}%
\pgfpathlineto{\pgfqpoint{1.172846in}{0.987325in}}%
\pgfpathlineto{\pgfqpoint{1.246645in}{1.344846in}}%
\pgfpathlineto{\pgfqpoint{1.322501in}{1.874728in}}%
\pgfpathlineto{\pgfqpoint{1.396760in}{2.239682in}}%
\pgfpathlineto{\pgfqpoint{1.473513in}{2.589436in}}%
\pgfpathlineto{\pgfqpoint{1.546544in}{2.872297in}}%
\pgfpathlineto{\pgfqpoint{1.621097in}{3.148842in}}%
\pgfpathlineto{\pgfqpoint{1.696748in}{3.316592in}}%
\pgfpathlineto{\pgfqpoint{1.771019in}{3.388146in}}%
\pgfpathlineto{\pgfqpoint{1.848364in}{3.447828in}}%
\pgfpathlineto{\pgfqpoint{1.920266in}{3.485764in}}%
\pgfpathlineto{\pgfqpoint{1.994195in}{3.507296in}}%
\pgfpathlineto{\pgfqpoint{2.068179in}{3.527716in}}%
\pgfpathlineto{\pgfqpoint{2.142269in}{3.506451in}}%
\pgfpathlineto{\pgfqpoint{2.216486in}{3.530340in}}%
\pgfpathlineto{\pgfqpoint{2.291371in}{3.555824in}}%
\pgfpathlineto{\pgfqpoint{2.365491in}{3.553536in}}%
\pgfpathlineto{\pgfqpoint{2.439982in}{3.579730in}}%
\pgfpathlineto{\pgfqpoint{2.514002in}{3.554164in}}%
\pgfpathlineto{\pgfqpoint{2.589901in}{3.554021in}}%
\pgfpathlineto{\pgfqpoint{2.663464in}{3.563023in}}%
\pgfpathlineto{\pgfqpoint{2.737949in}{3.573612in}}%
\pgfpathlineto{\pgfqpoint{2.812354in}{3.609140in}}%
\pgfpathlineto{\pgfqpoint{2.886469in}{3.616459in}}%
\pgfpathlineto{\pgfqpoint{2.960992in}{3.633508in}}%
\pgfpathlineto{\pgfqpoint{3.035396in}{3.652534in}}%
\pgfpathlineto{\pgfqpoint{3.109941in}{3.674541in}}%
\pgfpathlineto{\pgfqpoint{3.185940in}{3.687335in}}%
\pgfpathlineto{\pgfqpoint{3.258333in}{3.691343in}}%
\pgfpathlineto{\pgfqpoint{3.334870in}{3.691110in}}%
\pgfpathlineto{\pgfqpoint{3.409005in}{3.687159in}}%
\pgfpathlineto{\pgfqpoint{3.483374in}{3.683740in}}%
\pgfpathlineto{\pgfqpoint{3.557709in}{3.680826in}}%
\pgfpathlineto{\pgfqpoint{3.635481in}{3.675678in}}%
\pgfpathlineto{\pgfqpoint{3.706173in}{3.679590in}}%
\pgfpathlineto{\pgfqpoint{3.780753in}{3.680221in}}%
\pgfpathlineto{\pgfqpoint{3.855154in}{3.680541in}}%
\pgfpathlineto{\pgfqpoint{3.929719in}{3.668267in}}%
\pgfpathlineto{\pgfqpoint{4.003795in}{3.680027in}}%
\pgfpathlineto{\pgfqpoint{4.078260in}{3.697320in}}%
\pgfpathlineto{\pgfqpoint{4.152395in}{3.627613in}}%
\pgfpathlineto{\pgfqpoint{4.228322in}{3.611994in}}%
\pgfpathlineto{\pgfqpoint{4.302367in}{3.614911in}}%
\pgfpathlineto{\pgfqpoint{4.375819in}{3.659697in}}%
\pgfpathlineto{\pgfqpoint{4.451286in}{3.652071in}}%
\pgfpathlineto{\pgfqpoint{4.525804in}{3.639211in}}%
\pgfpathlineto{\pgfqpoint{4.600850in}{3.638705in}}%
\pgfpathlineto{\pgfqpoint{4.675521in}{3.642074in}}%
\pgfpathlineto{\pgfqpoint{4.752222in}{3.646738in}}%
\pgfpathlineto{\pgfqpoint{4.828037in}{3.659036in}}%
\pgfpathlineto{\pgfqpoint{4.902395in}{3.669353in}}%
\pgfpathlineto{\pgfqpoint{4.976465in}{3.735929in}}%
\pgfpathlineto{\pgfqpoint{5.051645in}{3.730041in}}%
\pgfpathlineto{\pgfqpoint{5.125255in}{3.670775in}}%
\pgfpathlineto{\pgfqpoint{5.199740in}{3.584650in}}%
\pgfpathlineto{\pgfqpoint{5.273953in}{3.551455in}}%
\pgfpathlineto{\pgfqpoint{5.348063in}{3.551045in}}%
\pgfpathlineto{\pgfqpoint{5.422504in}{3.690645in}}%
\pgfpathlineto{\pgfqpoint{5.497107in}{3.645863in}}%
\pgfusepath{stroke}%
\end{pgfscope}%
\begin{pgfscope}%
\pgfpathrectangle{\pgfqpoint{0.800000in}{0.528000in}}{\pgfqpoint{4.960000in}{3.696000in}}%
\pgfusepath{clip}%
\pgfsetrectcap%
\pgfsetroundjoin%
\pgfsetlinewidth{1.505625pt}%
\definecolor{currentstroke}{rgb}{0.172549,0.627451,0.172549}%
\pgfsetstrokecolor{currentstroke}%
\pgfsetdash{}{0pt}%
\pgfpathmoveto{\pgfqpoint{1.025455in}{0.765793in}}%
\pgfpathlineto{\pgfqpoint{1.099911in}{0.778590in}}%
\pgfpathlineto{\pgfqpoint{1.175324in}{1.101696in}}%
\pgfpathlineto{\pgfqpoint{1.250759in}{1.485395in}}%
\pgfpathlineto{\pgfqpoint{1.324208in}{1.960597in}}%
\pgfpathlineto{\pgfqpoint{1.398758in}{2.483777in}}%
\pgfpathlineto{\pgfqpoint{1.473239in}{2.891969in}}%
\pgfpathlineto{\pgfqpoint{1.547019in}{3.273010in}}%
\pgfpathlineto{\pgfqpoint{1.621688in}{3.570346in}}%
\pgfpathlineto{\pgfqpoint{1.695783in}{3.737484in}}%
\pgfpathlineto{\pgfqpoint{1.770461in}{3.723029in}}%
\pgfpathlineto{\pgfqpoint{1.847841in}{3.758239in}}%
\pgfpathlineto{\pgfqpoint{1.918992in}{3.806586in}}%
\pgfpathlineto{\pgfqpoint{1.995729in}{3.754568in}}%
\pgfpathlineto{\pgfqpoint{2.071400in}{3.692826in}}%
\pgfpathlineto{\pgfqpoint{2.144749in}{3.643533in}}%
\pgfpathlineto{\pgfqpoint{2.218626in}{3.614652in}}%
\pgfpathlineto{\pgfqpoint{2.293223in}{3.637023in}}%
\pgfpathlineto{\pgfqpoint{2.367600in}{3.571304in}}%
\pgfpathlineto{\pgfqpoint{2.441953in}{3.675501in}}%
\pgfpathlineto{\pgfqpoint{2.516471in}{3.635010in}}%
\pgfpathlineto{\pgfqpoint{2.590935in}{3.634243in}}%
\pgfpathlineto{\pgfqpoint{2.665483in}{3.625403in}}%
\pgfpathlineto{\pgfqpoint{2.739264in}{3.615084in}}%
\pgfpathlineto{\pgfqpoint{2.814189in}{3.605215in}}%
\pgfpathlineto{\pgfqpoint{2.889825in}{3.622084in}}%
\pgfpathlineto{\pgfqpoint{2.963704in}{3.622582in}}%
\pgfpathlineto{\pgfqpoint{3.037334in}{3.622381in}}%
\pgfpathlineto{\pgfqpoint{3.112176in}{3.616297in}}%
\pgfpathlineto{\pgfqpoint{3.186516in}{3.611273in}}%
\pgfpathlineto{\pgfqpoint{3.260694in}{3.617859in}}%
\pgfpathlineto{\pgfqpoint{3.335034in}{3.668712in}}%
\pgfpathlineto{\pgfqpoint{3.409356in}{3.666525in}}%
\pgfpathlineto{\pgfqpoint{3.484633in}{3.658872in}}%
\pgfpathlineto{\pgfqpoint{3.558101in}{3.645881in}}%
\pgfpathlineto{\pgfqpoint{3.632383in}{3.643402in}}%
\pgfpathlineto{\pgfqpoint{3.707011in}{3.718062in}}%
\pgfpathlineto{\pgfqpoint{3.783326in}{3.724008in}}%
\pgfpathlineto{\pgfqpoint{3.856908in}{3.651340in}}%
\pgfpathlineto{\pgfqpoint{3.930715in}{3.624424in}}%
\pgfpathlineto{\pgfqpoint{4.004643in}{3.608829in}}%
\pgfpathlineto{\pgfqpoint{4.078590in}{3.604696in}}%
\pgfpathlineto{\pgfqpoint{4.154100in}{3.598036in}}%
\pgfpathlineto{\pgfqpoint{4.228692in}{3.617646in}}%
\pgfpathlineto{\pgfqpoint{4.302687in}{3.635894in}}%
\pgfpathlineto{\pgfqpoint{4.377390in}{3.639674in}}%
\pgfpathlineto{\pgfqpoint{4.451459in}{3.645503in}}%
\pgfpathlineto{\pgfqpoint{4.525943in}{3.653420in}}%
\pgfpathlineto{\pgfqpoint{4.601122in}{3.653522in}}%
\pgfpathlineto{\pgfqpoint{4.675193in}{3.655071in}}%
\pgfpathlineto{\pgfqpoint{4.750877in}{3.645047in}}%
\pgfpathlineto{\pgfqpoint{4.824357in}{3.650157in}}%
\pgfpathlineto{\pgfqpoint{4.898213in}{3.650645in}}%
\pgfpathlineto{\pgfqpoint{4.972646in}{3.598496in}}%
\pgfpathlineto{\pgfqpoint{5.047668in}{3.594487in}}%
\pgfpathlineto{\pgfqpoint{5.121956in}{3.607148in}}%
\pgfpathlineto{\pgfqpoint{5.196176in}{3.684920in}}%
\pgfpathlineto{\pgfqpoint{5.270506in}{3.672852in}}%
\pgfpathlineto{\pgfqpoint{5.345043in}{3.654725in}}%
\pgfpathlineto{\pgfqpoint{5.419263in}{3.645731in}}%
\pgfpathlineto{\pgfqpoint{5.493699in}{3.679618in}}%
\pgfusepath{stroke}%
\end{pgfscope}%
\begin{pgfscope}%
\pgfpathrectangle{\pgfqpoint{0.800000in}{0.528000in}}{\pgfqpoint{4.960000in}{3.696000in}}%
\pgfusepath{clip}%
\pgfsetrectcap%
\pgfsetroundjoin%
\pgfsetlinewidth{1.505625pt}%
\definecolor{currentstroke}{rgb}{0.839216,0.152941,0.156863}%
\pgfsetstrokecolor{currentstroke}%
\pgfsetdash{}{0pt}%
\pgfpathmoveto{\pgfqpoint{1.025455in}{0.764512in}}%
\pgfpathlineto{\pgfqpoint{1.099317in}{0.769619in}}%
\pgfpathlineto{\pgfqpoint{1.173519in}{1.044913in}}%
\pgfpathlineto{\pgfqpoint{1.247511in}{1.400453in}}%
\pgfpathlineto{\pgfqpoint{1.322201in}{1.894242in}}%
\pgfpathlineto{\pgfqpoint{1.396396in}{2.435620in}}%
\pgfpathlineto{\pgfqpoint{1.470917in}{2.917505in}}%
\pgfpathlineto{\pgfqpoint{1.545546in}{3.273112in}}%
\pgfpathlineto{\pgfqpoint{1.619703in}{3.789598in}}%
\pgfpathlineto{\pgfqpoint{1.694012in}{3.825462in}}%
\pgfpathlineto{\pgfqpoint{1.768885in}{3.975413in}}%
\pgfpathlineto{\pgfqpoint{1.842986in}{3.891544in}}%
\pgfpathlineto{\pgfqpoint{1.919062in}{3.808414in}}%
\pgfpathlineto{\pgfqpoint{1.992356in}{3.719060in}}%
\pgfpathlineto{\pgfqpoint{2.066206in}{3.630064in}}%
\pgfpathlineto{\pgfqpoint{2.140388in}{3.547560in}}%
\pgfpathlineto{\pgfqpoint{2.214802in}{3.524054in}}%
\pgfpathlineto{\pgfqpoint{2.288995in}{3.553333in}}%
\pgfpathlineto{\pgfqpoint{2.363309in}{3.578748in}}%
\pgfpathlineto{\pgfqpoint{2.437570in}{3.591594in}}%
\pgfpathlineto{\pgfqpoint{2.512228in}{3.616208in}}%
\pgfpathlineto{\pgfqpoint{2.586632in}{3.628265in}}%
\pgfpathlineto{\pgfqpoint{2.661688in}{3.657657in}}%
\pgfpathlineto{\pgfqpoint{2.735747in}{3.734210in}}%
\pgfpathlineto{\pgfqpoint{2.809657in}{3.729330in}}%
\pgfpathlineto{\pgfqpoint{2.883897in}{3.693667in}}%
\pgfpathlineto{\pgfqpoint{2.958353in}{3.680043in}}%
\pgfpathlineto{\pgfqpoint{3.032628in}{3.624015in}}%
\pgfpathlineto{\pgfqpoint{3.107069in}{3.627603in}}%
\pgfpathlineto{\pgfqpoint{3.181518in}{3.608076in}}%
\pgfpathlineto{\pgfqpoint{3.255803in}{3.598243in}}%
\pgfpathlineto{\pgfqpoint{3.330082in}{3.588613in}}%
\pgfpathlineto{\pgfqpoint{3.406248in}{3.582035in}}%
\pgfpathlineto{\pgfqpoint{3.482771in}{3.576513in}}%
\pgfpathlineto{\pgfqpoint{3.555301in}{3.596836in}}%
\pgfpathlineto{\pgfqpoint{3.629360in}{3.617336in}}%
\pgfpathlineto{\pgfqpoint{3.703565in}{3.636909in}}%
\pgfpathlineto{\pgfqpoint{3.778024in}{3.639804in}}%
\pgfpathlineto{\pgfqpoint{3.852388in}{3.675679in}}%
\pgfpathlineto{\pgfqpoint{3.927125in}{3.686831in}}%
\pgfpathlineto{\pgfqpoint{4.002276in}{3.697293in}}%
\pgfpathlineto{\pgfqpoint{4.076613in}{3.693312in}}%
\pgfpathlineto{\pgfqpoint{4.150560in}{3.687977in}}%
\pgfpathlineto{\pgfqpoint{4.225595in}{3.666941in}}%
\pgfpathlineto{\pgfqpoint{4.302408in}{3.683137in}}%
\pgfpathlineto{\pgfqpoint{4.375572in}{3.652386in}}%
\pgfpathlineto{\pgfqpoint{4.449438in}{3.748699in}}%
\pgfpathlineto{\pgfqpoint{4.523700in}{3.679239in}}%
\pgfpathlineto{\pgfqpoint{4.598169in}{3.617208in}}%
\pgfpathlineto{\pgfqpoint{4.672232in}{3.571684in}}%
\pgfpathlineto{\pgfqpoint{4.747518in}{3.575020in}}%
\pgfpathlineto{\pgfqpoint{4.822072in}{3.543237in}}%
\pgfpathlineto{\pgfqpoint{4.897366in}{3.539970in}}%
\pgfpathlineto{\pgfqpoint{4.971408in}{3.546933in}}%
\pgfpathlineto{\pgfqpoint{5.045575in}{3.583280in}}%
\pgfpathlineto{\pgfqpoint{5.120129in}{3.595381in}}%
\pgfpathlineto{\pgfqpoint{5.197079in}{3.614207in}}%
\pgfpathlineto{\pgfqpoint{5.270598in}{3.679506in}}%
\pgfpathlineto{\pgfqpoint{5.345079in}{3.697385in}}%
\pgfpathlineto{\pgfqpoint{5.419659in}{3.709422in}}%
\pgfpathlineto{\pgfqpoint{5.493677in}{3.706048in}}%
\pgfusepath{stroke}%
\end{pgfscope}%
\begin{pgfscope}%
\pgfpathrectangle{\pgfqpoint{0.800000in}{0.528000in}}{\pgfqpoint{4.960000in}{3.696000in}}%
\pgfusepath{clip}%
\pgfsetrectcap%
\pgfsetroundjoin%
\pgfsetlinewidth{1.505625pt}%
\definecolor{currentstroke}{rgb}{0.580392,0.403922,0.741176}%
\pgfsetstrokecolor{currentstroke}%
\pgfsetdash{}{0pt}%
\pgfpathmoveto{\pgfqpoint{1.025455in}{0.738618in}}%
\pgfpathlineto{\pgfqpoint{1.100458in}{0.792579in}}%
\pgfpathlineto{\pgfqpoint{1.176281in}{1.035238in}}%
\pgfpathlineto{\pgfqpoint{1.250820in}{1.480665in}}%
\pgfpathlineto{\pgfqpoint{1.324749in}{1.935529in}}%
\pgfpathlineto{\pgfqpoint{1.398416in}{2.435732in}}%
\pgfpathlineto{\pgfqpoint{1.472778in}{2.927914in}}%
\pgfpathlineto{\pgfqpoint{1.547787in}{3.331854in}}%
\pgfpathlineto{\pgfqpoint{1.622083in}{3.708665in}}%
\pgfpathlineto{\pgfqpoint{1.696655in}{3.964779in}}%
\pgfpathlineto{\pgfqpoint{1.771695in}{4.012352in}}%
\pgfpathlineto{\pgfqpoint{1.846473in}{4.056000in}}%
\pgfpathlineto{\pgfqpoint{1.921333in}{3.800301in}}%
\pgfpathlineto{\pgfqpoint{1.997868in}{3.684595in}}%
\pgfpathlineto{\pgfqpoint{2.071678in}{3.590967in}}%
\pgfpathlineto{\pgfqpoint{2.145699in}{3.519844in}}%
\pgfpathlineto{\pgfqpoint{2.220025in}{3.464517in}}%
\pgfpathlineto{\pgfqpoint{2.294077in}{3.457185in}}%
\pgfpathlineto{\pgfqpoint{2.370747in}{3.504670in}}%
\pgfpathlineto{\pgfqpoint{2.443124in}{3.580497in}}%
\pgfpathlineto{\pgfqpoint{2.517862in}{3.582568in}}%
\pgfpathlineto{\pgfqpoint{2.591929in}{3.630165in}}%
\pgfpathlineto{\pgfqpoint{2.666223in}{3.673735in}}%
\pgfpathlineto{\pgfqpoint{2.740918in}{3.773446in}}%
\pgfpathlineto{\pgfqpoint{2.816862in}{3.865754in}}%
\pgfpathlineto{\pgfqpoint{2.890082in}{3.816657in}}%
\pgfpathlineto{\pgfqpoint{2.964029in}{3.739903in}}%
\pgfpathlineto{\pgfqpoint{3.038293in}{3.644611in}}%
\pgfpathlineto{\pgfqpoint{3.112766in}{3.639869in}}%
\pgfpathlineto{\pgfqpoint{3.186957in}{3.542315in}}%
\pgfpathlineto{\pgfqpoint{3.261805in}{3.535236in}}%
\pgfpathlineto{\pgfqpoint{3.337022in}{3.530606in}}%
\pgfpathlineto{\pgfqpoint{3.411398in}{3.528227in}}%
\pgfpathlineto{\pgfqpoint{3.486552in}{3.557848in}}%
\pgfpathlineto{\pgfqpoint{3.560422in}{3.569954in}}%
\pgfpathlineto{\pgfqpoint{3.634412in}{3.681334in}}%
\pgfpathlineto{\pgfqpoint{3.710497in}{3.696935in}}%
\pgfpathlineto{\pgfqpoint{3.783965in}{3.722865in}}%
\pgfpathlineto{\pgfqpoint{3.857775in}{3.721750in}}%
\pgfpathlineto{\pgfqpoint{3.931643in}{3.719083in}}%
\pgfpathlineto{\pgfqpoint{4.006370in}{3.711775in}}%
\pgfpathlineto{\pgfqpoint{4.080629in}{3.711661in}}%
\pgfpathlineto{\pgfqpoint{4.155515in}{3.714135in}}%
\pgfpathlineto{\pgfqpoint{4.229449in}{3.615028in}}%
\pgfpathlineto{\pgfqpoint{4.304294in}{3.531718in}}%
\pgfpathlineto{\pgfqpoint{4.378164in}{3.534218in}}%
\pgfpathlineto{\pgfqpoint{4.454060in}{3.567497in}}%
\pgfpathlineto{\pgfqpoint{4.528649in}{3.587800in}}%
\pgfpathlineto{\pgfqpoint{4.605303in}{3.593102in}}%
\pgfpathlineto{\pgfqpoint{4.678615in}{3.627127in}}%
\pgfpathlineto{\pgfqpoint{4.752839in}{3.671951in}}%
\pgfpathlineto{\pgfqpoint{4.827129in}{3.717907in}}%
\pgfpathlineto{\pgfqpoint{4.902033in}{3.737871in}}%
\pgfpathlineto{\pgfqpoint{4.977876in}{3.868177in}}%
\pgfpathlineto{\pgfqpoint{5.051683in}{3.825276in}}%
\pgfpathlineto{\pgfqpoint{5.125830in}{3.746545in}}%
\pgfpathlineto{\pgfqpoint{5.200384in}{3.706173in}}%
\pgfpathlineto{\pgfqpoint{5.274235in}{3.608876in}}%
\pgfpathlineto{\pgfqpoint{5.348695in}{3.518185in}}%
\pgfpathlineto{\pgfqpoint{5.423195in}{3.532442in}}%
\pgfpathlineto{\pgfqpoint{5.499079in}{3.530912in}}%
\pgfusepath{stroke}%
\end{pgfscope}%
\begin{pgfscope}%
\pgfpathrectangle{\pgfqpoint{0.800000in}{0.528000in}}{\pgfqpoint{4.960000in}{3.696000in}}%
\pgfusepath{clip}%
\pgfsetrectcap%
\pgfsetroundjoin%
\pgfsetlinewidth{1.505625pt}%
\definecolor{currentstroke}{rgb}{0.549020,0.337255,0.294118}%
\pgfsetstrokecolor{currentstroke}%
\pgfsetdash{}{0pt}%
\pgfpathmoveto{\pgfqpoint{1.025455in}{0.742256in}}%
\pgfpathlineto{\pgfqpoint{1.099647in}{0.778325in}}%
\pgfpathlineto{\pgfqpoint{1.174084in}{1.100498in}}%
\pgfpathlineto{\pgfqpoint{1.248390in}{1.493512in}}%
\pgfpathlineto{\pgfqpoint{1.322841in}{2.017005in}}%
\pgfpathlineto{\pgfqpoint{1.397621in}{2.506135in}}%
\pgfpathlineto{\pgfqpoint{1.472134in}{2.916922in}}%
\pgfpathlineto{\pgfqpoint{1.546661in}{3.490019in}}%
\pgfpathlineto{\pgfqpoint{1.622031in}{3.799356in}}%
\pgfpathlineto{\pgfqpoint{1.696124in}{3.997648in}}%
\pgfpathlineto{\pgfqpoint{1.770359in}{3.988773in}}%
\pgfpathlineto{\pgfqpoint{1.844945in}{3.952996in}}%
\pgfpathlineto{\pgfqpoint{1.919056in}{3.788517in}}%
\pgfpathlineto{\pgfqpoint{1.993644in}{3.591868in}}%
\pgfpathlineto{\pgfqpoint{2.067688in}{3.459063in}}%
\pgfpathlineto{\pgfqpoint{2.142953in}{3.408651in}}%
\pgfpathlineto{\pgfqpoint{2.216798in}{3.441772in}}%
\pgfpathlineto{\pgfqpoint{2.290796in}{3.469168in}}%
\pgfpathlineto{\pgfqpoint{2.365613in}{3.532840in}}%
\pgfpathlineto{\pgfqpoint{2.442489in}{3.641766in}}%
\pgfpathlineto{\pgfqpoint{2.515846in}{3.745393in}}%
\pgfpathlineto{\pgfqpoint{2.589371in}{3.804057in}}%
\pgfpathlineto{\pgfqpoint{2.663477in}{3.789106in}}%
\pgfpathlineto{\pgfqpoint{2.738336in}{3.806464in}}%
\pgfpathlineto{\pgfqpoint{2.812747in}{3.757123in}}%
\pgfpathlineto{\pgfqpoint{2.887592in}{3.751660in}}%
\pgfpathlineto{\pgfqpoint{2.961704in}{3.673283in}}%
\pgfpathlineto{\pgfqpoint{3.036167in}{3.619228in}}%
\pgfpathlineto{\pgfqpoint{3.109768in}{3.530612in}}%
\pgfpathlineto{\pgfqpoint{3.184331in}{3.533370in}}%
\pgfpathlineto{\pgfqpoint{3.258978in}{3.545157in}}%
\pgfpathlineto{\pgfqpoint{3.335761in}{3.602010in}}%
\pgfpathlineto{\pgfqpoint{3.410308in}{3.651377in}}%
\pgfpathlineto{\pgfqpoint{3.483913in}{3.693571in}}%
\pgfpathlineto{\pgfqpoint{3.558745in}{3.735727in}}%
\pgfpathlineto{\pgfqpoint{3.633531in}{3.747311in}}%
\pgfpathlineto{\pgfqpoint{3.708051in}{3.751267in}}%
\pgfpathlineto{\pgfqpoint{3.782670in}{3.742350in}}%
\pgfpathlineto{\pgfqpoint{3.856790in}{3.690299in}}%
\pgfpathlineto{\pgfqpoint{3.931143in}{3.634825in}}%
\pgfpathlineto{\pgfqpoint{4.005704in}{3.562697in}}%
\pgfpathlineto{\pgfqpoint{4.080418in}{3.544429in}}%
\pgfpathlineto{\pgfqpoint{4.155350in}{3.582879in}}%
\pgfpathlineto{\pgfqpoint{4.232876in}{3.599450in}}%
\pgfpathlineto{\pgfqpoint{4.305808in}{3.653715in}}%
\pgfpathlineto{\pgfqpoint{4.379205in}{3.671428in}}%
\pgfpathlineto{\pgfqpoint{4.453262in}{3.677173in}}%
\pgfpathlineto{\pgfqpoint{4.528297in}{3.687170in}}%
\pgfpathlineto{\pgfqpoint{4.603125in}{3.706063in}}%
\pgfpathlineto{\pgfqpoint{4.677816in}{3.717514in}}%
\pgfpathlineto{\pgfqpoint{4.751560in}{3.702962in}}%
\pgfpathlineto{\pgfqpoint{4.826891in}{3.685078in}}%
\pgfpathlineto{\pgfqpoint{4.901387in}{3.671163in}}%
\pgfpathlineto{\pgfqpoint{4.975726in}{3.643229in}}%
\pgfpathlineto{\pgfqpoint{5.050456in}{3.633133in}}%
\pgfpathlineto{\pgfqpoint{5.124767in}{3.628443in}}%
\pgfpathlineto{\pgfqpoint{5.199653in}{3.628585in}}%
\pgfpathlineto{\pgfqpoint{5.274321in}{3.696190in}}%
\pgfpathlineto{\pgfqpoint{5.349479in}{3.707169in}}%
\pgfpathlineto{\pgfqpoint{5.424210in}{3.748363in}}%
\pgfpathlineto{\pgfqpoint{5.499007in}{3.724123in}}%
\pgfusepath{stroke}%
\end{pgfscope}%
\begin{pgfscope}%
\pgfsetrectcap%
\pgfsetmiterjoin%
\pgfsetlinewidth{0.803000pt}%
\definecolor{currentstroke}{rgb}{0.000000,0.000000,0.000000}%
\pgfsetstrokecolor{currentstroke}%
\pgfsetdash{}{0pt}%
\pgfpathmoveto{\pgfqpoint{0.800000in}{0.528000in}}%
\pgfpathlineto{\pgfqpoint{0.800000in}{4.224000in}}%
\pgfusepath{stroke}%
\end{pgfscope}%
\begin{pgfscope}%
\pgfsetrectcap%
\pgfsetmiterjoin%
\pgfsetlinewidth{0.803000pt}%
\definecolor{currentstroke}{rgb}{0.000000,0.000000,0.000000}%
\pgfsetstrokecolor{currentstroke}%
\pgfsetdash{}{0pt}%
\pgfpathmoveto{\pgfqpoint{5.760000in}{0.528000in}}%
\pgfpathlineto{\pgfqpoint{5.760000in}{4.224000in}}%
\pgfusepath{stroke}%
\end{pgfscope}%
\begin{pgfscope}%
\pgfsetrectcap%
\pgfsetmiterjoin%
\pgfsetlinewidth{0.803000pt}%
\definecolor{currentstroke}{rgb}{0.000000,0.000000,0.000000}%
\pgfsetstrokecolor{currentstroke}%
\pgfsetdash{}{0pt}%
\pgfpathmoveto{\pgfqpoint{0.800000in}{0.528000in}}%
\pgfpathlineto{\pgfqpoint{5.760000in}{0.528000in}}%
\pgfusepath{stroke}%
\end{pgfscope}%
\begin{pgfscope}%
\pgfsetrectcap%
\pgfsetmiterjoin%
\pgfsetlinewidth{0.803000pt}%
\definecolor{currentstroke}{rgb}{0.000000,0.000000,0.000000}%
\pgfsetstrokecolor{currentstroke}%
\pgfsetdash{}{0pt}%
\pgfpathmoveto{\pgfqpoint{0.800000in}{4.224000in}}%
\pgfpathlineto{\pgfqpoint{5.760000in}{4.224000in}}%
\pgfusepath{stroke}%
\end{pgfscope}%
\begin{pgfscope}%
\definecolor{textcolor}{rgb}{0.000000,0.000000,0.000000}%
\pgfsetstrokecolor{textcolor}%
\pgfsetfillcolor{textcolor}%
\pgftext[x=3.280000in,y=4.307333in,,base]{\color{textcolor}\sffamily\fontsize{12.000000}{14.400000}\selectfont Yaw controller input}%
\end{pgfscope}%
\begin{pgfscope}%
\pgfsetbuttcap%
\pgfsetmiterjoin%
\definecolor{currentfill}{rgb}{1.000000,1.000000,1.000000}%
\pgfsetfillcolor{currentfill}%
\pgfsetfillopacity{0.800000}%
\pgfsetlinewidth{1.003750pt}%
\definecolor{currentstroke}{rgb}{0.800000,0.800000,0.800000}%
\pgfsetstrokecolor{currentstroke}%
\pgfsetstrokeopacity{0.800000}%
\pgfsetdash{}{0pt}%
\pgfpathmoveto{\pgfqpoint{4.953237in}{0.597444in}}%
\pgfpathlineto{\pgfqpoint{5.662778in}{0.597444in}}%
\pgfpathquadraticcurveto{\pgfqpoint{5.690556in}{0.597444in}}{\pgfqpoint{5.690556in}{0.625222in}}%
\pgfpathlineto{\pgfqpoint{5.690556in}{1.834477in}}%
\pgfpathquadraticcurveto{\pgfqpoint{5.690556in}{1.862254in}}{\pgfqpoint{5.662778in}{1.862254in}}%
\pgfpathlineto{\pgfqpoint{4.953237in}{1.862254in}}%
\pgfpathquadraticcurveto{\pgfqpoint{4.925460in}{1.862254in}}{\pgfqpoint{4.925460in}{1.834477in}}%
\pgfpathlineto{\pgfqpoint{4.925460in}{0.625222in}}%
\pgfpathquadraticcurveto{\pgfqpoint{4.925460in}{0.597444in}}{\pgfqpoint{4.953237in}{0.597444in}}%
\pgfpathlineto{\pgfqpoint{4.953237in}{0.597444in}}%
\pgfpathclose%
\pgfusepath{stroke,fill}%
\end{pgfscope}%
\begin{pgfscope}%
\pgfsetrectcap%
\pgfsetroundjoin%
\pgfsetlinewidth{1.505625pt}%
\definecolor{currentstroke}{rgb}{0.121569,0.466667,0.705882}%
\pgfsetstrokecolor{currentstroke}%
\pgfsetdash{}{0pt}%
\pgfpathmoveto{\pgfqpoint{4.981015in}{1.749787in}}%
\pgfpathlineto{\pgfqpoint{5.119904in}{1.749787in}}%
\pgfpathlineto{\pgfqpoint{5.258793in}{1.749787in}}%
\pgfusepath{stroke}%
\end{pgfscope}%
\begin{pgfscope}%
\definecolor{textcolor}{rgb}{0.000000,0.000000,0.000000}%
\pgfsetstrokecolor{textcolor}%
\pgfsetfillcolor{textcolor}%
\pgftext[x=5.369904in,y=1.701176in,left,base]{\color{textcolor}\sffamily\fontsize{10.000000}{12.000000}\selectfont 25}%
\end{pgfscope}%
\begin{pgfscope}%
\pgfsetrectcap%
\pgfsetroundjoin%
\pgfsetlinewidth{1.505625pt}%
\definecolor{currentstroke}{rgb}{1.000000,0.498039,0.054902}%
\pgfsetstrokecolor{currentstroke}%
\pgfsetdash{}{0pt}%
\pgfpathmoveto{\pgfqpoint{4.981015in}{1.545930in}}%
\pgfpathlineto{\pgfqpoint{5.119904in}{1.545930in}}%
\pgfpathlineto{\pgfqpoint{5.258793in}{1.545930in}}%
\pgfusepath{stroke}%
\end{pgfscope}%
\begin{pgfscope}%
\definecolor{textcolor}{rgb}{0.000000,0.000000,0.000000}%
\pgfsetstrokecolor{textcolor}%
\pgfsetfillcolor{textcolor}%
\pgftext[x=5.369904in,y=1.497319in,left,base]{\color{textcolor}\sffamily\fontsize{10.000000}{12.000000}\selectfont 50}%
\end{pgfscope}%
\begin{pgfscope}%
\pgfsetrectcap%
\pgfsetroundjoin%
\pgfsetlinewidth{1.505625pt}%
\definecolor{currentstroke}{rgb}{0.172549,0.627451,0.172549}%
\pgfsetstrokecolor{currentstroke}%
\pgfsetdash{}{0pt}%
\pgfpathmoveto{\pgfqpoint{4.981015in}{1.342073in}}%
\pgfpathlineto{\pgfqpoint{5.119904in}{1.342073in}}%
\pgfpathlineto{\pgfqpoint{5.258793in}{1.342073in}}%
\pgfusepath{stroke}%
\end{pgfscope}%
\begin{pgfscope}%
\definecolor{textcolor}{rgb}{0.000000,0.000000,0.000000}%
\pgfsetstrokecolor{textcolor}%
\pgfsetfillcolor{textcolor}%
\pgftext[x=5.369904in,y=1.293461in,left,base]{\color{textcolor}\sffamily\fontsize{10.000000}{12.000000}\selectfont 75}%
\end{pgfscope}%
\begin{pgfscope}%
\pgfsetrectcap%
\pgfsetroundjoin%
\pgfsetlinewidth{1.505625pt}%
\definecolor{currentstroke}{rgb}{0.839216,0.152941,0.156863}%
\pgfsetstrokecolor{currentstroke}%
\pgfsetdash{}{0pt}%
\pgfpathmoveto{\pgfqpoint{4.981015in}{1.138215in}}%
\pgfpathlineto{\pgfqpoint{5.119904in}{1.138215in}}%
\pgfpathlineto{\pgfqpoint{5.258793in}{1.138215in}}%
\pgfusepath{stroke}%
\end{pgfscope}%
\begin{pgfscope}%
\definecolor{textcolor}{rgb}{0.000000,0.000000,0.000000}%
\pgfsetstrokecolor{textcolor}%
\pgfsetfillcolor{textcolor}%
\pgftext[x=5.369904in,y=1.089604in,left,base]{\color{textcolor}\sffamily\fontsize{10.000000}{12.000000}\selectfont 100}%
\end{pgfscope}%
\begin{pgfscope}%
\pgfsetrectcap%
\pgfsetroundjoin%
\pgfsetlinewidth{1.505625pt}%
\definecolor{currentstroke}{rgb}{0.580392,0.403922,0.741176}%
\pgfsetstrokecolor{currentstroke}%
\pgfsetdash{}{0pt}%
\pgfpathmoveto{\pgfqpoint{4.981015in}{0.934358in}}%
\pgfpathlineto{\pgfqpoint{5.119904in}{0.934358in}}%
\pgfpathlineto{\pgfqpoint{5.258793in}{0.934358in}}%
\pgfusepath{stroke}%
\end{pgfscope}%
\begin{pgfscope}%
\definecolor{textcolor}{rgb}{0.000000,0.000000,0.000000}%
\pgfsetstrokecolor{textcolor}%
\pgfsetfillcolor{textcolor}%
\pgftext[x=5.369904in,y=0.885747in,left,base]{\color{textcolor}\sffamily\fontsize{10.000000}{12.000000}\selectfont 125}%
\end{pgfscope}%
\begin{pgfscope}%
\pgfsetrectcap%
\pgfsetroundjoin%
\pgfsetlinewidth{1.505625pt}%
\definecolor{currentstroke}{rgb}{0.549020,0.337255,0.294118}%
\pgfsetstrokecolor{currentstroke}%
\pgfsetdash{}{0pt}%
\pgfpathmoveto{\pgfqpoint{4.981015in}{0.730501in}}%
\pgfpathlineto{\pgfqpoint{5.119904in}{0.730501in}}%
\pgfpathlineto{\pgfqpoint{5.258793in}{0.730501in}}%
\pgfusepath{stroke}%
\end{pgfscope}%
\begin{pgfscope}%
\definecolor{textcolor}{rgb}{0.000000,0.000000,0.000000}%
\pgfsetstrokecolor{textcolor}%
\pgfsetfillcolor{textcolor}%
\pgftext[x=5.369904in,y=0.681890in,left,base]{\color{textcolor}\sffamily\fontsize{10.000000}{12.000000}\selectfont 150}%
\end{pgfscope}%
\end{pgfpicture}%
\makeatother%
\endgroup%
}
    \end{minipage}
    \begin{minipage}[t]{0.5\linewidth}
        \centering
        \scalebox{0.55}{%% Creator: Matplotlib, PGF backend
%%
%% To include the figure in your LaTeX document, write
%%   \input{<filename>.pgf}
%%
%% Make sure the required packages are loaded in your preamble
%%   \usepackage{pgf}
%%
%% Also ensure that all the required font packages are loaded; for instance,
%% the lmodern package is sometimes necessary when using math font.
%%   \usepackage{lmodern}
%%
%% Figures using additional raster images can only be included by \input if
%% they are in the same directory as the main LaTeX file. For loading figures
%% from other directories you can use the `import` package
%%   \usepackage{import}
%%
%% and then include the figures with
%%   \import{<path to file>}{<filename>.pgf}
%%
%% Matplotlib used the following preamble
%%   \usepackage{fontspec}
%%   \setmainfont{DejaVuSerif.ttf}[Path=\detokenize{/home/lgonz/tfg-aero/tfg-giaa-dronecontrol/venv/lib/python3.8/site-packages/matplotlib/mpl-data/fonts/ttf/}]
%%   \setsansfont{DejaVuSans.ttf}[Path=\detokenize{/home/lgonz/tfg-aero/tfg-giaa-dronecontrol/venv/lib/python3.8/site-packages/matplotlib/mpl-data/fonts/ttf/}]
%%   \setmonofont{DejaVuSansMono.ttf}[Path=\detokenize{/home/lgonz/tfg-aero/tfg-giaa-dronecontrol/venv/lib/python3.8/site-packages/matplotlib/mpl-data/fonts/ttf/}]
%%
\begingroup%
\makeatletter%
\begin{pgfpicture}%
\pgfpathrectangle{\pgfpointorigin}{\pgfqpoint{6.400000in}{4.800000in}}%
\pgfusepath{use as bounding box, clip}%
\begin{pgfscope}%
\pgfsetbuttcap%
\pgfsetmiterjoin%
\definecolor{currentfill}{rgb}{1.000000,1.000000,1.000000}%
\pgfsetfillcolor{currentfill}%
\pgfsetlinewidth{0.000000pt}%
\definecolor{currentstroke}{rgb}{1.000000,1.000000,1.000000}%
\pgfsetstrokecolor{currentstroke}%
\pgfsetdash{}{0pt}%
\pgfpathmoveto{\pgfqpoint{0.000000in}{0.000000in}}%
\pgfpathlineto{\pgfqpoint{6.400000in}{0.000000in}}%
\pgfpathlineto{\pgfqpoint{6.400000in}{4.800000in}}%
\pgfpathlineto{\pgfqpoint{0.000000in}{4.800000in}}%
\pgfpathlineto{\pgfqpoint{0.000000in}{0.000000in}}%
\pgfpathclose%
\pgfusepath{fill}%
\end{pgfscope}%
\begin{pgfscope}%
\pgfsetbuttcap%
\pgfsetmiterjoin%
\definecolor{currentfill}{rgb}{1.000000,1.000000,1.000000}%
\pgfsetfillcolor{currentfill}%
\pgfsetlinewidth{0.000000pt}%
\definecolor{currentstroke}{rgb}{0.000000,0.000000,0.000000}%
\pgfsetstrokecolor{currentstroke}%
\pgfsetstrokeopacity{0.000000}%
\pgfsetdash{}{0pt}%
\pgfpathmoveto{\pgfqpoint{0.800000in}{0.528000in}}%
\pgfpathlineto{\pgfqpoint{5.760000in}{0.528000in}}%
\pgfpathlineto{\pgfqpoint{5.760000in}{4.224000in}}%
\pgfpathlineto{\pgfqpoint{0.800000in}{4.224000in}}%
\pgfpathlineto{\pgfqpoint{0.800000in}{0.528000in}}%
\pgfpathclose%
\pgfusepath{fill}%
\end{pgfscope}%
\begin{pgfscope}%
\pgfpathrectangle{\pgfqpoint{0.800000in}{0.528000in}}{\pgfqpoint{4.960000in}{3.696000in}}%
\pgfusepath{clip}%
\pgfsetrectcap%
\pgfsetroundjoin%
\pgfsetlinewidth{0.803000pt}%
\definecolor{currentstroke}{rgb}{0.690196,0.690196,0.690196}%
\pgfsetstrokecolor{currentstroke}%
\pgfsetdash{}{0pt}%
\pgfpathmoveto{\pgfqpoint{1.025455in}{0.528000in}}%
\pgfpathlineto{\pgfqpoint{1.025455in}{4.224000in}}%
\pgfusepath{stroke}%
\end{pgfscope}%
\begin{pgfscope}%
\pgfsetbuttcap%
\pgfsetroundjoin%
\definecolor{currentfill}{rgb}{0.000000,0.000000,0.000000}%
\pgfsetfillcolor{currentfill}%
\pgfsetlinewidth{0.803000pt}%
\definecolor{currentstroke}{rgb}{0.000000,0.000000,0.000000}%
\pgfsetstrokecolor{currentstroke}%
\pgfsetdash{}{0pt}%
\pgfsys@defobject{currentmarker}{\pgfqpoint{0.000000in}{-0.048611in}}{\pgfqpoint{0.000000in}{0.000000in}}{%
\pgfpathmoveto{\pgfqpoint{0.000000in}{0.000000in}}%
\pgfpathlineto{\pgfqpoint{0.000000in}{-0.048611in}}%
\pgfusepath{stroke,fill}%
}%
\begin{pgfscope}%
\pgfsys@transformshift{1.025455in}{0.528000in}%
\pgfsys@useobject{currentmarker}{}%
\end{pgfscope}%
\end{pgfscope}%
\begin{pgfscope}%
\definecolor{textcolor}{rgb}{0.000000,0.000000,0.000000}%
\pgfsetstrokecolor{textcolor}%
\pgfsetfillcolor{textcolor}%
\pgftext[x=1.025455in,y=0.430778in,,top]{\color{textcolor}\sffamily\fontsize{10.000000}{12.000000}\selectfont 0.0}%
\end{pgfscope}%
\begin{pgfscope}%
\pgfpathrectangle{\pgfqpoint{0.800000in}{0.528000in}}{\pgfqpoint{4.960000in}{3.696000in}}%
\pgfusepath{clip}%
\pgfsetrectcap%
\pgfsetroundjoin%
\pgfsetlinewidth{0.803000pt}%
\definecolor{currentstroke}{rgb}{0.690196,0.690196,0.690196}%
\pgfsetstrokecolor{currentstroke}%
\pgfsetdash{}{0pt}%
\pgfpathmoveto{\pgfqpoint{1.583185in}{0.528000in}}%
\pgfpathlineto{\pgfqpoint{1.583185in}{4.224000in}}%
\pgfusepath{stroke}%
\end{pgfscope}%
\begin{pgfscope}%
\pgfsetbuttcap%
\pgfsetroundjoin%
\definecolor{currentfill}{rgb}{0.000000,0.000000,0.000000}%
\pgfsetfillcolor{currentfill}%
\pgfsetlinewidth{0.803000pt}%
\definecolor{currentstroke}{rgb}{0.000000,0.000000,0.000000}%
\pgfsetstrokecolor{currentstroke}%
\pgfsetdash{}{0pt}%
\pgfsys@defobject{currentmarker}{\pgfqpoint{0.000000in}{-0.048611in}}{\pgfqpoint{0.000000in}{0.000000in}}{%
\pgfpathmoveto{\pgfqpoint{0.000000in}{0.000000in}}%
\pgfpathlineto{\pgfqpoint{0.000000in}{-0.048611in}}%
\pgfusepath{stroke,fill}%
}%
\begin{pgfscope}%
\pgfsys@transformshift{1.583185in}{0.528000in}%
\pgfsys@useobject{currentmarker}{}%
\end{pgfscope}%
\end{pgfscope}%
\begin{pgfscope}%
\definecolor{textcolor}{rgb}{0.000000,0.000000,0.000000}%
\pgfsetstrokecolor{textcolor}%
\pgfsetfillcolor{textcolor}%
\pgftext[x=1.583185in,y=0.430778in,,top]{\color{textcolor}\sffamily\fontsize{10.000000}{12.000000}\selectfont 2.5}%
\end{pgfscope}%
\begin{pgfscope}%
\pgfpathrectangle{\pgfqpoint{0.800000in}{0.528000in}}{\pgfqpoint{4.960000in}{3.696000in}}%
\pgfusepath{clip}%
\pgfsetrectcap%
\pgfsetroundjoin%
\pgfsetlinewidth{0.803000pt}%
\definecolor{currentstroke}{rgb}{0.690196,0.690196,0.690196}%
\pgfsetstrokecolor{currentstroke}%
\pgfsetdash{}{0pt}%
\pgfpathmoveto{\pgfqpoint{2.140916in}{0.528000in}}%
\pgfpathlineto{\pgfqpoint{2.140916in}{4.224000in}}%
\pgfusepath{stroke}%
\end{pgfscope}%
\begin{pgfscope}%
\pgfsetbuttcap%
\pgfsetroundjoin%
\definecolor{currentfill}{rgb}{0.000000,0.000000,0.000000}%
\pgfsetfillcolor{currentfill}%
\pgfsetlinewidth{0.803000pt}%
\definecolor{currentstroke}{rgb}{0.000000,0.000000,0.000000}%
\pgfsetstrokecolor{currentstroke}%
\pgfsetdash{}{0pt}%
\pgfsys@defobject{currentmarker}{\pgfqpoint{0.000000in}{-0.048611in}}{\pgfqpoint{0.000000in}{0.000000in}}{%
\pgfpathmoveto{\pgfqpoint{0.000000in}{0.000000in}}%
\pgfpathlineto{\pgfqpoint{0.000000in}{-0.048611in}}%
\pgfusepath{stroke,fill}%
}%
\begin{pgfscope}%
\pgfsys@transformshift{2.140916in}{0.528000in}%
\pgfsys@useobject{currentmarker}{}%
\end{pgfscope}%
\end{pgfscope}%
\begin{pgfscope}%
\definecolor{textcolor}{rgb}{0.000000,0.000000,0.000000}%
\pgfsetstrokecolor{textcolor}%
\pgfsetfillcolor{textcolor}%
\pgftext[x=2.140916in,y=0.430778in,,top]{\color{textcolor}\sffamily\fontsize{10.000000}{12.000000}\selectfont 5.0}%
\end{pgfscope}%
\begin{pgfscope}%
\pgfpathrectangle{\pgfqpoint{0.800000in}{0.528000in}}{\pgfqpoint{4.960000in}{3.696000in}}%
\pgfusepath{clip}%
\pgfsetrectcap%
\pgfsetroundjoin%
\pgfsetlinewidth{0.803000pt}%
\definecolor{currentstroke}{rgb}{0.690196,0.690196,0.690196}%
\pgfsetstrokecolor{currentstroke}%
\pgfsetdash{}{0pt}%
\pgfpathmoveto{\pgfqpoint{2.698647in}{0.528000in}}%
\pgfpathlineto{\pgfqpoint{2.698647in}{4.224000in}}%
\pgfusepath{stroke}%
\end{pgfscope}%
\begin{pgfscope}%
\pgfsetbuttcap%
\pgfsetroundjoin%
\definecolor{currentfill}{rgb}{0.000000,0.000000,0.000000}%
\pgfsetfillcolor{currentfill}%
\pgfsetlinewidth{0.803000pt}%
\definecolor{currentstroke}{rgb}{0.000000,0.000000,0.000000}%
\pgfsetstrokecolor{currentstroke}%
\pgfsetdash{}{0pt}%
\pgfsys@defobject{currentmarker}{\pgfqpoint{0.000000in}{-0.048611in}}{\pgfqpoint{0.000000in}{0.000000in}}{%
\pgfpathmoveto{\pgfqpoint{0.000000in}{0.000000in}}%
\pgfpathlineto{\pgfqpoint{0.000000in}{-0.048611in}}%
\pgfusepath{stroke,fill}%
}%
\begin{pgfscope}%
\pgfsys@transformshift{2.698647in}{0.528000in}%
\pgfsys@useobject{currentmarker}{}%
\end{pgfscope}%
\end{pgfscope}%
\begin{pgfscope}%
\definecolor{textcolor}{rgb}{0.000000,0.000000,0.000000}%
\pgfsetstrokecolor{textcolor}%
\pgfsetfillcolor{textcolor}%
\pgftext[x=2.698647in,y=0.430778in,,top]{\color{textcolor}\sffamily\fontsize{10.000000}{12.000000}\selectfont 7.5}%
\end{pgfscope}%
\begin{pgfscope}%
\pgfpathrectangle{\pgfqpoint{0.800000in}{0.528000in}}{\pgfqpoint{4.960000in}{3.696000in}}%
\pgfusepath{clip}%
\pgfsetrectcap%
\pgfsetroundjoin%
\pgfsetlinewidth{0.803000pt}%
\definecolor{currentstroke}{rgb}{0.690196,0.690196,0.690196}%
\pgfsetstrokecolor{currentstroke}%
\pgfsetdash{}{0pt}%
\pgfpathmoveto{\pgfqpoint{3.256378in}{0.528000in}}%
\pgfpathlineto{\pgfqpoint{3.256378in}{4.224000in}}%
\pgfusepath{stroke}%
\end{pgfscope}%
\begin{pgfscope}%
\pgfsetbuttcap%
\pgfsetroundjoin%
\definecolor{currentfill}{rgb}{0.000000,0.000000,0.000000}%
\pgfsetfillcolor{currentfill}%
\pgfsetlinewidth{0.803000pt}%
\definecolor{currentstroke}{rgb}{0.000000,0.000000,0.000000}%
\pgfsetstrokecolor{currentstroke}%
\pgfsetdash{}{0pt}%
\pgfsys@defobject{currentmarker}{\pgfqpoint{0.000000in}{-0.048611in}}{\pgfqpoint{0.000000in}{0.000000in}}{%
\pgfpathmoveto{\pgfqpoint{0.000000in}{0.000000in}}%
\pgfpathlineto{\pgfqpoint{0.000000in}{-0.048611in}}%
\pgfusepath{stroke,fill}%
}%
\begin{pgfscope}%
\pgfsys@transformshift{3.256378in}{0.528000in}%
\pgfsys@useobject{currentmarker}{}%
\end{pgfscope}%
\end{pgfscope}%
\begin{pgfscope}%
\definecolor{textcolor}{rgb}{0.000000,0.000000,0.000000}%
\pgfsetstrokecolor{textcolor}%
\pgfsetfillcolor{textcolor}%
\pgftext[x=3.256378in,y=0.430778in,,top]{\color{textcolor}\sffamily\fontsize{10.000000}{12.000000}\selectfont 10.0}%
\end{pgfscope}%
\begin{pgfscope}%
\pgfpathrectangle{\pgfqpoint{0.800000in}{0.528000in}}{\pgfqpoint{4.960000in}{3.696000in}}%
\pgfusepath{clip}%
\pgfsetrectcap%
\pgfsetroundjoin%
\pgfsetlinewidth{0.803000pt}%
\definecolor{currentstroke}{rgb}{0.690196,0.690196,0.690196}%
\pgfsetstrokecolor{currentstroke}%
\pgfsetdash{}{0pt}%
\pgfpathmoveto{\pgfqpoint{3.814109in}{0.528000in}}%
\pgfpathlineto{\pgfqpoint{3.814109in}{4.224000in}}%
\pgfusepath{stroke}%
\end{pgfscope}%
\begin{pgfscope}%
\pgfsetbuttcap%
\pgfsetroundjoin%
\definecolor{currentfill}{rgb}{0.000000,0.000000,0.000000}%
\pgfsetfillcolor{currentfill}%
\pgfsetlinewidth{0.803000pt}%
\definecolor{currentstroke}{rgb}{0.000000,0.000000,0.000000}%
\pgfsetstrokecolor{currentstroke}%
\pgfsetdash{}{0pt}%
\pgfsys@defobject{currentmarker}{\pgfqpoint{0.000000in}{-0.048611in}}{\pgfqpoint{0.000000in}{0.000000in}}{%
\pgfpathmoveto{\pgfqpoint{0.000000in}{0.000000in}}%
\pgfpathlineto{\pgfqpoint{0.000000in}{-0.048611in}}%
\pgfusepath{stroke,fill}%
}%
\begin{pgfscope}%
\pgfsys@transformshift{3.814109in}{0.528000in}%
\pgfsys@useobject{currentmarker}{}%
\end{pgfscope}%
\end{pgfscope}%
\begin{pgfscope}%
\definecolor{textcolor}{rgb}{0.000000,0.000000,0.000000}%
\pgfsetstrokecolor{textcolor}%
\pgfsetfillcolor{textcolor}%
\pgftext[x=3.814109in,y=0.430778in,,top]{\color{textcolor}\sffamily\fontsize{10.000000}{12.000000}\selectfont 12.5}%
\end{pgfscope}%
\begin{pgfscope}%
\pgfpathrectangle{\pgfqpoint{0.800000in}{0.528000in}}{\pgfqpoint{4.960000in}{3.696000in}}%
\pgfusepath{clip}%
\pgfsetrectcap%
\pgfsetroundjoin%
\pgfsetlinewidth{0.803000pt}%
\definecolor{currentstroke}{rgb}{0.690196,0.690196,0.690196}%
\pgfsetstrokecolor{currentstroke}%
\pgfsetdash{}{0pt}%
\pgfpathmoveto{\pgfqpoint{4.371840in}{0.528000in}}%
\pgfpathlineto{\pgfqpoint{4.371840in}{4.224000in}}%
\pgfusepath{stroke}%
\end{pgfscope}%
\begin{pgfscope}%
\pgfsetbuttcap%
\pgfsetroundjoin%
\definecolor{currentfill}{rgb}{0.000000,0.000000,0.000000}%
\pgfsetfillcolor{currentfill}%
\pgfsetlinewidth{0.803000pt}%
\definecolor{currentstroke}{rgb}{0.000000,0.000000,0.000000}%
\pgfsetstrokecolor{currentstroke}%
\pgfsetdash{}{0pt}%
\pgfsys@defobject{currentmarker}{\pgfqpoint{0.000000in}{-0.048611in}}{\pgfqpoint{0.000000in}{0.000000in}}{%
\pgfpathmoveto{\pgfqpoint{0.000000in}{0.000000in}}%
\pgfpathlineto{\pgfqpoint{0.000000in}{-0.048611in}}%
\pgfusepath{stroke,fill}%
}%
\begin{pgfscope}%
\pgfsys@transformshift{4.371840in}{0.528000in}%
\pgfsys@useobject{currentmarker}{}%
\end{pgfscope}%
\end{pgfscope}%
\begin{pgfscope}%
\definecolor{textcolor}{rgb}{0.000000,0.000000,0.000000}%
\pgfsetstrokecolor{textcolor}%
\pgfsetfillcolor{textcolor}%
\pgftext[x=4.371840in,y=0.430778in,,top]{\color{textcolor}\sffamily\fontsize{10.000000}{12.000000}\selectfont 15.0}%
\end{pgfscope}%
\begin{pgfscope}%
\pgfpathrectangle{\pgfqpoint{0.800000in}{0.528000in}}{\pgfqpoint{4.960000in}{3.696000in}}%
\pgfusepath{clip}%
\pgfsetrectcap%
\pgfsetroundjoin%
\pgfsetlinewidth{0.803000pt}%
\definecolor{currentstroke}{rgb}{0.690196,0.690196,0.690196}%
\pgfsetstrokecolor{currentstroke}%
\pgfsetdash{}{0pt}%
\pgfpathmoveto{\pgfqpoint{4.929570in}{0.528000in}}%
\pgfpathlineto{\pgfqpoint{4.929570in}{4.224000in}}%
\pgfusepath{stroke}%
\end{pgfscope}%
\begin{pgfscope}%
\pgfsetbuttcap%
\pgfsetroundjoin%
\definecolor{currentfill}{rgb}{0.000000,0.000000,0.000000}%
\pgfsetfillcolor{currentfill}%
\pgfsetlinewidth{0.803000pt}%
\definecolor{currentstroke}{rgb}{0.000000,0.000000,0.000000}%
\pgfsetstrokecolor{currentstroke}%
\pgfsetdash{}{0pt}%
\pgfsys@defobject{currentmarker}{\pgfqpoint{0.000000in}{-0.048611in}}{\pgfqpoint{0.000000in}{0.000000in}}{%
\pgfpathmoveto{\pgfqpoint{0.000000in}{0.000000in}}%
\pgfpathlineto{\pgfqpoint{0.000000in}{-0.048611in}}%
\pgfusepath{stroke,fill}%
}%
\begin{pgfscope}%
\pgfsys@transformshift{4.929570in}{0.528000in}%
\pgfsys@useobject{currentmarker}{}%
\end{pgfscope}%
\end{pgfscope}%
\begin{pgfscope}%
\definecolor{textcolor}{rgb}{0.000000,0.000000,0.000000}%
\pgfsetstrokecolor{textcolor}%
\pgfsetfillcolor{textcolor}%
\pgftext[x=4.929570in,y=0.430778in,,top]{\color{textcolor}\sffamily\fontsize{10.000000}{12.000000}\selectfont 17.5}%
\end{pgfscope}%
\begin{pgfscope}%
\pgfpathrectangle{\pgfqpoint{0.800000in}{0.528000in}}{\pgfqpoint{4.960000in}{3.696000in}}%
\pgfusepath{clip}%
\pgfsetrectcap%
\pgfsetroundjoin%
\pgfsetlinewidth{0.803000pt}%
\definecolor{currentstroke}{rgb}{0.690196,0.690196,0.690196}%
\pgfsetstrokecolor{currentstroke}%
\pgfsetdash{}{0pt}%
\pgfpathmoveto{\pgfqpoint{5.487301in}{0.528000in}}%
\pgfpathlineto{\pgfqpoint{5.487301in}{4.224000in}}%
\pgfusepath{stroke}%
\end{pgfscope}%
\begin{pgfscope}%
\pgfsetbuttcap%
\pgfsetroundjoin%
\definecolor{currentfill}{rgb}{0.000000,0.000000,0.000000}%
\pgfsetfillcolor{currentfill}%
\pgfsetlinewidth{0.803000pt}%
\definecolor{currentstroke}{rgb}{0.000000,0.000000,0.000000}%
\pgfsetstrokecolor{currentstroke}%
\pgfsetdash{}{0pt}%
\pgfsys@defobject{currentmarker}{\pgfqpoint{0.000000in}{-0.048611in}}{\pgfqpoint{0.000000in}{0.000000in}}{%
\pgfpathmoveto{\pgfqpoint{0.000000in}{0.000000in}}%
\pgfpathlineto{\pgfqpoint{0.000000in}{-0.048611in}}%
\pgfusepath{stroke,fill}%
}%
\begin{pgfscope}%
\pgfsys@transformshift{5.487301in}{0.528000in}%
\pgfsys@useobject{currentmarker}{}%
\end{pgfscope}%
\end{pgfscope}%
\begin{pgfscope}%
\definecolor{textcolor}{rgb}{0.000000,0.000000,0.000000}%
\pgfsetstrokecolor{textcolor}%
\pgfsetfillcolor{textcolor}%
\pgftext[x=5.487301in,y=0.430778in,,top]{\color{textcolor}\sffamily\fontsize{10.000000}{12.000000}\selectfont 20.0}%
\end{pgfscope}%
\begin{pgfscope}%
\definecolor{textcolor}{rgb}{0.000000,0.000000,0.000000}%
\pgfsetstrokecolor{textcolor}%
\pgfsetfillcolor{textcolor}%
\pgftext[x=3.280000in,y=0.240809in,,top]{\color{textcolor}\sffamily\fontsize{10.000000}{12.000000}\selectfont time [s]}%
\end{pgfscope}%
\begin{pgfscope}%
\pgfpathrectangle{\pgfqpoint{0.800000in}{0.528000in}}{\pgfqpoint{4.960000in}{3.696000in}}%
\pgfusepath{clip}%
\pgfsetrectcap%
\pgfsetroundjoin%
\pgfsetlinewidth{0.803000pt}%
\definecolor{currentstroke}{rgb}{0.690196,0.690196,0.690196}%
\pgfsetstrokecolor{currentstroke}%
\pgfsetdash{}{0pt}%
\pgfpathmoveto{\pgfqpoint{0.800000in}{0.668997in}}%
\pgfpathlineto{\pgfqpoint{5.760000in}{0.668997in}}%
\pgfusepath{stroke}%
\end{pgfscope}%
\begin{pgfscope}%
\pgfsetbuttcap%
\pgfsetroundjoin%
\definecolor{currentfill}{rgb}{0.000000,0.000000,0.000000}%
\pgfsetfillcolor{currentfill}%
\pgfsetlinewidth{0.803000pt}%
\definecolor{currentstroke}{rgb}{0.000000,0.000000,0.000000}%
\pgfsetstrokecolor{currentstroke}%
\pgfsetdash{}{0pt}%
\pgfsys@defobject{currentmarker}{\pgfqpoint{-0.048611in}{0.000000in}}{\pgfqpoint{-0.000000in}{0.000000in}}{%
\pgfpathmoveto{\pgfqpoint{-0.000000in}{0.000000in}}%
\pgfpathlineto{\pgfqpoint{-0.048611in}{0.000000in}}%
\pgfusepath{stroke,fill}%
}%
\begin{pgfscope}%
\pgfsys@transformshift{0.800000in}{0.668997in}%
\pgfsys@useobject{currentmarker}{}%
\end{pgfscope}%
\end{pgfscope}%
\begin{pgfscope}%
\definecolor{textcolor}{rgb}{0.000000,0.000000,0.000000}%
\pgfsetstrokecolor{textcolor}%
\pgfsetfillcolor{textcolor}%
\pgftext[x=0.506387in, y=0.616236in, left, base]{\color{textcolor}\sffamily\fontsize{10.000000}{12.000000}\selectfont \ensuremath{-}2}%
\end{pgfscope}%
\begin{pgfscope}%
\pgfpathrectangle{\pgfqpoint{0.800000in}{0.528000in}}{\pgfqpoint{4.960000in}{3.696000in}}%
\pgfusepath{clip}%
\pgfsetrectcap%
\pgfsetroundjoin%
\pgfsetlinewidth{0.803000pt}%
\definecolor{currentstroke}{rgb}{0.690196,0.690196,0.690196}%
\pgfsetstrokecolor{currentstroke}%
\pgfsetdash{}{0pt}%
\pgfpathmoveto{\pgfqpoint{0.800000in}{1.152855in}}%
\pgfpathlineto{\pgfqpoint{5.760000in}{1.152855in}}%
\pgfusepath{stroke}%
\end{pgfscope}%
\begin{pgfscope}%
\pgfsetbuttcap%
\pgfsetroundjoin%
\definecolor{currentfill}{rgb}{0.000000,0.000000,0.000000}%
\pgfsetfillcolor{currentfill}%
\pgfsetlinewidth{0.803000pt}%
\definecolor{currentstroke}{rgb}{0.000000,0.000000,0.000000}%
\pgfsetstrokecolor{currentstroke}%
\pgfsetdash{}{0pt}%
\pgfsys@defobject{currentmarker}{\pgfqpoint{-0.048611in}{0.000000in}}{\pgfqpoint{-0.000000in}{0.000000in}}{%
\pgfpathmoveto{\pgfqpoint{-0.000000in}{0.000000in}}%
\pgfpathlineto{\pgfqpoint{-0.048611in}{0.000000in}}%
\pgfusepath{stroke,fill}%
}%
\begin{pgfscope}%
\pgfsys@transformshift{0.800000in}{1.152855in}%
\pgfsys@useobject{currentmarker}{}%
\end{pgfscope}%
\end{pgfscope}%
\begin{pgfscope}%
\definecolor{textcolor}{rgb}{0.000000,0.000000,0.000000}%
\pgfsetstrokecolor{textcolor}%
\pgfsetfillcolor{textcolor}%
\pgftext[x=0.506387in, y=1.100093in, left, base]{\color{textcolor}\sffamily\fontsize{10.000000}{12.000000}\selectfont \ensuremath{-}1}%
\end{pgfscope}%
\begin{pgfscope}%
\pgfpathrectangle{\pgfqpoint{0.800000in}{0.528000in}}{\pgfqpoint{4.960000in}{3.696000in}}%
\pgfusepath{clip}%
\pgfsetrectcap%
\pgfsetroundjoin%
\pgfsetlinewidth{0.803000pt}%
\definecolor{currentstroke}{rgb}{0.690196,0.690196,0.690196}%
\pgfsetstrokecolor{currentstroke}%
\pgfsetdash{}{0pt}%
\pgfpathmoveto{\pgfqpoint{0.800000in}{1.636712in}}%
\pgfpathlineto{\pgfqpoint{5.760000in}{1.636712in}}%
\pgfusepath{stroke}%
\end{pgfscope}%
\begin{pgfscope}%
\pgfsetbuttcap%
\pgfsetroundjoin%
\definecolor{currentfill}{rgb}{0.000000,0.000000,0.000000}%
\pgfsetfillcolor{currentfill}%
\pgfsetlinewidth{0.803000pt}%
\definecolor{currentstroke}{rgb}{0.000000,0.000000,0.000000}%
\pgfsetstrokecolor{currentstroke}%
\pgfsetdash{}{0pt}%
\pgfsys@defobject{currentmarker}{\pgfqpoint{-0.048611in}{0.000000in}}{\pgfqpoint{-0.000000in}{0.000000in}}{%
\pgfpathmoveto{\pgfqpoint{-0.000000in}{0.000000in}}%
\pgfpathlineto{\pgfqpoint{-0.048611in}{0.000000in}}%
\pgfusepath{stroke,fill}%
}%
\begin{pgfscope}%
\pgfsys@transformshift{0.800000in}{1.636712in}%
\pgfsys@useobject{currentmarker}{}%
\end{pgfscope}%
\end{pgfscope}%
\begin{pgfscope}%
\definecolor{textcolor}{rgb}{0.000000,0.000000,0.000000}%
\pgfsetstrokecolor{textcolor}%
\pgfsetfillcolor{textcolor}%
\pgftext[x=0.614412in, y=1.583951in, left, base]{\color{textcolor}\sffamily\fontsize{10.000000}{12.000000}\selectfont 0}%
\end{pgfscope}%
\begin{pgfscope}%
\pgfpathrectangle{\pgfqpoint{0.800000in}{0.528000in}}{\pgfqpoint{4.960000in}{3.696000in}}%
\pgfusepath{clip}%
\pgfsetrectcap%
\pgfsetroundjoin%
\pgfsetlinewidth{0.803000pt}%
\definecolor{currentstroke}{rgb}{0.690196,0.690196,0.690196}%
\pgfsetstrokecolor{currentstroke}%
\pgfsetdash{}{0pt}%
\pgfpathmoveto{\pgfqpoint{0.800000in}{2.120570in}}%
\pgfpathlineto{\pgfqpoint{5.760000in}{2.120570in}}%
\pgfusepath{stroke}%
\end{pgfscope}%
\begin{pgfscope}%
\pgfsetbuttcap%
\pgfsetroundjoin%
\definecolor{currentfill}{rgb}{0.000000,0.000000,0.000000}%
\pgfsetfillcolor{currentfill}%
\pgfsetlinewidth{0.803000pt}%
\definecolor{currentstroke}{rgb}{0.000000,0.000000,0.000000}%
\pgfsetstrokecolor{currentstroke}%
\pgfsetdash{}{0pt}%
\pgfsys@defobject{currentmarker}{\pgfqpoint{-0.048611in}{0.000000in}}{\pgfqpoint{-0.000000in}{0.000000in}}{%
\pgfpathmoveto{\pgfqpoint{-0.000000in}{0.000000in}}%
\pgfpathlineto{\pgfqpoint{-0.048611in}{0.000000in}}%
\pgfusepath{stroke,fill}%
}%
\begin{pgfscope}%
\pgfsys@transformshift{0.800000in}{2.120570in}%
\pgfsys@useobject{currentmarker}{}%
\end{pgfscope}%
\end{pgfscope}%
\begin{pgfscope}%
\definecolor{textcolor}{rgb}{0.000000,0.000000,0.000000}%
\pgfsetstrokecolor{textcolor}%
\pgfsetfillcolor{textcolor}%
\pgftext[x=0.614412in, y=2.067808in, left, base]{\color{textcolor}\sffamily\fontsize{10.000000}{12.000000}\selectfont 1}%
\end{pgfscope}%
\begin{pgfscope}%
\pgfpathrectangle{\pgfqpoint{0.800000in}{0.528000in}}{\pgfqpoint{4.960000in}{3.696000in}}%
\pgfusepath{clip}%
\pgfsetrectcap%
\pgfsetroundjoin%
\pgfsetlinewidth{0.803000pt}%
\definecolor{currentstroke}{rgb}{0.690196,0.690196,0.690196}%
\pgfsetstrokecolor{currentstroke}%
\pgfsetdash{}{0pt}%
\pgfpathmoveto{\pgfqpoint{0.800000in}{2.604427in}}%
\pgfpathlineto{\pgfqpoint{5.760000in}{2.604427in}}%
\pgfusepath{stroke}%
\end{pgfscope}%
\begin{pgfscope}%
\pgfsetbuttcap%
\pgfsetroundjoin%
\definecolor{currentfill}{rgb}{0.000000,0.000000,0.000000}%
\pgfsetfillcolor{currentfill}%
\pgfsetlinewidth{0.803000pt}%
\definecolor{currentstroke}{rgb}{0.000000,0.000000,0.000000}%
\pgfsetstrokecolor{currentstroke}%
\pgfsetdash{}{0pt}%
\pgfsys@defobject{currentmarker}{\pgfqpoint{-0.048611in}{0.000000in}}{\pgfqpoint{-0.000000in}{0.000000in}}{%
\pgfpathmoveto{\pgfqpoint{-0.000000in}{0.000000in}}%
\pgfpathlineto{\pgfqpoint{-0.048611in}{0.000000in}}%
\pgfusepath{stroke,fill}%
}%
\begin{pgfscope}%
\pgfsys@transformshift{0.800000in}{2.604427in}%
\pgfsys@useobject{currentmarker}{}%
\end{pgfscope}%
\end{pgfscope}%
\begin{pgfscope}%
\definecolor{textcolor}{rgb}{0.000000,0.000000,0.000000}%
\pgfsetstrokecolor{textcolor}%
\pgfsetfillcolor{textcolor}%
\pgftext[x=0.614412in, y=2.551666in, left, base]{\color{textcolor}\sffamily\fontsize{10.000000}{12.000000}\selectfont 2}%
\end{pgfscope}%
\begin{pgfscope}%
\pgfpathrectangle{\pgfqpoint{0.800000in}{0.528000in}}{\pgfqpoint{4.960000in}{3.696000in}}%
\pgfusepath{clip}%
\pgfsetrectcap%
\pgfsetroundjoin%
\pgfsetlinewidth{0.803000pt}%
\definecolor{currentstroke}{rgb}{0.690196,0.690196,0.690196}%
\pgfsetstrokecolor{currentstroke}%
\pgfsetdash{}{0pt}%
\pgfpathmoveto{\pgfqpoint{0.800000in}{3.088285in}}%
\pgfpathlineto{\pgfqpoint{5.760000in}{3.088285in}}%
\pgfusepath{stroke}%
\end{pgfscope}%
\begin{pgfscope}%
\pgfsetbuttcap%
\pgfsetroundjoin%
\definecolor{currentfill}{rgb}{0.000000,0.000000,0.000000}%
\pgfsetfillcolor{currentfill}%
\pgfsetlinewidth{0.803000pt}%
\definecolor{currentstroke}{rgb}{0.000000,0.000000,0.000000}%
\pgfsetstrokecolor{currentstroke}%
\pgfsetdash{}{0pt}%
\pgfsys@defobject{currentmarker}{\pgfqpoint{-0.048611in}{0.000000in}}{\pgfqpoint{-0.000000in}{0.000000in}}{%
\pgfpathmoveto{\pgfqpoint{-0.000000in}{0.000000in}}%
\pgfpathlineto{\pgfqpoint{-0.048611in}{0.000000in}}%
\pgfusepath{stroke,fill}%
}%
\begin{pgfscope}%
\pgfsys@transformshift{0.800000in}{3.088285in}%
\pgfsys@useobject{currentmarker}{}%
\end{pgfscope}%
\end{pgfscope}%
\begin{pgfscope}%
\definecolor{textcolor}{rgb}{0.000000,0.000000,0.000000}%
\pgfsetstrokecolor{textcolor}%
\pgfsetfillcolor{textcolor}%
\pgftext[x=0.614412in, y=3.035523in, left, base]{\color{textcolor}\sffamily\fontsize{10.000000}{12.000000}\selectfont 3}%
\end{pgfscope}%
\begin{pgfscope}%
\pgfpathrectangle{\pgfqpoint{0.800000in}{0.528000in}}{\pgfqpoint{4.960000in}{3.696000in}}%
\pgfusepath{clip}%
\pgfsetrectcap%
\pgfsetroundjoin%
\pgfsetlinewidth{0.803000pt}%
\definecolor{currentstroke}{rgb}{0.690196,0.690196,0.690196}%
\pgfsetstrokecolor{currentstroke}%
\pgfsetdash{}{0pt}%
\pgfpathmoveto{\pgfqpoint{0.800000in}{3.572142in}}%
\pgfpathlineto{\pgfqpoint{5.760000in}{3.572142in}}%
\pgfusepath{stroke}%
\end{pgfscope}%
\begin{pgfscope}%
\pgfsetbuttcap%
\pgfsetroundjoin%
\definecolor{currentfill}{rgb}{0.000000,0.000000,0.000000}%
\pgfsetfillcolor{currentfill}%
\pgfsetlinewidth{0.803000pt}%
\definecolor{currentstroke}{rgb}{0.000000,0.000000,0.000000}%
\pgfsetstrokecolor{currentstroke}%
\pgfsetdash{}{0pt}%
\pgfsys@defobject{currentmarker}{\pgfqpoint{-0.048611in}{0.000000in}}{\pgfqpoint{-0.000000in}{0.000000in}}{%
\pgfpathmoveto{\pgfqpoint{-0.000000in}{0.000000in}}%
\pgfpathlineto{\pgfqpoint{-0.048611in}{0.000000in}}%
\pgfusepath{stroke,fill}%
}%
\begin{pgfscope}%
\pgfsys@transformshift{0.800000in}{3.572142in}%
\pgfsys@useobject{currentmarker}{}%
\end{pgfscope}%
\end{pgfscope}%
\begin{pgfscope}%
\definecolor{textcolor}{rgb}{0.000000,0.000000,0.000000}%
\pgfsetstrokecolor{textcolor}%
\pgfsetfillcolor{textcolor}%
\pgftext[x=0.614412in, y=3.519381in, left, base]{\color{textcolor}\sffamily\fontsize{10.000000}{12.000000}\selectfont 4}%
\end{pgfscope}%
\begin{pgfscope}%
\pgfpathrectangle{\pgfqpoint{0.800000in}{0.528000in}}{\pgfqpoint{4.960000in}{3.696000in}}%
\pgfusepath{clip}%
\pgfsetrectcap%
\pgfsetroundjoin%
\pgfsetlinewidth{0.803000pt}%
\definecolor{currentstroke}{rgb}{0.690196,0.690196,0.690196}%
\pgfsetstrokecolor{currentstroke}%
\pgfsetdash{}{0pt}%
\pgfpathmoveto{\pgfqpoint{0.800000in}{4.056000in}}%
\pgfpathlineto{\pgfqpoint{5.760000in}{4.056000in}}%
\pgfusepath{stroke}%
\end{pgfscope}%
\begin{pgfscope}%
\pgfsetbuttcap%
\pgfsetroundjoin%
\definecolor{currentfill}{rgb}{0.000000,0.000000,0.000000}%
\pgfsetfillcolor{currentfill}%
\pgfsetlinewidth{0.803000pt}%
\definecolor{currentstroke}{rgb}{0.000000,0.000000,0.000000}%
\pgfsetstrokecolor{currentstroke}%
\pgfsetdash{}{0pt}%
\pgfsys@defobject{currentmarker}{\pgfqpoint{-0.048611in}{0.000000in}}{\pgfqpoint{-0.000000in}{0.000000in}}{%
\pgfpathmoveto{\pgfqpoint{-0.000000in}{0.000000in}}%
\pgfpathlineto{\pgfqpoint{-0.048611in}{0.000000in}}%
\pgfusepath{stroke,fill}%
}%
\begin{pgfscope}%
\pgfsys@transformshift{0.800000in}{4.056000in}%
\pgfsys@useobject{currentmarker}{}%
\end{pgfscope}%
\end{pgfscope}%
\begin{pgfscope}%
\definecolor{textcolor}{rgb}{0.000000,0.000000,0.000000}%
\pgfsetstrokecolor{textcolor}%
\pgfsetfillcolor{textcolor}%
\pgftext[x=0.614412in, y=4.003238in, left, base]{\color{textcolor}\sffamily\fontsize{10.000000}{12.000000}\selectfont 5}%
\end{pgfscope}%
\begin{pgfscope}%
\definecolor{textcolor}{rgb}{0.000000,0.000000,0.000000}%
\pgfsetstrokecolor{textcolor}%
\pgfsetfillcolor{textcolor}%
\pgftext[x=0.450832in,y=2.376000in,,bottom,rotate=90.000000]{\color{textcolor}\sffamily\fontsize{10.000000}{12.000000}\selectfont Output velocity [rad/s]}%
\end{pgfscope}%
\begin{pgfscope}%
\pgfpathrectangle{\pgfqpoint{0.800000in}{0.528000in}}{\pgfqpoint{4.960000in}{3.696000in}}%
\pgfusepath{clip}%
\pgfsetrectcap%
\pgfsetroundjoin%
\pgfsetlinewidth{1.505625pt}%
\definecolor{currentstroke}{rgb}{0.121569,0.466667,0.705882}%
\pgfsetstrokecolor{currentstroke}%
\pgfsetdash{}{0pt}%
\pgfpathmoveto{\pgfqpoint{1.025455in}{3.021440in}}%
\pgfpathlineto{\pgfqpoint{1.063776in}{3.028985in}}%
\pgfpathlineto{\pgfqpoint{1.138623in}{3.021800in}}%
\pgfpathlineto{\pgfqpoint{1.212806in}{2.960255in}}%
\pgfpathlineto{\pgfqpoint{1.286280in}{2.817989in}}%
\pgfpathlineto{\pgfqpoint{1.360571in}{2.683353in}}%
\pgfpathlineto{\pgfqpoint{1.434952in}{2.542907in}}%
\pgfpathlineto{\pgfqpoint{1.508998in}{2.470880in}}%
\pgfpathlineto{\pgfqpoint{1.584714in}{2.404935in}}%
\pgfpathlineto{\pgfqpoint{1.661929in}{2.328594in}}%
\pgfpathlineto{\pgfqpoint{1.735143in}{2.230337in}}%
\pgfpathlineto{\pgfqpoint{1.808687in}{2.176463in}}%
\pgfpathlineto{\pgfqpoint{1.882936in}{2.128627in}}%
\pgfpathlineto{\pgfqpoint{1.957436in}{2.078367in}}%
\pgfpathlineto{\pgfqpoint{2.032221in}{2.052460in}}%
\pgfpathlineto{\pgfqpoint{2.106580in}{2.014108in}}%
\pgfpathlineto{\pgfqpoint{2.181303in}{1.975085in}}%
\pgfpathlineto{\pgfqpoint{2.255350in}{1.957230in}}%
\pgfpathlineto{\pgfqpoint{2.330461in}{1.935736in}}%
\pgfpathlineto{\pgfqpoint{2.404739in}{1.898814in}}%
\pgfpathlineto{\pgfqpoint{2.479219in}{1.887970in}}%
\pgfpathlineto{\pgfqpoint{2.555322in}{1.874091in}}%
\pgfpathlineto{\pgfqpoint{2.629577in}{1.857998in}}%
\pgfpathlineto{\pgfqpoint{2.703086in}{1.841548in}}%
\pgfpathlineto{\pgfqpoint{2.777270in}{1.824123in}}%
\pgfpathlineto{\pgfqpoint{2.852756in}{1.804470in}}%
\pgfpathlineto{\pgfqpoint{2.927602in}{1.784992in}}%
\pgfpathlineto{\pgfqpoint{3.001582in}{1.727179in}}%
\pgfpathlineto{\pgfqpoint{3.076834in}{1.741638in}}%
\pgfpathlineto{\pgfqpoint{3.150951in}{1.742447in}}%
\pgfpathlineto{\pgfqpoint{3.227261in}{1.736042in}}%
\pgfpathlineto{\pgfqpoint{3.301360in}{1.700458in}}%
\pgfpathlineto{\pgfqpoint{3.375878in}{1.695838in}}%
\pgfpathlineto{\pgfqpoint{3.452246in}{1.679179in}}%
\pgfpathlineto{\pgfqpoint{3.527990in}{1.681133in}}%
\pgfpathlineto{\pgfqpoint{3.599038in}{1.673142in}}%
\pgfpathlineto{\pgfqpoint{3.673314in}{1.666042in}}%
\pgfpathlineto{\pgfqpoint{3.747844in}{1.674754in}}%
\pgfpathlineto{\pgfqpoint{3.822123in}{1.656541in}}%
\pgfpathlineto{\pgfqpoint{3.896442in}{1.650580in}}%
\pgfpathlineto{\pgfqpoint{3.970902in}{1.642761in}}%
\pgfpathlineto{\pgfqpoint{4.045637in}{1.640331in}}%
\pgfpathlineto{\pgfqpoint{4.119856in}{1.640487in}}%
\pgfpathlineto{\pgfqpoint{4.193845in}{1.641450in}}%
\pgfpathlineto{\pgfqpoint{4.268656in}{1.653980in}}%
\pgfpathlineto{\pgfqpoint{4.345033in}{1.635018in}}%
\pgfpathlineto{\pgfqpoint{4.418046in}{1.635697in}}%
\pgfpathlineto{\pgfqpoint{4.492418in}{1.635307in}}%
\pgfpathlineto{\pgfqpoint{4.566377in}{1.623191in}}%
\pgfpathlineto{\pgfqpoint{4.640637in}{1.625898in}}%
\pgfpathlineto{\pgfqpoint{4.715036in}{1.628274in}}%
\pgfpathlineto{\pgfqpoint{4.789632in}{1.626367in}}%
\pgfpathlineto{\pgfqpoint{4.864110in}{1.624036in}}%
\pgfpathlineto{\pgfqpoint{4.938524in}{1.598438in}}%
\pgfpathlineto{\pgfqpoint{5.012461in}{1.607912in}}%
\pgfpathlineto{\pgfqpoint{5.086863in}{1.615330in}}%
\pgfpathlineto{\pgfqpoint{5.161604in}{1.600651in}}%
\pgfpathlineto{\pgfqpoint{5.237893in}{1.634381in}}%
\pgfpathlineto{\pgfqpoint{5.311107in}{1.651454in}}%
\pgfpathlineto{\pgfqpoint{5.386056in}{1.651443in}}%
\pgfpathlineto{\pgfqpoint{5.459939in}{1.590271in}}%
\pgfpathlineto{\pgfqpoint{5.534545in}{1.605960in}}%
\pgfusepath{stroke}%
\end{pgfscope}%
\begin{pgfscope}%
\pgfpathrectangle{\pgfqpoint{0.800000in}{0.528000in}}{\pgfqpoint{4.960000in}{3.696000in}}%
\pgfusepath{clip}%
\pgfsetrectcap%
\pgfsetroundjoin%
\pgfsetlinewidth{1.505625pt}%
\definecolor{currentstroke}{rgb}{1.000000,0.498039,0.054902}%
\pgfsetstrokecolor{currentstroke}%
\pgfsetdash{}{0pt}%
\pgfpathmoveto{\pgfqpoint{1.025455in}{4.056000in}}%
\pgfpathlineto{\pgfqpoint{1.099065in}{4.056000in}}%
\pgfpathlineto{\pgfqpoint{1.172846in}{4.056000in}}%
\pgfpathlineto{\pgfqpoint{1.246645in}{3.812409in}}%
\pgfpathlineto{\pgfqpoint{1.322501in}{3.315192in}}%
\pgfpathlineto{\pgfqpoint{1.396760in}{2.972736in}}%
\pgfpathlineto{\pgfqpoint{1.473513in}{2.644543in}}%
\pgfpathlineto{\pgfqpoint{1.546544in}{2.379119in}}%
\pgfpathlineto{\pgfqpoint{1.621097in}{2.119622in}}%
\pgfpathlineto{\pgfqpoint{1.696748in}{1.962213in}}%
\pgfpathlineto{\pgfqpoint{1.771019in}{1.895070in}}%
\pgfpathlineto{\pgfqpoint{1.848364in}{1.839067in}}%
\pgfpathlineto{\pgfqpoint{1.920266in}{1.803469in}}%
\pgfpathlineto{\pgfqpoint{1.994195in}{1.783265in}}%
\pgfpathlineto{\pgfqpoint{2.068179in}{1.764104in}}%
\pgfpathlineto{\pgfqpoint{2.142269in}{1.784058in}}%
\pgfpathlineto{\pgfqpoint{2.216486in}{1.761642in}}%
\pgfpathlineto{\pgfqpoint{2.291371in}{1.737729in}}%
\pgfpathlineto{\pgfqpoint{2.365491in}{1.739876in}}%
\pgfpathlineto{\pgfqpoint{2.439982in}{1.715297in}}%
\pgfpathlineto{\pgfqpoint{2.514002in}{1.739287in}}%
\pgfpathlineto{\pgfqpoint{2.589901in}{1.739420in}}%
\pgfpathlineto{\pgfqpoint{2.663464in}{1.730973in}}%
\pgfpathlineto{\pgfqpoint{2.737949in}{1.721037in}}%
\pgfpathlineto{\pgfqpoint{2.812354in}{1.687699in}}%
\pgfpathlineto{\pgfqpoint{2.886469in}{1.680831in}}%
\pgfpathlineto{\pgfqpoint{2.960992in}{1.664834in}}%
\pgfpathlineto{\pgfqpoint{3.035396in}{1.646980in}}%
\pgfpathlineto{\pgfqpoint{3.109941in}{1.626330in}}%
\pgfpathlineto{\pgfqpoint{3.185940in}{1.614324in}}%
\pgfpathlineto{\pgfqpoint{3.258333in}{1.610564in}}%
\pgfpathlineto{\pgfqpoint{3.334870in}{1.610782in}}%
\pgfpathlineto{\pgfqpoint{3.409005in}{1.614489in}}%
\pgfpathlineto{\pgfqpoint{3.483374in}{1.617698in}}%
\pgfpathlineto{\pgfqpoint{3.557709in}{1.620433in}}%
\pgfpathlineto{\pgfqpoint{3.635481in}{1.625263in}}%
\pgfpathlineto{\pgfqpoint{3.706173in}{1.621592in}}%
\pgfpathlineto{\pgfqpoint{3.780753in}{1.621000in}}%
\pgfpathlineto{\pgfqpoint{3.855154in}{1.620699in}}%
\pgfpathlineto{\pgfqpoint{3.929719in}{1.632217in}}%
\pgfpathlineto{\pgfqpoint{4.003795in}{1.621182in}}%
\pgfpathlineto{\pgfqpoint{4.078260in}{1.604955in}}%
\pgfpathlineto{\pgfqpoint{4.152395in}{1.670365in}}%
\pgfpathlineto{\pgfqpoint{4.228322in}{1.685022in}}%
\pgfpathlineto{\pgfqpoint{4.302367in}{1.682284in}}%
\pgfpathlineto{\pgfqpoint{4.375819in}{1.640259in}}%
\pgfpathlineto{\pgfqpoint{4.451286in}{1.647415in}}%
\pgfpathlineto{\pgfqpoint{4.525804in}{1.659482in}}%
\pgfpathlineto{\pgfqpoint{4.600850in}{1.659957in}}%
\pgfpathlineto{\pgfqpoint{4.675521in}{1.656796in}}%
\pgfpathlineto{\pgfqpoint{4.752222in}{1.652419in}}%
\pgfpathlineto{\pgfqpoint{4.828037in}{1.640879in}}%
\pgfpathlineto{\pgfqpoint{4.902395in}{1.631198in}}%
\pgfpathlineto{\pgfqpoint{4.976465in}{1.568727in}}%
\pgfpathlineto{\pgfqpoint{5.051645in}{1.574251in}}%
\pgfpathlineto{\pgfqpoint{5.125255in}{1.629864in}}%
\pgfpathlineto{\pgfqpoint{5.199740in}{1.710680in}}%
\pgfpathlineto{\pgfqpoint{5.273953in}{1.741828in}}%
\pgfpathlineto{\pgfqpoint{5.348063in}{1.742213in}}%
\pgfpathlineto{\pgfqpoint{5.422504in}{1.611219in}}%
\pgfpathlineto{\pgfqpoint{5.497107in}{1.653240in}}%
\pgfusepath{stroke}%
\end{pgfscope}%
\begin{pgfscope}%
\pgfpathrectangle{\pgfqpoint{0.800000in}{0.528000in}}{\pgfqpoint{4.960000in}{3.696000in}}%
\pgfusepath{clip}%
\pgfsetrectcap%
\pgfsetroundjoin%
\pgfsetlinewidth{1.505625pt}%
\definecolor{currentstroke}{rgb}{0.172549,0.627451,0.172549}%
\pgfsetstrokecolor{currentstroke}%
\pgfsetdash{}{0pt}%
\pgfpathmoveto{\pgfqpoint{1.025455in}{4.056000in}}%
\pgfpathlineto{\pgfqpoint{1.099911in}{4.056000in}}%
\pgfpathlineto{\pgfqpoint{1.175324in}{4.056000in}}%
\pgfpathlineto{\pgfqpoint{1.250759in}{4.056000in}}%
\pgfpathlineto{\pgfqpoint{1.324208in}{4.033568in}}%
\pgfpathlineto{\pgfqpoint{1.398758in}{3.297176in}}%
\pgfpathlineto{\pgfqpoint{1.473239in}{2.722634in}}%
\pgfpathlineto{\pgfqpoint{1.547019in}{2.186306in}}%
\pgfpathlineto{\pgfqpoint{1.621688in}{1.767797in}}%
\pgfpathlineto{\pgfqpoint{1.695783in}{1.532545in}}%
\pgfpathlineto{\pgfqpoint{1.770461in}{1.552890in}}%
\pgfpathlineto{\pgfqpoint{1.847841in}{1.503332in}}%
\pgfpathlineto{\pgfqpoint{1.918992in}{1.435281in}}%
\pgfpathlineto{\pgfqpoint{1.995729in}{1.508498in}}%
\pgfpathlineto{\pgfqpoint{2.071400in}{1.595402in}}%
\pgfpathlineto{\pgfqpoint{2.144749in}{1.664784in}}%
\pgfpathlineto{\pgfqpoint{2.218626in}{1.705434in}}%
\pgfpathlineto{\pgfqpoint{2.293223in}{1.673947in}}%
\pgfpathlineto{\pgfqpoint{2.367600in}{1.766448in}}%
\pgfpathlineto{\pgfqpoint{2.441953in}{1.619788in}}%
\pgfpathlineto{\pgfqpoint{2.516471in}{1.676780in}}%
\pgfpathlineto{\pgfqpoint{2.590935in}{1.677859in}}%
\pgfpathlineto{\pgfqpoint{2.665483in}{1.690302in}}%
\pgfpathlineto{\pgfqpoint{2.739264in}{1.704827in}}%
\pgfpathlineto{\pgfqpoint{2.814189in}{1.718717in}}%
\pgfpathlineto{\pgfqpoint{2.889825in}{1.694974in}}%
\pgfpathlineto{\pgfqpoint{2.963704in}{1.694273in}}%
\pgfpathlineto{\pgfqpoint{3.037334in}{1.694556in}}%
\pgfpathlineto{\pgfqpoint{3.112176in}{1.703119in}}%
\pgfpathlineto{\pgfqpoint{3.186516in}{1.710191in}}%
\pgfpathlineto{\pgfqpoint{3.260694in}{1.700921in}}%
\pgfpathlineto{\pgfqpoint{3.335034in}{1.629344in}}%
\pgfpathlineto{\pgfqpoint{3.409356in}{1.632422in}}%
\pgfpathlineto{\pgfqpoint{3.484633in}{1.643194in}}%
\pgfpathlineto{\pgfqpoint{3.558101in}{1.661479in}}%
\pgfpathlineto{\pgfqpoint{3.632383in}{1.664968in}}%
\pgfpathlineto{\pgfqpoint{3.707011in}{1.559882in}}%
\pgfpathlineto{\pgfqpoint{3.783326in}{1.551512in}}%
\pgfpathlineto{\pgfqpoint{3.856908in}{1.653795in}}%
\pgfpathlineto{\pgfqpoint{3.930715in}{1.691680in}}%
\pgfpathlineto{\pgfqpoint{4.004643in}{1.713630in}}%
\pgfpathlineto{\pgfqpoint{4.078590in}{1.719447in}}%
\pgfpathlineto{\pgfqpoint{4.154100in}{1.728822in}}%
\pgfpathlineto{\pgfqpoint{4.228692in}{1.701220in}}%
\pgfpathlineto{\pgfqpoint{4.302687in}{1.675536in}}%
\pgfpathlineto{\pgfqpoint{4.377390in}{1.670215in}}%
\pgfpathlineto{\pgfqpoint{4.451459in}{1.662011in}}%
\pgfpathlineto{\pgfqpoint{4.525943in}{1.650868in}}%
\pgfpathlineto{\pgfqpoint{4.601122in}{1.650723in}}%
\pgfpathlineto{\pgfqpoint{4.675193in}{1.648543in}}%
\pgfpathlineto{\pgfqpoint{4.750877in}{1.662652in}}%
\pgfpathlineto{\pgfqpoint{4.824357in}{1.655460in}}%
\pgfpathlineto{\pgfqpoint{4.898213in}{1.654774in}}%
\pgfpathlineto{\pgfqpoint{4.972646in}{1.728174in}}%
\pgfpathlineto{\pgfqpoint{5.047668in}{1.733817in}}%
\pgfpathlineto{\pgfqpoint{5.121956in}{1.715996in}}%
\pgfpathlineto{\pgfqpoint{5.196176in}{1.606531in}}%
\pgfpathlineto{\pgfqpoint{5.270506in}{1.623516in}}%
\pgfpathlineto{\pgfqpoint{5.345043in}{1.649031in}}%
\pgfpathlineto{\pgfqpoint{5.419263in}{1.661690in}}%
\pgfpathlineto{\pgfqpoint{5.493699in}{1.613993in}}%
\pgfusepath{stroke}%
\end{pgfscope}%
\begin{pgfscope}%
\pgfpathrectangle{\pgfqpoint{0.800000in}{0.528000in}}{\pgfqpoint{4.960000in}{3.696000in}}%
\pgfusepath{clip}%
\pgfsetrectcap%
\pgfsetroundjoin%
\pgfsetlinewidth{1.505625pt}%
\definecolor{currentstroke}{rgb}{0.839216,0.152941,0.156863}%
\pgfsetstrokecolor{currentstroke}%
\pgfsetdash{}{0pt}%
\pgfpathmoveto{\pgfqpoint{1.025455in}{4.056000in}}%
\pgfpathlineto{\pgfqpoint{1.099317in}{4.056000in}}%
\pgfpathlineto{\pgfqpoint{1.173519in}{4.056000in}}%
\pgfpathlineto{\pgfqpoint{1.247511in}{4.056000in}}%
\pgfpathlineto{\pgfqpoint{1.322201in}{4.056000in}}%
\pgfpathlineto{\pgfqpoint{1.396396in}{3.941041in}}%
\pgfpathlineto{\pgfqpoint{1.470917in}{3.036683in}}%
\pgfpathlineto{\pgfqpoint{1.545546in}{2.369313in}}%
\pgfpathlineto{\pgfqpoint{1.619703in}{1.400020in}}%
\pgfpathlineto{\pgfqpoint{1.694012in}{1.332712in}}%
\pgfpathlineto{\pgfqpoint{1.768885in}{1.051299in}}%
\pgfpathlineto{\pgfqpoint{1.842986in}{1.208697in}}%
\pgfpathlineto{\pgfqpoint{1.919062in}{1.364708in}}%
\pgfpathlineto{\pgfqpoint{1.992356in}{1.532399in}}%
\pgfpathlineto{\pgfqpoint{2.066206in}{1.699418in}}%
\pgfpathlineto{\pgfqpoint{2.140388in}{1.854254in}}%
\pgfpathlineto{\pgfqpoint{2.214802in}{1.898369in}}%
\pgfpathlineto{\pgfqpoint{2.288995in}{1.843419in}}%
\pgfpathlineto{\pgfqpoint{2.363309in}{1.795724in}}%
\pgfpathlineto{\pgfqpoint{2.437570in}{1.771615in}}%
\pgfpathlineto{\pgfqpoint{2.512228in}{1.725422in}}%
\pgfpathlineto{\pgfqpoint{2.586632in}{1.702794in}}%
\pgfpathlineto{\pgfqpoint{2.661688in}{1.647634in}}%
\pgfpathlineto{\pgfqpoint{2.735747in}{1.503967in}}%
\pgfpathlineto{\pgfqpoint{2.809657in}{1.513126in}}%
\pgfpathlineto{\pgfqpoint{2.883897in}{1.580054in}}%
\pgfpathlineto{\pgfqpoint{2.958353in}{1.605623in}}%
\pgfpathlineto{\pgfqpoint{3.032628in}{1.710769in}}%
\pgfpathlineto{\pgfqpoint{3.107069in}{1.704037in}}%
\pgfpathlineto{\pgfqpoint{3.181518in}{1.740683in}}%
\pgfpathlineto{\pgfqpoint{3.255803in}{1.759137in}}%
\pgfpathlineto{\pgfqpoint{3.330082in}{1.777209in}}%
\pgfpathlineto{\pgfqpoint{3.406248in}{1.789555in}}%
\pgfpathlineto{\pgfqpoint{3.482771in}{1.799919in}}%
\pgfpathlineto{\pgfqpoint{3.555301in}{1.761778in}}%
\pgfpathlineto{\pgfqpoint{3.629360in}{1.723305in}}%
\pgfpathlineto{\pgfqpoint{3.703565in}{1.686573in}}%
\pgfpathlineto{\pgfqpoint{3.778024in}{1.681139in}}%
\pgfpathlineto{\pgfqpoint{3.852388in}{1.613812in}}%
\pgfpathlineto{\pgfqpoint{3.927125in}{1.592882in}}%
\pgfpathlineto{\pgfqpoint{4.002276in}{1.573249in}}%
\pgfpathlineto{\pgfqpoint{4.076613in}{1.580721in}}%
\pgfpathlineto{\pgfqpoint{4.150560in}{1.590732in}}%
\pgfpathlineto{\pgfqpoint{4.225595in}{1.630211in}}%
\pgfpathlineto{\pgfqpoint{4.302408in}{1.599815in}}%
\pgfpathlineto{\pgfqpoint{4.375572in}{1.657526in}}%
\pgfpathlineto{\pgfqpoint{4.449438in}{1.476775in}}%
\pgfpathlineto{\pgfqpoint{4.523700in}{1.607132in}}%
\pgfpathlineto{\pgfqpoint{4.598169in}{1.723546in}}%
\pgfpathlineto{\pgfqpoint{4.672232in}{1.808980in}}%
\pgfpathlineto{\pgfqpoint{4.747518in}{1.802720in}}%
\pgfpathlineto{\pgfqpoint{4.822072in}{1.862368in}}%
\pgfpathlineto{\pgfqpoint{4.897366in}{1.868498in}}%
\pgfpathlineto{\pgfqpoint{4.971408in}{1.855432in}}%
\pgfpathlineto{\pgfqpoint{5.045575in}{1.787219in}}%
\pgfpathlineto{\pgfqpoint{5.120129in}{1.764508in}}%
\pgfpathlineto{\pgfqpoint{5.197079in}{1.729177in}}%
\pgfpathlineto{\pgfqpoint{5.270598in}{1.606630in}}%
\pgfpathlineto{\pgfqpoint{5.345079in}{1.573076in}}%
\pgfpathlineto{\pgfqpoint{5.419659in}{1.550486in}}%
\pgfpathlineto{\pgfqpoint{5.493677in}{1.556819in}}%
\pgfusepath{stroke}%
\end{pgfscope}%
\begin{pgfscope}%
\pgfpathrectangle{\pgfqpoint{0.800000in}{0.528000in}}{\pgfqpoint{4.960000in}{3.696000in}}%
\pgfusepath{clip}%
\pgfsetrectcap%
\pgfsetroundjoin%
\pgfsetlinewidth{1.505625pt}%
\definecolor{currentstroke}{rgb}{0.580392,0.403922,0.741176}%
\pgfsetstrokecolor{currentstroke}%
\pgfsetdash{}{0pt}%
\pgfpathmoveto{\pgfqpoint{1.025455in}{4.056000in}}%
\pgfpathlineto{\pgfqpoint{1.100458in}{4.056000in}}%
\pgfpathlineto{\pgfqpoint{1.176281in}{4.056000in}}%
\pgfpathlineto{\pgfqpoint{1.250820in}{4.056000in}}%
\pgfpathlineto{\pgfqpoint{1.324749in}{4.056000in}}%
\pgfpathlineto{\pgfqpoint{1.398416in}{4.056000in}}%
\pgfpathlineto{\pgfqpoint{1.472778in}{3.362258in}}%
\pgfpathlineto{\pgfqpoint{1.547787in}{2.414660in}}%
\pgfpathlineto{\pgfqpoint{1.622083in}{1.530707in}}%
\pgfpathlineto{\pgfqpoint{1.696655in}{0.929890in}}%
\pgfpathlineto{\pgfqpoint{1.771695in}{0.818292in}}%
\pgfpathlineto{\pgfqpoint{1.846473in}{0.715897in}}%
\pgfpathlineto{\pgfqpoint{1.921333in}{1.315738in}}%
\pgfpathlineto{\pgfqpoint{1.997868in}{1.587171in}}%
\pgfpathlineto{\pgfqpoint{2.071678in}{1.806811in}}%
\pgfpathlineto{\pgfqpoint{2.145699in}{1.973657in}}%
\pgfpathlineto{\pgfqpoint{2.220025in}{2.103448in}}%
\pgfpathlineto{\pgfqpoint{2.294077in}{2.120648in}}%
\pgfpathlineto{\pgfqpoint{2.370747in}{2.009254in}}%
\pgfpathlineto{\pgfqpoint{2.443124in}{1.831374in}}%
\pgfpathlineto{\pgfqpoint{2.517862in}{1.826515in}}%
\pgfpathlineto{\pgfqpoint{2.591929in}{1.714858in}}%
\pgfpathlineto{\pgfqpoint{2.666223in}{1.612647in}}%
\pgfpathlineto{\pgfqpoint{2.740918in}{1.378737in}}%
\pgfpathlineto{\pgfqpoint{2.816862in}{1.162193in}}%
\pgfpathlineto{\pgfqpoint{2.890082in}{1.277368in}}%
\pgfpathlineto{\pgfqpoint{2.964029in}{1.457425in}}%
\pgfpathlineto{\pgfqpoint{3.038293in}{1.680969in}}%
\pgfpathlineto{\pgfqpoint{3.112766in}{1.692094in}}%
\pgfpathlineto{\pgfqpoint{3.186957in}{1.920945in}}%
\pgfpathlineto{\pgfqpoint{3.261805in}{1.937551in}}%
\pgfpathlineto{\pgfqpoint{3.337022in}{1.948411in}}%
\pgfpathlineto{\pgfqpoint{3.411398in}{1.953992in}}%
\pgfpathlineto{\pgfqpoint{3.486552in}{1.884505in}}%
\pgfpathlineto{\pgfqpoint{3.560422in}{1.856106in}}%
\pgfpathlineto{\pgfqpoint{3.634412in}{1.594821in}}%
\pgfpathlineto{\pgfqpoint{3.710497in}{1.558224in}}%
\pgfpathlineto{\pgfqpoint{3.783965in}{1.497394in}}%
\pgfpathlineto{\pgfqpoint{3.857775in}{1.500011in}}%
\pgfpathlineto{\pgfqpoint{3.931643in}{1.506266in}}%
\pgfpathlineto{\pgfqpoint{4.006370in}{1.523410in}}%
\pgfpathlineto{\pgfqpoint{4.080629in}{1.523678in}}%
\pgfpathlineto{\pgfqpoint{4.155515in}{1.517873in}}%
\pgfpathlineto{\pgfqpoint{4.229449in}{1.750368in}}%
\pgfpathlineto{\pgfqpoint{4.304294in}{1.945804in}}%
\pgfpathlineto{\pgfqpoint{4.378164in}{1.939939in}}%
\pgfpathlineto{\pgfqpoint{4.454060in}{1.861870in}}%
\pgfpathlineto{\pgfqpoint{4.528649in}{1.814241in}}%
\pgfpathlineto{\pgfqpoint{4.605303in}{1.801802in}}%
\pgfpathlineto{\pgfqpoint{4.678615in}{1.721985in}}%
\pgfpathlineto{\pgfqpoint{4.752839in}{1.616832in}}%
\pgfpathlineto{\pgfqpoint{4.827129in}{1.509026in}}%
\pgfpathlineto{\pgfqpoint{4.902033in}{1.462192in}}%
\pgfpathlineto{\pgfqpoint{4.977876in}{1.156509in}}%
\pgfpathlineto{\pgfqpoint{5.051683in}{1.257151in}}%
\pgfpathlineto{\pgfqpoint{5.125830in}{1.441844in}}%
\pgfpathlineto{\pgfqpoint{5.200384in}{1.536551in}}%
\pgfpathlineto{\pgfqpoint{5.274235in}{1.764799in}}%
\pgfpathlineto{\pgfqpoint{5.348695in}{1.977549in}}%
\pgfpathlineto{\pgfqpoint{5.423195in}{1.944105in}}%
\pgfpathlineto{\pgfqpoint{5.499079in}{1.947694in}}%
\pgfusepath{stroke}%
\end{pgfscope}%
\begin{pgfscope}%
\pgfpathrectangle{\pgfqpoint{0.800000in}{0.528000in}}{\pgfqpoint{4.960000in}{3.696000in}}%
\pgfusepath{clip}%
\pgfsetrectcap%
\pgfsetroundjoin%
\pgfsetlinewidth{1.505625pt}%
\definecolor{currentstroke}{rgb}{0.549020,0.337255,0.294118}%
\pgfsetstrokecolor{currentstroke}%
\pgfsetdash{}{0pt}%
\pgfpathmoveto{\pgfqpoint{1.025455in}{4.056000in}}%
\pgfpathlineto{\pgfqpoint{1.099647in}{4.056000in}}%
\pgfpathlineto{\pgfqpoint{1.174084in}{4.056000in}}%
\pgfpathlineto{\pgfqpoint{1.248390in}{4.056000in}}%
\pgfpathlineto{\pgfqpoint{1.322841in}{4.056000in}}%
\pgfpathlineto{\pgfqpoint{1.397621in}{4.056000in}}%
\pgfpathlineto{\pgfqpoint{1.472134in}{3.738312in}}%
\pgfpathlineto{\pgfqpoint{1.546661in}{2.125006in}}%
\pgfpathlineto{\pgfqpoint{1.622031in}{1.254203in}}%
\pgfpathlineto{\pgfqpoint{1.696124in}{0.696000in}}%
\pgfpathlineto{\pgfqpoint{1.770359in}{0.720984in}}%
\pgfpathlineto{\pgfqpoint{1.844945in}{0.821698in}}%
\pgfpathlineto{\pgfqpoint{1.919056in}{1.284716in}}%
\pgfpathlineto{\pgfqpoint{1.993644in}{1.838296in}}%
\pgfpathlineto{\pgfqpoint{2.067688in}{2.212149in}}%
\pgfpathlineto{\pgfqpoint{2.142953in}{2.354063in}}%
\pgfpathlineto{\pgfqpoint{2.216798in}{2.260825in}}%
\pgfpathlineto{\pgfqpoint{2.290796in}{2.183702in}}%
\pgfpathlineto{\pgfqpoint{2.365613in}{2.004463in}}%
\pgfpathlineto{\pgfqpoint{2.442489in}{1.697830in}}%
\pgfpathlineto{\pgfqpoint{2.515846in}{1.406111in}}%
\pgfpathlineto{\pgfqpoint{2.589371in}{1.240971in}}%
\pgfpathlineto{\pgfqpoint{2.663477in}{1.283059in}}%
\pgfpathlineto{\pgfqpoint{2.738336in}{1.234195in}}%
\pgfpathlineto{\pgfqpoint{2.812747in}{1.373091in}}%
\pgfpathlineto{\pgfqpoint{2.887592in}{1.388472in}}%
\pgfpathlineto{\pgfqpoint{2.961704in}{1.609106in}}%
\pgfpathlineto{\pgfqpoint{3.036167in}{1.761276in}}%
\pgfpathlineto{\pgfqpoint{3.109768in}{2.010735in}}%
\pgfpathlineto{\pgfqpoint{3.184331in}{2.002970in}}%
\pgfpathlineto{\pgfqpoint{3.258978in}{1.969789in}}%
\pgfpathlineto{\pgfqpoint{3.335761in}{1.809746in}}%
\pgfpathlineto{\pgfqpoint{3.410308in}{1.670774in}}%
\pgfpathlineto{\pgfqpoint{3.483913in}{1.551996in}}%
\pgfpathlineto{\pgfqpoint{3.558745in}{1.433322in}}%
\pgfpathlineto{\pgfqpoint{3.633531in}{1.400712in}}%
\pgfpathlineto{\pgfqpoint{3.708051in}{1.389578in}}%
\pgfpathlineto{\pgfqpoint{3.782670in}{1.414679in}}%
\pgfpathlineto{\pgfqpoint{3.856790in}{1.561205in}}%
\pgfpathlineto{\pgfqpoint{3.931143in}{1.717369in}}%
\pgfpathlineto{\pgfqpoint{4.005704in}{1.920413in}}%
\pgfpathlineto{\pgfqpoint{4.080418in}{1.971838in}}%
\pgfpathlineto{\pgfqpoint{4.155350in}{1.863599in}}%
\pgfpathlineto{\pgfqpoint{4.232876in}{1.816952in}}%
\pgfpathlineto{\pgfqpoint{4.305808in}{1.664193in}}%
\pgfpathlineto{\pgfqpoint{4.379205in}{1.614329in}}%
\pgfpathlineto{\pgfqpoint{4.453262in}{1.598158in}}%
\pgfpathlineto{\pgfqpoint{4.528297in}{1.570015in}}%
\pgfpathlineto{\pgfqpoint{4.603125in}{1.516828in}}%
\pgfpathlineto{\pgfqpoint{4.677816in}{1.484594in}}%
\pgfpathlineto{\pgfqpoint{4.751560in}{1.525558in}}%
\pgfpathlineto{\pgfqpoint{4.826891in}{1.575905in}}%
\pgfpathlineto{\pgfqpoint{4.901387in}{1.615073in}}%
\pgfpathlineto{\pgfqpoint{4.975726in}{1.693710in}}%
\pgfpathlineto{\pgfqpoint{5.050456in}{1.722132in}}%
\pgfpathlineto{\pgfqpoint{5.124767in}{1.735335in}}%
\pgfpathlineto{\pgfqpoint{5.199653in}{1.734935in}}%
\pgfpathlineto{\pgfqpoint{5.274321in}{1.544621in}}%
\pgfpathlineto{\pgfqpoint{5.349479in}{1.513715in}}%
\pgfpathlineto{\pgfqpoint{5.424210in}{1.397751in}}%
\pgfpathlineto{\pgfqpoint{5.499007in}{1.465990in}}%
\pgfusepath{stroke}%
\end{pgfscope}%
\begin{pgfscope}%
\pgfsetrectcap%
\pgfsetmiterjoin%
\pgfsetlinewidth{0.803000pt}%
\definecolor{currentstroke}{rgb}{0.000000,0.000000,0.000000}%
\pgfsetstrokecolor{currentstroke}%
\pgfsetdash{}{0pt}%
\pgfpathmoveto{\pgfqpoint{0.800000in}{0.528000in}}%
\pgfpathlineto{\pgfqpoint{0.800000in}{4.224000in}}%
\pgfusepath{stroke}%
\end{pgfscope}%
\begin{pgfscope}%
\pgfsetrectcap%
\pgfsetmiterjoin%
\pgfsetlinewidth{0.803000pt}%
\definecolor{currentstroke}{rgb}{0.000000,0.000000,0.000000}%
\pgfsetstrokecolor{currentstroke}%
\pgfsetdash{}{0pt}%
\pgfpathmoveto{\pgfqpoint{5.760000in}{0.528000in}}%
\pgfpathlineto{\pgfqpoint{5.760000in}{4.224000in}}%
\pgfusepath{stroke}%
\end{pgfscope}%
\begin{pgfscope}%
\pgfsetrectcap%
\pgfsetmiterjoin%
\pgfsetlinewidth{0.803000pt}%
\definecolor{currentstroke}{rgb}{0.000000,0.000000,0.000000}%
\pgfsetstrokecolor{currentstroke}%
\pgfsetdash{}{0pt}%
\pgfpathmoveto{\pgfqpoint{0.800000in}{0.528000in}}%
\pgfpathlineto{\pgfqpoint{5.760000in}{0.528000in}}%
\pgfusepath{stroke}%
\end{pgfscope}%
\begin{pgfscope}%
\pgfsetrectcap%
\pgfsetmiterjoin%
\pgfsetlinewidth{0.803000pt}%
\definecolor{currentstroke}{rgb}{0.000000,0.000000,0.000000}%
\pgfsetstrokecolor{currentstroke}%
\pgfsetdash{}{0pt}%
\pgfpathmoveto{\pgfqpoint{0.800000in}{4.224000in}}%
\pgfpathlineto{\pgfqpoint{5.760000in}{4.224000in}}%
\pgfusepath{stroke}%
\end{pgfscope}%
\begin{pgfscope}%
\definecolor{textcolor}{rgb}{0.000000,0.000000,0.000000}%
\pgfsetstrokecolor{textcolor}%
\pgfsetfillcolor{textcolor}%
\pgftext[x=3.280000in,y=4.307333in,,base]{\color{textcolor}\sffamily\fontsize{12.000000}{14.400000}\selectfont Yaw controller output}%
\end{pgfscope}%
\begin{pgfscope}%
\pgfsetbuttcap%
\pgfsetmiterjoin%
\definecolor{currentfill}{rgb}{1.000000,1.000000,1.000000}%
\pgfsetfillcolor{currentfill}%
\pgfsetfillopacity{0.800000}%
\pgfsetlinewidth{1.003750pt}%
\definecolor{currentstroke}{rgb}{0.800000,0.800000,0.800000}%
\pgfsetstrokecolor{currentstroke}%
\pgfsetstrokeopacity{0.800000}%
\pgfsetdash{}{0pt}%
\pgfpathmoveto{\pgfqpoint{4.953237in}{2.889746in}}%
\pgfpathlineto{\pgfqpoint{5.662778in}{2.889746in}}%
\pgfpathquadraticcurveto{\pgfqpoint{5.690556in}{2.889746in}}{\pgfqpoint{5.690556in}{2.917523in}}%
\pgfpathlineto{\pgfqpoint{5.690556in}{4.126778in}}%
\pgfpathquadraticcurveto{\pgfqpoint{5.690556in}{4.154556in}}{\pgfqpoint{5.662778in}{4.154556in}}%
\pgfpathlineto{\pgfqpoint{4.953237in}{4.154556in}}%
\pgfpathquadraticcurveto{\pgfqpoint{4.925460in}{4.154556in}}{\pgfqpoint{4.925460in}{4.126778in}}%
\pgfpathlineto{\pgfqpoint{4.925460in}{2.917523in}}%
\pgfpathquadraticcurveto{\pgfqpoint{4.925460in}{2.889746in}}{\pgfqpoint{4.953237in}{2.889746in}}%
\pgfpathlineto{\pgfqpoint{4.953237in}{2.889746in}}%
\pgfpathclose%
\pgfusepath{stroke,fill}%
\end{pgfscope}%
\begin{pgfscope}%
\pgfsetrectcap%
\pgfsetroundjoin%
\pgfsetlinewidth{1.505625pt}%
\definecolor{currentstroke}{rgb}{0.121569,0.466667,0.705882}%
\pgfsetstrokecolor{currentstroke}%
\pgfsetdash{}{0pt}%
\pgfpathmoveto{\pgfqpoint{4.981015in}{4.042088in}}%
\pgfpathlineto{\pgfqpoint{5.119904in}{4.042088in}}%
\pgfpathlineto{\pgfqpoint{5.258793in}{4.042088in}}%
\pgfusepath{stroke}%
\end{pgfscope}%
\begin{pgfscope}%
\definecolor{textcolor}{rgb}{0.000000,0.000000,0.000000}%
\pgfsetstrokecolor{textcolor}%
\pgfsetfillcolor{textcolor}%
\pgftext[x=5.369904in,y=3.993477in,left,base]{\color{textcolor}\sffamily\fontsize{10.000000}{12.000000}\selectfont 25}%
\end{pgfscope}%
\begin{pgfscope}%
\pgfsetrectcap%
\pgfsetroundjoin%
\pgfsetlinewidth{1.505625pt}%
\definecolor{currentstroke}{rgb}{1.000000,0.498039,0.054902}%
\pgfsetstrokecolor{currentstroke}%
\pgfsetdash{}{0pt}%
\pgfpathmoveto{\pgfqpoint{4.981015in}{3.838231in}}%
\pgfpathlineto{\pgfqpoint{5.119904in}{3.838231in}}%
\pgfpathlineto{\pgfqpoint{5.258793in}{3.838231in}}%
\pgfusepath{stroke}%
\end{pgfscope}%
\begin{pgfscope}%
\definecolor{textcolor}{rgb}{0.000000,0.000000,0.000000}%
\pgfsetstrokecolor{textcolor}%
\pgfsetfillcolor{textcolor}%
\pgftext[x=5.369904in,y=3.789620in,left,base]{\color{textcolor}\sffamily\fontsize{10.000000}{12.000000}\selectfont 50}%
\end{pgfscope}%
\begin{pgfscope}%
\pgfsetrectcap%
\pgfsetroundjoin%
\pgfsetlinewidth{1.505625pt}%
\definecolor{currentstroke}{rgb}{0.172549,0.627451,0.172549}%
\pgfsetstrokecolor{currentstroke}%
\pgfsetdash{}{0pt}%
\pgfpathmoveto{\pgfqpoint{4.981015in}{3.634374in}}%
\pgfpathlineto{\pgfqpoint{5.119904in}{3.634374in}}%
\pgfpathlineto{\pgfqpoint{5.258793in}{3.634374in}}%
\pgfusepath{stroke}%
\end{pgfscope}%
\begin{pgfscope}%
\definecolor{textcolor}{rgb}{0.000000,0.000000,0.000000}%
\pgfsetstrokecolor{textcolor}%
\pgfsetfillcolor{textcolor}%
\pgftext[x=5.369904in,y=3.585762in,left,base]{\color{textcolor}\sffamily\fontsize{10.000000}{12.000000}\selectfont 75}%
\end{pgfscope}%
\begin{pgfscope}%
\pgfsetrectcap%
\pgfsetroundjoin%
\pgfsetlinewidth{1.505625pt}%
\definecolor{currentstroke}{rgb}{0.839216,0.152941,0.156863}%
\pgfsetstrokecolor{currentstroke}%
\pgfsetdash{}{0pt}%
\pgfpathmoveto{\pgfqpoint{4.981015in}{3.430516in}}%
\pgfpathlineto{\pgfqpoint{5.119904in}{3.430516in}}%
\pgfpathlineto{\pgfqpoint{5.258793in}{3.430516in}}%
\pgfusepath{stroke}%
\end{pgfscope}%
\begin{pgfscope}%
\definecolor{textcolor}{rgb}{0.000000,0.000000,0.000000}%
\pgfsetstrokecolor{textcolor}%
\pgfsetfillcolor{textcolor}%
\pgftext[x=5.369904in,y=3.381905in,left,base]{\color{textcolor}\sffamily\fontsize{10.000000}{12.000000}\selectfont 100}%
\end{pgfscope}%
\begin{pgfscope}%
\pgfsetrectcap%
\pgfsetroundjoin%
\pgfsetlinewidth{1.505625pt}%
\definecolor{currentstroke}{rgb}{0.580392,0.403922,0.741176}%
\pgfsetstrokecolor{currentstroke}%
\pgfsetdash{}{0pt}%
\pgfpathmoveto{\pgfqpoint{4.981015in}{3.226659in}}%
\pgfpathlineto{\pgfqpoint{5.119904in}{3.226659in}}%
\pgfpathlineto{\pgfqpoint{5.258793in}{3.226659in}}%
\pgfusepath{stroke}%
\end{pgfscope}%
\begin{pgfscope}%
\definecolor{textcolor}{rgb}{0.000000,0.000000,0.000000}%
\pgfsetstrokecolor{textcolor}%
\pgfsetfillcolor{textcolor}%
\pgftext[x=5.369904in,y=3.178048in,left,base]{\color{textcolor}\sffamily\fontsize{10.000000}{12.000000}\selectfont 125}%
\end{pgfscope}%
\begin{pgfscope}%
\pgfsetrectcap%
\pgfsetroundjoin%
\pgfsetlinewidth{1.505625pt}%
\definecolor{currentstroke}{rgb}{0.549020,0.337255,0.294118}%
\pgfsetstrokecolor{currentstroke}%
\pgfsetdash{}{0pt}%
\pgfpathmoveto{\pgfqpoint{4.981015in}{3.022802in}}%
\pgfpathlineto{\pgfqpoint{5.119904in}{3.022802in}}%
\pgfpathlineto{\pgfqpoint{5.258793in}{3.022802in}}%
\pgfusepath{stroke}%
\end{pgfscope}%
\begin{pgfscope}%
\definecolor{textcolor}{rgb}{0.000000,0.000000,0.000000}%
\pgfsetstrokecolor{textcolor}%
\pgfsetfillcolor{textcolor}%
\pgftext[x=5.369904in,y=2.974191in,left,base]{\color{textcolor}\sffamily\fontsize{10.000000}{12.000000}\selectfont 150}%
\end{pgfscope}%
\end{pgfpicture}%
\makeatother%
\endgroup%
}
    \end{minipage}
    \\
    \begin{minipage}[t]{0.5\linewidth}
        \centering
        \scalebox{0.55}{%% Creator: Matplotlib, PGF backend
%%
%% To include the figure in your LaTeX document, write
%%   \input{<filename>.pgf}
%%
%% Make sure the required packages are loaded in your preamble
%%   \usepackage{pgf}
%%
%% Also ensure that all the required font packages are loaded; for instance,
%% the lmodern package is sometimes necessary when using math font.
%%   \usepackage{lmodern}
%%
%% Figures using additional raster images can only be included by \input if
%% they are in the same directory as the main LaTeX file. For loading figures
%% from other directories you can use the `import` package
%%   \usepackage{import}
%%
%% and then include the figures with
%%   \import{<path to file>}{<filename>.pgf}
%%
%% Matplotlib used the following preamble
%%   \usepackage{fontspec}
%%   \setmainfont{DejaVuSerif.ttf}[Path=\detokenize{/home/lgonz/tfg-aero/tfg-giaa-dronecontrol/venv/lib/python3.8/site-packages/matplotlib/mpl-data/fonts/ttf/}]
%%   \setsansfont{DejaVuSans.ttf}[Path=\detokenize{/home/lgonz/tfg-aero/tfg-giaa-dronecontrol/venv/lib/python3.8/site-packages/matplotlib/mpl-data/fonts/ttf/}]
%%   \setmonofont{DejaVuSansMono.ttf}[Path=\detokenize{/home/lgonz/tfg-aero/tfg-giaa-dronecontrol/venv/lib/python3.8/site-packages/matplotlib/mpl-data/fonts/ttf/}]
%%
\begingroup%
\makeatletter%
\begin{pgfpicture}%
\pgfpathrectangle{\pgfpointorigin}{\pgfqpoint{6.400000in}{4.800000in}}%
\pgfusepath{use as bounding box, clip}%
\begin{pgfscope}%
\pgfsetbuttcap%
\pgfsetmiterjoin%
\definecolor{currentfill}{rgb}{1.000000,1.000000,1.000000}%
\pgfsetfillcolor{currentfill}%
\pgfsetlinewidth{0.000000pt}%
\definecolor{currentstroke}{rgb}{1.000000,1.000000,1.000000}%
\pgfsetstrokecolor{currentstroke}%
\pgfsetdash{}{0pt}%
\pgfpathmoveto{\pgfqpoint{0.000000in}{0.000000in}}%
\pgfpathlineto{\pgfqpoint{6.400000in}{0.000000in}}%
\pgfpathlineto{\pgfqpoint{6.400000in}{4.800000in}}%
\pgfpathlineto{\pgfqpoint{0.000000in}{4.800000in}}%
\pgfpathlineto{\pgfqpoint{0.000000in}{0.000000in}}%
\pgfpathclose%
\pgfusepath{fill}%
\end{pgfscope}%
\begin{pgfscope}%
\pgfsetbuttcap%
\pgfsetmiterjoin%
\definecolor{currentfill}{rgb}{1.000000,1.000000,1.000000}%
\pgfsetfillcolor{currentfill}%
\pgfsetlinewidth{0.000000pt}%
\definecolor{currentstroke}{rgb}{0.000000,0.000000,0.000000}%
\pgfsetstrokecolor{currentstroke}%
\pgfsetstrokeopacity{0.000000}%
\pgfsetdash{}{0pt}%
\pgfpathmoveto{\pgfqpoint{0.800000in}{0.528000in}}%
\pgfpathlineto{\pgfqpoint{5.760000in}{0.528000in}}%
\pgfpathlineto{\pgfqpoint{5.760000in}{4.224000in}}%
\pgfpathlineto{\pgfqpoint{0.800000in}{4.224000in}}%
\pgfpathlineto{\pgfqpoint{0.800000in}{0.528000in}}%
\pgfpathclose%
\pgfusepath{fill}%
\end{pgfscope}%
\begin{pgfscope}%
\pgfpathrectangle{\pgfqpoint{0.800000in}{0.528000in}}{\pgfqpoint{4.960000in}{3.696000in}}%
\pgfusepath{clip}%
\pgfsetrectcap%
\pgfsetroundjoin%
\pgfsetlinewidth{0.803000pt}%
\definecolor{currentstroke}{rgb}{0.690196,0.690196,0.690196}%
\pgfsetstrokecolor{currentstroke}%
\pgfsetdash{}{0pt}%
\pgfpathmoveto{\pgfqpoint{1.025455in}{0.528000in}}%
\pgfpathlineto{\pgfqpoint{1.025455in}{4.224000in}}%
\pgfusepath{stroke}%
\end{pgfscope}%
\begin{pgfscope}%
\pgfsetbuttcap%
\pgfsetroundjoin%
\definecolor{currentfill}{rgb}{0.000000,0.000000,0.000000}%
\pgfsetfillcolor{currentfill}%
\pgfsetlinewidth{0.803000pt}%
\definecolor{currentstroke}{rgb}{0.000000,0.000000,0.000000}%
\pgfsetstrokecolor{currentstroke}%
\pgfsetdash{}{0pt}%
\pgfsys@defobject{currentmarker}{\pgfqpoint{0.000000in}{-0.048611in}}{\pgfqpoint{0.000000in}{0.000000in}}{%
\pgfpathmoveto{\pgfqpoint{0.000000in}{0.000000in}}%
\pgfpathlineto{\pgfqpoint{0.000000in}{-0.048611in}}%
\pgfusepath{stroke,fill}%
}%
\begin{pgfscope}%
\pgfsys@transformshift{1.025455in}{0.528000in}%
\pgfsys@useobject{currentmarker}{}%
\end{pgfscope}%
\end{pgfscope}%
\begin{pgfscope}%
\definecolor{textcolor}{rgb}{0.000000,0.000000,0.000000}%
\pgfsetstrokecolor{textcolor}%
\pgfsetfillcolor{textcolor}%
\pgftext[x=1.025455in,y=0.430778in,,top]{\color{textcolor}\sffamily\fontsize{10.000000}{12.000000}\selectfont 0.0}%
\end{pgfscope}%
\begin{pgfscope}%
\pgfpathrectangle{\pgfqpoint{0.800000in}{0.528000in}}{\pgfqpoint{4.960000in}{3.696000in}}%
\pgfusepath{clip}%
\pgfsetrectcap%
\pgfsetroundjoin%
\pgfsetlinewidth{0.803000pt}%
\definecolor{currentstroke}{rgb}{0.690196,0.690196,0.690196}%
\pgfsetstrokecolor{currentstroke}%
\pgfsetdash{}{0pt}%
\pgfpathmoveto{\pgfqpoint{1.583179in}{0.528000in}}%
\pgfpathlineto{\pgfqpoint{1.583179in}{4.224000in}}%
\pgfusepath{stroke}%
\end{pgfscope}%
\begin{pgfscope}%
\pgfsetbuttcap%
\pgfsetroundjoin%
\definecolor{currentfill}{rgb}{0.000000,0.000000,0.000000}%
\pgfsetfillcolor{currentfill}%
\pgfsetlinewidth{0.803000pt}%
\definecolor{currentstroke}{rgb}{0.000000,0.000000,0.000000}%
\pgfsetstrokecolor{currentstroke}%
\pgfsetdash{}{0pt}%
\pgfsys@defobject{currentmarker}{\pgfqpoint{0.000000in}{-0.048611in}}{\pgfqpoint{0.000000in}{0.000000in}}{%
\pgfpathmoveto{\pgfqpoint{0.000000in}{0.000000in}}%
\pgfpathlineto{\pgfqpoint{0.000000in}{-0.048611in}}%
\pgfusepath{stroke,fill}%
}%
\begin{pgfscope}%
\pgfsys@transformshift{1.583179in}{0.528000in}%
\pgfsys@useobject{currentmarker}{}%
\end{pgfscope}%
\end{pgfscope}%
\begin{pgfscope}%
\definecolor{textcolor}{rgb}{0.000000,0.000000,0.000000}%
\pgfsetstrokecolor{textcolor}%
\pgfsetfillcolor{textcolor}%
\pgftext[x=1.583179in,y=0.430778in,,top]{\color{textcolor}\sffamily\fontsize{10.000000}{12.000000}\selectfont 2.5}%
\end{pgfscope}%
\begin{pgfscope}%
\pgfpathrectangle{\pgfqpoint{0.800000in}{0.528000in}}{\pgfqpoint{4.960000in}{3.696000in}}%
\pgfusepath{clip}%
\pgfsetrectcap%
\pgfsetroundjoin%
\pgfsetlinewidth{0.803000pt}%
\definecolor{currentstroke}{rgb}{0.690196,0.690196,0.690196}%
\pgfsetstrokecolor{currentstroke}%
\pgfsetdash{}{0pt}%
\pgfpathmoveto{\pgfqpoint{2.140904in}{0.528000in}}%
\pgfpathlineto{\pgfqpoint{2.140904in}{4.224000in}}%
\pgfusepath{stroke}%
\end{pgfscope}%
\begin{pgfscope}%
\pgfsetbuttcap%
\pgfsetroundjoin%
\definecolor{currentfill}{rgb}{0.000000,0.000000,0.000000}%
\pgfsetfillcolor{currentfill}%
\pgfsetlinewidth{0.803000pt}%
\definecolor{currentstroke}{rgb}{0.000000,0.000000,0.000000}%
\pgfsetstrokecolor{currentstroke}%
\pgfsetdash{}{0pt}%
\pgfsys@defobject{currentmarker}{\pgfqpoint{0.000000in}{-0.048611in}}{\pgfqpoint{0.000000in}{0.000000in}}{%
\pgfpathmoveto{\pgfqpoint{0.000000in}{0.000000in}}%
\pgfpathlineto{\pgfqpoint{0.000000in}{-0.048611in}}%
\pgfusepath{stroke,fill}%
}%
\begin{pgfscope}%
\pgfsys@transformshift{2.140904in}{0.528000in}%
\pgfsys@useobject{currentmarker}{}%
\end{pgfscope}%
\end{pgfscope}%
\begin{pgfscope}%
\definecolor{textcolor}{rgb}{0.000000,0.000000,0.000000}%
\pgfsetstrokecolor{textcolor}%
\pgfsetfillcolor{textcolor}%
\pgftext[x=2.140904in,y=0.430778in,,top]{\color{textcolor}\sffamily\fontsize{10.000000}{12.000000}\selectfont 5.0}%
\end{pgfscope}%
\begin{pgfscope}%
\pgfpathrectangle{\pgfqpoint{0.800000in}{0.528000in}}{\pgfqpoint{4.960000in}{3.696000in}}%
\pgfusepath{clip}%
\pgfsetrectcap%
\pgfsetroundjoin%
\pgfsetlinewidth{0.803000pt}%
\definecolor{currentstroke}{rgb}{0.690196,0.690196,0.690196}%
\pgfsetstrokecolor{currentstroke}%
\pgfsetdash{}{0pt}%
\pgfpathmoveto{\pgfqpoint{2.698629in}{0.528000in}}%
\pgfpathlineto{\pgfqpoint{2.698629in}{4.224000in}}%
\pgfusepath{stroke}%
\end{pgfscope}%
\begin{pgfscope}%
\pgfsetbuttcap%
\pgfsetroundjoin%
\definecolor{currentfill}{rgb}{0.000000,0.000000,0.000000}%
\pgfsetfillcolor{currentfill}%
\pgfsetlinewidth{0.803000pt}%
\definecolor{currentstroke}{rgb}{0.000000,0.000000,0.000000}%
\pgfsetstrokecolor{currentstroke}%
\pgfsetdash{}{0pt}%
\pgfsys@defobject{currentmarker}{\pgfqpoint{0.000000in}{-0.048611in}}{\pgfqpoint{0.000000in}{0.000000in}}{%
\pgfpathmoveto{\pgfqpoint{0.000000in}{0.000000in}}%
\pgfpathlineto{\pgfqpoint{0.000000in}{-0.048611in}}%
\pgfusepath{stroke,fill}%
}%
\begin{pgfscope}%
\pgfsys@transformshift{2.698629in}{0.528000in}%
\pgfsys@useobject{currentmarker}{}%
\end{pgfscope}%
\end{pgfscope}%
\begin{pgfscope}%
\definecolor{textcolor}{rgb}{0.000000,0.000000,0.000000}%
\pgfsetstrokecolor{textcolor}%
\pgfsetfillcolor{textcolor}%
\pgftext[x=2.698629in,y=0.430778in,,top]{\color{textcolor}\sffamily\fontsize{10.000000}{12.000000}\selectfont 7.5}%
\end{pgfscope}%
\begin{pgfscope}%
\pgfpathrectangle{\pgfqpoint{0.800000in}{0.528000in}}{\pgfqpoint{4.960000in}{3.696000in}}%
\pgfusepath{clip}%
\pgfsetrectcap%
\pgfsetroundjoin%
\pgfsetlinewidth{0.803000pt}%
\definecolor{currentstroke}{rgb}{0.690196,0.690196,0.690196}%
\pgfsetstrokecolor{currentstroke}%
\pgfsetdash{}{0pt}%
\pgfpathmoveto{\pgfqpoint{3.256354in}{0.528000in}}%
\pgfpathlineto{\pgfqpoint{3.256354in}{4.224000in}}%
\pgfusepath{stroke}%
\end{pgfscope}%
\begin{pgfscope}%
\pgfsetbuttcap%
\pgfsetroundjoin%
\definecolor{currentfill}{rgb}{0.000000,0.000000,0.000000}%
\pgfsetfillcolor{currentfill}%
\pgfsetlinewidth{0.803000pt}%
\definecolor{currentstroke}{rgb}{0.000000,0.000000,0.000000}%
\pgfsetstrokecolor{currentstroke}%
\pgfsetdash{}{0pt}%
\pgfsys@defobject{currentmarker}{\pgfqpoint{0.000000in}{-0.048611in}}{\pgfqpoint{0.000000in}{0.000000in}}{%
\pgfpathmoveto{\pgfqpoint{0.000000in}{0.000000in}}%
\pgfpathlineto{\pgfqpoint{0.000000in}{-0.048611in}}%
\pgfusepath{stroke,fill}%
}%
\begin{pgfscope}%
\pgfsys@transformshift{3.256354in}{0.528000in}%
\pgfsys@useobject{currentmarker}{}%
\end{pgfscope}%
\end{pgfscope}%
\begin{pgfscope}%
\definecolor{textcolor}{rgb}{0.000000,0.000000,0.000000}%
\pgfsetstrokecolor{textcolor}%
\pgfsetfillcolor{textcolor}%
\pgftext[x=3.256354in,y=0.430778in,,top]{\color{textcolor}\sffamily\fontsize{10.000000}{12.000000}\selectfont 10.0}%
\end{pgfscope}%
\begin{pgfscope}%
\pgfpathrectangle{\pgfqpoint{0.800000in}{0.528000in}}{\pgfqpoint{4.960000in}{3.696000in}}%
\pgfusepath{clip}%
\pgfsetrectcap%
\pgfsetroundjoin%
\pgfsetlinewidth{0.803000pt}%
\definecolor{currentstroke}{rgb}{0.690196,0.690196,0.690196}%
\pgfsetstrokecolor{currentstroke}%
\pgfsetdash{}{0pt}%
\pgfpathmoveto{\pgfqpoint{3.814078in}{0.528000in}}%
\pgfpathlineto{\pgfqpoint{3.814078in}{4.224000in}}%
\pgfusepath{stroke}%
\end{pgfscope}%
\begin{pgfscope}%
\pgfsetbuttcap%
\pgfsetroundjoin%
\definecolor{currentfill}{rgb}{0.000000,0.000000,0.000000}%
\pgfsetfillcolor{currentfill}%
\pgfsetlinewidth{0.803000pt}%
\definecolor{currentstroke}{rgb}{0.000000,0.000000,0.000000}%
\pgfsetstrokecolor{currentstroke}%
\pgfsetdash{}{0pt}%
\pgfsys@defobject{currentmarker}{\pgfqpoint{0.000000in}{-0.048611in}}{\pgfqpoint{0.000000in}{0.000000in}}{%
\pgfpathmoveto{\pgfqpoint{0.000000in}{0.000000in}}%
\pgfpathlineto{\pgfqpoint{0.000000in}{-0.048611in}}%
\pgfusepath{stroke,fill}%
}%
\begin{pgfscope}%
\pgfsys@transformshift{3.814078in}{0.528000in}%
\pgfsys@useobject{currentmarker}{}%
\end{pgfscope}%
\end{pgfscope}%
\begin{pgfscope}%
\definecolor{textcolor}{rgb}{0.000000,0.000000,0.000000}%
\pgfsetstrokecolor{textcolor}%
\pgfsetfillcolor{textcolor}%
\pgftext[x=3.814078in,y=0.430778in,,top]{\color{textcolor}\sffamily\fontsize{10.000000}{12.000000}\selectfont 12.5}%
\end{pgfscope}%
\begin{pgfscope}%
\pgfpathrectangle{\pgfqpoint{0.800000in}{0.528000in}}{\pgfqpoint{4.960000in}{3.696000in}}%
\pgfusepath{clip}%
\pgfsetrectcap%
\pgfsetroundjoin%
\pgfsetlinewidth{0.803000pt}%
\definecolor{currentstroke}{rgb}{0.690196,0.690196,0.690196}%
\pgfsetstrokecolor{currentstroke}%
\pgfsetdash{}{0pt}%
\pgfpathmoveto{\pgfqpoint{4.371803in}{0.528000in}}%
\pgfpathlineto{\pgfqpoint{4.371803in}{4.224000in}}%
\pgfusepath{stroke}%
\end{pgfscope}%
\begin{pgfscope}%
\pgfsetbuttcap%
\pgfsetroundjoin%
\definecolor{currentfill}{rgb}{0.000000,0.000000,0.000000}%
\pgfsetfillcolor{currentfill}%
\pgfsetlinewidth{0.803000pt}%
\definecolor{currentstroke}{rgb}{0.000000,0.000000,0.000000}%
\pgfsetstrokecolor{currentstroke}%
\pgfsetdash{}{0pt}%
\pgfsys@defobject{currentmarker}{\pgfqpoint{0.000000in}{-0.048611in}}{\pgfqpoint{0.000000in}{0.000000in}}{%
\pgfpathmoveto{\pgfqpoint{0.000000in}{0.000000in}}%
\pgfpathlineto{\pgfqpoint{0.000000in}{-0.048611in}}%
\pgfusepath{stroke,fill}%
}%
\begin{pgfscope}%
\pgfsys@transformshift{4.371803in}{0.528000in}%
\pgfsys@useobject{currentmarker}{}%
\end{pgfscope}%
\end{pgfscope}%
\begin{pgfscope}%
\definecolor{textcolor}{rgb}{0.000000,0.000000,0.000000}%
\pgfsetstrokecolor{textcolor}%
\pgfsetfillcolor{textcolor}%
\pgftext[x=4.371803in,y=0.430778in,,top]{\color{textcolor}\sffamily\fontsize{10.000000}{12.000000}\selectfont 15.0}%
\end{pgfscope}%
\begin{pgfscope}%
\pgfpathrectangle{\pgfqpoint{0.800000in}{0.528000in}}{\pgfqpoint{4.960000in}{3.696000in}}%
\pgfusepath{clip}%
\pgfsetrectcap%
\pgfsetroundjoin%
\pgfsetlinewidth{0.803000pt}%
\definecolor{currentstroke}{rgb}{0.690196,0.690196,0.690196}%
\pgfsetstrokecolor{currentstroke}%
\pgfsetdash{}{0pt}%
\pgfpathmoveto{\pgfqpoint{4.929528in}{0.528000in}}%
\pgfpathlineto{\pgfqpoint{4.929528in}{4.224000in}}%
\pgfusepath{stroke}%
\end{pgfscope}%
\begin{pgfscope}%
\pgfsetbuttcap%
\pgfsetroundjoin%
\definecolor{currentfill}{rgb}{0.000000,0.000000,0.000000}%
\pgfsetfillcolor{currentfill}%
\pgfsetlinewidth{0.803000pt}%
\definecolor{currentstroke}{rgb}{0.000000,0.000000,0.000000}%
\pgfsetstrokecolor{currentstroke}%
\pgfsetdash{}{0pt}%
\pgfsys@defobject{currentmarker}{\pgfqpoint{0.000000in}{-0.048611in}}{\pgfqpoint{0.000000in}{0.000000in}}{%
\pgfpathmoveto{\pgfqpoint{0.000000in}{0.000000in}}%
\pgfpathlineto{\pgfqpoint{0.000000in}{-0.048611in}}%
\pgfusepath{stroke,fill}%
}%
\begin{pgfscope}%
\pgfsys@transformshift{4.929528in}{0.528000in}%
\pgfsys@useobject{currentmarker}{}%
\end{pgfscope}%
\end{pgfscope}%
\begin{pgfscope}%
\definecolor{textcolor}{rgb}{0.000000,0.000000,0.000000}%
\pgfsetstrokecolor{textcolor}%
\pgfsetfillcolor{textcolor}%
\pgftext[x=4.929528in,y=0.430778in,,top]{\color{textcolor}\sffamily\fontsize{10.000000}{12.000000}\selectfont 17.5}%
\end{pgfscope}%
\begin{pgfscope}%
\pgfpathrectangle{\pgfqpoint{0.800000in}{0.528000in}}{\pgfqpoint{4.960000in}{3.696000in}}%
\pgfusepath{clip}%
\pgfsetrectcap%
\pgfsetroundjoin%
\pgfsetlinewidth{0.803000pt}%
\definecolor{currentstroke}{rgb}{0.690196,0.690196,0.690196}%
\pgfsetstrokecolor{currentstroke}%
\pgfsetdash{}{0pt}%
\pgfpathmoveto{\pgfqpoint{5.487252in}{0.528000in}}%
\pgfpathlineto{\pgfqpoint{5.487252in}{4.224000in}}%
\pgfusepath{stroke}%
\end{pgfscope}%
\begin{pgfscope}%
\pgfsetbuttcap%
\pgfsetroundjoin%
\definecolor{currentfill}{rgb}{0.000000,0.000000,0.000000}%
\pgfsetfillcolor{currentfill}%
\pgfsetlinewidth{0.803000pt}%
\definecolor{currentstroke}{rgb}{0.000000,0.000000,0.000000}%
\pgfsetstrokecolor{currentstroke}%
\pgfsetdash{}{0pt}%
\pgfsys@defobject{currentmarker}{\pgfqpoint{0.000000in}{-0.048611in}}{\pgfqpoint{0.000000in}{0.000000in}}{%
\pgfpathmoveto{\pgfqpoint{0.000000in}{0.000000in}}%
\pgfpathlineto{\pgfqpoint{0.000000in}{-0.048611in}}%
\pgfusepath{stroke,fill}%
}%
\begin{pgfscope}%
\pgfsys@transformshift{5.487252in}{0.528000in}%
\pgfsys@useobject{currentmarker}{}%
\end{pgfscope}%
\end{pgfscope}%
\begin{pgfscope}%
\definecolor{textcolor}{rgb}{0.000000,0.000000,0.000000}%
\pgfsetstrokecolor{textcolor}%
\pgfsetfillcolor{textcolor}%
\pgftext[x=5.487252in,y=0.430778in,,top]{\color{textcolor}\sffamily\fontsize{10.000000}{12.000000}\selectfont 20.0}%
\end{pgfscope}%
\begin{pgfscope}%
\definecolor{textcolor}{rgb}{0.000000,0.000000,0.000000}%
\pgfsetstrokecolor{textcolor}%
\pgfsetfillcolor{textcolor}%
\pgftext[x=3.280000in,y=0.240809in,,top]{\color{textcolor}\sffamily\fontsize{10.000000}{12.000000}\selectfont time [s]}%
\end{pgfscope}%
\begin{pgfscope}%
\pgfpathrectangle{\pgfqpoint{0.800000in}{0.528000in}}{\pgfqpoint{4.960000in}{3.696000in}}%
\pgfusepath{clip}%
\pgfsetrectcap%
\pgfsetroundjoin%
\pgfsetlinewidth{0.803000pt}%
\definecolor{currentstroke}{rgb}{0.690196,0.690196,0.690196}%
\pgfsetstrokecolor{currentstroke}%
\pgfsetdash{}{0pt}%
\pgfpathmoveto{\pgfqpoint{0.800000in}{0.969957in}}%
\pgfpathlineto{\pgfqpoint{5.760000in}{0.969957in}}%
\pgfusepath{stroke}%
\end{pgfscope}%
\begin{pgfscope}%
\pgfsetbuttcap%
\pgfsetroundjoin%
\definecolor{currentfill}{rgb}{0.000000,0.000000,0.000000}%
\pgfsetfillcolor{currentfill}%
\pgfsetlinewidth{0.803000pt}%
\definecolor{currentstroke}{rgb}{0.000000,0.000000,0.000000}%
\pgfsetstrokecolor{currentstroke}%
\pgfsetdash{}{0pt}%
\pgfsys@defobject{currentmarker}{\pgfqpoint{-0.048611in}{0.000000in}}{\pgfqpoint{-0.000000in}{0.000000in}}{%
\pgfpathmoveto{\pgfqpoint{-0.000000in}{0.000000in}}%
\pgfpathlineto{\pgfqpoint{-0.048611in}{0.000000in}}%
\pgfusepath{stroke,fill}%
}%
\begin{pgfscope}%
\pgfsys@transformshift{0.800000in}{0.969957in}%
\pgfsys@useobject{currentmarker}{}%
\end{pgfscope}%
\end{pgfscope}%
\begin{pgfscope}%
\definecolor{textcolor}{rgb}{0.000000,0.000000,0.000000}%
\pgfsetstrokecolor{textcolor}%
\pgfsetfillcolor{textcolor}%
\pgftext[x=0.614412in, y=0.917195in, left, base]{\color{textcolor}\sffamily\fontsize{10.000000}{12.000000}\selectfont 8}%
\end{pgfscope}%
\begin{pgfscope}%
\pgfpathrectangle{\pgfqpoint{0.800000in}{0.528000in}}{\pgfqpoint{4.960000in}{3.696000in}}%
\pgfusepath{clip}%
\pgfsetrectcap%
\pgfsetroundjoin%
\pgfsetlinewidth{0.803000pt}%
\definecolor{currentstroke}{rgb}{0.690196,0.690196,0.690196}%
\pgfsetstrokecolor{currentstroke}%
\pgfsetdash{}{0pt}%
\pgfpathmoveto{\pgfqpoint{0.800000in}{1.507127in}}%
\pgfpathlineto{\pgfqpoint{5.760000in}{1.507127in}}%
\pgfusepath{stroke}%
\end{pgfscope}%
\begin{pgfscope}%
\pgfsetbuttcap%
\pgfsetroundjoin%
\definecolor{currentfill}{rgb}{0.000000,0.000000,0.000000}%
\pgfsetfillcolor{currentfill}%
\pgfsetlinewidth{0.803000pt}%
\definecolor{currentstroke}{rgb}{0.000000,0.000000,0.000000}%
\pgfsetstrokecolor{currentstroke}%
\pgfsetdash{}{0pt}%
\pgfsys@defobject{currentmarker}{\pgfqpoint{-0.048611in}{0.000000in}}{\pgfqpoint{-0.000000in}{0.000000in}}{%
\pgfpathmoveto{\pgfqpoint{-0.000000in}{0.000000in}}%
\pgfpathlineto{\pgfqpoint{-0.048611in}{0.000000in}}%
\pgfusepath{stroke,fill}%
}%
\begin{pgfscope}%
\pgfsys@transformshift{0.800000in}{1.507127in}%
\pgfsys@useobject{currentmarker}{}%
\end{pgfscope}%
\end{pgfscope}%
\begin{pgfscope}%
\definecolor{textcolor}{rgb}{0.000000,0.000000,0.000000}%
\pgfsetstrokecolor{textcolor}%
\pgfsetfillcolor{textcolor}%
\pgftext[x=0.526047in, y=1.454366in, left, base]{\color{textcolor}\sffamily\fontsize{10.000000}{12.000000}\selectfont 10}%
\end{pgfscope}%
\begin{pgfscope}%
\pgfpathrectangle{\pgfqpoint{0.800000in}{0.528000in}}{\pgfqpoint{4.960000in}{3.696000in}}%
\pgfusepath{clip}%
\pgfsetrectcap%
\pgfsetroundjoin%
\pgfsetlinewidth{0.803000pt}%
\definecolor{currentstroke}{rgb}{0.690196,0.690196,0.690196}%
\pgfsetstrokecolor{currentstroke}%
\pgfsetdash{}{0pt}%
\pgfpathmoveto{\pgfqpoint{0.800000in}{2.044297in}}%
\pgfpathlineto{\pgfqpoint{5.760000in}{2.044297in}}%
\pgfusepath{stroke}%
\end{pgfscope}%
\begin{pgfscope}%
\pgfsetbuttcap%
\pgfsetroundjoin%
\definecolor{currentfill}{rgb}{0.000000,0.000000,0.000000}%
\pgfsetfillcolor{currentfill}%
\pgfsetlinewidth{0.803000pt}%
\definecolor{currentstroke}{rgb}{0.000000,0.000000,0.000000}%
\pgfsetstrokecolor{currentstroke}%
\pgfsetdash{}{0pt}%
\pgfsys@defobject{currentmarker}{\pgfqpoint{-0.048611in}{0.000000in}}{\pgfqpoint{-0.000000in}{0.000000in}}{%
\pgfpathmoveto{\pgfqpoint{-0.000000in}{0.000000in}}%
\pgfpathlineto{\pgfqpoint{-0.048611in}{0.000000in}}%
\pgfusepath{stroke,fill}%
}%
\begin{pgfscope}%
\pgfsys@transformshift{0.800000in}{2.044297in}%
\pgfsys@useobject{currentmarker}{}%
\end{pgfscope}%
\end{pgfscope}%
\begin{pgfscope}%
\definecolor{textcolor}{rgb}{0.000000,0.000000,0.000000}%
\pgfsetstrokecolor{textcolor}%
\pgfsetfillcolor{textcolor}%
\pgftext[x=0.526047in, y=1.991536in, left, base]{\color{textcolor}\sffamily\fontsize{10.000000}{12.000000}\selectfont 12}%
\end{pgfscope}%
\begin{pgfscope}%
\pgfpathrectangle{\pgfqpoint{0.800000in}{0.528000in}}{\pgfqpoint{4.960000in}{3.696000in}}%
\pgfusepath{clip}%
\pgfsetrectcap%
\pgfsetroundjoin%
\pgfsetlinewidth{0.803000pt}%
\definecolor{currentstroke}{rgb}{0.690196,0.690196,0.690196}%
\pgfsetstrokecolor{currentstroke}%
\pgfsetdash{}{0pt}%
\pgfpathmoveto{\pgfqpoint{0.800000in}{2.581468in}}%
\pgfpathlineto{\pgfqpoint{5.760000in}{2.581468in}}%
\pgfusepath{stroke}%
\end{pgfscope}%
\begin{pgfscope}%
\pgfsetbuttcap%
\pgfsetroundjoin%
\definecolor{currentfill}{rgb}{0.000000,0.000000,0.000000}%
\pgfsetfillcolor{currentfill}%
\pgfsetlinewidth{0.803000pt}%
\definecolor{currentstroke}{rgb}{0.000000,0.000000,0.000000}%
\pgfsetstrokecolor{currentstroke}%
\pgfsetdash{}{0pt}%
\pgfsys@defobject{currentmarker}{\pgfqpoint{-0.048611in}{0.000000in}}{\pgfqpoint{-0.000000in}{0.000000in}}{%
\pgfpathmoveto{\pgfqpoint{-0.000000in}{0.000000in}}%
\pgfpathlineto{\pgfqpoint{-0.048611in}{0.000000in}}%
\pgfusepath{stroke,fill}%
}%
\begin{pgfscope}%
\pgfsys@transformshift{0.800000in}{2.581468in}%
\pgfsys@useobject{currentmarker}{}%
\end{pgfscope}%
\end{pgfscope}%
\begin{pgfscope}%
\definecolor{textcolor}{rgb}{0.000000,0.000000,0.000000}%
\pgfsetstrokecolor{textcolor}%
\pgfsetfillcolor{textcolor}%
\pgftext[x=0.526047in, y=2.528706in, left, base]{\color{textcolor}\sffamily\fontsize{10.000000}{12.000000}\selectfont 14}%
\end{pgfscope}%
\begin{pgfscope}%
\pgfpathrectangle{\pgfqpoint{0.800000in}{0.528000in}}{\pgfqpoint{4.960000in}{3.696000in}}%
\pgfusepath{clip}%
\pgfsetrectcap%
\pgfsetroundjoin%
\pgfsetlinewidth{0.803000pt}%
\definecolor{currentstroke}{rgb}{0.690196,0.690196,0.690196}%
\pgfsetstrokecolor{currentstroke}%
\pgfsetdash{}{0pt}%
\pgfpathmoveto{\pgfqpoint{0.800000in}{3.118638in}}%
\pgfpathlineto{\pgfqpoint{5.760000in}{3.118638in}}%
\pgfusepath{stroke}%
\end{pgfscope}%
\begin{pgfscope}%
\pgfsetbuttcap%
\pgfsetroundjoin%
\definecolor{currentfill}{rgb}{0.000000,0.000000,0.000000}%
\pgfsetfillcolor{currentfill}%
\pgfsetlinewidth{0.803000pt}%
\definecolor{currentstroke}{rgb}{0.000000,0.000000,0.000000}%
\pgfsetstrokecolor{currentstroke}%
\pgfsetdash{}{0pt}%
\pgfsys@defobject{currentmarker}{\pgfqpoint{-0.048611in}{0.000000in}}{\pgfqpoint{-0.000000in}{0.000000in}}{%
\pgfpathmoveto{\pgfqpoint{-0.000000in}{0.000000in}}%
\pgfpathlineto{\pgfqpoint{-0.048611in}{0.000000in}}%
\pgfusepath{stroke,fill}%
}%
\begin{pgfscope}%
\pgfsys@transformshift{0.800000in}{3.118638in}%
\pgfsys@useobject{currentmarker}{}%
\end{pgfscope}%
\end{pgfscope}%
\begin{pgfscope}%
\definecolor{textcolor}{rgb}{0.000000,0.000000,0.000000}%
\pgfsetstrokecolor{textcolor}%
\pgfsetfillcolor{textcolor}%
\pgftext[x=0.526047in, y=3.065876in, left, base]{\color{textcolor}\sffamily\fontsize{10.000000}{12.000000}\selectfont 16}%
\end{pgfscope}%
\begin{pgfscope}%
\pgfpathrectangle{\pgfqpoint{0.800000in}{0.528000in}}{\pgfqpoint{4.960000in}{3.696000in}}%
\pgfusepath{clip}%
\pgfsetrectcap%
\pgfsetroundjoin%
\pgfsetlinewidth{0.803000pt}%
\definecolor{currentstroke}{rgb}{0.690196,0.690196,0.690196}%
\pgfsetstrokecolor{currentstroke}%
\pgfsetdash{}{0pt}%
\pgfpathmoveto{\pgfqpoint{0.800000in}{3.655808in}}%
\pgfpathlineto{\pgfqpoint{5.760000in}{3.655808in}}%
\pgfusepath{stroke}%
\end{pgfscope}%
\begin{pgfscope}%
\pgfsetbuttcap%
\pgfsetroundjoin%
\definecolor{currentfill}{rgb}{0.000000,0.000000,0.000000}%
\pgfsetfillcolor{currentfill}%
\pgfsetlinewidth{0.803000pt}%
\definecolor{currentstroke}{rgb}{0.000000,0.000000,0.000000}%
\pgfsetstrokecolor{currentstroke}%
\pgfsetdash{}{0pt}%
\pgfsys@defobject{currentmarker}{\pgfqpoint{-0.048611in}{0.000000in}}{\pgfqpoint{-0.000000in}{0.000000in}}{%
\pgfpathmoveto{\pgfqpoint{-0.000000in}{0.000000in}}%
\pgfpathlineto{\pgfqpoint{-0.048611in}{0.000000in}}%
\pgfusepath{stroke,fill}%
}%
\begin{pgfscope}%
\pgfsys@transformshift{0.800000in}{3.655808in}%
\pgfsys@useobject{currentmarker}{}%
\end{pgfscope}%
\end{pgfscope}%
\begin{pgfscope}%
\definecolor{textcolor}{rgb}{0.000000,0.000000,0.000000}%
\pgfsetstrokecolor{textcolor}%
\pgfsetfillcolor{textcolor}%
\pgftext[x=0.526047in, y=3.603047in, left, base]{\color{textcolor}\sffamily\fontsize{10.000000}{12.000000}\selectfont 18}%
\end{pgfscope}%
\begin{pgfscope}%
\pgfpathrectangle{\pgfqpoint{0.800000in}{0.528000in}}{\pgfqpoint{4.960000in}{3.696000in}}%
\pgfusepath{clip}%
\pgfsetrectcap%
\pgfsetroundjoin%
\pgfsetlinewidth{0.803000pt}%
\definecolor{currentstroke}{rgb}{0.690196,0.690196,0.690196}%
\pgfsetstrokecolor{currentstroke}%
\pgfsetdash{}{0pt}%
\pgfpathmoveto{\pgfqpoint{0.800000in}{4.192978in}}%
\pgfpathlineto{\pgfqpoint{5.760000in}{4.192978in}}%
\pgfusepath{stroke}%
\end{pgfscope}%
\begin{pgfscope}%
\pgfsetbuttcap%
\pgfsetroundjoin%
\definecolor{currentfill}{rgb}{0.000000,0.000000,0.000000}%
\pgfsetfillcolor{currentfill}%
\pgfsetlinewidth{0.803000pt}%
\definecolor{currentstroke}{rgb}{0.000000,0.000000,0.000000}%
\pgfsetstrokecolor{currentstroke}%
\pgfsetdash{}{0pt}%
\pgfsys@defobject{currentmarker}{\pgfqpoint{-0.048611in}{0.000000in}}{\pgfqpoint{-0.000000in}{0.000000in}}{%
\pgfpathmoveto{\pgfqpoint{-0.000000in}{0.000000in}}%
\pgfpathlineto{\pgfqpoint{-0.048611in}{0.000000in}}%
\pgfusepath{stroke,fill}%
}%
\begin{pgfscope}%
\pgfsys@transformshift{0.800000in}{4.192978in}%
\pgfsys@useobject{currentmarker}{}%
\end{pgfscope}%
\end{pgfscope}%
\begin{pgfscope}%
\definecolor{textcolor}{rgb}{0.000000,0.000000,0.000000}%
\pgfsetstrokecolor{textcolor}%
\pgfsetfillcolor{textcolor}%
\pgftext[x=0.526047in, y=4.140217in, left, base]{\color{textcolor}\sffamily\fontsize{10.000000}{12.000000}\selectfont 20}%
\end{pgfscope}%
\begin{pgfscope}%
\definecolor{textcolor}{rgb}{0.000000,0.000000,0.000000}%
\pgfsetstrokecolor{textcolor}%
\pgfsetfillcolor{textcolor}%
\pgftext[x=0.470492in,y=2.376000in,,bottom,rotate=90.000000]{\color{textcolor}\sffamily\fontsize{10.000000}{12.000000}\selectfont Heading [deg]}%
\end{pgfscope}%
\begin{pgfscope}%
\pgfpathrectangle{\pgfqpoint{0.800000in}{0.528000in}}{\pgfqpoint{4.960000in}{3.696000in}}%
\pgfusepath{clip}%
\pgfsetrectcap%
\pgfsetroundjoin%
\pgfsetlinewidth{1.505625pt}%
\definecolor{currentstroke}{rgb}{0.121569,0.466667,0.705882}%
\pgfsetstrokecolor{currentstroke}%
\pgfsetdash{}{0pt}%
\pgfpathmoveto{\pgfqpoint{1.025455in}{0.706743in}}%
\pgfpathlineto{\pgfqpoint{1.063807in}{0.720173in}}%
\pgfpathlineto{\pgfqpoint{1.138689in}{0.849094in}}%
\pgfpathlineto{\pgfqpoint{1.212835in}{1.074705in}}%
\pgfpathlineto{\pgfqpoint{1.286293in}{1.316432in}}%
\pgfpathlineto{\pgfqpoint{1.360494in}{1.587703in}}%
\pgfpathlineto{\pgfqpoint{1.434923in}{1.840173in}}%
\pgfpathlineto{\pgfqpoint{1.508995in}{2.052355in}}%
\pgfpathlineto{\pgfqpoint{1.584752in}{2.248422in}}%
\pgfpathlineto{\pgfqpoint{1.662031in}{2.433746in}}%
\pgfpathlineto{\pgfqpoint{1.735165in}{2.581468in}}%
\pgfpathlineto{\pgfqpoint{1.808694in}{2.707703in}}%
\pgfpathlineto{\pgfqpoint{1.882920in}{2.825880in}}%
\pgfpathlineto{\pgfqpoint{1.957495in}{2.927942in}}%
\pgfpathlineto{\pgfqpoint{2.032291in}{3.011204in}}%
\pgfpathlineto{\pgfqpoint{2.106575in}{3.086408in}}%
\pgfpathlineto{\pgfqpoint{2.181248in}{3.156240in}}%
\pgfpathlineto{\pgfqpoint{2.255368in}{3.218014in}}%
\pgfpathlineto{\pgfqpoint{2.330471in}{3.271731in}}%
\pgfpathlineto{\pgfqpoint{2.404691in}{3.322763in}}%
\pgfpathlineto{\pgfqpoint{2.479207in}{3.368422in}}%
\pgfpathlineto{\pgfqpoint{2.555411in}{3.411396in}}%
\pgfpathlineto{\pgfqpoint{2.629606in}{3.448998in}}%
\pgfpathlineto{\pgfqpoint{2.703104in}{3.486600in}}%
\pgfpathlineto{\pgfqpoint{2.777292in}{3.521516in}}%
\pgfpathlineto{\pgfqpoint{2.852751in}{3.553746in}}%
\pgfpathlineto{\pgfqpoint{2.927608in}{3.585976in}}%
\pgfpathlineto{\pgfqpoint{3.001555in}{3.612835in}}%
\pgfpathlineto{\pgfqpoint{3.076794in}{3.634321in}}%
\pgfpathlineto{\pgfqpoint{3.151007in}{3.645065in}}%
\pgfpathlineto{\pgfqpoint{3.227248in}{3.661180in}}%
\pgfpathlineto{\pgfqpoint{3.301410in}{3.674609in}}%
\pgfpathlineto{\pgfqpoint{3.375998in}{3.690724in}}%
\pgfpathlineto{\pgfqpoint{3.452370in}{3.701468in}}%
\pgfpathlineto{\pgfqpoint{3.527983in}{3.712211in}}%
\pgfpathlineto{\pgfqpoint{3.599009in}{3.717583in}}%
\pgfpathlineto{\pgfqpoint{3.673289in}{3.722954in}}%
\pgfpathlineto{\pgfqpoint{3.747802in}{3.728326in}}%
\pgfpathlineto{\pgfqpoint{3.822038in}{3.736384in}}%
\pgfpathlineto{\pgfqpoint{3.896411in}{3.739070in}}%
\pgfpathlineto{\pgfqpoint{3.970918in}{3.741755in}}%
\pgfpathlineto{\pgfqpoint{4.045652in}{3.741755in}}%
\pgfpathlineto{\pgfqpoint{4.119791in}{3.744441in}}%
\pgfpathlineto{\pgfqpoint{4.193816in}{3.744441in}}%
\pgfpathlineto{\pgfqpoint{4.268642in}{3.744441in}}%
\pgfpathlineto{\pgfqpoint{4.345114in}{3.744441in}}%
\pgfpathlineto{\pgfqpoint{4.418039in}{3.744441in}}%
\pgfpathlineto{\pgfqpoint{4.492440in}{3.741755in}}%
\pgfpathlineto{\pgfqpoint{4.566331in}{3.741755in}}%
\pgfpathlineto{\pgfqpoint{4.640612in}{3.741755in}}%
\pgfpathlineto{\pgfqpoint{4.715028in}{3.739070in}}%
\pgfpathlineto{\pgfqpoint{4.789574in}{3.736384in}}%
\pgfpathlineto{\pgfqpoint{4.864101in}{3.736384in}}%
\pgfpathlineto{\pgfqpoint{4.938540in}{3.733698in}}%
\pgfpathlineto{\pgfqpoint{5.012377in}{3.731012in}}%
\pgfpathlineto{\pgfqpoint{5.086863in}{3.725640in}}%
\pgfpathlineto{\pgfqpoint{5.161582in}{3.722954in}}%
\pgfpathlineto{\pgfqpoint{5.237903in}{3.717583in}}%
\pgfpathlineto{\pgfqpoint{5.311101in}{3.709525in}}%
\pgfpathlineto{\pgfqpoint{5.386074in}{3.709525in}}%
\pgfpathlineto{\pgfqpoint{5.459923in}{3.714897in}}%
\pgfpathlineto{\pgfqpoint{5.534545in}{3.714897in}}%
\pgfusepath{stroke}%
\end{pgfscope}%
\begin{pgfscope}%
\pgfpathrectangle{\pgfqpoint{0.800000in}{0.528000in}}{\pgfqpoint{4.960000in}{3.696000in}}%
\pgfusepath{clip}%
\pgfsetrectcap%
\pgfsetroundjoin%
\pgfsetlinewidth{1.505625pt}%
\definecolor{currentstroke}{rgb}{1.000000,0.498039,0.054902}%
\pgfsetstrokecolor{currentstroke}%
\pgfsetdash{}{0pt}%
\pgfpathmoveto{\pgfqpoint{1.025455in}{0.696000in}}%
\pgfpathlineto{\pgfqpoint{1.099067in}{0.830293in}}%
\pgfpathlineto{\pgfqpoint{1.172823in}{1.133794in}}%
\pgfpathlineto{\pgfqpoint{1.246593in}{1.568902in}}%
\pgfpathlineto{\pgfqpoint{1.322459in}{2.036240in}}%
\pgfpathlineto{\pgfqpoint{1.396727in}{2.479405in}}%
\pgfpathlineto{\pgfqpoint{1.473435in}{2.841995in}}%
\pgfpathlineto{\pgfqpoint{1.546479in}{3.126695in}}%
\pgfpathlineto{\pgfqpoint{1.621080in}{3.357679in}}%
\pgfpathlineto{\pgfqpoint{1.696742in}{3.508086in}}%
\pgfpathlineto{\pgfqpoint{1.770997in}{3.602091in}}%
\pgfpathlineto{\pgfqpoint{1.848657in}{3.663866in}}%
\pgfpathlineto{\pgfqpoint{1.920302in}{3.696096in}}%
\pgfpathlineto{\pgfqpoint{1.994163in}{3.714897in}}%
\pgfpathlineto{\pgfqpoint{2.068175in}{3.720269in}}%
\pgfpathlineto{\pgfqpoint{2.142198in}{3.725640in}}%
\pgfpathlineto{\pgfqpoint{2.216474in}{3.728326in}}%
\pgfpathlineto{\pgfqpoint{2.291342in}{3.733698in}}%
\pgfpathlineto{\pgfqpoint{2.365521in}{3.739070in}}%
\pgfpathlineto{\pgfqpoint{2.440012in}{3.744441in}}%
\pgfpathlineto{\pgfqpoint{2.513997in}{3.747127in}}%
\pgfpathlineto{\pgfqpoint{2.589952in}{3.757871in}}%
\pgfpathlineto{\pgfqpoint{2.663420in}{3.765928in}}%
\pgfpathlineto{\pgfqpoint{2.737944in}{3.776671in}}%
\pgfpathlineto{\pgfqpoint{2.812320in}{3.784729in}}%
\pgfpathlineto{\pgfqpoint{2.886400in}{3.795472in}}%
\pgfpathlineto{\pgfqpoint{2.960971in}{3.803530in}}%
\pgfpathlineto{\pgfqpoint{3.035319in}{3.806216in}}%
\pgfpathlineto{\pgfqpoint{3.109983in}{3.808902in}}%
\pgfpathlineto{\pgfqpoint{3.185917in}{3.808902in}}%
\pgfpathlineto{\pgfqpoint{3.258235in}{3.803530in}}%
\pgfpathlineto{\pgfqpoint{3.334839in}{3.798158in}}%
\pgfpathlineto{\pgfqpoint{3.409040in}{3.792787in}}%
\pgfpathlineto{\pgfqpoint{3.483311in}{3.784729in}}%
\pgfpathlineto{\pgfqpoint{3.557682in}{3.776671in}}%
\pgfpathlineto{\pgfqpoint{3.635410in}{3.768614in}}%
\pgfpathlineto{\pgfqpoint{3.706113in}{3.763242in}}%
\pgfpathlineto{\pgfqpoint{3.780735in}{3.757871in}}%
\pgfpathlineto{\pgfqpoint{3.855108in}{3.752499in}}%
\pgfpathlineto{\pgfqpoint{3.929708in}{3.749813in}}%
\pgfpathlineto{\pgfqpoint{4.003734in}{3.744441in}}%
\pgfpathlineto{\pgfqpoint{4.078207in}{3.741755in}}%
\pgfpathlineto{\pgfqpoint{4.152372in}{3.739070in}}%
\pgfpathlineto{\pgfqpoint{4.228302in}{3.739070in}}%
\pgfpathlineto{\pgfqpoint{4.302367in}{3.744441in}}%
\pgfpathlineto{\pgfqpoint{4.375688in}{3.749813in}}%
\pgfpathlineto{\pgfqpoint{4.451228in}{3.755185in}}%
\pgfpathlineto{\pgfqpoint{4.525797in}{3.760556in}}%
\pgfpathlineto{\pgfqpoint{4.600718in}{3.765928in}}%
\pgfpathlineto{\pgfqpoint{4.675469in}{3.768614in}}%
\pgfpathlineto{\pgfqpoint{4.752587in}{3.776671in}}%
\pgfpathlineto{\pgfqpoint{4.828081in}{3.779357in}}%
\pgfpathlineto{\pgfqpoint{4.902364in}{3.782043in}}%
\pgfpathlineto{\pgfqpoint{4.976596in}{3.787415in}}%
\pgfpathlineto{\pgfqpoint{5.051701in}{3.784729in}}%
\pgfpathlineto{\pgfqpoint{5.125177in}{3.776671in}}%
\pgfpathlineto{\pgfqpoint{5.199714in}{3.768614in}}%
\pgfpathlineto{\pgfqpoint{5.273899in}{3.768614in}}%
\pgfpathlineto{\pgfqpoint{5.348033in}{3.776671in}}%
\pgfpathlineto{\pgfqpoint{5.422456in}{3.790101in}}%
\pgfpathlineto{\pgfqpoint{5.497007in}{3.806216in}}%
\pgfusepath{stroke}%
\end{pgfscope}%
\begin{pgfscope}%
\pgfpathrectangle{\pgfqpoint{0.800000in}{0.528000in}}{\pgfqpoint{4.960000in}{3.696000in}}%
\pgfusepath{clip}%
\pgfsetrectcap%
\pgfsetroundjoin%
\pgfsetlinewidth{1.505625pt}%
\definecolor{currentstroke}{rgb}{0.172549,0.627451,0.172549}%
\pgfsetstrokecolor{currentstroke}%
\pgfsetdash{}{0pt}%
\pgfpathmoveto{\pgfqpoint{1.025455in}{0.696000in}}%
\pgfpathlineto{\pgfqpoint{1.099824in}{0.835664in}}%
\pgfpathlineto{\pgfqpoint{1.175368in}{1.174082in}}%
\pgfpathlineto{\pgfqpoint{1.250734in}{1.617247in}}%
\pgfpathlineto{\pgfqpoint{1.324164in}{2.106072in}}%
\pgfpathlineto{\pgfqpoint{1.398714in}{2.576096in}}%
\pgfpathlineto{\pgfqpoint{1.473193in}{3.038062in}}%
\pgfpathlineto{\pgfqpoint{1.547047in}{3.371108in}}%
\pgfpathlineto{\pgfqpoint{1.621619in}{3.626264in}}%
\pgfpathlineto{\pgfqpoint{1.695733in}{3.763242in}}%
\pgfpathlineto{\pgfqpoint{1.770407in}{3.806216in}}%
\pgfpathlineto{\pgfqpoint{1.847793in}{3.790101in}}%
\pgfpathlineto{\pgfqpoint{1.918976in}{3.749813in}}%
\pgfpathlineto{\pgfqpoint{1.995825in}{3.690724in}}%
\pgfpathlineto{\pgfqpoint{2.071419in}{3.626264in}}%
\pgfpathlineto{\pgfqpoint{2.144728in}{3.569861in}}%
\pgfpathlineto{\pgfqpoint{2.218537in}{3.529573in}}%
\pgfpathlineto{\pgfqpoint{2.293167in}{3.510772in}}%
\pgfpathlineto{\pgfqpoint{2.367570in}{3.497343in}}%
\pgfpathlineto{\pgfqpoint{2.441957in}{3.491971in}}%
\pgfpathlineto{\pgfqpoint{2.516456in}{3.491971in}}%
\pgfpathlineto{\pgfqpoint{2.590956in}{3.489285in}}%
\pgfpathlineto{\pgfqpoint{2.665462in}{3.489285in}}%
\pgfpathlineto{\pgfqpoint{2.739194in}{3.491971in}}%
\pgfpathlineto{\pgfqpoint{2.814185in}{3.497343in}}%
\pgfpathlineto{\pgfqpoint{2.889918in}{3.505400in}}%
\pgfpathlineto{\pgfqpoint{2.963700in}{3.518830in}}%
\pgfpathlineto{\pgfqpoint{3.037258in}{3.529573in}}%
\pgfpathlineto{\pgfqpoint{3.112152in}{3.537631in}}%
\pgfpathlineto{\pgfqpoint{3.186417in}{3.551060in}}%
\pgfpathlineto{\pgfqpoint{3.260642in}{3.564489in}}%
\pgfpathlineto{\pgfqpoint{3.335015in}{3.577918in}}%
\pgfpathlineto{\pgfqpoint{3.409349in}{3.585976in}}%
\pgfpathlineto{\pgfqpoint{3.484590in}{3.588662in}}%
\pgfpathlineto{\pgfqpoint{3.558022in}{3.588662in}}%
\pgfpathlineto{\pgfqpoint{3.632308in}{3.591348in}}%
\pgfpathlineto{\pgfqpoint{3.706894in}{3.594034in}}%
\pgfpathlineto{\pgfqpoint{3.783333in}{3.594034in}}%
\pgfpathlineto{\pgfqpoint{3.856942in}{3.585976in}}%
\pgfpathlineto{\pgfqpoint{3.930686in}{3.577918in}}%
\pgfpathlineto{\pgfqpoint{4.004579in}{3.572547in}}%
\pgfpathlineto{\pgfqpoint{4.078500in}{3.580604in}}%
\pgfpathlineto{\pgfqpoint{4.154011in}{3.594034in}}%
\pgfpathlineto{\pgfqpoint{4.228644in}{3.604777in}}%
\pgfpathlineto{\pgfqpoint{4.302594in}{3.623578in}}%
\pgfpathlineto{\pgfqpoint{4.377267in}{3.639693in}}%
\pgfpathlineto{\pgfqpoint{4.451307in}{3.650436in}}%
\pgfpathlineto{\pgfqpoint{4.525813in}{3.658494in}}%
\pgfpathlineto{\pgfqpoint{4.601026in}{3.663866in}}%
\pgfpathlineto{\pgfqpoint{4.675062in}{3.669237in}}%
\pgfpathlineto{\pgfqpoint{4.750808in}{3.674609in}}%
\pgfpathlineto{\pgfqpoint{4.824314in}{3.677295in}}%
\pgfpathlineto{\pgfqpoint{4.898125in}{3.679981in}}%
\pgfpathlineto{\pgfqpoint{4.972539in}{3.685353in}}%
\pgfpathlineto{\pgfqpoint{5.047642in}{3.693410in}}%
\pgfpathlineto{\pgfqpoint{5.121926in}{3.709525in}}%
\pgfpathlineto{\pgfqpoint{5.196093in}{3.725640in}}%
\pgfpathlineto{\pgfqpoint{5.270461in}{3.739070in}}%
\pgfpathlineto{\pgfqpoint{5.344890in}{3.741755in}}%
\pgfpathlineto{\pgfqpoint{5.419185in}{3.747127in}}%
\pgfpathlineto{\pgfqpoint{5.493626in}{3.752499in}}%
\pgfusepath{stroke}%
\end{pgfscope}%
\begin{pgfscope}%
\pgfpathrectangle{\pgfqpoint{0.800000in}{0.528000in}}{\pgfqpoint{4.960000in}{3.696000in}}%
\pgfusepath{clip}%
\pgfsetrectcap%
\pgfsetroundjoin%
\pgfsetlinewidth{1.505625pt}%
\definecolor{currentstroke}{rgb}{0.839216,0.152941,0.156863}%
\pgfsetstrokecolor{currentstroke}%
\pgfsetdash{}{0pt}%
\pgfpathmoveto{\pgfqpoint{1.025455in}{0.701372in}}%
\pgfpathlineto{\pgfqpoint{1.099364in}{0.830293in}}%
\pgfpathlineto{\pgfqpoint{1.173463in}{1.163338in}}%
\pgfpathlineto{\pgfqpoint{1.247490in}{1.590388in}}%
\pgfpathlineto{\pgfqpoint{1.322184in}{2.049669in}}%
\pgfpathlineto{\pgfqpoint{1.396363in}{2.557295in}}%
\pgfpathlineto{\pgfqpoint{1.470989in}{3.062235in}}%
\pgfpathlineto{\pgfqpoint{1.545509in}{3.467799in}}%
\pgfpathlineto{\pgfqpoint{1.619641in}{3.784729in}}%
\pgfpathlineto{\pgfqpoint{1.693921in}{3.940508in}}%
\pgfpathlineto{\pgfqpoint{1.768886in}{3.961995in}}%
\pgfpathlineto{\pgfqpoint{1.842952in}{3.900221in}}%
\pgfpathlineto{\pgfqpoint{1.919041in}{3.782043in}}%
\pgfpathlineto{\pgfqpoint{1.992332in}{3.661180in}}%
\pgfpathlineto{\pgfqpoint{2.066213in}{3.567175in}}%
\pgfpathlineto{\pgfqpoint{2.140371in}{3.502715in}}%
\pgfpathlineto{\pgfqpoint{2.214782in}{3.478542in}}%
\pgfpathlineto{\pgfqpoint{2.288928in}{3.478542in}}%
\pgfpathlineto{\pgfqpoint{2.363307in}{3.500029in}}%
\pgfpathlineto{\pgfqpoint{2.437539in}{3.526887in}}%
\pgfpathlineto{\pgfqpoint{2.512156in}{3.556432in}}%
\pgfpathlineto{\pgfqpoint{2.586680in}{3.580604in}}%
\pgfpathlineto{\pgfqpoint{2.661652in}{3.599405in}}%
\pgfpathlineto{\pgfqpoint{2.735749in}{3.610149in}}%
\pgfpathlineto{\pgfqpoint{2.809602in}{3.610149in}}%
\pgfpathlineto{\pgfqpoint{2.883838in}{3.591348in}}%
\pgfpathlineto{\pgfqpoint{2.958319in}{3.572547in}}%
\pgfpathlineto{\pgfqpoint{3.032549in}{3.553746in}}%
\pgfpathlineto{\pgfqpoint{3.107006in}{3.548374in}}%
\pgfpathlineto{\pgfqpoint{3.181440in}{3.551060in}}%
\pgfpathlineto{\pgfqpoint{3.255767in}{3.559118in}}%
\pgfpathlineto{\pgfqpoint{3.329981in}{3.575233in}}%
\pgfpathlineto{\pgfqpoint{3.406303in}{3.596719in}}%
\pgfpathlineto{\pgfqpoint{3.482751in}{3.628950in}}%
\pgfpathlineto{\pgfqpoint{3.555317in}{3.658494in}}%
\pgfpathlineto{\pgfqpoint{3.629320in}{3.688038in}}%
\pgfpathlineto{\pgfqpoint{3.703480in}{3.714897in}}%
\pgfpathlineto{\pgfqpoint{3.778027in}{3.736384in}}%
\pgfpathlineto{\pgfqpoint{3.852264in}{3.747127in}}%
\pgfpathlineto{\pgfqpoint{3.927050in}{3.752499in}}%
\pgfpathlineto{\pgfqpoint{4.002198in}{3.755185in}}%
\pgfpathlineto{\pgfqpoint{4.076525in}{3.749813in}}%
\pgfpathlineto{\pgfqpoint{4.150513in}{3.739070in}}%
\pgfpathlineto{\pgfqpoint{4.225575in}{3.728326in}}%
\pgfpathlineto{\pgfqpoint{4.302402in}{3.717583in}}%
\pgfpathlineto{\pgfqpoint{4.375531in}{3.709525in}}%
\pgfpathlineto{\pgfqpoint{4.449413in}{3.704153in}}%
\pgfpathlineto{\pgfqpoint{4.523592in}{3.693410in}}%
\pgfpathlineto{\pgfqpoint{4.598120in}{3.679981in}}%
\pgfpathlineto{\pgfqpoint{4.672196in}{3.671923in}}%
\pgfpathlineto{\pgfqpoint{4.747452in}{3.685353in}}%
\pgfpathlineto{\pgfqpoint{4.822039in}{3.706839in}}%
\pgfpathlineto{\pgfqpoint{4.897357in}{3.744441in}}%
\pgfpathlineto{\pgfqpoint{4.971388in}{3.787415in}}%
\pgfpathlineto{\pgfqpoint{5.045497in}{3.833074in}}%
\pgfpathlineto{\pgfqpoint{5.120047in}{3.873362in}}%
\pgfpathlineto{\pgfqpoint{5.197082in}{3.910964in}}%
\pgfpathlineto{\pgfqpoint{5.270517in}{3.935137in}}%
\pgfpathlineto{\pgfqpoint{5.345016in}{3.951252in}}%
\pgfpathlineto{\pgfqpoint{5.419524in}{3.953938in}}%
\pgfpathlineto{\pgfqpoint{5.493599in}{3.945880in}}%
\pgfusepath{stroke}%
\end{pgfscope}%
\begin{pgfscope}%
\pgfpathrectangle{\pgfqpoint{0.800000in}{0.528000in}}{\pgfqpoint{4.960000in}{3.696000in}}%
\pgfusepath{clip}%
\pgfsetrectcap%
\pgfsetroundjoin%
\pgfsetlinewidth{1.505625pt}%
\definecolor{currentstroke}{rgb}{0.580392,0.403922,0.741176}%
\pgfsetstrokecolor{currentstroke}%
\pgfsetdash{}{0pt}%
\pgfpathmoveto{\pgfqpoint{1.025455in}{0.701372in}}%
\pgfpathlineto{\pgfqpoint{1.100470in}{0.841036in}}%
\pgfpathlineto{\pgfqpoint{1.176378in}{1.171396in}}%
\pgfpathlineto{\pgfqpoint{1.250865in}{1.611875in}}%
\pgfpathlineto{\pgfqpoint{1.324834in}{2.079213in}}%
\pgfpathlineto{\pgfqpoint{1.398474in}{2.570724in}}%
\pgfpathlineto{\pgfqpoint{1.472807in}{3.054177in}}%
\pgfpathlineto{\pgfqpoint{1.547834in}{3.521516in}}%
\pgfpathlineto{\pgfqpoint{1.622068in}{3.865305in}}%
\pgfpathlineto{\pgfqpoint{1.696680in}{4.034513in}}%
\pgfpathlineto{\pgfqpoint{1.771657in}{4.056000in}}%
\pgfpathlineto{\pgfqpoint{1.846512in}{3.951252in}}%
\pgfpathlineto{\pgfqpoint{1.921313in}{3.776671in}}%
\pgfpathlineto{\pgfqpoint{1.997965in}{3.594034in}}%
\pgfpathlineto{\pgfqpoint{2.071739in}{3.473170in}}%
\pgfpathlineto{\pgfqpoint{2.145849in}{3.400652in}}%
\pgfpathlineto{\pgfqpoint{2.220040in}{3.381851in}}%
\pgfpathlineto{\pgfqpoint{2.294057in}{3.408710in}}%
\pgfpathlineto{\pgfqpoint{2.370736in}{3.478542in}}%
\pgfpathlineto{\pgfqpoint{2.443144in}{3.548374in}}%
\pgfpathlineto{\pgfqpoint{2.517953in}{3.618206in}}%
\pgfpathlineto{\pgfqpoint{2.591959in}{3.669237in}}%
\pgfpathlineto{\pgfqpoint{2.666301in}{3.706839in}}%
\pgfpathlineto{\pgfqpoint{2.740975in}{3.728326in}}%
\pgfpathlineto{\pgfqpoint{2.816885in}{3.720269in}}%
\pgfpathlineto{\pgfqpoint{2.890126in}{3.671923in}}%
\pgfpathlineto{\pgfqpoint{2.964049in}{3.604777in}}%
\pgfpathlineto{\pgfqpoint{3.038333in}{3.532259in}}%
\pgfpathlineto{\pgfqpoint{3.112811in}{3.486600in}}%
\pgfpathlineto{\pgfqpoint{3.187013in}{3.473170in}}%
\pgfpathlineto{\pgfqpoint{3.261789in}{3.481228in}}%
\pgfpathlineto{\pgfqpoint{3.337060in}{3.518830in}}%
\pgfpathlineto{\pgfqpoint{3.411535in}{3.569861in}}%
\pgfpathlineto{\pgfqpoint{3.486495in}{3.634321in}}%
\pgfpathlineto{\pgfqpoint{3.560473in}{3.693410in}}%
\pgfpathlineto{\pgfqpoint{3.634416in}{3.749813in}}%
\pgfpathlineto{\pgfqpoint{3.710522in}{3.784729in}}%
\pgfpathlineto{\pgfqpoint{3.783974in}{3.795472in}}%
\pgfpathlineto{\pgfqpoint{3.857782in}{3.787415in}}%
\pgfpathlineto{\pgfqpoint{3.931661in}{3.768614in}}%
\pgfpathlineto{\pgfqpoint{4.006331in}{3.741755in}}%
\pgfpathlineto{\pgfqpoint{4.080577in}{3.712211in}}%
\pgfpathlineto{\pgfqpoint{4.155559in}{3.682667in}}%
\pgfpathlineto{\pgfqpoint{4.229512in}{3.655808in}}%
\pgfpathlineto{\pgfqpoint{4.304405in}{3.637007in}}%
\pgfpathlineto{\pgfqpoint{4.378201in}{3.653122in}}%
\pgfpathlineto{\pgfqpoint{4.454021in}{3.693410in}}%
\pgfpathlineto{\pgfqpoint{4.528621in}{3.741755in}}%
\pgfpathlineto{\pgfqpoint{4.605285in}{3.790101in}}%
\pgfpathlineto{\pgfqpoint{4.678608in}{3.833074in}}%
\pgfpathlineto{\pgfqpoint{4.752888in}{3.865305in}}%
\pgfpathlineto{\pgfqpoint{4.827075in}{3.884106in}}%
\pgfpathlineto{\pgfqpoint{4.901997in}{3.884106in}}%
\pgfpathlineto{\pgfqpoint{4.977955in}{3.867990in}}%
\pgfpathlineto{\pgfqpoint{5.051662in}{3.822331in}}%
\pgfpathlineto{\pgfqpoint{5.125936in}{3.744441in}}%
\pgfpathlineto{\pgfqpoint{5.200351in}{3.674609in}}%
\pgfpathlineto{\pgfqpoint{5.274279in}{3.626264in}}%
\pgfpathlineto{\pgfqpoint{5.348673in}{3.604777in}}%
\pgfpathlineto{\pgfqpoint{5.423245in}{3.618206in}}%
\pgfpathlineto{\pgfqpoint{5.499111in}{3.661180in}}%
\pgfusepath{stroke}%
\end{pgfscope}%
\begin{pgfscope}%
\pgfpathrectangle{\pgfqpoint{0.800000in}{0.528000in}}{\pgfqpoint{4.960000in}{3.696000in}}%
\pgfusepath{clip}%
\pgfsetrectcap%
\pgfsetroundjoin%
\pgfsetlinewidth{1.505625pt}%
\definecolor{currentstroke}{rgb}{0.549020,0.337255,0.294118}%
\pgfsetstrokecolor{currentstroke}%
\pgfsetdash{}{0pt}%
\pgfpathmoveto{\pgfqpoint{1.025455in}{0.701372in}}%
\pgfpathlineto{\pgfqpoint{1.099620in}{0.830293in}}%
\pgfpathlineto{\pgfqpoint{1.174149in}{1.163338in}}%
\pgfpathlineto{\pgfqpoint{1.248389in}{1.603818in}}%
\pgfpathlineto{\pgfqpoint{1.322944in}{2.052355in}}%
\pgfpathlineto{\pgfqpoint{1.397696in}{2.557295in}}%
\pgfpathlineto{\pgfqpoint{1.472152in}{3.067607in}}%
\pgfpathlineto{\pgfqpoint{1.546810in}{3.556432in}}%
\pgfpathlineto{\pgfqpoint{1.622112in}{3.889477in}}%
\pgfpathlineto{\pgfqpoint{1.696146in}{4.050628in}}%
\pgfpathlineto{\pgfqpoint{1.770403in}{4.029141in}}%
\pgfpathlineto{\pgfqpoint{1.845009in}{3.897535in}}%
\pgfpathlineto{\pgfqpoint{1.919095in}{3.701468in}}%
\pgfpathlineto{\pgfqpoint{1.993646in}{3.516144in}}%
\pgfpathlineto{\pgfqpoint{2.067728in}{3.408710in}}%
\pgfpathlineto{\pgfqpoint{2.142923in}{3.379165in}}%
\pgfpathlineto{\pgfqpoint{2.216782in}{3.424825in}}%
\pgfpathlineto{\pgfqpoint{2.290852in}{3.510772in}}%
\pgfpathlineto{\pgfqpoint{2.365584in}{3.618206in}}%
\pgfpathlineto{\pgfqpoint{2.442579in}{3.720269in}}%
\pgfpathlineto{\pgfqpoint{2.515939in}{3.792787in}}%
\pgfpathlineto{\pgfqpoint{2.589407in}{3.811588in}}%
\pgfpathlineto{\pgfqpoint{2.663517in}{3.782043in}}%
\pgfpathlineto{\pgfqpoint{2.738313in}{3.720269in}}%
\pgfpathlineto{\pgfqpoint{2.812768in}{3.647751in}}%
\pgfpathlineto{\pgfqpoint{2.887563in}{3.567175in}}%
\pgfpathlineto{\pgfqpoint{2.961811in}{3.494657in}}%
\pgfpathlineto{\pgfqpoint{3.036241in}{3.443626in}}%
\pgfpathlineto{\pgfqpoint{3.109806in}{3.424825in}}%
\pgfpathlineto{\pgfqpoint{3.184273in}{3.440940in}}%
\pgfpathlineto{\pgfqpoint{3.259041in}{3.486600in}}%
\pgfpathlineto{\pgfqpoint{3.335888in}{3.556432in}}%
\pgfpathlineto{\pgfqpoint{3.410340in}{3.615520in}}%
\pgfpathlineto{\pgfqpoint{3.483889in}{3.658494in}}%
\pgfpathlineto{\pgfqpoint{3.558769in}{3.677295in}}%
\pgfpathlineto{\pgfqpoint{3.633581in}{3.669237in}}%
\pgfpathlineto{\pgfqpoint{3.708079in}{3.637007in}}%
\pgfpathlineto{\pgfqpoint{3.782596in}{3.594034in}}%
\pgfpathlineto{\pgfqpoint{3.856782in}{3.548374in}}%
\pgfpathlineto{\pgfqpoint{3.931191in}{3.505400in}}%
\pgfpathlineto{\pgfqpoint{4.005692in}{3.486600in}}%
\pgfpathlineto{\pgfqpoint{4.080380in}{3.491971in}}%
\pgfpathlineto{\pgfqpoint{4.155377in}{3.532259in}}%
\pgfpathlineto{\pgfqpoint{4.233004in}{3.588662in}}%
\pgfpathlineto{\pgfqpoint{4.305852in}{3.637007in}}%
\pgfpathlineto{\pgfqpoint{4.379262in}{3.671923in}}%
\pgfpathlineto{\pgfqpoint{4.453286in}{3.690724in}}%
\pgfpathlineto{\pgfqpoint{4.528300in}{3.698782in}}%
\pgfpathlineto{\pgfqpoint{4.603131in}{3.698782in}}%
\pgfpathlineto{\pgfqpoint{4.677857in}{3.685353in}}%
\pgfpathlineto{\pgfqpoint{4.751488in}{3.663866in}}%
\pgfpathlineto{\pgfqpoint{4.826885in}{3.637007in}}%
\pgfpathlineto{\pgfqpoint{4.901354in}{3.612835in}}%
\pgfpathlineto{\pgfqpoint{4.975747in}{3.591348in}}%
\pgfpathlineto{\pgfqpoint{5.050397in}{3.583290in}}%
\pgfpathlineto{\pgfqpoint{5.124837in}{3.585976in}}%
\pgfpathlineto{\pgfqpoint{5.199617in}{3.596719in}}%
\pgfpathlineto{\pgfqpoint{5.274348in}{3.615520in}}%
\pgfpathlineto{\pgfqpoint{5.349444in}{3.623578in}}%
\pgfpathlineto{\pgfqpoint{5.424222in}{3.620892in}}%
\pgfpathlineto{\pgfqpoint{5.498989in}{3.604777in}}%
\pgfusepath{stroke}%
\end{pgfscope}%
\begin{pgfscope}%
\pgfpathrectangle{\pgfqpoint{0.800000in}{0.528000in}}{\pgfqpoint{4.960000in}{3.696000in}}%
\pgfusepath{clip}%
\pgfsetrectcap%
\pgfsetroundjoin%
\pgfsetlinewidth{1.505625pt}%
\definecolor{currentstroke}{rgb}{0.890196,0.466667,0.760784}%
\pgfsetstrokecolor{currentstroke}%
\pgfsetdash{}{0pt}%
\pgfpathmoveto{\pgfqpoint{1.025455in}{3.739051in}}%
\pgfpathlineto{\pgfqpoint{5.487252in}{3.739051in}}%
\pgfusepath{stroke}%
\end{pgfscope}%
\begin{pgfscope}%
\pgfsetrectcap%
\pgfsetmiterjoin%
\pgfsetlinewidth{0.803000pt}%
\definecolor{currentstroke}{rgb}{0.000000,0.000000,0.000000}%
\pgfsetstrokecolor{currentstroke}%
\pgfsetdash{}{0pt}%
\pgfpathmoveto{\pgfqpoint{0.800000in}{0.528000in}}%
\pgfpathlineto{\pgfqpoint{0.800000in}{4.224000in}}%
\pgfusepath{stroke}%
\end{pgfscope}%
\begin{pgfscope}%
\pgfsetrectcap%
\pgfsetmiterjoin%
\pgfsetlinewidth{0.803000pt}%
\definecolor{currentstroke}{rgb}{0.000000,0.000000,0.000000}%
\pgfsetstrokecolor{currentstroke}%
\pgfsetdash{}{0pt}%
\pgfpathmoveto{\pgfqpoint{5.760000in}{0.528000in}}%
\pgfpathlineto{\pgfqpoint{5.760000in}{4.224000in}}%
\pgfusepath{stroke}%
\end{pgfscope}%
\begin{pgfscope}%
\pgfsetrectcap%
\pgfsetmiterjoin%
\pgfsetlinewidth{0.803000pt}%
\definecolor{currentstroke}{rgb}{0.000000,0.000000,0.000000}%
\pgfsetstrokecolor{currentstroke}%
\pgfsetdash{}{0pt}%
\pgfpathmoveto{\pgfqpoint{0.800000in}{0.528000in}}%
\pgfpathlineto{\pgfqpoint{5.760000in}{0.528000in}}%
\pgfusepath{stroke}%
\end{pgfscope}%
\begin{pgfscope}%
\pgfsetrectcap%
\pgfsetmiterjoin%
\pgfsetlinewidth{0.803000pt}%
\definecolor{currentstroke}{rgb}{0.000000,0.000000,0.000000}%
\pgfsetstrokecolor{currentstroke}%
\pgfsetdash{}{0pt}%
\pgfpathmoveto{\pgfqpoint{0.800000in}{4.224000in}}%
\pgfpathlineto{\pgfqpoint{5.760000in}{4.224000in}}%
\pgfusepath{stroke}%
\end{pgfscope}%
\begin{pgfscope}%
\definecolor{textcolor}{rgb}{0.000000,0.000000,0.000000}%
\pgfsetstrokecolor{textcolor}%
\pgfsetfillcolor{textcolor}%
\pgftext[x=3.280000in,y=4.307333in,,base]{\color{textcolor}\sffamily\fontsize{12.000000}{14.400000}\selectfont Measured yaw position}%
\end{pgfscope}%
\begin{pgfscope}%
\pgfsetbuttcap%
\pgfsetmiterjoin%
\definecolor{currentfill}{rgb}{1.000000,1.000000,1.000000}%
\pgfsetfillcolor{currentfill}%
\pgfsetfillopacity{0.800000}%
\pgfsetlinewidth{1.003750pt}%
\definecolor{currentstroke}{rgb}{0.800000,0.800000,0.800000}%
\pgfsetstrokecolor{currentstroke}%
\pgfsetstrokeopacity{0.800000}%
\pgfsetdash{}{0pt}%
\pgfpathmoveto{\pgfqpoint{4.788646in}{0.597444in}}%
\pgfpathlineto{\pgfqpoint{5.662778in}{0.597444in}}%
\pgfpathquadraticcurveto{\pgfqpoint{5.690556in}{0.597444in}}{\pgfqpoint{5.690556in}{0.625222in}}%
\pgfpathlineto{\pgfqpoint{5.690556in}{2.038334in}}%
\pgfpathquadraticcurveto{\pgfqpoint{5.690556in}{2.066112in}}{\pgfqpoint{5.662778in}{2.066112in}}%
\pgfpathlineto{\pgfqpoint{4.788646in}{2.066112in}}%
\pgfpathquadraticcurveto{\pgfqpoint{4.760868in}{2.066112in}}{\pgfqpoint{4.760868in}{2.038334in}}%
\pgfpathlineto{\pgfqpoint{4.760868in}{0.625222in}}%
\pgfpathquadraticcurveto{\pgfqpoint{4.760868in}{0.597444in}}{\pgfqpoint{4.788646in}{0.597444in}}%
\pgfpathlineto{\pgfqpoint{4.788646in}{0.597444in}}%
\pgfpathclose%
\pgfusepath{stroke,fill}%
\end{pgfscope}%
\begin{pgfscope}%
\pgfsetrectcap%
\pgfsetroundjoin%
\pgfsetlinewidth{1.505625pt}%
\definecolor{currentstroke}{rgb}{0.121569,0.466667,0.705882}%
\pgfsetstrokecolor{currentstroke}%
\pgfsetdash{}{0pt}%
\pgfpathmoveto{\pgfqpoint{4.816424in}{1.953644in}}%
\pgfpathlineto{\pgfqpoint{4.955312in}{1.953644in}}%
\pgfpathlineto{\pgfqpoint{5.094201in}{1.953644in}}%
\pgfusepath{stroke}%
\end{pgfscope}%
\begin{pgfscope}%
\definecolor{textcolor}{rgb}{0.000000,0.000000,0.000000}%
\pgfsetstrokecolor{textcolor}%
\pgfsetfillcolor{textcolor}%
\pgftext[x=5.205312in,y=1.905033in,left,base]{\color{textcolor}\sffamily\fontsize{10.000000}{12.000000}\selectfont 25}%
\end{pgfscope}%
\begin{pgfscope}%
\pgfsetrectcap%
\pgfsetroundjoin%
\pgfsetlinewidth{1.505625pt}%
\definecolor{currentstroke}{rgb}{1.000000,0.498039,0.054902}%
\pgfsetstrokecolor{currentstroke}%
\pgfsetdash{}{0pt}%
\pgfpathmoveto{\pgfqpoint{4.816424in}{1.749787in}}%
\pgfpathlineto{\pgfqpoint{4.955312in}{1.749787in}}%
\pgfpathlineto{\pgfqpoint{5.094201in}{1.749787in}}%
\pgfusepath{stroke}%
\end{pgfscope}%
\begin{pgfscope}%
\definecolor{textcolor}{rgb}{0.000000,0.000000,0.000000}%
\pgfsetstrokecolor{textcolor}%
\pgfsetfillcolor{textcolor}%
\pgftext[x=5.205312in,y=1.701176in,left,base]{\color{textcolor}\sffamily\fontsize{10.000000}{12.000000}\selectfont 50}%
\end{pgfscope}%
\begin{pgfscope}%
\pgfsetrectcap%
\pgfsetroundjoin%
\pgfsetlinewidth{1.505625pt}%
\definecolor{currentstroke}{rgb}{0.172549,0.627451,0.172549}%
\pgfsetstrokecolor{currentstroke}%
\pgfsetdash{}{0pt}%
\pgfpathmoveto{\pgfqpoint{4.816424in}{1.545930in}}%
\pgfpathlineto{\pgfqpoint{4.955312in}{1.545930in}}%
\pgfpathlineto{\pgfqpoint{5.094201in}{1.545930in}}%
\pgfusepath{stroke}%
\end{pgfscope}%
\begin{pgfscope}%
\definecolor{textcolor}{rgb}{0.000000,0.000000,0.000000}%
\pgfsetstrokecolor{textcolor}%
\pgfsetfillcolor{textcolor}%
\pgftext[x=5.205312in,y=1.497319in,left,base]{\color{textcolor}\sffamily\fontsize{10.000000}{12.000000}\selectfont 75}%
\end{pgfscope}%
\begin{pgfscope}%
\pgfsetrectcap%
\pgfsetroundjoin%
\pgfsetlinewidth{1.505625pt}%
\definecolor{currentstroke}{rgb}{0.839216,0.152941,0.156863}%
\pgfsetstrokecolor{currentstroke}%
\pgfsetdash{}{0pt}%
\pgfpathmoveto{\pgfqpoint{4.816424in}{1.342073in}}%
\pgfpathlineto{\pgfqpoint{4.955312in}{1.342073in}}%
\pgfpathlineto{\pgfqpoint{5.094201in}{1.342073in}}%
\pgfusepath{stroke}%
\end{pgfscope}%
\begin{pgfscope}%
\definecolor{textcolor}{rgb}{0.000000,0.000000,0.000000}%
\pgfsetstrokecolor{textcolor}%
\pgfsetfillcolor{textcolor}%
\pgftext[x=5.205312in,y=1.293461in,left,base]{\color{textcolor}\sffamily\fontsize{10.000000}{12.000000}\selectfont 100}%
\end{pgfscope}%
\begin{pgfscope}%
\pgfsetrectcap%
\pgfsetroundjoin%
\pgfsetlinewidth{1.505625pt}%
\definecolor{currentstroke}{rgb}{0.580392,0.403922,0.741176}%
\pgfsetstrokecolor{currentstroke}%
\pgfsetdash{}{0pt}%
\pgfpathmoveto{\pgfqpoint{4.816424in}{1.138215in}}%
\pgfpathlineto{\pgfqpoint{4.955312in}{1.138215in}}%
\pgfpathlineto{\pgfqpoint{5.094201in}{1.138215in}}%
\pgfusepath{stroke}%
\end{pgfscope}%
\begin{pgfscope}%
\definecolor{textcolor}{rgb}{0.000000,0.000000,0.000000}%
\pgfsetstrokecolor{textcolor}%
\pgfsetfillcolor{textcolor}%
\pgftext[x=5.205312in,y=1.089604in,left,base]{\color{textcolor}\sffamily\fontsize{10.000000}{12.000000}\selectfont 125}%
\end{pgfscope}%
\begin{pgfscope}%
\pgfsetrectcap%
\pgfsetroundjoin%
\pgfsetlinewidth{1.505625pt}%
\definecolor{currentstroke}{rgb}{0.549020,0.337255,0.294118}%
\pgfsetstrokecolor{currentstroke}%
\pgfsetdash{}{0pt}%
\pgfpathmoveto{\pgfqpoint{4.816424in}{0.934358in}}%
\pgfpathlineto{\pgfqpoint{4.955312in}{0.934358in}}%
\pgfpathlineto{\pgfqpoint{5.094201in}{0.934358in}}%
\pgfusepath{stroke}%
\end{pgfscope}%
\begin{pgfscope}%
\definecolor{textcolor}{rgb}{0.000000,0.000000,0.000000}%
\pgfsetstrokecolor{textcolor}%
\pgfsetfillcolor{textcolor}%
\pgftext[x=5.205312in,y=0.885747in,left,base]{\color{textcolor}\sffamily\fontsize{10.000000}{12.000000}\selectfont 150}%
\end{pgfscope}%
\begin{pgfscope}%
\pgfsetrectcap%
\pgfsetroundjoin%
\pgfsetlinewidth{1.505625pt}%
\definecolor{currentstroke}{rgb}{0.890196,0.466667,0.760784}%
\pgfsetstrokecolor{currentstroke}%
\pgfsetdash{}{0pt}%
\pgfpathmoveto{\pgfqpoint{4.816424in}{0.730501in}}%
\pgfpathlineto{\pgfqpoint{4.955312in}{0.730501in}}%
\pgfpathlineto{\pgfqpoint{5.094201in}{0.730501in}}%
\pgfusepath{stroke}%
\end{pgfscope}%
\begin{pgfscope}%
\definecolor{textcolor}{rgb}{0.000000,0.000000,0.000000}%
\pgfsetstrokecolor{textcolor}%
\pgfsetfillcolor{textcolor}%
\pgftext[x=5.205312in,y=0.681890in,left,base]{\color{textcolor}\sffamily\fontsize{10.000000}{12.000000}\selectfont Target}%
\end{pgfscope}%
\end{pgfpicture}%
\makeatother%
\endgroup%
}
    \end{minipage}
    \begin{minipage}[t]{0.5\linewidth}
        \centering
        \scalebox{0.55}{%% Creator: Matplotlib, PGF backend
%%
%% To include the figure in your LaTeX document, write
%%   \input{<filename>.pgf}
%%
%% Make sure the required packages are loaded in your preamble
%%   \usepackage{pgf}
%%
%% Also ensure that all the required font packages are loaded; for instance,
%% the lmodern package is sometimes necessary when using math font.
%%   \usepackage{lmodern}
%%
%% Figures using additional raster images can only be included by \input if
%% they are in the same directory as the main LaTeX file. For loading figures
%% from other directories you can use the `import` package
%%   \usepackage{import}
%%
%% and then include the figures with
%%   \import{<path to file>}{<filename>.pgf}
%%
%% Matplotlib used the following preamble
%%   \usepackage{fontspec}
%%   \setmainfont{DejaVuSerif.ttf}[Path=\detokenize{/home/lgonz/tfg-aero/tfg-giaa-dronecontrol/venv/lib/python3.8/site-packages/matplotlib/mpl-data/fonts/ttf/}]
%%   \setsansfont{DejaVuSans.ttf}[Path=\detokenize{/home/lgonz/tfg-aero/tfg-giaa-dronecontrol/venv/lib/python3.8/site-packages/matplotlib/mpl-data/fonts/ttf/}]
%%   \setmonofont{DejaVuSansMono.ttf}[Path=\detokenize{/home/lgonz/tfg-aero/tfg-giaa-dronecontrol/venv/lib/python3.8/site-packages/matplotlib/mpl-data/fonts/ttf/}]
%%
\begingroup%
\makeatletter%
\begin{pgfpicture}%
\pgfpathrectangle{\pgfpointorigin}{\pgfqpoint{6.400000in}{4.800000in}}%
\pgfusepath{use as bounding box, clip}%
\begin{pgfscope}%
\pgfsetbuttcap%
\pgfsetmiterjoin%
\definecolor{currentfill}{rgb}{1.000000,1.000000,1.000000}%
\pgfsetfillcolor{currentfill}%
\pgfsetlinewidth{0.000000pt}%
\definecolor{currentstroke}{rgb}{1.000000,1.000000,1.000000}%
\pgfsetstrokecolor{currentstroke}%
\pgfsetdash{}{0pt}%
\pgfpathmoveto{\pgfqpoint{0.000000in}{0.000000in}}%
\pgfpathlineto{\pgfqpoint{6.400000in}{0.000000in}}%
\pgfpathlineto{\pgfqpoint{6.400000in}{4.800000in}}%
\pgfpathlineto{\pgfqpoint{0.000000in}{4.800000in}}%
\pgfpathlineto{\pgfqpoint{0.000000in}{0.000000in}}%
\pgfpathclose%
\pgfusepath{fill}%
\end{pgfscope}%
\begin{pgfscope}%
\pgfsetbuttcap%
\pgfsetmiterjoin%
\definecolor{currentfill}{rgb}{1.000000,1.000000,1.000000}%
\pgfsetfillcolor{currentfill}%
\pgfsetlinewidth{0.000000pt}%
\definecolor{currentstroke}{rgb}{0.000000,0.000000,0.000000}%
\pgfsetstrokecolor{currentstroke}%
\pgfsetstrokeopacity{0.000000}%
\pgfsetdash{}{0pt}%
\pgfpathmoveto{\pgfqpoint{0.800000in}{0.528000in}}%
\pgfpathlineto{\pgfqpoint{5.760000in}{0.528000in}}%
\pgfpathlineto{\pgfqpoint{5.760000in}{4.224000in}}%
\pgfpathlineto{\pgfqpoint{0.800000in}{4.224000in}}%
\pgfpathlineto{\pgfqpoint{0.800000in}{0.528000in}}%
\pgfpathclose%
\pgfusepath{fill}%
\end{pgfscope}%
\begin{pgfscope}%
\pgfpathrectangle{\pgfqpoint{0.800000in}{0.528000in}}{\pgfqpoint{4.960000in}{3.696000in}}%
\pgfusepath{clip}%
\pgfsetrectcap%
\pgfsetroundjoin%
\pgfsetlinewidth{0.803000pt}%
\definecolor{currentstroke}{rgb}{0.690196,0.690196,0.690196}%
\pgfsetstrokecolor{currentstroke}%
\pgfsetdash{}{0pt}%
\pgfpathmoveto{\pgfqpoint{1.025455in}{0.528000in}}%
\pgfpathlineto{\pgfqpoint{1.025455in}{4.224000in}}%
\pgfusepath{stroke}%
\end{pgfscope}%
\begin{pgfscope}%
\pgfsetbuttcap%
\pgfsetroundjoin%
\definecolor{currentfill}{rgb}{0.000000,0.000000,0.000000}%
\pgfsetfillcolor{currentfill}%
\pgfsetlinewidth{0.803000pt}%
\definecolor{currentstroke}{rgb}{0.000000,0.000000,0.000000}%
\pgfsetstrokecolor{currentstroke}%
\pgfsetdash{}{0pt}%
\pgfsys@defobject{currentmarker}{\pgfqpoint{0.000000in}{-0.048611in}}{\pgfqpoint{0.000000in}{0.000000in}}{%
\pgfpathmoveto{\pgfqpoint{0.000000in}{0.000000in}}%
\pgfpathlineto{\pgfqpoint{0.000000in}{-0.048611in}}%
\pgfusepath{stroke,fill}%
}%
\begin{pgfscope}%
\pgfsys@transformshift{1.025455in}{0.528000in}%
\pgfsys@useobject{currentmarker}{}%
\end{pgfscope}%
\end{pgfscope}%
\begin{pgfscope}%
\definecolor{textcolor}{rgb}{0.000000,0.000000,0.000000}%
\pgfsetstrokecolor{textcolor}%
\pgfsetfillcolor{textcolor}%
\pgftext[x=1.025455in,y=0.430778in,,top]{\color{textcolor}\sffamily\fontsize{10.000000}{12.000000}\selectfont 0.0}%
\end{pgfscope}%
\begin{pgfscope}%
\pgfpathrectangle{\pgfqpoint{0.800000in}{0.528000in}}{\pgfqpoint{4.960000in}{3.696000in}}%
\pgfusepath{clip}%
\pgfsetrectcap%
\pgfsetroundjoin%
\pgfsetlinewidth{0.803000pt}%
\definecolor{currentstroke}{rgb}{0.690196,0.690196,0.690196}%
\pgfsetstrokecolor{currentstroke}%
\pgfsetdash{}{0pt}%
\pgfpathmoveto{\pgfqpoint{1.583179in}{0.528000in}}%
\pgfpathlineto{\pgfqpoint{1.583179in}{4.224000in}}%
\pgfusepath{stroke}%
\end{pgfscope}%
\begin{pgfscope}%
\pgfsetbuttcap%
\pgfsetroundjoin%
\definecolor{currentfill}{rgb}{0.000000,0.000000,0.000000}%
\pgfsetfillcolor{currentfill}%
\pgfsetlinewidth{0.803000pt}%
\definecolor{currentstroke}{rgb}{0.000000,0.000000,0.000000}%
\pgfsetstrokecolor{currentstroke}%
\pgfsetdash{}{0pt}%
\pgfsys@defobject{currentmarker}{\pgfqpoint{0.000000in}{-0.048611in}}{\pgfqpoint{0.000000in}{0.000000in}}{%
\pgfpathmoveto{\pgfqpoint{0.000000in}{0.000000in}}%
\pgfpathlineto{\pgfqpoint{0.000000in}{-0.048611in}}%
\pgfusepath{stroke,fill}%
}%
\begin{pgfscope}%
\pgfsys@transformshift{1.583179in}{0.528000in}%
\pgfsys@useobject{currentmarker}{}%
\end{pgfscope}%
\end{pgfscope}%
\begin{pgfscope}%
\definecolor{textcolor}{rgb}{0.000000,0.000000,0.000000}%
\pgfsetstrokecolor{textcolor}%
\pgfsetfillcolor{textcolor}%
\pgftext[x=1.583179in,y=0.430778in,,top]{\color{textcolor}\sffamily\fontsize{10.000000}{12.000000}\selectfont 2.5}%
\end{pgfscope}%
\begin{pgfscope}%
\pgfpathrectangle{\pgfqpoint{0.800000in}{0.528000in}}{\pgfqpoint{4.960000in}{3.696000in}}%
\pgfusepath{clip}%
\pgfsetrectcap%
\pgfsetroundjoin%
\pgfsetlinewidth{0.803000pt}%
\definecolor{currentstroke}{rgb}{0.690196,0.690196,0.690196}%
\pgfsetstrokecolor{currentstroke}%
\pgfsetdash{}{0pt}%
\pgfpathmoveto{\pgfqpoint{2.140904in}{0.528000in}}%
\pgfpathlineto{\pgfqpoint{2.140904in}{4.224000in}}%
\pgfusepath{stroke}%
\end{pgfscope}%
\begin{pgfscope}%
\pgfsetbuttcap%
\pgfsetroundjoin%
\definecolor{currentfill}{rgb}{0.000000,0.000000,0.000000}%
\pgfsetfillcolor{currentfill}%
\pgfsetlinewidth{0.803000pt}%
\definecolor{currentstroke}{rgb}{0.000000,0.000000,0.000000}%
\pgfsetstrokecolor{currentstroke}%
\pgfsetdash{}{0pt}%
\pgfsys@defobject{currentmarker}{\pgfqpoint{0.000000in}{-0.048611in}}{\pgfqpoint{0.000000in}{0.000000in}}{%
\pgfpathmoveto{\pgfqpoint{0.000000in}{0.000000in}}%
\pgfpathlineto{\pgfqpoint{0.000000in}{-0.048611in}}%
\pgfusepath{stroke,fill}%
}%
\begin{pgfscope}%
\pgfsys@transformshift{2.140904in}{0.528000in}%
\pgfsys@useobject{currentmarker}{}%
\end{pgfscope}%
\end{pgfscope}%
\begin{pgfscope}%
\definecolor{textcolor}{rgb}{0.000000,0.000000,0.000000}%
\pgfsetstrokecolor{textcolor}%
\pgfsetfillcolor{textcolor}%
\pgftext[x=2.140904in,y=0.430778in,,top]{\color{textcolor}\sffamily\fontsize{10.000000}{12.000000}\selectfont 5.0}%
\end{pgfscope}%
\begin{pgfscope}%
\pgfpathrectangle{\pgfqpoint{0.800000in}{0.528000in}}{\pgfqpoint{4.960000in}{3.696000in}}%
\pgfusepath{clip}%
\pgfsetrectcap%
\pgfsetroundjoin%
\pgfsetlinewidth{0.803000pt}%
\definecolor{currentstroke}{rgb}{0.690196,0.690196,0.690196}%
\pgfsetstrokecolor{currentstroke}%
\pgfsetdash{}{0pt}%
\pgfpathmoveto{\pgfqpoint{2.698629in}{0.528000in}}%
\pgfpathlineto{\pgfqpoint{2.698629in}{4.224000in}}%
\pgfusepath{stroke}%
\end{pgfscope}%
\begin{pgfscope}%
\pgfsetbuttcap%
\pgfsetroundjoin%
\definecolor{currentfill}{rgb}{0.000000,0.000000,0.000000}%
\pgfsetfillcolor{currentfill}%
\pgfsetlinewidth{0.803000pt}%
\definecolor{currentstroke}{rgb}{0.000000,0.000000,0.000000}%
\pgfsetstrokecolor{currentstroke}%
\pgfsetdash{}{0pt}%
\pgfsys@defobject{currentmarker}{\pgfqpoint{0.000000in}{-0.048611in}}{\pgfqpoint{0.000000in}{0.000000in}}{%
\pgfpathmoveto{\pgfqpoint{0.000000in}{0.000000in}}%
\pgfpathlineto{\pgfqpoint{0.000000in}{-0.048611in}}%
\pgfusepath{stroke,fill}%
}%
\begin{pgfscope}%
\pgfsys@transformshift{2.698629in}{0.528000in}%
\pgfsys@useobject{currentmarker}{}%
\end{pgfscope}%
\end{pgfscope}%
\begin{pgfscope}%
\definecolor{textcolor}{rgb}{0.000000,0.000000,0.000000}%
\pgfsetstrokecolor{textcolor}%
\pgfsetfillcolor{textcolor}%
\pgftext[x=2.698629in,y=0.430778in,,top]{\color{textcolor}\sffamily\fontsize{10.000000}{12.000000}\selectfont 7.5}%
\end{pgfscope}%
\begin{pgfscope}%
\pgfpathrectangle{\pgfqpoint{0.800000in}{0.528000in}}{\pgfqpoint{4.960000in}{3.696000in}}%
\pgfusepath{clip}%
\pgfsetrectcap%
\pgfsetroundjoin%
\pgfsetlinewidth{0.803000pt}%
\definecolor{currentstroke}{rgb}{0.690196,0.690196,0.690196}%
\pgfsetstrokecolor{currentstroke}%
\pgfsetdash{}{0pt}%
\pgfpathmoveto{\pgfqpoint{3.256354in}{0.528000in}}%
\pgfpathlineto{\pgfqpoint{3.256354in}{4.224000in}}%
\pgfusepath{stroke}%
\end{pgfscope}%
\begin{pgfscope}%
\pgfsetbuttcap%
\pgfsetroundjoin%
\definecolor{currentfill}{rgb}{0.000000,0.000000,0.000000}%
\pgfsetfillcolor{currentfill}%
\pgfsetlinewidth{0.803000pt}%
\definecolor{currentstroke}{rgb}{0.000000,0.000000,0.000000}%
\pgfsetstrokecolor{currentstroke}%
\pgfsetdash{}{0pt}%
\pgfsys@defobject{currentmarker}{\pgfqpoint{0.000000in}{-0.048611in}}{\pgfqpoint{0.000000in}{0.000000in}}{%
\pgfpathmoveto{\pgfqpoint{0.000000in}{0.000000in}}%
\pgfpathlineto{\pgfqpoint{0.000000in}{-0.048611in}}%
\pgfusepath{stroke,fill}%
}%
\begin{pgfscope}%
\pgfsys@transformshift{3.256354in}{0.528000in}%
\pgfsys@useobject{currentmarker}{}%
\end{pgfscope}%
\end{pgfscope}%
\begin{pgfscope}%
\definecolor{textcolor}{rgb}{0.000000,0.000000,0.000000}%
\pgfsetstrokecolor{textcolor}%
\pgfsetfillcolor{textcolor}%
\pgftext[x=3.256354in,y=0.430778in,,top]{\color{textcolor}\sffamily\fontsize{10.000000}{12.000000}\selectfont 10.0}%
\end{pgfscope}%
\begin{pgfscope}%
\pgfpathrectangle{\pgfqpoint{0.800000in}{0.528000in}}{\pgfqpoint{4.960000in}{3.696000in}}%
\pgfusepath{clip}%
\pgfsetrectcap%
\pgfsetroundjoin%
\pgfsetlinewidth{0.803000pt}%
\definecolor{currentstroke}{rgb}{0.690196,0.690196,0.690196}%
\pgfsetstrokecolor{currentstroke}%
\pgfsetdash{}{0pt}%
\pgfpathmoveto{\pgfqpoint{3.814078in}{0.528000in}}%
\pgfpathlineto{\pgfqpoint{3.814078in}{4.224000in}}%
\pgfusepath{stroke}%
\end{pgfscope}%
\begin{pgfscope}%
\pgfsetbuttcap%
\pgfsetroundjoin%
\definecolor{currentfill}{rgb}{0.000000,0.000000,0.000000}%
\pgfsetfillcolor{currentfill}%
\pgfsetlinewidth{0.803000pt}%
\definecolor{currentstroke}{rgb}{0.000000,0.000000,0.000000}%
\pgfsetstrokecolor{currentstroke}%
\pgfsetdash{}{0pt}%
\pgfsys@defobject{currentmarker}{\pgfqpoint{0.000000in}{-0.048611in}}{\pgfqpoint{0.000000in}{0.000000in}}{%
\pgfpathmoveto{\pgfqpoint{0.000000in}{0.000000in}}%
\pgfpathlineto{\pgfqpoint{0.000000in}{-0.048611in}}%
\pgfusepath{stroke,fill}%
}%
\begin{pgfscope}%
\pgfsys@transformshift{3.814078in}{0.528000in}%
\pgfsys@useobject{currentmarker}{}%
\end{pgfscope}%
\end{pgfscope}%
\begin{pgfscope}%
\definecolor{textcolor}{rgb}{0.000000,0.000000,0.000000}%
\pgfsetstrokecolor{textcolor}%
\pgfsetfillcolor{textcolor}%
\pgftext[x=3.814078in,y=0.430778in,,top]{\color{textcolor}\sffamily\fontsize{10.000000}{12.000000}\selectfont 12.5}%
\end{pgfscope}%
\begin{pgfscope}%
\pgfpathrectangle{\pgfqpoint{0.800000in}{0.528000in}}{\pgfqpoint{4.960000in}{3.696000in}}%
\pgfusepath{clip}%
\pgfsetrectcap%
\pgfsetroundjoin%
\pgfsetlinewidth{0.803000pt}%
\definecolor{currentstroke}{rgb}{0.690196,0.690196,0.690196}%
\pgfsetstrokecolor{currentstroke}%
\pgfsetdash{}{0pt}%
\pgfpathmoveto{\pgfqpoint{4.371803in}{0.528000in}}%
\pgfpathlineto{\pgfqpoint{4.371803in}{4.224000in}}%
\pgfusepath{stroke}%
\end{pgfscope}%
\begin{pgfscope}%
\pgfsetbuttcap%
\pgfsetroundjoin%
\definecolor{currentfill}{rgb}{0.000000,0.000000,0.000000}%
\pgfsetfillcolor{currentfill}%
\pgfsetlinewidth{0.803000pt}%
\definecolor{currentstroke}{rgb}{0.000000,0.000000,0.000000}%
\pgfsetstrokecolor{currentstroke}%
\pgfsetdash{}{0pt}%
\pgfsys@defobject{currentmarker}{\pgfqpoint{0.000000in}{-0.048611in}}{\pgfqpoint{0.000000in}{0.000000in}}{%
\pgfpathmoveto{\pgfqpoint{0.000000in}{0.000000in}}%
\pgfpathlineto{\pgfqpoint{0.000000in}{-0.048611in}}%
\pgfusepath{stroke,fill}%
}%
\begin{pgfscope}%
\pgfsys@transformshift{4.371803in}{0.528000in}%
\pgfsys@useobject{currentmarker}{}%
\end{pgfscope}%
\end{pgfscope}%
\begin{pgfscope}%
\definecolor{textcolor}{rgb}{0.000000,0.000000,0.000000}%
\pgfsetstrokecolor{textcolor}%
\pgfsetfillcolor{textcolor}%
\pgftext[x=4.371803in,y=0.430778in,,top]{\color{textcolor}\sffamily\fontsize{10.000000}{12.000000}\selectfont 15.0}%
\end{pgfscope}%
\begin{pgfscope}%
\pgfpathrectangle{\pgfqpoint{0.800000in}{0.528000in}}{\pgfqpoint{4.960000in}{3.696000in}}%
\pgfusepath{clip}%
\pgfsetrectcap%
\pgfsetroundjoin%
\pgfsetlinewidth{0.803000pt}%
\definecolor{currentstroke}{rgb}{0.690196,0.690196,0.690196}%
\pgfsetstrokecolor{currentstroke}%
\pgfsetdash{}{0pt}%
\pgfpathmoveto{\pgfqpoint{4.929528in}{0.528000in}}%
\pgfpathlineto{\pgfqpoint{4.929528in}{4.224000in}}%
\pgfusepath{stroke}%
\end{pgfscope}%
\begin{pgfscope}%
\pgfsetbuttcap%
\pgfsetroundjoin%
\definecolor{currentfill}{rgb}{0.000000,0.000000,0.000000}%
\pgfsetfillcolor{currentfill}%
\pgfsetlinewidth{0.803000pt}%
\definecolor{currentstroke}{rgb}{0.000000,0.000000,0.000000}%
\pgfsetstrokecolor{currentstroke}%
\pgfsetdash{}{0pt}%
\pgfsys@defobject{currentmarker}{\pgfqpoint{0.000000in}{-0.048611in}}{\pgfqpoint{0.000000in}{0.000000in}}{%
\pgfpathmoveto{\pgfqpoint{0.000000in}{0.000000in}}%
\pgfpathlineto{\pgfqpoint{0.000000in}{-0.048611in}}%
\pgfusepath{stroke,fill}%
}%
\begin{pgfscope}%
\pgfsys@transformshift{4.929528in}{0.528000in}%
\pgfsys@useobject{currentmarker}{}%
\end{pgfscope}%
\end{pgfscope}%
\begin{pgfscope}%
\definecolor{textcolor}{rgb}{0.000000,0.000000,0.000000}%
\pgfsetstrokecolor{textcolor}%
\pgfsetfillcolor{textcolor}%
\pgftext[x=4.929528in,y=0.430778in,,top]{\color{textcolor}\sffamily\fontsize{10.000000}{12.000000}\selectfont 17.5}%
\end{pgfscope}%
\begin{pgfscope}%
\pgfpathrectangle{\pgfqpoint{0.800000in}{0.528000in}}{\pgfqpoint{4.960000in}{3.696000in}}%
\pgfusepath{clip}%
\pgfsetrectcap%
\pgfsetroundjoin%
\pgfsetlinewidth{0.803000pt}%
\definecolor{currentstroke}{rgb}{0.690196,0.690196,0.690196}%
\pgfsetstrokecolor{currentstroke}%
\pgfsetdash{}{0pt}%
\pgfpathmoveto{\pgfqpoint{5.487252in}{0.528000in}}%
\pgfpathlineto{\pgfqpoint{5.487252in}{4.224000in}}%
\pgfusepath{stroke}%
\end{pgfscope}%
\begin{pgfscope}%
\pgfsetbuttcap%
\pgfsetroundjoin%
\definecolor{currentfill}{rgb}{0.000000,0.000000,0.000000}%
\pgfsetfillcolor{currentfill}%
\pgfsetlinewidth{0.803000pt}%
\definecolor{currentstroke}{rgb}{0.000000,0.000000,0.000000}%
\pgfsetstrokecolor{currentstroke}%
\pgfsetdash{}{0pt}%
\pgfsys@defobject{currentmarker}{\pgfqpoint{0.000000in}{-0.048611in}}{\pgfqpoint{0.000000in}{0.000000in}}{%
\pgfpathmoveto{\pgfqpoint{0.000000in}{0.000000in}}%
\pgfpathlineto{\pgfqpoint{0.000000in}{-0.048611in}}%
\pgfusepath{stroke,fill}%
}%
\begin{pgfscope}%
\pgfsys@transformshift{5.487252in}{0.528000in}%
\pgfsys@useobject{currentmarker}{}%
\end{pgfscope}%
\end{pgfscope}%
\begin{pgfscope}%
\definecolor{textcolor}{rgb}{0.000000,0.000000,0.000000}%
\pgfsetstrokecolor{textcolor}%
\pgfsetfillcolor{textcolor}%
\pgftext[x=5.487252in,y=0.430778in,,top]{\color{textcolor}\sffamily\fontsize{10.000000}{12.000000}\selectfont 20.0}%
\end{pgfscope}%
\begin{pgfscope}%
\definecolor{textcolor}{rgb}{0.000000,0.000000,0.000000}%
\pgfsetstrokecolor{textcolor}%
\pgfsetfillcolor{textcolor}%
\pgftext[x=3.280000in,y=0.240809in,,top]{\color{textcolor}\sffamily\fontsize{10.000000}{12.000000}\selectfont time [s]}%
\end{pgfscope}%
\begin{pgfscope}%
\pgfpathrectangle{\pgfqpoint{0.800000in}{0.528000in}}{\pgfqpoint{4.960000in}{3.696000in}}%
\pgfusepath{clip}%
\pgfsetrectcap%
\pgfsetroundjoin%
\pgfsetlinewidth{0.803000pt}%
\definecolor{currentstroke}{rgb}{0.690196,0.690196,0.690196}%
\pgfsetstrokecolor{currentstroke}%
\pgfsetdash{}{0pt}%
\pgfpathmoveto{\pgfqpoint{0.800000in}{0.671793in}}%
\pgfpathlineto{\pgfqpoint{5.760000in}{0.671793in}}%
\pgfusepath{stroke}%
\end{pgfscope}%
\begin{pgfscope}%
\pgfsetbuttcap%
\pgfsetroundjoin%
\definecolor{currentfill}{rgb}{0.000000,0.000000,0.000000}%
\pgfsetfillcolor{currentfill}%
\pgfsetlinewidth{0.803000pt}%
\definecolor{currentstroke}{rgb}{0.000000,0.000000,0.000000}%
\pgfsetstrokecolor{currentstroke}%
\pgfsetdash{}{0pt}%
\pgfsys@defobject{currentmarker}{\pgfqpoint{-0.048611in}{0.000000in}}{\pgfqpoint{-0.000000in}{0.000000in}}{%
\pgfpathmoveto{\pgfqpoint{-0.000000in}{0.000000in}}%
\pgfpathlineto{\pgfqpoint{-0.048611in}{0.000000in}}%
\pgfusepath{stroke,fill}%
}%
\begin{pgfscope}%
\pgfsys@transformshift{0.800000in}{0.671793in}%
\pgfsys@useobject{currentmarker}{}%
\end{pgfscope}%
\end{pgfscope}%
\begin{pgfscope}%
\definecolor{textcolor}{rgb}{0.000000,0.000000,0.000000}%
\pgfsetstrokecolor{textcolor}%
\pgfsetfillcolor{textcolor}%
\pgftext[x=0.285508in, y=0.619031in, left, base]{\color{textcolor}\sffamily\fontsize{10.000000}{12.000000}\selectfont \ensuremath{-}0.04}%
\end{pgfscope}%
\begin{pgfscope}%
\pgfpathrectangle{\pgfqpoint{0.800000in}{0.528000in}}{\pgfqpoint{4.960000in}{3.696000in}}%
\pgfusepath{clip}%
\pgfsetrectcap%
\pgfsetroundjoin%
\pgfsetlinewidth{0.803000pt}%
\definecolor{currentstroke}{rgb}{0.690196,0.690196,0.690196}%
\pgfsetstrokecolor{currentstroke}%
\pgfsetdash{}{0pt}%
\pgfpathmoveto{\pgfqpoint{0.800000in}{1.165739in}}%
\pgfpathlineto{\pgfqpoint{5.760000in}{1.165739in}}%
\pgfusepath{stroke}%
\end{pgfscope}%
\begin{pgfscope}%
\pgfsetbuttcap%
\pgfsetroundjoin%
\definecolor{currentfill}{rgb}{0.000000,0.000000,0.000000}%
\pgfsetfillcolor{currentfill}%
\pgfsetlinewidth{0.803000pt}%
\definecolor{currentstroke}{rgb}{0.000000,0.000000,0.000000}%
\pgfsetstrokecolor{currentstroke}%
\pgfsetdash{}{0pt}%
\pgfsys@defobject{currentmarker}{\pgfqpoint{-0.048611in}{0.000000in}}{\pgfqpoint{-0.000000in}{0.000000in}}{%
\pgfpathmoveto{\pgfqpoint{-0.000000in}{0.000000in}}%
\pgfpathlineto{\pgfqpoint{-0.048611in}{0.000000in}}%
\pgfusepath{stroke,fill}%
}%
\begin{pgfscope}%
\pgfsys@transformshift{0.800000in}{1.165739in}%
\pgfsys@useobject{currentmarker}{}%
\end{pgfscope}%
\end{pgfscope}%
\begin{pgfscope}%
\definecolor{textcolor}{rgb}{0.000000,0.000000,0.000000}%
\pgfsetstrokecolor{textcolor}%
\pgfsetfillcolor{textcolor}%
\pgftext[x=0.285508in, y=1.112978in, left, base]{\color{textcolor}\sffamily\fontsize{10.000000}{12.000000}\selectfont \ensuremath{-}0.02}%
\end{pgfscope}%
\begin{pgfscope}%
\pgfpathrectangle{\pgfqpoint{0.800000in}{0.528000in}}{\pgfqpoint{4.960000in}{3.696000in}}%
\pgfusepath{clip}%
\pgfsetrectcap%
\pgfsetroundjoin%
\pgfsetlinewidth{0.803000pt}%
\definecolor{currentstroke}{rgb}{0.690196,0.690196,0.690196}%
\pgfsetstrokecolor{currentstroke}%
\pgfsetdash{}{0pt}%
\pgfpathmoveto{\pgfqpoint{0.800000in}{1.659686in}}%
\pgfpathlineto{\pgfqpoint{5.760000in}{1.659686in}}%
\pgfusepath{stroke}%
\end{pgfscope}%
\begin{pgfscope}%
\pgfsetbuttcap%
\pgfsetroundjoin%
\definecolor{currentfill}{rgb}{0.000000,0.000000,0.000000}%
\pgfsetfillcolor{currentfill}%
\pgfsetlinewidth{0.803000pt}%
\definecolor{currentstroke}{rgb}{0.000000,0.000000,0.000000}%
\pgfsetstrokecolor{currentstroke}%
\pgfsetdash{}{0pt}%
\pgfsys@defobject{currentmarker}{\pgfqpoint{-0.048611in}{0.000000in}}{\pgfqpoint{-0.000000in}{0.000000in}}{%
\pgfpathmoveto{\pgfqpoint{-0.000000in}{0.000000in}}%
\pgfpathlineto{\pgfqpoint{-0.048611in}{0.000000in}}%
\pgfusepath{stroke,fill}%
}%
\begin{pgfscope}%
\pgfsys@transformshift{0.800000in}{1.659686in}%
\pgfsys@useobject{currentmarker}{}%
\end{pgfscope}%
\end{pgfscope}%
\begin{pgfscope}%
\definecolor{textcolor}{rgb}{0.000000,0.000000,0.000000}%
\pgfsetstrokecolor{textcolor}%
\pgfsetfillcolor{textcolor}%
\pgftext[x=0.393533in, y=1.606924in, left, base]{\color{textcolor}\sffamily\fontsize{10.000000}{12.000000}\selectfont 0.00}%
\end{pgfscope}%
\begin{pgfscope}%
\pgfpathrectangle{\pgfqpoint{0.800000in}{0.528000in}}{\pgfqpoint{4.960000in}{3.696000in}}%
\pgfusepath{clip}%
\pgfsetrectcap%
\pgfsetroundjoin%
\pgfsetlinewidth{0.803000pt}%
\definecolor{currentstroke}{rgb}{0.690196,0.690196,0.690196}%
\pgfsetstrokecolor{currentstroke}%
\pgfsetdash{}{0pt}%
\pgfpathmoveto{\pgfqpoint{0.800000in}{2.153632in}}%
\pgfpathlineto{\pgfqpoint{5.760000in}{2.153632in}}%
\pgfusepath{stroke}%
\end{pgfscope}%
\begin{pgfscope}%
\pgfsetbuttcap%
\pgfsetroundjoin%
\definecolor{currentfill}{rgb}{0.000000,0.000000,0.000000}%
\pgfsetfillcolor{currentfill}%
\pgfsetlinewidth{0.803000pt}%
\definecolor{currentstroke}{rgb}{0.000000,0.000000,0.000000}%
\pgfsetstrokecolor{currentstroke}%
\pgfsetdash{}{0pt}%
\pgfsys@defobject{currentmarker}{\pgfqpoint{-0.048611in}{0.000000in}}{\pgfqpoint{-0.000000in}{0.000000in}}{%
\pgfpathmoveto{\pgfqpoint{-0.000000in}{0.000000in}}%
\pgfpathlineto{\pgfqpoint{-0.048611in}{0.000000in}}%
\pgfusepath{stroke,fill}%
}%
\begin{pgfscope}%
\pgfsys@transformshift{0.800000in}{2.153632in}%
\pgfsys@useobject{currentmarker}{}%
\end{pgfscope}%
\end{pgfscope}%
\begin{pgfscope}%
\definecolor{textcolor}{rgb}{0.000000,0.000000,0.000000}%
\pgfsetstrokecolor{textcolor}%
\pgfsetfillcolor{textcolor}%
\pgftext[x=0.393533in, y=2.100871in, left, base]{\color{textcolor}\sffamily\fontsize{10.000000}{12.000000}\selectfont 0.02}%
\end{pgfscope}%
\begin{pgfscope}%
\pgfpathrectangle{\pgfqpoint{0.800000in}{0.528000in}}{\pgfqpoint{4.960000in}{3.696000in}}%
\pgfusepath{clip}%
\pgfsetrectcap%
\pgfsetroundjoin%
\pgfsetlinewidth{0.803000pt}%
\definecolor{currentstroke}{rgb}{0.690196,0.690196,0.690196}%
\pgfsetstrokecolor{currentstroke}%
\pgfsetdash{}{0pt}%
\pgfpathmoveto{\pgfqpoint{0.800000in}{2.647579in}}%
\pgfpathlineto{\pgfqpoint{5.760000in}{2.647579in}}%
\pgfusepath{stroke}%
\end{pgfscope}%
\begin{pgfscope}%
\pgfsetbuttcap%
\pgfsetroundjoin%
\definecolor{currentfill}{rgb}{0.000000,0.000000,0.000000}%
\pgfsetfillcolor{currentfill}%
\pgfsetlinewidth{0.803000pt}%
\definecolor{currentstroke}{rgb}{0.000000,0.000000,0.000000}%
\pgfsetstrokecolor{currentstroke}%
\pgfsetdash{}{0pt}%
\pgfsys@defobject{currentmarker}{\pgfqpoint{-0.048611in}{0.000000in}}{\pgfqpoint{-0.000000in}{0.000000in}}{%
\pgfpathmoveto{\pgfqpoint{-0.000000in}{0.000000in}}%
\pgfpathlineto{\pgfqpoint{-0.048611in}{0.000000in}}%
\pgfusepath{stroke,fill}%
}%
\begin{pgfscope}%
\pgfsys@transformshift{0.800000in}{2.647579in}%
\pgfsys@useobject{currentmarker}{}%
\end{pgfscope}%
\end{pgfscope}%
\begin{pgfscope}%
\definecolor{textcolor}{rgb}{0.000000,0.000000,0.000000}%
\pgfsetstrokecolor{textcolor}%
\pgfsetfillcolor{textcolor}%
\pgftext[x=0.393533in, y=2.594817in, left, base]{\color{textcolor}\sffamily\fontsize{10.000000}{12.000000}\selectfont 0.04}%
\end{pgfscope}%
\begin{pgfscope}%
\pgfpathrectangle{\pgfqpoint{0.800000in}{0.528000in}}{\pgfqpoint{4.960000in}{3.696000in}}%
\pgfusepath{clip}%
\pgfsetrectcap%
\pgfsetroundjoin%
\pgfsetlinewidth{0.803000pt}%
\definecolor{currentstroke}{rgb}{0.690196,0.690196,0.690196}%
\pgfsetstrokecolor{currentstroke}%
\pgfsetdash{}{0pt}%
\pgfpathmoveto{\pgfqpoint{0.800000in}{3.141525in}}%
\pgfpathlineto{\pgfqpoint{5.760000in}{3.141525in}}%
\pgfusepath{stroke}%
\end{pgfscope}%
\begin{pgfscope}%
\pgfsetbuttcap%
\pgfsetroundjoin%
\definecolor{currentfill}{rgb}{0.000000,0.000000,0.000000}%
\pgfsetfillcolor{currentfill}%
\pgfsetlinewidth{0.803000pt}%
\definecolor{currentstroke}{rgb}{0.000000,0.000000,0.000000}%
\pgfsetstrokecolor{currentstroke}%
\pgfsetdash{}{0pt}%
\pgfsys@defobject{currentmarker}{\pgfqpoint{-0.048611in}{0.000000in}}{\pgfqpoint{-0.000000in}{0.000000in}}{%
\pgfpathmoveto{\pgfqpoint{-0.000000in}{0.000000in}}%
\pgfpathlineto{\pgfqpoint{-0.048611in}{0.000000in}}%
\pgfusepath{stroke,fill}%
}%
\begin{pgfscope}%
\pgfsys@transformshift{0.800000in}{3.141525in}%
\pgfsys@useobject{currentmarker}{}%
\end{pgfscope}%
\end{pgfscope}%
\begin{pgfscope}%
\definecolor{textcolor}{rgb}{0.000000,0.000000,0.000000}%
\pgfsetstrokecolor{textcolor}%
\pgfsetfillcolor{textcolor}%
\pgftext[x=0.393533in, y=3.088763in, left, base]{\color{textcolor}\sffamily\fontsize{10.000000}{12.000000}\selectfont 0.06}%
\end{pgfscope}%
\begin{pgfscope}%
\pgfpathrectangle{\pgfqpoint{0.800000in}{0.528000in}}{\pgfqpoint{4.960000in}{3.696000in}}%
\pgfusepath{clip}%
\pgfsetrectcap%
\pgfsetroundjoin%
\pgfsetlinewidth{0.803000pt}%
\definecolor{currentstroke}{rgb}{0.690196,0.690196,0.690196}%
\pgfsetstrokecolor{currentstroke}%
\pgfsetdash{}{0pt}%
\pgfpathmoveto{\pgfqpoint{0.800000in}{3.635471in}}%
\pgfpathlineto{\pgfqpoint{5.760000in}{3.635471in}}%
\pgfusepath{stroke}%
\end{pgfscope}%
\begin{pgfscope}%
\pgfsetbuttcap%
\pgfsetroundjoin%
\definecolor{currentfill}{rgb}{0.000000,0.000000,0.000000}%
\pgfsetfillcolor{currentfill}%
\pgfsetlinewidth{0.803000pt}%
\definecolor{currentstroke}{rgb}{0.000000,0.000000,0.000000}%
\pgfsetstrokecolor{currentstroke}%
\pgfsetdash{}{0pt}%
\pgfsys@defobject{currentmarker}{\pgfqpoint{-0.048611in}{0.000000in}}{\pgfqpoint{-0.000000in}{0.000000in}}{%
\pgfpathmoveto{\pgfqpoint{-0.000000in}{0.000000in}}%
\pgfpathlineto{\pgfqpoint{-0.048611in}{0.000000in}}%
\pgfusepath{stroke,fill}%
}%
\begin{pgfscope}%
\pgfsys@transformshift{0.800000in}{3.635471in}%
\pgfsys@useobject{currentmarker}{}%
\end{pgfscope}%
\end{pgfscope}%
\begin{pgfscope}%
\definecolor{textcolor}{rgb}{0.000000,0.000000,0.000000}%
\pgfsetstrokecolor{textcolor}%
\pgfsetfillcolor{textcolor}%
\pgftext[x=0.393533in, y=3.582710in, left, base]{\color{textcolor}\sffamily\fontsize{10.000000}{12.000000}\selectfont 0.08}%
\end{pgfscope}%
\begin{pgfscope}%
\pgfpathrectangle{\pgfqpoint{0.800000in}{0.528000in}}{\pgfqpoint{4.960000in}{3.696000in}}%
\pgfusepath{clip}%
\pgfsetrectcap%
\pgfsetroundjoin%
\pgfsetlinewidth{0.803000pt}%
\definecolor{currentstroke}{rgb}{0.690196,0.690196,0.690196}%
\pgfsetstrokecolor{currentstroke}%
\pgfsetdash{}{0pt}%
\pgfpathmoveto{\pgfqpoint{0.800000in}{4.129418in}}%
\pgfpathlineto{\pgfqpoint{5.760000in}{4.129418in}}%
\pgfusepath{stroke}%
\end{pgfscope}%
\begin{pgfscope}%
\pgfsetbuttcap%
\pgfsetroundjoin%
\definecolor{currentfill}{rgb}{0.000000,0.000000,0.000000}%
\pgfsetfillcolor{currentfill}%
\pgfsetlinewidth{0.803000pt}%
\definecolor{currentstroke}{rgb}{0.000000,0.000000,0.000000}%
\pgfsetstrokecolor{currentstroke}%
\pgfsetdash{}{0pt}%
\pgfsys@defobject{currentmarker}{\pgfqpoint{-0.048611in}{0.000000in}}{\pgfqpoint{-0.000000in}{0.000000in}}{%
\pgfpathmoveto{\pgfqpoint{-0.000000in}{0.000000in}}%
\pgfpathlineto{\pgfqpoint{-0.048611in}{0.000000in}}%
\pgfusepath{stroke,fill}%
}%
\begin{pgfscope}%
\pgfsys@transformshift{0.800000in}{4.129418in}%
\pgfsys@useobject{currentmarker}{}%
\end{pgfscope}%
\end{pgfscope}%
\begin{pgfscope}%
\definecolor{textcolor}{rgb}{0.000000,0.000000,0.000000}%
\pgfsetstrokecolor{textcolor}%
\pgfsetfillcolor{textcolor}%
\pgftext[x=0.393533in, y=4.076656in, left, base]{\color{textcolor}\sffamily\fontsize{10.000000}{12.000000}\selectfont 0.10}%
\end{pgfscope}%
\begin{pgfscope}%
\definecolor{textcolor}{rgb}{0.000000,0.000000,0.000000}%
\pgfsetstrokecolor{textcolor}%
\pgfsetfillcolor{textcolor}%
\pgftext[x=0.229952in,y=2.376000in,,bottom,rotate=90.000000]{\color{textcolor}\sffamily\fontsize{10.000000}{12.000000}\selectfont Velocity [rad/s]}%
\end{pgfscope}%
\begin{pgfscope}%
\pgfpathrectangle{\pgfqpoint{0.800000in}{0.528000in}}{\pgfqpoint{4.960000in}{3.696000in}}%
\pgfusepath{clip}%
\pgfsetrectcap%
\pgfsetroundjoin%
\pgfsetlinewidth{1.505625pt}%
\definecolor{currentstroke}{rgb}{0.121569,0.466667,0.705882}%
\pgfsetstrokecolor{currentstroke}%
\pgfsetdash{}{0pt}%
\pgfpathmoveto{\pgfqpoint{1.025455in}{1.631249in}}%
\pgfpathlineto{\pgfqpoint{1.063807in}{2.070865in}}%
\pgfpathlineto{\pgfqpoint{1.138689in}{2.548287in}}%
\pgfpathlineto{\pgfqpoint{1.212835in}{2.829660in}}%
\pgfpathlineto{\pgfqpoint{1.286293in}{2.981943in}}%
\pgfpathlineto{\pgfqpoint{1.360494in}{2.916869in}}%
\pgfpathlineto{\pgfqpoint{1.434923in}{2.782950in}}%
\pgfpathlineto{\pgfqpoint{1.508995in}{2.689896in}}%
\pgfpathlineto{\pgfqpoint{1.584752in}{2.545818in}}%
\pgfpathlineto{\pgfqpoint{1.662031in}{2.446119in}}%
\pgfpathlineto{\pgfqpoint{1.735165in}{2.349959in}}%
\pgfpathlineto{\pgfqpoint{1.808694in}{2.242041in}}%
\pgfpathlineto{\pgfqpoint{1.882920in}{2.186106in}}%
\pgfpathlineto{\pgfqpoint{1.957495in}{2.091300in}}%
\pgfpathlineto{\pgfqpoint{2.032291in}{2.027786in}}%
\pgfpathlineto{\pgfqpoint{2.106575in}{2.044421in}}%
\pgfpathlineto{\pgfqpoint{2.181248in}{1.940201in}}%
\pgfpathlineto{\pgfqpoint{2.255368in}{1.915636in}}%
\pgfpathlineto{\pgfqpoint{2.330471in}{1.891378in}}%
\pgfpathlineto{\pgfqpoint{2.404691in}{1.946402in}}%
\pgfpathlineto{\pgfqpoint{2.479207in}{1.886295in}}%
\pgfpathlineto{\pgfqpoint{2.555411in}{1.844178in}}%
\pgfpathlineto{\pgfqpoint{2.629606in}{1.804522in}}%
\pgfpathlineto{\pgfqpoint{2.703104in}{1.806190in}}%
\pgfpathlineto{\pgfqpoint{2.777292in}{1.823848in}}%
\pgfpathlineto{\pgfqpoint{2.852751in}{1.810314in}}%
\pgfpathlineto{\pgfqpoint{2.927608in}{1.770838in}}%
\pgfpathlineto{\pgfqpoint{3.001555in}{1.754091in}}%
\pgfpathlineto{\pgfqpoint{3.076794in}{1.740145in}}%
\pgfpathlineto{\pgfqpoint{3.151007in}{1.725020in}}%
\pgfpathlineto{\pgfqpoint{3.227248in}{1.710949in}}%
\pgfpathlineto{\pgfqpoint{3.301410in}{1.738204in}}%
\pgfpathlineto{\pgfqpoint{3.375998in}{1.750477in}}%
\pgfpathlineto{\pgfqpoint{3.452370in}{1.716263in}}%
\pgfpathlineto{\pgfqpoint{3.527983in}{1.719795in}}%
\pgfpathlineto{\pgfqpoint{3.599009in}{1.698404in}}%
\pgfpathlineto{\pgfqpoint{3.673289in}{1.669217in}}%
\pgfpathlineto{\pgfqpoint{3.747802in}{1.675229in}}%
\pgfpathlineto{\pgfqpoint{3.822038in}{1.681203in}}%
\pgfpathlineto{\pgfqpoint{3.896411in}{1.676622in}}%
\pgfpathlineto{\pgfqpoint{3.970918in}{1.677144in}}%
\pgfpathlineto{\pgfqpoint{4.045652in}{1.662890in}}%
\pgfpathlineto{\pgfqpoint{4.119791in}{1.676428in}}%
\pgfpathlineto{\pgfqpoint{4.193816in}{1.652267in}}%
\pgfpathlineto{\pgfqpoint{4.268642in}{1.653853in}}%
\pgfpathlineto{\pgfqpoint{4.345114in}{1.636619in}}%
\pgfpathlineto{\pgfqpoint{4.418039in}{1.629333in}}%
\pgfpathlineto{\pgfqpoint{4.492440in}{1.658165in}}%
\pgfpathlineto{\pgfqpoint{4.566331in}{1.631458in}}%
\pgfpathlineto{\pgfqpoint{4.640612in}{1.636180in}}%
\pgfpathlineto{\pgfqpoint{4.715028in}{1.625568in}}%
\pgfpathlineto{\pgfqpoint{4.789574in}{1.653363in}}%
\pgfpathlineto{\pgfqpoint{4.864101in}{1.655776in}}%
\pgfpathlineto{\pgfqpoint{4.938540in}{1.627751in}}%
\pgfpathlineto{\pgfqpoint{5.012377in}{1.659054in}}%
\pgfpathlineto{\pgfqpoint{5.086863in}{1.653139in}}%
\pgfpathlineto{\pgfqpoint{5.161582in}{1.650685in}}%
\pgfpathlineto{\pgfqpoint{5.237903in}{1.624078in}}%
\pgfpathlineto{\pgfqpoint{5.311101in}{1.653299in}}%
\pgfpathlineto{\pgfqpoint{5.386074in}{1.647135in}}%
\pgfpathlineto{\pgfqpoint{5.459923in}{1.649124in}}%
\pgfpathlineto{\pgfqpoint{5.534545in}{1.652404in}}%
\pgfusepath{stroke}%
\end{pgfscope}%
\begin{pgfscope}%
\pgfpathrectangle{\pgfqpoint{0.800000in}{0.528000in}}{\pgfqpoint{4.960000in}{3.696000in}}%
\pgfusepath{clip}%
\pgfsetrectcap%
\pgfsetroundjoin%
\pgfsetlinewidth{1.505625pt}%
\definecolor{currentstroke}{rgb}{1.000000,0.498039,0.054902}%
\pgfsetstrokecolor{currentstroke}%
\pgfsetdash{}{0pt}%
\pgfpathmoveto{\pgfqpoint{1.025455in}{1.740166in}}%
\pgfpathlineto{\pgfqpoint{1.099067in}{2.940289in}}%
\pgfpathlineto{\pgfqpoint{1.172823in}{3.525657in}}%
\pgfpathlineto{\pgfqpoint{1.246593in}{3.849358in}}%
\pgfpathlineto{\pgfqpoint{1.322459in}{3.888734in}}%
\pgfpathlineto{\pgfqpoint{1.396727in}{3.560737in}}%
\pgfpathlineto{\pgfqpoint{1.473435in}{3.236878in}}%
\pgfpathlineto{\pgfqpoint{1.546479in}{2.861589in}}%
\pgfpathlineto{\pgfqpoint{1.621080in}{2.523281in}}%
\pgfpathlineto{\pgfqpoint{1.696742in}{2.270084in}}%
\pgfpathlineto{\pgfqpoint{1.770997in}{1.991273in}}%
\pgfpathlineto{\pgfqpoint{1.848657in}{1.874345in}}%
\pgfpathlineto{\pgfqpoint{1.920302in}{1.757251in}}%
\pgfpathlineto{\pgfqpoint{1.994163in}{1.690953in}}%
\pgfpathlineto{\pgfqpoint{2.068175in}{1.661899in}}%
\pgfpathlineto{\pgfqpoint{2.142198in}{1.666397in}}%
\pgfpathlineto{\pgfqpoint{2.216474in}{1.669723in}}%
\pgfpathlineto{\pgfqpoint{2.291342in}{1.656486in}}%
\pgfpathlineto{\pgfqpoint{2.365521in}{1.698022in}}%
\pgfpathlineto{\pgfqpoint{2.440012in}{1.698606in}}%
\pgfpathlineto{\pgfqpoint{2.513997in}{1.689923in}}%
\pgfpathlineto{\pgfqpoint{2.589952in}{1.685830in}}%
\pgfpathlineto{\pgfqpoint{2.663420in}{1.741157in}}%
\pgfpathlineto{\pgfqpoint{2.737944in}{1.692276in}}%
\pgfpathlineto{\pgfqpoint{2.812320in}{1.716690in}}%
\pgfpathlineto{\pgfqpoint{2.886400in}{1.732093in}}%
\pgfpathlineto{\pgfqpoint{2.960971in}{1.694698in}}%
\pgfpathlineto{\pgfqpoint{3.035319in}{1.671425in}}%
\pgfpathlineto{\pgfqpoint{3.109983in}{1.624493in}}%
\pgfpathlineto{\pgfqpoint{3.185917in}{1.643401in}}%
\pgfpathlineto{\pgfqpoint{3.258235in}{1.650697in}}%
\pgfpathlineto{\pgfqpoint{3.334839in}{1.603159in}}%
\pgfpathlineto{\pgfqpoint{3.409040in}{1.627185in}}%
\pgfpathlineto{\pgfqpoint{3.483311in}{1.614847in}}%
\pgfpathlineto{\pgfqpoint{3.557682in}{1.609300in}}%
\pgfpathlineto{\pgfqpoint{3.635410in}{1.653390in}}%
\pgfpathlineto{\pgfqpoint{3.706113in}{1.631092in}}%
\pgfpathlineto{\pgfqpoint{3.780735in}{1.623244in}}%
\pgfpathlineto{\pgfqpoint{3.855108in}{1.632305in}}%
\pgfpathlineto{\pgfqpoint{3.929708in}{1.652522in}}%
\pgfpathlineto{\pgfqpoint{4.003734in}{1.626104in}}%
\pgfpathlineto{\pgfqpoint{4.078207in}{1.672283in}}%
\pgfpathlineto{\pgfqpoint{4.152372in}{1.622260in}}%
\pgfpathlineto{\pgfqpoint{4.228302in}{1.668533in}}%
\pgfpathlineto{\pgfqpoint{4.302367in}{1.665862in}}%
\pgfpathlineto{\pgfqpoint{4.375688in}{1.680415in}}%
\pgfpathlineto{\pgfqpoint{4.451228in}{1.664455in}}%
\pgfpathlineto{\pgfqpoint{4.525797in}{1.681630in}}%
\pgfpathlineto{\pgfqpoint{4.600718in}{1.698702in}}%
\pgfpathlineto{\pgfqpoint{4.675469in}{1.672373in}}%
\pgfpathlineto{\pgfqpoint{4.752587in}{1.680624in}}%
\pgfpathlineto{\pgfqpoint{4.828081in}{1.684383in}}%
\pgfpathlineto{\pgfqpoint{4.902364in}{1.650767in}}%
\pgfpathlineto{\pgfqpoint{4.976596in}{1.669330in}}%
\pgfpathlineto{\pgfqpoint{5.051701in}{1.642349in}}%
\pgfpathlineto{\pgfqpoint{5.125177in}{1.632734in}}%
\pgfpathlineto{\pgfqpoint{5.199714in}{1.639490in}}%
\pgfpathlineto{\pgfqpoint{5.273899in}{1.675514in}}%
\pgfpathlineto{\pgfqpoint{5.348033in}{1.738245in}}%
\pgfpathlineto{\pgfqpoint{5.422456in}{1.737675in}}%
\pgfpathlineto{\pgfqpoint{5.497007in}{1.685437in}}%
\pgfusepath{stroke}%
\end{pgfscope}%
\begin{pgfscope}%
\pgfpathrectangle{\pgfqpoint{0.800000in}{0.528000in}}{\pgfqpoint{4.960000in}{3.696000in}}%
\pgfusepath{clip}%
\pgfsetrectcap%
\pgfsetroundjoin%
\pgfsetlinewidth{1.505625pt}%
\definecolor{currentstroke}{rgb}{0.172549,0.627451,0.172549}%
\pgfsetstrokecolor{currentstroke}%
\pgfsetdash{}{0pt}%
\pgfpathmoveto{\pgfqpoint{1.025455in}{1.732789in}}%
\pgfpathlineto{\pgfqpoint{1.099824in}{2.918831in}}%
\pgfpathlineto{\pgfqpoint{1.175368in}{3.546864in}}%
\pgfpathlineto{\pgfqpoint{1.250734in}{3.868432in}}%
\pgfpathlineto{\pgfqpoint{1.324164in}{4.019308in}}%
\pgfpathlineto{\pgfqpoint{1.398714in}{4.032472in}}%
\pgfpathlineto{\pgfqpoint{1.473193in}{3.595834in}}%
\pgfpathlineto{\pgfqpoint{1.547047in}{3.143421in}}%
\pgfpathlineto{\pgfqpoint{1.621619in}{2.549996in}}%
\pgfpathlineto{\pgfqpoint{1.695733in}{2.004764in}}%
\pgfpathlineto{\pgfqpoint{1.770407in}{1.662406in}}%
\pgfpathlineto{\pgfqpoint{1.847793in}{1.474115in}}%
\pgfpathlineto{\pgfqpoint{1.918976in}{1.377959in}}%
\pgfpathlineto{\pgfqpoint{1.995825in}{1.336387in}}%
\pgfpathlineto{\pgfqpoint{2.071419in}{1.343451in}}%
\pgfpathlineto{\pgfqpoint{2.144728in}{1.387159in}}%
\pgfpathlineto{\pgfqpoint{2.218537in}{1.536581in}}%
\pgfpathlineto{\pgfqpoint{2.293167in}{1.578081in}}%
\pgfpathlineto{\pgfqpoint{2.367570in}{1.584756in}}%
\pgfpathlineto{\pgfqpoint{2.441957in}{1.675891in}}%
\pgfpathlineto{\pgfqpoint{2.516456in}{1.672389in}}%
\pgfpathlineto{\pgfqpoint{2.590956in}{1.681978in}}%
\pgfpathlineto{\pgfqpoint{2.665462in}{1.656070in}}%
\pgfpathlineto{\pgfqpoint{2.739194in}{1.650246in}}%
\pgfpathlineto{\pgfqpoint{2.814185in}{1.715827in}}%
\pgfpathlineto{\pgfqpoint{2.889918in}{1.695272in}}%
\pgfpathlineto{\pgfqpoint{2.963700in}{1.711151in}}%
\pgfpathlineto{\pgfqpoint{3.037258in}{1.718424in}}%
\pgfpathlineto{\pgfqpoint{3.112152in}{1.699637in}}%
\pgfpathlineto{\pgfqpoint{3.186417in}{1.692146in}}%
\pgfpathlineto{\pgfqpoint{3.260642in}{1.725717in}}%
\pgfpathlineto{\pgfqpoint{3.335015in}{1.692852in}}%
\pgfpathlineto{\pgfqpoint{3.409349in}{1.660546in}}%
\pgfpathlineto{\pgfqpoint{3.484590in}{1.659327in}}%
\pgfpathlineto{\pgfqpoint{3.558022in}{1.646971in}}%
\pgfpathlineto{\pgfqpoint{3.632308in}{1.703731in}}%
\pgfpathlineto{\pgfqpoint{3.706894in}{1.673098in}}%
\pgfpathlineto{\pgfqpoint{3.783333in}{1.644808in}}%
\pgfpathlineto{\pgfqpoint{3.856942in}{1.593425in}}%
\pgfpathlineto{\pgfqpoint{3.930686in}{1.630386in}}%
\pgfpathlineto{\pgfqpoint{4.004579in}{1.661748in}}%
\pgfpathlineto{\pgfqpoint{4.078500in}{1.676459in}}%
\pgfpathlineto{\pgfqpoint{4.154011in}{1.725711in}}%
\pgfpathlineto{\pgfqpoint{4.228644in}{1.736907in}}%
\pgfpathlineto{\pgfqpoint{4.302594in}{1.761249in}}%
\pgfpathlineto{\pgfqpoint{4.377267in}{1.751406in}}%
\pgfpathlineto{\pgfqpoint{4.451307in}{1.691057in}}%
\pgfpathlineto{\pgfqpoint{4.525813in}{1.691894in}}%
\pgfpathlineto{\pgfqpoint{4.601026in}{1.678166in}}%
\pgfpathlineto{\pgfqpoint{4.675062in}{1.693032in}}%
\pgfpathlineto{\pgfqpoint{4.750808in}{1.663004in}}%
\pgfpathlineto{\pgfqpoint{4.824314in}{1.644392in}}%
\pgfpathlineto{\pgfqpoint{4.898125in}{1.660084in}}%
\pgfpathlineto{\pgfqpoint{4.972539in}{1.675767in}}%
\pgfpathlineto{\pgfqpoint{5.047642in}{1.747980in}}%
\pgfpathlineto{\pgfqpoint{5.121926in}{1.696020in}}%
\pgfpathlineto{\pgfqpoint{5.196093in}{1.743939in}}%
\pgfpathlineto{\pgfqpoint{5.270461in}{1.701187in}}%
\pgfpathlineto{\pgfqpoint{5.344890in}{1.672501in}}%
\pgfpathlineto{\pgfqpoint{5.419185in}{1.663140in}}%
\pgfpathlineto{\pgfqpoint{5.493626in}{1.653003in}}%
\pgfusepath{stroke}%
\end{pgfscope}%
\begin{pgfscope}%
\pgfpathrectangle{\pgfqpoint{0.800000in}{0.528000in}}{\pgfqpoint{4.960000in}{3.696000in}}%
\pgfusepath{clip}%
\pgfsetrectcap%
\pgfsetroundjoin%
\pgfsetlinewidth{1.505625pt}%
\definecolor{currentstroke}{rgb}{0.839216,0.152941,0.156863}%
\pgfsetstrokecolor{currentstroke}%
\pgfsetdash{}{0pt}%
\pgfpathmoveto{\pgfqpoint{1.025455in}{1.740938in}}%
\pgfpathlineto{\pgfqpoint{1.099364in}{2.911101in}}%
\pgfpathlineto{\pgfqpoint{1.173463in}{3.517851in}}%
\pgfpathlineto{\pgfqpoint{1.247490in}{3.880327in}}%
\pgfpathlineto{\pgfqpoint{1.322184in}{4.022861in}}%
\pgfpathlineto{\pgfqpoint{1.396363in}{4.056000in}}%
\pgfpathlineto{\pgfqpoint{1.470989in}{3.959532in}}%
\pgfpathlineto{\pgfqpoint{1.545509in}{3.483664in}}%
\pgfpathlineto{\pgfqpoint{1.619641in}{2.793555in}}%
\pgfpathlineto{\pgfqpoint{1.693921in}{1.988121in}}%
\pgfpathlineto{\pgfqpoint{1.768886in}{1.539417in}}%
\pgfpathlineto{\pgfqpoint{1.842952in}{1.143651in}}%
\pgfpathlineto{\pgfqpoint{1.919041in}{1.078813in}}%
\pgfpathlineto{\pgfqpoint{1.992332in}{1.107481in}}%
\pgfpathlineto{\pgfqpoint{2.066213in}{1.254671in}}%
\pgfpathlineto{\pgfqpoint{2.140371in}{1.433430in}}%
\pgfpathlineto{\pgfqpoint{2.214782in}{1.636491in}}%
\pgfpathlineto{\pgfqpoint{2.288928in}{1.729905in}}%
\pgfpathlineto{\pgfqpoint{2.363307in}{1.776669in}}%
\pgfpathlineto{\pgfqpoint{2.437539in}{1.787823in}}%
\pgfpathlineto{\pgfqpoint{2.512156in}{1.783875in}}%
\pgfpathlineto{\pgfqpoint{2.586680in}{1.800749in}}%
\pgfpathlineto{\pgfqpoint{2.661652in}{1.728123in}}%
\pgfpathlineto{\pgfqpoint{2.735749in}{1.678525in}}%
\pgfpathlineto{\pgfqpoint{2.809602in}{1.610062in}}%
\pgfpathlineto{\pgfqpoint{2.883838in}{1.566525in}}%
\pgfpathlineto{\pgfqpoint{2.958319in}{1.538290in}}%
\pgfpathlineto{\pgfqpoint{3.032549in}{1.588473in}}%
\pgfpathlineto{\pgfqpoint{3.107006in}{1.650100in}}%
\pgfpathlineto{\pgfqpoint{3.181440in}{1.676844in}}%
\pgfpathlineto{\pgfqpoint{3.255767in}{1.673224in}}%
\pgfpathlineto{\pgfqpoint{3.329981in}{1.733683in}}%
\pgfpathlineto{\pgfqpoint{3.406303in}{1.791305in}}%
\pgfpathlineto{\pgfqpoint{3.482751in}{1.762065in}}%
\pgfpathlineto{\pgfqpoint{3.555317in}{1.791489in}}%
\pgfpathlineto{\pgfqpoint{3.629320in}{1.812350in}}%
\pgfpathlineto{\pgfqpoint{3.703480in}{1.775167in}}%
\pgfpathlineto{\pgfqpoint{3.778027in}{1.760150in}}%
\pgfpathlineto{\pgfqpoint{3.852264in}{1.700647in}}%
\pgfpathlineto{\pgfqpoint{3.927050in}{1.658256in}}%
\pgfpathlineto{\pgfqpoint{4.002198in}{1.642026in}}%
\pgfpathlineto{\pgfqpoint{4.076525in}{1.597165in}}%
\pgfpathlineto{\pgfqpoint{4.150513in}{1.600557in}}%
\pgfpathlineto{\pgfqpoint{4.225575in}{1.614092in}}%
\pgfpathlineto{\pgfqpoint{4.302402in}{1.625427in}}%
\pgfpathlineto{\pgfqpoint{4.375531in}{1.617888in}}%
\pgfpathlineto{\pgfqpoint{4.449413in}{1.649169in}}%
\pgfpathlineto{\pgfqpoint{4.523592in}{1.552545in}}%
\pgfpathlineto{\pgfqpoint{4.598120in}{1.611013in}}%
\pgfpathlineto{\pgfqpoint{4.672196in}{1.664777in}}%
\pgfpathlineto{\pgfqpoint{4.747452in}{1.778673in}}%
\pgfpathlineto{\pgfqpoint{4.822039in}{1.748376in}}%
\pgfpathlineto{\pgfqpoint{4.897357in}{1.857351in}}%
\pgfpathlineto{\pgfqpoint{4.971388in}{1.854111in}}%
\pgfpathlineto{\pgfqpoint{5.045497in}{1.886015in}}%
\pgfpathlineto{\pgfqpoint{5.120047in}{1.823996in}}%
\pgfpathlineto{\pgfqpoint{5.197082in}{1.846674in}}%
\pgfpathlineto{\pgfqpoint{5.270517in}{1.774630in}}%
\pgfpathlineto{\pgfqpoint{5.345016in}{1.686320in}}%
\pgfpathlineto{\pgfqpoint{5.419524in}{1.654822in}}%
\pgfpathlineto{\pgfqpoint{5.493599in}{1.616314in}}%
\pgfusepath{stroke}%
\end{pgfscope}%
\begin{pgfscope}%
\pgfpathrectangle{\pgfqpoint{0.800000in}{0.528000in}}{\pgfqpoint{4.960000in}{3.696000in}}%
\pgfusepath{clip}%
\pgfsetrectcap%
\pgfsetroundjoin%
\pgfsetlinewidth{1.505625pt}%
\definecolor{currentstroke}{rgb}{0.580392,0.403922,0.741176}%
\pgfsetstrokecolor{currentstroke}%
\pgfsetdash{}{0pt}%
\pgfpathmoveto{\pgfqpoint{1.025455in}{1.721738in}}%
\pgfpathlineto{\pgfqpoint{1.100470in}{2.913420in}}%
\pgfpathlineto{\pgfqpoint{1.176378in}{3.587920in}}%
\pgfpathlineto{\pgfqpoint{1.250865in}{3.914005in}}%
\pgfpathlineto{\pgfqpoint{1.324834in}{3.985154in}}%
\pgfpathlineto{\pgfqpoint{1.398474in}{4.042024in}}%
\pgfpathlineto{\pgfqpoint{1.472807in}{4.017967in}}%
\pgfpathlineto{\pgfqpoint{1.547834in}{3.623859in}}%
\pgfpathlineto{\pgfqpoint{1.622068in}{2.920491in}}%
\pgfpathlineto{\pgfqpoint{1.696680in}{2.115215in}}%
\pgfpathlineto{\pgfqpoint{1.771657in}{1.395641in}}%
\pgfpathlineto{\pgfqpoint{1.846512in}{0.918833in}}%
\pgfpathlineto{\pgfqpoint{1.921313in}{0.698285in}}%
\pgfpathlineto{\pgfqpoint{1.997965in}{0.912675in}}%
\pgfpathlineto{\pgfqpoint{2.071739in}{1.165289in}}%
\pgfpathlineto{\pgfqpoint{2.145849in}{1.496492in}}%
\pgfpathlineto{\pgfqpoint{2.220040in}{1.721802in}}%
\pgfpathlineto{\pgfqpoint{2.294057in}{1.937904in}}%
\pgfpathlineto{\pgfqpoint{2.370736in}{2.012843in}}%
\pgfpathlineto{\pgfqpoint{2.443144in}{2.015987in}}%
\pgfpathlineto{\pgfqpoint{2.517953in}{1.896163in}}%
\pgfpathlineto{\pgfqpoint{2.591959in}{1.903063in}}%
\pgfpathlineto{\pgfqpoint{2.666301in}{1.811954in}}%
\pgfpathlineto{\pgfqpoint{2.740975in}{1.666259in}}%
\pgfpathlineto{\pgfqpoint{2.816885in}{1.509882in}}%
\pgfpathlineto{\pgfqpoint{2.890126in}{1.352016in}}%
\pgfpathlineto{\pgfqpoint{2.964049in}{1.263109in}}%
\pgfpathlineto{\pgfqpoint{3.038333in}{1.388751in}}%
\pgfpathlineto{\pgfqpoint{3.112811in}{1.539876in}}%
\pgfpathlineto{\pgfqpoint{3.187013in}{1.649076in}}%
\pgfpathlineto{\pgfqpoint{3.261789in}{1.746276in}}%
\pgfpathlineto{\pgfqpoint{3.337060in}{1.888992in}}%
\pgfpathlineto{\pgfqpoint{3.411535in}{1.909356in}}%
\pgfpathlineto{\pgfqpoint{3.486495in}{1.948492in}}%
\pgfpathlineto{\pgfqpoint{3.560473in}{1.941066in}}%
\pgfpathlineto{\pgfqpoint{3.634416in}{1.923830in}}%
\pgfpathlineto{\pgfqpoint{3.710522in}{1.728316in}}%
\pgfpathlineto{\pgfqpoint{3.783974in}{1.661162in}}%
\pgfpathlineto{\pgfqpoint{3.857782in}{1.573443in}}%
\pgfpathlineto{\pgfqpoint{3.931661in}{1.542404in}}%
\pgfpathlineto{\pgfqpoint{4.006331in}{1.496052in}}%
\pgfpathlineto{\pgfqpoint{4.080577in}{1.523695in}}%
\pgfpathlineto{\pgfqpoint{4.155559in}{1.550350in}}%
\pgfpathlineto{\pgfqpoint{4.229512in}{1.513705in}}%
\pgfpathlineto{\pgfqpoint{4.304405in}{1.672064in}}%
\pgfpathlineto{\pgfqpoint{4.378201in}{1.815588in}}%
\pgfpathlineto{\pgfqpoint{4.454021in}{1.905658in}}%
\pgfpathlineto{\pgfqpoint{4.528621in}{1.902574in}}%
\pgfpathlineto{\pgfqpoint{4.605285in}{1.836229in}}%
\pgfpathlineto{\pgfqpoint{4.678608in}{1.846554in}}%
\pgfpathlineto{\pgfqpoint{4.752888in}{1.764283in}}%
\pgfpathlineto{\pgfqpoint{4.827075in}{1.667427in}}%
\pgfpathlineto{\pgfqpoint{4.901997in}{1.627341in}}%
\pgfpathlineto{\pgfqpoint{4.977955in}{1.535112in}}%
\pgfpathlineto{\pgfqpoint{5.051662in}{1.339337in}}%
\pgfpathlineto{\pgfqpoint{5.125936in}{1.317442in}}%
\pgfpathlineto{\pgfqpoint{5.200351in}{1.373053in}}%
\pgfpathlineto{\pgfqpoint{5.274279in}{1.474950in}}%
\pgfpathlineto{\pgfqpoint{5.348673in}{1.635832in}}%
\pgfpathlineto{\pgfqpoint{5.423245in}{1.823661in}}%
\pgfpathlineto{\pgfqpoint{5.499111in}{1.940765in}}%
\pgfusepath{stroke}%
\end{pgfscope}%
\begin{pgfscope}%
\pgfpathrectangle{\pgfqpoint{0.800000in}{0.528000in}}{\pgfqpoint{4.960000in}{3.696000in}}%
\pgfusepath{clip}%
\pgfsetrectcap%
\pgfsetroundjoin%
\pgfsetlinewidth{1.505625pt}%
\definecolor{currentstroke}{rgb}{0.549020,0.337255,0.294118}%
\pgfsetstrokecolor{currentstroke}%
\pgfsetdash{}{0pt}%
\pgfpathmoveto{\pgfqpoint{1.025455in}{1.764886in}}%
\pgfpathlineto{\pgfqpoint{1.099620in}{2.913898in}}%
\pgfpathlineto{\pgfqpoint{1.174149in}{3.558171in}}%
\pgfpathlineto{\pgfqpoint{1.248389in}{3.899072in}}%
\pgfpathlineto{\pgfqpoint{1.322944in}{4.015046in}}%
\pgfpathlineto{\pgfqpoint{1.397696in}{4.053408in}}%
\pgfpathlineto{\pgfqpoint{1.472152in}{4.038181in}}%
\pgfpathlineto{\pgfqpoint{1.546810in}{3.807461in}}%
\pgfpathlineto{\pgfqpoint{1.622112in}{2.892007in}}%
\pgfpathlineto{\pgfqpoint{1.696146in}{1.935276in}}%
\pgfpathlineto{\pgfqpoint{1.770403in}{1.142588in}}%
\pgfpathlineto{\pgfqpoint{1.845009in}{0.847727in}}%
\pgfpathlineto{\pgfqpoint{1.919095in}{0.696000in}}%
\pgfpathlineto{\pgfqpoint{1.993646in}{0.919549in}}%
\pgfpathlineto{\pgfqpoint{2.067728in}{1.308145in}}%
\pgfpathlineto{\pgfqpoint{2.142923in}{1.711025in}}%
\pgfpathlineto{\pgfqpoint{2.216782in}{2.014251in}}%
\pgfpathlineto{\pgfqpoint{2.290852in}{2.161265in}}%
\pgfpathlineto{\pgfqpoint{2.365584in}{2.148964in}}%
\pgfpathlineto{\pgfqpoint{2.442579in}{2.094579in}}%
\pgfpathlineto{\pgfqpoint{2.515939in}{1.849080in}}%
\pgfpathlineto{\pgfqpoint{2.589407in}{1.612202in}}%
\pgfpathlineto{\pgfqpoint{2.663517in}{1.416346in}}%
\pgfpathlineto{\pgfqpoint{2.738313in}{1.338891in}}%
\pgfpathlineto{\pgfqpoint{2.812768in}{1.266319in}}%
\pgfpathlineto{\pgfqpoint{2.887563in}{1.334445in}}%
\pgfpathlineto{\pgfqpoint{2.961811in}{1.351090in}}%
\pgfpathlineto{\pgfqpoint{3.036241in}{1.501029in}}%
\pgfpathlineto{\pgfqpoint{3.109806in}{1.663374in}}%
\pgfpathlineto{\pgfqpoint{3.184273in}{1.844019in}}%
\pgfpathlineto{\pgfqpoint{3.259041in}{1.957619in}}%
\pgfpathlineto{\pgfqpoint{3.335888in}{1.963351in}}%
\pgfpathlineto{\pgfqpoint{3.410340in}{1.928278in}}%
\pgfpathlineto{\pgfqpoint{3.483889in}{1.812865in}}%
\pgfpathlineto{\pgfqpoint{3.558769in}{1.676027in}}%
\pgfpathlineto{\pgfqpoint{3.633581in}{1.572422in}}%
\pgfpathlineto{\pgfqpoint{3.708079in}{1.456641in}}%
\pgfpathlineto{\pgfqpoint{3.782596in}{1.450489in}}%
\pgfpathlineto{\pgfqpoint{3.856782in}{1.382421in}}%
\pgfpathlineto{\pgfqpoint{3.931191in}{1.501874in}}%
\pgfpathlineto{\pgfqpoint{4.005692in}{1.614832in}}%
\pgfpathlineto{\pgfqpoint{4.080380in}{1.764748in}}%
\pgfpathlineto{\pgfqpoint{4.155377in}{1.893576in}}%
\pgfpathlineto{\pgfqpoint{4.233004in}{1.910011in}}%
\pgfpathlineto{\pgfqpoint{4.305852in}{1.863269in}}%
\pgfpathlineto{\pgfqpoint{4.379262in}{1.802717in}}%
\pgfpathlineto{\pgfqpoint{4.453286in}{1.715677in}}%
\pgfpathlineto{\pgfqpoint{4.528300in}{1.673333in}}%
\pgfpathlineto{\pgfqpoint{4.603131in}{1.643945in}}%
\pgfpathlineto{\pgfqpoint{4.677857in}{1.572343in}}%
\pgfpathlineto{\pgfqpoint{4.751488in}{1.555939in}}%
\pgfpathlineto{\pgfqpoint{4.826885in}{1.542800in}}%
\pgfpathlineto{\pgfqpoint{4.901354in}{1.597152in}}%
\pgfpathlineto{\pgfqpoint{4.975747in}{1.571953in}}%
\pgfpathlineto{\pgfqpoint{5.050397in}{1.609695in}}%
\pgfpathlineto{\pgfqpoint{5.124837in}{1.696509in}}%
\pgfpathlineto{\pgfqpoint{5.199617in}{1.743086in}}%
\pgfpathlineto{\pgfqpoint{5.274348in}{1.720460in}}%
\pgfpathlineto{\pgfqpoint{5.349444in}{1.668019in}}%
\pgfpathlineto{\pgfqpoint{5.424222in}{1.585668in}}%
\pgfpathlineto{\pgfqpoint{5.498989in}{1.533279in}}%
\pgfusepath{stroke}%
\end{pgfscope}%
\begin{pgfscope}%
\pgfsetrectcap%
\pgfsetmiterjoin%
\pgfsetlinewidth{0.803000pt}%
\definecolor{currentstroke}{rgb}{0.000000,0.000000,0.000000}%
\pgfsetstrokecolor{currentstroke}%
\pgfsetdash{}{0pt}%
\pgfpathmoveto{\pgfqpoint{0.800000in}{0.528000in}}%
\pgfpathlineto{\pgfqpoint{0.800000in}{4.224000in}}%
\pgfusepath{stroke}%
\end{pgfscope}%
\begin{pgfscope}%
\pgfsetrectcap%
\pgfsetmiterjoin%
\pgfsetlinewidth{0.803000pt}%
\definecolor{currentstroke}{rgb}{0.000000,0.000000,0.000000}%
\pgfsetstrokecolor{currentstroke}%
\pgfsetdash{}{0pt}%
\pgfpathmoveto{\pgfqpoint{5.760000in}{0.528000in}}%
\pgfpathlineto{\pgfqpoint{5.760000in}{4.224000in}}%
\pgfusepath{stroke}%
\end{pgfscope}%
\begin{pgfscope}%
\pgfsetrectcap%
\pgfsetmiterjoin%
\pgfsetlinewidth{0.803000pt}%
\definecolor{currentstroke}{rgb}{0.000000,0.000000,0.000000}%
\pgfsetstrokecolor{currentstroke}%
\pgfsetdash{}{0pt}%
\pgfpathmoveto{\pgfqpoint{0.800000in}{0.528000in}}%
\pgfpathlineto{\pgfqpoint{5.760000in}{0.528000in}}%
\pgfusepath{stroke}%
\end{pgfscope}%
\begin{pgfscope}%
\pgfsetrectcap%
\pgfsetmiterjoin%
\pgfsetlinewidth{0.803000pt}%
\definecolor{currentstroke}{rgb}{0.000000,0.000000,0.000000}%
\pgfsetstrokecolor{currentstroke}%
\pgfsetdash{}{0pt}%
\pgfpathmoveto{\pgfqpoint{0.800000in}{4.224000in}}%
\pgfpathlineto{\pgfqpoint{5.760000in}{4.224000in}}%
\pgfusepath{stroke}%
\end{pgfscope}%
\begin{pgfscope}%
\definecolor{textcolor}{rgb}{0.000000,0.000000,0.000000}%
\pgfsetstrokecolor{textcolor}%
\pgfsetfillcolor{textcolor}%
\pgftext[x=3.280000in,y=4.307333in,,base]{\color{textcolor}\sffamily\fontsize{12.000000}{14.400000}\selectfont Measured yaw speed}%
\end{pgfscope}%
\begin{pgfscope}%
\pgfsetbuttcap%
\pgfsetmiterjoin%
\definecolor{currentfill}{rgb}{1.000000,1.000000,1.000000}%
\pgfsetfillcolor{currentfill}%
\pgfsetfillopacity{0.800000}%
\pgfsetlinewidth{1.003750pt}%
\definecolor{currentstroke}{rgb}{0.800000,0.800000,0.800000}%
\pgfsetstrokecolor{currentstroke}%
\pgfsetstrokeopacity{0.800000}%
\pgfsetdash{}{0pt}%
\pgfpathmoveto{\pgfqpoint{4.953237in}{2.889746in}}%
\pgfpathlineto{\pgfqpoint{5.662778in}{2.889746in}}%
\pgfpathquadraticcurveto{\pgfqpoint{5.690556in}{2.889746in}}{\pgfqpoint{5.690556in}{2.917523in}}%
\pgfpathlineto{\pgfqpoint{5.690556in}{4.126778in}}%
\pgfpathquadraticcurveto{\pgfqpoint{5.690556in}{4.154556in}}{\pgfqpoint{5.662778in}{4.154556in}}%
\pgfpathlineto{\pgfqpoint{4.953237in}{4.154556in}}%
\pgfpathquadraticcurveto{\pgfqpoint{4.925460in}{4.154556in}}{\pgfqpoint{4.925460in}{4.126778in}}%
\pgfpathlineto{\pgfqpoint{4.925460in}{2.917523in}}%
\pgfpathquadraticcurveto{\pgfqpoint{4.925460in}{2.889746in}}{\pgfqpoint{4.953237in}{2.889746in}}%
\pgfpathlineto{\pgfqpoint{4.953237in}{2.889746in}}%
\pgfpathclose%
\pgfusepath{stroke,fill}%
\end{pgfscope}%
\begin{pgfscope}%
\pgfsetrectcap%
\pgfsetroundjoin%
\pgfsetlinewidth{1.505625pt}%
\definecolor{currentstroke}{rgb}{0.121569,0.466667,0.705882}%
\pgfsetstrokecolor{currentstroke}%
\pgfsetdash{}{0pt}%
\pgfpathmoveto{\pgfqpoint{4.981015in}{4.042088in}}%
\pgfpathlineto{\pgfqpoint{5.119904in}{4.042088in}}%
\pgfpathlineto{\pgfqpoint{5.258793in}{4.042088in}}%
\pgfusepath{stroke}%
\end{pgfscope}%
\begin{pgfscope}%
\definecolor{textcolor}{rgb}{0.000000,0.000000,0.000000}%
\pgfsetstrokecolor{textcolor}%
\pgfsetfillcolor{textcolor}%
\pgftext[x=5.369904in,y=3.993477in,left,base]{\color{textcolor}\sffamily\fontsize{10.000000}{12.000000}\selectfont 25}%
\end{pgfscope}%
\begin{pgfscope}%
\pgfsetrectcap%
\pgfsetroundjoin%
\pgfsetlinewidth{1.505625pt}%
\definecolor{currentstroke}{rgb}{1.000000,0.498039,0.054902}%
\pgfsetstrokecolor{currentstroke}%
\pgfsetdash{}{0pt}%
\pgfpathmoveto{\pgfqpoint{4.981015in}{3.838231in}}%
\pgfpathlineto{\pgfqpoint{5.119904in}{3.838231in}}%
\pgfpathlineto{\pgfqpoint{5.258793in}{3.838231in}}%
\pgfusepath{stroke}%
\end{pgfscope}%
\begin{pgfscope}%
\definecolor{textcolor}{rgb}{0.000000,0.000000,0.000000}%
\pgfsetstrokecolor{textcolor}%
\pgfsetfillcolor{textcolor}%
\pgftext[x=5.369904in,y=3.789620in,left,base]{\color{textcolor}\sffamily\fontsize{10.000000}{12.000000}\selectfont 50}%
\end{pgfscope}%
\begin{pgfscope}%
\pgfsetrectcap%
\pgfsetroundjoin%
\pgfsetlinewidth{1.505625pt}%
\definecolor{currentstroke}{rgb}{0.172549,0.627451,0.172549}%
\pgfsetstrokecolor{currentstroke}%
\pgfsetdash{}{0pt}%
\pgfpathmoveto{\pgfqpoint{4.981015in}{3.634374in}}%
\pgfpathlineto{\pgfqpoint{5.119904in}{3.634374in}}%
\pgfpathlineto{\pgfqpoint{5.258793in}{3.634374in}}%
\pgfusepath{stroke}%
\end{pgfscope}%
\begin{pgfscope}%
\definecolor{textcolor}{rgb}{0.000000,0.000000,0.000000}%
\pgfsetstrokecolor{textcolor}%
\pgfsetfillcolor{textcolor}%
\pgftext[x=5.369904in,y=3.585762in,left,base]{\color{textcolor}\sffamily\fontsize{10.000000}{12.000000}\selectfont 75}%
\end{pgfscope}%
\begin{pgfscope}%
\pgfsetrectcap%
\pgfsetroundjoin%
\pgfsetlinewidth{1.505625pt}%
\definecolor{currentstroke}{rgb}{0.839216,0.152941,0.156863}%
\pgfsetstrokecolor{currentstroke}%
\pgfsetdash{}{0pt}%
\pgfpathmoveto{\pgfqpoint{4.981015in}{3.430516in}}%
\pgfpathlineto{\pgfqpoint{5.119904in}{3.430516in}}%
\pgfpathlineto{\pgfqpoint{5.258793in}{3.430516in}}%
\pgfusepath{stroke}%
\end{pgfscope}%
\begin{pgfscope}%
\definecolor{textcolor}{rgb}{0.000000,0.000000,0.000000}%
\pgfsetstrokecolor{textcolor}%
\pgfsetfillcolor{textcolor}%
\pgftext[x=5.369904in,y=3.381905in,left,base]{\color{textcolor}\sffamily\fontsize{10.000000}{12.000000}\selectfont 100}%
\end{pgfscope}%
\begin{pgfscope}%
\pgfsetrectcap%
\pgfsetroundjoin%
\pgfsetlinewidth{1.505625pt}%
\definecolor{currentstroke}{rgb}{0.580392,0.403922,0.741176}%
\pgfsetstrokecolor{currentstroke}%
\pgfsetdash{}{0pt}%
\pgfpathmoveto{\pgfqpoint{4.981015in}{3.226659in}}%
\pgfpathlineto{\pgfqpoint{5.119904in}{3.226659in}}%
\pgfpathlineto{\pgfqpoint{5.258793in}{3.226659in}}%
\pgfusepath{stroke}%
\end{pgfscope}%
\begin{pgfscope}%
\definecolor{textcolor}{rgb}{0.000000,0.000000,0.000000}%
\pgfsetstrokecolor{textcolor}%
\pgfsetfillcolor{textcolor}%
\pgftext[x=5.369904in,y=3.178048in,left,base]{\color{textcolor}\sffamily\fontsize{10.000000}{12.000000}\selectfont 125}%
\end{pgfscope}%
\begin{pgfscope}%
\pgfsetrectcap%
\pgfsetroundjoin%
\pgfsetlinewidth{1.505625pt}%
\definecolor{currentstroke}{rgb}{0.549020,0.337255,0.294118}%
\pgfsetstrokecolor{currentstroke}%
\pgfsetdash{}{0pt}%
\pgfpathmoveto{\pgfqpoint{4.981015in}{3.022802in}}%
\pgfpathlineto{\pgfqpoint{5.119904in}{3.022802in}}%
\pgfpathlineto{\pgfqpoint{5.258793in}{3.022802in}}%
\pgfusepath{stroke}%
\end{pgfscope}%
\begin{pgfscope}%
\definecolor{textcolor}{rgb}{0.000000,0.000000,0.000000}%
\pgfsetstrokecolor{textcolor}%
\pgfsetfillcolor{textcolor}%
\pgftext[x=5.369904in,y=2.974191in,left,base]{\color{textcolor}\sffamily\fontsize{10.000000}{12.000000}\selectfont 150}%
\end{pgfscope}%
\end{pgfpicture}%
\makeatother%
\endgroup%
}
    \end{minipage}
    \label{fig:tune-yaw-prop}
    \caption{Variation of (a) input position and (b) output velocity for different values of $K_{P}$ and $K_I=0$, $K_D=0$ while the yaw controller is engaged.}
\end{figure}


\subsubsection{Integral component}
\subsubsection{Derivative component}

%%%%%%%%%%%%%%%%%%%%%%%%%

% \begin{figure}
%   \centering
%   \makebox[\textwidth][c]{
%   \includegraphics[width=.52\textwidth]{img/pid/yaw/yaw_pos_prop_i0_d0.png}
%   \includegraphics[width=.52\textwidth]{img/pid/yaw/yaw_vel_prop_i0_d0.jpg}}
%   \caption{Variation of (a) input position and (b) output velocity for different values of $K_{P}$ and $K_I=0$, $K_D=0$ while the yaw controller is engaged.}\label{fig:tune-yaw-prop}
% \end{figure}





\subsection{Forward controller}

After the yaw controller is tuned, it is time to move to the forward controller.
The process will be the same as for the yaw controller. Looking at the process gain in this case, it can be noted that a positive output in the controller creates a positive velocity in the forward direction, drawing the vehicle closer to the target person. The input variable is the height of the detected person in the camera field of view, normalized to the height of the field of view, which increases as the drone gets closer to the target. Therefore, a positive output velocity results in an increasing input to the controller, which means that the controller gain already has the correct sign.

The starting position for tuning the forward controller needs to present an offset from the reference position in the distance between the person and the vehicle (x-axis). To achieve this, the person model will be situated at the $x=500, y=0$ coordinates, with the vehicle remaining at the origin in the simulated world. The chosen starting position (closer than the reference point) will result in an initial negative velocity output to move the vehicle away from the person. Figure \ref{fig:tune-ref-pos-fwd} shows the starting position in the simulator.


\begin{figure}[H]
  \centering
  \includegraphics[width=\textwidth, keepaspectratio]{img/pid/tune-ref-pos-fwd.jpg}
  \caption{Starting position of the simulator for tuning the forward controller. The human model is situated 450 units forward and centred from the vehicle position.}\label{fig:tune-ref-pos-fwd}
\end{figure}



\subsection{PID tuning validation}
\label{subsec:pid-test-controller}

The final validation of the tuning obtained for the controllers will be performed using the \texttt{test-controller} tool described in Section \ref{subsec:pid-tools}. The goal is to check the step response of the controllers for different starting distances and validate their performance when engaged simultaneously. For the first test, the starting distances will vary along the y-axis, and for the second one, they will vary along the x-axis.

In the first test, the positions will vary along the y-axis, meaning that the figure will move from left to right in the field of view of the vehicle. The y-coordinates to be tested will range from -150 to 150 units in increments of 50, while the x-coordinate of the figure in the simulated world will remain fixed at $x=500$.

The results of the first run are shown in Figure \ref{fig:validate-yaw}. The y-coordinates tested range from -150 to 150 units in increments of 50, while the x-coordinate of the figure in the simulated world remains fixed at $x=500$. These changes in position mean that the figure moves from left to right in the field of view of the vehicle, following a line parallel to the lateral axis of the vehicle. To counteract this movement, both the yaw and forward controllers need to engage to reach the reference position.

The plots in Figure \ref{fig:validate-yaw} depict the changes in the normalized horizontal distance and normalized figure height detected by the person recognition algorithm during the time it takes for the vehicle to reach the target distance from the human figure for each tested start position. The target is considered reached when the error is less than 2\% and the output speed at the controller is less than 10\% of the maximum value. The full execution of the \texttt{test-controller} tool with varying lateral positions can be seen in the video found in the project's \href{https://l-gonz.github.io/tfg-giaa-dronecontrol/videos/test-yaw-controller}{page}\footnote{\url{https://l-gonz.github.io/tfg-giaa-dronecontrol/videos/test-yaw-controller}}.

\begin{figure}[H]
  \centering
  \makebox[\textwidth][c]{
  \includegraphics[width=.52\linewidth]{img/pid/validation_yaw.png}
  \includegraphics[width=.52\linewidth]{img/pid/validation_yaw_2.png}}
  \caption{Changes over time in detected horizontal position and height as input for the controllers with different starting positions in the y-axis.}
  \label{fig:validate-yaw}
\end{figure}

During execution, the yaw controller introduces a negative yaw velocity when the figure is on the left half of the camera's field of view and a positive yaw velocity when the figure is on the right half, aiming to achieve a target horizontal distance of 0.5 (centred in the camera image). On the other hand, the forward controller outputs a negative forward velocity to reduce the detected height of the figure from around 0.44 to the target value of 0.36.

Looking at Figure \ref{fig:validate-yaw}a, it can be observed that most of the time is spent initiating the movement towards the target. Once the vehicle starts moving, there is not much difference between the -50 and the -150 steps in the time it takes to reach the target position, with the former taking around 3 seconds and the latter taking approximately 3.6 seconds. 

In Figure \ref{fig:validate-yaw}b, the trajectories appear quite similar since the starting distance to the target is the same for all the cases. For each run, the controller guides the vehicle to move backwards, ensuring that the figure stays sufficiently far away, resulting in a decrease in the detected height. Additionally, a detection anomaly can be observed in Figure \ref{fig:validate-yaw}b. For the starting position at $y=100$ (yellow line), there is a brief initial period where the detected height is very small due to a detection error in the first frames processed by the computer vision algorithm. However, within half a second, the detection stabilizes, and the controller successfully guides the vehicle to the target position without significantly impacting the time taken or the final position. This demonstrates that the controllers are capable of recovering from detection errors without compromising the vehicle's movement.


\begin{figure}[H]
  \centering
  \includegraphics[width=.7\textwidth, keepaspectratio]{img/pid/validation_fwd.png}
  \caption{Changes over time in detected height as input for the forward controller with different starting positions in the x-axis.}
  \label{fig:validate-fwd}
\end{figure}


To further validate the performance of the forward controller, the \texttt{test-controller} tool can be used with varying positions along the x-axis. This means that the tests are conducted with the human figure at different distances along the longitudinal axis of the vehicle, closer and further away than the reference distance. Throughout the process, the figure is kept centred in the camera's field of view (y position remains 0). Therefore, in this scenario, the yaw controller does not need to be considered.

Figure \ref{fig:validate-fwd} illustrates the changes over time in the input to the forward controller for each starting position as the vehicle moves towards the target position. The graph highlights significant differences in how the controller responds to positions closer or further away from the target distance. When the person is very close to the vehicle, there are substantial differences in detected heights for minor changes in longitudinal distance, leading to the rapid movement of the vehicle away from its start position. Conversely, when the person is further away from the vehicle than the target, the same differences in distance are associated with minor differences in detected height. As a result, the controller determines a smaller velocity output compared to the cases when the person is closer to the vehicle. Consequently, it takes a longer time for the vehicle to reach the target position.

Notably, even when the person is so close to the vehicle that part of their figure falls outside the camera's field of view, the pose detection mechanism functions well enough to estimate the person's continued position outside the image. This can be seen on Figure \ref{fig:validate-fwd} for the case of $x=300$, where the detected height starts lower than it should be and gradually increases as the full person starts to fit in the camera's field of view.
\\ \\


\begin{figure}[H]
  \centering
  \includegraphics[width=\textwidth, keepaspectratio]{img/video-follow-sitl.png}
  \caption{Single frame from the video showing the movement of the drone in response to changes in the position of the tracked person.}
  \label{fig:airsim-test-follow}
\end{figure}

Overall, the results of the validation process indicate that the tuned controllers perform effectively, and accurately follow the person across different starting positions, even when detection errors occur. They successfully respond to variations in the person's distance and angle from the vehicle, allowing for accurate tracking and movement towards the target relative position.

The selected coefficients for the controllers can now be applied to the complete follow solution in order to assess the expected performance of the vehicle in real flights. To provide a visual demonstration, a video showcasing the drone's movement using these parameter values can be accessed \href{https://l-gonz.github.io/tfg-giaa-dronecontrol/videos/test-sitl-follow}{here}\footnote{\url{https://l-gonz.github.io/tfg-giaa-dronecontrol/videos/test-sitl-follow}}. Additionally, Figure \ref{fig:airsim-test-follow} displays a frame extracted from the video, giving a glimpse of the drone's behaviour during the follow operation.




\section{PX4 HITL simulation and validation}
\label{sec:test-4-hitl}


This section will delve into the practical implementation of the hardware-in-the-loop (HITL) mode using QGroundControl, Pixhawk 4 board, and the AirSim simulator. The objective is to transition from using a simulated version of the flight stack running on Linux (SITL) to executing the PX4 software natively on a physical Pixhawk board with simulated input and output. This simulation mode allows for real-time testing and validation of the system by integrating physical hardware with simulated environments. 

Achieving a seamless interaction between the simulator, board, and external control applications requires several configuration steps and setting up wired connections. This configuration will be explored to achieve a system that can run the developed control solution, replicating the same behaviour demonstrated in the previous section. In the first part, the DroneVisionControl application will be run from the simulation computer, as outlined in Figure \ref{fig:hitl-connections}. In the second part, after the performance of the flight board is validated, the execution of the DroneVisionControl application will be moved to a separate companion computer, the Raspberry Pi 4. This dual approach allows testing both the offboard and onboard configurations.

\subsection{HITL validation with simulation computer (offboard configuration)}

The purpose of this section is to test the offboard configuration on the ground before any flight tests, with the PX4 flight stack running on the dedicated autopilot board and the DroneVisionControl application running on the simulation computer. To run the flight stack on the Pixhawk board without flying, it is necessary to use the HITL simulation mode, where the flight mechanics will still be simulated by AirSim. The steps for setting up the HITL simulation environment include configuring the flight board through QGroundControl, configuring AirSim to connect to a physical board, and adding new communication channels for additional control mechanisms (DroneVisionControl and RC). QGroundControl provides a specific quadcopter HITL airframe configuration, which initializes the board with all the necessary parameters to activate the simulation mode. The details of these parameters can be found in Appendix \ref{app:install-hitl}. Configuring the board using QGroundControl is a straightforward process. Simply connecting the Pixhawk 4 board to the computer through its debug Micro-USB port will make QGroundControl automatically detect and establish a connection with the board.


On the AirSim side, enabling the simulator to work in HITL mode requires modifying its configuration file. Specifically, the option to accept serial connections needs to be enabled. This file is described in detail in Appendix \ref{app:airsim-config}.
It is important to note that both QGroundControl and AirSim cannot simultaneously establish a connection to the Pixhawk board through the same USB port. Only one of them can be active and connected to the board at any given time. Consequently, QGroundControl must be shut off while conducting the simulation to allow AirSim to establish the necessary communication with the board.


To test the complete system configuration for HITL, as outlined in Figure \ref{fig:hitl-connections}, the Pixhawk board requires an additional communication channel dedicated to the MAVLink exchange with the DroneVisionControl application. This channel is achieved by adding a telemetry radio to the board, which will connect to a counterpart radio on the simulation computer. This wireless link, along with the already existing USB connection, will provide the two separate MAVLink channels needed to complete the environment.
As the AirSim simulator requires a higher update rate compared to the DroneVisionControl application, it is important to keep the Pixhawk to AirSim connection on the wired link. The DroneVisionControl application, on the other hand, sends and receives commands from the Pixhawk at a lower rate, as it depends on the results of the slower computer vision algorithms. The data transmission rate of the radio link is, therefore, sufficient for this application.

Since the flight stack now runs on a physical controller, it becomes possible to attach an RC antenna to the \texttt{PPM RC} port of the board. This antenna allows the vehicle to be flown using an independent RC controller \cite{configure-rc}. By configuring the switches in the RC unit for flight mode change or as a kill switch, additional tests can be carried out to verify the developed safety features described in Section \ref{subsec:safety}, like interrupting autonomous flight upon flight mode changes or upon loss of signal from the RC controller.


After all the necessary connections are set up and the AirSim simulator started, the control program can be started with the following command:
\begin{minted}[breaklines, fontsize=\footnotesize, baselinestretch=1]{bash}
dronevisioncontrol follow --sim --serial COM[X]:57600
\end{minted}
The \texttt{COM[X]:57600} section describes the serial connection to the telemetry radio, where the COM port number will vary depending on the particular USB port to which the telemetry radio is connected, and the baudrate specified is 57600.



\subsection{HITL validation with Raspberry Pi (onboard configuration)}
\label{sec:test-5-rpi}

The next crucial step in transitioning from a fully simulated environment to real flight is to establish a connection between the future onboard computer, the Raspberry Pi 4, and the Pixhawk flight controller. This connection enables conducting more realistic tests using the same hardware as during flight tests to allow the identification of potential hardware performance issues. The tests mimic the onboard computer configuration outlined in Figure \ref{fig:onboard-config}, with the DroneVisionControl application running on an onboard companion computer.

Figure \ref{fig:hitl-setup-picture} provides an overview of the connections used for HITL testing with the Raspberry Pi. The main addition from the previous section is the direct connection between the Pixhawk board and the Raspberry Pi's I/O pins. While the inclusion of the telemetry radio is not strictly required in this scenario, it enables maintaining a simultaneous connection to QGroundControl on the ground station or simulation computer, facilitating better oversight and monitoring.

\begin{figure}
  \centering
  \includegraphics[width=0.7\textwidth, keepaspectratio]{img/hitl-setup-picture.jpg}
  \caption{Pixhawk 4 board connected to a Raspberry Pi running the DroneVisionControl application and a Windows computer running the AirSim simulator. The setup includes a telemetry radio for QGroundControl and an RC receiver for manual control.}
  \label{fig:hitl-setup-picture}
\end{figure}

Before any tests can begin, the operating system of the Raspberry needs to be set up for the DroneVisionControl application. A detailed explanation of the complete installation process, along with all the necessary libraries and dependencies for the Raspberry Pi, is included in Appendix \ref{app:install-hitl}. To conveniently control the Raspberry Pi during the installation, a remote desktop connection is recommended. This allows for transmitting screen contents, as well as mouse and keyboard input, over a local network. Thus, accessing the Pi's desktop from the ground station computer becomes feasible, even during flight. One available option to achieve this is XRDP \cite{xrdp-front}, an open-source implementation of a Microsoft Remote Desktop Protocol server that is compatible with the Raspberry OS.

To ensure successful progress towards autonomous flight, several key characteristics of the Raspberry Pi must be addressed:
\begin{enumerate}
    \item Power supply. Capacity to function when powered by a battery.
    \item Serial connection. Pixhawk to Raspberry Pi and Pixhawk to the simulation computer.
    \item DroneVisionControl software. Ability to run the DroneVisionControl application and its dependencies.
    \item Performance of computer vision algorithms with limited processing power.
\end{enumerate}


\subsubsection{Verify power supply}

The first step is to verify that the Raspberry Pi can be powered by a secondary battery connected to the board's power port by a USB-C cable instead of the standard AC power supply. Through this battery, sufficient power must be provided to the Raspberry Pi to maintain enough processing speed for the image processing software and to power the camera connected to the board. The performance differences between powering the Raspberry board with the AC supply and the battery are detailed in Section \ref{subsec:performance}. At this point, it is enough to check that the \texttt{test-camera} utility can run at an acceptable frame rate with either hand or pose detection enabled on previously recorded images from the camera or simulator.


\subsubsection{Verify serial connections}

The most important connection that needs to be verified is the MAVLink communication channel between the flight controller and the onboard computer. The custom connector described in Table \ref{tab:wiring-telem} is used for this purpose, with the \texttt{TELEM2} port on the Pixhawk board connected to the TX/RX UART pins on the Raspberry Pi's GPIO header. Additionally, configuration is required for the flight controller and the companion computer. For the PX4 flight controller, only the \texttt{TELEM1} port of the Pixhawk is configured for telemetry radio. An additional MAVLink channel must be configured in QGroundControl by setting the correct values to the parameters listed in Table \ref{tab:telem2-params}.

\begin{table}[h!]
 \begin{center}
  \begin{tabular}{l|l}
    Parameter name & Value \\ \hline
    MAV\_1\_CONFIG & TELEM2 \\
    SER\_TEL2\_BAUD & 921600 \\
  \end{tabular}
  \caption{PX4 parameters that require configuration to enable MAVLink communication through the secondary telemetry port.}
  \label{tab:telem2-params}
 \end{center}
\end{table}

On the Raspberry Pi side, the serial port is configured by default to exchange shell messages. This needs to be disabled using the \texttt{raspi-config} command-line utility and selecting the following steps: Interface options -> Serial Port -> Disable login shell, enable serial port hardware (see Figure \ref{fig:serial-connection}). After making these changes, the \texttt{/dev/serial0} address can be used to communicate with the device at the baud rate configured in QGroundControl.

\begin{figure}
  \centering
  \makebox[\textwidth][c]{
  \includegraphics[width=.615\textwidth, keepaspectratio]{img/raspi-config.png}
  \includegraphics[width=.385\textwidth, keepaspectratio]{img/rpi-pixhawk-serial.jpg}}
  \caption{a) Picture of Raspberry's \texttt{raspi-config} and b) close-up of Pixhawk to Pi cable connection.}
  \label{fig:serial-connection}
\end{figure}


The remaining connections serve to communicate the Pixhawk board and the simulation computer. Two different channels are needed to be able to exchange information with QGroundControl and AirSim at the same time. The connection to AirSim will be implemented through the development-only micro-USB port on the Pixhawk board and the connection to QGroundControl through the telemetry radio. Due to the substantially lower throughput of the telemetry radio in comparison with the USB cable, using the telemetry radio to communicate between the Pixhawk board and AirSim will result in inferior performance since there is a more significant amount of messages transferred than between QGroundControl and the Pixhawk.



\subsubsection{Verify DroneVisionControl software}

To validate the complete configuration, the test camera utility will be used by executing the following command in the Raspberry Pi:

\begin{minted}[breaklines, fontsize=\footnotesize, baselinestretch=1]{bash}
dronevisioncontrol tools test-camera --hardware /dev/serial0:921600 --sim <AirSim host IP> --pose-detection
\end{minted}

This command connects DroneVisionControl to PX4 via the hardware address \texttt{/dev/serial0} at a baudrate of 921600. Meanwhile, the images from the simulator's virtual camera are sent to the Raspberry Pi over the local area network. The \texttt{<AirSim host IP>} parameter defines the IP the application will connect to to receive these images.

The result from the execution is shown in Figure \ref{fig:rpi-airsim-test}. On the right side, the remote connection to the Raspberry Pi's desktop is displayed, showing the output of the DroneVisionControl program running the pose detection algorithm on the images received from the simulator. On the left side of the figure, the AirSim simulator renders the movements of the vehicle as it responds to the instructions from the flight controller that listens to the companion computer.

\begin{figure}[H]
  \centering
  \includegraphics[width=\textwidth, keepaspectratio]{img/airsim-rpi-test.png}
  \caption{Left: AirSim simulator on Windows host. Right: RPi desktop with DroneVisionControl application and pose output.}
  \label{fig:rpi-airsim-test}
\end{figure}


\subsection{Performance analysis}
\label{subsec:performance}

One crucial question that remains to be answered before the vehicle can take flight using this hardware and software configuration is whether the Raspberry Pi 4b's modest processor, a quad-core ARM Cortex-A72 64-bit SoC running at 1.5GHz, can handle the detection and tracking algorithms with sufficient performance and respond effectively to real-time movement. To address this matter, the average time spent by the program on each task in the running loop can be calculated and analyzed under different scenarios. This analysis will help estimate the maximum speed at which the algorithm can track a person.


Based on the running loop for the follow solution described in Section \ref{sec:follow} and shown in Figure \ref{fig:follow-loop}, the processing cost can be divided into several distinct areas that can be measured independently: image processing, offboard control, manual input, and released thread (overhead from the async execution). Figure \ref{fig:perf-sitl-sim} illustrates the time allocation for each task during an average run of the follow solution with PX4 running in SITL mode and AirSim as the simulator. DroneVisionControl was also run on the simulation computer. The time consumption was obtained by measuring the time at the start and end of each task, calculating the time difference between them and averaging across every iteration of the main loop. The utility for this process can be found in the measure function inside the application's utils module\footnote{\url{https://github.com/l-gonz/tfg-giaa-dronecontrol/blob/main/dronecontrol/common/utils.py}}.

\begin{figure}[H]
  \centering
  \includegraphics[width=.9\textwidth, keepaspectratio]{img/sitl-performance.png}
  \caption{Average percentages of a loop spent by each task in the follow solution with their corresponding average absolute times in seconds (SITL + AirSim configuration).}
  \label{fig:perf-sitl-sim}
\end{figure}


The most time-consuming task is image processing, which accounts for approximately 89\% of the total loop time. To gain further insight into the time allocation within the image processing task, the work can be further divided into three subtasks:
\begin{enumerate}
    \item \textbf{Get frame}, which involves requesting a new frame from the video source.
    \item \textbf{Process pose}, which entails sending the frame to the MediaPipe library for detection and tracking.
    \item \textbf{Detect}, which involves calculating the bounding box coordinates and determining whether it represents a valid pose.
\end{enumerate}

As shown in \ref{fig:perf-sitl-sim} (right), the subtasks take 40\%, 48\%, and 1\% of the total loop time, respectively. The average time for each iteration of the loop is 0.095 seconds, resulting in an average performance of 10.5 \acrfull{fps}.

Similar measurements were conducted for different hardware combinations, employing varying degrees of simulation while running the follow solution in offboard mode and connected to the AirSim simulator. The hardware combinations tested include:
\begin{enumerate}
    \item All simulated hardware: PX4 in SITL mode + DroneVisionControl on the simulation computer + images from the AirSim simulator as the video source.
    \item Simulated hardware with real images: PX4 in SITL mode + DroneVisionControl on the simulation computer + images from an attached camera as the video source.
    \item Test hardware with AC power supply: PX4 in HITL mode on Pixhawk 4 + DroneVisionControl on Raspberry Pi powered by AC + images from an attached camera as the video source.
    \item Test hardware with battery: PX4 in HITL mode on Pixhawk 4 + DroneVisionControl on Raspberry Pi powered by a battery + images from an attached camera as the video source.
\end{enumerate}


\begin{figure}[H]
  \centering
  \includegraphics[width=\textwidth, keepaspectratio]{img/performance-graph.png}
  \caption{Average FPS and time spent on each task per iteration of the follow solution for the different hardware configurations.}
  \label{fig:perf-analysis}
\end{figure}


Figure \ref{fig:perf-analysis} presents the average measurements obtained for all the analyzed hardware combinations. In the first test, it is observed that retrieving each new frame from the AirSim simulator takes significantly more time compared to using an external camera. This discrepancy arises from performance limitations in the simulation running on Unreal Engine, which will not be a factor once the simulation engine is no longer used. In the second test, replacing the simulator images with the feed from an external camera results in faster image retrieval and the highest performance among all the tests, averaging 14 FPS with peaks exceeding 20 FPS.

In the third and fourth tests, the image processing calculations are performed on the onboard Raspberry Pi computer, leading to a 400\% to 500\% increase in the time required for pose processing. Additionally, there is also a noticeable difference in performance observed for the Raspberry's processor between powering the computer through the AC power supply and through an external battery. This difference stems from the fact that the former supplies 3A of current to the board and the latter only 2A.

These measurements provide insights into the board's expected behaviour during actual flights, with the fourth test (hardware with battery) representing the closest approximation to real flight conditions. From the results, it is expected that a performance of approximately 3 FPS can be sustained during flight, which gives a time between frames of approximately 0.3 seconds. Considering that the camera's field of view for flight tests covers approximately four meters at the target follow distance, the person being tracked should be able to move at a speed of 3-4 m/s with the drone maintaining line of sight. This performance is satisfactory enough for this project, but there is room for improvement by, for example, optimizing the image processing algorithm or using more powerful hardware for the onboard companion computer.

\section{Quadcopter flight tests}

After validating the performance and safety of the control algorithms in the simulated environment, the next step involves conducting flight tests with a fully-built physical UAV. This final phase of the validation process aims to assess the performance of the developed software with all the previously analysed components together in a real quadcopter during flight. To achieve this, first, the base vehicle will be constructed using the chosen development kit. Subsequently, all additional components required for this project, such as the companion computer and camera, will be integrated into the base frame. Once the vehicle can successfully fly with the full payload under remote control, the developed control solutions will be tested.

The initial test will run the hand-control solution to verify that the autopilot can receive flight commands from an offboard computer outside of the simulation. Next, the follow mechanism will be started to confirm that the companion computer can function in flight as well as it did during the simulation tests.

The exact steps that will be executed one after the other to ensure that safety is maintained during the whole process are as follows:
\begin{enumerate}
    \item Assemble the quadcopter with its basic components.
    \item Attach the custom payload.
    \item Conduct a test flight using only the remote control and factory autopilot while monitoring through QGroundControl.
    \item Perform a flight using the custom software from an offboard computer, utilizing the \texttt{test-camera} tool.
    \item Conduct a flight using the \texttt{test-camera} tool from the onboard computer.
    \item Perform a flight using the custom hand-gesture control solution from the offboard computer.
    \item Conduct a flight using the custom follow solution from the onboard computer.
\end{enumerate}


\subsection{Build process}
\label{sec:test-7-builddrone}

The chosen vehicle for this project is the Holybro X500, specifically designed to be compatible with PX4. The PX4 documentation\footnote{\url{https://docs.px4.io/main/en/frames_multicopter/holybro_x500_pixhawk4.html}} provides detailed instructions on how to build the vehicle using its Development Kit. Figure \ref{fig:x500-dev-kit} illustrates all the components required to construct the complete vehicle.

\begin{figure}[H]
  \centering
  \includegraphics[width=.7\textwidth, keepaspectratio]{img/x500-dev-kit.jpg}
  \caption{Development kit for the Holybro X500.}
  \source{Adapted from \citetitle{px4-guide} \cite{px4-guide}.}
  \label{fig:x500-dev-kit}
\end{figure}


Once the standard parts are assembled, the custom additions can be integrated into the remaining space within the frame. The Raspberry Pi companion computer will be positioned between the autopilot and GPS antenna. This placement facilitates a convenient connection between the autopilot and the Raspberry Pi's I/O pins using short cables, preventing excessive wire clutter within the frame. 

During flight, the Raspberry Pi will be powered by a dedicated external battery, which supplies power through a 2-ampere USB port. This port will be connected to the Raspberry Pi's original power cable, utilizing its USB-C power supply socket. As explained in Section \ref{sec:test-5-rpi}, this power supply is sufficient to operate the connected camera and run the developed software with satisfactory performance. The battery will be positioned beneath the autopilot, as depicted in Figure \ref{fig:full-build}. 


To ensure the camera is securely mounted on the vehicle's frame, the custom-built support system described in Section \ref{subsec:onboard} will be utilized. The camera holder will be attached to the slide bars beneath the main frame, positioning the camera's weight as close to the centre of mass as possible behind the GPS platform. The main battery, responsible for powering the engines and autopilot, and located on the underside of the carbon frame, can be shifted along the forward axis to balance the added weight from the companion computer, its battery and the camera so that the centre of gravity falls approximately in the centre of the vehicle. Figure \ref{fig:camera-holder-closeup} shows the vehicle's underside with the attached camera and the strap for holding the main battery.

\begin{figure}
  \centering
  \includegraphics[width=1\textwidth, keepaspectratio]{img/full-build.jpg}
  \caption{Complete build of the quadcopter with the main components highlighted.}\label{fig:full-build}
\end{figure}

\begin{figure}
  \centering
  \includegraphics[width=0.7\textwidth, keepaspectratio]{img/underside-2.jpg}
  \caption{Underside of the vehicle, with the supports for holding the main battery and the camera in place.}
  \label{fig:camera-holder-closeup}
\end{figure}


% After the vehicle has been built, there are additional installation and calibration steps that must be carried out before it can fly, also contained in the guide mentioned above.
% Any simulation modes previously activated for testing must be deactivated from the Safety section of the vehicle configuration and the \texttt{MAV\_1\_CONFIG} parameter set to \texttt{TELEM2}, as described in section \ref{subsec:offboard}.
% Then all the different sensors present, both embedded on the flight controller board and attached to the outside frame, need to be calibrated for this particular build.
% The QGroundControl \ref{subsec:qgc} ground station application contains a configuration screen with all the calibration tools needed for the vehicle setup, shown in figure \ref{fig:qgc-config}.
% The vehicle can be configured either by connecting the flight controller directly to the computer via the micro-USB port on its side or through a wireless connection by plugging the companion telemetry radio into the computer running QGroundControl.
Once the vehicle construction is complete, additional installation and calibration steps are necessary before it can be flown. These steps, outlined in the aforementioned guide, include the deactivation of any previously activated simulation modes from the Safety section of the vehicle configuration. Furthermore, the \texttt{MAV_1_CONFIG} parameter must be set to \texttt{TELEM2}, as explained in Section \ref{subsec:offboard}. Calibration of all the onboard and externally attached sensors specific to this build is also required. The QGroundControl ground station application (refer to Section \ref{subsec:qgc}) provides a configuration screen with the necessary calibration tools for vehicle setup, as depicted in Figure \ref{fig:qgc-config}. The vehicle configuration can be performed either by directly connecting the flight controller to a computer via the micro-USB port or wirelessly by connecting the companion telemetry radio to the computer running QGroundControl.


\begin{figure}
  \centering
  \makebox[\textwidth][c]{
  \includegraphics[width=0.5\textwidth, keepaspectratio]{img/qgc-config-1.png}
  \includegraphics[width=0.5\textwidth, keepaspectratio]{img/qgc-config-3.png}}\\
  \includegraphics[width=0.5\textwidth, keepaspectratio]{img/qgc-config-2.png}
  \caption{Screenshot from the QGroundControl calibration and setup tools used to configure the vehicle}\label{fig:qgc-config}
\end{figure}


\subsection{Initial tests}
\label{sec:test-8-flight}

% Setup:    flight plan
% Test:     - assisted takeoff, fly with RC
%           - tools/test_camera + record video
% Results:  video of flying, adjust pid set-point

\subsubsection{Baseline flight with factory software}
\label{subsec:fl-test-1}

% Once the vehicle is fully configured, the RC controller and QGroundControl can be used to test assisted takeoff and landing.
% At this point, the drone should be able to maintain stable flight while using autopilot-assisted flight modes like Position Mode, where the roll and pitch sticks control the acceleration over the ground of the vehicle in the forward/backward and left/right directions relative to the heading the vehicle is facing.
% The throttle controls the speed of ascent and descent. 
% With the sticks centred, the vehicle will actively remain locked to a position in 3D space, compensating for wind and other forces.
% This is the safest manual mode to test that the standard autopilot works as expected.
Once the vehicle is fully configured, testing the assisted takeoff and landing can be conducted using the RC controller and QGroundControl. At this stage, the drone should be capable of maintaining stable flight using autopilot-assisted flight modes, such as Position Mode. In Position Mode, the roll and pitch sticks control the vehicle's acceleration over the ground in the forward/backward and left/right directions relative to the vehicle's heading. The throttle stick controls the ascent and descent speed. When the sticks are centered, the vehicle actively maintains its position in 3D space, compensating for wind and external forces. This manual mode serves as a safe means to verify the functionality of the standard autopilot.

% Through QGroundControl it is possible to map the different switches in the RC controller to various autopilot commands.
% For this test, one of the switches with two positions will be mapped to arm/disarm, which controls whether the engines of the quadcopter can start or not. One of the switches with three positions will be mapped to the landing/takeoff/position flight modes, respectively. The main autopilot modes can be tested by switching between the available positions during flight.
% This configuration exhausts all the channels available in the RC controller employed.
% Other flight modes can be set by using the QGroundControl interface directly.
Through QGroundControl, it is possible to map different switches on the RC controller to various autopilot commands. For the test, a two-position switch will be mapped to arm/disarm, controlling the quadcopter's engine startup. A three-position switch will be assigned to the landing/takeoff/position flight modes, respectively. By switching between the available positions during flight, the primary autopilot modes can be tested. It's worth noting that this configuration utilizes all available channels on the RC controller. Additional flight modes can be set directly through the QGroundControl interface.

% To carry out the flight, first, the main battery is connected to the socket in the power module.
% This starts up the autopilot, the GPS antenna, the telemetry radio, and the RC receiver.
% Afterwards, QGroundControl can be started on a computer connected to the second telemetry radio via USB.
% If everything has worked correctly, the ground station application will automatically connect to the vehicle and situate its position on a satellite map.
% Turning on the RC controller will likewise make it connect to the vehicle, as long as it has been paired correctly, as indicated in the guide linked in the first step of the build process.
% Once all the wireless connections have been established, the drone can take off by first switching to the armed state and then switching to the takeoff flight mode.
% While the drone is in the air, switching to the position flight mode will allow direct control through the joysticks in the controller.
To initiate the flight, the main battery is connected to the power module socket. This action powers up the autopilot, GPS antenna, telemetry radio, and RC receiver. Subsequently, QGroundControl is launched on a computer connected to the second telemetry radio via USB. If all connections have been established correctly, the ground station application will automatically connect to the vehicle and display its position on a satellite map. Similarly, turning on the RC controller, provided it has been correctly paired as outlined in the guide from the initial build process, establishes the connection with the vehicle. Once all wireless connections are established, the drone can take off by switching to the armed state, followed by selecting the takeoff flight mode. While the drone is airborne, switching to the position flight mode enables direct control through the joysticks on the controller.


\subsubsection{Offboard computer flight with test tool}
\label{subsec:fl-test-2}

% The second test flight will aim to ensure that the custom software can send takeoff and landing commands through a wireless MAVlink channel from the offboard computer (using the telemetry radio through the developed test tool).
% For this flight, the QGroundControl application cannot be connected to the vehicle since the DroneVisionControl application will block the telemetry radio channel.
% The RC controller will therefore be used as a backup in case anything goes wrong with the software.
% At any moment, the controller can switch flight mode and override the input generated from the DroneVisionControl application, recovering manual control.
The second test flight aims to verify that the custom software can wirelessly transmit takeoff and landing commands from the offboard computer using a MAVlink channel (utilizing the telemetry radio through the developed test tool). During this flight, it is important to note that the QGroundControl application cannot be connected to the vehicle as it will interfere with the telemetry radio channel used by the DroneVisionControl application. Consequently, the RC controller will serve as a backup in case of any issues with the software. The controller can be used to switch flight modes and override inputs from the DroneVisionControl application, providing manual control if necessary.


% Since the DroneVisionControl application can now easily arm the vehicle on its own while sending a takeoff command,
% the two-way switch of the controller will be mapped for all the tests from now on to the command to kill the power to the engines.
% This command could be helpful in edge cases to protect the vehicle or the surrounding area if the autopilot were to destabilize during takeoff and landing or completely lose control over the vehicle.
% Now, once the main battery is connected again to the power module, the test tool is run with the following command for a Windows or a Linux machine, respectively:
% \begin{minted}[breaklines, fontsize=\footnotesize, baselinestretch=1]{bash}
% dronevisioncontrol tools test-camera -r COM<X>:57600
% \end{minted}
% or
% \begin{minted}[breaklines, fontsize=\footnotesize, baselinestretch=1]{bash}
% dronevisioncontrol tools test-camera -r /dev/ttyUSB0:57600
% \end{minted}
To ensure flexibility and safety during the tests, the two-way switch on the controller will be mapped to the command to cut off power to the engines. This command can be valuable in exceptional situations where it is necessary to protect the vehicle or the surrounding area, such as during destabilization or complete loss of control by the autopilot during takeoff, landing, or flight.

To initiate the test, reconnect the main battery to the power module. Then, execute the following command on the offboard computer, depending on whether it is a Windows or Linux machine:

For Windows: dronevisioncontrol tools test-camera -r COM<X>:57600
For Linux: dronevisioncontrol tools test-camera -r /dev/ttyUSB0:57600

% After successfully connecting to the vehicle, the T and L keys in the computer keyboard can be used for takeoff and landing, respectively. The O key can be used to set the autopilot in offboard flight mode to enable it to receive velocity commands.
% Afterwards, the WASD keys can be used to control the forward and sideways velocity of the vehicle and the QE keys to control its yaw velocity.
% Figure \ref{fig:flight-test-cam-offboard} displays the output on the computer's terminal window, where the connection process and the sent velocity commands are shown, and the output on the camera from the offboard computer.
% A video of the entire process can be found \href{https://l-gonz.github.io/tfg-giaa-dronecontrol/videos/flight-test-offboard}{here}\footnote{\url{https://l-gonz.github.io/tfg-giaa-dronecontrol/videos/flight-test-offboard}}.
After successfully establishing the connection with the vehicle, the computer keyboard can be used to control the flight. Pressing the T key triggers takeoff, while the L key initiates the landing process. The O key sets the autopilot to offboard flight mode, enabling it to receive velocity commands. Subsequently, the WASD keys control the forward and sideways velocity of the vehicle, and the QE keys adjust its yaw velocity.

Figure \ref{fig:flight-test-cam-offboard} illustrates the output displayed in the computer's terminal window, showcasing the connection process, the sent velocity commands, and the camera output from the offboard computer. Additionally, a video capturing the entire process can be accessed using the provided link.

\begin{figure}
  \centering
  \includegraphics[width=\textwidth, keepaspectratio]{img/video-field-test-offboard.png}
  \caption{Terminal output from the \texttt{test-camera} tool running on an offboard computer and image of the drone flying in response}
  \label{fig:flight-test-cam-offboard}
\end{figure}

\subsubsection{Onboard computer flight with test tool}
\label{subsec:fl-test-3}

% The third and last test flight in this section will ensure that the custom software can send takeoff and landing commands through a cabled MAVlink channel from the onboard computer,
% as well as ensuring that the onboard camera can obtain a good image of the field of view of the vehicle during flight.
% For that, the same tool will be used as in the last test, 
% but in this instance, it will be run on the Raspberry Pi, and the connection will be established through the wired serial link between this onboard computer and the Pixhawk autopilot board.
% Since the camera connected to the computer sending commands is now looking down on the pilot, it is possible to activate pose detection on the test images received from this onboard camera.
The third and final test flight in this section aims to confirm that the custom software can send takeoff and landing commands through a cabled MAVlink channel from the onboard computer. Additionally, it ensures that the onboard camera can capture a clear image of the vehicle's field of view during flight.

To perform this test, the same tool used in the previous test will be executed on the Raspberry Pi, utilizing the wired serial link between the onboard computer and the Pixhawk autopilot board. As the camera connected to the computer is positioned to look down on the pilot, pose detection can be activated on the received test images from this onboard camera.

% To start the flight test, the main and secondary batteries need to be attached, respectively, to the power module and to the Raspberry Pi.
% After the onboard computer has started, the easiest way to control it is with a remote desktop connection through WiFi, as explained in section \ref{sec:devenv}.
% By this connection, a terminal window can be opened on the desktop, and the following command run:
To initiate the flight test, attach the main battery to the power module and the secondary battery to the Raspberry Pi. Once the onboard computer has started, establish a remote desktop connection via WiFi, following the instructions provided in section \ref{sec:devenv}. Through this connection, open a terminal window on the desktop and execute the following command:
\begin{minted}[breaklines, fontsize=\footnotesize, baselinestretch=1]{bash}
dronevisioncontrol tools test-camera -r /dev/serial0:921600 -p
\end{minted}


% As opposed to the flight using the telemetry radio, in this test, the serial connection runs at a baudrate of 921600, which matches the configured baudrate on the \texttt{TELEM2} port of the Pixhawk board.
% The "-p" option enables pose detection in the output images.
% A video of the entire process can be found \href{https://l-gonz.github.io/tfg-giaa-dronecontrol/videos/flight-test-onboard}{here}\footnote{\url{https://l-gonz.github.io/tfg-giaa-dronecontrol/videos/flight-test-onboard}} and an image extracted form it can be seen in figure \ref{fig:flight-test-cam-onboard}.
Unlike the telemetry radio flight, this test employs a serial connection running at a baud rate of 921600, matching the configured baud rate on the \texttt{TELEM2} port of the Pixhawk board. The "-p" option enables pose detection in the output images.

A video documenting the entire process can be accessed at the following link: Flight Test Onboard Video. Additionally, an image extracted from the video can be seen in Figure \ref{fig:flight-test-cam-onboard}.


\begin{figure}
  \centering
  \includegraphics[width=\textwidth, keepaspectratio]{img/video-field-test-onboard.png}
  \caption{Pose detection algorithm running on images taken during flight}
  \label{fig:flight-test-cam-onboard}
\end{figure}


\subsection{Hand gesture control}
\label{subsec:fl-test-4}

% During basic flight tests, all the connections and individual parts of the software were validated in actual flight.
% Now, it is time to integrate the piloting system with the image recognition results to test the developed vision-based control solutions.
% The first solution to be used in flight will be the hand-gesture guidance system, which runs on an offboard computer with more available processing resources and no dependence on battery-supplied power to work.
% The setup will be identical to the second test flight (\ref{subsec:fl-test-2}) with the telemetry radio as the serial link and the onboard companion computer turned off.
% Once the autopilot board is powered up, the control solution can be started with the following command:
% \begin{minted}[breaklines, fontsize=\footnotesize, baselinestretch=1]{bash}
% dronevisioncontrol hand -s <device>:57600
% \end{minted}
% where <device> is the COM port or TTY device the telemetry radio is attached to, depending on the platform.
During the basic flight tests, all the connections and individual components of the software were validated in actual flight scenarios. Now, the focus shifts to integrating the piloting system with the results of image recognition to test the developed vision-based control solutions.

The first solution to be tested in flight is the hand-gesture guidance system, which runs on an offboard computer with ample processing resources and doesn't rely on battery power. The setup for this test will be the same as the second test flight described in section \ref{subsec:fl-test-2}, utilizing the telemetry radio as the serial link and turning off the onboard companion computer. Once the autopilot board is powered up, initiate the control solution by executing the following command:
dronevisioncontrol hand -s <device>:57600
Replace <device> with the appropriate COM port or TTY device to which the telemetry radio is connected, depending on the platform.


% After the pilot connects, the image from the computer's webcam will appear on the screen with an outline over any detected hand.
% An open palm should be shown to the camera to start controlling the vehicle.
% Then, a closed fist will make the drone take off, and pointing up with the index finger will start the offboard flight mode.
% Afterwards, moving the index finger right or left will make the vehicle mirror the movement, 
% and moving the thumb right or left will make the vehicle move forward and backwards, respectively.
% At any point during the test, an open hand will make the drone hover at its current place, as will losing sight of the controlling hand.
Once the pilot connection is established, the image from the computer's webcam will be displayed on the screen with an outline highlighting any detected hand. To begin controlling the vehicle, show an open palm to the camera. Closing the hand into a fist will initiate takeoff, and pointing up with the index finger will activate the offboard flight mode. Moving the index finger right or left will cause the vehicle to mirror the movement, while moving the thumb right or left will make the vehicle move forward and backward, respectively. At any point during the test, displaying an open hand will cause the drone to hover in its current position. Losing sight of the controlling hand will also trigger this hovering behavior.


% A video of the entire process can be found \href{https://l-gonz.github.io/tfg-giaa-dronecontrol/videos/flight-test-hand}{here}\footnote{\url{https://l-gonz.github.io/tfg-giaa-dronecontrol/videos/flight-test-hand}} and an image extracted form it can be seen in figure \ref{fig:flight-test-hand}.
A video showcasing the entire process can be accessed at the following link: Flight Test Hand Video. Additionally, an image extracted from the video can be seen in Figure \ref{fig:flight-test-hand}.

\begin{figure}
  \centering
  \includegraphics[width=\textwidth, keepaspectratio]{img/video-field-test-hand.png}
  \caption{Image taken during flight controlled by the hand-gesture solution. The vehicle is moving forward.}
  \label{fig:flight-test-hand}
\end{figure}

\subsection{Target detecting, tracking and following}
\label{subsec:fl-test-5}


% Finally, it only remains to test the follow control solution.
% In this section, the companion computer will be running the follow program, and it will be validated whether it can keep track of and follow a moving target during a non-simulated flight.
% The setup will be identical to the third test in section \ref{subsec:fl-test-3}, without needing a wireless telemetry connection.
% The telemetry radio is, therefore, free to be used, for example, to track the vehicle's path through the QGroundControl application on a secondary, offboard computer.
% The control application will be started with the following command:
% \begin{minted}[breaklines, fontsize=\footnotesize, baselinestretch=1]{bash}
% dronevisioncontrol follow -s /dev/serial0:921600
% \end{minted}
% This \href{https://l-gonz.github.io/tfg-giaa-dronecontrol/videos/flight-test-follow}{video}\footnote{\url{https://l-gonz.github.io/tfg-giaa-dronecontrol/videos/flight-test-follow}} shows the process of getting the vehicle to takeoff (T key), activating offboard flight mode (O key), and starting movement tracking of the detected figure.
% Figure \ref{fig:flight-test-follow} shows an image extracted from this video.
To test the follow control solution, the companion computer will run the follow program, aiming to track and follow a moving target during an actual non-simulated flight. The setup for this test will be the same as the third test flight described in section \ref{subsec:fl-test-3}, without the need for a wireless telemetry connection. The telemetry radio can be utilized, for example, to track the vehicle's path using the QGroundControl application on a secondary offboard computer. Initiate the control application by executing the following command:
dronevisioncontrol follow -s /dev/serial0:921600
This video here showcases the process of the vehicle taking off (T key), activating offboard flight mode (O key), and initiating the tracking of a detected figure. Additionally, Figure \ref{fig:flight-test-follow} presents an image extracted from this video.


\begin{figure}
  \centering
  \includegraphics[width=\textwidth, keepaspectratio]{img/video-field-test-follow.png}
  \caption{Terminal and image output of the DroneVisionControl follow solution running on the Raspberry Pi}
  \label{fig:flight-test-follow}
\end{figure}

% The maximum frames per second managed by the program running on the Pi is around 6 FPS when the follow mechanism is engaged and around 8 FPS if it is disabled by switching out of offboard flight mode.
% In practice, this means that the person being tracked by the drone has to move quite slowly for the camera not to lose sight of them before the autopilot can send the command to the vehicle to move to the previously detected position.
% However, for a proof-of-concept scenario, this is an acceptable performance.
During the flight test, the program running on the Raspberry Pi achieves a maximum frame rate of around 6 FPS when the follow mechanism is active, and approximately 8 FPS when it is disabled by switching out of offboard flight mode. In practical terms, this means that the person being tracked by the drone needs to move relatively slowly to ensure that the camera does not lose sight of them before the autopilot can send the command to the vehicle to move to the previously detected position. However, for a proof-of-concept scenario, this performance is acceptable.

% At the end of the program run, the average loop time and average runtime for each of the tasks in the main loop are shown in the terminal.
% From the measures obtained for the test flight carried out, the average frame rate calculates to be 3.58 FPS.
% If these measurements are compared to those analyzed in section \ref{subsec:performance}, as shown in figure \ref{fig:flight-performance}, particularly to the test configuration most closely resembling actual flight with the autopilot board running on HITL mode and the companion computer powered by the secondary battery, it is possible to appreciate how close this configuration matches the behaviour during actual flight and therefore validating it as an appropriate environment to test the performance of this type of algorithms.
At the end of the program execution, the average loop time and average runtime for each task in the main loop will be displayed in the terminal. Based on the measurements obtained during the test flight, the average frame rate calculates to be 3.58 FPS. Comparing these measurements with those analyzed in section \ref{subsec:performance}, particularly with the test configuration that closely resembles actual flight with the autopilot board running on HITL mode and the companion computer powered by the secondary battery (as shown in Figure \ref{fig:flight-performance}), it is evident that this configuration closely matches the behavior during actual flight. This validates it as a suitable environment for testing the performance of such algorithms.

Overall, this test flight verifies the effectiveness of the follow control solution and its ability to track and follow a moving target during a real flight scenario.


\begin{figure}
  \centering
  \includegraphics[width=\textwidth, keepaspectratio]{img/perf-hitl-flight.png}
  \caption{Terminal and image output of the DroneVisionControl follow solution running on the Raspberry Pi}
  \label{fig:flight-performance}
\end{figure}

