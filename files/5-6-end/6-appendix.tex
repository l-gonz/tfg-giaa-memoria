\chapter{Installation manuals}

This appendix 
\url{https://github.com/l-gonz/tfg-giaa-dronecontrol}

\section{SITL: Development environment}
\label{app:install-dev-env}

This section describes the process of installing all the necessary applications to set up a development environment to run PX4's SITL simulation in a system similar to the one described in Section \ref{sec:devenv}.
The instructions assume a computer running Windows 10/11 as the operating system and Windows Subsystem for Linux installed.
PX4 details the installation steps of their source code for several platforms in their documentation \cite{install-px4}, where Ubuntu is the recommended platform.
To make it easier, the DroneVisionControl repository contains a small shell script that aggregates all the steps and installs all the dependencies with the folder structure that the project expects.
This includes installing and setting up PX4 and QGroundControl and creating a virtual environment for the project, installing the Python packages and the DroneVisionControl application.

To run the script, simply clone the repository, navigate to the project folder and execute:
\begin{minted}[breaklines, fontsize=\footnotesize, baselinestretch=1]{bash}
./install.sh
\end{minted}

To test the installation of PX4, execute the following line (requires a graphic interface for WSL):
\begin{minted}[breaklines, fontsize=\footnotesize, baselinestretch=1]{bash}
./simulator.sh --gazebo
\end{minted}

To test the installation of DroneVisionControl, execute:
\begin{minted}[breaklines, fontsize=\footnotesize, baselinestretch=1]{bash}
dronevisioncontrol tools test-camera -s -c
\end{minted}
Camera input is not supported from within WSL, but it should be possible to control the simulated drone with the keyboard.

The next step is to install DroneVisionControl in the Windows machine to be able to access an integrated or USB camera.
It requires having Python already installed.
First, clone the project repository, navigate to the folder and set up the virtual environment:
\begin{minted}[breaklines, fontsize=\footnotesize, baselinestretch=1]{bash}
pip install virtualenv
virtualenv venv
venv\Scripts\activate
\end{minted}

Then install DroneVisionControl:
\begin{minted}[breaklines, fontsize=\footnotesize, baselinestretch=1]{bash}
pip install -r requirements.txt
pip install -e .
\end{minted}

Additionally, to make the simulated PX4 application broadcast to the Windows machine on port 14550, it is necessary to edit the file \texttt{px4-rc.mavlink} located in \texttt{Firmware/PX4-Autopilot/etc/init.d-posix} on the project folder in WSL.
To the mavlink start command for the ground control link on line 14, append -p to enable broadcasting:
\begin{minted}[breaklines, fontsize=\footnotesize, baselinestretch=1]{bash}
mavlink start -x -u $udp_gcs_port_local -r 4000000 -f
\end{minted}

To test the installation, with the PX4 simulator still running in WSL, execute:
\begin{minted}[breaklines, fontsize=\footnotesize, baselinestretch=1]{bash}
dronevisioncontrol tools test-camera -s -c
\end{minted}
It should now be possible to both control the drone with the keyboard and obtain images from a camera attached to the computer, as well as run the hand control solution in the simulator.

\subsection{Installation of AirSim}
\label{app:install-airsim}

To run the PX4 software-in-the-loop simulator in AirSim and test the follow solution,
first, install Unreal Engine at version 4.27 at least from the Epic Games Launcher\footnote{\url{https://www.unrealengine.com/download}} and open the environment found in the \texttt{data} folder of the repository.
To use AirSim with a different Unreal environment, follow the guide in the AirSim documentation\footnote{\url{https://microsoft.github.io/AirSim/unreal_custenv/}}
After starting play mode in Unreal for the first time, a \texttt{settings.json} file will appear in an AirSim folder in the user's Documents folder.
The contents of this file need to be replaced with the configuration file found in Appendix \ref{app:airsim-config} to be able to interact with PX4 running inside WSL, selecting the correct value for \texttt{UseSerial} and exchanging the \texttt{LocalHostIp} in the file for the IP of the Windows machine in the virtual WSL network.
This IP can be obtained by typing \texttt{ipconfig} in the Windows command prompt and looking for the IPv4 address under "Ethernet Adapter vEthernet (WSL)".

After the settings have been set, restart play mode in Unreal; the output log should show a message saying "Waiting for TCP connection on port 4560, local IP <Windows-IP>".
It is now possible to build and start PX4 by executing in WSL:
\begin{minted}[breaklines, fontsize=\footnotesize, baselinestretch=1]{bash}
./simulator.sh --airsim
\end{minted}

If the IP has been set correctly in the AirSim settings, 
PX4 and the simulator will find each other correctly and connect.
The test-camera tool can be run either from Linux within WSL or from Windows by executing:
\begin{minted}[breaklines, fontsize=\footnotesize, baselinestretch=1]{bash}
dronevisioncontrol tools test-camera -s -w
\end{minted}

The follow control solution can be run from Linux with:
\begin{minted}[breaklines, fontsize=\footnotesize, baselinestretch=1]{bash}
dronevisioncontrol follow --sim 172.19.112.1
\end{minted}
and from Windows with:
\begin{minted}[breaklines, fontsize=\footnotesize, baselinestretch=1]{bash}
dronevisioncontrol follow --sim -p 14550
\end{minted}

\section{HITL: Installation on a Raspberry Pi 4}
\label{app:install-hitl}

For the Raspberry Pi, the operating system of choice for the tests carried out in this project is the native Raspberry OS, but the configuration and results will be similar if using any other Linux distribution. The main requirement for the OS is that it counts with a graphical interface so that the output of the DroneVisionControl application can be visualised.

The installation of the DroneVisionControl application is done the same way as in the WSL system. The most significant difference with the SITL configuration comes from the setup of the Pixhawk board to run the hardware-in-the-loop simulation. The changes in configuration are done through the QGroundControl application, which can be connected to the board through a micro-USB cable or wirelessly through the telemetry radios. The steps to enable the HITL simulation are as follows:
\begin{enumerate}
    \item Configure the airframe for simulation through the Vehicle Setup screen (Vehicle Setup -> Airframe -> HIL Quadcopter X).
    \item Enable HITL simulation in the Safety settings (Vehicle Setup -> Safety -> HITL enabled).
    \item Enable a new MAVLink channel on the secondary telemetry port (Vehicle Setup -> Parameters -> MAV\_1\_CONFIG.
\end{enumerate}

Some additional steps need to be taken to enable a UART serial connection through the RX/TX pins on the GPIO header of the Raspberry Pi. A description of this process can be found in \cite{rpi-uart-setup}.
Other helpful documentation for setting up XRDP on the Raspberry Pi to start a remote desktop from a standalone PC can be found on \cite{install-xrdp}.


\section{AirSim configuration file}
\label{app:airsim-config}

Listing \ref{lst:airsim-settings} provides the configuration file used for AirSim. The file includes various settings and parameters tailored to the specific requirements of the project.

The \texttt{PX4} section contains the configuration settings for the PX4-type flight controller. It tells the simulator to connect to the specific external flight stack. The \texttt{LockStep} parameter is set to true to indicate synchronized step execution between the simulator and the flight controller and the rest of the connection settings replicate the network configuration described in Figure \ref{fig:px4-ports}.

The \texttt{Parameters} section includes specific PX4 parameters relevant only for simulation, such as disabling safety checks for RC control and the start world location of the vehicle.
These parameters can be customized based on the desired behaviour and performance of the flight controller.

The only settings that need to be edited depending on the individual machine and current simulation mode are the \texttt{LocalHostIp} and \texttt{UseSerial}.
The \texttt{LocalHostIp} should be replaced with the IP of the Windows host for the WSL virtual network interface to allow communication with the Linux subsystem. This is only necessary for SITL mode.
The \texttt{UseSerial} parameter determines the communication channel with the flight controller and is set to either \texttt{true} for HITL simulation mode through USB or \texttt{false} for SITL simulation mode to use UDP instead.

The \texttt{CameraDefaults} section defines the settings for the camera images that can be retrieved through the \texttt{airlib} library from external code. 

\begin{listing}[h]
    \caption{AirSim's \texttt{settings.json} file, located in the computer's Documents folder, with settings required for configuring this project.}
    \label{lst:airsim-settings}
    \begin{minted}[breaklines, fontsize=\footnotesize, baselinestretch=1]{json}
{
    "SettingsVersion": 1.2,
    "SimMode": "Multirotor",
    "ClockType": "SteppableClock",
    "Vehicles": {
        "PX4": {
            "VehicleType": "PX4Multirotor",
            "LockStep": true,
            "UseSerial": "<SITL: false, HITL: true>"
            "UseTcp": true,
            "TcpPort": 4560,
            "ControlIp": "remote",
            "ControlPortLocal": 14550,
            "ControlPortRemote": 18570,
            "LocalHostIp": "<Windows-IP>",
            "Sensors":{
                "Barometer":{
                    "SensorType": 1,
                    "Enabled": true
                }
            },
            "Parameters": {
                "NAV_RCL_ACT": 1,
                "NAV_DLL_ACT": 0,
                "COM_RCL_EXCEPT": 7,
                "LPE_LAT": 47.641468,
                "LPE_LON": -122.140165
            }
        }
    },
    "CameraDefaults": {
        "CaptureSettings": [
            {
                "ImageType": 0,
                "Width": 640,
                "Height": 400
            }
        ]
    }
}
    \end{minted}
\end{listing}

\chapter{Command-line interface of the application}
\label{app:cli}

\begin{minted}[breaklines, fontsize=\footnotesize, baselinestretch=1]{text}
Usage: dronevisioncontrol tools test-camera [OPTIONS]

Options:
  -s, --sim TEXT        attach to a simulator through UDP, optionally provide
                        the IP the simulator listens at
  -r, --hardware TEXT   attach to a hardware drone through serial, optionally
                        provide the address of the device that connects to PX4
  -w, --wsl             expects the program to run on a Linux WSL OS
  -c, --camera          use a physical camera as source
  -h, --hand-detection  use hand detection for image processing
  -p, --pose-detection  use pose detection for image processing
  -f, --file TEXT       file name to use as video source
  --help                Show this message and exit.


Usage: dronevisioncontrol tools tune [OPTIONS]

Options:
  --yaw / --forward     test the controller yaw or forward movement
  --manual              manual tuning
  -t, --time INTEGER    sample time for each of the values to test
  -p, --kp-values TEXT  values to test for Kp parameter
  -i, --ki-values TEXT  values to test for Ki parameter
  -d, --kd-values TEXT  values to test for Kd parameter
  -h, --help            Show this message and exit.


Usage: dronevisioncontrol hand [OPTIONS]

Options:
  -i, --ip TEXT       pilot IP address, ignored if serial is provided
  -p, --port INTEGER  port for UDP connections
  -s, --serial TEXT   connect to drone system through serial, default device
                      is /dev/ttyUSB0
  -f, --file PATH     file to use as source instead of the camera
  -l, --log           log important info and save video
  -h, --help          Show this message and exit.


Usage: dronevisioncontrol follow [OPTIONS]

Options:
  --ip TEXT          pilot IP address, ignored if serial is provided
  -p, --port TEXT    pilot UDP port, ignored if serial is provided, default is
                     14540
  --sim TEXT         run with AirSim as flight engine, optionally provide ip
                     the sim listens to
  -l, --log          log important info and save video
  -s, --serial TEXT  use serial to connect to PX4 (HITL), optionally provide
                     the address of the serial port
  -h, --help         Show this message and exit.
\end{minted}


\chapter{PID tuning graphs}

\section{Yaw controller}
\label{app:yaw-pid-results}

\begin{figure}[H]
    \begin{minipage}[t]{0.5\linewidth}
        \centering
        \scalebox{0.55}{%% Creator: Matplotlib, PGF backend
%%
%% To include the figure in your LaTeX document, write
%%   \input{<filename>.pgf}
%%
%% Make sure the required packages are loaded in your preamble
%%   \usepackage{pgf}
%%
%% Also ensure that all the required font packages are loaded; for instance,
%% the lmodern package is sometimes necessary when using math font.
%%   \usepackage{lmodern}
%%
%% Figures using additional raster images can only be included by \input if
%% they are in the same directory as the main LaTeX file. For loading figures
%% from other directories you can use the `import` package
%%   \usepackage{import}
%%
%% and then include the figures with
%%   \import{<path to file>}{<filename>.pgf}
%%
%% Matplotlib used the following preamble
%%   \usepackage{fontspec}
%%   \setmainfont{DejaVuSerif.ttf}[Path=\detokenize{/home/lgonz/tfg-aero/tfg-giaa-dronecontrol/venv/lib/python3.8/site-packages/matplotlib/mpl-data/fonts/ttf/}]
%%   \setsansfont{DejaVuSans.ttf}[Path=\detokenize{/home/lgonz/tfg-aero/tfg-giaa-dronecontrol/venv/lib/python3.8/site-packages/matplotlib/mpl-data/fonts/ttf/}]
%%   \setmonofont{DejaVuSansMono.ttf}[Path=\detokenize{/home/lgonz/tfg-aero/tfg-giaa-dronecontrol/venv/lib/python3.8/site-packages/matplotlib/mpl-data/fonts/ttf/}]
%%
\begingroup%
\makeatletter%
\begin{pgfpicture}%
\pgfpathrectangle{\pgfpointorigin}{\pgfqpoint{6.400000in}{4.800000in}}%
\pgfusepath{use as bounding box, clip}%
\begin{pgfscope}%
\pgfsetbuttcap%
\pgfsetmiterjoin%
\definecolor{currentfill}{rgb}{1.000000,1.000000,1.000000}%
\pgfsetfillcolor{currentfill}%
\pgfsetlinewidth{0.000000pt}%
\definecolor{currentstroke}{rgb}{1.000000,1.000000,1.000000}%
\pgfsetstrokecolor{currentstroke}%
\pgfsetdash{}{0pt}%
\pgfpathmoveto{\pgfqpoint{0.000000in}{0.000000in}}%
\pgfpathlineto{\pgfqpoint{6.400000in}{0.000000in}}%
\pgfpathlineto{\pgfqpoint{6.400000in}{4.800000in}}%
\pgfpathlineto{\pgfqpoint{0.000000in}{4.800000in}}%
\pgfpathlineto{\pgfqpoint{0.000000in}{0.000000in}}%
\pgfpathclose%
\pgfusepath{fill}%
\end{pgfscope}%
\begin{pgfscope}%
\pgfsetbuttcap%
\pgfsetmiterjoin%
\definecolor{currentfill}{rgb}{1.000000,1.000000,1.000000}%
\pgfsetfillcolor{currentfill}%
\pgfsetlinewidth{0.000000pt}%
\definecolor{currentstroke}{rgb}{0.000000,0.000000,0.000000}%
\pgfsetstrokecolor{currentstroke}%
\pgfsetstrokeopacity{0.000000}%
\pgfsetdash{}{0pt}%
\pgfpathmoveto{\pgfqpoint{0.800000in}{0.528000in}}%
\pgfpathlineto{\pgfqpoint{5.760000in}{0.528000in}}%
\pgfpathlineto{\pgfqpoint{5.760000in}{4.224000in}}%
\pgfpathlineto{\pgfqpoint{0.800000in}{4.224000in}}%
\pgfpathlineto{\pgfqpoint{0.800000in}{0.528000in}}%
\pgfpathclose%
\pgfusepath{fill}%
\end{pgfscope}%
\begin{pgfscope}%
\pgfpathrectangle{\pgfqpoint{0.800000in}{0.528000in}}{\pgfqpoint{4.960000in}{3.696000in}}%
\pgfusepath{clip}%
\pgfsetrectcap%
\pgfsetroundjoin%
\pgfsetlinewidth{0.803000pt}%
\definecolor{currentstroke}{rgb}{0.690196,0.690196,0.690196}%
\pgfsetstrokecolor{currentstroke}%
\pgfsetdash{}{0pt}%
\pgfpathmoveto{\pgfqpoint{1.025455in}{0.528000in}}%
\pgfpathlineto{\pgfqpoint{1.025455in}{4.224000in}}%
\pgfusepath{stroke}%
\end{pgfscope}%
\begin{pgfscope}%
\pgfsetbuttcap%
\pgfsetroundjoin%
\definecolor{currentfill}{rgb}{0.000000,0.000000,0.000000}%
\pgfsetfillcolor{currentfill}%
\pgfsetlinewidth{0.803000pt}%
\definecolor{currentstroke}{rgb}{0.000000,0.000000,0.000000}%
\pgfsetstrokecolor{currentstroke}%
\pgfsetdash{}{0pt}%
\pgfsys@defobject{currentmarker}{\pgfqpoint{0.000000in}{-0.048611in}}{\pgfqpoint{0.000000in}{0.000000in}}{%
\pgfpathmoveto{\pgfqpoint{0.000000in}{0.000000in}}%
\pgfpathlineto{\pgfqpoint{0.000000in}{-0.048611in}}%
\pgfusepath{stroke,fill}%
}%
\begin{pgfscope}%
\pgfsys@transformshift{1.025455in}{0.528000in}%
\pgfsys@useobject{currentmarker}{}%
\end{pgfscope}%
\end{pgfscope}%
\begin{pgfscope}%
\definecolor{textcolor}{rgb}{0.000000,0.000000,0.000000}%
\pgfsetstrokecolor{textcolor}%
\pgfsetfillcolor{textcolor}%
\pgftext[x=1.025455in,y=0.430778in,,top]{\color{textcolor}\sffamily\fontsize{10.000000}{12.000000}\selectfont 0}%
\end{pgfscope}%
\begin{pgfscope}%
\pgfpathrectangle{\pgfqpoint{0.800000in}{0.528000in}}{\pgfqpoint{4.960000in}{3.696000in}}%
\pgfusepath{clip}%
\pgfsetrectcap%
\pgfsetroundjoin%
\pgfsetlinewidth{0.803000pt}%
\definecolor{currentstroke}{rgb}{0.690196,0.690196,0.690196}%
\pgfsetstrokecolor{currentstroke}%
\pgfsetdash{}{0pt}%
\pgfpathmoveto{\pgfqpoint{1.775888in}{0.528000in}}%
\pgfpathlineto{\pgfqpoint{1.775888in}{4.224000in}}%
\pgfusepath{stroke}%
\end{pgfscope}%
\begin{pgfscope}%
\pgfsetbuttcap%
\pgfsetroundjoin%
\definecolor{currentfill}{rgb}{0.000000,0.000000,0.000000}%
\pgfsetfillcolor{currentfill}%
\pgfsetlinewidth{0.803000pt}%
\definecolor{currentstroke}{rgb}{0.000000,0.000000,0.000000}%
\pgfsetstrokecolor{currentstroke}%
\pgfsetdash{}{0pt}%
\pgfsys@defobject{currentmarker}{\pgfqpoint{0.000000in}{-0.048611in}}{\pgfqpoint{0.000000in}{0.000000in}}{%
\pgfpathmoveto{\pgfqpoint{0.000000in}{0.000000in}}%
\pgfpathlineto{\pgfqpoint{0.000000in}{-0.048611in}}%
\pgfusepath{stroke,fill}%
}%
\begin{pgfscope}%
\pgfsys@transformshift{1.775888in}{0.528000in}%
\pgfsys@useobject{currentmarker}{}%
\end{pgfscope}%
\end{pgfscope}%
\begin{pgfscope}%
\definecolor{textcolor}{rgb}{0.000000,0.000000,0.000000}%
\pgfsetstrokecolor{textcolor}%
\pgfsetfillcolor{textcolor}%
\pgftext[x=1.775888in,y=0.430778in,,top]{\color{textcolor}\sffamily\fontsize{10.000000}{12.000000}\selectfont 5}%
\end{pgfscope}%
\begin{pgfscope}%
\pgfpathrectangle{\pgfqpoint{0.800000in}{0.528000in}}{\pgfqpoint{4.960000in}{3.696000in}}%
\pgfusepath{clip}%
\pgfsetrectcap%
\pgfsetroundjoin%
\pgfsetlinewidth{0.803000pt}%
\definecolor{currentstroke}{rgb}{0.690196,0.690196,0.690196}%
\pgfsetstrokecolor{currentstroke}%
\pgfsetdash{}{0pt}%
\pgfpathmoveto{\pgfqpoint{2.526321in}{0.528000in}}%
\pgfpathlineto{\pgfqpoint{2.526321in}{4.224000in}}%
\pgfusepath{stroke}%
\end{pgfscope}%
\begin{pgfscope}%
\pgfsetbuttcap%
\pgfsetroundjoin%
\definecolor{currentfill}{rgb}{0.000000,0.000000,0.000000}%
\pgfsetfillcolor{currentfill}%
\pgfsetlinewidth{0.803000pt}%
\definecolor{currentstroke}{rgb}{0.000000,0.000000,0.000000}%
\pgfsetstrokecolor{currentstroke}%
\pgfsetdash{}{0pt}%
\pgfsys@defobject{currentmarker}{\pgfqpoint{0.000000in}{-0.048611in}}{\pgfqpoint{0.000000in}{0.000000in}}{%
\pgfpathmoveto{\pgfqpoint{0.000000in}{0.000000in}}%
\pgfpathlineto{\pgfqpoint{0.000000in}{-0.048611in}}%
\pgfusepath{stroke,fill}%
}%
\begin{pgfscope}%
\pgfsys@transformshift{2.526321in}{0.528000in}%
\pgfsys@useobject{currentmarker}{}%
\end{pgfscope}%
\end{pgfscope}%
\begin{pgfscope}%
\definecolor{textcolor}{rgb}{0.000000,0.000000,0.000000}%
\pgfsetstrokecolor{textcolor}%
\pgfsetfillcolor{textcolor}%
\pgftext[x=2.526321in,y=0.430778in,,top]{\color{textcolor}\sffamily\fontsize{10.000000}{12.000000}\selectfont 10}%
\end{pgfscope}%
\begin{pgfscope}%
\pgfpathrectangle{\pgfqpoint{0.800000in}{0.528000in}}{\pgfqpoint{4.960000in}{3.696000in}}%
\pgfusepath{clip}%
\pgfsetrectcap%
\pgfsetroundjoin%
\pgfsetlinewidth{0.803000pt}%
\definecolor{currentstroke}{rgb}{0.690196,0.690196,0.690196}%
\pgfsetstrokecolor{currentstroke}%
\pgfsetdash{}{0pt}%
\pgfpathmoveto{\pgfqpoint{3.276754in}{0.528000in}}%
\pgfpathlineto{\pgfqpoint{3.276754in}{4.224000in}}%
\pgfusepath{stroke}%
\end{pgfscope}%
\begin{pgfscope}%
\pgfsetbuttcap%
\pgfsetroundjoin%
\definecolor{currentfill}{rgb}{0.000000,0.000000,0.000000}%
\pgfsetfillcolor{currentfill}%
\pgfsetlinewidth{0.803000pt}%
\definecolor{currentstroke}{rgb}{0.000000,0.000000,0.000000}%
\pgfsetstrokecolor{currentstroke}%
\pgfsetdash{}{0pt}%
\pgfsys@defobject{currentmarker}{\pgfqpoint{0.000000in}{-0.048611in}}{\pgfqpoint{0.000000in}{0.000000in}}{%
\pgfpathmoveto{\pgfqpoint{0.000000in}{0.000000in}}%
\pgfpathlineto{\pgfqpoint{0.000000in}{-0.048611in}}%
\pgfusepath{stroke,fill}%
}%
\begin{pgfscope}%
\pgfsys@transformshift{3.276754in}{0.528000in}%
\pgfsys@useobject{currentmarker}{}%
\end{pgfscope}%
\end{pgfscope}%
\begin{pgfscope}%
\definecolor{textcolor}{rgb}{0.000000,0.000000,0.000000}%
\pgfsetstrokecolor{textcolor}%
\pgfsetfillcolor{textcolor}%
\pgftext[x=3.276754in,y=0.430778in,,top]{\color{textcolor}\sffamily\fontsize{10.000000}{12.000000}\selectfont 15}%
\end{pgfscope}%
\begin{pgfscope}%
\pgfpathrectangle{\pgfqpoint{0.800000in}{0.528000in}}{\pgfqpoint{4.960000in}{3.696000in}}%
\pgfusepath{clip}%
\pgfsetrectcap%
\pgfsetroundjoin%
\pgfsetlinewidth{0.803000pt}%
\definecolor{currentstroke}{rgb}{0.690196,0.690196,0.690196}%
\pgfsetstrokecolor{currentstroke}%
\pgfsetdash{}{0pt}%
\pgfpathmoveto{\pgfqpoint{4.027187in}{0.528000in}}%
\pgfpathlineto{\pgfqpoint{4.027187in}{4.224000in}}%
\pgfusepath{stroke}%
\end{pgfscope}%
\begin{pgfscope}%
\pgfsetbuttcap%
\pgfsetroundjoin%
\definecolor{currentfill}{rgb}{0.000000,0.000000,0.000000}%
\pgfsetfillcolor{currentfill}%
\pgfsetlinewidth{0.803000pt}%
\definecolor{currentstroke}{rgb}{0.000000,0.000000,0.000000}%
\pgfsetstrokecolor{currentstroke}%
\pgfsetdash{}{0pt}%
\pgfsys@defobject{currentmarker}{\pgfqpoint{0.000000in}{-0.048611in}}{\pgfqpoint{0.000000in}{0.000000in}}{%
\pgfpathmoveto{\pgfqpoint{0.000000in}{0.000000in}}%
\pgfpathlineto{\pgfqpoint{0.000000in}{-0.048611in}}%
\pgfusepath{stroke,fill}%
}%
\begin{pgfscope}%
\pgfsys@transformshift{4.027187in}{0.528000in}%
\pgfsys@useobject{currentmarker}{}%
\end{pgfscope}%
\end{pgfscope}%
\begin{pgfscope}%
\definecolor{textcolor}{rgb}{0.000000,0.000000,0.000000}%
\pgfsetstrokecolor{textcolor}%
\pgfsetfillcolor{textcolor}%
\pgftext[x=4.027187in,y=0.430778in,,top]{\color{textcolor}\sffamily\fontsize{10.000000}{12.000000}\selectfont 20}%
\end{pgfscope}%
\begin{pgfscope}%
\pgfpathrectangle{\pgfqpoint{0.800000in}{0.528000in}}{\pgfqpoint{4.960000in}{3.696000in}}%
\pgfusepath{clip}%
\pgfsetrectcap%
\pgfsetroundjoin%
\pgfsetlinewidth{0.803000pt}%
\definecolor{currentstroke}{rgb}{0.690196,0.690196,0.690196}%
\pgfsetstrokecolor{currentstroke}%
\pgfsetdash{}{0pt}%
\pgfpathmoveto{\pgfqpoint{4.777620in}{0.528000in}}%
\pgfpathlineto{\pgfqpoint{4.777620in}{4.224000in}}%
\pgfusepath{stroke}%
\end{pgfscope}%
\begin{pgfscope}%
\pgfsetbuttcap%
\pgfsetroundjoin%
\definecolor{currentfill}{rgb}{0.000000,0.000000,0.000000}%
\pgfsetfillcolor{currentfill}%
\pgfsetlinewidth{0.803000pt}%
\definecolor{currentstroke}{rgb}{0.000000,0.000000,0.000000}%
\pgfsetstrokecolor{currentstroke}%
\pgfsetdash{}{0pt}%
\pgfsys@defobject{currentmarker}{\pgfqpoint{0.000000in}{-0.048611in}}{\pgfqpoint{0.000000in}{0.000000in}}{%
\pgfpathmoveto{\pgfqpoint{0.000000in}{0.000000in}}%
\pgfpathlineto{\pgfqpoint{0.000000in}{-0.048611in}}%
\pgfusepath{stroke,fill}%
}%
\begin{pgfscope}%
\pgfsys@transformshift{4.777620in}{0.528000in}%
\pgfsys@useobject{currentmarker}{}%
\end{pgfscope}%
\end{pgfscope}%
\begin{pgfscope}%
\definecolor{textcolor}{rgb}{0.000000,0.000000,0.000000}%
\pgfsetstrokecolor{textcolor}%
\pgfsetfillcolor{textcolor}%
\pgftext[x=4.777620in,y=0.430778in,,top]{\color{textcolor}\sffamily\fontsize{10.000000}{12.000000}\selectfont 25}%
\end{pgfscope}%
\begin{pgfscope}%
\pgfpathrectangle{\pgfqpoint{0.800000in}{0.528000in}}{\pgfqpoint{4.960000in}{3.696000in}}%
\pgfusepath{clip}%
\pgfsetrectcap%
\pgfsetroundjoin%
\pgfsetlinewidth{0.803000pt}%
\definecolor{currentstroke}{rgb}{0.690196,0.690196,0.690196}%
\pgfsetstrokecolor{currentstroke}%
\pgfsetdash{}{0pt}%
\pgfpathmoveto{\pgfqpoint{5.528053in}{0.528000in}}%
\pgfpathlineto{\pgfqpoint{5.528053in}{4.224000in}}%
\pgfusepath{stroke}%
\end{pgfscope}%
\begin{pgfscope}%
\pgfsetbuttcap%
\pgfsetroundjoin%
\definecolor{currentfill}{rgb}{0.000000,0.000000,0.000000}%
\pgfsetfillcolor{currentfill}%
\pgfsetlinewidth{0.803000pt}%
\definecolor{currentstroke}{rgb}{0.000000,0.000000,0.000000}%
\pgfsetstrokecolor{currentstroke}%
\pgfsetdash{}{0pt}%
\pgfsys@defobject{currentmarker}{\pgfqpoint{0.000000in}{-0.048611in}}{\pgfqpoint{0.000000in}{0.000000in}}{%
\pgfpathmoveto{\pgfqpoint{0.000000in}{0.000000in}}%
\pgfpathlineto{\pgfqpoint{0.000000in}{-0.048611in}}%
\pgfusepath{stroke,fill}%
}%
\begin{pgfscope}%
\pgfsys@transformshift{5.528053in}{0.528000in}%
\pgfsys@useobject{currentmarker}{}%
\end{pgfscope}%
\end{pgfscope}%
\begin{pgfscope}%
\definecolor{textcolor}{rgb}{0.000000,0.000000,0.000000}%
\pgfsetstrokecolor{textcolor}%
\pgfsetfillcolor{textcolor}%
\pgftext[x=5.528053in,y=0.430778in,,top]{\color{textcolor}\sffamily\fontsize{10.000000}{12.000000}\selectfont 30}%
\end{pgfscope}%
\begin{pgfscope}%
\definecolor{textcolor}{rgb}{0.000000,0.000000,0.000000}%
\pgfsetstrokecolor{textcolor}%
\pgfsetfillcolor{textcolor}%
\pgftext[x=3.280000in,y=0.240809in,,top]{\color{textcolor}\sffamily\fontsize{10.000000}{12.000000}\selectfont time [s]}%
\end{pgfscope}%
\begin{pgfscope}%
\pgfpathrectangle{\pgfqpoint{0.800000in}{0.528000in}}{\pgfqpoint{4.960000in}{3.696000in}}%
\pgfusepath{clip}%
\pgfsetrectcap%
\pgfsetroundjoin%
\pgfsetlinewidth{0.803000pt}%
\definecolor{currentstroke}{rgb}{0.690196,0.690196,0.690196}%
\pgfsetstrokecolor{currentstroke}%
\pgfsetdash{}{0pt}%
\pgfpathmoveto{\pgfqpoint{0.800000in}{0.679390in}}%
\pgfpathlineto{\pgfqpoint{5.760000in}{0.679390in}}%
\pgfusepath{stroke}%
\end{pgfscope}%
\begin{pgfscope}%
\pgfsetbuttcap%
\pgfsetroundjoin%
\definecolor{currentfill}{rgb}{0.000000,0.000000,0.000000}%
\pgfsetfillcolor{currentfill}%
\pgfsetlinewidth{0.803000pt}%
\definecolor{currentstroke}{rgb}{0.000000,0.000000,0.000000}%
\pgfsetstrokecolor{currentstroke}%
\pgfsetdash{}{0pt}%
\pgfsys@defobject{currentmarker}{\pgfqpoint{-0.048611in}{0.000000in}}{\pgfqpoint{-0.000000in}{0.000000in}}{%
\pgfpathmoveto{\pgfqpoint{-0.000000in}{0.000000in}}%
\pgfpathlineto{\pgfqpoint{-0.048611in}{0.000000in}}%
\pgfusepath{stroke,fill}%
}%
\begin{pgfscope}%
\pgfsys@transformshift{0.800000in}{0.679390in}%
\pgfsys@useobject{currentmarker}{}%
\end{pgfscope}%
\end{pgfscope}%
\begin{pgfscope}%
\definecolor{textcolor}{rgb}{0.000000,0.000000,0.000000}%
\pgfsetstrokecolor{textcolor}%
\pgfsetfillcolor{textcolor}%
\pgftext[x=0.197143in, y=0.626628in, left, base]{\color{textcolor}\sffamily\fontsize{10.000000}{12.000000}\selectfont \ensuremath{-}0.150}%
\end{pgfscope}%
\begin{pgfscope}%
\pgfpathrectangle{\pgfqpoint{0.800000in}{0.528000in}}{\pgfqpoint{4.960000in}{3.696000in}}%
\pgfusepath{clip}%
\pgfsetrectcap%
\pgfsetroundjoin%
\pgfsetlinewidth{0.803000pt}%
\definecolor{currentstroke}{rgb}{0.690196,0.690196,0.690196}%
\pgfsetstrokecolor{currentstroke}%
\pgfsetdash{}{0pt}%
\pgfpathmoveto{\pgfqpoint{0.800000in}{1.117491in}}%
\pgfpathlineto{\pgfqpoint{5.760000in}{1.117491in}}%
\pgfusepath{stroke}%
\end{pgfscope}%
\begin{pgfscope}%
\pgfsetbuttcap%
\pgfsetroundjoin%
\definecolor{currentfill}{rgb}{0.000000,0.000000,0.000000}%
\pgfsetfillcolor{currentfill}%
\pgfsetlinewidth{0.803000pt}%
\definecolor{currentstroke}{rgb}{0.000000,0.000000,0.000000}%
\pgfsetstrokecolor{currentstroke}%
\pgfsetdash{}{0pt}%
\pgfsys@defobject{currentmarker}{\pgfqpoint{-0.048611in}{0.000000in}}{\pgfqpoint{-0.000000in}{0.000000in}}{%
\pgfpathmoveto{\pgfqpoint{-0.000000in}{0.000000in}}%
\pgfpathlineto{\pgfqpoint{-0.048611in}{0.000000in}}%
\pgfusepath{stroke,fill}%
}%
\begin{pgfscope}%
\pgfsys@transformshift{0.800000in}{1.117491in}%
\pgfsys@useobject{currentmarker}{}%
\end{pgfscope}%
\end{pgfscope}%
\begin{pgfscope}%
\definecolor{textcolor}{rgb}{0.000000,0.000000,0.000000}%
\pgfsetstrokecolor{textcolor}%
\pgfsetfillcolor{textcolor}%
\pgftext[x=0.197143in, y=1.064730in, left, base]{\color{textcolor}\sffamily\fontsize{10.000000}{12.000000}\selectfont \ensuremath{-}0.125}%
\end{pgfscope}%
\begin{pgfscope}%
\pgfpathrectangle{\pgfqpoint{0.800000in}{0.528000in}}{\pgfqpoint{4.960000in}{3.696000in}}%
\pgfusepath{clip}%
\pgfsetrectcap%
\pgfsetroundjoin%
\pgfsetlinewidth{0.803000pt}%
\definecolor{currentstroke}{rgb}{0.690196,0.690196,0.690196}%
\pgfsetstrokecolor{currentstroke}%
\pgfsetdash{}{0pt}%
\pgfpathmoveto{\pgfqpoint{0.800000in}{1.555593in}}%
\pgfpathlineto{\pgfqpoint{5.760000in}{1.555593in}}%
\pgfusepath{stroke}%
\end{pgfscope}%
\begin{pgfscope}%
\pgfsetbuttcap%
\pgfsetroundjoin%
\definecolor{currentfill}{rgb}{0.000000,0.000000,0.000000}%
\pgfsetfillcolor{currentfill}%
\pgfsetlinewidth{0.803000pt}%
\definecolor{currentstroke}{rgb}{0.000000,0.000000,0.000000}%
\pgfsetstrokecolor{currentstroke}%
\pgfsetdash{}{0pt}%
\pgfsys@defobject{currentmarker}{\pgfqpoint{-0.048611in}{0.000000in}}{\pgfqpoint{-0.000000in}{0.000000in}}{%
\pgfpathmoveto{\pgfqpoint{-0.000000in}{0.000000in}}%
\pgfpathlineto{\pgfqpoint{-0.048611in}{0.000000in}}%
\pgfusepath{stroke,fill}%
}%
\begin{pgfscope}%
\pgfsys@transformshift{0.800000in}{1.555593in}%
\pgfsys@useobject{currentmarker}{}%
\end{pgfscope}%
\end{pgfscope}%
\begin{pgfscope}%
\definecolor{textcolor}{rgb}{0.000000,0.000000,0.000000}%
\pgfsetstrokecolor{textcolor}%
\pgfsetfillcolor{textcolor}%
\pgftext[x=0.197143in, y=1.502831in, left, base]{\color{textcolor}\sffamily\fontsize{10.000000}{12.000000}\selectfont \ensuremath{-}0.100}%
\end{pgfscope}%
\begin{pgfscope}%
\pgfpathrectangle{\pgfqpoint{0.800000in}{0.528000in}}{\pgfqpoint{4.960000in}{3.696000in}}%
\pgfusepath{clip}%
\pgfsetrectcap%
\pgfsetroundjoin%
\pgfsetlinewidth{0.803000pt}%
\definecolor{currentstroke}{rgb}{0.690196,0.690196,0.690196}%
\pgfsetstrokecolor{currentstroke}%
\pgfsetdash{}{0pt}%
\pgfpathmoveto{\pgfqpoint{0.800000in}{1.993694in}}%
\pgfpathlineto{\pgfqpoint{5.760000in}{1.993694in}}%
\pgfusepath{stroke}%
\end{pgfscope}%
\begin{pgfscope}%
\pgfsetbuttcap%
\pgfsetroundjoin%
\definecolor{currentfill}{rgb}{0.000000,0.000000,0.000000}%
\pgfsetfillcolor{currentfill}%
\pgfsetlinewidth{0.803000pt}%
\definecolor{currentstroke}{rgb}{0.000000,0.000000,0.000000}%
\pgfsetstrokecolor{currentstroke}%
\pgfsetdash{}{0pt}%
\pgfsys@defobject{currentmarker}{\pgfqpoint{-0.048611in}{0.000000in}}{\pgfqpoint{-0.000000in}{0.000000in}}{%
\pgfpathmoveto{\pgfqpoint{-0.000000in}{0.000000in}}%
\pgfpathlineto{\pgfqpoint{-0.048611in}{0.000000in}}%
\pgfusepath{stroke,fill}%
}%
\begin{pgfscope}%
\pgfsys@transformshift{0.800000in}{1.993694in}%
\pgfsys@useobject{currentmarker}{}%
\end{pgfscope}%
\end{pgfscope}%
\begin{pgfscope}%
\definecolor{textcolor}{rgb}{0.000000,0.000000,0.000000}%
\pgfsetstrokecolor{textcolor}%
\pgfsetfillcolor{textcolor}%
\pgftext[x=0.197143in, y=1.940933in, left, base]{\color{textcolor}\sffamily\fontsize{10.000000}{12.000000}\selectfont \ensuremath{-}0.075}%
\end{pgfscope}%
\begin{pgfscope}%
\pgfpathrectangle{\pgfqpoint{0.800000in}{0.528000in}}{\pgfqpoint{4.960000in}{3.696000in}}%
\pgfusepath{clip}%
\pgfsetrectcap%
\pgfsetroundjoin%
\pgfsetlinewidth{0.803000pt}%
\definecolor{currentstroke}{rgb}{0.690196,0.690196,0.690196}%
\pgfsetstrokecolor{currentstroke}%
\pgfsetdash{}{0pt}%
\pgfpathmoveto{\pgfqpoint{0.800000in}{2.431796in}}%
\pgfpathlineto{\pgfqpoint{5.760000in}{2.431796in}}%
\pgfusepath{stroke}%
\end{pgfscope}%
\begin{pgfscope}%
\pgfsetbuttcap%
\pgfsetroundjoin%
\definecolor{currentfill}{rgb}{0.000000,0.000000,0.000000}%
\pgfsetfillcolor{currentfill}%
\pgfsetlinewidth{0.803000pt}%
\definecolor{currentstroke}{rgb}{0.000000,0.000000,0.000000}%
\pgfsetstrokecolor{currentstroke}%
\pgfsetdash{}{0pt}%
\pgfsys@defobject{currentmarker}{\pgfqpoint{-0.048611in}{0.000000in}}{\pgfqpoint{-0.000000in}{0.000000in}}{%
\pgfpathmoveto{\pgfqpoint{-0.000000in}{0.000000in}}%
\pgfpathlineto{\pgfqpoint{-0.048611in}{0.000000in}}%
\pgfusepath{stroke,fill}%
}%
\begin{pgfscope}%
\pgfsys@transformshift{0.800000in}{2.431796in}%
\pgfsys@useobject{currentmarker}{}%
\end{pgfscope}%
\end{pgfscope}%
\begin{pgfscope}%
\definecolor{textcolor}{rgb}{0.000000,0.000000,0.000000}%
\pgfsetstrokecolor{textcolor}%
\pgfsetfillcolor{textcolor}%
\pgftext[x=0.197143in, y=2.379034in, left, base]{\color{textcolor}\sffamily\fontsize{10.000000}{12.000000}\selectfont \ensuremath{-}0.050}%
\end{pgfscope}%
\begin{pgfscope}%
\pgfpathrectangle{\pgfqpoint{0.800000in}{0.528000in}}{\pgfqpoint{4.960000in}{3.696000in}}%
\pgfusepath{clip}%
\pgfsetrectcap%
\pgfsetroundjoin%
\pgfsetlinewidth{0.803000pt}%
\definecolor{currentstroke}{rgb}{0.690196,0.690196,0.690196}%
\pgfsetstrokecolor{currentstroke}%
\pgfsetdash{}{0pt}%
\pgfpathmoveto{\pgfqpoint{0.800000in}{2.869897in}}%
\pgfpathlineto{\pgfqpoint{5.760000in}{2.869897in}}%
\pgfusepath{stroke}%
\end{pgfscope}%
\begin{pgfscope}%
\pgfsetbuttcap%
\pgfsetroundjoin%
\definecolor{currentfill}{rgb}{0.000000,0.000000,0.000000}%
\pgfsetfillcolor{currentfill}%
\pgfsetlinewidth{0.803000pt}%
\definecolor{currentstroke}{rgb}{0.000000,0.000000,0.000000}%
\pgfsetstrokecolor{currentstroke}%
\pgfsetdash{}{0pt}%
\pgfsys@defobject{currentmarker}{\pgfqpoint{-0.048611in}{0.000000in}}{\pgfqpoint{-0.000000in}{0.000000in}}{%
\pgfpathmoveto{\pgfqpoint{-0.000000in}{0.000000in}}%
\pgfpathlineto{\pgfqpoint{-0.048611in}{0.000000in}}%
\pgfusepath{stroke,fill}%
}%
\begin{pgfscope}%
\pgfsys@transformshift{0.800000in}{2.869897in}%
\pgfsys@useobject{currentmarker}{}%
\end{pgfscope}%
\end{pgfscope}%
\begin{pgfscope}%
\definecolor{textcolor}{rgb}{0.000000,0.000000,0.000000}%
\pgfsetstrokecolor{textcolor}%
\pgfsetfillcolor{textcolor}%
\pgftext[x=0.197143in, y=2.817136in, left, base]{\color{textcolor}\sffamily\fontsize{10.000000}{12.000000}\selectfont \ensuremath{-}0.025}%
\end{pgfscope}%
\begin{pgfscope}%
\pgfpathrectangle{\pgfqpoint{0.800000in}{0.528000in}}{\pgfqpoint{4.960000in}{3.696000in}}%
\pgfusepath{clip}%
\pgfsetrectcap%
\pgfsetroundjoin%
\pgfsetlinewidth{0.803000pt}%
\definecolor{currentstroke}{rgb}{0.690196,0.690196,0.690196}%
\pgfsetstrokecolor{currentstroke}%
\pgfsetdash{}{0pt}%
\pgfpathmoveto{\pgfqpoint{0.800000in}{3.307999in}}%
\pgfpathlineto{\pgfqpoint{5.760000in}{3.307999in}}%
\pgfusepath{stroke}%
\end{pgfscope}%
\begin{pgfscope}%
\pgfsetbuttcap%
\pgfsetroundjoin%
\definecolor{currentfill}{rgb}{0.000000,0.000000,0.000000}%
\pgfsetfillcolor{currentfill}%
\pgfsetlinewidth{0.803000pt}%
\definecolor{currentstroke}{rgb}{0.000000,0.000000,0.000000}%
\pgfsetstrokecolor{currentstroke}%
\pgfsetdash{}{0pt}%
\pgfsys@defobject{currentmarker}{\pgfqpoint{-0.048611in}{0.000000in}}{\pgfqpoint{-0.000000in}{0.000000in}}{%
\pgfpathmoveto{\pgfqpoint{-0.000000in}{0.000000in}}%
\pgfpathlineto{\pgfqpoint{-0.048611in}{0.000000in}}%
\pgfusepath{stroke,fill}%
}%
\begin{pgfscope}%
\pgfsys@transformshift{0.800000in}{3.307999in}%
\pgfsys@useobject{currentmarker}{}%
\end{pgfscope}%
\end{pgfscope}%
\begin{pgfscope}%
\definecolor{textcolor}{rgb}{0.000000,0.000000,0.000000}%
\pgfsetstrokecolor{textcolor}%
\pgfsetfillcolor{textcolor}%
\pgftext[x=0.305168in, y=3.255237in, left, base]{\color{textcolor}\sffamily\fontsize{10.000000}{12.000000}\selectfont 0.000}%
\end{pgfscope}%
\begin{pgfscope}%
\pgfpathrectangle{\pgfqpoint{0.800000in}{0.528000in}}{\pgfqpoint{4.960000in}{3.696000in}}%
\pgfusepath{clip}%
\pgfsetrectcap%
\pgfsetroundjoin%
\pgfsetlinewidth{0.803000pt}%
\definecolor{currentstroke}{rgb}{0.690196,0.690196,0.690196}%
\pgfsetstrokecolor{currentstroke}%
\pgfsetdash{}{0pt}%
\pgfpathmoveto{\pgfqpoint{0.800000in}{3.746100in}}%
\pgfpathlineto{\pgfqpoint{5.760000in}{3.746100in}}%
\pgfusepath{stroke}%
\end{pgfscope}%
\begin{pgfscope}%
\pgfsetbuttcap%
\pgfsetroundjoin%
\definecolor{currentfill}{rgb}{0.000000,0.000000,0.000000}%
\pgfsetfillcolor{currentfill}%
\pgfsetlinewidth{0.803000pt}%
\definecolor{currentstroke}{rgb}{0.000000,0.000000,0.000000}%
\pgfsetstrokecolor{currentstroke}%
\pgfsetdash{}{0pt}%
\pgfsys@defobject{currentmarker}{\pgfqpoint{-0.048611in}{0.000000in}}{\pgfqpoint{-0.000000in}{0.000000in}}{%
\pgfpathmoveto{\pgfqpoint{-0.000000in}{0.000000in}}%
\pgfpathlineto{\pgfqpoint{-0.048611in}{0.000000in}}%
\pgfusepath{stroke,fill}%
}%
\begin{pgfscope}%
\pgfsys@transformshift{0.800000in}{3.746100in}%
\pgfsys@useobject{currentmarker}{}%
\end{pgfscope}%
\end{pgfscope}%
\begin{pgfscope}%
\definecolor{textcolor}{rgb}{0.000000,0.000000,0.000000}%
\pgfsetstrokecolor{textcolor}%
\pgfsetfillcolor{textcolor}%
\pgftext[x=0.305168in, y=3.693339in, left, base]{\color{textcolor}\sffamily\fontsize{10.000000}{12.000000}\selectfont 0.025}%
\end{pgfscope}%
\begin{pgfscope}%
\pgfpathrectangle{\pgfqpoint{0.800000in}{0.528000in}}{\pgfqpoint{4.960000in}{3.696000in}}%
\pgfusepath{clip}%
\pgfsetrectcap%
\pgfsetroundjoin%
\pgfsetlinewidth{0.803000pt}%
\definecolor{currentstroke}{rgb}{0.690196,0.690196,0.690196}%
\pgfsetstrokecolor{currentstroke}%
\pgfsetdash{}{0pt}%
\pgfpathmoveto{\pgfqpoint{0.800000in}{4.184202in}}%
\pgfpathlineto{\pgfqpoint{5.760000in}{4.184202in}}%
\pgfusepath{stroke}%
\end{pgfscope}%
\begin{pgfscope}%
\pgfsetbuttcap%
\pgfsetroundjoin%
\definecolor{currentfill}{rgb}{0.000000,0.000000,0.000000}%
\pgfsetfillcolor{currentfill}%
\pgfsetlinewidth{0.803000pt}%
\definecolor{currentstroke}{rgb}{0.000000,0.000000,0.000000}%
\pgfsetstrokecolor{currentstroke}%
\pgfsetdash{}{0pt}%
\pgfsys@defobject{currentmarker}{\pgfqpoint{-0.048611in}{0.000000in}}{\pgfqpoint{-0.000000in}{0.000000in}}{%
\pgfpathmoveto{\pgfqpoint{-0.000000in}{0.000000in}}%
\pgfpathlineto{\pgfqpoint{-0.048611in}{0.000000in}}%
\pgfusepath{stroke,fill}%
}%
\begin{pgfscope}%
\pgfsys@transformshift{0.800000in}{4.184202in}%
\pgfsys@useobject{currentmarker}{}%
\end{pgfscope}%
\end{pgfscope}%
\begin{pgfscope}%
\definecolor{textcolor}{rgb}{0.000000,0.000000,0.000000}%
\pgfsetstrokecolor{textcolor}%
\pgfsetfillcolor{textcolor}%
\pgftext[x=0.305168in, y=4.131440in, left, base]{\color{textcolor}\sffamily\fontsize{10.000000}{12.000000}\selectfont 0.050}%
\end{pgfscope}%
\begin{pgfscope}%
\definecolor{textcolor}{rgb}{0.000000,0.000000,0.000000}%
\pgfsetstrokecolor{textcolor}%
\pgfsetfillcolor{textcolor}%
\pgftext[x=0.141587in,y=2.376000in,,bottom,rotate=90.000000]{\color{textcolor}\sffamily\fontsize{10.000000}{12.000000}\selectfont Computed error [-]}%
\end{pgfscope}%
\begin{pgfscope}%
\pgfpathrectangle{\pgfqpoint{0.800000in}{0.528000in}}{\pgfqpoint{4.960000in}{3.696000in}}%
\pgfusepath{clip}%
\pgfsetrectcap%
\pgfsetroundjoin%
\pgfsetlinewidth{1.505625pt}%
\definecolor{currentstroke}{rgb}{0.121569,0.466667,0.705882}%
\pgfsetstrokecolor{currentstroke}%
\pgfsetdash{}{0pt}%
\pgfpathmoveto{\pgfqpoint{1.025455in}{0.696000in}}%
\pgfpathlineto{\pgfqpoint{1.080166in}{0.698391in}}%
\pgfpathlineto{\pgfqpoint{1.134635in}{0.876975in}}%
\pgfpathlineto{\pgfqpoint{1.189307in}{1.151486in}}%
\pgfpathlineto{\pgfqpoint{1.243360in}{1.443642in}}%
\pgfpathlineto{\pgfqpoint{1.298042in}{1.806387in}}%
\pgfpathlineto{\pgfqpoint{1.352583in}{2.102451in}}%
\pgfpathlineto{\pgfqpoint{1.406773in}{2.429054in}}%
\pgfpathlineto{\pgfqpoint{1.460553in}{2.810362in}}%
\pgfpathlineto{\pgfqpoint{1.515040in}{3.062529in}}%
\pgfpathlineto{\pgfqpoint{1.570820in}{3.442273in}}%
\pgfpathlineto{\pgfqpoint{1.625051in}{3.720213in}}%
\pgfpathlineto{\pgfqpoint{1.678843in}{3.915147in}}%
\pgfpathlineto{\pgfqpoint{1.733208in}{3.928056in}}%
\pgfpathlineto{\pgfqpoint{1.787527in}{3.898496in}}%
\pgfpathlineto{\pgfqpoint{1.841593in}{3.882170in}}%
\pgfpathlineto{\pgfqpoint{1.895731in}{3.760923in}}%
\pgfpathlineto{\pgfqpoint{1.950398in}{3.701694in}}%
\pgfpathlineto{\pgfqpoint{2.004622in}{3.603871in}}%
\pgfpathlineto{\pgfqpoint{2.059015in}{3.560168in}}%
\pgfpathlineto{\pgfqpoint{2.113207in}{3.539210in}}%
\pgfpathlineto{\pgfqpoint{2.168203in}{3.526114in}}%
\pgfpathlineto{\pgfqpoint{2.222993in}{3.557756in}}%
\pgfpathlineto{\pgfqpoint{2.276364in}{3.625312in}}%
\pgfpathlineto{\pgfqpoint{2.330589in}{3.610647in}}%
\pgfpathlineto{\pgfqpoint{2.384772in}{3.645792in}}%
\pgfpathlineto{\pgfqpoint{2.438965in}{3.636980in}}%
\pgfpathlineto{\pgfqpoint{2.493235in}{3.625268in}}%
\pgfpathlineto{\pgfqpoint{2.547792in}{3.636236in}}%
\pgfpathlineto{\pgfqpoint{2.602437in}{3.625419in}}%
\pgfpathlineto{\pgfqpoint{2.656286in}{3.593169in}}%
\pgfpathlineto{\pgfqpoint{2.710680in}{3.530796in}}%
\pgfpathlineto{\pgfqpoint{2.765200in}{3.509926in}}%
\pgfpathlineto{\pgfqpoint{2.820957in}{3.522187in}}%
\pgfpathlineto{\pgfqpoint{2.874965in}{3.456245in}}%
\pgfpathlineto{\pgfqpoint{2.928279in}{3.459982in}}%
\pgfpathlineto{\pgfqpoint{2.982366in}{3.465730in}}%
\pgfpathlineto{\pgfqpoint{3.036772in}{3.469439in}}%
\pgfpathlineto{\pgfqpoint{3.091062in}{3.465406in}}%
\pgfpathlineto{\pgfqpoint{3.145509in}{3.488905in}}%
\pgfpathlineto{\pgfqpoint{3.200139in}{3.517322in}}%
\pgfpathlineto{\pgfqpoint{3.254271in}{3.497597in}}%
\pgfpathlineto{\pgfqpoint{3.308336in}{3.464800in}}%
\pgfpathlineto{\pgfqpoint{3.362592in}{3.439347in}}%
\pgfpathlineto{\pgfqpoint{3.416708in}{3.432070in}}%
\pgfpathlineto{\pgfqpoint{3.472125in}{3.424141in}}%
\pgfpathlineto{\pgfqpoint{3.525649in}{3.403347in}}%
\pgfpathlineto{\pgfqpoint{3.579557in}{3.382454in}}%
\pgfpathlineto{\pgfqpoint{3.634116in}{3.369354in}}%
\pgfpathlineto{\pgfqpoint{3.688284in}{3.361037in}}%
\pgfpathlineto{\pgfqpoint{3.742596in}{3.421908in}}%
\pgfpathlineto{\pgfqpoint{3.797042in}{3.457190in}}%
\pgfpathlineto{\pgfqpoint{3.851612in}{3.467298in}}%
\pgfpathlineto{\pgfqpoint{3.905832in}{3.458296in}}%
\pgfpathlineto{\pgfqpoint{3.959908in}{3.397326in}}%
\pgfpathlineto{\pgfqpoint{4.015150in}{3.391723in}}%
\pgfpathlineto{\pgfqpoint{4.071012in}{3.367464in}}%
\pgfpathlineto{\pgfqpoint{4.124062in}{3.378896in}}%
\pgfpathlineto{\pgfqpoint{4.178078in}{3.342275in}}%
\pgfpathlineto{\pgfqpoint{4.232513in}{3.375506in}}%
\pgfpathlineto{\pgfqpoint{4.286824in}{3.396943in}}%
\pgfpathlineto{\pgfqpoint{4.341796in}{3.362728in}}%
\pgfpathlineto{\pgfqpoint{4.395095in}{3.401945in}}%
\pgfpathlineto{\pgfqpoint{4.449434in}{3.387106in}}%
\pgfpathlineto{\pgfqpoint{4.503604in}{3.365612in}}%
\pgfpathlineto{\pgfqpoint{4.557835in}{3.337903in}}%
\pgfpathlineto{\pgfqpoint{4.612134in}{3.333549in}}%
\pgfpathlineto{\pgfqpoint{4.666205in}{3.336541in}}%
\pgfpathlineto{\pgfqpoint{4.721780in}{3.351502in}}%
\pgfpathlineto{\pgfqpoint{4.775177in}{3.358400in}}%
\pgfpathlineto{\pgfqpoint{4.828950in}{3.359744in}}%
\pgfpathlineto{\pgfqpoint{4.884582in}{3.361955in}}%
\pgfpathlineto{\pgfqpoint{4.937200in}{3.359021in}}%
\pgfpathlineto{\pgfqpoint{4.991438in}{3.359623in}}%
\pgfpathlineto{\pgfqpoint{5.045564in}{3.361481in}}%
\pgfpathlineto{\pgfqpoint{5.100256in}{3.359468in}}%
\pgfpathlineto{\pgfqpoint{5.154242in}{3.351477in}}%
\pgfpathlineto{\pgfqpoint{5.208504in}{3.354984in}}%
\pgfpathlineto{\pgfqpoint{5.262395in}{3.352427in}}%
\pgfpathlineto{\pgfqpoint{5.316854in}{3.342353in}}%
\pgfpathlineto{\pgfqpoint{5.372435in}{3.418843in}}%
\pgfpathlineto{\pgfqpoint{5.426190in}{3.396020in}}%
\pgfpathlineto{\pgfqpoint{5.480011in}{3.396946in}}%
\pgfpathlineto{\pgfqpoint{5.533997in}{3.388261in}}%
\pgfusepath{stroke}%
\end{pgfscope}%
\begin{pgfscope}%
\pgfpathrectangle{\pgfqpoint{0.800000in}{0.528000in}}{\pgfqpoint{4.960000in}{3.696000in}}%
\pgfusepath{clip}%
\pgfsetrectcap%
\pgfsetroundjoin%
\pgfsetlinewidth{1.505625pt}%
\definecolor{currentstroke}{rgb}{1.000000,0.498039,0.054902}%
\pgfsetstrokecolor{currentstroke}%
\pgfsetdash{}{0pt}%
\pgfpathmoveto{\pgfqpoint{1.025455in}{0.848360in}}%
\pgfpathlineto{\pgfqpoint{1.079750in}{0.817147in}}%
\pgfpathlineto{\pgfqpoint{1.133888in}{0.953409in}}%
\pgfpathlineto{\pgfqpoint{1.188434in}{1.202285in}}%
\pgfpathlineto{\pgfqpoint{1.242282in}{1.509981in}}%
\pgfpathlineto{\pgfqpoint{1.296663in}{1.852216in}}%
\pgfpathlineto{\pgfqpoint{1.350562in}{2.098417in}}%
\pgfpathlineto{\pgfqpoint{1.405085in}{2.481441in}}%
\pgfpathlineto{\pgfqpoint{1.459266in}{2.757117in}}%
\pgfpathlineto{\pgfqpoint{1.513703in}{3.044728in}}%
\pgfpathlineto{\pgfqpoint{1.568960in}{3.259789in}}%
\pgfpathlineto{\pgfqpoint{1.622471in}{3.353245in}}%
\pgfpathlineto{\pgfqpoint{1.676403in}{3.334492in}}%
\pgfpathlineto{\pgfqpoint{1.730801in}{3.369528in}}%
\pgfpathlineto{\pgfqpoint{1.785347in}{3.348634in}}%
\pgfpathlineto{\pgfqpoint{1.839619in}{3.296363in}}%
\pgfpathlineto{\pgfqpoint{1.893542in}{3.272906in}}%
\pgfpathlineto{\pgfqpoint{1.947895in}{3.243412in}}%
\pgfpathlineto{\pgfqpoint{2.002284in}{3.231715in}}%
\pgfpathlineto{\pgfqpoint{2.056356in}{3.221333in}}%
\pgfpathlineto{\pgfqpoint{2.110971in}{3.227302in}}%
\pgfpathlineto{\pgfqpoint{2.165312in}{3.230502in}}%
\pgfpathlineto{\pgfqpoint{2.220969in}{3.243719in}}%
\pgfpathlineto{\pgfqpoint{2.274336in}{3.283713in}}%
\pgfpathlineto{\pgfqpoint{2.330352in}{3.309460in}}%
\pgfpathlineto{\pgfqpoint{2.382294in}{3.307373in}}%
\pgfpathlineto{\pgfqpoint{2.437054in}{3.333968in}}%
\pgfpathlineto{\pgfqpoint{2.491640in}{3.349542in}}%
\pgfpathlineto{\pgfqpoint{2.546087in}{3.339596in}}%
\pgfpathlineto{\pgfqpoint{2.599862in}{3.338072in}}%
\pgfpathlineto{\pgfqpoint{2.656657in}{3.330999in}}%
\pgfpathlineto{\pgfqpoint{2.708769in}{3.336893in}}%
\pgfpathlineto{\pgfqpoint{2.762909in}{3.331898in}}%
\pgfpathlineto{\pgfqpoint{2.818748in}{3.326871in}}%
\pgfpathlineto{\pgfqpoint{2.872578in}{3.327862in}}%
\pgfpathlineto{\pgfqpoint{2.926187in}{3.318662in}}%
\pgfpathlineto{\pgfqpoint{2.980158in}{3.326445in}}%
\pgfpathlineto{\pgfqpoint{3.034313in}{3.322592in}}%
\pgfpathlineto{\pgfqpoint{3.089249in}{3.332750in}}%
\pgfpathlineto{\pgfqpoint{3.146544in}{3.318591in}}%
\pgfpathlineto{\pgfqpoint{3.197223in}{3.330001in}}%
\pgfpathlineto{\pgfqpoint{3.251288in}{3.300147in}}%
\pgfpathlineto{\pgfqpoint{3.305522in}{3.280880in}}%
\pgfpathlineto{\pgfqpoint{3.361667in}{3.261894in}}%
\pgfpathlineto{\pgfqpoint{3.415360in}{3.254930in}}%
\pgfpathlineto{\pgfqpoint{3.470374in}{3.254750in}}%
\pgfpathlineto{\pgfqpoint{3.523981in}{3.253201in}}%
\pgfpathlineto{\pgfqpoint{3.578327in}{3.254577in}}%
\pgfpathlineto{\pgfqpoint{3.632938in}{3.259591in}}%
\pgfpathlineto{\pgfqpoint{3.686791in}{3.263051in}}%
\pgfpathlineto{\pgfqpoint{3.740763in}{3.287566in}}%
\pgfpathlineto{\pgfqpoint{3.794976in}{3.288369in}}%
\pgfpathlineto{\pgfqpoint{3.849426in}{3.320174in}}%
\pgfpathlineto{\pgfqpoint{3.903699in}{3.337726in}}%
\pgfpathlineto{\pgfqpoint{3.959208in}{3.336495in}}%
\pgfpathlineto{\pgfqpoint{4.013277in}{3.333952in}}%
\pgfpathlineto{\pgfqpoint{4.066807in}{3.331393in}}%
\pgfpathlineto{\pgfqpoint{4.121033in}{3.330851in}}%
\pgfpathlineto{\pgfqpoint{4.175145in}{3.302278in}}%
\pgfpathlineto{\pgfqpoint{4.229366in}{3.345539in}}%
\pgfpathlineto{\pgfqpoint{4.283535in}{3.297484in}}%
\pgfpathlineto{\pgfqpoint{4.337744in}{3.277635in}}%
\pgfpathlineto{\pgfqpoint{4.392696in}{3.271977in}}%
\pgfpathlineto{\pgfqpoint{4.446908in}{3.287843in}}%
\pgfpathlineto{\pgfqpoint{4.501094in}{3.278236in}}%
\pgfpathlineto{\pgfqpoint{4.556994in}{3.277351in}}%
\pgfpathlineto{\pgfqpoint{4.610504in}{3.278063in}}%
\pgfpathlineto{\pgfqpoint{4.664207in}{3.258432in}}%
\pgfpathlineto{\pgfqpoint{4.718082in}{3.257988in}}%
\pgfpathlineto{\pgfqpoint{4.772504in}{3.258024in}}%
\pgfpathlineto{\pgfqpoint{4.826657in}{3.270017in}}%
\pgfpathlineto{\pgfqpoint{4.880848in}{3.270493in}}%
\pgfpathlineto{\pgfqpoint{4.935639in}{3.273849in}}%
\pgfpathlineto{\pgfqpoint{4.989162in}{3.276070in}}%
\pgfpathlineto{\pgfqpoint{5.043500in}{3.296749in}}%
\pgfpathlineto{\pgfqpoint{5.098109in}{3.298890in}}%
\pgfpathlineto{\pgfqpoint{5.152079in}{3.381233in}}%
\pgfpathlineto{\pgfqpoint{5.207951in}{3.349345in}}%
\pgfpathlineto{\pgfqpoint{5.262151in}{3.351376in}}%
\pgfpathlineto{\pgfqpoint{5.315309in}{3.333035in}}%
\pgfpathlineto{\pgfqpoint{5.369464in}{3.331927in}}%
\pgfpathlineto{\pgfqpoint{5.423533in}{3.315827in}}%
\pgfpathlineto{\pgfqpoint{5.477683in}{3.302708in}}%
\pgfpathlineto{\pgfqpoint{5.531876in}{3.302499in}}%
\pgfusepath{stroke}%
\end{pgfscope}%
\begin{pgfscope}%
\pgfpathrectangle{\pgfqpoint{0.800000in}{0.528000in}}{\pgfqpoint{4.960000in}{3.696000in}}%
\pgfusepath{clip}%
\pgfsetrectcap%
\pgfsetroundjoin%
\pgfsetlinewidth{1.505625pt}%
\definecolor{currentstroke}{rgb}{0.172549,0.627451,0.172549}%
\pgfsetstrokecolor{currentstroke}%
\pgfsetdash{}{0pt}%
\pgfpathmoveto{\pgfqpoint{1.025455in}{0.733721in}}%
\pgfpathlineto{\pgfqpoint{1.079234in}{0.723379in}}%
\pgfpathlineto{\pgfqpoint{1.133723in}{0.843562in}}%
\pgfpathlineto{\pgfqpoint{1.187856in}{1.175413in}}%
\pgfpathlineto{\pgfqpoint{1.242214in}{1.426004in}}%
\pgfpathlineto{\pgfqpoint{1.296445in}{1.794711in}}%
\pgfpathlineto{\pgfqpoint{1.351799in}{2.105557in}}%
\pgfpathlineto{\pgfqpoint{1.405264in}{2.433790in}}%
\pgfpathlineto{\pgfqpoint{1.459281in}{2.798002in}}%
\pgfpathlineto{\pgfqpoint{1.513873in}{3.145025in}}%
\pgfpathlineto{\pgfqpoint{1.568033in}{3.389570in}}%
\pgfpathlineto{\pgfqpoint{1.622414in}{3.631318in}}%
\pgfpathlineto{\pgfqpoint{1.676742in}{3.759801in}}%
\pgfpathlineto{\pgfqpoint{1.730796in}{3.799491in}}%
\pgfpathlineto{\pgfqpoint{1.784969in}{3.913238in}}%
\pgfpathlineto{\pgfqpoint{1.839395in}{3.813910in}}%
\pgfpathlineto{\pgfqpoint{1.893423in}{3.769903in}}%
\pgfpathlineto{\pgfqpoint{1.948020in}{3.667669in}}%
\pgfpathlineto{\pgfqpoint{2.004111in}{3.641305in}}%
\pgfpathlineto{\pgfqpoint{2.057931in}{3.599187in}}%
\pgfpathlineto{\pgfqpoint{2.112063in}{3.611703in}}%
\pgfpathlineto{\pgfqpoint{2.165953in}{3.593064in}}%
\pgfpathlineto{\pgfqpoint{2.220218in}{3.570275in}}%
\pgfpathlineto{\pgfqpoint{2.274710in}{3.576231in}}%
\pgfpathlineto{\pgfqpoint{2.328663in}{3.633115in}}%
\pgfpathlineto{\pgfqpoint{2.383755in}{3.646013in}}%
\pgfpathlineto{\pgfqpoint{2.438423in}{3.619240in}}%
\pgfpathlineto{\pgfqpoint{2.492365in}{3.501781in}}%
\pgfpathlineto{\pgfqpoint{2.546597in}{3.484441in}}%
\pgfpathlineto{\pgfqpoint{2.602253in}{3.483472in}}%
\pgfpathlineto{\pgfqpoint{2.656772in}{3.568844in}}%
\pgfpathlineto{\pgfqpoint{2.710354in}{3.560377in}}%
\pgfpathlineto{\pgfqpoint{2.764377in}{3.557460in}}%
\pgfpathlineto{\pgfqpoint{2.818541in}{3.555234in}}%
\pgfpathlineto{\pgfqpoint{2.872733in}{3.500652in}}%
\pgfpathlineto{\pgfqpoint{2.927078in}{3.508642in}}%
\pgfpathlineto{\pgfqpoint{2.981209in}{3.519574in}}%
\pgfpathlineto{\pgfqpoint{3.036167in}{3.469879in}}%
\pgfpathlineto{\pgfqpoint{3.090014in}{3.480781in}}%
\pgfpathlineto{\pgfqpoint{3.144250in}{3.455484in}}%
\pgfpathlineto{\pgfqpoint{3.198300in}{3.457944in}}%
\pgfpathlineto{\pgfqpoint{3.253835in}{3.482172in}}%
\pgfpathlineto{\pgfqpoint{3.307738in}{3.492183in}}%
\pgfpathlineto{\pgfqpoint{3.362872in}{3.482710in}}%
\pgfpathlineto{\pgfqpoint{3.415982in}{3.469910in}}%
\pgfpathlineto{\pgfqpoint{3.470306in}{3.465033in}}%
\pgfpathlineto{\pgfqpoint{3.524501in}{3.481442in}}%
\pgfpathlineto{\pgfqpoint{3.578509in}{3.493684in}}%
\pgfpathlineto{\pgfqpoint{3.633194in}{3.482423in}}%
\pgfpathlineto{\pgfqpoint{3.687058in}{3.460184in}}%
\pgfpathlineto{\pgfqpoint{3.741744in}{3.446190in}}%
\pgfpathlineto{\pgfqpoint{3.795463in}{3.435521in}}%
\pgfpathlineto{\pgfqpoint{3.851010in}{3.437282in}}%
\pgfpathlineto{\pgfqpoint{3.904699in}{3.435505in}}%
\pgfpathlineto{\pgfqpoint{3.958547in}{3.429705in}}%
\pgfpathlineto{\pgfqpoint{4.013533in}{3.421477in}}%
\pgfpathlineto{\pgfqpoint{4.067391in}{3.421033in}}%
\pgfpathlineto{\pgfqpoint{4.121392in}{3.407666in}}%
\pgfpathlineto{\pgfqpoint{4.175667in}{3.396156in}}%
\pgfpathlineto{\pgfqpoint{4.230121in}{3.386571in}}%
\pgfpathlineto{\pgfqpoint{4.284602in}{3.431107in}}%
\pgfpathlineto{\pgfqpoint{4.338955in}{3.456179in}}%
\pgfpathlineto{\pgfqpoint{4.393221in}{3.452638in}}%
\pgfpathlineto{\pgfqpoint{4.448144in}{3.406576in}}%
\pgfpathlineto{\pgfqpoint{4.503135in}{3.394685in}}%
\pgfpathlineto{\pgfqpoint{4.556389in}{3.364071in}}%
\pgfpathlineto{\pgfqpoint{4.610159in}{3.335685in}}%
\pgfpathlineto{\pgfqpoint{4.664807in}{3.313062in}}%
\pgfpathlineto{\pgfqpoint{4.719074in}{3.307203in}}%
\pgfpathlineto{\pgfqpoint{4.773451in}{3.303383in}}%
\pgfpathlineto{\pgfqpoint{4.827889in}{3.316411in}}%
\pgfpathlineto{\pgfqpoint{4.882397in}{3.345692in}}%
\pgfpathlineto{\pgfqpoint{4.935989in}{3.417619in}}%
\pgfpathlineto{\pgfqpoint{4.990264in}{3.423747in}}%
\pgfpathlineto{\pgfqpoint{5.046432in}{3.395628in}}%
\pgfpathlineto{\pgfqpoint{5.100510in}{3.324079in}}%
\pgfpathlineto{\pgfqpoint{5.154473in}{3.285936in}}%
\pgfpathlineto{\pgfqpoint{5.208241in}{3.263004in}}%
\pgfpathlineto{\pgfqpoint{5.262422in}{3.272045in}}%
\pgfpathlineto{\pgfqpoint{5.316691in}{3.274112in}}%
\pgfpathlineto{\pgfqpoint{5.371032in}{3.279673in}}%
\pgfpathlineto{\pgfqpoint{5.426445in}{3.392721in}}%
\pgfpathlineto{\pgfqpoint{5.479666in}{3.359100in}}%
\pgfpathlineto{\pgfqpoint{5.533619in}{3.346478in}}%
\pgfusepath{stroke}%
\end{pgfscope}%
\begin{pgfscope}%
\pgfpathrectangle{\pgfqpoint{0.800000in}{0.528000in}}{\pgfqpoint{4.960000in}{3.696000in}}%
\pgfusepath{clip}%
\pgfsetrectcap%
\pgfsetroundjoin%
\pgfsetlinewidth{1.505625pt}%
\definecolor{currentstroke}{rgb}{0.839216,0.152941,0.156863}%
\pgfsetstrokecolor{currentstroke}%
\pgfsetdash{}{0pt}%
\pgfpathmoveto{\pgfqpoint{1.025455in}{0.785085in}}%
\pgfpathlineto{\pgfqpoint{1.079980in}{0.803138in}}%
\pgfpathlineto{\pgfqpoint{1.134597in}{1.041187in}}%
\pgfpathlineto{\pgfqpoint{1.188728in}{1.282262in}}%
\pgfpathlineto{\pgfqpoint{1.242825in}{1.599808in}}%
\pgfpathlineto{\pgfqpoint{1.297292in}{1.958356in}}%
\pgfpathlineto{\pgfqpoint{1.351312in}{2.253295in}}%
\pgfpathlineto{\pgfqpoint{1.405887in}{2.712623in}}%
\pgfpathlineto{\pgfqpoint{1.460285in}{2.957086in}}%
\pgfpathlineto{\pgfqpoint{1.514354in}{3.253804in}}%
\pgfpathlineto{\pgfqpoint{1.569017in}{3.559800in}}%
\pgfpathlineto{\pgfqpoint{1.622677in}{3.869208in}}%
\pgfpathlineto{\pgfqpoint{1.677009in}{4.008462in}}%
\pgfpathlineto{\pgfqpoint{1.732580in}{4.056000in}}%
\pgfpathlineto{\pgfqpoint{1.786694in}{4.054122in}}%
\pgfpathlineto{\pgfqpoint{1.840752in}{4.007907in}}%
\pgfpathlineto{\pgfqpoint{1.894556in}{3.985721in}}%
\pgfpathlineto{\pgfqpoint{1.948848in}{3.942634in}}%
\pgfpathlineto{\pgfqpoint{2.003216in}{3.837438in}}%
\pgfpathlineto{\pgfqpoint{2.057766in}{3.796464in}}%
\pgfpathlineto{\pgfqpoint{2.112133in}{3.768639in}}%
\pgfpathlineto{\pgfqpoint{2.166679in}{3.744308in}}%
\pgfpathlineto{\pgfqpoint{2.220665in}{3.741443in}}%
\pgfpathlineto{\pgfqpoint{2.274773in}{3.708353in}}%
\pgfpathlineto{\pgfqpoint{2.328950in}{3.696624in}}%
\pgfpathlineto{\pgfqpoint{2.384095in}{3.676416in}}%
\pgfpathlineto{\pgfqpoint{2.437575in}{3.606897in}}%
\pgfpathlineto{\pgfqpoint{2.491663in}{3.547733in}}%
\pgfpathlineto{\pgfqpoint{2.545879in}{3.517098in}}%
\pgfpathlineto{\pgfqpoint{2.599970in}{3.495341in}}%
\pgfpathlineto{\pgfqpoint{2.654219in}{3.480041in}}%
\pgfpathlineto{\pgfqpoint{2.709374in}{3.481523in}}%
\pgfpathlineto{\pgfqpoint{2.763997in}{3.478472in}}%
\pgfpathlineto{\pgfqpoint{2.817975in}{3.489609in}}%
\pgfpathlineto{\pgfqpoint{2.871924in}{3.496539in}}%
\pgfpathlineto{\pgfqpoint{2.926113in}{3.501374in}}%
\pgfpathlineto{\pgfqpoint{2.982102in}{3.496322in}}%
\pgfpathlineto{\pgfqpoint{3.035437in}{3.467562in}}%
\pgfpathlineto{\pgfqpoint{3.089546in}{3.463146in}}%
\pgfpathlineto{\pgfqpoint{3.143619in}{3.440068in}}%
\pgfpathlineto{\pgfqpoint{3.197969in}{3.392565in}}%
\pgfpathlineto{\pgfqpoint{3.252444in}{3.417109in}}%
\pgfpathlineto{\pgfqpoint{3.307143in}{3.398025in}}%
\pgfpathlineto{\pgfqpoint{3.361390in}{3.368145in}}%
\pgfpathlineto{\pgfqpoint{3.415978in}{3.367821in}}%
\pgfpathlineto{\pgfqpoint{3.469575in}{3.363243in}}%
\pgfpathlineto{\pgfqpoint{3.524316in}{3.346348in}}%
\pgfpathlineto{\pgfqpoint{3.579516in}{3.332399in}}%
\pgfpathlineto{\pgfqpoint{3.632768in}{3.316365in}}%
\pgfpathlineto{\pgfqpoint{3.686758in}{3.375022in}}%
\pgfpathlineto{\pgfqpoint{3.740714in}{3.381455in}}%
\pgfpathlineto{\pgfqpoint{3.794948in}{3.379084in}}%
\pgfpathlineto{\pgfqpoint{3.849494in}{3.312048in}}%
\pgfpathlineto{\pgfqpoint{3.903841in}{3.315858in}}%
\pgfpathlineto{\pgfqpoint{3.958346in}{3.298692in}}%
\pgfpathlineto{\pgfqpoint{4.012774in}{3.283456in}}%
\pgfpathlineto{\pgfqpoint{4.066848in}{3.272003in}}%
\pgfpathlineto{\pgfqpoint{4.121072in}{3.268081in}}%
\pgfpathlineto{\pgfqpoint{4.175089in}{3.290651in}}%
\pgfpathlineto{\pgfqpoint{4.230259in}{3.293933in}}%
\pgfpathlineto{\pgfqpoint{4.284046in}{3.295714in}}%
\pgfpathlineto{\pgfqpoint{4.338109in}{3.296968in}}%
\pgfpathlineto{\pgfqpoint{4.392484in}{3.297546in}}%
\pgfpathlineto{\pgfqpoint{4.446605in}{3.300513in}}%
\pgfpathlineto{\pgfqpoint{4.500903in}{3.334988in}}%
\pgfpathlineto{\pgfqpoint{4.555341in}{3.341061in}}%
\pgfpathlineto{\pgfqpoint{4.611108in}{3.323025in}}%
\pgfpathlineto{\pgfqpoint{4.665816in}{3.400666in}}%
\pgfpathlineto{\pgfqpoint{4.719430in}{3.331086in}}%
\pgfpathlineto{\pgfqpoint{4.773188in}{3.314078in}}%
\pgfpathlineto{\pgfqpoint{4.827429in}{3.290918in}}%
\pgfpathlineto{\pgfqpoint{4.882588in}{3.274947in}}%
\pgfpathlineto{\pgfqpoint{4.936268in}{3.264781in}}%
\pgfpathlineto{\pgfqpoint{4.990289in}{3.295167in}}%
\pgfpathlineto{\pgfqpoint{5.044253in}{3.312261in}}%
\pgfpathlineto{\pgfqpoint{5.098808in}{3.314268in}}%
\pgfpathlineto{\pgfqpoint{5.152836in}{3.326343in}}%
\pgfpathlineto{\pgfqpoint{5.207623in}{3.325025in}}%
\pgfpathlineto{\pgfqpoint{5.262551in}{3.354667in}}%
\pgfpathlineto{\pgfqpoint{5.316455in}{3.373925in}}%
\pgfpathlineto{\pgfqpoint{5.370761in}{3.323991in}}%
\pgfpathlineto{\pgfqpoint{5.424557in}{3.358786in}}%
\pgfpathlineto{\pgfqpoint{5.480693in}{3.322748in}}%
\pgfpathlineto{\pgfqpoint{5.534545in}{3.264861in}}%
\pgfusepath{stroke}%
\end{pgfscope}%
\begin{pgfscope}%
\pgfpathrectangle{\pgfqpoint{0.800000in}{0.528000in}}{\pgfqpoint{4.960000in}{3.696000in}}%
\pgfusepath{clip}%
\pgfsetrectcap%
\pgfsetroundjoin%
\pgfsetlinewidth{1.505625pt}%
\definecolor{currentstroke}{rgb}{0.580392,0.403922,0.741176}%
\pgfsetstrokecolor{currentstroke}%
\pgfsetdash{}{0pt}%
\pgfpathmoveto{\pgfqpoint{1.025455in}{0.728043in}}%
\pgfpathlineto{\pgfqpoint{1.079612in}{0.756637in}}%
\pgfpathlineto{\pgfqpoint{1.133728in}{1.005969in}}%
\pgfpathlineto{\pgfqpoint{1.188278in}{1.320597in}}%
\pgfpathlineto{\pgfqpoint{1.242625in}{1.616855in}}%
\pgfpathlineto{\pgfqpoint{1.297054in}{1.975188in}}%
\pgfpathlineto{\pgfqpoint{1.350867in}{2.315602in}}%
\pgfpathlineto{\pgfqpoint{1.407778in}{2.634814in}}%
\pgfpathlineto{\pgfqpoint{1.459536in}{2.947831in}}%
\pgfpathlineto{\pgfqpoint{1.516888in}{3.296648in}}%
\pgfpathlineto{\pgfqpoint{1.568114in}{3.607521in}}%
\pgfpathlineto{\pgfqpoint{1.622203in}{3.854465in}}%
\pgfpathlineto{\pgfqpoint{1.676843in}{3.965766in}}%
\pgfpathlineto{\pgfqpoint{1.731122in}{3.944646in}}%
\pgfpathlineto{\pgfqpoint{1.786391in}{3.994522in}}%
\pgfpathlineto{\pgfqpoint{1.839042in}{3.980007in}}%
\pgfpathlineto{\pgfqpoint{1.893272in}{3.860406in}}%
\pgfpathlineto{\pgfqpoint{1.948145in}{3.783219in}}%
\pgfpathlineto{\pgfqpoint{2.002288in}{3.750905in}}%
\pgfpathlineto{\pgfqpoint{2.056741in}{3.711625in}}%
\pgfpathlineto{\pgfqpoint{2.110706in}{3.710596in}}%
\pgfpathlineto{\pgfqpoint{2.166371in}{3.686550in}}%
\pgfpathlineto{\pgfqpoint{2.220878in}{3.640762in}}%
\pgfpathlineto{\pgfqpoint{2.274947in}{3.558584in}}%
\pgfpathlineto{\pgfqpoint{2.328962in}{3.459699in}}%
\pgfpathlineto{\pgfqpoint{2.383201in}{3.420370in}}%
\pgfpathlineto{\pgfqpoint{2.437301in}{3.426489in}}%
\pgfpathlineto{\pgfqpoint{2.491747in}{3.420843in}}%
\pgfpathlineto{\pgfqpoint{2.545961in}{3.404092in}}%
\pgfpathlineto{\pgfqpoint{2.600272in}{3.396559in}}%
\pgfpathlineto{\pgfqpoint{2.656414in}{3.414920in}}%
\pgfpathlineto{\pgfqpoint{2.708666in}{3.390170in}}%
\pgfpathlineto{\pgfqpoint{2.764570in}{3.381512in}}%
\pgfpathlineto{\pgfqpoint{2.818290in}{3.367420in}}%
\pgfpathlineto{\pgfqpoint{2.871706in}{3.355964in}}%
\pgfpathlineto{\pgfqpoint{2.925827in}{3.351019in}}%
\pgfpathlineto{\pgfqpoint{2.980157in}{3.350818in}}%
\pgfpathlineto{\pgfqpoint{3.034053in}{3.353402in}}%
\pgfpathlineto{\pgfqpoint{3.088412in}{3.355562in}}%
\pgfpathlineto{\pgfqpoint{3.142668in}{3.335863in}}%
\pgfpathlineto{\pgfqpoint{3.196718in}{3.339356in}}%
\pgfpathlineto{\pgfqpoint{3.250994in}{3.337448in}}%
\pgfpathlineto{\pgfqpoint{3.305298in}{3.308477in}}%
\pgfpathlineto{\pgfqpoint{3.359939in}{3.295606in}}%
\pgfpathlineto{\pgfqpoint{3.415995in}{3.294579in}}%
\pgfpathlineto{\pgfqpoint{3.469035in}{3.309320in}}%
\pgfpathlineto{\pgfqpoint{3.522587in}{3.310317in}}%
\pgfpathlineto{\pgfqpoint{3.576535in}{3.401026in}}%
\pgfpathlineto{\pgfqpoint{3.630804in}{3.402487in}}%
\pgfpathlineto{\pgfqpoint{3.685400in}{3.332393in}}%
\pgfpathlineto{\pgfqpoint{3.739759in}{3.279781in}}%
\pgfpathlineto{\pgfqpoint{3.794077in}{3.277007in}}%
\pgfpathlineto{\pgfqpoint{3.848019in}{3.271912in}}%
\pgfpathlineto{\pgfqpoint{3.904291in}{3.367568in}}%
\pgfpathlineto{\pgfqpoint{3.958163in}{3.333887in}}%
\pgfpathlineto{\pgfqpoint{4.012002in}{3.380594in}}%
\pgfpathlineto{\pgfqpoint{4.065890in}{3.317143in}}%
\pgfpathlineto{\pgfqpoint{4.120538in}{3.256396in}}%
\pgfpathlineto{\pgfqpoint{4.174575in}{3.328451in}}%
\pgfpathlineto{\pgfqpoint{4.229997in}{3.356941in}}%
\pgfpathlineto{\pgfqpoint{4.284433in}{3.357775in}}%
\pgfpathlineto{\pgfqpoint{4.338741in}{3.281898in}}%
\pgfpathlineto{\pgfqpoint{4.393199in}{3.255670in}}%
\pgfpathlineto{\pgfqpoint{4.447253in}{3.251735in}}%
\pgfpathlineto{\pgfqpoint{4.501498in}{3.284259in}}%
\pgfpathlineto{\pgfqpoint{4.557188in}{3.255663in}}%
\pgfpathlineto{\pgfqpoint{4.610632in}{3.279486in}}%
\pgfpathlineto{\pgfqpoint{4.664969in}{3.287325in}}%
\pgfpathlineto{\pgfqpoint{4.719042in}{3.293143in}}%
\pgfpathlineto{\pgfqpoint{4.773240in}{3.279843in}}%
\pgfpathlineto{\pgfqpoint{4.827408in}{3.274731in}}%
\pgfpathlineto{\pgfqpoint{4.881508in}{3.277366in}}%
\pgfpathlineto{\pgfqpoint{4.936140in}{3.320055in}}%
\pgfpathlineto{\pgfqpoint{4.990022in}{3.340035in}}%
\pgfpathlineto{\pgfqpoint{5.044450in}{3.395594in}}%
\pgfpathlineto{\pgfqpoint{5.098913in}{3.337951in}}%
\pgfpathlineto{\pgfqpoint{5.153034in}{3.366585in}}%
\pgfpathlineto{\pgfqpoint{5.208996in}{3.302245in}}%
\pgfpathlineto{\pgfqpoint{5.262675in}{3.261506in}}%
\pgfpathlineto{\pgfqpoint{5.317141in}{3.274686in}}%
\pgfpathlineto{\pgfqpoint{5.370575in}{3.301391in}}%
\pgfpathlineto{\pgfqpoint{5.424782in}{3.269053in}}%
\pgfpathlineto{\pgfqpoint{5.479037in}{3.271303in}}%
\pgfpathlineto{\pgfqpoint{5.533378in}{3.257356in}}%
\pgfusepath{stroke}%
\end{pgfscope}%
\begin{pgfscope}%
\pgfpathrectangle{\pgfqpoint{0.800000in}{0.528000in}}{\pgfqpoint{4.960000in}{3.696000in}}%
\pgfusepath{clip}%
\pgfsetrectcap%
\pgfsetroundjoin%
\pgfsetlinewidth{1.505625pt}%
\definecolor{currentstroke}{rgb}{0.549020,0.337255,0.294118}%
\pgfsetstrokecolor{currentstroke}%
\pgfsetdash{}{0pt}%
\pgfpathmoveto{\pgfqpoint{1.025455in}{0.833094in}}%
\pgfpathlineto{\pgfqpoint{1.079685in}{0.831490in}}%
\pgfpathlineto{\pgfqpoint{1.133950in}{1.046884in}}%
\pgfpathlineto{\pgfqpoint{1.188380in}{1.262310in}}%
\pgfpathlineto{\pgfqpoint{1.242661in}{1.599784in}}%
\pgfpathlineto{\pgfqpoint{1.296693in}{1.951872in}}%
\pgfpathlineto{\pgfqpoint{1.350861in}{2.285205in}}%
\pgfpathlineto{\pgfqpoint{1.405072in}{2.637044in}}%
\pgfpathlineto{\pgfqpoint{1.459209in}{2.914465in}}%
\pgfpathlineto{\pgfqpoint{1.517935in}{3.219489in}}%
\pgfpathlineto{\pgfqpoint{1.569225in}{3.538654in}}%
\pgfpathlineto{\pgfqpoint{1.623079in}{3.790234in}}%
\pgfpathlineto{\pgfqpoint{1.677664in}{3.871582in}}%
\pgfpathlineto{\pgfqpoint{1.731891in}{3.955387in}}%
\pgfpathlineto{\pgfqpoint{1.785793in}{3.908555in}}%
\pgfpathlineto{\pgfqpoint{1.840216in}{3.817051in}}%
\pgfpathlineto{\pgfqpoint{1.894779in}{3.796564in}}%
\pgfpathlineto{\pgfqpoint{1.948755in}{3.748618in}}%
\pgfpathlineto{\pgfqpoint{2.003100in}{3.709206in}}%
\pgfpathlineto{\pgfqpoint{2.057969in}{3.650897in}}%
\pgfpathlineto{\pgfqpoint{2.112760in}{3.694616in}}%
\pgfpathlineto{\pgfqpoint{2.167550in}{3.638261in}}%
\pgfpathlineto{\pgfqpoint{2.221171in}{3.561024in}}%
\pgfpathlineto{\pgfqpoint{2.275106in}{3.446192in}}%
\pgfpathlineto{\pgfqpoint{2.330373in}{3.411592in}}%
\pgfpathlineto{\pgfqpoint{2.384469in}{3.346450in}}%
\pgfpathlineto{\pgfqpoint{2.438635in}{3.344165in}}%
\pgfpathlineto{\pgfqpoint{2.493008in}{3.340744in}}%
\pgfpathlineto{\pgfqpoint{2.547177in}{3.354927in}}%
\pgfpathlineto{\pgfqpoint{2.601370in}{3.345873in}}%
\pgfpathlineto{\pgfqpoint{2.656184in}{3.325818in}}%
\pgfpathlineto{\pgfqpoint{2.711210in}{3.331559in}}%
\pgfpathlineto{\pgfqpoint{2.766803in}{3.339426in}}%
\pgfpathlineto{\pgfqpoint{2.820458in}{3.365013in}}%
\pgfpathlineto{\pgfqpoint{2.874422in}{3.325936in}}%
\pgfpathlineto{\pgfqpoint{2.928908in}{3.311677in}}%
\pgfpathlineto{\pgfqpoint{2.982855in}{3.301009in}}%
\pgfpathlineto{\pgfqpoint{3.036961in}{3.304835in}}%
\pgfpathlineto{\pgfqpoint{3.091462in}{3.311543in}}%
\pgfpathlineto{\pgfqpoint{3.145671in}{3.283756in}}%
\pgfpathlineto{\pgfqpoint{3.199515in}{3.279957in}}%
\pgfpathlineto{\pgfqpoint{3.253880in}{3.370264in}}%
\pgfpathlineto{\pgfqpoint{3.309202in}{3.330885in}}%
\pgfpathlineto{\pgfqpoint{3.363164in}{3.297540in}}%
\pgfpathlineto{\pgfqpoint{3.418277in}{3.274248in}}%
\pgfpathlineto{\pgfqpoint{3.472086in}{3.310390in}}%
\pgfpathlineto{\pgfqpoint{3.526526in}{3.282291in}}%
\pgfpathlineto{\pgfqpoint{3.581203in}{3.280793in}}%
\pgfpathlineto{\pgfqpoint{3.635095in}{3.270717in}}%
\pgfpathlineto{\pgfqpoint{3.689165in}{3.301876in}}%
\pgfpathlineto{\pgfqpoint{3.743641in}{3.299242in}}%
\pgfpathlineto{\pgfqpoint{3.797741in}{3.360678in}}%
\pgfpathlineto{\pgfqpoint{3.852674in}{3.365804in}}%
\pgfpathlineto{\pgfqpoint{3.906592in}{3.315474in}}%
\pgfpathlineto{\pgfqpoint{3.960588in}{3.247815in}}%
\pgfpathlineto{\pgfqpoint{4.015476in}{3.248384in}}%
\pgfpathlineto{\pgfqpoint{4.070544in}{3.257173in}}%
\pgfpathlineto{\pgfqpoint{4.124128in}{3.267180in}}%
\pgfpathlineto{\pgfqpoint{4.177774in}{3.305933in}}%
\pgfpathlineto{\pgfqpoint{4.231779in}{3.356854in}}%
\pgfpathlineto{\pgfqpoint{4.286009in}{3.353476in}}%
\pgfpathlineto{\pgfqpoint{4.340192in}{3.299919in}}%
\pgfpathlineto{\pgfqpoint{4.394409in}{3.258839in}}%
\pgfpathlineto{\pgfqpoint{4.448987in}{3.263579in}}%
\pgfpathlineto{\pgfqpoint{4.503260in}{3.264716in}}%
\pgfpathlineto{\pgfqpoint{4.557021in}{3.230407in}}%
\pgfpathlineto{\pgfqpoint{4.611398in}{3.302879in}}%
\pgfpathlineto{\pgfqpoint{4.667188in}{3.252898in}}%
\pgfpathlineto{\pgfqpoint{4.720436in}{3.264521in}}%
\pgfpathlineto{\pgfqpoint{4.774201in}{3.353665in}}%
\pgfpathlineto{\pgfqpoint{4.828159in}{3.334133in}}%
\pgfpathlineto{\pgfqpoint{4.882334in}{3.293287in}}%
\pgfpathlineto{\pgfqpoint{4.936546in}{3.279225in}}%
\pgfpathlineto{\pgfqpoint{4.991058in}{3.289737in}}%
\pgfpathlineto{\pgfqpoint{5.045460in}{3.305907in}}%
\pgfpathlineto{\pgfqpoint{5.099913in}{3.318982in}}%
\pgfpathlineto{\pgfqpoint{5.153394in}{3.315390in}}%
\pgfpathlineto{\pgfqpoint{5.207586in}{3.327145in}}%
\pgfpathlineto{\pgfqpoint{5.261878in}{3.402366in}}%
\pgfpathlineto{\pgfqpoint{5.317607in}{3.417075in}}%
\pgfpathlineto{\pgfqpoint{5.371075in}{3.329617in}}%
\pgfpathlineto{\pgfqpoint{5.424720in}{3.277686in}}%
\pgfpathlineto{\pgfqpoint{5.478932in}{3.263273in}}%
\pgfpathlineto{\pgfqpoint{5.533007in}{3.263351in}}%
\pgfusepath{stroke}%
\end{pgfscope}%
\begin{pgfscope}%
\pgfsetrectcap%
\pgfsetmiterjoin%
\pgfsetlinewidth{0.803000pt}%
\definecolor{currentstroke}{rgb}{0.000000,0.000000,0.000000}%
\pgfsetstrokecolor{currentstroke}%
\pgfsetdash{}{0pt}%
\pgfpathmoveto{\pgfqpoint{0.800000in}{0.528000in}}%
\pgfpathlineto{\pgfqpoint{0.800000in}{4.224000in}}%
\pgfusepath{stroke}%
\end{pgfscope}%
\begin{pgfscope}%
\pgfsetrectcap%
\pgfsetmiterjoin%
\pgfsetlinewidth{0.803000pt}%
\definecolor{currentstroke}{rgb}{0.000000,0.000000,0.000000}%
\pgfsetstrokecolor{currentstroke}%
\pgfsetdash{}{0pt}%
\pgfpathmoveto{\pgfqpoint{5.760000in}{0.528000in}}%
\pgfpathlineto{\pgfqpoint{5.760000in}{4.224000in}}%
\pgfusepath{stroke}%
\end{pgfscope}%
\begin{pgfscope}%
\pgfsetrectcap%
\pgfsetmiterjoin%
\pgfsetlinewidth{0.803000pt}%
\definecolor{currentstroke}{rgb}{0.000000,0.000000,0.000000}%
\pgfsetstrokecolor{currentstroke}%
\pgfsetdash{}{0pt}%
\pgfpathmoveto{\pgfqpoint{0.800000in}{0.528000in}}%
\pgfpathlineto{\pgfqpoint{5.760000in}{0.528000in}}%
\pgfusepath{stroke}%
\end{pgfscope}%
\begin{pgfscope}%
\pgfsetrectcap%
\pgfsetmiterjoin%
\pgfsetlinewidth{0.803000pt}%
\definecolor{currentstroke}{rgb}{0.000000,0.000000,0.000000}%
\pgfsetstrokecolor{currentstroke}%
\pgfsetdash{}{0pt}%
\pgfpathmoveto{\pgfqpoint{0.800000in}{4.224000in}}%
\pgfpathlineto{\pgfqpoint{5.760000in}{4.224000in}}%
\pgfusepath{stroke}%
\end{pgfscope}%
\begin{pgfscope}%
\definecolor{textcolor}{rgb}{0.000000,0.000000,0.000000}%
\pgfsetstrokecolor{textcolor}%
\pgfsetfillcolor{textcolor}%
\pgftext[x=3.280000in,y=4.307333in,,base]{\color{textcolor}\sffamily\fontsize{12.000000}{14.400000}\selectfont Yaw controller input}%
\end{pgfscope}%
\begin{pgfscope}%
\pgfsetbuttcap%
\pgfsetmiterjoin%
\definecolor{currentfill}{rgb}{1.000000,1.000000,1.000000}%
\pgfsetfillcolor{currentfill}%
\pgfsetfillopacity{0.800000}%
\pgfsetlinewidth{1.003750pt}%
\definecolor{currentstroke}{rgb}{0.800000,0.800000,0.800000}%
\pgfsetstrokecolor{currentstroke}%
\pgfsetstrokeopacity{0.800000}%
\pgfsetdash{}{0pt}%
\pgfpathmoveto{\pgfqpoint{5.041603in}{0.597444in}}%
\pgfpathlineto{\pgfqpoint{5.662778in}{0.597444in}}%
\pgfpathquadraticcurveto{\pgfqpoint{5.690556in}{0.597444in}}{\pgfqpoint{5.690556in}{0.625222in}}%
\pgfpathlineto{\pgfqpoint{5.690556in}{1.834477in}}%
\pgfpathquadraticcurveto{\pgfqpoint{5.690556in}{1.862254in}}{\pgfqpoint{5.662778in}{1.862254in}}%
\pgfpathlineto{\pgfqpoint{5.041603in}{1.862254in}}%
\pgfpathquadraticcurveto{\pgfqpoint{5.013825in}{1.862254in}}{\pgfqpoint{5.013825in}{1.834477in}}%
\pgfpathlineto{\pgfqpoint{5.013825in}{0.625222in}}%
\pgfpathquadraticcurveto{\pgfqpoint{5.013825in}{0.597444in}}{\pgfqpoint{5.041603in}{0.597444in}}%
\pgfpathlineto{\pgfqpoint{5.041603in}{0.597444in}}%
\pgfpathclose%
\pgfusepath{stroke,fill}%
\end{pgfscope}%
\begin{pgfscope}%
\pgfsetrectcap%
\pgfsetroundjoin%
\pgfsetlinewidth{1.505625pt}%
\definecolor{currentstroke}{rgb}{0.121569,0.466667,0.705882}%
\pgfsetstrokecolor{currentstroke}%
\pgfsetdash{}{0pt}%
\pgfpathmoveto{\pgfqpoint{5.069380in}{1.749787in}}%
\pgfpathlineto{\pgfqpoint{5.208269in}{1.749787in}}%
\pgfpathlineto{\pgfqpoint{5.347158in}{1.749787in}}%
\pgfusepath{stroke}%
\end{pgfscope}%
\begin{pgfscope}%
\definecolor{textcolor}{rgb}{0.000000,0.000000,0.000000}%
\pgfsetstrokecolor{textcolor}%
\pgfsetfillcolor{textcolor}%
\pgftext[x=5.458269in,y=1.701176in,left,base]{\color{textcolor}\sffamily\fontsize{10.000000}{12.000000}\selectfont 0}%
\end{pgfscope}%
\begin{pgfscope}%
\pgfsetrectcap%
\pgfsetroundjoin%
\pgfsetlinewidth{1.505625pt}%
\definecolor{currentstroke}{rgb}{1.000000,0.498039,0.054902}%
\pgfsetstrokecolor{currentstroke}%
\pgfsetdash{}{0pt}%
\pgfpathmoveto{\pgfqpoint{5.069380in}{1.545930in}}%
\pgfpathlineto{\pgfqpoint{5.208269in}{1.545930in}}%
\pgfpathlineto{\pgfqpoint{5.347158in}{1.545930in}}%
\pgfusepath{stroke}%
\end{pgfscope}%
\begin{pgfscope}%
\definecolor{textcolor}{rgb}{0.000000,0.000000,0.000000}%
\pgfsetstrokecolor{textcolor}%
\pgfsetfillcolor{textcolor}%
\pgftext[x=5.458269in,y=1.497319in,left,base]{\color{textcolor}\sffamily\fontsize{10.000000}{12.000000}\selectfont 10}%
\end{pgfscope}%
\begin{pgfscope}%
\pgfsetrectcap%
\pgfsetroundjoin%
\pgfsetlinewidth{1.505625pt}%
\definecolor{currentstroke}{rgb}{0.172549,0.627451,0.172549}%
\pgfsetstrokecolor{currentstroke}%
\pgfsetdash{}{0pt}%
\pgfpathmoveto{\pgfqpoint{5.069380in}{1.342073in}}%
\pgfpathlineto{\pgfqpoint{5.208269in}{1.342073in}}%
\pgfpathlineto{\pgfqpoint{5.347158in}{1.342073in}}%
\pgfusepath{stroke}%
\end{pgfscope}%
\begin{pgfscope}%
\definecolor{textcolor}{rgb}{0.000000,0.000000,0.000000}%
\pgfsetstrokecolor{textcolor}%
\pgfsetfillcolor{textcolor}%
\pgftext[x=5.458269in,y=1.293461in,left,base]{\color{textcolor}\sffamily\fontsize{10.000000}{12.000000}\selectfont 20}%
\end{pgfscope}%
\begin{pgfscope}%
\pgfsetrectcap%
\pgfsetroundjoin%
\pgfsetlinewidth{1.505625pt}%
\definecolor{currentstroke}{rgb}{0.839216,0.152941,0.156863}%
\pgfsetstrokecolor{currentstroke}%
\pgfsetdash{}{0pt}%
\pgfpathmoveto{\pgfqpoint{5.069380in}{1.138215in}}%
\pgfpathlineto{\pgfqpoint{5.208269in}{1.138215in}}%
\pgfpathlineto{\pgfqpoint{5.347158in}{1.138215in}}%
\pgfusepath{stroke}%
\end{pgfscope}%
\begin{pgfscope}%
\definecolor{textcolor}{rgb}{0.000000,0.000000,0.000000}%
\pgfsetstrokecolor{textcolor}%
\pgfsetfillcolor{textcolor}%
\pgftext[x=5.458269in,y=1.089604in,left,base]{\color{textcolor}\sffamily\fontsize{10.000000}{12.000000}\selectfont 30}%
\end{pgfscope}%
\begin{pgfscope}%
\pgfsetrectcap%
\pgfsetroundjoin%
\pgfsetlinewidth{1.505625pt}%
\definecolor{currentstroke}{rgb}{0.580392,0.403922,0.741176}%
\pgfsetstrokecolor{currentstroke}%
\pgfsetdash{}{0pt}%
\pgfpathmoveto{\pgfqpoint{5.069380in}{0.934358in}}%
\pgfpathlineto{\pgfqpoint{5.208269in}{0.934358in}}%
\pgfpathlineto{\pgfqpoint{5.347158in}{0.934358in}}%
\pgfusepath{stroke}%
\end{pgfscope}%
\begin{pgfscope}%
\definecolor{textcolor}{rgb}{0.000000,0.000000,0.000000}%
\pgfsetstrokecolor{textcolor}%
\pgfsetfillcolor{textcolor}%
\pgftext[x=5.458269in,y=0.885747in,left,base]{\color{textcolor}\sffamily\fontsize{10.000000}{12.000000}\selectfont 40}%
\end{pgfscope}%
\begin{pgfscope}%
\pgfsetrectcap%
\pgfsetroundjoin%
\pgfsetlinewidth{1.505625pt}%
\definecolor{currentstroke}{rgb}{0.549020,0.337255,0.294118}%
\pgfsetstrokecolor{currentstroke}%
\pgfsetdash{}{0pt}%
\pgfpathmoveto{\pgfqpoint{5.069380in}{0.730501in}}%
\pgfpathlineto{\pgfqpoint{5.208269in}{0.730501in}}%
\pgfpathlineto{\pgfqpoint{5.347158in}{0.730501in}}%
\pgfusepath{stroke}%
\end{pgfscope}%
\begin{pgfscope}%
\definecolor{textcolor}{rgb}{0.000000,0.000000,0.000000}%
\pgfsetstrokecolor{textcolor}%
\pgfsetfillcolor{textcolor}%
\pgftext[x=5.458269in,y=0.681890in,left,base]{\color{textcolor}\sffamily\fontsize{10.000000}{12.000000}\selectfont 50}%
\end{pgfscope}%
\end{pgfpicture}%
\makeatother%
\endgroup%
}
    \end{minipage}
    \begin{minipage}[t]{0.5\linewidth}
        \centering
        \scalebox{0.55}{%% Creator: Matplotlib, PGF backend
%%
%% To include the figure in your LaTeX document, write
%%   \input{<filename>.pgf}
%%
%% Make sure the required packages are loaded in your preamble
%%   \usepackage{pgf}
%%
%% Also ensure that all the required font packages are loaded; for instance,
%% the lmodern package is sometimes necessary when using math font.
%%   \usepackage{lmodern}
%%
%% Figures using additional raster images can only be included by \input if
%% they are in the same directory as the main LaTeX file. For loading figures
%% from other directories you can use the `import` package
%%   \usepackage{import}
%%
%% and then include the figures with
%%   \import{<path to file>}{<filename>.pgf}
%%
%% Matplotlib used the following preamble
%%   \usepackage{fontspec}
%%   \setmainfont{DejaVuSerif.ttf}[Path=\detokenize{/home/lgonz/tfg-aero/tfg-giaa-dronecontrol/venv/lib/python3.8/site-packages/matplotlib/mpl-data/fonts/ttf/}]
%%   \setsansfont{DejaVuSans.ttf}[Path=\detokenize{/home/lgonz/tfg-aero/tfg-giaa-dronecontrol/venv/lib/python3.8/site-packages/matplotlib/mpl-data/fonts/ttf/}]
%%   \setmonofont{DejaVuSansMono.ttf}[Path=\detokenize{/home/lgonz/tfg-aero/tfg-giaa-dronecontrol/venv/lib/python3.8/site-packages/matplotlib/mpl-data/fonts/ttf/}]
%%
\begingroup%
\makeatletter%
\begin{pgfpicture}%
\pgfpathrectangle{\pgfpointorigin}{\pgfqpoint{6.400000in}{4.800000in}}%
\pgfusepath{use as bounding box, clip}%
\begin{pgfscope}%
\pgfsetbuttcap%
\pgfsetmiterjoin%
\definecolor{currentfill}{rgb}{1.000000,1.000000,1.000000}%
\pgfsetfillcolor{currentfill}%
\pgfsetlinewidth{0.000000pt}%
\definecolor{currentstroke}{rgb}{1.000000,1.000000,1.000000}%
\pgfsetstrokecolor{currentstroke}%
\pgfsetdash{}{0pt}%
\pgfpathmoveto{\pgfqpoint{0.000000in}{0.000000in}}%
\pgfpathlineto{\pgfqpoint{6.400000in}{0.000000in}}%
\pgfpathlineto{\pgfqpoint{6.400000in}{4.800000in}}%
\pgfpathlineto{\pgfqpoint{0.000000in}{4.800000in}}%
\pgfpathlineto{\pgfqpoint{0.000000in}{0.000000in}}%
\pgfpathclose%
\pgfusepath{fill}%
\end{pgfscope}%
\begin{pgfscope}%
\pgfsetbuttcap%
\pgfsetmiterjoin%
\definecolor{currentfill}{rgb}{1.000000,1.000000,1.000000}%
\pgfsetfillcolor{currentfill}%
\pgfsetlinewidth{0.000000pt}%
\definecolor{currentstroke}{rgb}{0.000000,0.000000,0.000000}%
\pgfsetstrokecolor{currentstroke}%
\pgfsetstrokeopacity{0.000000}%
\pgfsetdash{}{0pt}%
\pgfpathmoveto{\pgfqpoint{0.800000in}{0.528000in}}%
\pgfpathlineto{\pgfqpoint{5.760000in}{0.528000in}}%
\pgfpathlineto{\pgfqpoint{5.760000in}{4.224000in}}%
\pgfpathlineto{\pgfqpoint{0.800000in}{4.224000in}}%
\pgfpathlineto{\pgfqpoint{0.800000in}{0.528000in}}%
\pgfpathclose%
\pgfusepath{fill}%
\end{pgfscope}%
\begin{pgfscope}%
\pgfpathrectangle{\pgfqpoint{0.800000in}{0.528000in}}{\pgfqpoint{4.960000in}{3.696000in}}%
\pgfusepath{clip}%
\pgfsetrectcap%
\pgfsetroundjoin%
\pgfsetlinewidth{0.803000pt}%
\definecolor{currentstroke}{rgb}{0.690196,0.690196,0.690196}%
\pgfsetstrokecolor{currentstroke}%
\pgfsetdash{}{0pt}%
\pgfpathmoveto{\pgfqpoint{1.025455in}{0.528000in}}%
\pgfpathlineto{\pgfqpoint{1.025455in}{4.224000in}}%
\pgfusepath{stroke}%
\end{pgfscope}%
\begin{pgfscope}%
\pgfsetbuttcap%
\pgfsetroundjoin%
\definecolor{currentfill}{rgb}{0.000000,0.000000,0.000000}%
\pgfsetfillcolor{currentfill}%
\pgfsetlinewidth{0.803000pt}%
\definecolor{currentstroke}{rgb}{0.000000,0.000000,0.000000}%
\pgfsetstrokecolor{currentstroke}%
\pgfsetdash{}{0pt}%
\pgfsys@defobject{currentmarker}{\pgfqpoint{0.000000in}{-0.048611in}}{\pgfqpoint{0.000000in}{0.000000in}}{%
\pgfpathmoveto{\pgfqpoint{0.000000in}{0.000000in}}%
\pgfpathlineto{\pgfqpoint{0.000000in}{-0.048611in}}%
\pgfusepath{stroke,fill}%
}%
\begin{pgfscope}%
\pgfsys@transformshift{1.025455in}{0.528000in}%
\pgfsys@useobject{currentmarker}{}%
\end{pgfscope}%
\end{pgfscope}%
\begin{pgfscope}%
\definecolor{textcolor}{rgb}{0.000000,0.000000,0.000000}%
\pgfsetstrokecolor{textcolor}%
\pgfsetfillcolor{textcolor}%
\pgftext[x=1.025455in,y=0.430778in,,top]{\color{textcolor}\sffamily\fontsize{10.000000}{12.000000}\selectfont 0}%
\end{pgfscope}%
\begin{pgfscope}%
\pgfpathrectangle{\pgfqpoint{0.800000in}{0.528000in}}{\pgfqpoint{4.960000in}{3.696000in}}%
\pgfusepath{clip}%
\pgfsetrectcap%
\pgfsetroundjoin%
\pgfsetlinewidth{0.803000pt}%
\definecolor{currentstroke}{rgb}{0.690196,0.690196,0.690196}%
\pgfsetstrokecolor{currentstroke}%
\pgfsetdash{}{0pt}%
\pgfpathmoveto{\pgfqpoint{1.775888in}{0.528000in}}%
\pgfpathlineto{\pgfqpoint{1.775888in}{4.224000in}}%
\pgfusepath{stroke}%
\end{pgfscope}%
\begin{pgfscope}%
\pgfsetbuttcap%
\pgfsetroundjoin%
\definecolor{currentfill}{rgb}{0.000000,0.000000,0.000000}%
\pgfsetfillcolor{currentfill}%
\pgfsetlinewidth{0.803000pt}%
\definecolor{currentstroke}{rgb}{0.000000,0.000000,0.000000}%
\pgfsetstrokecolor{currentstroke}%
\pgfsetdash{}{0pt}%
\pgfsys@defobject{currentmarker}{\pgfqpoint{0.000000in}{-0.048611in}}{\pgfqpoint{0.000000in}{0.000000in}}{%
\pgfpathmoveto{\pgfqpoint{0.000000in}{0.000000in}}%
\pgfpathlineto{\pgfqpoint{0.000000in}{-0.048611in}}%
\pgfusepath{stroke,fill}%
}%
\begin{pgfscope}%
\pgfsys@transformshift{1.775888in}{0.528000in}%
\pgfsys@useobject{currentmarker}{}%
\end{pgfscope}%
\end{pgfscope}%
\begin{pgfscope}%
\definecolor{textcolor}{rgb}{0.000000,0.000000,0.000000}%
\pgfsetstrokecolor{textcolor}%
\pgfsetfillcolor{textcolor}%
\pgftext[x=1.775888in,y=0.430778in,,top]{\color{textcolor}\sffamily\fontsize{10.000000}{12.000000}\selectfont 5}%
\end{pgfscope}%
\begin{pgfscope}%
\pgfpathrectangle{\pgfqpoint{0.800000in}{0.528000in}}{\pgfqpoint{4.960000in}{3.696000in}}%
\pgfusepath{clip}%
\pgfsetrectcap%
\pgfsetroundjoin%
\pgfsetlinewidth{0.803000pt}%
\definecolor{currentstroke}{rgb}{0.690196,0.690196,0.690196}%
\pgfsetstrokecolor{currentstroke}%
\pgfsetdash{}{0pt}%
\pgfpathmoveto{\pgfqpoint{2.526321in}{0.528000in}}%
\pgfpathlineto{\pgfqpoint{2.526321in}{4.224000in}}%
\pgfusepath{stroke}%
\end{pgfscope}%
\begin{pgfscope}%
\pgfsetbuttcap%
\pgfsetroundjoin%
\definecolor{currentfill}{rgb}{0.000000,0.000000,0.000000}%
\pgfsetfillcolor{currentfill}%
\pgfsetlinewidth{0.803000pt}%
\definecolor{currentstroke}{rgb}{0.000000,0.000000,0.000000}%
\pgfsetstrokecolor{currentstroke}%
\pgfsetdash{}{0pt}%
\pgfsys@defobject{currentmarker}{\pgfqpoint{0.000000in}{-0.048611in}}{\pgfqpoint{0.000000in}{0.000000in}}{%
\pgfpathmoveto{\pgfqpoint{0.000000in}{0.000000in}}%
\pgfpathlineto{\pgfqpoint{0.000000in}{-0.048611in}}%
\pgfusepath{stroke,fill}%
}%
\begin{pgfscope}%
\pgfsys@transformshift{2.526321in}{0.528000in}%
\pgfsys@useobject{currentmarker}{}%
\end{pgfscope}%
\end{pgfscope}%
\begin{pgfscope}%
\definecolor{textcolor}{rgb}{0.000000,0.000000,0.000000}%
\pgfsetstrokecolor{textcolor}%
\pgfsetfillcolor{textcolor}%
\pgftext[x=2.526321in,y=0.430778in,,top]{\color{textcolor}\sffamily\fontsize{10.000000}{12.000000}\selectfont 10}%
\end{pgfscope}%
\begin{pgfscope}%
\pgfpathrectangle{\pgfqpoint{0.800000in}{0.528000in}}{\pgfqpoint{4.960000in}{3.696000in}}%
\pgfusepath{clip}%
\pgfsetrectcap%
\pgfsetroundjoin%
\pgfsetlinewidth{0.803000pt}%
\definecolor{currentstroke}{rgb}{0.690196,0.690196,0.690196}%
\pgfsetstrokecolor{currentstroke}%
\pgfsetdash{}{0pt}%
\pgfpathmoveto{\pgfqpoint{3.276754in}{0.528000in}}%
\pgfpathlineto{\pgfqpoint{3.276754in}{4.224000in}}%
\pgfusepath{stroke}%
\end{pgfscope}%
\begin{pgfscope}%
\pgfsetbuttcap%
\pgfsetroundjoin%
\definecolor{currentfill}{rgb}{0.000000,0.000000,0.000000}%
\pgfsetfillcolor{currentfill}%
\pgfsetlinewidth{0.803000pt}%
\definecolor{currentstroke}{rgb}{0.000000,0.000000,0.000000}%
\pgfsetstrokecolor{currentstroke}%
\pgfsetdash{}{0pt}%
\pgfsys@defobject{currentmarker}{\pgfqpoint{0.000000in}{-0.048611in}}{\pgfqpoint{0.000000in}{0.000000in}}{%
\pgfpathmoveto{\pgfqpoint{0.000000in}{0.000000in}}%
\pgfpathlineto{\pgfqpoint{0.000000in}{-0.048611in}}%
\pgfusepath{stroke,fill}%
}%
\begin{pgfscope}%
\pgfsys@transformshift{3.276754in}{0.528000in}%
\pgfsys@useobject{currentmarker}{}%
\end{pgfscope}%
\end{pgfscope}%
\begin{pgfscope}%
\definecolor{textcolor}{rgb}{0.000000,0.000000,0.000000}%
\pgfsetstrokecolor{textcolor}%
\pgfsetfillcolor{textcolor}%
\pgftext[x=3.276754in,y=0.430778in,,top]{\color{textcolor}\sffamily\fontsize{10.000000}{12.000000}\selectfont 15}%
\end{pgfscope}%
\begin{pgfscope}%
\pgfpathrectangle{\pgfqpoint{0.800000in}{0.528000in}}{\pgfqpoint{4.960000in}{3.696000in}}%
\pgfusepath{clip}%
\pgfsetrectcap%
\pgfsetroundjoin%
\pgfsetlinewidth{0.803000pt}%
\definecolor{currentstroke}{rgb}{0.690196,0.690196,0.690196}%
\pgfsetstrokecolor{currentstroke}%
\pgfsetdash{}{0pt}%
\pgfpathmoveto{\pgfqpoint{4.027187in}{0.528000in}}%
\pgfpathlineto{\pgfqpoint{4.027187in}{4.224000in}}%
\pgfusepath{stroke}%
\end{pgfscope}%
\begin{pgfscope}%
\pgfsetbuttcap%
\pgfsetroundjoin%
\definecolor{currentfill}{rgb}{0.000000,0.000000,0.000000}%
\pgfsetfillcolor{currentfill}%
\pgfsetlinewidth{0.803000pt}%
\definecolor{currentstroke}{rgb}{0.000000,0.000000,0.000000}%
\pgfsetstrokecolor{currentstroke}%
\pgfsetdash{}{0pt}%
\pgfsys@defobject{currentmarker}{\pgfqpoint{0.000000in}{-0.048611in}}{\pgfqpoint{0.000000in}{0.000000in}}{%
\pgfpathmoveto{\pgfqpoint{0.000000in}{0.000000in}}%
\pgfpathlineto{\pgfqpoint{0.000000in}{-0.048611in}}%
\pgfusepath{stroke,fill}%
}%
\begin{pgfscope}%
\pgfsys@transformshift{4.027187in}{0.528000in}%
\pgfsys@useobject{currentmarker}{}%
\end{pgfscope}%
\end{pgfscope}%
\begin{pgfscope}%
\definecolor{textcolor}{rgb}{0.000000,0.000000,0.000000}%
\pgfsetstrokecolor{textcolor}%
\pgfsetfillcolor{textcolor}%
\pgftext[x=4.027187in,y=0.430778in,,top]{\color{textcolor}\sffamily\fontsize{10.000000}{12.000000}\selectfont 20}%
\end{pgfscope}%
\begin{pgfscope}%
\pgfpathrectangle{\pgfqpoint{0.800000in}{0.528000in}}{\pgfqpoint{4.960000in}{3.696000in}}%
\pgfusepath{clip}%
\pgfsetrectcap%
\pgfsetroundjoin%
\pgfsetlinewidth{0.803000pt}%
\definecolor{currentstroke}{rgb}{0.690196,0.690196,0.690196}%
\pgfsetstrokecolor{currentstroke}%
\pgfsetdash{}{0pt}%
\pgfpathmoveto{\pgfqpoint{4.777620in}{0.528000in}}%
\pgfpathlineto{\pgfqpoint{4.777620in}{4.224000in}}%
\pgfusepath{stroke}%
\end{pgfscope}%
\begin{pgfscope}%
\pgfsetbuttcap%
\pgfsetroundjoin%
\definecolor{currentfill}{rgb}{0.000000,0.000000,0.000000}%
\pgfsetfillcolor{currentfill}%
\pgfsetlinewidth{0.803000pt}%
\definecolor{currentstroke}{rgb}{0.000000,0.000000,0.000000}%
\pgfsetstrokecolor{currentstroke}%
\pgfsetdash{}{0pt}%
\pgfsys@defobject{currentmarker}{\pgfqpoint{0.000000in}{-0.048611in}}{\pgfqpoint{0.000000in}{0.000000in}}{%
\pgfpathmoveto{\pgfqpoint{0.000000in}{0.000000in}}%
\pgfpathlineto{\pgfqpoint{0.000000in}{-0.048611in}}%
\pgfusepath{stroke,fill}%
}%
\begin{pgfscope}%
\pgfsys@transformshift{4.777620in}{0.528000in}%
\pgfsys@useobject{currentmarker}{}%
\end{pgfscope}%
\end{pgfscope}%
\begin{pgfscope}%
\definecolor{textcolor}{rgb}{0.000000,0.000000,0.000000}%
\pgfsetstrokecolor{textcolor}%
\pgfsetfillcolor{textcolor}%
\pgftext[x=4.777620in,y=0.430778in,,top]{\color{textcolor}\sffamily\fontsize{10.000000}{12.000000}\selectfont 25}%
\end{pgfscope}%
\begin{pgfscope}%
\pgfpathrectangle{\pgfqpoint{0.800000in}{0.528000in}}{\pgfqpoint{4.960000in}{3.696000in}}%
\pgfusepath{clip}%
\pgfsetrectcap%
\pgfsetroundjoin%
\pgfsetlinewidth{0.803000pt}%
\definecolor{currentstroke}{rgb}{0.690196,0.690196,0.690196}%
\pgfsetstrokecolor{currentstroke}%
\pgfsetdash{}{0pt}%
\pgfpathmoveto{\pgfqpoint{5.528053in}{0.528000in}}%
\pgfpathlineto{\pgfqpoint{5.528053in}{4.224000in}}%
\pgfusepath{stroke}%
\end{pgfscope}%
\begin{pgfscope}%
\pgfsetbuttcap%
\pgfsetroundjoin%
\definecolor{currentfill}{rgb}{0.000000,0.000000,0.000000}%
\pgfsetfillcolor{currentfill}%
\pgfsetlinewidth{0.803000pt}%
\definecolor{currentstroke}{rgb}{0.000000,0.000000,0.000000}%
\pgfsetstrokecolor{currentstroke}%
\pgfsetdash{}{0pt}%
\pgfsys@defobject{currentmarker}{\pgfqpoint{0.000000in}{-0.048611in}}{\pgfqpoint{0.000000in}{0.000000in}}{%
\pgfpathmoveto{\pgfqpoint{0.000000in}{0.000000in}}%
\pgfpathlineto{\pgfqpoint{0.000000in}{-0.048611in}}%
\pgfusepath{stroke,fill}%
}%
\begin{pgfscope}%
\pgfsys@transformshift{5.528053in}{0.528000in}%
\pgfsys@useobject{currentmarker}{}%
\end{pgfscope}%
\end{pgfscope}%
\begin{pgfscope}%
\definecolor{textcolor}{rgb}{0.000000,0.000000,0.000000}%
\pgfsetstrokecolor{textcolor}%
\pgfsetfillcolor{textcolor}%
\pgftext[x=5.528053in,y=0.430778in,,top]{\color{textcolor}\sffamily\fontsize{10.000000}{12.000000}\selectfont 30}%
\end{pgfscope}%
\begin{pgfscope}%
\definecolor{textcolor}{rgb}{0.000000,0.000000,0.000000}%
\pgfsetstrokecolor{textcolor}%
\pgfsetfillcolor{textcolor}%
\pgftext[x=3.280000in,y=0.240809in,,top]{\color{textcolor}\sffamily\fontsize{10.000000}{12.000000}\selectfont time [s]}%
\end{pgfscope}%
\begin{pgfscope}%
\pgfpathrectangle{\pgfqpoint{0.800000in}{0.528000in}}{\pgfqpoint{4.960000in}{3.696000in}}%
\pgfusepath{clip}%
\pgfsetrectcap%
\pgfsetroundjoin%
\pgfsetlinewidth{0.803000pt}%
\definecolor{currentstroke}{rgb}{0.690196,0.690196,0.690196}%
\pgfsetstrokecolor{currentstroke}%
\pgfsetdash{}{0pt}%
\pgfpathmoveto{\pgfqpoint{0.800000in}{0.569081in}}%
\pgfpathlineto{\pgfqpoint{5.760000in}{0.569081in}}%
\pgfusepath{stroke}%
\end{pgfscope}%
\begin{pgfscope}%
\pgfsetbuttcap%
\pgfsetroundjoin%
\definecolor{currentfill}{rgb}{0.000000,0.000000,0.000000}%
\pgfsetfillcolor{currentfill}%
\pgfsetlinewidth{0.803000pt}%
\definecolor{currentstroke}{rgb}{0.000000,0.000000,0.000000}%
\pgfsetstrokecolor{currentstroke}%
\pgfsetdash{}{0pt}%
\pgfsys@defobject{currentmarker}{\pgfqpoint{-0.048611in}{0.000000in}}{\pgfqpoint{-0.000000in}{0.000000in}}{%
\pgfpathmoveto{\pgfqpoint{-0.000000in}{0.000000in}}%
\pgfpathlineto{\pgfqpoint{-0.048611in}{0.000000in}}%
\pgfusepath{stroke,fill}%
}%
\begin{pgfscope}%
\pgfsys@transformshift{0.800000in}{0.569081in}%
\pgfsys@useobject{currentmarker}{}%
\end{pgfscope}%
\end{pgfscope}%
\begin{pgfscope}%
\definecolor{textcolor}{rgb}{0.000000,0.000000,0.000000}%
\pgfsetstrokecolor{textcolor}%
\pgfsetfillcolor{textcolor}%
\pgftext[x=0.506387in, y=0.516320in, left, base]{\color{textcolor}\sffamily\fontsize{10.000000}{12.000000}\selectfont \ensuremath{-}2}%
\end{pgfscope}%
\begin{pgfscope}%
\pgfpathrectangle{\pgfqpoint{0.800000in}{0.528000in}}{\pgfqpoint{4.960000in}{3.696000in}}%
\pgfusepath{clip}%
\pgfsetrectcap%
\pgfsetroundjoin%
\pgfsetlinewidth{0.803000pt}%
\definecolor{currentstroke}{rgb}{0.690196,0.690196,0.690196}%
\pgfsetstrokecolor{currentstroke}%
\pgfsetdash{}{0pt}%
\pgfpathmoveto{\pgfqpoint{0.800000in}{1.067212in}}%
\pgfpathlineto{\pgfqpoint{5.760000in}{1.067212in}}%
\pgfusepath{stroke}%
\end{pgfscope}%
\begin{pgfscope}%
\pgfsetbuttcap%
\pgfsetroundjoin%
\definecolor{currentfill}{rgb}{0.000000,0.000000,0.000000}%
\pgfsetfillcolor{currentfill}%
\pgfsetlinewidth{0.803000pt}%
\definecolor{currentstroke}{rgb}{0.000000,0.000000,0.000000}%
\pgfsetstrokecolor{currentstroke}%
\pgfsetdash{}{0pt}%
\pgfsys@defobject{currentmarker}{\pgfqpoint{-0.048611in}{0.000000in}}{\pgfqpoint{-0.000000in}{0.000000in}}{%
\pgfpathmoveto{\pgfqpoint{-0.000000in}{0.000000in}}%
\pgfpathlineto{\pgfqpoint{-0.048611in}{0.000000in}}%
\pgfusepath{stroke,fill}%
}%
\begin{pgfscope}%
\pgfsys@transformshift{0.800000in}{1.067212in}%
\pgfsys@useobject{currentmarker}{}%
\end{pgfscope}%
\end{pgfscope}%
\begin{pgfscope}%
\definecolor{textcolor}{rgb}{0.000000,0.000000,0.000000}%
\pgfsetstrokecolor{textcolor}%
\pgfsetfillcolor{textcolor}%
\pgftext[x=0.506387in, y=1.014451in, left, base]{\color{textcolor}\sffamily\fontsize{10.000000}{12.000000}\selectfont \ensuremath{-}1}%
\end{pgfscope}%
\begin{pgfscope}%
\pgfpathrectangle{\pgfqpoint{0.800000in}{0.528000in}}{\pgfqpoint{4.960000in}{3.696000in}}%
\pgfusepath{clip}%
\pgfsetrectcap%
\pgfsetroundjoin%
\pgfsetlinewidth{0.803000pt}%
\definecolor{currentstroke}{rgb}{0.690196,0.690196,0.690196}%
\pgfsetstrokecolor{currentstroke}%
\pgfsetdash{}{0pt}%
\pgfpathmoveto{\pgfqpoint{0.800000in}{1.565344in}}%
\pgfpathlineto{\pgfqpoint{5.760000in}{1.565344in}}%
\pgfusepath{stroke}%
\end{pgfscope}%
\begin{pgfscope}%
\pgfsetbuttcap%
\pgfsetroundjoin%
\definecolor{currentfill}{rgb}{0.000000,0.000000,0.000000}%
\pgfsetfillcolor{currentfill}%
\pgfsetlinewidth{0.803000pt}%
\definecolor{currentstroke}{rgb}{0.000000,0.000000,0.000000}%
\pgfsetstrokecolor{currentstroke}%
\pgfsetdash{}{0pt}%
\pgfsys@defobject{currentmarker}{\pgfqpoint{-0.048611in}{0.000000in}}{\pgfqpoint{-0.000000in}{0.000000in}}{%
\pgfpathmoveto{\pgfqpoint{-0.000000in}{0.000000in}}%
\pgfpathlineto{\pgfqpoint{-0.048611in}{0.000000in}}%
\pgfusepath{stroke,fill}%
}%
\begin{pgfscope}%
\pgfsys@transformshift{0.800000in}{1.565344in}%
\pgfsys@useobject{currentmarker}{}%
\end{pgfscope}%
\end{pgfscope}%
\begin{pgfscope}%
\definecolor{textcolor}{rgb}{0.000000,0.000000,0.000000}%
\pgfsetstrokecolor{textcolor}%
\pgfsetfillcolor{textcolor}%
\pgftext[x=0.614412in, y=1.512582in, left, base]{\color{textcolor}\sffamily\fontsize{10.000000}{12.000000}\selectfont 0}%
\end{pgfscope}%
\begin{pgfscope}%
\pgfpathrectangle{\pgfqpoint{0.800000in}{0.528000in}}{\pgfqpoint{4.960000in}{3.696000in}}%
\pgfusepath{clip}%
\pgfsetrectcap%
\pgfsetroundjoin%
\pgfsetlinewidth{0.803000pt}%
\definecolor{currentstroke}{rgb}{0.690196,0.690196,0.690196}%
\pgfsetstrokecolor{currentstroke}%
\pgfsetdash{}{0pt}%
\pgfpathmoveto{\pgfqpoint{0.800000in}{2.063475in}}%
\pgfpathlineto{\pgfqpoint{5.760000in}{2.063475in}}%
\pgfusepath{stroke}%
\end{pgfscope}%
\begin{pgfscope}%
\pgfsetbuttcap%
\pgfsetroundjoin%
\definecolor{currentfill}{rgb}{0.000000,0.000000,0.000000}%
\pgfsetfillcolor{currentfill}%
\pgfsetlinewidth{0.803000pt}%
\definecolor{currentstroke}{rgb}{0.000000,0.000000,0.000000}%
\pgfsetstrokecolor{currentstroke}%
\pgfsetdash{}{0pt}%
\pgfsys@defobject{currentmarker}{\pgfqpoint{-0.048611in}{0.000000in}}{\pgfqpoint{-0.000000in}{0.000000in}}{%
\pgfpathmoveto{\pgfqpoint{-0.000000in}{0.000000in}}%
\pgfpathlineto{\pgfqpoint{-0.048611in}{0.000000in}}%
\pgfusepath{stroke,fill}%
}%
\begin{pgfscope}%
\pgfsys@transformshift{0.800000in}{2.063475in}%
\pgfsys@useobject{currentmarker}{}%
\end{pgfscope}%
\end{pgfscope}%
\begin{pgfscope}%
\definecolor{textcolor}{rgb}{0.000000,0.000000,0.000000}%
\pgfsetstrokecolor{textcolor}%
\pgfsetfillcolor{textcolor}%
\pgftext[x=0.614412in, y=2.010713in, left, base]{\color{textcolor}\sffamily\fontsize{10.000000}{12.000000}\selectfont 1}%
\end{pgfscope}%
\begin{pgfscope}%
\pgfpathrectangle{\pgfqpoint{0.800000in}{0.528000in}}{\pgfqpoint{4.960000in}{3.696000in}}%
\pgfusepath{clip}%
\pgfsetrectcap%
\pgfsetroundjoin%
\pgfsetlinewidth{0.803000pt}%
\definecolor{currentstroke}{rgb}{0.690196,0.690196,0.690196}%
\pgfsetstrokecolor{currentstroke}%
\pgfsetdash{}{0pt}%
\pgfpathmoveto{\pgfqpoint{0.800000in}{2.561606in}}%
\pgfpathlineto{\pgfqpoint{5.760000in}{2.561606in}}%
\pgfusepath{stroke}%
\end{pgfscope}%
\begin{pgfscope}%
\pgfsetbuttcap%
\pgfsetroundjoin%
\definecolor{currentfill}{rgb}{0.000000,0.000000,0.000000}%
\pgfsetfillcolor{currentfill}%
\pgfsetlinewidth{0.803000pt}%
\definecolor{currentstroke}{rgb}{0.000000,0.000000,0.000000}%
\pgfsetstrokecolor{currentstroke}%
\pgfsetdash{}{0pt}%
\pgfsys@defobject{currentmarker}{\pgfqpoint{-0.048611in}{0.000000in}}{\pgfqpoint{-0.000000in}{0.000000in}}{%
\pgfpathmoveto{\pgfqpoint{-0.000000in}{0.000000in}}%
\pgfpathlineto{\pgfqpoint{-0.048611in}{0.000000in}}%
\pgfusepath{stroke,fill}%
}%
\begin{pgfscope}%
\pgfsys@transformshift{0.800000in}{2.561606in}%
\pgfsys@useobject{currentmarker}{}%
\end{pgfscope}%
\end{pgfscope}%
\begin{pgfscope}%
\definecolor{textcolor}{rgb}{0.000000,0.000000,0.000000}%
\pgfsetstrokecolor{textcolor}%
\pgfsetfillcolor{textcolor}%
\pgftext[x=0.614412in, y=2.508845in, left, base]{\color{textcolor}\sffamily\fontsize{10.000000}{12.000000}\selectfont 2}%
\end{pgfscope}%
\begin{pgfscope}%
\pgfpathrectangle{\pgfqpoint{0.800000in}{0.528000in}}{\pgfqpoint{4.960000in}{3.696000in}}%
\pgfusepath{clip}%
\pgfsetrectcap%
\pgfsetroundjoin%
\pgfsetlinewidth{0.803000pt}%
\definecolor{currentstroke}{rgb}{0.690196,0.690196,0.690196}%
\pgfsetstrokecolor{currentstroke}%
\pgfsetdash{}{0pt}%
\pgfpathmoveto{\pgfqpoint{0.800000in}{3.059737in}}%
\pgfpathlineto{\pgfqpoint{5.760000in}{3.059737in}}%
\pgfusepath{stroke}%
\end{pgfscope}%
\begin{pgfscope}%
\pgfsetbuttcap%
\pgfsetroundjoin%
\definecolor{currentfill}{rgb}{0.000000,0.000000,0.000000}%
\pgfsetfillcolor{currentfill}%
\pgfsetlinewidth{0.803000pt}%
\definecolor{currentstroke}{rgb}{0.000000,0.000000,0.000000}%
\pgfsetstrokecolor{currentstroke}%
\pgfsetdash{}{0pt}%
\pgfsys@defobject{currentmarker}{\pgfqpoint{-0.048611in}{0.000000in}}{\pgfqpoint{-0.000000in}{0.000000in}}{%
\pgfpathmoveto{\pgfqpoint{-0.000000in}{0.000000in}}%
\pgfpathlineto{\pgfqpoint{-0.048611in}{0.000000in}}%
\pgfusepath{stroke,fill}%
}%
\begin{pgfscope}%
\pgfsys@transformshift{0.800000in}{3.059737in}%
\pgfsys@useobject{currentmarker}{}%
\end{pgfscope}%
\end{pgfscope}%
\begin{pgfscope}%
\definecolor{textcolor}{rgb}{0.000000,0.000000,0.000000}%
\pgfsetstrokecolor{textcolor}%
\pgfsetfillcolor{textcolor}%
\pgftext[x=0.614412in, y=3.006976in, left, base]{\color{textcolor}\sffamily\fontsize{10.000000}{12.000000}\selectfont 3}%
\end{pgfscope}%
\begin{pgfscope}%
\pgfpathrectangle{\pgfqpoint{0.800000in}{0.528000in}}{\pgfqpoint{4.960000in}{3.696000in}}%
\pgfusepath{clip}%
\pgfsetrectcap%
\pgfsetroundjoin%
\pgfsetlinewidth{0.803000pt}%
\definecolor{currentstroke}{rgb}{0.690196,0.690196,0.690196}%
\pgfsetstrokecolor{currentstroke}%
\pgfsetdash{}{0pt}%
\pgfpathmoveto{\pgfqpoint{0.800000in}{3.557869in}}%
\pgfpathlineto{\pgfqpoint{5.760000in}{3.557869in}}%
\pgfusepath{stroke}%
\end{pgfscope}%
\begin{pgfscope}%
\pgfsetbuttcap%
\pgfsetroundjoin%
\definecolor{currentfill}{rgb}{0.000000,0.000000,0.000000}%
\pgfsetfillcolor{currentfill}%
\pgfsetlinewidth{0.803000pt}%
\definecolor{currentstroke}{rgb}{0.000000,0.000000,0.000000}%
\pgfsetstrokecolor{currentstroke}%
\pgfsetdash{}{0pt}%
\pgfsys@defobject{currentmarker}{\pgfqpoint{-0.048611in}{0.000000in}}{\pgfqpoint{-0.000000in}{0.000000in}}{%
\pgfpathmoveto{\pgfqpoint{-0.000000in}{0.000000in}}%
\pgfpathlineto{\pgfqpoint{-0.048611in}{0.000000in}}%
\pgfusepath{stroke,fill}%
}%
\begin{pgfscope}%
\pgfsys@transformshift{0.800000in}{3.557869in}%
\pgfsys@useobject{currentmarker}{}%
\end{pgfscope}%
\end{pgfscope}%
\begin{pgfscope}%
\definecolor{textcolor}{rgb}{0.000000,0.000000,0.000000}%
\pgfsetstrokecolor{textcolor}%
\pgfsetfillcolor{textcolor}%
\pgftext[x=0.614412in, y=3.505107in, left, base]{\color{textcolor}\sffamily\fontsize{10.000000}{12.000000}\selectfont 4}%
\end{pgfscope}%
\begin{pgfscope}%
\pgfpathrectangle{\pgfqpoint{0.800000in}{0.528000in}}{\pgfqpoint{4.960000in}{3.696000in}}%
\pgfusepath{clip}%
\pgfsetrectcap%
\pgfsetroundjoin%
\pgfsetlinewidth{0.803000pt}%
\definecolor{currentstroke}{rgb}{0.690196,0.690196,0.690196}%
\pgfsetstrokecolor{currentstroke}%
\pgfsetdash{}{0pt}%
\pgfpathmoveto{\pgfqpoint{0.800000in}{4.056000in}}%
\pgfpathlineto{\pgfqpoint{5.760000in}{4.056000in}}%
\pgfusepath{stroke}%
\end{pgfscope}%
\begin{pgfscope}%
\pgfsetbuttcap%
\pgfsetroundjoin%
\definecolor{currentfill}{rgb}{0.000000,0.000000,0.000000}%
\pgfsetfillcolor{currentfill}%
\pgfsetlinewidth{0.803000pt}%
\definecolor{currentstroke}{rgb}{0.000000,0.000000,0.000000}%
\pgfsetstrokecolor{currentstroke}%
\pgfsetdash{}{0pt}%
\pgfsys@defobject{currentmarker}{\pgfqpoint{-0.048611in}{0.000000in}}{\pgfqpoint{-0.000000in}{0.000000in}}{%
\pgfpathmoveto{\pgfqpoint{-0.000000in}{0.000000in}}%
\pgfpathlineto{\pgfqpoint{-0.048611in}{0.000000in}}%
\pgfusepath{stroke,fill}%
}%
\begin{pgfscope}%
\pgfsys@transformshift{0.800000in}{4.056000in}%
\pgfsys@useobject{currentmarker}{}%
\end{pgfscope}%
\end{pgfscope}%
\begin{pgfscope}%
\definecolor{textcolor}{rgb}{0.000000,0.000000,0.000000}%
\pgfsetstrokecolor{textcolor}%
\pgfsetfillcolor{textcolor}%
\pgftext[x=0.614412in, y=4.003238in, left, base]{\color{textcolor}\sffamily\fontsize{10.000000}{12.000000}\selectfont 5}%
\end{pgfscope}%
\begin{pgfscope}%
\definecolor{textcolor}{rgb}{0.000000,0.000000,0.000000}%
\pgfsetstrokecolor{textcolor}%
\pgfsetfillcolor{textcolor}%
\pgftext[x=0.450832in,y=2.376000in,,bottom,rotate=90.000000]{\color{textcolor}\sffamily\fontsize{10.000000}{12.000000}\selectfont Output velocity [deg/s]}%
\end{pgfscope}%
\begin{pgfscope}%
\pgfpathrectangle{\pgfqpoint{0.800000in}{0.528000in}}{\pgfqpoint{4.960000in}{3.696000in}}%
\pgfusepath{clip}%
\pgfsetrectcap%
\pgfsetroundjoin%
\pgfsetlinewidth{1.505625pt}%
\definecolor{currentstroke}{rgb}{0.121569,0.466667,0.705882}%
\pgfsetstrokecolor{currentstroke}%
\pgfsetdash{}{0pt}%
\pgfpathmoveto{\pgfqpoint{1.025455in}{4.056000in}}%
\pgfpathlineto{\pgfqpoint{1.080166in}{4.056000in}}%
\pgfpathlineto{\pgfqpoint{1.134635in}{4.056000in}}%
\pgfpathlineto{\pgfqpoint{1.189307in}{4.056000in}}%
\pgfpathlineto{\pgfqpoint{1.243360in}{4.056000in}}%
\pgfpathlineto{\pgfqpoint{1.298042in}{4.056000in}}%
\pgfpathlineto{\pgfqpoint{1.352583in}{4.056000in}}%
\pgfpathlineto{\pgfqpoint{1.406773in}{4.056000in}}%
\pgfpathlineto{\pgfqpoint{1.460553in}{4.056000in}}%
\pgfpathlineto{\pgfqpoint{1.515040in}{3.644961in}}%
\pgfpathlineto{\pgfqpoint{1.570820in}{2.551332in}}%
\pgfpathlineto{\pgfqpoint{1.625051in}{1.718932in}}%
\pgfpathlineto{\pgfqpoint{1.678843in}{1.102966in}}%
\pgfpathlineto{\pgfqpoint{1.733208in}{1.002427in}}%
\pgfpathlineto{\pgfqpoint{1.787527in}{1.025704in}}%
\pgfpathlineto{\pgfqpoint{1.841593in}{1.013317in}}%
\pgfpathlineto{\pgfqpoint{1.895731in}{1.311528in}}%
\pgfpathlineto{\pgfqpoint{1.950398in}{1.439130in}}%
\pgfpathlineto{\pgfqpoint{2.004622in}{1.686810in}}%
\pgfpathlineto{\pgfqpoint{2.059015in}{1.785063in}}%
\pgfpathlineto{\pgfqpoint{2.113207in}{1.820905in}}%
\pgfpathlineto{\pgfqpoint{2.168203in}{1.835413in}}%
\pgfpathlineto{\pgfqpoint{2.222993in}{1.719552in}}%
\pgfpathlineto{\pgfqpoint{2.276364in}{1.495445in}}%
\pgfpathlineto{\pgfqpoint{2.330589in}{1.506051in}}%
\pgfpathlineto{\pgfqpoint{2.384772in}{1.371483in}}%
\pgfpathlineto{\pgfqpoint{2.438965in}{1.362765in}}%
\pgfpathlineto{\pgfqpoint{2.493235in}{1.363448in}}%
\pgfpathlineto{\pgfqpoint{2.547792in}{1.298354in}}%
\pgfpathlineto{\pgfqpoint{2.602437in}{1.296251in}}%
\pgfpathlineto{\pgfqpoint{2.656286in}{1.358841in}}%
\pgfpathlineto{\pgfqpoint{2.710680in}{1.513186in}}%
\pgfpathlineto{\pgfqpoint{2.765200in}{1.551661in}}%
\pgfpathlineto{\pgfqpoint{2.820957in}{1.494190in}}%
\pgfpathlineto{\pgfqpoint{2.874965in}{1.666470in}}%
\pgfpathlineto{\pgfqpoint{2.928279in}{1.640499in}}%
\pgfpathlineto{\pgfqpoint{2.982366in}{1.608004in}}%
\pgfpathlineto{\pgfqpoint{3.036772in}{1.580827in}}%
\pgfpathlineto{\pgfqpoint{3.091062in}{1.576103in}}%
\pgfpathlineto{\pgfqpoint{3.145509in}{1.490653in}}%
\pgfpathlineto{\pgfqpoint{3.200139in}{1.388217in}}%
\pgfpathlineto{\pgfqpoint{3.254271in}{1.424848in}}%
\pgfpathlineto{\pgfqpoint{3.308336in}{1.502021in}}%
\pgfpathlineto{\pgfqpoint{3.362592in}{1.560876in}}%
\pgfpathlineto{\pgfqpoint{3.416708in}{1.568842in}}%
\pgfpathlineto{\pgfqpoint{3.472125in}{1.579192in}}%
\pgfpathlineto{\pgfqpoint{3.525649in}{1.628635in}}%
\pgfpathlineto{\pgfqpoint{3.579557in}{1.680423in}}%
\pgfpathlineto{\pgfqpoint{3.634116in}{1.711319in}}%
\pgfpathlineto{\pgfqpoint{3.688284in}{1.729519in}}%
\pgfpathlineto{\pgfqpoint{3.742596in}{1.544773in}}%
\pgfpathlineto{\pgfqpoint{3.797042in}{1.429101in}}%
\pgfpathlineto{\pgfqpoint{3.851612in}{1.383902in}}%
\pgfpathlineto{\pgfqpoint{3.905832in}{1.394056in}}%
\pgfpathlineto{\pgfqpoint{3.959908in}{1.558218in}}%
\pgfpathlineto{\pgfqpoint{4.015150in}{1.565385in}}%
\pgfpathlineto{\pgfqpoint{4.071012in}{1.628053in}}%
\pgfpathlineto{\pgfqpoint{4.124062in}{1.588432in}}%
\pgfpathlineto{\pgfqpoint{4.178078in}{1.689022in}}%
\pgfpathlineto{\pgfqpoint{4.232513in}{1.587604in}}%
\pgfpathlineto{\pgfqpoint{4.286824in}{1.517517in}}%
\pgfpathlineto{\pgfqpoint{4.341796in}{1.609079in}}%
\pgfpathlineto{\pgfqpoint{4.395095in}{1.488119in}}%
\pgfpathlineto{\pgfqpoint{4.449434in}{1.522158in}}%
\pgfpathlineto{\pgfqpoint{4.503604in}{1.577345in}}%
\pgfpathlineto{\pgfqpoint{4.557835in}{1.653037in}}%
\pgfpathlineto{\pgfqpoint{4.612134in}{1.662786in}}%
\pgfpathlineto{\pgfqpoint{4.666205in}{1.651358in}}%
\pgfpathlineto{\pgfqpoint{4.721780in}{1.604252in}}%
\pgfpathlineto{\pgfqpoint{4.775177in}{1.579547in}}%
\pgfpathlineto{\pgfqpoint{4.828950in}{1.570455in}}%
\pgfpathlineto{\pgfqpoint{4.884582in}{1.558487in}}%
\pgfpathlineto{\pgfqpoint{4.937200in}{1.561742in}}%
\pgfpathlineto{\pgfqpoint{4.991438in}{1.554729in}}%
\pgfpathlineto{\pgfqpoint{5.045564in}{1.543963in}}%
\pgfpathlineto{\pgfqpoint{5.100256in}{1.544354in}}%
\pgfpathlineto{\pgfqpoint{5.154242in}{1.562622in}}%
\pgfpathlineto{\pgfqpoint{5.208504in}{1.547826in}}%
\pgfpathlineto{\pgfqpoint{5.262395in}{1.550558in}}%
\pgfpathlineto{\pgfqpoint{5.316854in}{1.575653in}}%
\pgfpathlineto{\pgfqpoint{5.372435in}{1.346556in}}%
\pgfpathlineto{\pgfqpoint{5.426190in}{1.402472in}}%
\pgfpathlineto{\pgfqpoint{5.480011in}{1.390772in}}%
\pgfpathlineto{\pgfqpoint{5.533997in}{1.407252in}}%
\pgfusepath{stroke}%
\end{pgfscope}%
\begin{pgfscope}%
\pgfpathrectangle{\pgfqpoint{0.800000in}{0.528000in}}{\pgfqpoint{4.960000in}{3.696000in}}%
\pgfusepath{clip}%
\pgfsetrectcap%
\pgfsetroundjoin%
\pgfsetlinewidth{1.505625pt}%
\definecolor{currentstroke}{rgb}{1.000000,0.498039,0.054902}%
\pgfsetstrokecolor{currentstroke}%
\pgfsetdash{}{0pt}%
\pgfpathmoveto{\pgfqpoint{1.025455in}{4.056000in}}%
\pgfpathlineto{\pgfqpoint{1.079750in}{4.056000in}}%
\pgfpathlineto{\pgfqpoint{1.133888in}{4.056000in}}%
\pgfpathlineto{\pgfqpoint{1.188434in}{4.056000in}}%
\pgfpathlineto{\pgfqpoint{1.242282in}{4.056000in}}%
\pgfpathlineto{\pgfqpoint{1.296663in}{4.056000in}}%
\pgfpathlineto{\pgfqpoint{1.350562in}{4.056000in}}%
\pgfpathlineto{\pgfqpoint{1.405085in}{3.615168in}}%
\pgfpathlineto{\pgfqpoint{1.459266in}{2.914187in}}%
\pgfpathlineto{\pgfqpoint{1.513703in}{2.088305in}}%
\pgfpathlineto{\pgfqpoint{1.568960in}{1.536336in}}%
\pgfpathlineto{\pgfqpoint{1.622471in}{1.362219in}}%
\pgfpathlineto{\pgfqpoint{1.676403in}{1.504871in}}%
\pgfpathlineto{\pgfqpoint{1.730801in}{1.362967in}}%
\pgfpathlineto{\pgfqpoint{1.785347in}{1.466178in}}%
\pgfpathlineto{\pgfqpoint{1.839619in}{1.639508in}}%
\pgfpathlineto{\pgfqpoint{1.893542in}{1.683657in}}%
\pgfpathlineto{\pgfqpoint{1.947895in}{1.772086in}}%
\pgfpathlineto{\pgfqpoint{2.002284in}{1.791359in}}%
\pgfpathlineto{\pgfqpoint{2.056356in}{1.819886in}}%
\pgfpathlineto{\pgfqpoint{2.110971in}{1.790068in}}%
\pgfpathlineto{\pgfqpoint{2.165312in}{1.783121in}}%
\pgfpathlineto{\pgfqpoint{2.220969in}{1.737932in}}%
\pgfpathlineto{\pgfqpoint{2.274336in}{1.602406in}}%
\pgfpathlineto{\pgfqpoint{2.330352in}{1.541581in}}%
\pgfpathlineto{\pgfqpoint{2.382294in}{1.568835in}}%
\pgfpathlineto{\pgfqpoint{2.437054in}{1.470804in}}%
\pgfpathlineto{\pgfqpoint{2.491640in}{1.435084in}}%
\pgfpathlineto{\pgfqpoint{2.546087in}{1.483322in}}%
\pgfpathlineto{\pgfqpoint{2.599862in}{1.481067in}}%
\pgfpathlineto{\pgfqpoint{2.656657in}{1.505277in}}%
\pgfpathlineto{\pgfqpoint{2.708769in}{1.478386in}}%
\pgfpathlineto{\pgfqpoint{2.762909in}{1.501345in}}%
\pgfpathlineto{\pgfqpoint{2.818748in}{1.515540in}}%
\pgfpathlineto{\pgfqpoint{2.872578in}{1.508095in}}%
\pgfpathlineto{\pgfqpoint{2.926187in}{1.542354in}}%
\pgfpathlineto{\pgfqpoint{2.980158in}{1.506757in}}%
\pgfpathlineto{\pgfqpoint{3.034313in}{1.526896in}}%
\pgfpathlineto{\pgfqpoint{3.089249in}{1.487098in}}%
\pgfpathlineto{\pgfqpoint{3.146544in}{1.545776in}}%
\pgfpathlineto{\pgfqpoint{3.197223in}{1.493194in}}%
\pgfpathlineto{\pgfqpoint{3.251288in}{1.611222in}}%
\pgfpathlineto{\pgfqpoint{3.305522in}{1.657586in}}%
\pgfpathlineto{\pgfqpoint{3.361667in}{1.710828in}}%
\pgfpathlineto{\pgfqpoint{3.415360in}{1.721727in}}%
\pgfpathlineto{\pgfqpoint{3.470374in}{1.716845in}}%
\pgfpathlineto{\pgfqpoint{3.523981in}{1.722344in}}%
\pgfpathlineto{\pgfqpoint{3.578327in}{1.716118in}}%
\pgfpathlineto{\pgfqpoint{3.632938in}{1.699029in}}%
\pgfpathlineto{\pgfqpoint{3.686791in}{1.690369in}}%
\pgfpathlineto{\pgfqpoint{3.740763in}{1.604048in}}%
\pgfpathlineto{\pgfqpoint{3.794976in}{1.620510in}}%
\pgfpathlineto{\pgfqpoint{3.849426in}{1.505814in}}%
\pgfpathlineto{\pgfqpoint{3.903699in}{1.467045in}}%
\pgfpathlineto{\pgfqpoint{3.959208in}{1.485287in}}%
\pgfpathlineto{\pgfqpoint{4.013277in}{1.493576in}}%
\pgfpathlineto{\pgfqpoint{4.066807in}{1.500884in}}%
\pgfpathlineto{\pgfqpoint{4.121033in}{1.500813in}}%
\pgfpathlineto{\pgfqpoint{4.175145in}{1.604132in}}%
\pgfpathlineto{\pgfqpoint{4.229366in}{1.424593in}}%
\pgfpathlineto{\pgfqpoint{4.283535in}{1.633079in}}%
\pgfpathlineto{\pgfqpoint{4.337744in}{1.667276in}}%
\pgfpathlineto{\pgfqpoint{4.392696in}{1.672131in}}%
\pgfpathlineto{\pgfqpoint{4.446908in}{1.610151in}}%
\pgfpathlineto{\pgfqpoint{4.501094in}{1.657512in}}%
\pgfpathlineto{\pgfqpoint{4.556994in}{1.653136in}}%
\pgfpathlineto{\pgfqpoint{4.610504in}{1.649869in}}%
\pgfpathlineto{\pgfqpoint{4.664207in}{1.721834in}}%
\pgfpathlineto{\pgfqpoint{4.718082in}{1.707852in}}%
\pgfpathlineto{\pgfqpoint{4.772504in}{1.707372in}}%
\pgfpathlineto{\pgfqpoint{4.826657in}{1.663861in}}%
\pgfpathlineto{\pgfqpoint{4.880848in}{1.671581in}}%
\pgfpathlineto{\pgfqpoint{4.935639in}{1.659803in}}%
\pgfpathlineto{\pgfqpoint{4.989162in}{1.654332in}}%
\pgfpathlineto{\pgfqpoint{5.043500in}{1.581085in}}%
\pgfpathlineto{\pgfqpoint{5.098109in}{1.589562in}}%
\pgfpathlineto{\pgfqpoint{5.152079in}{1.292080in}}%
\pgfpathlineto{\pgfqpoint{5.207951in}{1.472163in}}%
\pgfpathlineto{\pgfqpoint{5.262151in}{1.440441in}}%
\pgfpathlineto{\pgfqpoint{5.315309in}{1.508896in}}%
\pgfpathlineto{\pgfqpoint{5.369464in}{1.498201in}}%
\pgfpathlineto{\pgfqpoint{5.423533in}{1.555795in}}%
\pgfpathlineto{\pgfqpoint{5.477683in}{1.590718in}}%
\pgfpathlineto{\pgfqpoint{5.531876in}{1.581143in}}%
\pgfusepath{stroke}%
\end{pgfscope}%
\begin{pgfscope}%
\pgfpathrectangle{\pgfqpoint{0.800000in}{0.528000in}}{\pgfqpoint{4.960000in}{3.696000in}}%
\pgfusepath{clip}%
\pgfsetrectcap%
\pgfsetroundjoin%
\pgfsetlinewidth{1.505625pt}%
\definecolor{currentstroke}{rgb}{0.172549,0.627451,0.172549}%
\pgfsetstrokecolor{currentstroke}%
\pgfsetdash{}{0pt}%
\pgfpathmoveto{\pgfqpoint{1.025455in}{4.056000in}}%
\pgfpathlineto{\pgfqpoint{1.079234in}{4.056000in}}%
\pgfpathlineto{\pgfqpoint{1.133723in}{4.056000in}}%
\pgfpathlineto{\pgfqpoint{1.187856in}{4.056000in}}%
\pgfpathlineto{\pgfqpoint{1.242214in}{4.056000in}}%
\pgfpathlineto{\pgfqpoint{1.296445in}{4.056000in}}%
\pgfpathlineto{\pgfqpoint{1.351799in}{4.056000in}}%
\pgfpathlineto{\pgfqpoint{1.405264in}{4.056000in}}%
\pgfpathlineto{\pgfqpoint{1.459281in}{3.792170in}}%
\pgfpathlineto{\pgfqpoint{1.513873in}{2.855511in}}%
\pgfpathlineto{\pgfqpoint{1.568033in}{2.309138in}}%
\pgfpathlineto{\pgfqpoint{1.622414in}{1.594608in}}%
\pgfpathlineto{\pgfqpoint{1.676742in}{1.360418in}}%
\pgfpathlineto{\pgfqpoint{1.730796in}{1.336422in}}%
\pgfpathlineto{\pgfqpoint{1.784969in}{0.834486in}}%
\pgfpathlineto{\pgfqpoint{1.839395in}{1.399558in}}%
\pgfpathlineto{\pgfqpoint{1.893423in}{1.391168in}}%
\pgfpathlineto{\pgfqpoint{1.948020in}{1.734856in}}%
\pgfpathlineto{\pgfqpoint{2.004111in}{1.654718in}}%
\pgfpathlineto{\pgfqpoint{2.057931in}{1.771431in}}%
\pgfpathlineto{\pgfqpoint{2.112063in}{1.618212in}}%
\pgfpathlineto{\pgfqpoint{2.165953in}{1.691340in}}%
\pgfpathlineto{\pgfqpoint{2.220218in}{1.735486in}}%
\pgfpathlineto{\pgfqpoint{2.274710in}{1.645715in}}%
\pgfpathlineto{\pgfqpoint{2.328663in}{1.370160in}}%
\pgfpathlineto{\pgfqpoint{2.383755in}{1.368214in}}%
\pgfpathlineto{\pgfqpoint{2.438423in}{1.473856in}}%
\pgfpathlineto{\pgfqpoint{2.492365in}{1.931953in}}%
\pgfpathlineto{\pgfqpoint{2.546597in}{1.804606in}}%
\pgfpathlineto{\pgfqpoint{2.602253in}{1.763066in}}%
\pgfpathlineto{\pgfqpoint{2.656772in}{1.358363in}}%
\pgfpathlineto{\pgfqpoint{2.710354in}{1.503908in}}%
\pgfpathlineto{\pgfqpoint{2.764377in}{1.477803in}}%
\pgfpathlineto{\pgfqpoint{2.818541in}{1.457667in}}%
\pgfpathlineto{\pgfqpoint{2.872733in}{1.675479in}}%
\pgfpathlineto{\pgfqpoint{2.927078in}{1.533630in}}%
\pgfpathlineto{\pgfqpoint{2.981209in}{1.476177in}}%
\pgfpathlineto{\pgfqpoint{3.036167in}{1.694981in}}%
\pgfpathlineto{\pgfqpoint{3.090014in}{1.551933in}}%
\pgfpathlineto{\pgfqpoint{3.144250in}{1.665769in}}%
\pgfpathlineto{\pgfqpoint{3.198300in}{1.599742in}}%
\pgfpathlineto{\pgfqpoint{3.253835in}{1.479212in}}%
\pgfpathlineto{\pgfqpoint{3.307738in}{1.453330in}}%
\pgfpathlineto{\pgfqpoint{3.362872in}{1.492521in}}%
\pgfpathlineto{\pgfqpoint{3.415982in}{1.518525in}}%
\pgfpathlineto{\pgfqpoint{3.470306in}{1.503330in}}%
\pgfpathlineto{\pgfqpoint{3.524501in}{1.405385in}}%
\pgfpathlineto{\pgfqpoint{3.578509in}{1.358087in}}%
\pgfpathlineto{\pgfqpoint{3.633194in}{1.408943in}}%
\pgfpathlineto{\pgfqpoint{3.687058in}{1.474291in}}%
\pgfpathlineto{\pgfqpoint{3.741744in}{1.486363in}}%
\pgfpathlineto{\pgfqpoint{3.795463in}{1.498829in}}%
\pgfpathlineto{\pgfqpoint{3.851010in}{1.460573in}}%
\pgfpathlineto{\pgfqpoint{3.904699in}{1.458184in}}%
\pgfpathlineto{\pgfqpoint{3.958547in}{1.468629in}}%
\pgfpathlineto{\pgfqpoint{4.013533in}{1.483774in}}%
\pgfpathlineto{\pgfqpoint{4.067391in}{1.461443in}}%
\pgfpathlineto{\pgfqpoint{4.121392in}{1.509665in}}%
\pgfpathlineto{\pgfqpoint{4.175667in}{1.530295in}}%
\pgfpathlineto{\pgfqpoint{4.230121in}{1.546360in}}%
\pgfpathlineto{\pgfqpoint{4.284602in}{1.322290in}}%
\pgfpathlineto{\pgfqpoint{4.338955in}{1.266159in}}%
\pgfpathlineto{\pgfqpoint{4.393221in}{1.306288in}}%
\pgfpathlineto{\pgfqpoint{4.448144in}{1.492964in}}%
\pgfpathlineto{\pgfqpoint{4.503135in}{1.464622in}}%
\pgfpathlineto{\pgfqpoint{4.556389in}{1.576586in}}%
\pgfpathlineto{\pgfqpoint{4.610159in}{1.650453in}}%
\pgfpathlineto{\pgfqpoint{4.664807in}{1.704514in}}%
\pgfpathlineto{\pgfqpoint{4.719074in}{1.695141in}}%
\pgfpathlineto{\pgfqpoint{4.773451in}{1.703254in}}%
\pgfpathlineto{\pgfqpoint{4.827889in}{1.638940in}}%
\pgfpathlineto{\pgfqpoint{4.882397in}{1.526400in}}%
\pgfpathlineto{\pgfqpoint{4.935989in}{1.242134in}}%
\pgfpathlineto{\pgfqpoint{4.990264in}{1.317703in}}%
\pgfpathlineto{\pgfqpoint{5.046432in}{1.440661in}}%
\pgfpathlineto{\pgfqpoint{5.100510in}{1.712572in}}%
\pgfpathlineto{\pgfqpoint{5.154473in}{1.770669in}}%
\pgfpathlineto{\pgfqpoint{5.208241in}{1.816516in}}%
\pgfpathlineto{\pgfqpoint{5.262422in}{1.743879in}}%
\pgfpathlineto{\pgfqpoint{5.316691in}{1.752473in}}%
\pgfpathlineto{\pgfqpoint{5.371032in}{1.734100in}}%
\pgfpathlineto{\pgfqpoint{5.426445in}{1.238520in}}%
\pgfpathlineto{\pgfqpoint{5.479666in}{1.556913in}}%
\pgfpathlineto{\pgfqpoint{5.533619in}{1.554920in}}%
\pgfusepath{stroke}%
\end{pgfscope}%
\begin{pgfscope}%
\pgfpathrectangle{\pgfqpoint{0.800000in}{0.528000in}}{\pgfqpoint{4.960000in}{3.696000in}}%
\pgfusepath{clip}%
\pgfsetrectcap%
\pgfsetroundjoin%
\pgfsetlinewidth{1.505625pt}%
\definecolor{currentstroke}{rgb}{0.839216,0.152941,0.156863}%
\pgfsetstrokecolor{currentstroke}%
\pgfsetdash{}{0pt}%
\pgfpathmoveto{\pgfqpoint{1.025455in}{4.056000in}}%
\pgfpathlineto{\pgfqpoint{1.079980in}{4.056000in}}%
\pgfpathlineto{\pgfqpoint{1.134597in}{4.056000in}}%
\pgfpathlineto{\pgfqpoint{1.188728in}{4.056000in}}%
\pgfpathlineto{\pgfqpoint{1.242825in}{4.056000in}}%
\pgfpathlineto{\pgfqpoint{1.297292in}{4.056000in}}%
\pgfpathlineto{\pgfqpoint{1.351312in}{4.056000in}}%
\pgfpathlineto{\pgfqpoint{1.405887in}{4.056000in}}%
\pgfpathlineto{\pgfqpoint{1.460285in}{4.056000in}}%
\pgfpathlineto{\pgfqpoint{1.514354in}{3.469995in}}%
\pgfpathlineto{\pgfqpoint{1.569017in}{2.533957in}}%
\pgfpathlineto{\pgfqpoint{1.622677in}{1.518835in}}%
\pgfpathlineto{\pgfqpoint{1.677009in}{1.388805in}}%
\pgfpathlineto{\pgfqpoint{1.732580in}{1.314773in}}%
\pgfpathlineto{\pgfqpoint{1.786694in}{1.281103in}}%
\pgfpathlineto{\pgfqpoint{1.840752in}{1.374133in}}%
\pgfpathlineto{\pgfqpoint{1.894556in}{1.242430in}}%
\pgfpathlineto{\pgfqpoint{1.948848in}{1.283193in}}%
\pgfpathlineto{\pgfqpoint{2.003216in}{1.619258in}}%
\pgfpathlineto{\pgfqpoint{2.057766in}{1.483284in}}%
\pgfpathlineto{\pgfqpoint{2.112133in}{1.436888in}}%
\pgfpathlineto{\pgfqpoint{2.166679in}{1.407488in}}%
\pgfpathlineto{\pgfqpoint{2.220665in}{1.276695in}}%
\pgfpathlineto{\pgfqpoint{2.274773in}{1.360187in}}%
\pgfpathlineto{\pgfqpoint{2.328950in}{1.263207in}}%
\pgfpathlineto{\pgfqpoint{2.384095in}{1.262886in}}%
\pgfpathlineto{\pgfqpoint{2.437575in}{1.519422in}}%
\pgfpathlineto{\pgfqpoint{2.491663in}{1.612107in}}%
\pgfpathlineto{\pgfqpoint{2.545879in}{1.588568in}}%
\pgfpathlineto{\pgfqpoint{2.599970in}{1.591187in}}%
\pgfpathlineto{\pgfqpoint{2.654219in}{1.583943in}}%
\pgfpathlineto{\pgfqpoint{2.709374in}{1.503948in}}%
\pgfpathlineto{\pgfqpoint{2.763997in}{1.487932in}}%
\pgfpathlineto{\pgfqpoint{2.817975in}{1.385584in}}%
\pgfpathlineto{\pgfqpoint{2.871924in}{1.337323in}}%
\pgfpathlineto{\pgfqpoint{2.926113in}{1.288912in}}%
\pgfpathlineto{\pgfqpoint{2.982102in}{1.286295in}}%
\pgfpathlineto{\pgfqpoint{3.035437in}{1.393283in}}%
\pgfpathlineto{\pgfqpoint{3.089546in}{1.315461in}}%
\pgfpathlineto{\pgfqpoint{3.143619in}{1.398196in}}%
\pgfpathlineto{\pgfqpoint{3.197969in}{1.573054in}}%
\pgfpathlineto{\pgfqpoint{3.252444in}{1.311237in}}%
\pgfpathlineto{\pgfqpoint{3.307143in}{1.449154in}}%
\pgfpathlineto{\pgfqpoint{3.361390in}{1.547573in}}%
\pgfpathlineto{\pgfqpoint{3.415978in}{1.466388in}}%
\pgfpathlineto{\pgfqpoint{3.469575in}{1.478361in}}%
\pgfpathlineto{\pgfqpoint{3.524316in}{1.547003in}}%
\pgfpathlineto{\pgfqpoint{3.579516in}{1.574393in}}%
\pgfpathlineto{\pgfqpoint{3.632768in}{1.624477in}}%
\pgfpathlineto{\pgfqpoint{3.686758in}{1.266444in}}%
\pgfpathlineto{\pgfqpoint{3.740714in}{1.356938in}}%
\pgfpathlineto{\pgfqpoint{3.794948in}{1.369930in}}%
\pgfpathlineto{\pgfqpoint{3.849494in}{1.711343in}}%
\pgfpathlineto{\pgfqpoint{3.903841in}{1.532631in}}%
\pgfpathlineto{\pgfqpoint{3.958346in}{1.632629in}}%
\pgfpathlineto{\pgfqpoint{4.012774in}{1.676521in}}%
\pgfpathlineto{\pgfqpoint{4.066848in}{1.707725in}}%
\pgfpathlineto{\pgfqpoint{4.121072in}{1.709220in}}%
\pgfpathlineto{\pgfqpoint{4.175089in}{1.585883in}}%
\pgfpathlineto{\pgfqpoint{4.230259in}{1.625353in}}%
\pgfpathlineto{\pgfqpoint{4.284046in}{1.626168in}}%
\pgfpathlineto{\pgfqpoint{4.338109in}{1.626137in}}%
\pgfpathlineto{\pgfqpoint{4.392484in}{1.628252in}}%
\pgfpathlineto{\pgfqpoint{4.446605in}{1.615695in}}%
\pgfpathlineto{\pgfqpoint{4.500903in}{1.437903in}}%
\pgfpathlineto{\pgfqpoint{4.555341in}{1.480808in}}%
\pgfpathlineto{\pgfqpoint{4.611108in}{1.584570in}}%
\pgfpathlineto{\pgfqpoint{4.665816in}{1.121639in}}%
\pgfpathlineto{\pgfqpoint{4.719430in}{1.662476in}}%
\pgfpathlineto{\pgfqpoint{4.773188in}{1.583973in}}%
\pgfpathlineto{\pgfqpoint{4.827429in}{1.667474in}}%
\pgfpathlineto{\pgfqpoint{4.882588in}{1.702187in}}%
\pgfpathlineto{\pgfqpoint{4.936268in}{1.727052in}}%
\pgfpathlineto{\pgfqpoint{4.990289in}{1.547075in}}%
\pgfpathlineto{\pgfqpoint{5.044253in}{1.529058in}}%
\pgfpathlineto{\pgfqpoint{5.098808in}{1.557893in}}%
\pgfpathlineto{\pgfqpoint{5.152836in}{1.495921in}}%
\pgfpathlineto{\pgfqpoint{5.207623in}{1.527818in}}%
\pgfpathlineto{\pgfqpoint{5.262551in}{1.361702in}}%
\pgfpathlineto{\pgfqpoint{5.316455in}{1.316837in}}%
\pgfpathlineto{\pgfqpoint{5.370761in}{1.618900in}}%
\pgfpathlineto{\pgfqpoint{5.424557in}{1.309176in}}%
\pgfpathlineto{\pgfqpoint{5.480693in}{1.573435in}}%
\pgfpathlineto{\pgfqpoint{5.534545in}{1.802187in}}%
\pgfusepath{stroke}%
\end{pgfscope}%
\begin{pgfscope}%
\pgfpathrectangle{\pgfqpoint{0.800000in}{0.528000in}}{\pgfqpoint{4.960000in}{3.696000in}}%
\pgfusepath{clip}%
\pgfsetrectcap%
\pgfsetroundjoin%
\pgfsetlinewidth{1.505625pt}%
\definecolor{currentstroke}{rgb}{0.580392,0.403922,0.741176}%
\pgfsetstrokecolor{currentstroke}%
\pgfsetdash{}{0pt}%
\pgfpathmoveto{\pgfqpoint{1.025455in}{4.056000in}}%
\pgfpathlineto{\pgfqpoint{1.079612in}{4.056000in}}%
\pgfpathlineto{\pgfqpoint{1.133728in}{4.056000in}}%
\pgfpathlineto{\pgfqpoint{1.188278in}{4.056000in}}%
\pgfpathlineto{\pgfqpoint{1.242625in}{4.056000in}}%
\pgfpathlineto{\pgfqpoint{1.297054in}{4.056000in}}%
\pgfpathlineto{\pgfqpoint{1.350867in}{4.056000in}}%
\pgfpathlineto{\pgfqpoint{1.407778in}{4.056000in}}%
\pgfpathlineto{\pgfqpoint{1.459536in}{4.047747in}}%
\pgfpathlineto{\pgfqpoint{1.516888in}{3.050348in}}%
\pgfpathlineto{\pgfqpoint{1.568114in}{2.081785in}}%
\pgfpathlineto{\pgfqpoint{1.622203in}{1.468414in}}%
\pgfpathlineto{\pgfqpoint{1.676843in}{1.379310in}}%
\pgfpathlineto{\pgfqpoint{1.731122in}{1.657032in}}%
\pgfpathlineto{\pgfqpoint{1.786391in}{1.079260in}}%
\pgfpathlineto{\pgfqpoint{1.839042in}{1.120531in}}%
\pgfpathlineto{\pgfqpoint{1.893272in}{1.619603in}}%
\pgfpathlineto{\pgfqpoint{1.948145in}{1.554543in}}%
\pgfpathlineto{\pgfqpoint{2.002288in}{1.371938in}}%
\pgfpathlineto{\pgfqpoint{2.056741in}{1.379976in}}%
\pgfpathlineto{\pgfqpoint{2.110706in}{1.139602in}}%
\pgfpathlineto{\pgfqpoint{2.166371in}{1.158702in}}%
\pgfpathlineto{\pgfqpoint{2.220878in}{1.255423in}}%
\pgfpathlineto{\pgfqpoint{2.274947in}{1.528054in}}%
\pgfpathlineto{\pgfqpoint{2.328962in}{1.815622in}}%
\pgfpathlineto{\pgfqpoint{2.383201in}{1.704120in}}%
\pgfpathlineto{\pgfqpoint{2.437301in}{1.507264in}}%
\pgfpathlineto{\pgfqpoint{2.491747in}{1.525402in}}%
\pgfpathlineto{\pgfqpoint{2.545961in}{1.578446in}}%
\pgfpathlineto{\pgfqpoint{2.600272in}{1.543476in}}%
\pgfpathlineto{\pgfqpoint{2.656414in}{1.377695in}}%
\pgfpathlineto{\pgfqpoint{2.708666in}{1.560295in}}%
\pgfpathlineto{\pgfqpoint{2.764570in}{1.507152in}}%
\pgfpathlineto{\pgfqpoint{2.818290in}{1.547408in}}%
\pgfpathlineto{\pgfqpoint{2.871706in}{1.557249in}}%
\pgfpathlineto{\pgfqpoint{2.925827in}{1.537069in}}%
\pgfpathlineto{\pgfqpoint{2.980157in}{1.509463in}}%
\pgfpathlineto{\pgfqpoint{3.034053in}{1.479401in}}%
\pgfpathlineto{\pgfqpoint{3.088412in}{1.459969in}}%
\pgfpathlineto{\pgfqpoint{3.142668in}{1.576116in}}%
\pgfpathlineto{\pgfqpoint{3.196718in}{1.483570in}}%
\pgfpathlineto{\pgfqpoint{3.250994in}{1.496941in}}%
\pgfpathlineto{\pgfqpoint{3.305298in}{1.664189in}}%
\pgfpathlineto{\pgfqpoint{3.359939in}{1.653775in}}%
\pgfpathlineto{\pgfqpoint{3.415995in}{1.623897in}}%
\pgfpathlineto{\pgfqpoint{3.469035in}{1.531045in}}%
\pgfpathlineto{\pgfqpoint{3.522587in}{1.571755in}}%
\pgfpathlineto{\pgfqpoint{3.576535in}{1.001632in}}%
\pgfpathlineto{\pgfqpoint{3.630804in}{1.250688in}}%
\pgfpathlineto{\pgfqpoint{3.685400in}{1.666061in}}%
\pgfpathlineto{\pgfqpoint{3.739759in}{1.770396in}}%
\pgfpathlineto{\pgfqpoint{3.794077in}{1.631392in}}%
\pgfpathlineto{\pgfqpoint{3.848019in}{1.664341in}}%
\pgfpathlineto{\pgfqpoint{3.904291in}{1.067171in}}%
\pgfpathlineto{\pgfqpoint{3.958163in}{1.551789in}}%
\pgfpathlineto{\pgfqpoint{4.012002in}{1.142078in}}%
\pgfpathlineto{\pgfqpoint{4.065890in}{1.668605in}}%
\pgfpathlineto{\pgfqpoint{4.120538in}{1.846075in}}%
\pgfpathlineto{\pgfqpoint{4.174575in}{1.217720in}}%
\pgfpathlineto{\pgfqpoint{4.229997in}{1.261153in}}%
\pgfpathlineto{\pgfqpoint{4.284433in}{1.328495in}}%
\pgfpathlineto{\pgfqpoint{4.338741in}{1.793267in}}%
\pgfpathlineto{\pgfqpoint{4.393199in}{1.727788in}}%
\pgfpathlineto{\pgfqpoint{4.447253in}{1.686486in}}%
\pgfpathlineto{\pgfqpoint{4.501498in}{1.486607in}}%
\pgfpathlineto{\pgfqpoint{4.557188in}{1.774406in}}%
\pgfpathlineto{\pgfqpoint{4.610632in}{1.551640in}}%
\pgfpathlineto{\pgfqpoint{4.664969in}{1.587194in}}%
\pgfpathlineto{\pgfqpoint{4.719042in}{1.581481in}}%
\pgfpathlineto{\pgfqpoint{4.773240in}{1.688190in}}%
\pgfpathlineto{\pgfqpoint{4.827408in}{1.687192in}}%
\pgfpathlineto{\pgfqpoint{4.881508in}{1.664699in}}%
\pgfpathlineto{\pgfqpoint{4.936140in}{1.414575in}}%
\pgfpathlineto{\pgfqpoint{4.990022in}{1.418045in}}%
\pgfpathlineto{\pgfqpoint{5.044450in}{1.122103in}}%
\pgfpathlineto{\pgfqpoint{5.098913in}{1.631504in}}%
\pgfpathlineto{\pgfqpoint{5.153034in}{1.261188in}}%
\pgfpathlineto{\pgfqpoint{5.208996in}{1.732401in}}%
\pgfpathlineto{\pgfqpoint{5.262675in}{1.795692in}}%
\pgfpathlineto{\pgfqpoint{5.317141in}{1.597731in}}%
\pgfpathlineto{\pgfqpoint{5.370575in}{1.479830in}}%
\pgfpathlineto{\pgfqpoint{5.424782in}{1.770846in}}%
\pgfpathlineto{\pgfqpoint{5.479037in}{1.666878in}}%
\pgfpathlineto{\pgfqpoint{5.533378in}{1.773035in}}%
\pgfusepath{stroke}%
\end{pgfscope}%
\begin{pgfscope}%
\pgfpathrectangle{\pgfqpoint{0.800000in}{0.528000in}}{\pgfqpoint{4.960000in}{3.696000in}}%
\pgfusepath{clip}%
\pgfsetrectcap%
\pgfsetroundjoin%
\pgfsetlinewidth{1.505625pt}%
\definecolor{currentstroke}{rgb}{0.549020,0.337255,0.294118}%
\pgfsetstrokecolor{currentstroke}%
\pgfsetdash{}{0pt}%
\pgfpathmoveto{\pgfqpoint{1.025455in}{4.056000in}}%
\pgfpathlineto{\pgfqpoint{1.079685in}{4.056000in}}%
\pgfpathlineto{\pgfqpoint{1.133950in}{4.056000in}}%
\pgfpathlineto{\pgfqpoint{1.188380in}{4.056000in}}%
\pgfpathlineto{\pgfqpoint{1.242661in}{4.056000in}}%
\pgfpathlineto{\pgfqpoint{1.296693in}{4.056000in}}%
\pgfpathlineto{\pgfqpoint{1.350861in}{4.056000in}}%
\pgfpathlineto{\pgfqpoint{1.405072in}{4.056000in}}%
\pgfpathlineto{\pgfqpoint{1.459209in}{4.056000in}}%
\pgfpathlineto{\pgfqpoint{1.517935in}{3.199624in}}%
\pgfpathlineto{\pgfqpoint{1.569225in}{1.983333in}}%
\pgfpathlineto{\pgfqpoint{1.623079in}{1.402345in}}%
\pgfpathlineto{\pgfqpoint{1.677664in}{1.616655in}}%
\pgfpathlineto{\pgfqpoint{1.731891in}{1.100717in}}%
\pgfpathlineto{\pgfqpoint{1.785793in}{1.503615in}}%
\pgfpathlineto{\pgfqpoint{1.840216in}{1.727140in}}%
\pgfpathlineto{\pgfqpoint{1.894779in}{1.304869in}}%
\pgfpathlineto{\pgfqpoint{1.948755in}{1.370384in}}%
\pgfpathlineto{\pgfqpoint{2.003100in}{1.282449in}}%
\pgfpathlineto{\pgfqpoint{2.057969in}{1.377650in}}%
\pgfpathlineto{\pgfqpoint{2.112760in}{0.696000in}}%
\pgfpathlineto{\pgfqpoint{2.167550in}{1.108719in}}%
\pgfpathlineto{\pgfqpoint{2.221171in}{1.313345in}}%
\pgfpathlineto{\pgfqpoint{2.275106in}{1.730200in}}%
\pgfpathlineto{\pgfqpoint{2.330373in}{1.464557in}}%
\pgfpathlineto{\pgfqpoint{2.384469in}{1.757296in}}%
\pgfpathlineto{\pgfqpoint{2.438635in}{1.501080in}}%
\pgfpathlineto{\pgfqpoint{2.493008in}{1.501737in}}%
\pgfpathlineto{\pgfqpoint{2.547177in}{1.372885in}}%
\pgfpathlineto{\pgfqpoint{2.601370in}{1.474567in}}%
\pgfpathlineto{\pgfqpoint{2.656184in}{1.566583in}}%
\pgfpathlineto{\pgfqpoint{2.711210in}{1.440141in}}%
\pgfpathlineto{\pgfqpoint{2.766803in}{1.396606in}}%
\pgfpathlineto{\pgfqpoint{2.820458in}{1.229162in}}%
\pgfpathlineto{\pgfqpoint{2.874422in}{1.589102in}}%
\pgfpathlineto{\pgfqpoint{2.928908in}{1.529477in}}%
\pgfpathlineto{\pgfqpoint{2.982855in}{1.549008in}}%
\pgfpathlineto{\pgfqpoint{3.036961in}{1.482167in}}%
\pgfpathlineto{\pgfqpoint{3.091462in}{1.450461in}}%
\pgfpathlineto{\pgfqpoint{3.145671in}{1.675005in}}%
\pgfpathlineto{\pgfqpoint{3.199515in}{1.602947in}}%
\pgfpathlineto{\pgfqpoint{3.253880in}{0.951207in}}%
\pgfpathlineto{\pgfqpoint{3.309202in}{1.559741in}}%
\pgfpathlineto{\pgfqpoint{3.363164in}{1.638775in}}%
\pgfpathlineto{\pgfqpoint{3.418277in}{1.677407in}}%
\pgfpathlineto{\pgfqpoint{3.472086in}{1.340268in}}%
\pgfpathlineto{\pgfqpoint{3.526526in}{1.684125in}}%
\pgfpathlineto{\pgfqpoint{3.581203in}{1.595396in}}%
\pgfpathlineto{\pgfqpoint{3.635095in}{1.673293in}}%
\pgfpathlineto{\pgfqpoint{3.689165in}{1.424422in}}%
\pgfpathlineto{\pgfqpoint{3.743641in}{1.568766in}}%
\pgfpathlineto{\pgfqpoint{3.797741in}{1.119977in}}%
\pgfpathlineto{\pgfqpoint{3.852674in}{1.303694in}}%
\pgfpathlineto{\pgfqpoint{3.906592in}{1.662730in}}%
\pgfpathlineto{\pgfqpoint{3.960588in}{1.947844in}}%
\pgfpathlineto{\pgfqpoint{4.015476in}{1.701513in}}%
\pgfpathlineto{\pgfqpoint{4.070544in}{1.665897in}}%
\pgfpathlineto{\pgfqpoint{4.124128in}{1.648234in}}%
\pgfpathlineto{\pgfqpoint{4.177774in}{1.424656in}}%
\pgfpathlineto{\pgfqpoint{4.231779in}{1.212882in}}%
\pgfpathlineto{\pgfqpoint{4.286009in}{1.418221in}}%
\pgfpathlineto{\pgfqpoint{4.340192in}{1.771350in}}%
\pgfpathlineto{\pgfqpoint{4.394409in}{1.859079in}}%
\pgfpathlineto{\pgfqpoint{4.448987in}{1.683819in}}%
\pgfpathlineto{\pgfqpoint{4.503260in}{1.712446in}}%
\pgfpathlineto{\pgfqpoint{4.557021in}{1.982164in}}%
\pgfpathlineto{\pgfqpoint{4.611398in}{1.357843in}}%
\pgfpathlineto{\pgfqpoint{4.667188in}{1.998615in}}%
\pgfpathlineto{\pgfqpoint{4.720436in}{1.745445in}}%
\pgfpathlineto{\pgfqpoint{4.774201in}{1.166323in}}%
\pgfpathlineto{\pgfqpoint{4.828159in}{1.642066in}}%
\pgfpathlineto{\pgfqpoint{4.882334in}{1.847825in}}%
\pgfpathlineto{\pgfqpoint{4.936546in}{1.794106in}}%
\pgfpathlineto{\pgfqpoint{4.991058in}{1.675306in}}%
\pgfpathlineto{\pgfqpoint{5.045460in}{1.607938in}}%
\pgfpathlineto{\pgfqpoint{5.099913in}{1.578420in}}%
\pgfpathlineto{\pgfqpoint{5.153394in}{1.651184in}}%
\pgfpathlineto{\pgfqpoint{5.207586in}{1.549314in}}%
\pgfpathlineto{\pgfqpoint{5.261878in}{1.047402in}}%
\pgfpathlineto{\pgfqpoint{5.317607in}{1.198783in}}%
\pgfpathlineto{\pgfqpoint{5.371075in}{1.843853in}}%
\pgfpathlineto{\pgfqpoint{5.424720in}{1.861369in}}%
\pgfpathlineto{\pgfqpoint{5.478932in}{1.770925in}}%
\pgfpathlineto{\pgfqpoint{5.533007in}{1.731967in}}%
\pgfusepath{stroke}%
\end{pgfscope}%
\begin{pgfscope}%
\pgfsetrectcap%
\pgfsetmiterjoin%
\pgfsetlinewidth{0.803000pt}%
\definecolor{currentstroke}{rgb}{0.000000,0.000000,0.000000}%
\pgfsetstrokecolor{currentstroke}%
\pgfsetdash{}{0pt}%
\pgfpathmoveto{\pgfqpoint{0.800000in}{0.528000in}}%
\pgfpathlineto{\pgfqpoint{0.800000in}{4.224000in}}%
\pgfusepath{stroke}%
\end{pgfscope}%
\begin{pgfscope}%
\pgfsetrectcap%
\pgfsetmiterjoin%
\pgfsetlinewidth{0.803000pt}%
\definecolor{currentstroke}{rgb}{0.000000,0.000000,0.000000}%
\pgfsetstrokecolor{currentstroke}%
\pgfsetdash{}{0pt}%
\pgfpathmoveto{\pgfqpoint{5.760000in}{0.528000in}}%
\pgfpathlineto{\pgfqpoint{5.760000in}{4.224000in}}%
\pgfusepath{stroke}%
\end{pgfscope}%
\begin{pgfscope}%
\pgfsetrectcap%
\pgfsetmiterjoin%
\pgfsetlinewidth{0.803000pt}%
\definecolor{currentstroke}{rgb}{0.000000,0.000000,0.000000}%
\pgfsetstrokecolor{currentstroke}%
\pgfsetdash{}{0pt}%
\pgfpathmoveto{\pgfqpoint{0.800000in}{0.528000in}}%
\pgfpathlineto{\pgfqpoint{5.760000in}{0.528000in}}%
\pgfusepath{stroke}%
\end{pgfscope}%
\begin{pgfscope}%
\pgfsetrectcap%
\pgfsetmiterjoin%
\pgfsetlinewidth{0.803000pt}%
\definecolor{currentstroke}{rgb}{0.000000,0.000000,0.000000}%
\pgfsetstrokecolor{currentstroke}%
\pgfsetdash{}{0pt}%
\pgfpathmoveto{\pgfqpoint{0.800000in}{4.224000in}}%
\pgfpathlineto{\pgfqpoint{5.760000in}{4.224000in}}%
\pgfusepath{stroke}%
\end{pgfscope}%
\begin{pgfscope}%
\definecolor{textcolor}{rgb}{0.000000,0.000000,0.000000}%
\pgfsetstrokecolor{textcolor}%
\pgfsetfillcolor{textcolor}%
\pgftext[x=3.280000in,y=4.307333in,,base]{\color{textcolor}\sffamily\fontsize{12.000000}{14.400000}\selectfont Yaw controller output}%
\end{pgfscope}%
\begin{pgfscope}%
\pgfsetbuttcap%
\pgfsetmiterjoin%
\definecolor{currentfill}{rgb}{1.000000,1.000000,1.000000}%
\pgfsetfillcolor{currentfill}%
\pgfsetfillopacity{0.800000}%
\pgfsetlinewidth{1.003750pt}%
\definecolor{currentstroke}{rgb}{0.800000,0.800000,0.800000}%
\pgfsetstrokecolor{currentstroke}%
\pgfsetstrokeopacity{0.800000}%
\pgfsetdash{}{0pt}%
\pgfpathmoveto{\pgfqpoint{5.041603in}{2.889746in}}%
\pgfpathlineto{\pgfqpoint{5.662778in}{2.889746in}}%
\pgfpathquadraticcurveto{\pgfqpoint{5.690556in}{2.889746in}}{\pgfqpoint{5.690556in}{2.917523in}}%
\pgfpathlineto{\pgfqpoint{5.690556in}{4.126778in}}%
\pgfpathquadraticcurveto{\pgfqpoint{5.690556in}{4.154556in}}{\pgfqpoint{5.662778in}{4.154556in}}%
\pgfpathlineto{\pgfqpoint{5.041603in}{4.154556in}}%
\pgfpathquadraticcurveto{\pgfqpoint{5.013825in}{4.154556in}}{\pgfqpoint{5.013825in}{4.126778in}}%
\pgfpathlineto{\pgfqpoint{5.013825in}{2.917523in}}%
\pgfpathquadraticcurveto{\pgfqpoint{5.013825in}{2.889746in}}{\pgfqpoint{5.041603in}{2.889746in}}%
\pgfpathlineto{\pgfqpoint{5.041603in}{2.889746in}}%
\pgfpathclose%
\pgfusepath{stroke,fill}%
\end{pgfscope}%
\begin{pgfscope}%
\pgfsetrectcap%
\pgfsetroundjoin%
\pgfsetlinewidth{1.505625pt}%
\definecolor{currentstroke}{rgb}{0.121569,0.466667,0.705882}%
\pgfsetstrokecolor{currentstroke}%
\pgfsetdash{}{0pt}%
\pgfpathmoveto{\pgfqpoint{5.069380in}{4.042088in}}%
\pgfpathlineto{\pgfqpoint{5.208269in}{4.042088in}}%
\pgfpathlineto{\pgfqpoint{5.347158in}{4.042088in}}%
\pgfusepath{stroke}%
\end{pgfscope}%
\begin{pgfscope}%
\definecolor{textcolor}{rgb}{0.000000,0.000000,0.000000}%
\pgfsetstrokecolor{textcolor}%
\pgfsetfillcolor{textcolor}%
\pgftext[x=5.458269in,y=3.993477in,left,base]{\color{textcolor}\sffamily\fontsize{10.000000}{12.000000}\selectfont 0}%
\end{pgfscope}%
\begin{pgfscope}%
\pgfsetrectcap%
\pgfsetroundjoin%
\pgfsetlinewidth{1.505625pt}%
\definecolor{currentstroke}{rgb}{1.000000,0.498039,0.054902}%
\pgfsetstrokecolor{currentstroke}%
\pgfsetdash{}{0pt}%
\pgfpathmoveto{\pgfqpoint{5.069380in}{3.838231in}}%
\pgfpathlineto{\pgfqpoint{5.208269in}{3.838231in}}%
\pgfpathlineto{\pgfqpoint{5.347158in}{3.838231in}}%
\pgfusepath{stroke}%
\end{pgfscope}%
\begin{pgfscope}%
\definecolor{textcolor}{rgb}{0.000000,0.000000,0.000000}%
\pgfsetstrokecolor{textcolor}%
\pgfsetfillcolor{textcolor}%
\pgftext[x=5.458269in,y=3.789620in,left,base]{\color{textcolor}\sffamily\fontsize{10.000000}{12.000000}\selectfont 10}%
\end{pgfscope}%
\begin{pgfscope}%
\pgfsetrectcap%
\pgfsetroundjoin%
\pgfsetlinewidth{1.505625pt}%
\definecolor{currentstroke}{rgb}{0.172549,0.627451,0.172549}%
\pgfsetstrokecolor{currentstroke}%
\pgfsetdash{}{0pt}%
\pgfpathmoveto{\pgfqpoint{5.069380in}{3.634374in}}%
\pgfpathlineto{\pgfqpoint{5.208269in}{3.634374in}}%
\pgfpathlineto{\pgfqpoint{5.347158in}{3.634374in}}%
\pgfusepath{stroke}%
\end{pgfscope}%
\begin{pgfscope}%
\definecolor{textcolor}{rgb}{0.000000,0.000000,0.000000}%
\pgfsetstrokecolor{textcolor}%
\pgfsetfillcolor{textcolor}%
\pgftext[x=5.458269in,y=3.585762in,left,base]{\color{textcolor}\sffamily\fontsize{10.000000}{12.000000}\selectfont 20}%
\end{pgfscope}%
\begin{pgfscope}%
\pgfsetrectcap%
\pgfsetroundjoin%
\pgfsetlinewidth{1.505625pt}%
\definecolor{currentstroke}{rgb}{0.839216,0.152941,0.156863}%
\pgfsetstrokecolor{currentstroke}%
\pgfsetdash{}{0pt}%
\pgfpathmoveto{\pgfqpoint{5.069380in}{3.430516in}}%
\pgfpathlineto{\pgfqpoint{5.208269in}{3.430516in}}%
\pgfpathlineto{\pgfqpoint{5.347158in}{3.430516in}}%
\pgfusepath{stroke}%
\end{pgfscope}%
\begin{pgfscope}%
\definecolor{textcolor}{rgb}{0.000000,0.000000,0.000000}%
\pgfsetstrokecolor{textcolor}%
\pgfsetfillcolor{textcolor}%
\pgftext[x=5.458269in,y=3.381905in,left,base]{\color{textcolor}\sffamily\fontsize{10.000000}{12.000000}\selectfont 30}%
\end{pgfscope}%
\begin{pgfscope}%
\pgfsetrectcap%
\pgfsetroundjoin%
\pgfsetlinewidth{1.505625pt}%
\definecolor{currentstroke}{rgb}{0.580392,0.403922,0.741176}%
\pgfsetstrokecolor{currentstroke}%
\pgfsetdash{}{0pt}%
\pgfpathmoveto{\pgfqpoint{5.069380in}{3.226659in}}%
\pgfpathlineto{\pgfqpoint{5.208269in}{3.226659in}}%
\pgfpathlineto{\pgfqpoint{5.347158in}{3.226659in}}%
\pgfusepath{stroke}%
\end{pgfscope}%
\begin{pgfscope}%
\definecolor{textcolor}{rgb}{0.000000,0.000000,0.000000}%
\pgfsetstrokecolor{textcolor}%
\pgfsetfillcolor{textcolor}%
\pgftext[x=5.458269in,y=3.178048in,left,base]{\color{textcolor}\sffamily\fontsize{10.000000}{12.000000}\selectfont 40}%
\end{pgfscope}%
\begin{pgfscope}%
\pgfsetrectcap%
\pgfsetroundjoin%
\pgfsetlinewidth{1.505625pt}%
\definecolor{currentstroke}{rgb}{0.549020,0.337255,0.294118}%
\pgfsetstrokecolor{currentstroke}%
\pgfsetdash{}{0pt}%
\pgfpathmoveto{\pgfqpoint{5.069380in}{3.022802in}}%
\pgfpathlineto{\pgfqpoint{5.208269in}{3.022802in}}%
\pgfpathlineto{\pgfqpoint{5.347158in}{3.022802in}}%
\pgfusepath{stroke}%
\end{pgfscope}%
\begin{pgfscope}%
\definecolor{textcolor}{rgb}{0.000000,0.000000,0.000000}%
\pgfsetstrokecolor{textcolor}%
\pgfsetfillcolor{textcolor}%
\pgftext[x=5.458269in,y=2.974191in,left,base]{\color{textcolor}\sffamily\fontsize{10.000000}{12.000000}\selectfont 50}%
\end{pgfscope}%
\end{pgfpicture}%
\makeatother%
\endgroup%
}
    \end{minipage}
    \caption{Variation of (a) computed error and (b) output velocity for different values of $K_{I}$ and $K_P=100$, $K_D=0$ while the yaw controller is engaged.}
    \label{fig:tune-yaw-int-io}
\end{figure}
\begin{figure}[H]
    \begin{minipage}[t]{0.5\linewidth}
        \centering
        \scalebox{0.55}{%% Creator: Matplotlib, PGF backend
%%
%% To include the figure in your LaTeX document, write
%%   \input{<filename>.pgf}
%%
%% Make sure the required packages are loaded in your preamble
%%   \usepackage{pgf}
%%
%% Also ensure that all the required font packages are loaded; for instance,
%% the lmodern package is sometimes necessary when using math font.
%%   \usepackage{lmodern}
%%
%% Figures using additional raster images can only be included by \input if
%% they are in the same directory as the main LaTeX file. For loading figures
%% from other directories you can use the `import` package
%%   \usepackage{import}
%%
%% and then include the figures with
%%   \import{<path to file>}{<filename>.pgf}
%%
%% Matplotlib used the following preamble
%%   \usepackage{fontspec}
%%   \setmainfont{DejaVuSerif.ttf}[Path=\detokenize{/home/lgonz/tfg-aero/tfg-giaa-dronecontrol/venv/lib/python3.8/site-packages/matplotlib/mpl-data/fonts/ttf/}]
%%   \setsansfont{DejaVuSans.ttf}[Path=\detokenize{/home/lgonz/tfg-aero/tfg-giaa-dronecontrol/venv/lib/python3.8/site-packages/matplotlib/mpl-data/fonts/ttf/}]
%%   \setmonofont{DejaVuSansMono.ttf}[Path=\detokenize{/home/lgonz/tfg-aero/tfg-giaa-dronecontrol/venv/lib/python3.8/site-packages/matplotlib/mpl-data/fonts/ttf/}]
%%
\begingroup%
\makeatletter%
\begin{pgfpicture}%
\pgfpathrectangle{\pgfpointorigin}{\pgfqpoint{6.400000in}{4.800000in}}%
\pgfusepath{use as bounding box, clip}%
\begin{pgfscope}%
\pgfsetbuttcap%
\pgfsetmiterjoin%
\definecolor{currentfill}{rgb}{1.000000,1.000000,1.000000}%
\pgfsetfillcolor{currentfill}%
\pgfsetlinewidth{0.000000pt}%
\definecolor{currentstroke}{rgb}{1.000000,1.000000,1.000000}%
\pgfsetstrokecolor{currentstroke}%
\pgfsetdash{}{0pt}%
\pgfpathmoveto{\pgfqpoint{0.000000in}{0.000000in}}%
\pgfpathlineto{\pgfqpoint{6.400000in}{0.000000in}}%
\pgfpathlineto{\pgfqpoint{6.400000in}{4.800000in}}%
\pgfpathlineto{\pgfqpoint{0.000000in}{4.800000in}}%
\pgfpathlineto{\pgfqpoint{0.000000in}{0.000000in}}%
\pgfpathclose%
\pgfusepath{fill}%
\end{pgfscope}%
\begin{pgfscope}%
\pgfsetbuttcap%
\pgfsetmiterjoin%
\definecolor{currentfill}{rgb}{1.000000,1.000000,1.000000}%
\pgfsetfillcolor{currentfill}%
\pgfsetlinewidth{0.000000pt}%
\definecolor{currentstroke}{rgb}{0.000000,0.000000,0.000000}%
\pgfsetstrokecolor{currentstroke}%
\pgfsetstrokeopacity{0.000000}%
\pgfsetdash{}{0pt}%
\pgfpathmoveto{\pgfqpoint{0.800000in}{0.528000in}}%
\pgfpathlineto{\pgfqpoint{5.760000in}{0.528000in}}%
\pgfpathlineto{\pgfqpoint{5.760000in}{4.224000in}}%
\pgfpathlineto{\pgfqpoint{0.800000in}{4.224000in}}%
\pgfpathlineto{\pgfqpoint{0.800000in}{0.528000in}}%
\pgfpathclose%
\pgfusepath{fill}%
\end{pgfscope}%
\begin{pgfscope}%
\pgfpathrectangle{\pgfqpoint{0.800000in}{0.528000in}}{\pgfqpoint{4.960000in}{3.696000in}}%
\pgfusepath{clip}%
\pgfsetrectcap%
\pgfsetroundjoin%
\pgfsetlinewidth{0.803000pt}%
\definecolor{currentstroke}{rgb}{0.690196,0.690196,0.690196}%
\pgfsetstrokecolor{currentstroke}%
\pgfsetdash{}{0pt}%
\pgfpathmoveto{\pgfqpoint{1.025455in}{0.528000in}}%
\pgfpathlineto{\pgfqpoint{1.025455in}{4.224000in}}%
\pgfusepath{stroke}%
\end{pgfscope}%
\begin{pgfscope}%
\pgfsetbuttcap%
\pgfsetroundjoin%
\definecolor{currentfill}{rgb}{0.000000,0.000000,0.000000}%
\pgfsetfillcolor{currentfill}%
\pgfsetlinewidth{0.803000pt}%
\definecolor{currentstroke}{rgb}{0.000000,0.000000,0.000000}%
\pgfsetstrokecolor{currentstroke}%
\pgfsetdash{}{0pt}%
\pgfsys@defobject{currentmarker}{\pgfqpoint{0.000000in}{-0.048611in}}{\pgfqpoint{0.000000in}{0.000000in}}{%
\pgfpathmoveto{\pgfqpoint{0.000000in}{0.000000in}}%
\pgfpathlineto{\pgfqpoint{0.000000in}{-0.048611in}}%
\pgfusepath{stroke,fill}%
}%
\begin{pgfscope}%
\pgfsys@transformshift{1.025455in}{0.528000in}%
\pgfsys@useobject{currentmarker}{}%
\end{pgfscope}%
\end{pgfscope}%
\begin{pgfscope}%
\definecolor{textcolor}{rgb}{0.000000,0.000000,0.000000}%
\pgfsetstrokecolor{textcolor}%
\pgfsetfillcolor{textcolor}%
\pgftext[x=1.025455in,y=0.430778in,,top]{\color{textcolor}\sffamily\fontsize{10.000000}{12.000000}\selectfont 0}%
\end{pgfscope}%
\begin{pgfscope}%
\pgfpathrectangle{\pgfqpoint{0.800000in}{0.528000in}}{\pgfqpoint{4.960000in}{3.696000in}}%
\pgfusepath{clip}%
\pgfsetrectcap%
\pgfsetroundjoin%
\pgfsetlinewidth{0.803000pt}%
\definecolor{currentstroke}{rgb}{0.690196,0.690196,0.690196}%
\pgfsetstrokecolor{currentstroke}%
\pgfsetdash{}{0pt}%
\pgfpathmoveto{\pgfqpoint{1.775877in}{0.528000in}}%
\pgfpathlineto{\pgfqpoint{1.775877in}{4.224000in}}%
\pgfusepath{stroke}%
\end{pgfscope}%
\begin{pgfscope}%
\pgfsetbuttcap%
\pgfsetroundjoin%
\definecolor{currentfill}{rgb}{0.000000,0.000000,0.000000}%
\pgfsetfillcolor{currentfill}%
\pgfsetlinewidth{0.803000pt}%
\definecolor{currentstroke}{rgb}{0.000000,0.000000,0.000000}%
\pgfsetstrokecolor{currentstroke}%
\pgfsetdash{}{0pt}%
\pgfsys@defobject{currentmarker}{\pgfqpoint{0.000000in}{-0.048611in}}{\pgfqpoint{0.000000in}{0.000000in}}{%
\pgfpathmoveto{\pgfqpoint{0.000000in}{0.000000in}}%
\pgfpathlineto{\pgfqpoint{0.000000in}{-0.048611in}}%
\pgfusepath{stroke,fill}%
}%
\begin{pgfscope}%
\pgfsys@transformshift{1.775877in}{0.528000in}%
\pgfsys@useobject{currentmarker}{}%
\end{pgfscope}%
\end{pgfscope}%
\begin{pgfscope}%
\definecolor{textcolor}{rgb}{0.000000,0.000000,0.000000}%
\pgfsetstrokecolor{textcolor}%
\pgfsetfillcolor{textcolor}%
\pgftext[x=1.775877in,y=0.430778in,,top]{\color{textcolor}\sffamily\fontsize{10.000000}{12.000000}\selectfont 5}%
\end{pgfscope}%
\begin{pgfscope}%
\pgfpathrectangle{\pgfqpoint{0.800000in}{0.528000in}}{\pgfqpoint{4.960000in}{3.696000in}}%
\pgfusepath{clip}%
\pgfsetrectcap%
\pgfsetroundjoin%
\pgfsetlinewidth{0.803000pt}%
\definecolor{currentstroke}{rgb}{0.690196,0.690196,0.690196}%
\pgfsetstrokecolor{currentstroke}%
\pgfsetdash{}{0pt}%
\pgfpathmoveto{\pgfqpoint{2.526299in}{0.528000in}}%
\pgfpathlineto{\pgfqpoint{2.526299in}{4.224000in}}%
\pgfusepath{stroke}%
\end{pgfscope}%
\begin{pgfscope}%
\pgfsetbuttcap%
\pgfsetroundjoin%
\definecolor{currentfill}{rgb}{0.000000,0.000000,0.000000}%
\pgfsetfillcolor{currentfill}%
\pgfsetlinewidth{0.803000pt}%
\definecolor{currentstroke}{rgb}{0.000000,0.000000,0.000000}%
\pgfsetstrokecolor{currentstroke}%
\pgfsetdash{}{0pt}%
\pgfsys@defobject{currentmarker}{\pgfqpoint{0.000000in}{-0.048611in}}{\pgfqpoint{0.000000in}{0.000000in}}{%
\pgfpathmoveto{\pgfqpoint{0.000000in}{0.000000in}}%
\pgfpathlineto{\pgfqpoint{0.000000in}{-0.048611in}}%
\pgfusepath{stroke,fill}%
}%
\begin{pgfscope}%
\pgfsys@transformshift{2.526299in}{0.528000in}%
\pgfsys@useobject{currentmarker}{}%
\end{pgfscope}%
\end{pgfscope}%
\begin{pgfscope}%
\definecolor{textcolor}{rgb}{0.000000,0.000000,0.000000}%
\pgfsetstrokecolor{textcolor}%
\pgfsetfillcolor{textcolor}%
\pgftext[x=2.526299in,y=0.430778in,,top]{\color{textcolor}\sffamily\fontsize{10.000000}{12.000000}\selectfont 10}%
\end{pgfscope}%
\begin{pgfscope}%
\pgfpathrectangle{\pgfqpoint{0.800000in}{0.528000in}}{\pgfqpoint{4.960000in}{3.696000in}}%
\pgfusepath{clip}%
\pgfsetrectcap%
\pgfsetroundjoin%
\pgfsetlinewidth{0.803000pt}%
\definecolor{currentstroke}{rgb}{0.690196,0.690196,0.690196}%
\pgfsetstrokecolor{currentstroke}%
\pgfsetdash{}{0pt}%
\pgfpathmoveto{\pgfqpoint{3.276722in}{0.528000in}}%
\pgfpathlineto{\pgfqpoint{3.276722in}{4.224000in}}%
\pgfusepath{stroke}%
\end{pgfscope}%
\begin{pgfscope}%
\pgfsetbuttcap%
\pgfsetroundjoin%
\definecolor{currentfill}{rgb}{0.000000,0.000000,0.000000}%
\pgfsetfillcolor{currentfill}%
\pgfsetlinewidth{0.803000pt}%
\definecolor{currentstroke}{rgb}{0.000000,0.000000,0.000000}%
\pgfsetstrokecolor{currentstroke}%
\pgfsetdash{}{0pt}%
\pgfsys@defobject{currentmarker}{\pgfqpoint{0.000000in}{-0.048611in}}{\pgfqpoint{0.000000in}{0.000000in}}{%
\pgfpathmoveto{\pgfqpoint{0.000000in}{0.000000in}}%
\pgfpathlineto{\pgfqpoint{0.000000in}{-0.048611in}}%
\pgfusepath{stroke,fill}%
}%
\begin{pgfscope}%
\pgfsys@transformshift{3.276722in}{0.528000in}%
\pgfsys@useobject{currentmarker}{}%
\end{pgfscope}%
\end{pgfscope}%
\begin{pgfscope}%
\definecolor{textcolor}{rgb}{0.000000,0.000000,0.000000}%
\pgfsetstrokecolor{textcolor}%
\pgfsetfillcolor{textcolor}%
\pgftext[x=3.276722in,y=0.430778in,,top]{\color{textcolor}\sffamily\fontsize{10.000000}{12.000000}\selectfont 15}%
\end{pgfscope}%
\begin{pgfscope}%
\pgfpathrectangle{\pgfqpoint{0.800000in}{0.528000in}}{\pgfqpoint{4.960000in}{3.696000in}}%
\pgfusepath{clip}%
\pgfsetrectcap%
\pgfsetroundjoin%
\pgfsetlinewidth{0.803000pt}%
\definecolor{currentstroke}{rgb}{0.690196,0.690196,0.690196}%
\pgfsetstrokecolor{currentstroke}%
\pgfsetdash{}{0pt}%
\pgfpathmoveto{\pgfqpoint{4.027144in}{0.528000in}}%
\pgfpathlineto{\pgfqpoint{4.027144in}{4.224000in}}%
\pgfusepath{stroke}%
\end{pgfscope}%
\begin{pgfscope}%
\pgfsetbuttcap%
\pgfsetroundjoin%
\definecolor{currentfill}{rgb}{0.000000,0.000000,0.000000}%
\pgfsetfillcolor{currentfill}%
\pgfsetlinewidth{0.803000pt}%
\definecolor{currentstroke}{rgb}{0.000000,0.000000,0.000000}%
\pgfsetstrokecolor{currentstroke}%
\pgfsetdash{}{0pt}%
\pgfsys@defobject{currentmarker}{\pgfqpoint{0.000000in}{-0.048611in}}{\pgfqpoint{0.000000in}{0.000000in}}{%
\pgfpathmoveto{\pgfqpoint{0.000000in}{0.000000in}}%
\pgfpathlineto{\pgfqpoint{0.000000in}{-0.048611in}}%
\pgfusepath{stroke,fill}%
}%
\begin{pgfscope}%
\pgfsys@transformshift{4.027144in}{0.528000in}%
\pgfsys@useobject{currentmarker}{}%
\end{pgfscope}%
\end{pgfscope}%
\begin{pgfscope}%
\definecolor{textcolor}{rgb}{0.000000,0.000000,0.000000}%
\pgfsetstrokecolor{textcolor}%
\pgfsetfillcolor{textcolor}%
\pgftext[x=4.027144in,y=0.430778in,,top]{\color{textcolor}\sffamily\fontsize{10.000000}{12.000000}\selectfont 20}%
\end{pgfscope}%
\begin{pgfscope}%
\pgfpathrectangle{\pgfqpoint{0.800000in}{0.528000in}}{\pgfqpoint{4.960000in}{3.696000in}}%
\pgfusepath{clip}%
\pgfsetrectcap%
\pgfsetroundjoin%
\pgfsetlinewidth{0.803000pt}%
\definecolor{currentstroke}{rgb}{0.690196,0.690196,0.690196}%
\pgfsetstrokecolor{currentstroke}%
\pgfsetdash{}{0pt}%
\pgfpathmoveto{\pgfqpoint{4.777566in}{0.528000in}}%
\pgfpathlineto{\pgfqpoint{4.777566in}{4.224000in}}%
\pgfusepath{stroke}%
\end{pgfscope}%
\begin{pgfscope}%
\pgfsetbuttcap%
\pgfsetroundjoin%
\definecolor{currentfill}{rgb}{0.000000,0.000000,0.000000}%
\pgfsetfillcolor{currentfill}%
\pgfsetlinewidth{0.803000pt}%
\definecolor{currentstroke}{rgb}{0.000000,0.000000,0.000000}%
\pgfsetstrokecolor{currentstroke}%
\pgfsetdash{}{0pt}%
\pgfsys@defobject{currentmarker}{\pgfqpoint{0.000000in}{-0.048611in}}{\pgfqpoint{0.000000in}{0.000000in}}{%
\pgfpathmoveto{\pgfqpoint{0.000000in}{0.000000in}}%
\pgfpathlineto{\pgfqpoint{0.000000in}{-0.048611in}}%
\pgfusepath{stroke,fill}%
}%
\begin{pgfscope}%
\pgfsys@transformshift{4.777566in}{0.528000in}%
\pgfsys@useobject{currentmarker}{}%
\end{pgfscope}%
\end{pgfscope}%
\begin{pgfscope}%
\definecolor{textcolor}{rgb}{0.000000,0.000000,0.000000}%
\pgfsetstrokecolor{textcolor}%
\pgfsetfillcolor{textcolor}%
\pgftext[x=4.777566in,y=0.430778in,,top]{\color{textcolor}\sffamily\fontsize{10.000000}{12.000000}\selectfont 25}%
\end{pgfscope}%
\begin{pgfscope}%
\pgfpathrectangle{\pgfqpoint{0.800000in}{0.528000in}}{\pgfqpoint{4.960000in}{3.696000in}}%
\pgfusepath{clip}%
\pgfsetrectcap%
\pgfsetroundjoin%
\pgfsetlinewidth{0.803000pt}%
\definecolor{currentstroke}{rgb}{0.690196,0.690196,0.690196}%
\pgfsetstrokecolor{currentstroke}%
\pgfsetdash{}{0pt}%
\pgfpathmoveto{\pgfqpoint{5.527989in}{0.528000in}}%
\pgfpathlineto{\pgfqpoint{5.527989in}{4.224000in}}%
\pgfusepath{stroke}%
\end{pgfscope}%
\begin{pgfscope}%
\pgfsetbuttcap%
\pgfsetroundjoin%
\definecolor{currentfill}{rgb}{0.000000,0.000000,0.000000}%
\pgfsetfillcolor{currentfill}%
\pgfsetlinewidth{0.803000pt}%
\definecolor{currentstroke}{rgb}{0.000000,0.000000,0.000000}%
\pgfsetstrokecolor{currentstroke}%
\pgfsetdash{}{0pt}%
\pgfsys@defobject{currentmarker}{\pgfqpoint{0.000000in}{-0.048611in}}{\pgfqpoint{0.000000in}{0.000000in}}{%
\pgfpathmoveto{\pgfqpoint{0.000000in}{0.000000in}}%
\pgfpathlineto{\pgfqpoint{0.000000in}{-0.048611in}}%
\pgfusepath{stroke,fill}%
}%
\begin{pgfscope}%
\pgfsys@transformshift{5.527989in}{0.528000in}%
\pgfsys@useobject{currentmarker}{}%
\end{pgfscope}%
\end{pgfscope}%
\begin{pgfscope}%
\definecolor{textcolor}{rgb}{0.000000,0.000000,0.000000}%
\pgfsetstrokecolor{textcolor}%
\pgfsetfillcolor{textcolor}%
\pgftext[x=5.527989in,y=0.430778in,,top]{\color{textcolor}\sffamily\fontsize{10.000000}{12.000000}\selectfont 30}%
\end{pgfscope}%
\begin{pgfscope}%
\definecolor{textcolor}{rgb}{0.000000,0.000000,0.000000}%
\pgfsetstrokecolor{textcolor}%
\pgfsetfillcolor{textcolor}%
\pgftext[x=3.280000in,y=0.240809in,,top]{\color{textcolor}\sffamily\fontsize{10.000000}{12.000000}\selectfont time [s]}%
\end{pgfscope}%
\begin{pgfscope}%
\pgfpathrectangle{\pgfqpoint{0.800000in}{0.528000in}}{\pgfqpoint{4.960000in}{3.696000in}}%
\pgfusepath{clip}%
\pgfsetrectcap%
\pgfsetroundjoin%
\pgfsetlinewidth{0.803000pt}%
\definecolor{currentstroke}{rgb}{0.690196,0.690196,0.690196}%
\pgfsetstrokecolor{currentstroke}%
\pgfsetdash{}{0pt}%
\pgfpathmoveto{\pgfqpoint{0.800000in}{0.789999in}}%
\pgfpathlineto{\pgfqpoint{5.760000in}{0.789999in}}%
\pgfusepath{stroke}%
\end{pgfscope}%
\begin{pgfscope}%
\pgfsetbuttcap%
\pgfsetroundjoin%
\definecolor{currentfill}{rgb}{0.000000,0.000000,0.000000}%
\pgfsetfillcolor{currentfill}%
\pgfsetlinewidth{0.803000pt}%
\definecolor{currentstroke}{rgb}{0.000000,0.000000,0.000000}%
\pgfsetstrokecolor{currentstroke}%
\pgfsetdash{}{0pt}%
\pgfsys@defobject{currentmarker}{\pgfqpoint{-0.048611in}{0.000000in}}{\pgfqpoint{-0.000000in}{0.000000in}}{%
\pgfpathmoveto{\pgfqpoint{-0.000000in}{0.000000in}}%
\pgfpathlineto{\pgfqpoint{-0.048611in}{0.000000in}}%
\pgfusepath{stroke,fill}%
}%
\begin{pgfscope}%
\pgfsys@transformshift{0.800000in}{0.789999in}%
\pgfsys@useobject{currentmarker}{}%
\end{pgfscope}%
\end{pgfscope}%
\begin{pgfscope}%
\definecolor{textcolor}{rgb}{0.000000,0.000000,0.000000}%
\pgfsetstrokecolor{textcolor}%
\pgfsetfillcolor{textcolor}%
\pgftext[x=0.481898in, y=0.737237in, left, base]{\color{textcolor}\sffamily\fontsize{10.000000}{12.000000}\selectfont 7.5}%
\end{pgfscope}%
\begin{pgfscope}%
\pgfpathrectangle{\pgfqpoint{0.800000in}{0.528000in}}{\pgfqpoint{4.960000in}{3.696000in}}%
\pgfusepath{clip}%
\pgfsetrectcap%
\pgfsetroundjoin%
\pgfsetlinewidth{0.803000pt}%
\definecolor{currentstroke}{rgb}{0.690196,0.690196,0.690196}%
\pgfsetstrokecolor{currentstroke}%
\pgfsetdash{}{0pt}%
\pgfpathmoveto{\pgfqpoint{0.800000in}{1.250778in}}%
\pgfpathlineto{\pgfqpoint{5.760000in}{1.250778in}}%
\pgfusepath{stroke}%
\end{pgfscope}%
\begin{pgfscope}%
\pgfsetbuttcap%
\pgfsetroundjoin%
\definecolor{currentfill}{rgb}{0.000000,0.000000,0.000000}%
\pgfsetfillcolor{currentfill}%
\pgfsetlinewidth{0.803000pt}%
\definecolor{currentstroke}{rgb}{0.000000,0.000000,0.000000}%
\pgfsetstrokecolor{currentstroke}%
\pgfsetdash{}{0pt}%
\pgfsys@defobject{currentmarker}{\pgfqpoint{-0.048611in}{0.000000in}}{\pgfqpoint{-0.000000in}{0.000000in}}{%
\pgfpathmoveto{\pgfqpoint{-0.000000in}{0.000000in}}%
\pgfpathlineto{\pgfqpoint{-0.048611in}{0.000000in}}%
\pgfusepath{stroke,fill}%
}%
\begin{pgfscope}%
\pgfsys@transformshift{0.800000in}{1.250778in}%
\pgfsys@useobject{currentmarker}{}%
\end{pgfscope}%
\end{pgfscope}%
\begin{pgfscope}%
\definecolor{textcolor}{rgb}{0.000000,0.000000,0.000000}%
\pgfsetstrokecolor{textcolor}%
\pgfsetfillcolor{textcolor}%
\pgftext[x=0.393533in, y=1.198016in, left, base]{\color{textcolor}\sffamily\fontsize{10.000000}{12.000000}\selectfont 10.0}%
\end{pgfscope}%
\begin{pgfscope}%
\pgfpathrectangle{\pgfqpoint{0.800000in}{0.528000in}}{\pgfqpoint{4.960000in}{3.696000in}}%
\pgfusepath{clip}%
\pgfsetrectcap%
\pgfsetroundjoin%
\pgfsetlinewidth{0.803000pt}%
\definecolor{currentstroke}{rgb}{0.690196,0.690196,0.690196}%
\pgfsetstrokecolor{currentstroke}%
\pgfsetdash{}{0pt}%
\pgfpathmoveto{\pgfqpoint{0.800000in}{1.711557in}}%
\pgfpathlineto{\pgfqpoint{5.760000in}{1.711557in}}%
\pgfusepath{stroke}%
\end{pgfscope}%
\begin{pgfscope}%
\pgfsetbuttcap%
\pgfsetroundjoin%
\definecolor{currentfill}{rgb}{0.000000,0.000000,0.000000}%
\pgfsetfillcolor{currentfill}%
\pgfsetlinewidth{0.803000pt}%
\definecolor{currentstroke}{rgb}{0.000000,0.000000,0.000000}%
\pgfsetstrokecolor{currentstroke}%
\pgfsetdash{}{0pt}%
\pgfsys@defobject{currentmarker}{\pgfqpoint{-0.048611in}{0.000000in}}{\pgfqpoint{-0.000000in}{0.000000in}}{%
\pgfpathmoveto{\pgfqpoint{-0.000000in}{0.000000in}}%
\pgfpathlineto{\pgfqpoint{-0.048611in}{0.000000in}}%
\pgfusepath{stroke,fill}%
}%
\begin{pgfscope}%
\pgfsys@transformshift{0.800000in}{1.711557in}%
\pgfsys@useobject{currentmarker}{}%
\end{pgfscope}%
\end{pgfscope}%
\begin{pgfscope}%
\definecolor{textcolor}{rgb}{0.000000,0.000000,0.000000}%
\pgfsetstrokecolor{textcolor}%
\pgfsetfillcolor{textcolor}%
\pgftext[x=0.393533in, y=1.658795in, left, base]{\color{textcolor}\sffamily\fontsize{10.000000}{12.000000}\selectfont 12.5}%
\end{pgfscope}%
\begin{pgfscope}%
\pgfpathrectangle{\pgfqpoint{0.800000in}{0.528000in}}{\pgfqpoint{4.960000in}{3.696000in}}%
\pgfusepath{clip}%
\pgfsetrectcap%
\pgfsetroundjoin%
\pgfsetlinewidth{0.803000pt}%
\definecolor{currentstroke}{rgb}{0.690196,0.690196,0.690196}%
\pgfsetstrokecolor{currentstroke}%
\pgfsetdash{}{0pt}%
\pgfpathmoveto{\pgfqpoint{0.800000in}{2.172336in}}%
\pgfpathlineto{\pgfqpoint{5.760000in}{2.172336in}}%
\pgfusepath{stroke}%
\end{pgfscope}%
\begin{pgfscope}%
\pgfsetbuttcap%
\pgfsetroundjoin%
\definecolor{currentfill}{rgb}{0.000000,0.000000,0.000000}%
\pgfsetfillcolor{currentfill}%
\pgfsetlinewidth{0.803000pt}%
\definecolor{currentstroke}{rgb}{0.000000,0.000000,0.000000}%
\pgfsetstrokecolor{currentstroke}%
\pgfsetdash{}{0pt}%
\pgfsys@defobject{currentmarker}{\pgfqpoint{-0.048611in}{0.000000in}}{\pgfqpoint{-0.000000in}{0.000000in}}{%
\pgfpathmoveto{\pgfqpoint{-0.000000in}{0.000000in}}%
\pgfpathlineto{\pgfqpoint{-0.048611in}{0.000000in}}%
\pgfusepath{stroke,fill}%
}%
\begin{pgfscope}%
\pgfsys@transformshift{0.800000in}{2.172336in}%
\pgfsys@useobject{currentmarker}{}%
\end{pgfscope}%
\end{pgfscope}%
\begin{pgfscope}%
\definecolor{textcolor}{rgb}{0.000000,0.000000,0.000000}%
\pgfsetstrokecolor{textcolor}%
\pgfsetfillcolor{textcolor}%
\pgftext[x=0.393533in, y=2.119574in, left, base]{\color{textcolor}\sffamily\fontsize{10.000000}{12.000000}\selectfont 15.0}%
\end{pgfscope}%
\begin{pgfscope}%
\pgfpathrectangle{\pgfqpoint{0.800000in}{0.528000in}}{\pgfqpoint{4.960000in}{3.696000in}}%
\pgfusepath{clip}%
\pgfsetrectcap%
\pgfsetroundjoin%
\pgfsetlinewidth{0.803000pt}%
\definecolor{currentstroke}{rgb}{0.690196,0.690196,0.690196}%
\pgfsetstrokecolor{currentstroke}%
\pgfsetdash{}{0pt}%
\pgfpathmoveto{\pgfqpoint{0.800000in}{2.633115in}}%
\pgfpathlineto{\pgfqpoint{5.760000in}{2.633115in}}%
\pgfusepath{stroke}%
\end{pgfscope}%
\begin{pgfscope}%
\pgfsetbuttcap%
\pgfsetroundjoin%
\definecolor{currentfill}{rgb}{0.000000,0.000000,0.000000}%
\pgfsetfillcolor{currentfill}%
\pgfsetlinewidth{0.803000pt}%
\definecolor{currentstroke}{rgb}{0.000000,0.000000,0.000000}%
\pgfsetstrokecolor{currentstroke}%
\pgfsetdash{}{0pt}%
\pgfsys@defobject{currentmarker}{\pgfqpoint{-0.048611in}{0.000000in}}{\pgfqpoint{-0.000000in}{0.000000in}}{%
\pgfpathmoveto{\pgfqpoint{-0.000000in}{0.000000in}}%
\pgfpathlineto{\pgfqpoint{-0.048611in}{0.000000in}}%
\pgfusepath{stroke,fill}%
}%
\begin{pgfscope}%
\pgfsys@transformshift{0.800000in}{2.633115in}%
\pgfsys@useobject{currentmarker}{}%
\end{pgfscope}%
\end{pgfscope}%
\begin{pgfscope}%
\definecolor{textcolor}{rgb}{0.000000,0.000000,0.000000}%
\pgfsetstrokecolor{textcolor}%
\pgfsetfillcolor{textcolor}%
\pgftext[x=0.393533in, y=2.580353in, left, base]{\color{textcolor}\sffamily\fontsize{10.000000}{12.000000}\selectfont 17.5}%
\end{pgfscope}%
\begin{pgfscope}%
\pgfpathrectangle{\pgfqpoint{0.800000in}{0.528000in}}{\pgfqpoint{4.960000in}{3.696000in}}%
\pgfusepath{clip}%
\pgfsetrectcap%
\pgfsetroundjoin%
\pgfsetlinewidth{0.803000pt}%
\definecolor{currentstroke}{rgb}{0.690196,0.690196,0.690196}%
\pgfsetstrokecolor{currentstroke}%
\pgfsetdash{}{0pt}%
\pgfpathmoveto{\pgfqpoint{0.800000in}{3.093894in}}%
\pgfpathlineto{\pgfqpoint{5.760000in}{3.093894in}}%
\pgfusepath{stroke}%
\end{pgfscope}%
\begin{pgfscope}%
\pgfsetbuttcap%
\pgfsetroundjoin%
\definecolor{currentfill}{rgb}{0.000000,0.000000,0.000000}%
\pgfsetfillcolor{currentfill}%
\pgfsetlinewidth{0.803000pt}%
\definecolor{currentstroke}{rgb}{0.000000,0.000000,0.000000}%
\pgfsetstrokecolor{currentstroke}%
\pgfsetdash{}{0pt}%
\pgfsys@defobject{currentmarker}{\pgfqpoint{-0.048611in}{0.000000in}}{\pgfqpoint{-0.000000in}{0.000000in}}{%
\pgfpathmoveto{\pgfqpoint{-0.000000in}{0.000000in}}%
\pgfpathlineto{\pgfqpoint{-0.048611in}{0.000000in}}%
\pgfusepath{stroke,fill}%
}%
\begin{pgfscope}%
\pgfsys@transformshift{0.800000in}{3.093894in}%
\pgfsys@useobject{currentmarker}{}%
\end{pgfscope}%
\end{pgfscope}%
\begin{pgfscope}%
\definecolor{textcolor}{rgb}{0.000000,0.000000,0.000000}%
\pgfsetstrokecolor{textcolor}%
\pgfsetfillcolor{textcolor}%
\pgftext[x=0.393533in, y=3.041132in, left, base]{\color{textcolor}\sffamily\fontsize{10.000000}{12.000000}\selectfont 20.0}%
\end{pgfscope}%
\begin{pgfscope}%
\pgfpathrectangle{\pgfqpoint{0.800000in}{0.528000in}}{\pgfqpoint{4.960000in}{3.696000in}}%
\pgfusepath{clip}%
\pgfsetrectcap%
\pgfsetroundjoin%
\pgfsetlinewidth{0.803000pt}%
\definecolor{currentstroke}{rgb}{0.690196,0.690196,0.690196}%
\pgfsetstrokecolor{currentstroke}%
\pgfsetdash{}{0pt}%
\pgfpathmoveto{\pgfqpoint{0.800000in}{3.554673in}}%
\pgfpathlineto{\pgfqpoint{5.760000in}{3.554673in}}%
\pgfusepath{stroke}%
\end{pgfscope}%
\begin{pgfscope}%
\pgfsetbuttcap%
\pgfsetroundjoin%
\definecolor{currentfill}{rgb}{0.000000,0.000000,0.000000}%
\pgfsetfillcolor{currentfill}%
\pgfsetlinewidth{0.803000pt}%
\definecolor{currentstroke}{rgb}{0.000000,0.000000,0.000000}%
\pgfsetstrokecolor{currentstroke}%
\pgfsetdash{}{0pt}%
\pgfsys@defobject{currentmarker}{\pgfqpoint{-0.048611in}{0.000000in}}{\pgfqpoint{-0.000000in}{0.000000in}}{%
\pgfpathmoveto{\pgfqpoint{-0.000000in}{0.000000in}}%
\pgfpathlineto{\pgfqpoint{-0.048611in}{0.000000in}}%
\pgfusepath{stroke,fill}%
}%
\begin{pgfscope}%
\pgfsys@transformshift{0.800000in}{3.554673in}%
\pgfsys@useobject{currentmarker}{}%
\end{pgfscope}%
\end{pgfscope}%
\begin{pgfscope}%
\definecolor{textcolor}{rgb}{0.000000,0.000000,0.000000}%
\pgfsetstrokecolor{textcolor}%
\pgfsetfillcolor{textcolor}%
\pgftext[x=0.393533in, y=3.501911in, left, base]{\color{textcolor}\sffamily\fontsize{10.000000}{12.000000}\selectfont 22.5}%
\end{pgfscope}%
\begin{pgfscope}%
\pgfpathrectangle{\pgfqpoint{0.800000in}{0.528000in}}{\pgfqpoint{4.960000in}{3.696000in}}%
\pgfusepath{clip}%
\pgfsetrectcap%
\pgfsetroundjoin%
\pgfsetlinewidth{0.803000pt}%
\definecolor{currentstroke}{rgb}{0.690196,0.690196,0.690196}%
\pgfsetstrokecolor{currentstroke}%
\pgfsetdash{}{0pt}%
\pgfpathmoveto{\pgfqpoint{0.800000in}{4.015451in}}%
\pgfpathlineto{\pgfqpoint{5.760000in}{4.015451in}}%
\pgfusepath{stroke}%
\end{pgfscope}%
\begin{pgfscope}%
\pgfsetbuttcap%
\pgfsetroundjoin%
\definecolor{currentfill}{rgb}{0.000000,0.000000,0.000000}%
\pgfsetfillcolor{currentfill}%
\pgfsetlinewidth{0.803000pt}%
\definecolor{currentstroke}{rgb}{0.000000,0.000000,0.000000}%
\pgfsetstrokecolor{currentstroke}%
\pgfsetdash{}{0pt}%
\pgfsys@defobject{currentmarker}{\pgfqpoint{-0.048611in}{0.000000in}}{\pgfqpoint{-0.000000in}{0.000000in}}{%
\pgfpathmoveto{\pgfqpoint{-0.000000in}{0.000000in}}%
\pgfpathlineto{\pgfqpoint{-0.048611in}{0.000000in}}%
\pgfusepath{stroke,fill}%
}%
\begin{pgfscope}%
\pgfsys@transformshift{0.800000in}{4.015451in}%
\pgfsys@useobject{currentmarker}{}%
\end{pgfscope}%
\end{pgfscope}%
\begin{pgfscope}%
\definecolor{textcolor}{rgb}{0.000000,0.000000,0.000000}%
\pgfsetstrokecolor{textcolor}%
\pgfsetfillcolor{textcolor}%
\pgftext[x=0.393533in, y=3.962690in, left, base]{\color{textcolor}\sffamily\fontsize{10.000000}{12.000000}\selectfont 25.0}%
\end{pgfscope}%
\begin{pgfscope}%
\definecolor{textcolor}{rgb}{0.000000,0.000000,0.000000}%
\pgfsetstrokecolor{textcolor}%
\pgfsetfillcolor{textcolor}%
\pgftext[x=0.337977in,y=2.376000in,,bottom,rotate=90.000000]{\color{textcolor}\sffamily\fontsize{10.000000}{12.000000}\selectfont Heading [deg]}%
\end{pgfscope}%
\begin{pgfscope}%
\pgfpathrectangle{\pgfqpoint{0.800000in}{0.528000in}}{\pgfqpoint{4.960000in}{3.696000in}}%
\pgfusepath{clip}%
\pgfsetrectcap%
\pgfsetroundjoin%
\pgfsetlinewidth{1.505625pt}%
\definecolor{currentstroke}{rgb}{0.121569,0.466667,0.705882}%
\pgfsetstrokecolor{currentstroke}%
\pgfsetdash{}{0pt}%
\pgfpathmoveto{\pgfqpoint{1.025455in}{0.697843in}}%
\pgfpathlineto{\pgfqpoint{1.080216in}{0.793685in}}%
\pgfpathlineto{\pgfqpoint{1.134673in}{1.027761in}}%
\pgfpathlineto{\pgfqpoint{1.189391in}{1.330032in}}%
\pgfpathlineto{\pgfqpoint{1.243361in}{1.635989in}}%
\pgfpathlineto{\pgfqpoint{1.298163in}{1.984338in}}%
\pgfpathlineto{\pgfqpoint{1.352591in}{2.332687in}}%
\pgfpathlineto{\pgfqpoint{1.406735in}{2.653389in}}%
\pgfpathlineto{\pgfqpoint{1.460531in}{2.998052in}}%
\pgfpathlineto{\pgfqpoint{1.515070in}{3.337185in}}%
\pgfpathlineto{\pgfqpoint{1.570910in}{3.654201in}}%
\pgfpathlineto{\pgfqpoint{1.625119in}{3.897492in}}%
\pgfpathlineto{\pgfqpoint{1.678888in}{4.024667in}}%
\pgfpathlineto{\pgfqpoint{1.733213in}{4.056000in}}%
\pgfpathlineto{\pgfqpoint{1.787534in}{4.013608in}}%
\pgfpathlineto{\pgfqpoint{1.841591in}{3.926982in}}%
\pgfpathlineto{\pgfqpoint{1.895738in}{3.821924in}}%
\pgfpathlineto{\pgfqpoint{1.950412in}{3.724239in}}%
\pgfpathlineto{\pgfqpoint{2.004612in}{3.652358in}}%
\pgfpathlineto{\pgfqpoint{2.058989in}{3.606280in}}%
\pgfpathlineto{\pgfqpoint{2.113203in}{3.589692in}}%
\pgfpathlineto{\pgfqpoint{2.168179in}{3.593378in}}%
\pgfpathlineto{\pgfqpoint{2.223038in}{3.613652in}}%
\pgfpathlineto{\pgfqpoint{2.276359in}{3.635770in}}%
\pgfpathlineto{\pgfqpoint{2.330619in}{3.648671in}}%
\pgfpathlineto{\pgfqpoint{2.384796in}{3.648671in}}%
\pgfpathlineto{\pgfqpoint{2.438985in}{3.635770in}}%
\pgfpathlineto{\pgfqpoint{2.493267in}{3.613652in}}%
\pgfpathlineto{\pgfqpoint{2.547795in}{3.587849in}}%
\pgfpathlineto{\pgfqpoint{2.602412in}{3.552829in}}%
\pgfpathlineto{\pgfqpoint{2.656260in}{3.517810in}}%
\pgfpathlineto{\pgfqpoint{2.710635in}{3.484634in}}%
\pgfpathlineto{\pgfqpoint{2.765174in}{3.458830in}}%
\pgfpathlineto{\pgfqpoint{2.821035in}{3.442242in}}%
\pgfpathlineto{\pgfqpoint{2.875002in}{3.434870in}}%
\pgfpathlineto{\pgfqpoint{2.928293in}{3.433027in}}%
\pgfpathlineto{\pgfqpoint{2.982386in}{3.442242in}}%
\pgfpathlineto{\pgfqpoint{3.036745in}{3.451458in}}%
\pgfpathlineto{\pgfqpoint{3.091018in}{3.460674in}}%
\pgfpathlineto{\pgfqpoint{3.145517in}{3.468046in}}%
\pgfpathlineto{\pgfqpoint{3.200134in}{3.468046in}}%
\pgfpathlineto{\pgfqpoint{3.254238in}{3.462517in}}%
\pgfpathlineto{\pgfqpoint{3.308284in}{3.447772in}}%
\pgfpathlineto{\pgfqpoint{3.362604in}{3.431184in}}%
\pgfpathlineto{\pgfqpoint{3.416655in}{3.421968in}}%
\pgfpathlineto{\pgfqpoint{3.472117in}{3.420125in}}%
\pgfpathlineto{\pgfqpoint{3.525701in}{3.420125in}}%
\pgfpathlineto{\pgfqpoint{3.579550in}{3.423811in}}%
\pgfpathlineto{\pgfqpoint{3.634058in}{3.431184in}}%
\pgfpathlineto{\pgfqpoint{3.688272in}{3.445929in}}%
\pgfpathlineto{\pgfqpoint{3.742615in}{3.464360in}}%
\pgfpathlineto{\pgfqpoint{3.797093in}{3.479105in}}%
\pgfpathlineto{\pgfqpoint{3.851657in}{3.482791in}}%
\pgfpathlineto{\pgfqpoint{3.905802in}{3.471732in}}%
\pgfpathlineto{\pgfqpoint{3.959907in}{3.453301in}}%
\pgfpathlineto{\pgfqpoint{4.015085in}{3.438556in}}%
\pgfpathlineto{\pgfqpoint{4.071056in}{3.431184in}}%
\pgfpathlineto{\pgfqpoint{4.124047in}{3.429341in}}%
\pgfpathlineto{\pgfqpoint{4.178083in}{3.431184in}}%
\pgfpathlineto{\pgfqpoint{4.232485in}{3.440399in}}%
\pgfpathlineto{\pgfqpoint{4.286809in}{3.449615in}}%
\pgfpathlineto{\pgfqpoint{4.341813in}{3.455144in}}%
\pgfpathlineto{\pgfqpoint{4.395067in}{3.456987in}}%
\pgfpathlineto{\pgfqpoint{4.449392in}{3.456987in}}%
\pgfpathlineto{\pgfqpoint{4.503590in}{3.455144in}}%
\pgfpathlineto{\pgfqpoint{4.557804in}{3.451458in}}%
\pgfpathlineto{\pgfqpoint{4.612129in}{3.455144in}}%
\pgfpathlineto{\pgfqpoint{4.666171in}{3.462517in}}%
\pgfpathlineto{\pgfqpoint{4.721790in}{3.473575in}}%
\pgfpathlineto{\pgfqpoint{4.775189in}{3.482791in}}%
\pgfpathlineto{\pgfqpoint{4.828905in}{3.492007in}}%
\pgfpathlineto{\pgfqpoint{4.884564in}{3.495693in}}%
\pgfpathlineto{\pgfqpoint{4.937184in}{3.495693in}}%
\pgfpathlineto{\pgfqpoint{4.991430in}{3.493850in}}%
\pgfpathlineto{\pgfqpoint{5.045525in}{3.493850in}}%
\pgfpathlineto{\pgfqpoint{5.100230in}{3.490163in}}%
\pgfpathlineto{\pgfqpoint{5.154248in}{3.486477in}}%
\pgfpathlineto{\pgfqpoint{5.208466in}{3.486477in}}%
\pgfpathlineto{\pgfqpoint{5.262379in}{3.484634in}}%
\pgfpathlineto{\pgfqpoint{5.316827in}{3.482791in}}%
\pgfpathlineto{\pgfqpoint{5.372401in}{3.479105in}}%
\pgfpathlineto{\pgfqpoint{5.426140in}{3.471732in}}%
\pgfpathlineto{\pgfqpoint{5.479980in}{3.453301in}}%
\pgfpathlineto{\pgfqpoint{5.534002in}{3.433027in}}%
\pgfusepath{stroke}%
\end{pgfscope}%
\begin{pgfscope}%
\pgfpathrectangle{\pgfqpoint{0.800000in}{0.528000in}}{\pgfqpoint{4.960000in}{3.696000in}}%
\pgfusepath{clip}%
\pgfsetrectcap%
\pgfsetroundjoin%
\pgfsetlinewidth{1.505625pt}%
\definecolor{currentstroke}{rgb}{1.000000,0.498039,0.054902}%
\pgfsetstrokecolor{currentstroke}%
\pgfsetdash{}{0pt}%
\pgfpathmoveto{\pgfqpoint{1.025455in}{0.699686in}}%
\pgfpathlineto{\pgfqpoint{1.079773in}{0.801058in}}%
\pgfpathlineto{\pgfqpoint{1.133941in}{1.022231in}}%
\pgfpathlineto{\pgfqpoint{1.188423in}{1.315287in}}%
\pgfpathlineto{\pgfqpoint{1.242250in}{1.648891in}}%
\pgfpathlineto{\pgfqpoint{1.296677in}{1.978809in}}%
\pgfpathlineto{\pgfqpoint{1.350545in}{2.329001in}}%
\pgfpathlineto{\pgfqpoint{1.405110in}{2.668134in}}%
\pgfpathlineto{\pgfqpoint{1.459241in}{2.977777in}}%
\pgfpathlineto{\pgfqpoint{1.513709in}{3.248715in}}%
\pgfpathlineto{\pgfqpoint{1.568972in}{3.425654in}}%
\pgfpathlineto{\pgfqpoint{1.622550in}{3.508595in}}%
\pgfpathlineto{\pgfqpoint{1.676384in}{3.519653in}}%
\pgfpathlineto{\pgfqpoint{1.730808in}{3.493850in}}%
\pgfpathlineto{\pgfqpoint{1.785404in}{3.449615in}}%
\pgfpathlineto{\pgfqpoint{1.839622in}{3.403537in}}%
\pgfpathlineto{\pgfqpoint{1.893526in}{3.364832in}}%
\pgfpathlineto{\pgfqpoint{1.947841in}{3.342714in}}%
\pgfpathlineto{\pgfqpoint{2.002297in}{3.335342in}}%
\pgfpathlineto{\pgfqpoint{2.056312in}{3.340871in}}%
\pgfpathlineto{\pgfqpoint{2.110933in}{3.353773in}}%
\pgfpathlineto{\pgfqpoint{2.165308in}{3.374047in}}%
\pgfpathlineto{\pgfqpoint{2.220974in}{3.398008in}}%
\pgfpathlineto{\pgfqpoint{2.274318in}{3.418282in}}%
\pgfpathlineto{\pgfqpoint{2.330368in}{3.434870in}}%
\pgfpathlineto{\pgfqpoint{2.382269in}{3.440399in}}%
\pgfpathlineto{\pgfqpoint{2.437082in}{3.438556in}}%
\pgfpathlineto{\pgfqpoint{2.491662in}{3.431184in}}%
\pgfpathlineto{\pgfqpoint{2.546067in}{3.416439in}}%
\pgfpathlineto{\pgfqpoint{2.599800in}{3.399851in}}%
\pgfpathlineto{\pgfqpoint{2.656658in}{3.383263in}}%
\pgfpathlineto{\pgfqpoint{2.708777in}{3.368518in}}%
\pgfpathlineto{\pgfqpoint{2.762940in}{3.355616in}}%
\pgfpathlineto{\pgfqpoint{2.818853in}{3.340871in}}%
\pgfpathlineto{\pgfqpoint{2.872585in}{3.333499in}}%
\pgfpathlineto{\pgfqpoint{2.926161in}{3.324283in}}%
\pgfpathlineto{\pgfqpoint{2.980126in}{3.316911in}}%
\pgfpathlineto{\pgfqpoint{3.034292in}{3.311381in}}%
\pgfpathlineto{\pgfqpoint{3.089254in}{3.307695in}}%
\pgfpathlineto{\pgfqpoint{3.146567in}{3.300323in}}%
\pgfpathlineto{\pgfqpoint{3.197229in}{3.294793in}}%
\pgfpathlineto{\pgfqpoint{3.251303in}{3.289264in}}%
\pgfpathlineto{\pgfqpoint{3.305470in}{3.285578in}}%
\pgfpathlineto{\pgfqpoint{3.361729in}{3.291107in}}%
\pgfpathlineto{\pgfqpoint{3.415373in}{3.302166in}}%
\pgfpathlineto{\pgfqpoint{3.470332in}{3.320597in}}%
\pgfpathlineto{\pgfqpoint{3.523973in}{3.339028in}}%
\pgfpathlineto{\pgfqpoint{3.578305in}{3.362988in}}%
\pgfpathlineto{\pgfqpoint{3.632915in}{3.386949in}}%
\pgfpathlineto{\pgfqpoint{3.686793in}{3.407223in}}%
\pgfpathlineto{\pgfqpoint{3.740778in}{3.427498in}}%
\pgfpathlineto{\pgfqpoint{3.794981in}{3.440399in}}%
\pgfpathlineto{\pgfqpoint{3.849441in}{3.449615in}}%
\pgfpathlineto{\pgfqpoint{3.903701in}{3.451458in}}%
\pgfpathlineto{\pgfqpoint{3.959232in}{3.445929in}}%
\pgfpathlineto{\pgfqpoint{4.013244in}{3.436713in}}%
\pgfpathlineto{\pgfqpoint{4.066783in}{3.427498in}}%
\pgfpathlineto{\pgfqpoint{4.120996in}{3.416439in}}%
\pgfpathlineto{\pgfqpoint{4.175078in}{3.407223in}}%
\pgfpathlineto{\pgfqpoint{4.229349in}{3.401694in}}%
\pgfpathlineto{\pgfqpoint{4.283489in}{3.392478in}}%
\pgfpathlineto{\pgfqpoint{4.337676in}{3.386949in}}%
\pgfpathlineto{\pgfqpoint{4.392684in}{3.390635in}}%
\pgfpathlineto{\pgfqpoint{4.446850in}{3.399851in}}%
\pgfpathlineto{\pgfqpoint{4.501075in}{3.409066in}}%
\pgfpathlineto{\pgfqpoint{4.557052in}{3.421968in}}%
\pgfpathlineto{\pgfqpoint{4.610505in}{3.433027in}}%
\pgfpathlineto{\pgfqpoint{4.664180in}{3.447772in}}%
\pgfpathlineto{\pgfqpoint{4.718027in}{3.462517in}}%
\pgfpathlineto{\pgfqpoint{4.772427in}{3.479105in}}%
\pgfpathlineto{\pgfqpoint{4.826597in}{3.497536in}}%
\pgfpathlineto{\pgfqpoint{4.880809in}{3.515967in}}%
\pgfpathlineto{\pgfqpoint{4.935589in}{3.530712in}}%
\pgfpathlineto{\pgfqpoint{4.989152in}{3.545457in}}%
\pgfpathlineto{\pgfqpoint{5.043488in}{3.556516in}}%
\pgfpathlineto{\pgfqpoint{5.098030in}{3.567574in}}%
\pgfpathlineto{\pgfqpoint{5.152024in}{3.571261in}}%
\pgfpathlineto{\pgfqpoint{5.207931in}{3.563888in}}%
\pgfpathlineto{\pgfqpoint{5.262101in}{3.545457in}}%
\pgfpathlineto{\pgfqpoint{5.315263in}{3.527026in}}%
\pgfpathlineto{\pgfqpoint{5.369397in}{3.508595in}}%
\pgfpathlineto{\pgfqpoint{5.423464in}{3.495693in}}%
\pgfpathlineto{\pgfqpoint{5.477630in}{3.486477in}}%
\pgfpathlineto{\pgfqpoint{5.531809in}{3.484634in}}%
\pgfusepath{stroke}%
\end{pgfscope}%
\begin{pgfscope}%
\pgfpathrectangle{\pgfqpoint{0.800000in}{0.528000in}}{\pgfqpoint{4.960000in}{3.696000in}}%
\pgfusepath{clip}%
\pgfsetrectcap%
\pgfsetroundjoin%
\pgfsetlinewidth{1.505625pt}%
\definecolor{currentstroke}{rgb}{0.172549,0.627451,0.172549}%
\pgfsetstrokecolor{currentstroke}%
\pgfsetdash{}{0pt}%
\pgfpathmoveto{\pgfqpoint{1.025455in}{0.699686in}}%
\pgfpathlineto{\pgfqpoint{1.079158in}{0.793685in}}%
\pgfpathlineto{\pgfqpoint{1.133729in}{1.024075in}}%
\pgfpathlineto{\pgfqpoint{1.187760in}{1.318973in}}%
\pgfpathlineto{\pgfqpoint{1.242179in}{1.635989in}}%
\pgfpathlineto{\pgfqpoint{1.296414in}{1.984338in}}%
\pgfpathlineto{\pgfqpoint{1.351748in}{2.325314in}}%
\pgfpathlineto{\pgfqpoint{1.405245in}{2.673663in}}%
\pgfpathlineto{\pgfqpoint{1.459210in}{2.998052in}}%
\pgfpathlineto{\pgfqpoint{1.513825in}{3.326126in}}%
\pgfpathlineto{\pgfqpoint{1.568002in}{3.587849in}}%
\pgfpathlineto{\pgfqpoint{1.622370in}{3.764788in}}%
\pgfpathlineto{\pgfqpoint{1.676696in}{3.862473in}}%
\pgfpathlineto{\pgfqpoint{1.730722in}{3.877218in}}%
\pgfpathlineto{\pgfqpoint{1.784898in}{3.847728in}}%
\pgfpathlineto{\pgfqpoint{1.839317in}{3.774003in}}%
\pgfpathlineto{\pgfqpoint{1.893308in}{3.687377in}}%
\pgfpathlineto{\pgfqpoint{1.947958in}{3.622868in}}%
\pgfpathlineto{\pgfqpoint{2.004105in}{3.571261in}}%
\pgfpathlineto{\pgfqpoint{2.057864in}{3.550986in}}%
\pgfpathlineto{\pgfqpoint{2.111981in}{3.545457in}}%
\pgfpathlineto{\pgfqpoint{2.165847in}{3.549143in}}%
\pgfpathlineto{\pgfqpoint{2.220123in}{3.552829in}}%
\pgfpathlineto{\pgfqpoint{2.274627in}{3.562045in}}%
\pgfpathlineto{\pgfqpoint{2.328598in}{3.573104in}}%
\pgfpathlineto{\pgfqpoint{2.383709in}{3.571261in}}%
\pgfpathlineto{\pgfqpoint{2.438391in}{3.554673in}}%
\pgfpathlineto{\pgfqpoint{2.492295in}{3.534398in}}%
\pgfpathlineto{\pgfqpoint{2.546508in}{3.534398in}}%
\pgfpathlineto{\pgfqpoint{2.602202in}{3.554673in}}%
\pgfpathlineto{\pgfqpoint{2.656694in}{3.578633in}}%
\pgfpathlineto{\pgfqpoint{2.710302in}{3.589692in}}%
\pgfpathlineto{\pgfqpoint{2.764258in}{3.587849in}}%
\pgfpathlineto{\pgfqpoint{2.818486in}{3.576790in}}%
\pgfpathlineto{\pgfqpoint{2.872608in}{3.563888in}}%
\pgfpathlineto{\pgfqpoint{2.927022in}{3.556516in}}%
\pgfpathlineto{\pgfqpoint{2.981128in}{3.558359in}}%
\pgfpathlineto{\pgfqpoint{3.036103in}{3.552829in}}%
\pgfpathlineto{\pgfqpoint{3.089907in}{3.550986in}}%
\pgfpathlineto{\pgfqpoint{3.144169in}{3.556516in}}%
\pgfpathlineto{\pgfqpoint{3.198240in}{3.562045in}}%
\pgfpathlineto{\pgfqpoint{3.253780in}{3.571261in}}%
\pgfpathlineto{\pgfqpoint{3.307666in}{3.571261in}}%
\pgfpathlineto{\pgfqpoint{3.362779in}{3.565731in}}%
\pgfpathlineto{\pgfqpoint{3.415915in}{3.554673in}}%
\pgfpathlineto{\pgfqpoint{3.470194in}{3.545457in}}%
\pgfpathlineto{\pgfqpoint{3.524372in}{3.538084in}}%
\pgfpathlineto{\pgfqpoint{3.578400in}{3.525183in}}%
\pgfpathlineto{\pgfqpoint{3.633197in}{3.506752in}}%
\pgfpathlineto{\pgfqpoint{3.686950in}{3.482791in}}%
\pgfpathlineto{\pgfqpoint{3.741647in}{3.464360in}}%
\pgfpathlineto{\pgfqpoint{3.795351in}{3.449615in}}%
\pgfpathlineto{\pgfqpoint{3.850948in}{3.436713in}}%
\pgfpathlineto{\pgfqpoint{3.904639in}{3.425654in}}%
\pgfpathlineto{\pgfqpoint{3.958434in}{3.412753in}}%
\pgfpathlineto{\pgfqpoint{4.013462in}{3.401694in}}%
\pgfpathlineto{\pgfqpoint{4.067332in}{3.390635in}}%
\pgfpathlineto{\pgfqpoint{4.121295in}{3.379577in}}%
\pgfpathlineto{\pgfqpoint{4.175573in}{3.372204in}}%
\pgfpathlineto{\pgfqpoint{4.230016in}{3.366675in}}%
\pgfpathlineto{\pgfqpoint{4.284549in}{3.361145in}}%
\pgfpathlineto{\pgfqpoint{4.338807in}{3.350087in}}%
\pgfpathlineto{\pgfqpoint{4.393133in}{3.326126in}}%
\pgfpathlineto{\pgfqpoint{4.448119in}{3.294793in}}%
\pgfpathlineto{\pgfqpoint{4.503070in}{3.265303in}}%
\pgfpathlineto{\pgfqpoint{4.556342in}{3.246872in}}%
\pgfpathlineto{\pgfqpoint{4.610059in}{3.237657in}}%
\pgfpathlineto{\pgfqpoint{4.664681in}{3.237657in}}%
\pgfpathlineto{\pgfqpoint{4.718993in}{3.245029in}}%
\pgfpathlineto{\pgfqpoint{4.773333in}{3.261617in}}%
\pgfpathlineto{\pgfqpoint{4.827813in}{3.280048in}}%
\pgfpathlineto{\pgfqpoint{4.882264in}{3.296636in}}%
\pgfpathlineto{\pgfqpoint{4.935857in}{3.309538in}}%
\pgfpathlineto{\pgfqpoint{4.990112in}{3.302166in}}%
\pgfpathlineto{\pgfqpoint{5.046308in}{3.274519in}}%
\pgfpathlineto{\pgfqpoint{5.100418in}{3.250558in}}%
\pgfpathlineto{\pgfqpoint{5.154363in}{3.239500in}}%
\pgfpathlineto{\pgfqpoint{5.208108in}{3.248715in}}%
\pgfpathlineto{\pgfqpoint{5.262316in}{3.268990in}}%
\pgfpathlineto{\pgfqpoint{5.316607in}{3.296636in}}%
\pgfpathlineto{\pgfqpoint{5.370885in}{3.326126in}}%
\pgfpathlineto{\pgfqpoint{5.426348in}{3.351930in}}%
\pgfpathlineto{\pgfqpoint{5.479566in}{3.357459in}}%
\pgfpathlineto{\pgfqpoint{5.533485in}{3.350087in}}%
\pgfusepath{stroke}%
\end{pgfscope}%
\begin{pgfscope}%
\pgfpathrectangle{\pgfqpoint{0.800000in}{0.528000in}}{\pgfqpoint{4.960000in}{3.696000in}}%
\pgfusepath{clip}%
\pgfsetrectcap%
\pgfsetroundjoin%
\pgfsetlinewidth{1.505625pt}%
\definecolor{currentstroke}{rgb}{0.839216,0.152941,0.156863}%
\pgfsetstrokecolor{currentstroke}%
\pgfsetdash{}{0pt}%
\pgfpathmoveto{\pgfqpoint{1.025455in}{0.696000in}}%
\pgfpathlineto{\pgfqpoint{1.080090in}{0.791842in}}%
\pgfpathlineto{\pgfqpoint{1.134614in}{1.025918in}}%
\pgfpathlineto{\pgfqpoint{1.188791in}{1.320816in}}%
\pgfpathlineto{\pgfqpoint{1.242834in}{1.637832in}}%
\pgfpathlineto{\pgfqpoint{1.297310in}{1.984338in}}%
\pgfpathlineto{\pgfqpoint{1.351337in}{2.334530in}}%
\pgfpathlineto{\pgfqpoint{1.405893in}{2.653389in}}%
\pgfpathlineto{\pgfqpoint{1.460365in}{2.998052in}}%
\pgfpathlineto{\pgfqpoint{1.514397in}{3.337185in}}%
\pgfpathlineto{\pgfqpoint{1.569063in}{3.637613in}}%
\pgfpathlineto{\pgfqpoint{1.622725in}{3.864316in}}%
\pgfpathlineto{\pgfqpoint{1.677042in}{3.993334in}}%
\pgfpathlineto{\pgfqpoint{1.732657in}{4.020981in}}%
\pgfpathlineto{\pgfqpoint{1.786735in}{3.995177in}}%
\pgfpathlineto{\pgfqpoint{1.840785in}{3.939884in}}%
\pgfpathlineto{\pgfqpoint{1.894526in}{3.873532in}}%
\pgfpathlineto{\pgfqpoint{1.948829in}{3.810866in}}%
\pgfpathlineto{\pgfqpoint{2.003223in}{3.742670in}}%
\pgfpathlineto{\pgfqpoint{2.057862in}{3.691063in}}%
\pgfpathlineto{\pgfqpoint{2.112141in}{3.656044in}}%
\pgfpathlineto{\pgfqpoint{2.166850in}{3.628397in}}%
\pgfpathlineto{\pgfqpoint{2.220651in}{3.598907in}}%
\pgfpathlineto{\pgfqpoint{2.274741in}{3.563888in}}%
\pgfpathlineto{\pgfqpoint{2.329019in}{3.532555in}}%
\pgfpathlineto{\pgfqpoint{2.384092in}{3.492007in}}%
\pgfpathlineto{\pgfqpoint{2.437556in}{3.449615in}}%
\pgfpathlineto{\pgfqpoint{2.491648in}{3.420125in}}%
\pgfpathlineto{\pgfqpoint{2.545872in}{3.403537in}}%
\pgfpathlineto{\pgfqpoint{2.599952in}{3.398008in}}%
\pgfpathlineto{\pgfqpoint{2.654185in}{3.401694in}}%
\pgfpathlineto{\pgfqpoint{2.709328in}{3.407223in}}%
\pgfpathlineto{\pgfqpoint{2.763965in}{3.409066in}}%
\pgfpathlineto{\pgfqpoint{2.818019in}{3.405380in}}%
\pgfpathlineto{\pgfqpoint{2.871918in}{3.396165in}}%
\pgfpathlineto{\pgfqpoint{2.926157in}{3.379577in}}%
\pgfpathlineto{\pgfqpoint{2.982161in}{3.350087in}}%
\pgfpathlineto{\pgfqpoint{3.035462in}{3.316911in}}%
\pgfpathlineto{\pgfqpoint{3.089529in}{3.283735in}}%
\pgfpathlineto{\pgfqpoint{3.143567in}{3.250558in}}%
\pgfpathlineto{\pgfqpoint{3.197957in}{3.221069in}}%
\pgfpathlineto{\pgfqpoint{3.252456in}{3.204481in}}%
\pgfpathlineto{\pgfqpoint{3.307163in}{3.187892in}}%
\pgfpathlineto{\pgfqpoint{3.361367in}{3.169461in}}%
\pgfpathlineto{\pgfqpoint{3.416005in}{3.156560in}}%
\pgfpathlineto{\pgfqpoint{3.469577in}{3.147344in}}%
\pgfpathlineto{\pgfqpoint{3.524277in}{3.139971in}}%
\pgfpathlineto{\pgfqpoint{3.579539in}{3.132599in}}%
\pgfpathlineto{\pgfqpoint{3.632766in}{3.128913in}}%
\pgfpathlineto{\pgfqpoint{3.686708in}{3.132599in}}%
\pgfpathlineto{\pgfqpoint{3.740680in}{3.125227in}}%
\pgfpathlineto{\pgfqpoint{3.794915in}{3.104952in}}%
\pgfpathlineto{\pgfqpoint{3.849429in}{3.082835in}}%
\pgfpathlineto{\pgfqpoint{3.903904in}{3.071776in}}%
\pgfpathlineto{\pgfqpoint{3.958372in}{3.071776in}}%
\pgfpathlineto{\pgfqpoint{4.012788in}{3.073619in}}%
\pgfpathlineto{\pgfqpoint{4.066862in}{3.084678in}}%
\pgfpathlineto{\pgfqpoint{4.121087in}{3.099423in}}%
\pgfpathlineto{\pgfqpoint{4.175062in}{3.119697in}}%
\pgfpathlineto{\pgfqpoint{4.230270in}{3.134442in}}%
\pgfpathlineto{\pgfqpoint{4.284046in}{3.147344in}}%
\pgfpathlineto{\pgfqpoint{4.338082in}{3.156560in}}%
\pgfpathlineto{\pgfqpoint{4.392432in}{3.163932in}}%
\pgfpathlineto{\pgfqpoint{4.446568in}{3.173148in}}%
\pgfpathlineto{\pgfqpoint{4.500838in}{3.182363in}}%
\pgfpathlineto{\pgfqpoint{4.555400in}{3.182363in}}%
\pgfpathlineto{\pgfqpoint{4.611048in}{3.174991in}}%
\pgfpathlineto{\pgfqpoint{4.665792in}{3.167618in}}%
\pgfpathlineto{\pgfqpoint{4.719415in}{3.147344in}}%
\pgfpathlineto{\pgfqpoint{4.773154in}{3.125227in}}%
\pgfpathlineto{\pgfqpoint{4.827416in}{3.117854in}}%
\pgfpathlineto{\pgfqpoint{4.882576in}{3.119697in}}%
\pgfpathlineto{\pgfqpoint{4.936212in}{3.128913in}}%
\pgfpathlineto{\pgfqpoint{4.990277in}{3.147344in}}%
\pgfpathlineto{\pgfqpoint{5.044179in}{3.158403in}}%
\pgfpathlineto{\pgfqpoint{5.098722in}{3.163932in}}%
\pgfpathlineto{\pgfqpoint{5.152774in}{3.163932in}}%
\pgfpathlineto{\pgfqpoint{5.207633in}{3.162089in}}%
\pgfpathlineto{\pgfqpoint{5.262571in}{3.158403in}}%
\pgfpathlineto{\pgfqpoint{5.316400in}{3.147344in}}%
\pgfpathlineto{\pgfqpoint{5.370708in}{3.125227in}}%
\pgfpathlineto{\pgfqpoint{5.424473in}{3.108639in}}%
\pgfpathlineto{\pgfqpoint{5.480658in}{3.092050in}}%
\pgfpathlineto{\pgfqpoint{5.534545in}{3.077306in}}%
\pgfusepath{stroke}%
\end{pgfscope}%
\begin{pgfscope}%
\pgfpathrectangle{\pgfqpoint{0.800000in}{0.528000in}}{\pgfqpoint{4.960000in}{3.696000in}}%
\pgfusepath{clip}%
\pgfsetrectcap%
\pgfsetroundjoin%
\pgfsetlinewidth{1.505625pt}%
\definecolor{currentstroke}{rgb}{0.580392,0.403922,0.741176}%
\pgfsetstrokecolor{currentstroke}%
\pgfsetdash{}{0pt}%
\pgfpathmoveto{\pgfqpoint{1.025455in}{0.697843in}}%
\pgfpathlineto{\pgfqpoint{1.079605in}{0.797371in}}%
\pgfpathlineto{\pgfqpoint{1.133670in}{1.011173in}}%
\pgfpathlineto{\pgfqpoint{1.188230in}{1.313444in}}%
\pgfpathlineto{\pgfqpoint{1.242578in}{1.647048in}}%
\pgfpathlineto{\pgfqpoint{1.297052in}{1.995397in}}%
\pgfpathlineto{\pgfqpoint{1.350859in}{2.319785in}}%
\pgfpathlineto{\pgfqpoint{1.407735in}{2.686565in}}%
\pgfpathlineto{\pgfqpoint{1.459492in}{3.010953in}}%
\pgfpathlineto{\pgfqpoint{1.516916in}{3.348244in}}%
\pgfpathlineto{\pgfqpoint{1.568065in}{3.621025in}}%
\pgfpathlineto{\pgfqpoint{1.622219in}{3.818238in}}%
\pgfpathlineto{\pgfqpoint{1.676853in}{3.904865in}}%
\pgfpathlineto{\pgfqpoint{1.731124in}{3.912237in}}%
\pgfpathlineto{\pgfqpoint{1.786380in}{3.890120in}}%
\pgfpathlineto{\pgfqpoint{1.839002in}{3.844042in}}%
\pgfpathlineto{\pgfqpoint{1.893188in}{3.768474in}}%
\pgfpathlineto{\pgfqpoint{1.948113in}{3.702122in}}%
\pgfpathlineto{\pgfqpoint{2.002244in}{3.657887in}}%
\pgfpathlineto{\pgfqpoint{2.056740in}{3.621025in}}%
\pgfpathlineto{\pgfqpoint{2.110716in}{3.580476in}}%
\pgfpathlineto{\pgfqpoint{2.166356in}{3.532555in}}%
\pgfpathlineto{\pgfqpoint{2.220821in}{3.473575in}}%
\pgfpathlineto{\pgfqpoint{2.274923in}{3.414596in}}%
\pgfpathlineto{\pgfqpoint{2.328877in}{3.374047in}}%
\pgfpathlineto{\pgfqpoint{2.383110in}{3.362988in}}%
\pgfpathlineto{\pgfqpoint{2.437242in}{3.368518in}}%
\pgfpathlineto{\pgfqpoint{2.491696in}{3.377733in}}%
\pgfpathlineto{\pgfqpoint{2.545915in}{3.379577in}}%
\pgfpathlineto{\pgfqpoint{2.600198in}{3.379577in}}%
\pgfpathlineto{\pgfqpoint{2.656347in}{3.379577in}}%
\pgfpathlineto{\pgfqpoint{2.708615in}{3.374047in}}%
\pgfpathlineto{\pgfqpoint{2.764603in}{3.364832in}}%
\pgfpathlineto{\pgfqpoint{2.818235in}{3.357459in}}%
\pgfpathlineto{\pgfqpoint{2.871697in}{3.351930in}}%
\pgfpathlineto{\pgfqpoint{2.925756in}{3.348244in}}%
\pgfpathlineto{\pgfqpoint{2.980093in}{3.346400in}}%
\pgfpathlineto{\pgfqpoint{3.033976in}{3.342714in}}%
\pgfpathlineto{\pgfqpoint{3.088369in}{3.337185in}}%
\pgfpathlineto{\pgfqpoint{3.142601in}{3.327969in}}%
\pgfpathlineto{\pgfqpoint{3.196633in}{3.320597in}}%
\pgfpathlineto{\pgfqpoint{3.250931in}{3.315067in}}%
\pgfpathlineto{\pgfqpoint{3.305224in}{3.307695in}}%
\pgfpathlineto{\pgfqpoint{3.359910in}{3.305852in}}%
\pgfpathlineto{\pgfqpoint{3.416017in}{3.313224in}}%
\pgfpathlineto{\pgfqpoint{3.468994in}{3.322440in}}%
\pgfpathlineto{\pgfqpoint{3.522561in}{3.329812in}}%
\pgfpathlineto{\pgfqpoint{3.576493in}{3.329812in}}%
\pgfpathlineto{\pgfqpoint{3.630789in}{3.309538in}}%
\pgfpathlineto{\pgfqpoint{3.685304in}{3.267146in}}%
\pgfpathlineto{\pgfqpoint{3.739675in}{3.237657in}}%
\pgfpathlineto{\pgfqpoint{3.794020in}{3.237657in}}%
\pgfpathlineto{\pgfqpoint{3.847922in}{3.246872in}}%
\pgfpathlineto{\pgfqpoint{3.904301in}{3.259774in}}%
\pgfpathlineto{\pgfqpoint{3.958124in}{3.250558in}}%
\pgfpathlineto{\pgfqpoint{4.011933in}{3.228441in}}%
\pgfpathlineto{\pgfqpoint{4.065834in}{3.202637in}}%
\pgfpathlineto{\pgfqpoint{4.120476in}{3.178677in}}%
\pgfpathlineto{\pgfqpoint{4.174488in}{3.184206in}}%
\pgfpathlineto{\pgfqpoint{4.229905in}{3.182363in}}%
\pgfpathlineto{\pgfqpoint{4.284419in}{3.158403in}}%
\pgfpathlineto{\pgfqpoint{4.338673in}{3.128913in}}%
\pgfpathlineto{\pgfqpoint{4.393106in}{3.116011in}}%
\pgfpathlineto{\pgfqpoint{4.447219in}{3.123383in}}%
\pgfpathlineto{\pgfqpoint{4.501451in}{3.139971in}}%
\pgfpathlineto{\pgfqpoint{4.557178in}{3.147344in}}%
\pgfpathlineto{\pgfqpoint{4.610548in}{3.160246in}}%
\pgfpathlineto{\pgfqpoint{4.664967in}{3.174991in}}%
\pgfpathlineto{\pgfqpoint{4.718950in}{3.182363in}}%
\pgfpathlineto{\pgfqpoint{4.773196in}{3.187892in}}%
\pgfpathlineto{\pgfqpoint{4.827365in}{3.197108in}}%
\pgfpathlineto{\pgfqpoint{4.881443in}{3.208167in}}%
\pgfpathlineto{\pgfqpoint{4.936019in}{3.221069in}}%
\pgfpathlineto{\pgfqpoint{4.989937in}{3.226598in}}%
\pgfpathlineto{\pgfqpoint{5.044358in}{3.215539in}}%
\pgfpathlineto{\pgfqpoint{5.098849in}{3.187892in}}%
\pgfpathlineto{\pgfqpoint{5.152964in}{3.163932in}}%
\pgfpathlineto{\pgfqpoint{5.208944in}{3.139971in}}%
\pgfpathlineto{\pgfqpoint{5.262618in}{3.127070in}}%
\pgfpathlineto{\pgfqpoint{5.317091in}{3.134442in}}%
\pgfpathlineto{\pgfqpoint{5.370508in}{3.149187in}}%
\pgfpathlineto{\pgfqpoint{5.424712in}{3.154716in}}%
\pgfpathlineto{\pgfqpoint{5.478946in}{3.163932in}}%
\pgfpathlineto{\pgfqpoint{5.533303in}{3.178677in}}%
\pgfusepath{stroke}%
\end{pgfscope}%
\begin{pgfscope}%
\pgfpathrectangle{\pgfqpoint{0.800000in}{0.528000in}}{\pgfqpoint{4.960000in}{3.696000in}}%
\pgfusepath{clip}%
\pgfsetrectcap%
\pgfsetroundjoin%
\pgfsetlinewidth{1.505625pt}%
\definecolor{currentstroke}{rgb}{0.549020,0.337255,0.294118}%
\pgfsetstrokecolor{currentstroke}%
\pgfsetdash{}{0pt}%
\pgfpathmoveto{\pgfqpoint{1.025455in}{0.697843in}}%
\pgfpathlineto{\pgfqpoint{1.079692in}{0.795528in}}%
\pgfpathlineto{\pgfqpoint{1.133976in}{1.007487in}}%
\pgfpathlineto{\pgfqpoint{1.188386in}{1.307914in}}%
\pgfpathlineto{\pgfqpoint{1.242691in}{1.641518in}}%
\pgfpathlineto{\pgfqpoint{1.296714in}{1.964064in}}%
\pgfpathlineto{\pgfqpoint{1.350821in}{2.314256in}}%
\pgfpathlineto{\pgfqpoint{1.405133in}{2.662604in}}%
\pgfpathlineto{\pgfqpoint{1.459224in}{2.988836in}}%
\pgfpathlineto{\pgfqpoint{1.517979in}{3.348244in}}%
\pgfpathlineto{\pgfqpoint{1.569207in}{3.626554in}}%
\pgfpathlineto{\pgfqpoint{1.623073in}{3.827454in}}%
\pgfpathlineto{\pgfqpoint{1.677598in}{3.908551in}}%
\pgfpathlineto{\pgfqpoint{1.731832in}{3.923296in}}%
\pgfpathlineto{\pgfqpoint{1.785768in}{3.886433in}}%
\pgfpathlineto{\pgfqpoint{1.840176in}{3.831140in}}%
\pgfpathlineto{\pgfqpoint{1.894752in}{3.792434in}}%
\pgfpathlineto{\pgfqpoint{1.948746in}{3.751886in}}%
\pgfpathlineto{\pgfqpoint{2.003098in}{3.703965in}}%
\pgfpathlineto{\pgfqpoint{2.057918in}{3.656044in}}%
\pgfpathlineto{\pgfqpoint{2.112762in}{3.604437in}}%
\pgfpathlineto{\pgfqpoint{2.167505in}{3.534398in}}%
\pgfpathlineto{\pgfqpoint{2.221144in}{3.449615in}}%
\pgfpathlineto{\pgfqpoint{2.275093in}{3.385106in}}%
\pgfpathlineto{\pgfqpoint{2.330331in}{3.344557in}}%
\pgfpathlineto{\pgfqpoint{2.384453in}{3.326126in}}%
\pgfpathlineto{\pgfqpoint{2.438577in}{3.322440in}}%
\pgfpathlineto{\pgfqpoint{2.492984in}{3.326126in}}%
\pgfpathlineto{\pgfqpoint{2.547164in}{3.326126in}}%
\pgfpathlineto{\pgfqpoint{2.601347in}{3.315067in}}%
\pgfpathlineto{\pgfqpoint{2.656122in}{3.304009in}}%
\pgfpathlineto{\pgfqpoint{2.711189in}{3.292950in}}%
\pgfpathlineto{\pgfqpoint{2.766842in}{3.285578in}}%
\pgfpathlineto{\pgfqpoint{2.820432in}{3.270833in}}%
\pgfpathlineto{\pgfqpoint{2.874410in}{3.245029in}}%
\pgfpathlineto{\pgfqpoint{2.928966in}{3.224755in}}%
\pgfpathlineto{\pgfqpoint{2.982825in}{3.215539in}}%
\pgfpathlineto{\pgfqpoint{3.036906in}{3.208167in}}%
\pgfpathlineto{\pgfqpoint{3.091367in}{3.202637in}}%
\pgfpathlineto{\pgfqpoint{3.145583in}{3.195265in}}%
\pgfpathlineto{\pgfqpoint{3.199468in}{3.189736in}}%
\pgfpathlineto{\pgfqpoint{3.253811in}{3.195265in}}%
\pgfpathlineto{\pgfqpoint{3.309196in}{3.174991in}}%
\pgfpathlineto{\pgfqpoint{3.363106in}{3.143658in}}%
\pgfpathlineto{\pgfqpoint{3.418267in}{3.134442in}}%
\pgfpathlineto{\pgfqpoint{3.472061in}{3.136285in}}%
\pgfpathlineto{\pgfqpoint{3.526446in}{3.132599in}}%
\pgfpathlineto{\pgfqpoint{3.581136in}{3.130756in}}%
\pgfpathlineto{\pgfqpoint{3.635046in}{3.136285in}}%
\pgfpathlineto{\pgfqpoint{3.689103in}{3.143658in}}%
\pgfpathlineto{\pgfqpoint{3.743607in}{3.147344in}}%
\pgfpathlineto{\pgfqpoint{3.797723in}{3.145501in}}%
\pgfpathlineto{\pgfqpoint{3.852686in}{3.127070in}}%
\pgfpathlineto{\pgfqpoint{3.906573in}{3.093894in}}%
\pgfpathlineto{\pgfqpoint{3.960513in}{3.071776in}}%
\pgfpathlineto{\pgfqpoint{4.015470in}{3.079149in}}%
\pgfpathlineto{\pgfqpoint{4.070565in}{3.103109in}}%
\pgfpathlineto{\pgfqpoint{4.124081in}{3.125227in}}%
\pgfpathlineto{\pgfqpoint{4.177739in}{3.141815in}}%
\pgfpathlineto{\pgfqpoint{4.231745in}{3.151030in}}%
\pgfpathlineto{\pgfqpoint{4.285985in}{3.136285in}}%
\pgfpathlineto{\pgfqpoint{4.340108in}{3.112325in}}%
\pgfpathlineto{\pgfqpoint{4.394324in}{3.101266in}}%
\pgfpathlineto{\pgfqpoint{4.448968in}{3.114168in}}%
\pgfpathlineto{\pgfqpoint{4.503189in}{3.132599in}}%
\pgfpathlineto{\pgfqpoint{4.556981in}{3.154716in}}%
\pgfpathlineto{\pgfqpoint{4.611351in}{3.186049in}}%
\pgfpathlineto{\pgfqpoint{4.667208in}{3.206324in}}%
\pgfpathlineto{\pgfqpoint{4.720427in}{3.226598in}}%
\pgfpathlineto{\pgfqpoint{4.774123in}{3.254245in}}%
\pgfpathlineto{\pgfqpoint{4.828055in}{3.259774in}}%
\pgfpathlineto{\pgfqpoint{4.882233in}{3.252402in}}%
\pgfpathlineto{\pgfqpoint{4.936459in}{3.259774in}}%
\pgfpathlineto{\pgfqpoint{4.991013in}{3.278205in}}%
\pgfpathlineto{\pgfqpoint{5.045390in}{3.298479in}}%
\pgfpathlineto{\pgfqpoint{5.099821in}{3.313224in}}%
\pgfpathlineto{\pgfqpoint{5.153345in}{3.322440in}}%
\pgfpathlineto{\pgfqpoint{5.207458in}{3.329812in}}%
\pgfpathlineto{\pgfqpoint{5.261814in}{3.331656in}}%
\pgfpathlineto{\pgfqpoint{5.317542in}{3.311381in}}%
\pgfpathlineto{\pgfqpoint{5.371046in}{3.268990in}}%
\pgfpathlineto{\pgfqpoint{5.424626in}{3.246872in}}%
\pgfpathlineto{\pgfqpoint{5.478878in}{3.254245in}}%
\pgfpathlineto{\pgfqpoint{5.532902in}{3.274519in}}%
\pgfusepath{stroke}%
\end{pgfscope}%
\begin{pgfscope}%
\pgfpathrectangle{\pgfqpoint{0.800000in}{0.528000in}}{\pgfqpoint{4.960000in}{3.696000in}}%
\pgfusepath{clip}%
\pgfsetrectcap%
\pgfsetroundjoin%
\pgfsetlinewidth{1.505625pt}%
\definecolor{currentstroke}{rgb}{0.890196,0.466667,0.760784}%
\pgfsetstrokecolor{currentstroke}%
\pgfsetdash{}{0pt}%
\pgfpathmoveto{\pgfqpoint{1.025455in}{3.166235in}}%
\pgfpathlineto{\pgfqpoint{5.527989in}{3.166235in}}%
\pgfusepath{stroke}%
\end{pgfscope}%
\begin{pgfscope}%
\pgfsetrectcap%
\pgfsetmiterjoin%
\pgfsetlinewidth{0.803000pt}%
\definecolor{currentstroke}{rgb}{0.000000,0.000000,0.000000}%
\pgfsetstrokecolor{currentstroke}%
\pgfsetdash{}{0pt}%
\pgfpathmoveto{\pgfqpoint{0.800000in}{0.528000in}}%
\pgfpathlineto{\pgfqpoint{0.800000in}{4.224000in}}%
\pgfusepath{stroke}%
\end{pgfscope}%
\begin{pgfscope}%
\pgfsetrectcap%
\pgfsetmiterjoin%
\pgfsetlinewidth{0.803000pt}%
\definecolor{currentstroke}{rgb}{0.000000,0.000000,0.000000}%
\pgfsetstrokecolor{currentstroke}%
\pgfsetdash{}{0pt}%
\pgfpathmoveto{\pgfqpoint{5.760000in}{0.528000in}}%
\pgfpathlineto{\pgfqpoint{5.760000in}{4.224000in}}%
\pgfusepath{stroke}%
\end{pgfscope}%
\begin{pgfscope}%
\pgfsetrectcap%
\pgfsetmiterjoin%
\pgfsetlinewidth{0.803000pt}%
\definecolor{currentstroke}{rgb}{0.000000,0.000000,0.000000}%
\pgfsetstrokecolor{currentstroke}%
\pgfsetdash{}{0pt}%
\pgfpathmoveto{\pgfqpoint{0.800000in}{0.528000in}}%
\pgfpathlineto{\pgfqpoint{5.760000in}{0.528000in}}%
\pgfusepath{stroke}%
\end{pgfscope}%
\begin{pgfscope}%
\pgfsetrectcap%
\pgfsetmiterjoin%
\pgfsetlinewidth{0.803000pt}%
\definecolor{currentstroke}{rgb}{0.000000,0.000000,0.000000}%
\pgfsetstrokecolor{currentstroke}%
\pgfsetdash{}{0pt}%
\pgfpathmoveto{\pgfqpoint{0.800000in}{4.224000in}}%
\pgfpathlineto{\pgfqpoint{5.760000in}{4.224000in}}%
\pgfusepath{stroke}%
\end{pgfscope}%
\begin{pgfscope}%
\definecolor{textcolor}{rgb}{0.000000,0.000000,0.000000}%
\pgfsetstrokecolor{textcolor}%
\pgfsetfillcolor{textcolor}%
\pgftext[x=3.280000in,y=4.307333in,,base]{\color{textcolor}\sffamily\fontsize{12.000000}{14.400000}\selectfont Measured yaw position}%
\end{pgfscope}%
\begin{pgfscope}%
\pgfsetbuttcap%
\pgfsetmiterjoin%
\definecolor{currentfill}{rgb}{1.000000,1.000000,1.000000}%
\pgfsetfillcolor{currentfill}%
\pgfsetfillopacity{0.800000}%
\pgfsetlinewidth{1.003750pt}%
\definecolor{currentstroke}{rgb}{0.800000,0.800000,0.800000}%
\pgfsetstrokecolor{currentstroke}%
\pgfsetstrokeopacity{0.800000}%
\pgfsetdash{}{0pt}%
\pgfpathmoveto{\pgfqpoint{4.788646in}{0.597444in}}%
\pgfpathlineto{\pgfqpoint{5.662778in}{0.597444in}}%
\pgfpathquadraticcurveto{\pgfqpoint{5.690556in}{0.597444in}}{\pgfqpoint{5.690556in}{0.625222in}}%
\pgfpathlineto{\pgfqpoint{5.690556in}{2.038334in}}%
\pgfpathquadraticcurveto{\pgfqpoint{5.690556in}{2.066112in}}{\pgfqpoint{5.662778in}{2.066112in}}%
\pgfpathlineto{\pgfqpoint{4.788646in}{2.066112in}}%
\pgfpathquadraticcurveto{\pgfqpoint{4.760868in}{2.066112in}}{\pgfqpoint{4.760868in}{2.038334in}}%
\pgfpathlineto{\pgfqpoint{4.760868in}{0.625222in}}%
\pgfpathquadraticcurveto{\pgfqpoint{4.760868in}{0.597444in}}{\pgfqpoint{4.788646in}{0.597444in}}%
\pgfpathlineto{\pgfqpoint{4.788646in}{0.597444in}}%
\pgfpathclose%
\pgfusepath{stroke,fill}%
\end{pgfscope}%
\begin{pgfscope}%
\pgfsetrectcap%
\pgfsetroundjoin%
\pgfsetlinewidth{1.505625pt}%
\definecolor{currentstroke}{rgb}{0.121569,0.466667,0.705882}%
\pgfsetstrokecolor{currentstroke}%
\pgfsetdash{}{0pt}%
\pgfpathmoveto{\pgfqpoint{4.816424in}{1.953644in}}%
\pgfpathlineto{\pgfqpoint{4.955312in}{1.953644in}}%
\pgfpathlineto{\pgfqpoint{5.094201in}{1.953644in}}%
\pgfusepath{stroke}%
\end{pgfscope}%
\begin{pgfscope}%
\definecolor{textcolor}{rgb}{0.000000,0.000000,0.000000}%
\pgfsetstrokecolor{textcolor}%
\pgfsetfillcolor{textcolor}%
\pgftext[x=5.205312in,y=1.905033in,left,base]{\color{textcolor}\sffamily\fontsize{10.000000}{12.000000}\selectfont 0}%
\end{pgfscope}%
\begin{pgfscope}%
\pgfsetrectcap%
\pgfsetroundjoin%
\pgfsetlinewidth{1.505625pt}%
\definecolor{currentstroke}{rgb}{1.000000,0.498039,0.054902}%
\pgfsetstrokecolor{currentstroke}%
\pgfsetdash{}{0pt}%
\pgfpathmoveto{\pgfqpoint{4.816424in}{1.749787in}}%
\pgfpathlineto{\pgfqpoint{4.955312in}{1.749787in}}%
\pgfpathlineto{\pgfqpoint{5.094201in}{1.749787in}}%
\pgfusepath{stroke}%
\end{pgfscope}%
\begin{pgfscope}%
\definecolor{textcolor}{rgb}{0.000000,0.000000,0.000000}%
\pgfsetstrokecolor{textcolor}%
\pgfsetfillcolor{textcolor}%
\pgftext[x=5.205312in,y=1.701176in,left,base]{\color{textcolor}\sffamily\fontsize{10.000000}{12.000000}\selectfont 10}%
\end{pgfscope}%
\begin{pgfscope}%
\pgfsetrectcap%
\pgfsetroundjoin%
\pgfsetlinewidth{1.505625pt}%
\definecolor{currentstroke}{rgb}{0.172549,0.627451,0.172549}%
\pgfsetstrokecolor{currentstroke}%
\pgfsetdash{}{0pt}%
\pgfpathmoveto{\pgfqpoint{4.816424in}{1.545930in}}%
\pgfpathlineto{\pgfqpoint{4.955312in}{1.545930in}}%
\pgfpathlineto{\pgfqpoint{5.094201in}{1.545930in}}%
\pgfusepath{stroke}%
\end{pgfscope}%
\begin{pgfscope}%
\definecolor{textcolor}{rgb}{0.000000,0.000000,0.000000}%
\pgfsetstrokecolor{textcolor}%
\pgfsetfillcolor{textcolor}%
\pgftext[x=5.205312in,y=1.497319in,left,base]{\color{textcolor}\sffamily\fontsize{10.000000}{12.000000}\selectfont 20}%
\end{pgfscope}%
\begin{pgfscope}%
\pgfsetrectcap%
\pgfsetroundjoin%
\pgfsetlinewidth{1.505625pt}%
\definecolor{currentstroke}{rgb}{0.839216,0.152941,0.156863}%
\pgfsetstrokecolor{currentstroke}%
\pgfsetdash{}{0pt}%
\pgfpathmoveto{\pgfqpoint{4.816424in}{1.342073in}}%
\pgfpathlineto{\pgfqpoint{4.955312in}{1.342073in}}%
\pgfpathlineto{\pgfqpoint{5.094201in}{1.342073in}}%
\pgfusepath{stroke}%
\end{pgfscope}%
\begin{pgfscope}%
\definecolor{textcolor}{rgb}{0.000000,0.000000,0.000000}%
\pgfsetstrokecolor{textcolor}%
\pgfsetfillcolor{textcolor}%
\pgftext[x=5.205312in,y=1.293461in,left,base]{\color{textcolor}\sffamily\fontsize{10.000000}{12.000000}\selectfont 30}%
\end{pgfscope}%
\begin{pgfscope}%
\pgfsetrectcap%
\pgfsetroundjoin%
\pgfsetlinewidth{1.505625pt}%
\definecolor{currentstroke}{rgb}{0.580392,0.403922,0.741176}%
\pgfsetstrokecolor{currentstroke}%
\pgfsetdash{}{0pt}%
\pgfpathmoveto{\pgfqpoint{4.816424in}{1.138215in}}%
\pgfpathlineto{\pgfqpoint{4.955312in}{1.138215in}}%
\pgfpathlineto{\pgfqpoint{5.094201in}{1.138215in}}%
\pgfusepath{stroke}%
\end{pgfscope}%
\begin{pgfscope}%
\definecolor{textcolor}{rgb}{0.000000,0.000000,0.000000}%
\pgfsetstrokecolor{textcolor}%
\pgfsetfillcolor{textcolor}%
\pgftext[x=5.205312in,y=1.089604in,left,base]{\color{textcolor}\sffamily\fontsize{10.000000}{12.000000}\selectfont 40}%
\end{pgfscope}%
\begin{pgfscope}%
\pgfsetrectcap%
\pgfsetroundjoin%
\pgfsetlinewidth{1.505625pt}%
\definecolor{currentstroke}{rgb}{0.549020,0.337255,0.294118}%
\pgfsetstrokecolor{currentstroke}%
\pgfsetdash{}{0pt}%
\pgfpathmoveto{\pgfqpoint{4.816424in}{0.934358in}}%
\pgfpathlineto{\pgfqpoint{4.955312in}{0.934358in}}%
\pgfpathlineto{\pgfqpoint{5.094201in}{0.934358in}}%
\pgfusepath{stroke}%
\end{pgfscope}%
\begin{pgfscope}%
\definecolor{textcolor}{rgb}{0.000000,0.000000,0.000000}%
\pgfsetstrokecolor{textcolor}%
\pgfsetfillcolor{textcolor}%
\pgftext[x=5.205312in,y=0.885747in,left,base]{\color{textcolor}\sffamily\fontsize{10.000000}{12.000000}\selectfont 50}%
\end{pgfscope}%
\begin{pgfscope}%
\pgfsetrectcap%
\pgfsetroundjoin%
\pgfsetlinewidth{1.505625pt}%
\definecolor{currentstroke}{rgb}{0.890196,0.466667,0.760784}%
\pgfsetstrokecolor{currentstroke}%
\pgfsetdash{}{0pt}%
\pgfpathmoveto{\pgfqpoint{4.816424in}{0.730501in}}%
\pgfpathlineto{\pgfqpoint{4.955312in}{0.730501in}}%
\pgfpathlineto{\pgfqpoint{5.094201in}{0.730501in}}%
\pgfusepath{stroke}%
\end{pgfscope}%
\begin{pgfscope}%
\definecolor{textcolor}{rgb}{0.000000,0.000000,0.000000}%
\pgfsetstrokecolor{textcolor}%
\pgfsetfillcolor{textcolor}%
\pgftext[x=5.205312in,y=0.681890in,left,base]{\color{textcolor}\sffamily\fontsize{10.000000}{12.000000}\selectfont Target}%
\end{pgfscope}%
\end{pgfpicture}%
\makeatother%
\endgroup%
}
    \end{minipage}
    \begin{minipage}[t]{0.5\linewidth}
        \centering
        \scalebox{0.55}{%% Creator: Matplotlib, PGF backend
%%
%% To include the figure in your LaTeX document, write
%%   \input{<filename>.pgf}
%%
%% Make sure the required packages are loaded in your preamble
%%   \usepackage{pgf}
%%
%% Also ensure that all the required font packages are loaded; for instance,
%% the lmodern package is sometimes necessary when using math font.
%%   \usepackage{lmodern}
%%
%% Figures using additional raster images can only be included by \input if
%% they are in the same directory as the main LaTeX file. For loading figures
%% from other directories you can use the `import` package
%%   \usepackage{import}
%%
%% and then include the figures with
%%   \import{<path to file>}{<filename>.pgf}
%%
%% Matplotlib used the following preamble
%%   \usepackage{fontspec}
%%   \setmainfont{DejaVuSerif.ttf}[Path=\detokenize{/home/lgonz/tfg-aero/tfg-giaa-dronecontrol/venv/lib/python3.8/site-packages/matplotlib/mpl-data/fonts/ttf/}]
%%   \setsansfont{DejaVuSans.ttf}[Path=\detokenize{/home/lgonz/tfg-aero/tfg-giaa-dronecontrol/venv/lib/python3.8/site-packages/matplotlib/mpl-data/fonts/ttf/}]
%%   \setmonofont{DejaVuSansMono.ttf}[Path=\detokenize{/home/lgonz/tfg-aero/tfg-giaa-dronecontrol/venv/lib/python3.8/site-packages/matplotlib/mpl-data/fonts/ttf/}]
%%
\begingroup%
\makeatletter%
\begin{pgfpicture}%
\pgfpathrectangle{\pgfpointorigin}{\pgfqpoint{6.400000in}{4.800000in}}%
\pgfusepath{use as bounding box, clip}%
\begin{pgfscope}%
\pgfsetbuttcap%
\pgfsetmiterjoin%
\definecolor{currentfill}{rgb}{1.000000,1.000000,1.000000}%
\pgfsetfillcolor{currentfill}%
\pgfsetlinewidth{0.000000pt}%
\definecolor{currentstroke}{rgb}{1.000000,1.000000,1.000000}%
\pgfsetstrokecolor{currentstroke}%
\pgfsetdash{}{0pt}%
\pgfpathmoveto{\pgfqpoint{0.000000in}{0.000000in}}%
\pgfpathlineto{\pgfqpoint{6.400000in}{0.000000in}}%
\pgfpathlineto{\pgfqpoint{6.400000in}{4.800000in}}%
\pgfpathlineto{\pgfqpoint{0.000000in}{4.800000in}}%
\pgfpathlineto{\pgfqpoint{0.000000in}{0.000000in}}%
\pgfpathclose%
\pgfusepath{fill}%
\end{pgfscope}%
\begin{pgfscope}%
\pgfsetbuttcap%
\pgfsetmiterjoin%
\definecolor{currentfill}{rgb}{1.000000,1.000000,1.000000}%
\pgfsetfillcolor{currentfill}%
\pgfsetlinewidth{0.000000pt}%
\definecolor{currentstroke}{rgb}{0.000000,0.000000,0.000000}%
\pgfsetstrokecolor{currentstroke}%
\pgfsetstrokeopacity{0.000000}%
\pgfsetdash{}{0pt}%
\pgfpathmoveto{\pgfqpoint{0.800000in}{0.528000in}}%
\pgfpathlineto{\pgfqpoint{5.760000in}{0.528000in}}%
\pgfpathlineto{\pgfqpoint{5.760000in}{4.224000in}}%
\pgfpathlineto{\pgfqpoint{0.800000in}{4.224000in}}%
\pgfpathlineto{\pgfqpoint{0.800000in}{0.528000in}}%
\pgfpathclose%
\pgfusepath{fill}%
\end{pgfscope}%
\begin{pgfscope}%
\pgfpathrectangle{\pgfqpoint{0.800000in}{0.528000in}}{\pgfqpoint{4.960000in}{3.696000in}}%
\pgfusepath{clip}%
\pgfsetrectcap%
\pgfsetroundjoin%
\pgfsetlinewidth{0.803000pt}%
\definecolor{currentstroke}{rgb}{0.690196,0.690196,0.690196}%
\pgfsetstrokecolor{currentstroke}%
\pgfsetdash{}{0pt}%
\pgfpathmoveto{\pgfqpoint{1.025455in}{0.528000in}}%
\pgfpathlineto{\pgfqpoint{1.025455in}{4.224000in}}%
\pgfusepath{stroke}%
\end{pgfscope}%
\begin{pgfscope}%
\pgfsetbuttcap%
\pgfsetroundjoin%
\definecolor{currentfill}{rgb}{0.000000,0.000000,0.000000}%
\pgfsetfillcolor{currentfill}%
\pgfsetlinewidth{0.803000pt}%
\definecolor{currentstroke}{rgb}{0.000000,0.000000,0.000000}%
\pgfsetstrokecolor{currentstroke}%
\pgfsetdash{}{0pt}%
\pgfsys@defobject{currentmarker}{\pgfqpoint{0.000000in}{-0.048611in}}{\pgfqpoint{0.000000in}{0.000000in}}{%
\pgfpathmoveto{\pgfqpoint{0.000000in}{0.000000in}}%
\pgfpathlineto{\pgfqpoint{0.000000in}{-0.048611in}}%
\pgfusepath{stroke,fill}%
}%
\begin{pgfscope}%
\pgfsys@transformshift{1.025455in}{0.528000in}%
\pgfsys@useobject{currentmarker}{}%
\end{pgfscope}%
\end{pgfscope}%
\begin{pgfscope}%
\definecolor{textcolor}{rgb}{0.000000,0.000000,0.000000}%
\pgfsetstrokecolor{textcolor}%
\pgfsetfillcolor{textcolor}%
\pgftext[x=1.025455in,y=0.430778in,,top]{\color{textcolor}\sffamily\fontsize{10.000000}{12.000000}\selectfont 0}%
\end{pgfscope}%
\begin{pgfscope}%
\pgfpathrectangle{\pgfqpoint{0.800000in}{0.528000in}}{\pgfqpoint{4.960000in}{3.696000in}}%
\pgfusepath{clip}%
\pgfsetrectcap%
\pgfsetroundjoin%
\pgfsetlinewidth{0.803000pt}%
\definecolor{currentstroke}{rgb}{0.690196,0.690196,0.690196}%
\pgfsetstrokecolor{currentstroke}%
\pgfsetdash{}{0pt}%
\pgfpathmoveto{\pgfqpoint{1.586289in}{0.528000in}}%
\pgfpathlineto{\pgfqpoint{1.586289in}{4.224000in}}%
\pgfusepath{stroke}%
\end{pgfscope}%
\begin{pgfscope}%
\pgfsetbuttcap%
\pgfsetroundjoin%
\definecolor{currentfill}{rgb}{0.000000,0.000000,0.000000}%
\pgfsetfillcolor{currentfill}%
\pgfsetlinewidth{0.803000pt}%
\definecolor{currentstroke}{rgb}{0.000000,0.000000,0.000000}%
\pgfsetstrokecolor{currentstroke}%
\pgfsetdash{}{0pt}%
\pgfsys@defobject{currentmarker}{\pgfqpoint{0.000000in}{-0.048611in}}{\pgfqpoint{0.000000in}{0.000000in}}{%
\pgfpathmoveto{\pgfqpoint{0.000000in}{0.000000in}}%
\pgfpathlineto{\pgfqpoint{0.000000in}{-0.048611in}}%
\pgfusepath{stroke,fill}%
}%
\begin{pgfscope}%
\pgfsys@transformshift{1.586289in}{0.528000in}%
\pgfsys@useobject{currentmarker}{}%
\end{pgfscope}%
\end{pgfscope}%
\begin{pgfscope}%
\definecolor{textcolor}{rgb}{0.000000,0.000000,0.000000}%
\pgfsetstrokecolor{textcolor}%
\pgfsetfillcolor{textcolor}%
\pgftext[x=1.586289in,y=0.430778in,,top]{\color{textcolor}\sffamily\fontsize{10.000000}{12.000000}\selectfont 5}%
\end{pgfscope}%
\begin{pgfscope}%
\pgfpathrectangle{\pgfqpoint{0.800000in}{0.528000in}}{\pgfqpoint{4.960000in}{3.696000in}}%
\pgfusepath{clip}%
\pgfsetrectcap%
\pgfsetroundjoin%
\pgfsetlinewidth{0.803000pt}%
\definecolor{currentstroke}{rgb}{0.690196,0.690196,0.690196}%
\pgfsetstrokecolor{currentstroke}%
\pgfsetdash{}{0pt}%
\pgfpathmoveto{\pgfqpoint{2.147123in}{0.528000in}}%
\pgfpathlineto{\pgfqpoint{2.147123in}{4.224000in}}%
\pgfusepath{stroke}%
\end{pgfscope}%
\begin{pgfscope}%
\pgfsetbuttcap%
\pgfsetroundjoin%
\definecolor{currentfill}{rgb}{0.000000,0.000000,0.000000}%
\pgfsetfillcolor{currentfill}%
\pgfsetlinewidth{0.803000pt}%
\definecolor{currentstroke}{rgb}{0.000000,0.000000,0.000000}%
\pgfsetstrokecolor{currentstroke}%
\pgfsetdash{}{0pt}%
\pgfsys@defobject{currentmarker}{\pgfqpoint{0.000000in}{-0.048611in}}{\pgfqpoint{0.000000in}{0.000000in}}{%
\pgfpathmoveto{\pgfqpoint{0.000000in}{0.000000in}}%
\pgfpathlineto{\pgfqpoint{0.000000in}{-0.048611in}}%
\pgfusepath{stroke,fill}%
}%
\begin{pgfscope}%
\pgfsys@transformshift{2.147123in}{0.528000in}%
\pgfsys@useobject{currentmarker}{}%
\end{pgfscope}%
\end{pgfscope}%
\begin{pgfscope}%
\definecolor{textcolor}{rgb}{0.000000,0.000000,0.000000}%
\pgfsetstrokecolor{textcolor}%
\pgfsetfillcolor{textcolor}%
\pgftext[x=2.147123in,y=0.430778in,,top]{\color{textcolor}\sffamily\fontsize{10.000000}{12.000000}\selectfont 10}%
\end{pgfscope}%
\begin{pgfscope}%
\pgfpathrectangle{\pgfqpoint{0.800000in}{0.528000in}}{\pgfqpoint{4.960000in}{3.696000in}}%
\pgfusepath{clip}%
\pgfsetrectcap%
\pgfsetroundjoin%
\pgfsetlinewidth{0.803000pt}%
\definecolor{currentstroke}{rgb}{0.690196,0.690196,0.690196}%
\pgfsetstrokecolor{currentstroke}%
\pgfsetdash{}{0pt}%
\pgfpathmoveto{\pgfqpoint{2.707957in}{0.528000in}}%
\pgfpathlineto{\pgfqpoint{2.707957in}{4.224000in}}%
\pgfusepath{stroke}%
\end{pgfscope}%
\begin{pgfscope}%
\pgfsetbuttcap%
\pgfsetroundjoin%
\definecolor{currentfill}{rgb}{0.000000,0.000000,0.000000}%
\pgfsetfillcolor{currentfill}%
\pgfsetlinewidth{0.803000pt}%
\definecolor{currentstroke}{rgb}{0.000000,0.000000,0.000000}%
\pgfsetstrokecolor{currentstroke}%
\pgfsetdash{}{0pt}%
\pgfsys@defobject{currentmarker}{\pgfqpoint{0.000000in}{-0.048611in}}{\pgfqpoint{0.000000in}{0.000000in}}{%
\pgfpathmoveto{\pgfqpoint{0.000000in}{0.000000in}}%
\pgfpathlineto{\pgfqpoint{0.000000in}{-0.048611in}}%
\pgfusepath{stroke,fill}%
}%
\begin{pgfscope}%
\pgfsys@transformshift{2.707957in}{0.528000in}%
\pgfsys@useobject{currentmarker}{}%
\end{pgfscope}%
\end{pgfscope}%
\begin{pgfscope}%
\definecolor{textcolor}{rgb}{0.000000,0.000000,0.000000}%
\pgfsetstrokecolor{textcolor}%
\pgfsetfillcolor{textcolor}%
\pgftext[x=2.707957in,y=0.430778in,,top]{\color{textcolor}\sffamily\fontsize{10.000000}{12.000000}\selectfont 15}%
\end{pgfscope}%
\begin{pgfscope}%
\pgfpathrectangle{\pgfqpoint{0.800000in}{0.528000in}}{\pgfqpoint{4.960000in}{3.696000in}}%
\pgfusepath{clip}%
\pgfsetrectcap%
\pgfsetroundjoin%
\pgfsetlinewidth{0.803000pt}%
\definecolor{currentstroke}{rgb}{0.690196,0.690196,0.690196}%
\pgfsetstrokecolor{currentstroke}%
\pgfsetdash{}{0pt}%
\pgfpathmoveto{\pgfqpoint{3.268791in}{0.528000in}}%
\pgfpathlineto{\pgfqpoint{3.268791in}{4.224000in}}%
\pgfusepath{stroke}%
\end{pgfscope}%
\begin{pgfscope}%
\pgfsetbuttcap%
\pgfsetroundjoin%
\definecolor{currentfill}{rgb}{0.000000,0.000000,0.000000}%
\pgfsetfillcolor{currentfill}%
\pgfsetlinewidth{0.803000pt}%
\definecolor{currentstroke}{rgb}{0.000000,0.000000,0.000000}%
\pgfsetstrokecolor{currentstroke}%
\pgfsetdash{}{0pt}%
\pgfsys@defobject{currentmarker}{\pgfqpoint{0.000000in}{-0.048611in}}{\pgfqpoint{0.000000in}{0.000000in}}{%
\pgfpathmoveto{\pgfqpoint{0.000000in}{0.000000in}}%
\pgfpathlineto{\pgfqpoint{0.000000in}{-0.048611in}}%
\pgfusepath{stroke,fill}%
}%
\begin{pgfscope}%
\pgfsys@transformshift{3.268791in}{0.528000in}%
\pgfsys@useobject{currentmarker}{}%
\end{pgfscope}%
\end{pgfscope}%
\begin{pgfscope}%
\definecolor{textcolor}{rgb}{0.000000,0.000000,0.000000}%
\pgfsetstrokecolor{textcolor}%
\pgfsetfillcolor{textcolor}%
\pgftext[x=3.268791in,y=0.430778in,,top]{\color{textcolor}\sffamily\fontsize{10.000000}{12.000000}\selectfont 20}%
\end{pgfscope}%
\begin{pgfscope}%
\pgfpathrectangle{\pgfqpoint{0.800000in}{0.528000in}}{\pgfqpoint{4.960000in}{3.696000in}}%
\pgfusepath{clip}%
\pgfsetrectcap%
\pgfsetroundjoin%
\pgfsetlinewidth{0.803000pt}%
\definecolor{currentstroke}{rgb}{0.690196,0.690196,0.690196}%
\pgfsetstrokecolor{currentstroke}%
\pgfsetdash{}{0pt}%
\pgfpathmoveto{\pgfqpoint{3.829625in}{0.528000in}}%
\pgfpathlineto{\pgfqpoint{3.829625in}{4.224000in}}%
\pgfusepath{stroke}%
\end{pgfscope}%
\begin{pgfscope}%
\pgfsetbuttcap%
\pgfsetroundjoin%
\definecolor{currentfill}{rgb}{0.000000,0.000000,0.000000}%
\pgfsetfillcolor{currentfill}%
\pgfsetlinewidth{0.803000pt}%
\definecolor{currentstroke}{rgb}{0.000000,0.000000,0.000000}%
\pgfsetstrokecolor{currentstroke}%
\pgfsetdash{}{0pt}%
\pgfsys@defobject{currentmarker}{\pgfqpoint{0.000000in}{-0.048611in}}{\pgfqpoint{0.000000in}{0.000000in}}{%
\pgfpathmoveto{\pgfqpoint{0.000000in}{0.000000in}}%
\pgfpathlineto{\pgfqpoint{0.000000in}{-0.048611in}}%
\pgfusepath{stroke,fill}%
}%
\begin{pgfscope}%
\pgfsys@transformshift{3.829625in}{0.528000in}%
\pgfsys@useobject{currentmarker}{}%
\end{pgfscope}%
\end{pgfscope}%
\begin{pgfscope}%
\definecolor{textcolor}{rgb}{0.000000,0.000000,0.000000}%
\pgfsetstrokecolor{textcolor}%
\pgfsetfillcolor{textcolor}%
\pgftext[x=3.829625in,y=0.430778in,,top]{\color{textcolor}\sffamily\fontsize{10.000000}{12.000000}\selectfont 25}%
\end{pgfscope}%
\begin{pgfscope}%
\pgfpathrectangle{\pgfqpoint{0.800000in}{0.528000in}}{\pgfqpoint{4.960000in}{3.696000in}}%
\pgfusepath{clip}%
\pgfsetrectcap%
\pgfsetroundjoin%
\pgfsetlinewidth{0.803000pt}%
\definecolor{currentstroke}{rgb}{0.690196,0.690196,0.690196}%
\pgfsetstrokecolor{currentstroke}%
\pgfsetdash{}{0pt}%
\pgfpathmoveto{\pgfqpoint{4.390459in}{0.528000in}}%
\pgfpathlineto{\pgfqpoint{4.390459in}{4.224000in}}%
\pgfusepath{stroke}%
\end{pgfscope}%
\begin{pgfscope}%
\pgfsetbuttcap%
\pgfsetroundjoin%
\definecolor{currentfill}{rgb}{0.000000,0.000000,0.000000}%
\pgfsetfillcolor{currentfill}%
\pgfsetlinewidth{0.803000pt}%
\definecolor{currentstroke}{rgb}{0.000000,0.000000,0.000000}%
\pgfsetstrokecolor{currentstroke}%
\pgfsetdash{}{0pt}%
\pgfsys@defobject{currentmarker}{\pgfqpoint{0.000000in}{-0.048611in}}{\pgfqpoint{0.000000in}{0.000000in}}{%
\pgfpathmoveto{\pgfqpoint{0.000000in}{0.000000in}}%
\pgfpathlineto{\pgfqpoint{0.000000in}{-0.048611in}}%
\pgfusepath{stroke,fill}%
}%
\begin{pgfscope}%
\pgfsys@transformshift{4.390459in}{0.528000in}%
\pgfsys@useobject{currentmarker}{}%
\end{pgfscope}%
\end{pgfscope}%
\begin{pgfscope}%
\definecolor{textcolor}{rgb}{0.000000,0.000000,0.000000}%
\pgfsetstrokecolor{textcolor}%
\pgfsetfillcolor{textcolor}%
\pgftext[x=4.390459in,y=0.430778in,,top]{\color{textcolor}\sffamily\fontsize{10.000000}{12.000000}\selectfont 30}%
\end{pgfscope}%
\begin{pgfscope}%
\pgfpathrectangle{\pgfqpoint{0.800000in}{0.528000in}}{\pgfqpoint{4.960000in}{3.696000in}}%
\pgfusepath{clip}%
\pgfsetrectcap%
\pgfsetroundjoin%
\pgfsetlinewidth{0.803000pt}%
\definecolor{currentstroke}{rgb}{0.690196,0.690196,0.690196}%
\pgfsetstrokecolor{currentstroke}%
\pgfsetdash{}{0pt}%
\pgfpathmoveto{\pgfqpoint{4.951293in}{0.528000in}}%
\pgfpathlineto{\pgfqpoint{4.951293in}{4.224000in}}%
\pgfusepath{stroke}%
\end{pgfscope}%
\begin{pgfscope}%
\pgfsetbuttcap%
\pgfsetroundjoin%
\definecolor{currentfill}{rgb}{0.000000,0.000000,0.000000}%
\pgfsetfillcolor{currentfill}%
\pgfsetlinewidth{0.803000pt}%
\definecolor{currentstroke}{rgb}{0.000000,0.000000,0.000000}%
\pgfsetstrokecolor{currentstroke}%
\pgfsetdash{}{0pt}%
\pgfsys@defobject{currentmarker}{\pgfqpoint{0.000000in}{-0.048611in}}{\pgfqpoint{0.000000in}{0.000000in}}{%
\pgfpathmoveto{\pgfqpoint{0.000000in}{0.000000in}}%
\pgfpathlineto{\pgfqpoint{0.000000in}{-0.048611in}}%
\pgfusepath{stroke,fill}%
}%
\begin{pgfscope}%
\pgfsys@transformshift{4.951293in}{0.528000in}%
\pgfsys@useobject{currentmarker}{}%
\end{pgfscope}%
\end{pgfscope}%
\begin{pgfscope}%
\definecolor{textcolor}{rgb}{0.000000,0.000000,0.000000}%
\pgfsetstrokecolor{textcolor}%
\pgfsetfillcolor{textcolor}%
\pgftext[x=4.951293in,y=0.430778in,,top]{\color{textcolor}\sffamily\fontsize{10.000000}{12.000000}\selectfont 35}%
\end{pgfscope}%
\begin{pgfscope}%
\pgfpathrectangle{\pgfqpoint{0.800000in}{0.528000in}}{\pgfqpoint{4.960000in}{3.696000in}}%
\pgfusepath{clip}%
\pgfsetrectcap%
\pgfsetroundjoin%
\pgfsetlinewidth{0.803000pt}%
\definecolor{currentstroke}{rgb}{0.690196,0.690196,0.690196}%
\pgfsetstrokecolor{currentstroke}%
\pgfsetdash{}{0pt}%
\pgfpathmoveto{\pgfqpoint{5.512127in}{0.528000in}}%
\pgfpathlineto{\pgfqpoint{5.512127in}{4.224000in}}%
\pgfusepath{stroke}%
\end{pgfscope}%
\begin{pgfscope}%
\pgfsetbuttcap%
\pgfsetroundjoin%
\definecolor{currentfill}{rgb}{0.000000,0.000000,0.000000}%
\pgfsetfillcolor{currentfill}%
\pgfsetlinewidth{0.803000pt}%
\definecolor{currentstroke}{rgb}{0.000000,0.000000,0.000000}%
\pgfsetstrokecolor{currentstroke}%
\pgfsetdash{}{0pt}%
\pgfsys@defobject{currentmarker}{\pgfqpoint{0.000000in}{-0.048611in}}{\pgfqpoint{0.000000in}{0.000000in}}{%
\pgfpathmoveto{\pgfqpoint{0.000000in}{0.000000in}}%
\pgfpathlineto{\pgfqpoint{0.000000in}{-0.048611in}}%
\pgfusepath{stroke,fill}%
}%
\begin{pgfscope}%
\pgfsys@transformshift{5.512127in}{0.528000in}%
\pgfsys@useobject{currentmarker}{}%
\end{pgfscope}%
\end{pgfscope}%
\begin{pgfscope}%
\definecolor{textcolor}{rgb}{0.000000,0.000000,0.000000}%
\pgfsetstrokecolor{textcolor}%
\pgfsetfillcolor{textcolor}%
\pgftext[x=5.512127in,y=0.430778in,,top]{\color{textcolor}\sffamily\fontsize{10.000000}{12.000000}\selectfont 40}%
\end{pgfscope}%
\begin{pgfscope}%
\definecolor{textcolor}{rgb}{0.000000,0.000000,0.000000}%
\pgfsetstrokecolor{textcolor}%
\pgfsetfillcolor{textcolor}%
\pgftext[x=3.280000in,y=0.240809in,,top]{\color{textcolor}\sffamily\fontsize{10.000000}{12.000000}\selectfont time [s]}%
\end{pgfscope}%
\begin{pgfscope}%
\pgfpathrectangle{\pgfqpoint{0.800000in}{0.528000in}}{\pgfqpoint{4.960000in}{3.696000in}}%
\pgfusepath{clip}%
\pgfsetrectcap%
\pgfsetroundjoin%
\pgfsetlinewidth{0.803000pt}%
\definecolor{currentstroke}{rgb}{0.690196,0.690196,0.690196}%
\pgfsetstrokecolor{currentstroke}%
\pgfsetdash{}{0pt}%
\pgfpathmoveto{\pgfqpoint{0.800000in}{0.604844in}}%
\pgfpathlineto{\pgfqpoint{5.760000in}{0.604844in}}%
\pgfusepath{stroke}%
\end{pgfscope}%
\begin{pgfscope}%
\pgfsetbuttcap%
\pgfsetroundjoin%
\definecolor{currentfill}{rgb}{0.000000,0.000000,0.000000}%
\pgfsetfillcolor{currentfill}%
\pgfsetlinewidth{0.803000pt}%
\definecolor{currentstroke}{rgb}{0.000000,0.000000,0.000000}%
\pgfsetstrokecolor{currentstroke}%
\pgfsetdash{}{0pt}%
\pgfsys@defobject{currentmarker}{\pgfqpoint{-0.048611in}{0.000000in}}{\pgfqpoint{-0.000000in}{0.000000in}}{%
\pgfpathmoveto{\pgfqpoint{-0.000000in}{0.000000in}}%
\pgfpathlineto{\pgfqpoint{-0.048611in}{0.000000in}}%
\pgfusepath{stroke,fill}%
}%
\begin{pgfscope}%
\pgfsys@transformshift{0.800000in}{0.604844in}%
\pgfsys@useobject{currentmarker}{}%
\end{pgfscope}%
\end{pgfscope}%
\begin{pgfscope}%
\definecolor{textcolor}{rgb}{0.000000,0.000000,0.000000}%
\pgfsetstrokecolor{textcolor}%
\pgfsetfillcolor{textcolor}%
\pgftext[x=0.506387in, y=0.552082in, left, base]{\color{textcolor}\sffamily\fontsize{10.000000}{12.000000}\selectfont \ensuremath{-}2}%
\end{pgfscope}%
\begin{pgfscope}%
\pgfpathrectangle{\pgfqpoint{0.800000in}{0.528000in}}{\pgfqpoint{4.960000in}{3.696000in}}%
\pgfusepath{clip}%
\pgfsetrectcap%
\pgfsetroundjoin%
\pgfsetlinewidth{0.803000pt}%
\definecolor{currentstroke}{rgb}{0.690196,0.690196,0.690196}%
\pgfsetstrokecolor{currentstroke}%
\pgfsetdash{}{0pt}%
\pgfpathmoveto{\pgfqpoint{0.800000in}{1.058071in}}%
\pgfpathlineto{\pgfqpoint{5.760000in}{1.058071in}}%
\pgfusepath{stroke}%
\end{pgfscope}%
\begin{pgfscope}%
\pgfsetbuttcap%
\pgfsetroundjoin%
\definecolor{currentfill}{rgb}{0.000000,0.000000,0.000000}%
\pgfsetfillcolor{currentfill}%
\pgfsetlinewidth{0.803000pt}%
\definecolor{currentstroke}{rgb}{0.000000,0.000000,0.000000}%
\pgfsetstrokecolor{currentstroke}%
\pgfsetdash{}{0pt}%
\pgfsys@defobject{currentmarker}{\pgfqpoint{-0.048611in}{0.000000in}}{\pgfqpoint{-0.000000in}{0.000000in}}{%
\pgfpathmoveto{\pgfqpoint{-0.000000in}{0.000000in}}%
\pgfpathlineto{\pgfqpoint{-0.048611in}{0.000000in}}%
\pgfusepath{stroke,fill}%
}%
\begin{pgfscope}%
\pgfsys@transformshift{0.800000in}{1.058071in}%
\pgfsys@useobject{currentmarker}{}%
\end{pgfscope}%
\end{pgfscope}%
\begin{pgfscope}%
\definecolor{textcolor}{rgb}{0.000000,0.000000,0.000000}%
\pgfsetstrokecolor{textcolor}%
\pgfsetfillcolor{textcolor}%
\pgftext[x=0.506387in, y=1.005309in, left, base]{\color{textcolor}\sffamily\fontsize{10.000000}{12.000000}\selectfont \ensuremath{-}1}%
\end{pgfscope}%
\begin{pgfscope}%
\pgfpathrectangle{\pgfqpoint{0.800000in}{0.528000in}}{\pgfqpoint{4.960000in}{3.696000in}}%
\pgfusepath{clip}%
\pgfsetrectcap%
\pgfsetroundjoin%
\pgfsetlinewidth{0.803000pt}%
\definecolor{currentstroke}{rgb}{0.690196,0.690196,0.690196}%
\pgfsetstrokecolor{currentstroke}%
\pgfsetdash{}{0pt}%
\pgfpathmoveto{\pgfqpoint{0.800000in}{1.511297in}}%
\pgfpathlineto{\pgfqpoint{5.760000in}{1.511297in}}%
\pgfusepath{stroke}%
\end{pgfscope}%
\begin{pgfscope}%
\pgfsetbuttcap%
\pgfsetroundjoin%
\definecolor{currentfill}{rgb}{0.000000,0.000000,0.000000}%
\pgfsetfillcolor{currentfill}%
\pgfsetlinewidth{0.803000pt}%
\definecolor{currentstroke}{rgb}{0.000000,0.000000,0.000000}%
\pgfsetstrokecolor{currentstroke}%
\pgfsetdash{}{0pt}%
\pgfsys@defobject{currentmarker}{\pgfqpoint{-0.048611in}{0.000000in}}{\pgfqpoint{-0.000000in}{0.000000in}}{%
\pgfpathmoveto{\pgfqpoint{-0.000000in}{0.000000in}}%
\pgfpathlineto{\pgfqpoint{-0.048611in}{0.000000in}}%
\pgfusepath{stroke,fill}%
}%
\begin{pgfscope}%
\pgfsys@transformshift{0.800000in}{1.511297in}%
\pgfsys@useobject{currentmarker}{}%
\end{pgfscope}%
\end{pgfscope}%
\begin{pgfscope}%
\definecolor{textcolor}{rgb}{0.000000,0.000000,0.000000}%
\pgfsetstrokecolor{textcolor}%
\pgfsetfillcolor{textcolor}%
\pgftext[x=0.614412in, y=1.458536in, left, base]{\color{textcolor}\sffamily\fontsize{10.000000}{12.000000}\selectfont 0}%
\end{pgfscope}%
\begin{pgfscope}%
\pgfpathrectangle{\pgfqpoint{0.800000in}{0.528000in}}{\pgfqpoint{4.960000in}{3.696000in}}%
\pgfusepath{clip}%
\pgfsetrectcap%
\pgfsetroundjoin%
\pgfsetlinewidth{0.803000pt}%
\definecolor{currentstroke}{rgb}{0.690196,0.690196,0.690196}%
\pgfsetstrokecolor{currentstroke}%
\pgfsetdash{}{0pt}%
\pgfpathmoveto{\pgfqpoint{0.800000in}{1.964524in}}%
\pgfpathlineto{\pgfqpoint{5.760000in}{1.964524in}}%
\pgfusepath{stroke}%
\end{pgfscope}%
\begin{pgfscope}%
\pgfsetbuttcap%
\pgfsetroundjoin%
\definecolor{currentfill}{rgb}{0.000000,0.000000,0.000000}%
\pgfsetfillcolor{currentfill}%
\pgfsetlinewidth{0.803000pt}%
\definecolor{currentstroke}{rgb}{0.000000,0.000000,0.000000}%
\pgfsetstrokecolor{currentstroke}%
\pgfsetdash{}{0pt}%
\pgfsys@defobject{currentmarker}{\pgfqpoint{-0.048611in}{0.000000in}}{\pgfqpoint{-0.000000in}{0.000000in}}{%
\pgfpathmoveto{\pgfqpoint{-0.000000in}{0.000000in}}%
\pgfpathlineto{\pgfqpoint{-0.048611in}{0.000000in}}%
\pgfusepath{stroke,fill}%
}%
\begin{pgfscope}%
\pgfsys@transformshift{0.800000in}{1.964524in}%
\pgfsys@useobject{currentmarker}{}%
\end{pgfscope}%
\end{pgfscope}%
\begin{pgfscope}%
\definecolor{textcolor}{rgb}{0.000000,0.000000,0.000000}%
\pgfsetstrokecolor{textcolor}%
\pgfsetfillcolor{textcolor}%
\pgftext[x=0.614412in, y=1.911762in, left, base]{\color{textcolor}\sffamily\fontsize{10.000000}{12.000000}\selectfont 1}%
\end{pgfscope}%
\begin{pgfscope}%
\pgfpathrectangle{\pgfqpoint{0.800000in}{0.528000in}}{\pgfqpoint{4.960000in}{3.696000in}}%
\pgfusepath{clip}%
\pgfsetrectcap%
\pgfsetroundjoin%
\pgfsetlinewidth{0.803000pt}%
\definecolor{currentstroke}{rgb}{0.690196,0.690196,0.690196}%
\pgfsetstrokecolor{currentstroke}%
\pgfsetdash{}{0pt}%
\pgfpathmoveto{\pgfqpoint{0.800000in}{2.417750in}}%
\pgfpathlineto{\pgfqpoint{5.760000in}{2.417750in}}%
\pgfusepath{stroke}%
\end{pgfscope}%
\begin{pgfscope}%
\pgfsetbuttcap%
\pgfsetroundjoin%
\definecolor{currentfill}{rgb}{0.000000,0.000000,0.000000}%
\pgfsetfillcolor{currentfill}%
\pgfsetlinewidth{0.803000pt}%
\definecolor{currentstroke}{rgb}{0.000000,0.000000,0.000000}%
\pgfsetstrokecolor{currentstroke}%
\pgfsetdash{}{0pt}%
\pgfsys@defobject{currentmarker}{\pgfqpoint{-0.048611in}{0.000000in}}{\pgfqpoint{-0.000000in}{0.000000in}}{%
\pgfpathmoveto{\pgfqpoint{-0.000000in}{0.000000in}}%
\pgfpathlineto{\pgfqpoint{-0.048611in}{0.000000in}}%
\pgfusepath{stroke,fill}%
}%
\begin{pgfscope}%
\pgfsys@transformshift{0.800000in}{2.417750in}%
\pgfsys@useobject{currentmarker}{}%
\end{pgfscope}%
\end{pgfscope}%
\begin{pgfscope}%
\definecolor{textcolor}{rgb}{0.000000,0.000000,0.000000}%
\pgfsetstrokecolor{textcolor}%
\pgfsetfillcolor{textcolor}%
\pgftext[x=0.614412in, y=2.364989in, left, base]{\color{textcolor}\sffamily\fontsize{10.000000}{12.000000}\selectfont 2}%
\end{pgfscope}%
\begin{pgfscope}%
\pgfpathrectangle{\pgfqpoint{0.800000in}{0.528000in}}{\pgfqpoint{4.960000in}{3.696000in}}%
\pgfusepath{clip}%
\pgfsetrectcap%
\pgfsetroundjoin%
\pgfsetlinewidth{0.803000pt}%
\definecolor{currentstroke}{rgb}{0.690196,0.690196,0.690196}%
\pgfsetstrokecolor{currentstroke}%
\pgfsetdash{}{0pt}%
\pgfpathmoveto{\pgfqpoint{0.800000in}{2.870977in}}%
\pgfpathlineto{\pgfqpoint{5.760000in}{2.870977in}}%
\pgfusepath{stroke}%
\end{pgfscope}%
\begin{pgfscope}%
\pgfsetbuttcap%
\pgfsetroundjoin%
\definecolor{currentfill}{rgb}{0.000000,0.000000,0.000000}%
\pgfsetfillcolor{currentfill}%
\pgfsetlinewidth{0.803000pt}%
\definecolor{currentstroke}{rgb}{0.000000,0.000000,0.000000}%
\pgfsetstrokecolor{currentstroke}%
\pgfsetdash{}{0pt}%
\pgfsys@defobject{currentmarker}{\pgfqpoint{-0.048611in}{0.000000in}}{\pgfqpoint{-0.000000in}{0.000000in}}{%
\pgfpathmoveto{\pgfqpoint{-0.000000in}{0.000000in}}%
\pgfpathlineto{\pgfqpoint{-0.048611in}{0.000000in}}%
\pgfusepath{stroke,fill}%
}%
\begin{pgfscope}%
\pgfsys@transformshift{0.800000in}{2.870977in}%
\pgfsys@useobject{currentmarker}{}%
\end{pgfscope}%
\end{pgfscope}%
\begin{pgfscope}%
\definecolor{textcolor}{rgb}{0.000000,0.000000,0.000000}%
\pgfsetstrokecolor{textcolor}%
\pgfsetfillcolor{textcolor}%
\pgftext[x=0.614412in, y=2.818215in, left, base]{\color{textcolor}\sffamily\fontsize{10.000000}{12.000000}\selectfont 3}%
\end{pgfscope}%
\begin{pgfscope}%
\pgfpathrectangle{\pgfqpoint{0.800000in}{0.528000in}}{\pgfqpoint{4.960000in}{3.696000in}}%
\pgfusepath{clip}%
\pgfsetrectcap%
\pgfsetroundjoin%
\pgfsetlinewidth{0.803000pt}%
\definecolor{currentstroke}{rgb}{0.690196,0.690196,0.690196}%
\pgfsetstrokecolor{currentstroke}%
\pgfsetdash{}{0pt}%
\pgfpathmoveto{\pgfqpoint{0.800000in}{3.324203in}}%
\pgfpathlineto{\pgfqpoint{5.760000in}{3.324203in}}%
\pgfusepath{stroke}%
\end{pgfscope}%
\begin{pgfscope}%
\pgfsetbuttcap%
\pgfsetroundjoin%
\definecolor{currentfill}{rgb}{0.000000,0.000000,0.000000}%
\pgfsetfillcolor{currentfill}%
\pgfsetlinewidth{0.803000pt}%
\definecolor{currentstroke}{rgb}{0.000000,0.000000,0.000000}%
\pgfsetstrokecolor{currentstroke}%
\pgfsetdash{}{0pt}%
\pgfsys@defobject{currentmarker}{\pgfqpoint{-0.048611in}{0.000000in}}{\pgfqpoint{-0.000000in}{0.000000in}}{%
\pgfpathmoveto{\pgfqpoint{-0.000000in}{0.000000in}}%
\pgfpathlineto{\pgfqpoint{-0.048611in}{0.000000in}}%
\pgfusepath{stroke,fill}%
}%
\begin{pgfscope}%
\pgfsys@transformshift{0.800000in}{3.324203in}%
\pgfsys@useobject{currentmarker}{}%
\end{pgfscope}%
\end{pgfscope}%
\begin{pgfscope}%
\definecolor{textcolor}{rgb}{0.000000,0.000000,0.000000}%
\pgfsetstrokecolor{textcolor}%
\pgfsetfillcolor{textcolor}%
\pgftext[x=0.614412in, y=3.271442in, left, base]{\color{textcolor}\sffamily\fontsize{10.000000}{12.000000}\selectfont 4}%
\end{pgfscope}%
\begin{pgfscope}%
\pgfpathrectangle{\pgfqpoint{0.800000in}{0.528000in}}{\pgfqpoint{4.960000in}{3.696000in}}%
\pgfusepath{clip}%
\pgfsetrectcap%
\pgfsetroundjoin%
\pgfsetlinewidth{0.803000pt}%
\definecolor{currentstroke}{rgb}{0.690196,0.690196,0.690196}%
\pgfsetstrokecolor{currentstroke}%
\pgfsetdash{}{0pt}%
\pgfpathmoveto{\pgfqpoint{0.800000in}{3.777430in}}%
\pgfpathlineto{\pgfqpoint{5.760000in}{3.777430in}}%
\pgfusepath{stroke}%
\end{pgfscope}%
\begin{pgfscope}%
\pgfsetbuttcap%
\pgfsetroundjoin%
\definecolor{currentfill}{rgb}{0.000000,0.000000,0.000000}%
\pgfsetfillcolor{currentfill}%
\pgfsetlinewidth{0.803000pt}%
\definecolor{currentstroke}{rgb}{0.000000,0.000000,0.000000}%
\pgfsetstrokecolor{currentstroke}%
\pgfsetdash{}{0pt}%
\pgfsys@defobject{currentmarker}{\pgfqpoint{-0.048611in}{0.000000in}}{\pgfqpoint{-0.000000in}{0.000000in}}{%
\pgfpathmoveto{\pgfqpoint{-0.000000in}{0.000000in}}%
\pgfpathlineto{\pgfqpoint{-0.048611in}{0.000000in}}%
\pgfusepath{stroke,fill}%
}%
\begin{pgfscope}%
\pgfsys@transformshift{0.800000in}{3.777430in}%
\pgfsys@useobject{currentmarker}{}%
\end{pgfscope}%
\end{pgfscope}%
\begin{pgfscope}%
\definecolor{textcolor}{rgb}{0.000000,0.000000,0.000000}%
\pgfsetstrokecolor{textcolor}%
\pgfsetfillcolor{textcolor}%
\pgftext[x=0.614412in, y=3.724668in, left, base]{\color{textcolor}\sffamily\fontsize{10.000000}{12.000000}\selectfont 5}%
\end{pgfscope}%
\begin{pgfscope}%
\definecolor{textcolor}{rgb}{0.000000,0.000000,0.000000}%
\pgfsetstrokecolor{textcolor}%
\pgfsetfillcolor{textcolor}%
\pgftext[x=0.450832in,y=2.376000in,,bottom,rotate=90.000000]{\color{textcolor}\sffamily\fontsize{10.000000}{12.000000}\selectfont Velocity [rad/s]}%
\end{pgfscope}%
\begin{pgfscope}%
\pgfpathrectangle{\pgfqpoint{0.800000in}{0.528000in}}{\pgfqpoint{4.960000in}{3.696000in}}%
\pgfusepath{clip}%
\pgfsetrectcap%
\pgfsetroundjoin%
\pgfsetlinewidth{1.505625pt}%
\definecolor{currentstroke}{rgb}{0.121569,0.466667,0.705882}%
\pgfsetstrokecolor{currentstroke}%
\pgfsetdash{}{0pt}%
\pgfpathmoveto{\pgfqpoint{1.025455in}{1.771236in}}%
\pgfpathlineto{\pgfqpoint{1.065167in}{2.857400in}}%
\pgfpathlineto{\pgfqpoint{1.105817in}{3.516538in}}%
\pgfpathlineto{\pgfqpoint{1.146107in}{3.858557in}}%
\pgfpathlineto{\pgfqpoint{1.186615in}{4.000091in}}%
\pgfpathlineto{\pgfqpoint{1.227153in}{4.056000in}}%
\pgfpathlineto{\pgfqpoint{1.267835in}{4.020795in}}%
\pgfpathlineto{\pgfqpoint{1.308567in}{3.952892in}}%
\pgfpathlineto{\pgfqpoint{1.349067in}{3.633564in}}%
\pgfpathlineto{\pgfqpoint{1.389451in}{3.055827in}}%
\pgfpathlineto{\pgfqpoint{1.430695in}{2.288293in}}%
\pgfpathlineto{\pgfqpoint{1.471889in}{1.576544in}}%
\pgfpathlineto{\pgfqpoint{1.511611in}{1.060195in}}%
\pgfpathlineto{\pgfqpoint{1.552108in}{0.878967in}}%
\pgfpathlineto{\pgfqpoint{1.593206in}{0.944192in}}%
\pgfpathlineto{\pgfqpoint{1.633759in}{1.112422in}}%
\pgfpathlineto{\pgfqpoint{1.674031in}{1.349024in}}%
\pgfpathlineto{\pgfqpoint{1.714960in}{1.577721in}}%
\pgfpathlineto{\pgfqpoint{1.756053in}{1.593659in}}%
\pgfpathlineto{\pgfqpoint{1.796587in}{1.717099in}}%
\pgfpathlineto{\pgfqpoint{1.837135in}{1.779797in}}%
\pgfpathlineto{\pgfqpoint{1.877436in}{1.833323in}}%
\pgfpathlineto{\pgfqpoint{1.918964in}{1.816451in}}%
\pgfpathlineto{\pgfqpoint{1.958990in}{1.830539in}}%
\pgfpathlineto{\pgfqpoint{1.999575in}{1.781138in}}%
\pgfpathlineto{\pgfqpoint{2.039803in}{1.675309in}}%
\pgfpathlineto{\pgfqpoint{2.080304in}{1.621340in}}%
\pgfpathlineto{\pgfqpoint{2.121033in}{1.529519in}}%
\pgfpathlineto{\pgfqpoint{2.161440in}{1.539260in}}%
\pgfpathlineto{\pgfqpoint{2.202137in}{1.457805in}}%
\pgfpathlineto{\pgfqpoint{2.243156in}{1.440185in}}%
\pgfpathlineto{\pgfqpoint{2.283453in}{1.499080in}}%
\pgfpathlineto{\pgfqpoint{2.323912in}{1.499493in}}%
\pgfpathlineto{\pgfqpoint{2.364623in}{1.491002in}}%
\pgfpathlineto{\pgfqpoint{2.406336in}{1.484282in}}%
\pgfpathlineto{\pgfqpoint{2.446308in}{1.471112in}}%
\pgfpathlineto{\pgfqpoint{2.486292in}{1.431247in}}%
\pgfpathlineto{\pgfqpoint{2.526670in}{1.455108in}}%
\pgfpathlineto{\pgfqpoint{2.567213in}{1.442228in}}%
\pgfpathlineto{\pgfqpoint{2.607812in}{1.489585in}}%
\pgfpathlineto{\pgfqpoint{2.648513in}{1.466403in}}%
\pgfpathlineto{\pgfqpoint{2.689074in}{1.458820in}}%
\pgfpathlineto{\pgfqpoint{2.729883in}{1.458502in}}%
\pgfpathlineto{\pgfqpoint{2.769927in}{1.461167in}}%
\pgfpathlineto{\pgfqpoint{2.810427in}{1.462372in}}%
\pgfpathlineto{\pgfqpoint{2.850784in}{1.479548in}}%
\pgfpathlineto{\pgfqpoint{2.891979in}{1.495471in}}%
\pgfpathlineto{\pgfqpoint{2.932080in}{1.465669in}}%
\pgfpathlineto{\pgfqpoint{2.972448in}{1.446358in}}%
\pgfpathlineto{\pgfqpoint{3.012905in}{1.465063in}}%
\pgfpathlineto{\pgfqpoint{3.053907in}{1.382908in}}%
\pgfpathlineto{\pgfqpoint{3.094997in}{1.444553in}}%
\pgfpathlineto{\pgfqpoint{3.137441in}{1.429336in}}%
\pgfpathlineto{\pgfqpoint{3.176427in}{1.430220in}}%
\pgfpathlineto{\pgfqpoint{3.216884in}{1.384878in}}%
\pgfpathlineto{\pgfqpoint{3.257335in}{1.365803in}}%
\pgfpathlineto{\pgfqpoint{3.297832in}{1.394146in}}%
\pgfpathlineto{\pgfqpoint{3.339567in}{1.388425in}}%
\pgfpathlineto{\pgfqpoint{3.379160in}{1.413080in}}%
\pgfpathlineto{\pgfqpoint{3.419553in}{1.476237in}}%
\pgfpathlineto{\pgfqpoint{3.459911in}{1.442963in}}%
\pgfpathlineto{\pgfqpoint{3.500453in}{1.529218in}}%
\pgfpathlineto{\pgfqpoint{3.540877in}{1.508279in}}%
\pgfpathlineto{\pgfqpoint{3.581344in}{1.544965in}}%
\pgfpathlineto{\pgfqpoint{3.622530in}{1.577131in}}%
\pgfpathlineto{\pgfqpoint{3.663092in}{1.555587in}}%
\pgfpathlineto{\pgfqpoint{3.703574in}{1.566935in}}%
\pgfpathlineto{\pgfqpoint{3.743918in}{1.570134in}}%
\pgfpathlineto{\pgfqpoint{3.784683in}{1.510617in}}%
\pgfpathlineto{\pgfqpoint{3.826265in}{1.475746in}}%
\pgfpathlineto{\pgfqpoint{3.866122in}{1.413425in}}%
\pgfpathlineto{\pgfqpoint{3.906444in}{1.406422in}}%
\pgfpathlineto{\pgfqpoint{3.946680in}{1.302478in}}%
\pgfpathlineto{\pgfqpoint{3.987188in}{1.407125in}}%
\pgfpathlineto{\pgfqpoint{4.027560in}{1.440027in}}%
\pgfpathlineto{\pgfqpoint{4.068377in}{1.532526in}}%
\pgfpathlineto{\pgfqpoint{4.108705in}{1.594615in}}%
\pgfpathlineto{\pgfqpoint{4.149612in}{1.660858in}}%
\pgfpathlineto{\pgfqpoint{4.189834in}{1.674063in}}%
\pgfpathlineto{\pgfqpoint{4.230320in}{1.602483in}}%
\pgfpathlineto{\pgfqpoint{4.270705in}{1.536084in}}%
\pgfpathlineto{\pgfqpoint{4.311851in}{1.534757in}}%
\pgfpathlineto{\pgfqpoint{4.352018in}{1.497567in}}%
\pgfpathlineto{\pgfqpoint{4.392331in}{1.524565in}}%
\pgfpathlineto{\pgfqpoint{4.432762in}{1.638231in}}%
\pgfpathlineto{\pgfqpoint{4.473590in}{1.643075in}}%
\pgfpathlineto{\pgfqpoint{4.514073in}{1.598628in}}%
\pgfpathlineto{\pgfqpoint{4.555047in}{1.625420in}}%
\pgfpathlineto{\pgfqpoint{4.595243in}{1.642520in}}%
\pgfpathlineto{\pgfqpoint{4.635559in}{1.613249in}}%
\pgfpathlineto{\pgfqpoint{4.676181in}{1.625673in}}%
\pgfpathlineto{\pgfqpoint{4.717589in}{1.561069in}}%
\pgfpathlineto{\pgfqpoint{4.759328in}{1.485772in}}%
\pgfpathlineto{\pgfqpoint{4.801586in}{1.464585in}}%
\pgfpathlineto{\pgfqpoint{4.841154in}{1.394501in}}%
\pgfpathlineto{\pgfqpoint{4.881393in}{1.246202in}}%
\pgfpathlineto{\pgfqpoint{4.921325in}{1.205269in}}%
\pgfpathlineto{\pgfqpoint{4.961813in}{1.182720in}}%
\pgfpathlineto{\pgfqpoint{5.002305in}{1.243584in}}%
\pgfpathlineto{\pgfqpoint{5.042789in}{1.290978in}}%
\pgfpathlineto{\pgfqpoint{5.084166in}{1.499941in}}%
\pgfpathlineto{\pgfqpoint{5.124005in}{1.611562in}}%
\pgfpathlineto{\pgfqpoint{5.164280in}{1.675024in}}%
\pgfpathlineto{\pgfqpoint{5.204977in}{1.613870in}}%
\pgfpathlineto{\pgfqpoint{5.245600in}{1.580104in}}%
\pgfpathlineto{\pgfqpoint{5.286480in}{1.575590in}}%
\pgfpathlineto{\pgfqpoint{5.326842in}{1.589130in}}%
\pgfpathlineto{\pgfqpoint{5.368258in}{1.560019in}}%
\pgfpathlineto{\pgfqpoint{5.408402in}{1.509392in}}%
\pgfpathlineto{\pgfqpoint{5.448455in}{1.374927in}}%
\pgfpathlineto{\pgfqpoint{5.488920in}{1.300082in}}%
\pgfpathlineto{\pgfqpoint{5.529783in}{1.314155in}}%
\pgfusepath{stroke}%
\end{pgfscope}%
\begin{pgfscope}%
\pgfpathrectangle{\pgfqpoint{0.800000in}{0.528000in}}{\pgfqpoint{4.960000in}{3.696000in}}%
\pgfusepath{clip}%
\pgfsetrectcap%
\pgfsetroundjoin%
\pgfsetlinewidth{1.505625pt}%
\definecolor{currentstroke}{rgb}{1.000000,0.498039,0.054902}%
\pgfsetstrokecolor{currentstroke}%
\pgfsetdash{}{0pt}%
\pgfpathmoveto{\pgfqpoint{1.025455in}{1.755623in}}%
\pgfpathlineto{\pgfqpoint{1.066066in}{2.915231in}}%
\pgfpathlineto{\pgfqpoint{1.106747in}{3.553688in}}%
\pgfpathlineto{\pgfqpoint{1.146999in}{3.886188in}}%
\pgfpathlineto{\pgfqpoint{1.187691in}{3.976890in}}%
\pgfpathlineto{\pgfqpoint{1.228114in}{4.020109in}}%
\pgfpathlineto{\pgfqpoint{1.270682in}{4.005177in}}%
\pgfpathlineto{\pgfqpoint{1.310601in}{3.948117in}}%
\pgfpathlineto{\pgfqpoint{1.350845in}{3.345741in}}%
\pgfpathlineto{\pgfqpoint{1.391418in}{2.816903in}}%
\pgfpathlineto{\pgfqpoint{1.432324in}{2.006156in}}%
\pgfpathlineto{\pgfqpoint{1.473020in}{1.526295in}}%
\pgfpathlineto{\pgfqpoint{1.513247in}{1.332105in}}%
\pgfpathlineto{\pgfqpoint{1.553736in}{1.278657in}}%
\pgfpathlineto{\pgfqpoint{1.594282in}{1.223724in}}%
\pgfpathlineto{\pgfqpoint{1.635543in}{1.192099in}}%
\pgfpathlineto{\pgfqpoint{1.675938in}{1.253130in}}%
\pgfpathlineto{\pgfqpoint{1.717007in}{1.300403in}}%
\pgfpathlineto{\pgfqpoint{1.758424in}{1.528941in}}%
\pgfpathlineto{\pgfqpoint{1.798591in}{1.554270in}}%
\pgfpathlineto{\pgfqpoint{1.838942in}{1.553105in}}%
\pgfpathlineto{\pgfqpoint{1.879383in}{1.617573in}}%
\pgfpathlineto{\pgfqpoint{1.919862in}{1.592456in}}%
\pgfpathlineto{\pgfqpoint{1.960608in}{1.556906in}}%
\pgfpathlineto{\pgfqpoint{2.000699in}{1.621688in}}%
\pgfpathlineto{\pgfqpoint{2.041595in}{1.641445in}}%
\pgfpathlineto{\pgfqpoint{2.081816in}{1.638775in}}%
\pgfpathlineto{\pgfqpoint{2.123079in}{1.631769in}}%
\pgfpathlineto{\pgfqpoint{2.164534in}{1.607928in}}%
\pgfpathlineto{\pgfqpoint{2.204565in}{1.388434in}}%
\pgfpathlineto{\pgfqpoint{2.244839in}{1.360205in}}%
\pgfpathlineto{\pgfqpoint{2.285499in}{1.331888in}}%
\pgfpathlineto{\pgfqpoint{2.325985in}{1.283626in}}%
\pgfpathlineto{\pgfqpoint{2.366506in}{1.328648in}}%
\pgfpathlineto{\pgfqpoint{2.406629in}{1.332222in}}%
\pgfpathlineto{\pgfqpoint{2.448242in}{1.354338in}}%
\pgfpathlineto{\pgfqpoint{2.488456in}{1.367485in}}%
\pgfpathlineto{\pgfqpoint{2.529053in}{1.425120in}}%
\pgfpathlineto{\pgfqpoint{2.569739in}{1.521999in}}%
\pgfpathlineto{\pgfqpoint{2.610216in}{1.653792in}}%
\pgfpathlineto{\pgfqpoint{2.651644in}{1.699393in}}%
\pgfpathlineto{\pgfqpoint{2.691736in}{1.675005in}}%
\pgfpathlineto{\pgfqpoint{2.731990in}{1.708627in}}%
\pgfpathlineto{\pgfqpoint{2.772678in}{1.453462in}}%
\pgfpathlineto{\pgfqpoint{2.813308in}{1.335672in}}%
\pgfpathlineto{\pgfqpoint{2.853768in}{1.377195in}}%
\pgfpathlineto{\pgfqpoint{2.894341in}{1.386857in}}%
\pgfpathlineto{\pgfqpoint{2.935058in}{1.480524in}}%
\pgfpathlineto{\pgfqpoint{2.975545in}{1.578734in}}%
\pgfpathlineto{\pgfqpoint{3.016117in}{1.617621in}}%
\pgfpathlineto{\pgfqpoint{3.056445in}{1.645353in}}%
\pgfpathlineto{\pgfqpoint{3.097115in}{1.581634in}}%
\pgfpathlineto{\pgfqpoint{3.138435in}{1.474018in}}%
\pgfpathlineto{\pgfqpoint{3.178401in}{1.438473in}}%
\pgfpathlineto{\pgfqpoint{3.218985in}{1.547181in}}%
\pgfpathlineto{\pgfqpoint{3.259516in}{1.635376in}}%
\pgfpathlineto{\pgfqpoint{3.300172in}{1.660289in}}%
\pgfpathlineto{\pgfqpoint{3.340882in}{1.618412in}}%
\pgfpathlineto{\pgfqpoint{3.381386in}{1.607149in}}%
\pgfpathlineto{\pgfqpoint{3.421776in}{1.576751in}}%
\pgfpathlineto{\pgfqpoint{3.462286in}{1.612054in}}%
\pgfpathlineto{\pgfqpoint{3.502956in}{1.582779in}}%
\pgfpathlineto{\pgfqpoint{3.543672in}{1.590050in}}%
\pgfpathlineto{\pgfqpoint{3.584058in}{1.602268in}}%
\pgfpathlineto{\pgfqpoint{3.625477in}{1.592429in}}%
\pgfpathlineto{\pgfqpoint{3.665651in}{1.578023in}}%
\pgfpathlineto{\pgfqpoint{3.706007in}{1.517281in}}%
\pgfpathlineto{\pgfqpoint{3.746739in}{1.496378in}}%
\pgfpathlineto{\pgfqpoint{3.787334in}{1.496924in}}%
\pgfpathlineto{\pgfqpoint{3.827958in}{1.558891in}}%
\pgfpathlineto{\pgfqpoint{3.868481in}{1.605592in}}%
\pgfpathlineto{\pgfqpoint{3.908760in}{1.542730in}}%
\pgfpathlineto{\pgfqpoint{3.949501in}{1.574432in}}%
\pgfpathlineto{\pgfqpoint{3.990445in}{1.559302in}}%
\pgfpathlineto{\pgfqpoint{4.030856in}{1.596351in}}%
\pgfpathlineto{\pgfqpoint{4.072060in}{1.574857in}}%
\pgfpathlineto{\pgfqpoint{4.113385in}{1.585283in}}%
\pgfpathlineto{\pgfqpoint{4.153722in}{1.539465in}}%
\pgfpathlineto{\pgfqpoint{4.193964in}{1.572538in}}%
\pgfpathlineto{\pgfqpoint{4.234361in}{1.500205in}}%
\pgfpathlineto{\pgfqpoint{4.275160in}{1.549629in}}%
\pgfpathlineto{\pgfqpoint{4.315453in}{1.564125in}}%
\pgfpathlineto{\pgfqpoint{4.355649in}{1.549659in}}%
\pgfpathlineto{\pgfqpoint{4.397572in}{1.603549in}}%
\pgfpathlineto{\pgfqpoint{4.437145in}{1.581093in}}%
\pgfpathlineto{\pgfqpoint{4.477907in}{1.565978in}}%
\pgfpathlineto{\pgfqpoint{4.518440in}{1.590864in}}%
\pgfpathlineto{\pgfqpoint{4.559241in}{1.579574in}}%
\pgfpathlineto{\pgfqpoint{4.601326in}{1.496528in}}%
\pgfpathlineto{\pgfqpoint{4.640443in}{1.387476in}}%
\pgfpathlineto{\pgfqpoint{4.680845in}{1.418049in}}%
\pgfpathlineto{\pgfqpoint{4.721296in}{1.395331in}}%
\pgfpathlineto{\pgfqpoint{4.761936in}{1.370344in}}%
\pgfpathlineto{\pgfqpoint{4.802281in}{1.420693in}}%
\pgfpathlineto{\pgfqpoint{4.842749in}{1.415454in}}%
\pgfpathlineto{\pgfqpoint{4.883113in}{1.435138in}}%
\pgfpathlineto{\pgfqpoint{4.923754in}{1.477461in}}%
\pgfpathlineto{\pgfqpoint{4.964381in}{1.461501in}}%
\pgfpathlineto{\pgfqpoint{5.005551in}{1.463864in}}%
\pgfpathlineto{\pgfqpoint{5.045605in}{1.486416in}}%
\pgfpathlineto{\pgfqpoint{5.087838in}{1.463134in}}%
\pgfpathlineto{\pgfqpoint{5.128061in}{1.456731in}}%
\pgfpathlineto{\pgfqpoint{5.167879in}{1.488476in}}%
\pgfpathlineto{\pgfqpoint{5.208409in}{1.458418in}}%
\pgfpathlineto{\pgfqpoint{5.248962in}{1.492634in}}%
\pgfpathlineto{\pgfqpoint{5.289531in}{1.525890in}}%
\pgfpathlineto{\pgfqpoint{5.329846in}{1.544059in}}%
\pgfpathlineto{\pgfqpoint{5.372381in}{1.524028in}}%
\pgfpathlineto{\pgfqpoint{5.411517in}{1.514997in}}%
\pgfpathlineto{\pgfqpoint{5.451793in}{1.502408in}}%
\pgfpathlineto{\pgfqpoint{5.494245in}{1.482064in}}%
\pgfpathlineto{\pgfqpoint{5.534545in}{1.551188in}}%
\pgfusepath{stroke}%
\end{pgfscope}%
\begin{pgfscope}%
\pgfpathrectangle{\pgfqpoint{0.800000in}{0.528000in}}{\pgfqpoint{4.960000in}{3.696000in}}%
\pgfusepath{clip}%
\pgfsetrectcap%
\pgfsetroundjoin%
\pgfsetlinewidth{1.505625pt}%
\definecolor{currentstroke}{rgb}{0.172549,0.627451,0.172549}%
\pgfsetstrokecolor{currentstroke}%
\pgfsetdash{}{0pt}%
\pgfpathmoveto{\pgfqpoint{1.025455in}{1.680639in}}%
\pgfpathlineto{\pgfqpoint{1.065942in}{2.890803in}}%
\pgfpathlineto{\pgfqpoint{1.106347in}{3.530001in}}%
\pgfpathlineto{\pgfqpoint{1.146953in}{3.838639in}}%
\pgfpathlineto{\pgfqpoint{1.187698in}{4.007099in}}%
\pgfpathlineto{\pgfqpoint{1.228069in}{4.024743in}}%
\pgfpathlineto{\pgfqpoint{1.269084in}{4.034469in}}%
\pgfpathlineto{\pgfqpoint{1.308995in}{3.980449in}}%
\pgfpathlineto{\pgfqpoint{1.349690in}{3.976291in}}%
\pgfpathlineto{\pgfqpoint{1.390093in}{3.827663in}}%
\pgfpathlineto{\pgfqpoint{1.432309in}{3.216843in}}%
\pgfpathlineto{\pgfqpoint{1.472123in}{2.519065in}}%
\pgfpathlineto{\pgfqpoint{1.512374in}{1.811501in}}%
\pgfpathlineto{\pgfqpoint{1.552764in}{1.375001in}}%
\pgfpathlineto{\pgfqpoint{1.593156in}{1.204122in}}%
\pgfpathlineto{\pgfqpoint{1.633578in}{1.123067in}}%
\pgfpathlineto{\pgfqpoint{1.674228in}{0.870954in}}%
\pgfpathlineto{\pgfqpoint{1.714522in}{0.864327in}}%
\pgfpathlineto{\pgfqpoint{1.755143in}{1.146980in}}%
\pgfpathlineto{\pgfqpoint{1.795636in}{1.336114in}}%
\pgfpathlineto{\pgfqpoint{1.836468in}{1.496573in}}%
\pgfpathlineto{\pgfqpoint{1.877162in}{1.538865in}}%
\pgfpathlineto{\pgfqpoint{1.918804in}{1.657532in}}%
\pgfpathlineto{\pgfqpoint{1.958872in}{1.574202in}}%
\pgfpathlineto{\pgfqpoint{1.999128in}{1.445025in}}%
\pgfpathlineto{\pgfqpoint{2.039133in}{1.344858in}}%
\pgfpathlineto{\pgfqpoint{2.079792in}{1.275979in}}%
\pgfpathlineto{\pgfqpoint{2.120186in}{1.352083in}}%
\pgfpathlineto{\pgfqpoint{2.161007in}{1.290169in}}%
\pgfpathlineto{\pgfqpoint{2.201257in}{1.305892in}}%
\pgfpathlineto{\pgfqpoint{2.241732in}{1.350664in}}%
\pgfpathlineto{\pgfqpoint{2.282417in}{1.596059in}}%
\pgfpathlineto{\pgfqpoint{2.323025in}{1.656744in}}%
\pgfpathlineto{\pgfqpoint{2.364298in}{1.602545in}}%
\pgfpathlineto{\pgfqpoint{2.405946in}{1.678300in}}%
\pgfpathlineto{\pgfqpoint{2.445811in}{1.655883in}}%
\pgfpathlineto{\pgfqpoint{2.486378in}{1.701753in}}%
\pgfpathlineto{\pgfqpoint{2.526667in}{1.668911in}}%
\pgfpathlineto{\pgfqpoint{2.567280in}{1.659922in}}%
\pgfpathlineto{\pgfqpoint{2.607813in}{1.560630in}}%
\pgfpathlineto{\pgfqpoint{2.648615in}{1.443721in}}%
\pgfpathlineto{\pgfqpoint{2.689401in}{1.423099in}}%
\pgfpathlineto{\pgfqpoint{2.729940in}{1.414459in}}%
\pgfpathlineto{\pgfqpoint{2.770428in}{1.411701in}}%
\pgfpathlineto{\pgfqpoint{2.811347in}{1.416018in}}%
\pgfpathlineto{\pgfqpoint{2.853021in}{1.403466in}}%
\pgfpathlineto{\pgfqpoint{2.892717in}{1.512555in}}%
\pgfpathlineto{\pgfqpoint{2.932837in}{1.543903in}}%
\pgfpathlineto{\pgfqpoint{2.974066in}{1.565941in}}%
\pgfpathlineto{\pgfqpoint{3.013823in}{1.550828in}}%
\pgfpathlineto{\pgfqpoint{3.054985in}{1.630715in}}%
\pgfpathlineto{\pgfqpoint{3.095377in}{1.628679in}}%
\pgfpathlineto{\pgfqpoint{3.136107in}{1.592283in}}%
\pgfpathlineto{\pgfqpoint{3.176243in}{1.578422in}}%
\pgfpathlineto{\pgfqpoint{3.216921in}{1.543599in}}%
\pgfpathlineto{\pgfqpoint{3.257428in}{1.495499in}}%
\pgfpathlineto{\pgfqpoint{3.298082in}{1.444360in}}%
\pgfpathlineto{\pgfqpoint{3.339345in}{1.448434in}}%
\pgfpathlineto{\pgfqpoint{3.379704in}{1.409876in}}%
\pgfpathlineto{\pgfqpoint{3.419816in}{1.444050in}}%
\pgfpathlineto{\pgfqpoint{3.460360in}{1.418058in}}%
\pgfpathlineto{\pgfqpoint{3.501387in}{1.402484in}}%
\pgfpathlineto{\pgfqpoint{3.541843in}{1.452398in}}%
\pgfpathlineto{\pgfqpoint{3.582412in}{1.466301in}}%
\pgfpathlineto{\pgfqpoint{3.623130in}{1.330102in}}%
\pgfpathlineto{\pgfqpoint{3.663328in}{1.391920in}}%
\pgfpathlineto{\pgfqpoint{3.703796in}{1.492919in}}%
\pgfpathlineto{\pgfqpoint{3.744443in}{1.592104in}}%
\pgfpathlineto{\pgfqpoint{3.786108in}{1.630089in}}%
\pgfpathlineto{\pgfqpoint{3.825918in}{1.583792in}}%
\pgfpathlineto{\pgfqpoint{3.866219in}{1.495230in}}%
\pgfpathlineto{\pgfqpoint{3.906276in}{1.423558in}}%
\pgfpathlineto{\pgfqpoint{3.946968in}{1.384442in}}%
\pgfpathlineto{\pgfqpoint{3.987502in}{1.204528in}}%
\pgfpathlineto{\pgfqpoint{4.028085in}{1.275977in}}%
\pgfpathlineto{\pgfqpoint{4.068546in}{1.329125in}}%
\pgfpathlineto{\pgfqpoint{4.108820in}{1.337282in}}%
\pgfpathlineto{\pgfqpoint{4.149664in}{1.386987in}}%
\pgfpathlineto{\pgfqpoint{4.190171in}{1.535409in}}%
\pgfpathlineto{\pgfqpoint{4.230645in}{1.617463in}}%
\pgfpathlineto{\pgfqpoint{4.272551in}{1.447788in}}%
\pgfpathlineto{\pgfqpoint{4.312595in}{1.516931in}}%
\pgfpathlineto{\pgfqpoint{4.352719in}{1.555445in}}%
\pgfpathlineto{\pgfqpoint{4.393074in}{1.624747in}}%
\pgfpathlineto{\pgfqpoint{4.433722in}{1.444880in}}%
\pgfpathlineto{\pgfqpoint{4.474177in}{1.491991in}}%
\pgfpathlineto{\pgfqpoint{4.515023in}{1.573542in}}%
\pgfpathlineto{\pgfqpoint{4.555782in}{1.649096in}}%
\pgfpathlineto{\pgfqpoint{4.595853in}{1.679000in}}%
\pgfpathlineto{\pgfqpoint{4.636412in}{1.740214in}}%
\pgfpathlineto{\pgfqpoint{4.676884in}{1.756754in}}%
\pgfpathlineto{\pgfqpoint{4.718576in}{1.528827in}}%
\pgfpathlineto{\pgfqpoint{4.758691in}{1.525044in}}%
\pgfpathlineto{\pgfqpoint{4.799091in}{1.528545in}}%
\pgfpathlineto{\pgfqpoint{4.839664in}{1.508483in}}%
\pgfpathlineto{\pgfqpoint{4.880330in}{1.441235in}}%
\pgfpathlineto{\pgfqpoint{4.921320in}{1.451023in}}%
\pgfpathlineto{\pgfqpoint{4.961564in}{1.452670in}}%
\pgfpathlineto{\pgfqpoint{5.002273in}{1.487521in}}%
\pgfpathlineto{\pgfqpoint{5.043004in}{1.490679in}}%
\pgfpathlineto{\pgfqpoint{5.083380in}{1.502920in}}%
\pgfpathlineto{\pgfqpoint{5.123853in}{1.462268in}}%
\pgfpathlineto{\pgfqpoint{5.165503in}{1.437582in}}%
\pgfpathlineto{\pgfqpoint{5.205502in}{1.458772in}}%
\pgfpathlineto{\pgfqpoint{5.245786in}{1.402716in}}%
\pgfpathlineto{\pgfqpoint{5.286022in}{1.443416in}}%
\pgfpathlineto{\pgfqpoint{5.326616in}{1.532899in}}%
\pgfpathlineto{\pgfqpoint{5.367125in}{1.550296in}}%
\pgfpathlineto{\pgfqpoint{5.407403in}{1.584977in}}%
\pgfpathlineto{\pgfqpoint{5.448340in}{1.634697in}}%
\pgfpathlineto{\pgfqpoint{5.488720in}{1.626001in}}%
\pgfpathlineto{\pgfqpoint{5.529151in}{1.648571in}}%
\pgfusepath{stroke}%
\end{pgfscope}%
\begin{pgfscope}%
\pgfpathrectangle{\pgfqpoint{0.800000in}{0.528000in}}{\pgfqpoint{4.960000in}{3.696000in}}%
\pgfusepath{clip}%
\pgfsetrectcap%
\pgfsetroundjoin%
\pgfsetlinewidth{1.505625pt}%
\definecolor{currentstroke}{rgb}{0.839216,0.152941,0.156863}%
\pgfsetstrokecolor{currentstroke}%
\pgfsetdash{}{0pt}%
\pgfpathmoveto{\pgfqpoint{1.025455in}{1.673275in}}%
\pgfpathlineto{\pgfqpoint{1.065837in}{2.957377in}}%
\pgfpathlineto{\pgfqpoint{1.107540in}{3.547333in}}%
\pgfpathlineto{\pgfqpoint{1.148030in}{3.854073in}}%
\pgfpathlineto{\pgfqpoint{1.188150in}{3.981880in}}%
\pgfpathlineto{\pgfqpoint{1.228519in}{4.052705in}}%
\pgfpathlineto{\pgfqpoint{1.268785in}{4.036743in}}%
\pgfpathlineto{\pgfqpoint{1.309371in}{3.993351in}}%
\pgfpathlineto{\pgfqpoint{1.350880in}{3.999617in}}%
\pgfpathlineto{\pgfqpoint{1.391303in}{3.932998in}}%
\pgfpathlineto{\pgfqpoint{1.431578in}{3.427044in}}%
\pgfpathlineto{\pgfqpoint{1.472178in}{2.950639in}}%
\pgfpathlineto{\pgfqpoint{1.512640in}{2.267300in}}%
\pgfpathlineto{\pgfqpoint{1.554227in}{1.674765in}}%
\pgfpathlineto{\pgfqpoint{1.594380in}{1.275302in}}%
\pgfpathlineto{\pgfqpoint{1.634354in}{1.172353in}}%
\pgfpathlineto{\pgfqpoint{1.674931in}{0.985334in}}%
\pgfpathlineto{\pgfqpoint{1.715589in}{0.914121in}}%
\pgfpathlineto{\pgfqpoint{1.756197in}{1.079666in}}%
\pgfpathlineto{\pgfqpoint{1.797029in}{1.030462in}}%
\pgfpathlineto{\pgfqpoint{1.839153in}{1.019155in}}%
\pgfpathlineto{\pgfqpoint{1.877870in}{1.163080in}}%
\pgfpathlineto{\pgfqpoint{1.918180in}{1.202380in}}%
\pgfpathlineto{\pgfqpoint{1.960155in}{1.173432in}}%
\pgfpathlineto{\pgfqpoint{2.000615in}{1.141399in}}%
\pgfpathlineto{\pgfqpoint{2.040561in}{1.246139in}}%
\pgfpathlineto{\pgfqpoint{2.080958in}{1.311878in}}%
\pgfpathlineto{\pgfqpoint{2.121190in}{1.440554in}}%
\pgfpathlineto{\pgfqpoint{2.161635in}{1.541253in}}%
\pgfpathlineto{\pgfqpoint{2.203254in}{1.567776in}}%
\pgfpathlineto{\pgfqpoint{2.243918in}{1.510654in}}%
\pgfpathlineto{\pgfqpoint{2.284182in}{1.379990in}}%
\pgfpathlineto{\pgfqpoint{2.324663in}{1.286856in}}%
\pgfpathlineto{\pgfqpoint{2.365329in}{1.451181in}}%
\pgfpathlineto{\pgfqpoint{2.405862in}{1.497504in}}%
\pgfpathlineto{\pgfqpoint{2.446514in}{1.542759in}}%
\pgfpathlineto{\pgfqpoint{2.487114in}{1.510802in}}%
\pgfpathlineto{\pgfqpoint{2.527720in}{1.468914in}}%
\pgfpathlineto{\pgfqpoint{2.567846in}{1.435749in}}%
\pgfpathlineto{\pgfqpoint{2.608409in}{1.520637in}}%
\pgfpathlineto{\pgfqpoint{2.649068in}{1.449271in}}%
\pgfpathlineto{\pgfqpoint{2.689568in}{1.474629in}}%
\pgfpathlineto{\pgfqpoint{2.731723in}{1.528400in}}%
\pgfpathlineto{\pgfqpoint{2.771742in}{1.558767in}}%
\pgfpathlineto{\pgfqpoint{2.812196in}{1.586862in}}%
\pgfpathlineto{\pgfqpoint{2.852865in}{1.528592in}}%
\pgfpathlineto{\pgfqpoint{2.893419in}{1.549660in}}%
\pgfpathlineto{\pgfqpoint{2.933960in}{1.557237in}}%
\pgfpathlineto{\pgfqpoint{2.974495in}{1.521043in}}%
\pgfpathlineto{\pgfqpoint{3.015117in}{1.479481in}}%
\pgfpathlineto{\pgfqpoint{3.055675in}{1.511881in}}%
\pgfpathlineto{\pgfqpoint{3.096045in}{1.329612in}}%
\pgfpathlineto{\pgfqpoint{3.136979in}{1.329440in}}%
\pgfpathlineto{\pgfqpoint{3.178855in}{1.341773in}}%
\pgfpathlineto{\pgfqpoint{3.219084in}{1.397493in}}%
\pgfpathlineto{\pgfqpoint{3.259130in}{1.358815in}}%
\pgfpathlineto{\pgfqpoint{3.299494in}{1.345870in}}%
\pgfpathlineto{\pgfqpoint{3.339857in}{1.407975in}}%
\pgfpathlineto{\pgfqpoint{3.380426in}{1.471081in}}%
\pgfpathlineto{\pgfqpoint{3.421187in}{1.518374in}}%
\pgfpathlineto{\pgfqpoint{3.462127in}{1.516750in}}%
\pgfpathlineto{\pgfqpoint{3.502542in}{1.517694in}}%
\pgfpathlineto{\pgfqpoint{3.542845in}{1.544281in}}%
\pgfpathlineto{\pgfqpoint{3.583455in}{1.531075in}}%
\pgfpathlineto{\pgfqpoint{3.624002in}{1.485201in}}%
\pgfpathlineto{\pgfqpoint{3.666085in}{1.324644in}}%
\pgfpathlineto{\pgfqpoint{3.705893in}{1.308145in}}%
\pgfpathlineto{\pgfqpoint{3.746190in}{1.227131in}}%
\pgfpathlineto{\pgfqpoint{3.786719in}{1.243124in}}%
\pgfpathlineto{\pgfqpoint{3.827297in}{1.335955in}}%
\pgfpathlineto{\pgfqpoint{3.867731in}{1.452249in}}%
\pgfpathlineto{\pgfqpoint{3.909023in}{1.539616in}}%
\pgfpathlineto{\pgfqpoint{3.949367in}{1.652482in}}%
\pgfpathlineto{\pgfqpoint{3.989702in}{1.700123in}}%
\pgfpathlineto{\pgfqpoint{4.030393in}{1.723365in}}%
\pgfpathlineto{\pgfqpoint{4.070711in}{1.531413in}}%
\pgfpathlineto{\pgfqpoint{4.111271in}{1.480325in}}%
\pgfpathlineto{\pgfqpoint{4.152764in}{1.391257in}}%
\pgfpathlineto{\pgfqpoint{4.192655in}{1.326843in}}%
\pgfpathlineto{\pgfqpoint{4.232948in}{1.297392in}}%
\pgfpathlineto{\pgfqpoint{4.273340in}{1.330432in}}%
\pgfpathlineto{\pgfqpoint{4.314061in}{1.512306in}}%
\pgfpathlineto{\pgfqpoint{4.354560in}{1.579173in}}%
\pgfpathlineto{\pgfqpoint{4.395456in}{1.609931in}}%
\pgfpathlineto{\pgfqpoint{4.436186in}{1.484204in}}%
\pgfpathlineto{\pgfqpoint{4.476438in}{1.487355in}}%
\pgfpathlineto{\pgfqpoint{4.517043in}{1.560171in}}%
\pgfpathlineto{\pgfqpoint{4.557747in}{1.599584in}}%
\pgfpathlineto{\pgfqpoint{4.599471in}{1.461168in}}%
\pgfpathlineto{\pgfqpoint{4.639494in}{1.479127in}}%
\pgfpathlineto{\pgfqpoint{4.679910in}{1.508392in}}%
\pgfpathlineto{\pgfqpoint{4.720349in}{1.429889in}}%
\pgfpathlineto{\pgfqpoint{4.760964in}{1.462161in}}%
\pgfpathlineto{\pgfqpoint{4.801858in}{1.603258in}}%
\pgfpathlineto{\pgfqpoint{4.841929in}{1.618684in}}%
\pgfpathlineto{\pgfqpoint{4.882129in}{1.673145in}}%
\pgfpathlineto{\pgfqpoint{4.922759in}{1.634907in}}%
\pgfpathlineto{\pgfqpoint{4.963352in}{1.677774in}}%
\pgfpathlineto{\pgfqpoint{5.003815in}{1.586144in}}%
\pgfpathlineto{\pgfqpoint{5.044510in}{1.632158in}}%
\pgfpathlineto{\pgfqpoint{5.086900in}{1.800717in}}%
\pgfpathlineto{\pgfqpoint{5.126487in}{1.807170in}}%
\pgfpathlineto{\pgfqpoint{5.166603in}{1.759221in}}%
\pgfpathlineto{\pgfqpoint{5.206999in}{1.379433in}}%
\pgfpathlineto{\pgfqpoint{5.247454in}{1.243100in}}%
\pgfpathlineto{\pgfqpoint{5.287977in}{1.256026in}}%
\pgfpathlineto{\pgfqpoint{5.328610in}{1.365441in}}%
\pgfpathlineto{\pgfqpoint{5.369087in}{1.298171in}}%
\pgfpathlineto{\pgfqpoint{5.409971in}{1.512554in}}%
\pgfpathlineto{\pgfqpoint{5.450783in}{1.759091in}}%
\pgfpathlineto{\pgfqpoint{5.491247in}{1.723922in}}%
\pgfpathlineto{\pgfqpoint{5.532506in}{1.588668in}}%
\pgfusepath{stroke}%
\end{pgfscope}%
\begin{pgfscope}%
\pgfpathrectangle{\pgfqpoint{0.800000in}{0.528000in}}{\pgfqpoint{4.960000in}{3.696000in}}%
\pgfusepath{clip}%
\pgfsetrectcap%
\pgfsetroundjoin%
\pgfsetlinewidth{1.505625pt}%
\definecolor{currentstroke}{rgb}{0.580392,0.403922,0.741176}%
\pgfsetstrokecolor{currentstroke}%
\pgfsetdash{}{0pt}%
\pgfpathmoveto{\pgfqpoint{1.025455in}{1.784832in}}%
\pgfpathlineto{\pgfqpoint{1.067683in}{2.982776in}}%
\pgfpathlineto{\pgfqpoint{1.107290in}{3.550495in}}%
\pgfpathlineto{\pgfqpoint{1.150310in}{3.875620in}}%
\pgfpathlineto{\pgfqpoint{1.189323in}{4.023098in}}%
\pgfpathlineto{\pgfqpoint{1.229714in}{4.031853in}}%
\pgfpathlineto{\pgfqpoint{1.270362in}{4.025396in}}%
\pgfpathlineto{\pgfqpoint{1.310474in}{4.000989in}}%
\pgfpathlineto{\pgfqpoint{1.351342in}{3.980591in}}%
\pgfpathlineto{\pgfqpoint{1.391372in}{3.869837in}}%
\pgfpathlineto{\pgfqpoint{1.432503in}{3.479595in}}%
\pgfpathlineto{\pgfqpoint{1.473364in}{2.698699in}}%
\pgfpathlineto{\pgfqpoint{1.513937in}{1.903577in}}%
\pgfpathlineto{\pgfqpoint{1.555394in}{1.442406in}}%
\pgfpathlineto{\pgfqpoint{1.595397in}{1.252240in}}%
\pgfpathlineto{\pgfqpoint{1.635699in}{1.099295in}}%
\pgfpathlineto{\pgfqpoint{1.676331in}{0.803906in}}%
\pgfpathlineto{\pgfqpoint{1.716922in}{0.843364in}}%
\pgfpathlineto{\pgfqpoint{1.757656in}{0.887141in}}%
\pgfpathlineto{\pgfqpoint{1.798234in}{0.973509in}}%
\pgfpathlineto{\pgfqpoint{1.838795in}{1.106241in}}%
\pgfpathlineto{\pgfqpoint{1.879124in}{1.083970in}}%
\pgfpathlineto{\pgfqpoint{1.921577in}{0.984056in}}%
\pgfpathlineto{\pgfqpoint{1.960475in}{1.062444in}}%
\pgfpathlineto{\pgfqpoint{2.000778in}{1.462579in}}%
\pgfpathlineto{\pgfqpoint{2.043146in}{1.569252in}}%
\pgfpathlineto{\pgfqpoint{2.083412in}{1.495195in}}%
\pgfpathlineto{\pgfqpoint{2.123718in}{1.463453in}}%
\pgfpathlineto{\pgfqpoint{2.164381in}{1.421943in}}%
\pgfpathlineto{\pgfqpoint{2.205019in}{1.348051in}}%
\pgfpathlineto{\pgfqpoint{2.245548in}{1.427589in}}%
\pgfpathlineto{\pgfqpoint{2.285831in}{1.421820in}}%
\pgfpathlineto{\pgfqpoint{2.326488in}{1.440527in}}%
\pgfpathlineto{\pgfqpoint{2.366901in}{1.417906in}}%
\pgfpathlineto{\pgfqpoint{2.407697in}{1.451141in}}%
\pgfpathlineto{\pgfqpoint{2.448720in}{1.394678in}}%
\pgfpathlineto{\pgfqpoint{2.490641in}{1.450737in}}%
\pgfpathlineto{\pgfqpoint{2.531132in}{1.457951in}}%
\pgfpathlineto{\pgfqpoint{2.570834in}{1.541072in}}%
\pgfpathlineto{\pgfqpoint{2.611169in}{1.549549in}}%
\pgfpathlineto{\pgfqpoint{2.651724in}{1.608728in}}%
\pgfpathlineto{\pgfqpoint{2.692035in}{1.692682in}}%
\pgfpathlineto{\pgfqpoint{2.732550in}{1.560453in}}%
\pgfpathlineto{\pgfqpoint{2.773597in}{1.477824in}}%
\pgfpathlineto{\pgfqpoint{2.813674in}{1.416043in}}%
\pgfpathlineto{\pgfqpoint{2.854476in}{1.470744in}}%
\pgfpathlineto{\pgfqpoint{2.895048in}{1.642707in}}%
\pgfpathlineto{\pgfqpoint{2.935784in}{1.705391in}}%
\pgfpathlineto{\pgfqpoint{2.976719in}{1.718157in}}%
\pgfpathlineto{\pgfqpoint{3.017062in}{1.702564in}}%
\pgfpathlineto{\pgfqpoint{3.057322in}{1.626944in}}%
\pgfpathlineto{\pgfqpoint{3.097933in}{1.425529in}}%
\pgfpathlineto{\pgfqpoint{3.138512in}{1.220589in}}%
\pgfpathlineto{\pgfqpoint{3.179056in}{1.193997in}}%
\pgfpathlineto{\pgfqpoint{3.219658in}{1.341998in}}%
\pgfpathlineto{\pgfqpoint{3.260223in}{1.541016in}}%
\pgfpathlineto{\pgfqpoint{3.300661in}{1.609306in}}%
\pgfpathlineto{\pgfqpoint{3.341435in}{1.691427in}}%
\pgfpathlineto{\pgfqpoint{3.383778in}{1.677545in}}%
\pgfpathlineto{\pgfqpoint{3.423782in}{1.570038in}}%
\pgfpathlineto{\pgfqpoint{3.463494in}{1.484717in}}%
\pgfpathlineto{\pgfqpoint{3.504318in}{1.475598in}}%
\pgfpathlineto{\pgfqpoint{3.544828in}{1.422372in}}%
\pgfpathlineto{\pgfqpoint{3.585171in}{1.438292in}}%
\pgfpathlineto{\pgfqpoint{3.625984in}{1.438494in}}%
\pgfpathlineto{\pgfqpoint{3.666622in}{1.404948in}}%
\pgfpathlineto{\pgfqpoint{3.707196in}{1.350673in}}%
\pgfpathlineto{\pgfqpoint{3.747556in}{1.375852in}}%
\pgfpathlineto{\pgfqpoint{3.788425in}{1.471325in}}%
\pgfpathlineto{\pgfqpoint{3.830005in}{1.482362in}}%
\pgfpathlineto{\pgfqpoint{3.869826in}{1.553656in}}%
\pgfpathlineto{\pgfqpoint{3.910046in}{1.528152in}}%
\pgfpathlineto{\pgfqpoint{3.950283in}{1.453777in}}%
\pgfpathlineto{\pgfqpoint{3.990708in}{1.290333in}}%
\pgfpathlineto{\pgfqpoint{4.031134in}{1.313275in}}%
\pgfpathlineto{\pgfqpoint{4.072580in}{1.323198in}}%
\pgfpathlineto{\pgfqpoint{4.113318in}{1.409107in}}%
\pgfpathlineto{\pgfqpoint{4.153374in}{1.422368in}}%
\pgfpathlineto{\pgfqpoint{4.194044in}{1.443599in}}%
\pgfpathlineto{\pgfqpoint{4.234731in}{1.445225in}}%
\pgfpathlineto{\pgfqpoint{4.275232in}{1.494624in}}%
\pgfpathlineto{\pgfqpoint{4.316762in}{1.503508in}}%
\pgfpathlineto{\pgfqpoint{4.357267in}{1.511906in}}%
\pgfpathlineto{\pgfqpoint{4.397709in}{1.439455in}}%
\pgfpathlineto{\pgfqpoint{4.437744in}{1.402101in}}%
\pgfpathlineto{\pgfqpoint{4.478174in}{1.494907in}}%
\pgfpathlineto{\pgfqpoint{4.518728in}{1.567651in}}%
\pgfpathlineto{\pgfqpoint{4.560022in}{1.498514in}}%
\pgfpathlineto{\pgfqpoint{4.600507in}{1.307873in}}%
\pgfpathlineto{\pgfqpoint{4.641060in}{1.300932in}}%
\pgfpathlineto{\pgfqpoint{4.681495in}{1.464097in}}%
\pgfpathlineto{\pgfqpoint{4.722049in}{1.491960in}}%
\pgfpathlineto{\pgfqpoint{4.762980in}{1.551910in}}%
\pgfpathlineto{\pgfqpoint{4.805032in}{1.689566in}}%
\pgfpathlineto{\pgfqpoint{4.845389in}{1.632753in}}%
\pgfpathlineto{\pgfqpoint{4.886100in}{1.580402in}}%
\pgfpathlineto{\pgfqpoint{4.925758in}{1.359654in}}%
\pgfpathlineto{\pgfqpoint{4.966245in}{1.278379in}}%
\pgfpathlineto{\pgfqpoint{5.007169in}{1.363648in}}%
\pgfpathlineto{\pgfqpoint{5.047214in}{1.533566in}}%
\pgfpathlineto{\pgfqpoint{5.087943in}{1.654814in}}%
\pgfpathlineto{\pgfqpoint{5.128869in}{1.758935in}}%
\pgfpathlineto{\pgfqpoint{5.169096in}{1.814533in}}%
\pgfpathlineto{\pgfqpoint{5.210596in}{1.842596in}}%
\pgfpathlineto{\pgfqpoint{5.250553in}{1.650418in}}%
\pgfpathlineto{\pgfqpoint{5.290840in}{1.429022in}}%
\pgfpathlineto{\pgfqpoint{5.331359in}{1.382208in}}%
\pgfpathlineto{\pgfqpoint{5.372117in}{1.423265in}}%
\pgfpathlineto{\pgfqpoint{5.412364in}{1.492497in}}%
\pgfpathlineto{\pgfqpoint{5.453645in}{1.517448in}}%
\pgfpathlineto{\pgfqpoint{5.493892in}{1.252375in}}%
\pgfpathlineto{\pgfqpoint{5.534434in}{1.224771in}}%
\pgfusepath{stroke}%
\end{pgfscope}%
\begin{pgfscope}%
\pgfpathrectangle{\pgfqpoint{0.800000in}{0.528000in}}{\pgfqpoint{4.960000in}{3.696000in}}%
\pgfusepath{clip}%
\pgfsetrectcap%
\pgfsetroundjoin%
\pgfsetlinewidth{1.505625pt}%
\definecolor{currentstroke}{rgb}{0.549020,0.337255,0.294118}%
\pgfsetstrokecolor{currentstroke}%
\pgfsetdash{}{0pt}%
\pgfpathmoveto{\pgfqpoint{1.025455in}{1.738282in}}%
\pgfpathlineto{\pgfqpoint{1.066223in}{2.875776in}}%
\pgfpathlineto{\pgfqpoint{1.106499in}{3.530166in}}%
\pgfpathlineto{\pgfqpoint{1.148337in}{3.888444in}}%
\pgfpathlineto{\pgfqpoint{1.188029in}{3.984470in}}%
\pgfpathlineto{\pgfqpoint{1.228111in}{4.023304in}}%
\pgfpathlineto{\pgfqpoint{1.268566in}{3.991144in}}%
\pgfpathlineto{\pgfqpoint{1.309141in}{4.007504in}}%
\pgfpathlineto{\pgfqpoint{1.350020in}{3.964805in}}%
\pgfpathlineto{\pgfqpoint{1.390648in}{3.846940in}}%
\pgfpathlineto{\pgfqpoint{1.431724in}{3.342401in}}%
\pgfpathlineto{\pgfqpoint{1.471905in}{2.626758in}}%
\pgfpathlineto{\pgfqpoint{1.512654in}{1.733263in}}%
\pgfpathlineto{\pgfqpoint{1.553221in}{1.400260in}}%
\pgfpathlineto{\pgfqpoint{1.595260in}{1.384917in}}%
\pgfpathlineto{\pgfqpoint{1.636935in}{1.332273in}}%
\pgfpathlineto{\pgfqpoint{1.675254in}{1.162444in}}%
\pgfpathlineto{\pgfqpoint{1.715503in}{0.696000in}}%
\pgfpathlineto{\pgfqpoint{1.755929in}{0.894483in}}%
\pgfpathlineto{\pgfqpoint{1.796365in}{0.887152in}}%
\pgfpathlineto{\pgfqpoint{1.837451in}{1.024033in}}%
\pgfpathlineto{\pgfqpoint{1.878600in}{1.077871in}}%
\pgfpathlineto{\pgfqpoint{1.919127in}{1.010725in}}%
\pgfpathlineto{\pgfqpoint{1.959106in}{1.220699in}}%
\pgfpathlineto{\pgfqpoint{1.999788in}{1.331481in}}%
\pgfpathlineto{\pgfqpoint{2.041459in}{1.317294in}}%
\pgfpathlineto{\pgfqpoint{2.081255in}{1.283444in}}%
\pgfpathlineto{\pgfqpoint{2.121635in}{1.466926in}}%
\pgfpathlineto{\pgfqpoint{2.161899in}{1.418851in}}%
\pgfpathlineto{\pgfqpoint{2.202364in}{1.464467in}}%
\pgfpathlineto{\pgfqpoint{2.242971in}{1.538681in}}%
\pgfpathlineto{\pgfqpoint{2.283544in}{1.584901in}}%
\pgfpathlineto{\pgfqpoint{2.324971in}{1.536788in}}%
\pgfpathlineto{\pgfqpoint{2.366268in}{1.534714in}}%
\pgfpathlineto{\pgfqpoint{2.406552in}{1.550121in}}%
\pgfpathlineto{\pgfqpoint{2.446935in}{1.605001in}}%
\pgfpathlineto{\pgfqpoint{2.489607in}{1.600831in}}%
\pgfpathlineto{\pgfqpoint{2.528542in}{1.633172in}}%
\pgfpathlineto{\pgfqpoint{2.568732in}{1.642655in}}%
\pgfpathlineto{\pgfqpoint{2.609857in}{1.649378in}}%
\pgfpathlineto{\pgfqpoint{2.649399in}{1.662196in}}%
\pgfpathlineto{\pgfqpoint{2.690145in}{1.582759in}}%
\pgfpathlineto{\pgfqpoint{2.730788in}{1.551166in}}%
\pgfpathlineto{\pgfqpoint{2.771304in}{1.460230in}}%
\pgfpathlineto{\pgfqpoint{2.811958in}{1.445872in}}%
\pgfpathlineto{\pgfqpoint{2.852327in}{1.374310in}}%
\pgfpathlineto{\pgfqpoint{2.892871in}{1.412638in}}%
\pgfpathlineto{\pgfqpoint{2.933625in}{1.408174in}}%
\pgfpathlineto{\pgfqpoint{2.977152in}{1.395793in}}%
\pgfpathlineto{\pgfqpoint{3.014926in}{1.414408in}}%
\pgfpathlineto{\pgfqpoint{3.055095in}{1.596309in}}%
\pgfpathlineto{\pgfqpoint{3.096718in}{1.665436in}}%
\pgfpathlineto{\pgfqpoint{3.136424in}{1.666811in}}%
\pgfpathlineto{\pgfqpoint{3.177036in}{1.603158in}}%
\pgfpathlineto{\pgfqpoint{3.218179in}{1.556927in}}%
\pgfpathlineto{\pgfqpoint{3.258648in}{1.591426in}}%
\pgfpathlineto{\pgfqpoint{3.299755in}{1.663143in}}%
\pgfpathlineto{\pgfqpoint{3.340133in}{1.698690in}}%
\pgfpathlineto{\pgfqpoint{3.380363in}{1.643966in}}%
\pgfpathlineto{\pgfqpoint{3.420858in}{1.634715in}}%
\pgfpathlineto{\pgfqpoint{3.462221in}{1.611011in}}%
\pgfpathlineto{\pgfqpoint{3.502129in}{1.519279in}}%
\pgfpathlineto{\pgfqpoint{3.542548in}{1.383111in}}%
\pgfpathlineto{\pgfqpoint{3.583245in}{1.385507in}}%
\pgfpathlineto{\pgfqpoint{3.623562in}{1.395910in}}%
\pgfpathlineto{\pgfqpoint{3.664149in}{1.390536in}}%
\pgfpathlineto{\pgfqpoint{3.704697in}{1.353861in}}%
\pgfpathlineto{\pgfqpoint{3.745350in}{1.382681in}}%
\pgfpathlineto{\pgfqpoint{3.785431in}{1.258879in}}%
\pgfpathlineto{\pgfqpoint{3.826117in}{1.134319in}}%
\pgfpathlineto{\pgfqpoint{3.866700in}{1.363409in}}%
\pgfpathlineto{\pgfqpoint{3.908857in}{1.769249in}}%
\pgfpathlineto{\pgfqpoint{3.948531in}{1.783751in}}%
\pgfpathlineto{\pgfqpoint{3.988387in}{1.733901in}}%
\pgfpathlineto{\pgfqpoint{4.028631in}{1.678121in}}%
\pgfpathlineto{\pgfqpoint{4.069478in}{1.597989in}}%
\pgfpathlineto{\pgfqpoint{4.109521in}{1.518189in}}%
\pgfpathlineto{\pgfqpoint{4.150173in}{1.190702in}}%
\pgfpathlineto{\pgfqpoint{4.191107in}{1.183549in}}%
\pgfpathlineto{\pgfqpoint{4.231721in}{1.384649in}}%
\pgfpathlineto{\pgfqpoint{4.272097in}{1.512972in}}%
\pgfpathlineto{\pgfqpoint{4.312947in}{1.565344in}}%
\pgfpathlineto{\pgfqpoint{4.354453in}{1.584973in}}%
\pgfpathlineto{\pgfqpoint{4.394197in}{1.643893in}}%
\pgfpathlineto{\pgfqpoint{4.434516in}{1.640154in}}%
\pgfpathlineto{\pgfqpoint{4.474795in}{1.529990in}}%
\pgfpathlineto{\pgfqpoint{4.515437in}{1.545724in}}%
\pgfpathlineto{\pgfqpoint{4.556114in}{1.512609in}}%
\pgfpathlineto{\pgfqpoint{4.596366in}{1.350014in}}%
\pgfpathlineto{\pgfqpoint{4.636941in}{1.090371in}}%
\pgfpathlineto{\pgfqpoint{4.680201in}{1.177825in}}%
\pgfpathlineto{\pgfqpoint{4.718390in}{1.347936in}}%
\pgfpathlineto{\pgfqpoint{4.759105in}{1.413766in}}%
\pgfpathlineto{\pgfqpoint{4.801667in}{1.618661in}}%
\pgfpathlineto{\pgfqpoint{4.841323in}{1.581851in}}%
\pgfpathlineto{\pgfqpoint{4.881504in}{1.603248in}}%
\pgfpathlineto{\pgfqpoint{4.922071in}{1.514565in}}%
\pgfpathlineto{\pgfqpoint{4.962759in}{1.617429in}}%
\pgfpathlineto{\pgfqpoint{5.003261in}{1.674104in}}%
\pgfpathlineto{\pgfqpoint{5.043942in}{1.557161in}}%
\pgfpathlineto{\pgfqpoint{5.084723in}{1.232911in}}%
\pgfpathlineto{\pgfqpoint{5.125090in}{1.126285in}}%
\pgfpathlineto{\pgfqpoint{5.165918in}{1.152323in}}%
\pgfpathlineto{\pgfqpoint{5.207093in}{1.488646in}}%
\pgfpathlineto{\pgfqpoint{5.246975in}{1.625669in}}%
\pgfpathlineto{\pgfqpoint{5.287356in}{1.571326in}}%
\pgfpathlineto{\pgfqpoint{5.327748in}{1.588697in}}%
\pgfpathlineto{\pgfqpoint{5.368195in}{1.564669in}}%
\pgfpathlineto{\pgfqpoint{5.409062in}{1.553566in}}%
\pgfpathlineto{\pgfqpoint{5.449271in}{1.555410in}}%
\pgfpathlineto{\pgfqpoint{5.490010in}{1.524005in}}%
\pgfpathlineto{\pgfqpoint{5.530944in}{1.538896in}}%
\pgfusepath{stroke}%
\end{pgfscope}%
\begin{pgfscope}%
\pgfsetrectcap%
\pgfsetmiterjoin%
\pgfsetlinewidth{0.803000pt}%
\definecolor{currentstroke}{rgb}{0.000000,0.000000,0.000000}%
\pgfsetstrokecolor{currentstroke}%
\pgfsetdash{}{0pt}%
\pgfpathmoveto{\pgfqpoint{0.800000in}{0.528000in}}%
\pgfpathlineto{\pgfqpoint{0.800000in}{4.224000in}}%
\pgfusepath{stroke}%
\end{pgfscope}%
\begin{pgfscope}%
\pgfsetrectcap%
\pgfsetmiterjoin%
\pgfsetlinewidth{0.803000pt}%
\definecolor{currentstroke}{rgb}{0.000000,0.000000,0.000000}%
\pgfsetstrokecolor{currentstroke}%
\pgfsetdash{}{0pt}%
\pgfpathmoveto{\pgfqpoint{5.760000in}{0.528000in}}%
\pgfpathlineto{\pgfqpoint{5.760000in}{4.224000in}}%
\pgfusepath{stroke}%
\end{pgfscope}%
\begin{pgfscope}%
\pgfsetrectcap%
\pgfsetmiterjoin%
\pgfsetlinewidth{0.803000pt}%
\definecolor{currentstroke}{rgb}{0.000000,0.000000,0.000000}%
\pgfsetstrokecolor{currentstroke}%
\pgfsetdash{}{0pt}%
\pgfpathmoveto{\pgfqpoint{0.800000in}{0.528000in}}%
\pgfpathlineto{\pgfqpoint{5.760000in}{0.528000in}}%
\pgfusepath{stroke}%
\end{pgfscope}%
\begin{pgfscope}%
\pgfsetrectcap%
\pgfsetmiterjoin%
\pgfsetlinewidth{0.803000pt}%
\definecolor{currentstroke}{rgb}{0.000000,0.000000,0.000000}%
\pgfsetstrokecolor{currentstroke}%
\pgfsetdash{}{0pt}%
\pgfpathmoveto{\pgfqpoint{0.800000in}{4.224000in}}%
\pgfpathlineto{\pgfqpoint{5.760000in}{4.224000in}}%
\pgfusepath{stroke}%
\end{pgfscope}%
\begin{pgfscope}%
\definecolor{textcolor}{rgb}{0.000000,0.000000,0.000000}%
\pgfsetstrokecolor{textcolor}%
\pgfsetfillcolor{textcolor}%
\pgftext[x=3.280000in,y=4.307333in,,base]{\color{textcolor}\sffamily\fontsize{12.000000}{14.400000}\selectfont Measured yaw speed}%
\end{pgfscope}%
\begin{pgfscope}%
\pgfsetbuttcap%
\pgfsetmiterjoin%
\definecolor{currentfill}{rgb}{1.000000,1.000000,1.000000}%
\pgfsetfillcolor{currentfill}%
\pgfsetfillopacity{0.800000}%
\pgfsetlinewidth{1.003750pt}%
\definecolor{currentstroke}{rgb}{0.800000,0.800000,0.800000}%
\pgfsetstrokecolor{currentstroke}%
\pgfsetstrokeopacity{0.800000}%
\pgfsetdash{}{0pt}%
\pgfpathmoveto{\pgfqpoint{5.041603in}{2.889746in}}%
\pgfpathlineto{\pgfqpoint{5.662778in}{2.889746in}}%
\pgfpathquadraticcurveto{\pgfqpoint{5.690556in}{2.889746in}}{\pgfqpoint{5.690556in}{2.917523in}}%
\pgfpathlineto{\pgfqpoint{5.690556in}{4.126778in}}%
\pgfpathquadraticcurveto{\pgfqpoint{5.690556in}{4.154556in}}{\pgfqpoint{5.662778in}{4.154556in}}%
\pgfpathlineto{\pgfqpoint{5.041603in}{4.154556in}}%
\pgfpathquadraticcurveto{\pgfqpoint{5.013825in}{4.154556in}}{\pgfqpoint{5.013825in}{4.126778in}}%
\pgfpathlineto{\pgfqpoint{5.013825in}{2.917523in}}%
\pgfpathquadraticcurveto{\pgfqpoint{5.013825in}{2.889746in}}{\pgfqpoint{5.041603in}{2.889746in}}%
\pgfpathlineto{\pgfqpoint{5.041603in}{2.889746in}}%
\pgfpathclose%
\pgfusepath{stroke,fill}%
\end{pgfscope}%
\begin{pgfscope}%
\pgfsetrectcap%
\pgfsetroundjoin%
\pgfsetlinewidth{1.505625pt}%
\definecolor{currentstroke}{rgb}{0.121569,0.466667,0.705882}%
\pgfsetstrokecolor{currentstroke}%
\pgfsetdash{}{0pt}%
\pgfpathmoveto{\pgfqpoint{5.069380in}{4.042088in}}%
\pgfpathlineto{\pgfqpoint{5.208269in}{4.042088in}}%
\pgfpathlineto{\pgfqpoint{5.347158in}{4.042088in}}%
\pgfusepath{stroke}%
\end{pgfscope}%
\begin{pgfscope}%
\definecolor{textcolor}{rgb}{0.000000,0.000000,0.000000}%
\pgfsetstrokecolor{textcolor}%
\pgfsetfillcolor{textcolor}%
\pgftext[x=5.458269in,y=3.993477in,left,base]{\color{textcolor}\sffamily\fontsize{10.000000}{12.000000}\selectfont 0}%
\end{pgfscope}%
\begin{pgfscope}%
\pgfsetrectcap%
\pgfsetroundjoin%
\pgfsetlinewidth{1.505625pt}%
\definecolor{currentstroke}{rgb}{1.000000,0.498039,0.054902}%
\pgfsetstrokecolor{currentstroke}%
\pgfsetdash{}{0pt}%
\pgfpathmoveto{\pgfqpoint{5.069380in}{3.838231in}}%
\pgfpathlineto{\pgfqpoint{5.208269in}{3.838231in}}%
\pgfpathlineto{\pgfqpoint{5.347158in}{3.838231in}}%
\pgfusepath{stroke}%
\end{pgfscope}%
\begin{pgfscope}%
\definecolor{textcolor}{rgb}{0.000000,0.000000,0.000000}%
\pgfsetstrokecolor{textcolor}%
\pgfsetfillcolor{textcolor}%
\pgftext[x=5.458269in,y=3.789620in,left,base]{\color{textcolor}\sffamily\fontsize{10.000000}{12.000000}\selectfont 10}%
\end{pgfscope}%
\begin{pgfscope}%
\pgfsetrectcap%
\pgfsetroundjoin%
\pgfsetlinewidth{1.505625pt}%
\definecolor{currentstroke}{rgb}{0.172549,0.627451,0.172549}%
\pgfsetstrokecolor{currentstroke}%
\pgfsetdash{}{0pt}%
\pgfpathmoveto{\pgfqpoint{5.069380in}{3.634374in}}%
\pgfpathlineto{\pgfqpoint{5.208269in}{3.634374in}}%
\pgfpathlineto{\pgfqpoint{5.347158in}{3.634374in}}%
\pgfusepath{stroke}%
\end{pgfscope}%
\begin{pgfscope}%
\definecolor{textcolor}{rgb}{0.000000,0.000000,0.000000}%
\pgfsetstrokecolor{textcolor}%
\pgfsetfillcolor{textcolor}%
\pgftext[x=5.458269in,y=3.585762in,left,base]{\color{textcolor}\sffamily\fontsize{10.000000}{12.000000}\selectfont 20}%
\end{pgfscope}%
\begin{pgfscope}%
\pgfsetrectcap%
\pgfsetroundjoin%
\pgfsetlinewidth{1.505625pt}%
\definecolor{currentstroke}{rgb}{0.839216,0.152941,0.156863}%
\pgfsetstrokecolor{currentstroke}%
\pgfsetdash{}{0pt}%
\pgfpathmoveto{\pgfqpoint{5.069380in}{3.430516in}}%
\pgfpathlineto{\pgfqpoint{5.208269in}{3.430516in}}%
\pgfpathlineto{\pgfqpoint{5.347158in}{3.430516in}}%
\pgfusepath{stroke}%
\end{pgfscope}%
\begin{pgfscope}%
\definecolor{textcolor}{rgb}{0.000000,0.000000,0.000000}%
\pgfsetstrokecolor{textcolor}%
\pgfsetfillcolor{textcolor}%
\pgftext[x=5.458269in,y=3.381905in,left,base]{\color{textcolor}\sffamily\fontsize{10.000000}{12.000000}\selectfont 30}%
\end{pgfscope}%
\begin{pgfscope}%
\pgfsetrectcap%
\pgfsetroundjoin%
\pgfsetlinewidth{1.505625pt}%
\definecolor{currentstroke}{rgb}{0.580392,0.403922,0.741176}%
\pgfsetstrokecolor{currentstroke}%
\pgfsetdash{}{0pt}%
\pgfpathmoveto{\pgfqpoint{5.069380in}{3.226659in}}%
\pgfpathlineto{\pgfqpoint{5.208269in}{3.226659in}}%
\pgfpathlineto{\pgfqpoint{5.347158in}{3.226659in}}%
\pgfusepath{stroke}%
\end{pgfscope}%
\begin{pgfscope}%
\definecolor{textcolor}{rgb}{0.000000,0.000000,0.000000}%
\pgfsetstrokecolor{textcolor}%
\pgfsetfillcolor{textcolor}%
\pgftext[x=5.458269in,y=3.178048in,left,base]{\color{textcolor}\sffamily\fontsize{10.000000}{12.000000}\selectfont 40}%
\end{pgfscope}%
\begin{pgfscope}%
\pgfsetrectcap%
\pgfsetroundjoin%
\pgfsetlinewidth{1.505625pt}%
\definecolor{currentstroke}{rgb}{0.549020,0.337255,0.294118}%
\pgfsetstrokecolor{currentstroke}%
\pgfsetdash{}{0pt}%
\pgfpathmoveto{\pgfqpoint{5.069380in}{3.022802in}}%
\pgfpathlineto{\pgfqpoint{5.208269in}{3.022802in}}%
\pgfpathlineto{\pgfqpoint{5.347158in}{3.022802in}}%
\pgfusepath{stroke}%
\end{pgfscope}%
\begin{pgfscope}%
\definecolor{textcolor}{rgb}{0.000000,0.000000,0.000000}%
\pgfsetstrokecolor{textcolor}%
\pgfsetfillcolor{textcolor}%
\pgftext[x=5.458269in,y=2.974191in,left,base]{\color{textcolor}\sffamily\fontsize{10.000000}{12.000000}\selectfont 50}%
\end{pgfscope}%
\end{pgfpicture}%
\makeatother%
\endgroup%
}
    \end{minipage}
    \caption{Variation of (a) measured yaw heading and (b) measured yaw velocity for different values of $K_{I}$ and $K_P=100$, $K_D=0$ while the yaw controller is engaged.}
    \label{fig:tune-yaw-int-measures}
\end{figure}

\begin{figure}[H]
    \begin{minipage}[t]{0.5\linewidth}
        \centering
        \scalebox{0.55}{%% Creator: Matplotlib, PGF backend
%%
%% To include the figure in your LaTeX document, write
%%   \input{<filename>.pgf}
%%
%% Make sure the required packages are loaded in your preamble
%%   \usepackage{pgf}
%%
%% Also ensure that all the required font packages are loaded; for instance,
%% the lmodern package is sometimes necessary when using math font.
%%   \usepackage{lmodern}
%%
%% Figures using additional raster images can only be included by \input if
%% they are in the same directory as the main LaTeX file. For loading figures
%% from other directories you can use the `import` package
%%   \usepackage{import}
%%
%% and then include the figures with
%%   \import{<path to file>}{<filename>.pgf}
%%
%% Matplotlib used the following preamble
%%   \usepackage{fontspec}
%%   \setmainfont{DejaVuSerif.ttf}[Path=\detokenize{/home/lgonz/tfg-aero/tfg-giaa-dronecontrol/venv/lib/python3.8/site-packages/matplotlib/mpl-data/fonts/ttf/}]
%%   \setsansfont{DejaVuSans.ttf}[Path=\detokenize{/home/lgonz/tfg-aero/tfg-giaa-dronecontrol/venv/lib/python3.8/site-packages/matplotlib/mpl-data/fonts/ttf/}]
%%   \setmonofont{DejaVuSansMono.ttf}[Path=\detokenize{/home/lgonz/tfg-aero/tfg-giaa-dronecontrol/venv/lib/python3.8/site-packages/matplotlib/mpl-data/fonts/ttf/}]
%%
\begingroup%
\makeatletter%
\begin{pgfpicture}%
\pgfpathrectangle{\pgfpointorigin}{\pgfqpoint{6.400000in}{4.800000in}}%
\pgfusepath{use as bounding box, clip}%
\begin{pgfscope}%
\pgfsetbuttcap%
\pgfsetmiterjoin%
\definecolor{currentfill}{rgb}{1.000000,1.000000,1.000000}%
\pgfsetfillcolor{currentfill}%
\pgfsetlinewidth{0.000000pt}%
\definecolor{currentstroke}{rgb}{1.000000,1.000000,1.000000}%
\pgfsetstrokecolor{currentstroke}%
\pgfsetdash{}{0pt}%
\pgfpathmoveto{\pgfqpoint{0.000000in}{0.000000in}}%
\pgfpathlineto{\pgfqpoint{6.400000in}{0.000000in}}%
\pgfpathlineto{\pgfqpoint{6.400000in}{4.800000in}}%
\pgfpathlineto{\pgfqpoint{0.000000in}{4.800000in}}%
\pgfpathlineto{\pgfqpoint{0.000000in}{0.000000in}}%
\pgfpathclose%
\pgfusepath{fill}%
\end{pgfscope}%
\begin{pgfscope}%
\pgfsetbuttcap%
\pgfsetmiterjoin%
\definecolor{currentfill}{rgb}{1.000000,1.000000,1.000000}%
\pgfsetfillcolor{currentfill}%
\pgfsetlinewidth{0.000000pt}%
\definecolor{currentstroke}{rgb}{0.000000,0.000000,0.000000}%
\pgfsetstrokecolor{currentstroke}%
\pgfsetstrokeopacity{0.000000}%
\pgfsetdash{}{0pt}%
\pgfpathmoveto{\pgfqpoint{0.800000in}{0.528000in}}%
\pgfpathlineto{\pgfqpoint{5.760000in}{0.528000in}}%
\pgfpathlineto{\pgfqpoint{5.760000in}{4.224000in}}%
\pgfpathlineto{\pgfqpoint{0.800000in}{4.224000in}}%
\pgfpathlineto{\pgfqpoint{0.800000in}{0.528000in}}%
\pgfpathclose%
\pgfusepath{fill}%
\end{pgfscope}%
\begin{pgfscope}%
\pgfpathrectangle{\pgfqpoint{0.800000in}{0.528000in}}{\pgfqpoint{4.960000in}{3.696000in}}%
\pgfusepath{clip}%
\pgfsetrectcap%
\pgfsetroundjoin%
\pgfsetlinewidth{0.803000pt}%
\definecolor{currentstroke}{rgb}{0.690196,0.690196,0.690196}%
\pgfsetstrokecolor{currentstroke}%
\pgfsetdash{}{0pt}%
\pgfpathmoveto{\pgfqpoint{1.025455in}{0.528000in}}%
\pgfpathlineto{\pgfqpoint{1.025455in}{4.224000in}}%
\pgfusepath{stroke}%
\end{pgfscope}%
\begin{pgfscope}%
\pgfsetbuttcap%
\pgfsetroundjoin%
\definecolor{currentfill}{rgb}{0.000000,0.000000,0.000000}%
\pgfsetfillcolor{currentfill}%
\pgfsetlinewidth{0.803000pt}%
\definecolor{currentstroke}{rgb}{0.000000,0.000000,0.000000}%
\pgfsetstrokecolor{currentstroke}%
\pgfsetdash{}{0pt}%
\pgfsys@defobject{currentmarker}{\pgfqpoint{0.000000in}{-0.048611in}}{\pgfqpoint{0.000000in}{0.000000in}}{%
\pgfpathmoveto{\pgfqpoint{0.000000in}{0.000000in}}%
\pgfpathlineto{\pgfqpoint{0.000000in}{-0.048611in}}%
\pgfusepath{stroke,fill}%
}%
\begin{pgfscope}%
\pgfsys@transformshift{1.025455in}{0.528000in}%
\pgfsys@useobject{currentmarker}{}%
\end{pgfscope}%
\end{pgfscope}%
\begin{pgfscope}%
\definecolor{textcolor}{rgb}{0.000000,0.000000,0.000000}%
\pgfsetstrokecolor{textcolor}%
\pgfsetfillcolor{textcolor}%
\pgftext[x=1.025455in,y=0.430778in,,top]{\color{textcolor}\sffamily\fontsize{10.000000}{12.000000}\selectfont 0}%
\end{pgfscope}%
\begin{pgfscope}%
\pgfpathrectangle{\pgfqpoint{0.800000in}{0.528000in}}{\pgfqpoint{4.960000in}{3.696000in}}%
\pgfusepath{clip}%
\pgfsetrectcap%
\pgfsetroundjoin%
\pgfsetlinewidth{0.803000pt}%
\definecolor{currentstroke}{rgb}{0.690196,0.690196,0.690196}%
\pgfsetstrokecolor{currentstroke}%
\pgfsetdash{}{0pt}%
\pgfpathmoveto{\pgfqpoint{1.775826in}{0.528000in}}%
\pgfpathlineto{\pgfqpoint{1.775826in}{4.224000in}}%
\pgfusepath{stroke}%
\end{pgfscope}%
\begin{pgfscope}%
\pgfsetbuttcap%
\pgfsetroundjoin%
\definecolor{currentfill}{rgb}{0.000000,0.000000,0.000000}%
\pgfsetfillcolor{currentfill}%
\pgfsetlinewidth{0.803000pt}%
\definecolor{currentstroke}{rgb}{0.000000,0.000000,0.000000}%
\pgfsetstrokecolor{currentstroke}%
\pgfsetdash{}{0pt}%
\pgfsys@defobject{currentmarker}{\pgfqpoint{0.000000in}{-0.048611in}}{\pgfqpoint{0.000000in}{0.000000in}}{%
\pgfpathmoveto{\pgfqpoint{0.000000in}{0.000000in}}%
\pgfpathlineto{\pgfqpoint{0.000000in}{-0.048611in}}%
\pgfusepath{stroke,fill}%
}%
\begin{pgfscope}%
\pgfsys@transformshift{1.775826in}{0.528000in}%
\pgfsys@useobject{currentmarker}{}%
\end{pgfscope}%
\end{pgfscope}%
\begin{pgfscope}%
\definecolor{textcolor}{rgb}{0.000000,0.000000,0.000000}%
\pgfsetstrokecolor{textcolor}%
\pgfsetfillcolor{textcolor}%
\pgftext[x=1.775826in,y=0.430778in,,top]{\color{textcolor}\sffamily\fontsize{10.000000}{12.000000}\selectfont 5}%
\end{pgfscope}%
\begin{pgfscope}%
\pgfpathrectangle{\pgfqpoint{0.800000in}{0.528000in}}{\pgfqpoint{4.960000in}{3.696000in}}%
\pgfusepath{clip}%
\pgfsetrectcap%
\pgfsetroundjoin%
\pgfsetlinewidth{0.803000pt}%
\definecolor{currentstroke}{rgb}{0.690196,0.690196,0.690196}%
\pgfsetstrokecolor{currentstroke}%
\pgfsetdash{}{0pt}%
\pgfpathmoveto{\pgfqpoint{2.526198in}{0.528000in}}%
\pgfpathlineto{\pgfqpoint{2.526198in}{4.224000in}}%
\pgfusepath{stroke}%
\end{pgfscope}%
\begin{pgfscope}%
\pgfsetbuttcap%
\pgfsetroundjoin%
\definecolor{currentfill}{rgb}{0.000000,0.000000,0.000000}%
\pgfsetfillcolor{currentfill}%
\pgfsetlinewidth{0.803000pt}%
\definecolor{currentstroke}{rgb}{0.000000,0.000000,0.000000}%
\pgfsetstrokecolor{currentstroke}%
\pgfsetdash{}{0pt}%
\pgfsys@defobject{currentmarker}{\pgfqpoint{0.000000in}{-0.048611in}}{\pgfqpoint{0.000000in}{0.000000in}}{%
\pgfpathmoveto{\pgfqpoint{0.000000in}{0.000000in}}%
\pgfpathlineto{\pgfqpoint{0.000000in}{-0.048611in}}%
\pgfusepath{stroke,fill}%
}%
\begin{pgfscope}%
\pgfsys@transformshift{2.526198in}{0.528000in}%
\pgfsys@useobject{currentmarker}{}%
\end{pgfscope}%
\end{pgfscope}%
\begin{pgfscope}%
\definecolor{textcolor}{rgb}{0.000000,0.000000,0.000000}%
\pgfsetstrokecolor{textcolor}%
\pgfsetfillcolor{textcolor}%
\pgftext[x=2.526198in,y=0.430778in,,top]{\color{textcolor}\sffamily\fontsize{10.000000}{12.000000}\selectfont 10}%
\end{pgfscope}%
\begin{pgfscope}%
\pgfpathrectangle{\pgfqpoint{0.800000in}{0.528000in}}{\pgfqpoint{4.960000in}{3.696000in}}%
\pgfusepath{clip}%
\pgfsetrectcap%
\pgfsetroundjoin%
\pgfsetlinewidth{0.803000pt}%
\definecolor{currentstroke}{rgb}{0.690196,0.690196,0.690196}%
\pgfsetstrokecolor{currentstroke}%
\pgfsetdash{}{0pt}%
\pgfpathmoveto{\pgfqpoint{3.276569in}{0.528000in}}%
\pgfpathlineto{\pgfqpoint{3.276569in}{4.224000in}}%
\pgfusepath{stroke}%
\end{pgfscope}%
\begin{pgfscope}%
\pgfsetbuttcap%
\pgfsetroundjoin%
\definecolor{currentfill}{rgb}{0.000000,0.000000,0.000000}%
\pgfsetfillcolor{currentfill}%
\pgfsetlinewidth{0.803000pt}%
\definecolor{currentstroke}{rgb}{0.000000,0.000000,0.000000}%
\pgfsetstrokecolor{currentstroke}%
\pgfsetdash{}{0pt}%
\pgfsys@defobject{currentmarker}{\pgfqpoint{0.000000in}{-0.048611in}}{\pgfqpoint{0.000000in}{0.000000in}}{%
\pgfpathmoveto{\pgfqpoint{0.000000in}{0.000000in}}%
\pgfpathlineto{\pgfqpoint{0.000000in}{-0.048611in}}%
\pgfusepath{stroke,fill}%
}%
\begin{pgfscope}%
\pgfsys@transformshift{3.276569in}{0.528000in}%
\pgfsys@useobject{currentmarker}{}%
\end{pgfscope}%
\end{pgfscope}%
\begin{pgfscope}%
\definecolor{textcolor}{rgb}{0.000000,0.000000,0.000000}%
\pgfsetstrokecolor{textcolor}%
\pgfsetfillcolor{textcolor}%
\pgftext[x=3.276569in,y=0.430778in,,top]{\color{textcolor}\sffamily\fontsize{10.000000}{12.000000}\selectfont 15}%
\end{pgfscope}%
\begin{pgfscope}%
\pgfpathrectangle{\pgfqpoint{0.800000in}{0.528000in}}{\pgfqpoint{4.960000in}{3.696000in}}%
\pgfusepath{clip}%
\pgfsetrectcap%
\pgfsetroundjoin%
\pgfsetlinewidth{0.803000pt}%
\definecolor{currentstroke}{rgb}{0.690196,0.690196,0.690196}%
\pgfsetstrokecolor{currentstroke}%
\pgfsetdash{}{0pt}%
\pgfpathmoveto{\pgfqpoint{4.026940in}{0.528000in}}%
\pgfpathlineto{\pgfqpoint{4.026940in}{4.224000in}}%
\pgfusepath{stroke}%
\end{pgfscope}%
\begin{pgfscope}%
\pgfsetbuttcap%
\pgfsetroundjoin%
\definecolor{currentfill}{rgb}{0.000000,0.000000,0.000000}%
\pgfsetfillcolor{currentfill}%
\pgfsetlinewidth{0.803000pt}%
\definecolor{currentstroke}{rgb}{0.000000,0.000000,0.000000}%
\pgfsetstrokecolor{currentstroke}%
\pgfsetdash{}{0pt}%
\pgfsys@defobject{currentmarker}{\pgfqpoint{0.000000in}{-0.048611in}}{\pgfqpoint{0.000000in}{0.000000in}}{%
\pgfpathmoveto{\pgfqpoint{0.000000in}{0.000000in}}%
\pgfpathlineto{\pgfqpoint{0.000000in}{-0.048611in}}%
\pgfusepath{stroke,fill}%
}%
\begin{pgfscope}%
\pgfsys@transformshift{4.026940in}{0.528000in}%
\pgfsys@useobject{currentmarker}{}%
\end{pgfscope}%
\end{pgfscope}%
\begin{pgfscope}%
\definecolor{textcolor}{rgb}{0.000000,0.000000,0.000000}%
\pgfsetstrokecolor{textcolor}%
\pgfsetfillcolor{textcolor}%
\pgftext[x=4.026940in,y=0.430778in,,top]{\color{textcolor}\sffamily\fontsize{10.000000}{12.000000}\selectfont 20}%
\end{pgfscope}%
\begin{pgfscope}%
\pgfpathrectangle{\pgfqpoint{0.800000in}{0.528000in}}{\pgfqpoint{4.960000in}{3.696000in}}%
\pgfusepath{clip}%
\pgfsetrectcap%
\pgfsetroundjoin%
\pgfsetlinewidth{0.803000pt}%
\definecolor{currentstroke}{rgb}{0.690196,0.690196,0.690196}%
\pgfsetstrokecolor{currentstroke}%
\pgfsetdash{}{0pt}%
\pgfpathmoveto{\pgfqpoint{4.777312in}{0.528000in}}%
\pgfpathlineto{\pgfqpoint{4.777312in}{4.224000in}}%
\pgfusepath{stroke}%
\end{pgfscope}%
\begin{pgfscope}%
\pgfsetbuttcap%
\pgfsetroundjoin%
\definecolor{currentfill}{rgb}{0.000000,0.000000,0.000000}%
\pgfsetfillcolor{currentfill}%
\pgfsetlinewidth{0.803000pt}%
\definecolor{currentstroke}{rgb}{0.000000,0.000000,0.000000}%
\pgfsetstrokecolor{currentstroke}%
\pgfsetdash{}{0pt}%
\pgfsys@defobject{currentmarker}{\pgfqpoint{0.000000in}{-0.048611in}}{\pgfqpoint{0.000000in}{0.000000in}}{%
\pgfpathmoveto{\pgfqpoint{0.000000in}{0.000000in}}%
\pgfpathlineto{\pgfqpoint{0.000000in}{-0.048611in}}%
\pgfusepath{stroke,fill}%
}%
\begin{pgfscope}%
\pgfsys@transformshift{4.777312in}{0.528000in}%
\pgfsys@useobject{currentmarker}{}%
\end{pgfscope}%
\end{pgfscope}%
\begin{pgfscope}%
\definecolor{textcolor}{rgb}{0.000000,0.000000,0.000000}%
\pgfsetstrokecolor{textcolor}%
\pgfsetfillcolor{textcolor}%
\pgftext[x=4.777312in,y=0.430778in,,top]{\color{textcolor}\sffamily\fontsize{10.000000}{12.000000}\selectfont 25}%
\end{pgfscope}%
\begin{pgfscope}%
\pgfpathrectangle{\pgfqpoint{0.800000in}{0.528000in}}{\pgfqpoint{4.960000in}{3.696000in}}%
\pgfusepath{clip}%
\pgfsetrectcap%
\pgfsetroundjoin%
\pgfsetlinewidth{0.803000pt}%
\definecolor{currentstroke}{rgb}{0.690196,0.690196,0.690196}%
\pgfsetstrokecolor{currentstroke}%
\pgfsetdash{}{0pt}%
\pgfpathmoveto{\pgfqpoint{5.527683in}{0.528000in}}%
\pgfpathlineto{\pgfqpoint{5.527683in}{4.224000in}}%
\pgfusepath{stroke}%
\end{pgfscope}%
\begin{pgfscope}%
\pgfsetbuttcap%
\pgfsetroundjoin%
\definecolor{currentfill}{rgb}{0.000000,0.000000,0.000000}%
\pgfsetfillcolor{currentfill}%
\pgfsetlinewidth{0.803000pt}%
\definecolor{currentstroke}{rgb}{0.000000,0.000000,0.000000}%
\pgfsetstrokecolor{currentstroke}%
\pgfsetdash{}{0pt}%
\pgfsys@defobject{currentmarker}{\pgfqpoint{0.000000in}{-0.048611in}}{\pgfqpoint{0.000000in}{0.000000in}}{%
\pgfpathmoveto{\pgfqpoint{0.000000in}{0.000000in}}%
\pgfpathlineto{\pgfqpoint{0.000000in}{-0.048611in}}%
\pgfusepath{stroke,fill}%
}%
\begin{pgfscope}%
\pgfsys@transformshift{5.527683in}{0.528000in}%
\pgfsys@useobject{currentmarker}{}%
\end{pgfscope}%
\end{pgfscope}%
\begin{pgfscope}%
\definecolor{textcolor}{rgb}{0.000000,0.000000,0.000000}%
\pgfsetstrokecolor{textcolor}%
\pgfsetfillcolor{textcolor}%
\pgftext[x=5.527683in,y=0.430778in,,top]{\color{textcolor}\sffamily\fontsize{10.000000}{12.000000}\selectfont 30}%
\end{pgfscope}%
\begin{pgfscope}%
\definecolor{textcolor}{rgb}{0.000000,0.000000,0.000000}%
\pgfsetstrokecolor{textcolor}%
\pgfsetfillcolor{textcolor}%
\pgftext[x=3.280000in,y=0.240809in,,top]{\color{textcolor}\sffamily\fontsize{10.000000}{12.000000}\selectfont time [s]}%
\end{pgfscope}%
\begin{pgfscope}%
\pgfpathrectangle{\pgfqpoint{0.800000in}{0.528000in}}{\pgfqpoint{4.960000in}{3.696000in}}%
\pgfusepath{clip}%
\pgfsetrectcap%
\pgfsetroundjoin%
\pgfsetlinewidth{0.803000pt}%
\definecolor{currentstroke}{rgb}{0.690196,0.690196,0.690196}%
\pgfsetstrokecolor{currentstroke}%
\pgfsetdash{}{0pt}%
\pgfpathmoveto{\pgfqpoint{0.800000in}{0.938410in}}%
\pgfpathlineto{\pgfqpoint{5.760000in}{0.938410in}}%
\pgfusepath{stroke}%
\end{pgfscope}%
\begin{pgfscope}%
\pgfsetbuttcap%
\pgfsetroundjoin%
\definecolor{currentfill}{rgb}{0.000000,0.000000,0.000000}%
\pgfsetfillcolor{currentfill}%
\pgfsetlinewidth{0.803000pt}%
\definecolor{currentstroke}{rgb}{0.000000,0.000000,0.000000}%
\pgfsetstrokecolor{currentstroke}%
\pgfsetdash{}{0pt}%
\pgfsys@defobject{currentmarker}{\pgfqpoint{-0.048611in}{0.000000in}}{\pgfqpoint{-0.000000in}{0.000000in}}{%
\pgfpathmoveto{\pgfqpoint{-0.000000in}{0.000000in}}%
\pgfpathlineto{\pgfqpoint{-0.048611in}{0.000000in}}%
\pgfusepath{stroke,fill}%
}%
\begin{pgfscope}%
\pgfsys@transformshift{0.800000in}{0.938410in}%
\pgfsys@useobject{currentmarker}{}%
\end{pgfscope}%
\end{pgfscope}%
\begin{pgfscope}%
\definecolor{textcolor}{rgb}{0.000000,0.000000,0.000000}%
\pgfsetstrokecolor{textcolor}%
\pgfsetfillcolor{textcolor}%
\pgftext[x=0.197143in, y=0.885648in, left, base]{\color{textcolor}\sffamily\fontsize{10.000000}{12.000000}\selectfont \ensuremath{-}0.125}%
\end{pgfscope}%
\begin{pgfscope}%
\pgfpathrectangle{\pgfqpoint{0.800000in}{0.528000in}}{\pgfqpoint{4.960000in}{3.696000in}}%
\pgfusepath{clip}%
\pgfsetrectcap%
\pgfsetroundjoin%
\pgfsetlinewidth{0.803000pt}%
\definecolor{currentstroke}{rgb}{0.690196,0.690196,0.690196}%
\pgfsetstrokecolor{currentstroke}%
\pgfsetdash{}{0pt}%
\pgfpathmoveto{\pgfqpoint{0.800000in}{1.372927in}}%
\pgfpathlineto{\pgfqpoint{5.760000in}{1.372927in}}%
\pgfusepath{stroke}%
\end{pgfscope}%
\begin{pgfscope}%
\pgfsetbuttcap%
\pgfsetroundjoin%
\definecolor{currentfill}{rgb}{0.000000,0.000000,0.000000}%
\pgfsetfillcolor{currentfill}%
\pgfsetlinewidth{0.803000pt}%
\definecolor{currentstroke}{rgb}{0.000000,0.000000,0.000000}%
\pgfsetstrokecolor{currentstroke}%
\pgfsetdash{}{0pt}%
\pgfsys@defobject{currentmarker}{\pgfqpoint{-0.048611in}{0.000000in}}{\pgfqpoint{-0.000000in}{0.000000in}}{%
\pgfpathmoveto{\pgfqpoint{-0.000000in}{0.000000in}}%
\pgfpathlineto{\pgfqpoint{-0.048611in}{0.000000in}}%
\pgfusepath{stroke,fill}%
}%
\begin{pgfscope}%
\pgfsys@transformshift{0.800000in}{1.372927in}%
\pgfsys@useobject{currentmarker}{}%
\end{pgfscope}%
\end{pgfscope}%
\begin{pgfscope}%
\definecolor{textcolor}{rgb}{0.000000,0.000000,0.000000}%
\pgfsetstrokecolor{textcolor}%
\pgfsetfillcolor{textcolor}%
\pgftext[x=0.197143in, y=1.320166in, left, base]{\color{textcolor}\sffamily\fontsize{10.000000}{12.000000}\selectfont \ensuremath{-}0.100}%
\end{pgfscope}%
\begin{pgfscope}%
\pgfpathrectangle{\pgfqpoint{0.800000in}{0.528000in}}{\pgfqpoint{4.960000in}{3.696000in}}%
\pgfusepath{clip}%
\pgfsetrectcap%
\pgfsetroundjoin%
\pgfsetlinewidth{0.803000pt}%
\definecolor{currentstroke}{rgb}{0.690196,0.690196,0.690196}%
\pgfsetstrokecolor{currentstroke}%
\pgfsetdash{}{0pt}%
\pgfpathmoveto{\pgfqpoint{0.800000in}{1.807445in}}%
\pgfpathlineto{\pgfqpoint{5.760000in}{1.807445in}}%
\pgfusepath{stroke}%
\end{pgfscope}%
\begin{pgfscope}%
\pgfsetbuttcap%
\pgfsetroundjoin%
\definecolor{currentfill}{rgb}{0.000000,0.000000,0.000000}%
\pgfsetfillcolor{currentfill}%
\pgfsetlinewidth{0.803000pt}%
\definecolor{currentstroke}{rgb}{0.000000,0.000000,0.000000}%
\pgfsetstrokecolor{currentstroke}%
\pgfsetdash{}{0pt}%
\pgfsys@defobject{currentmarker}{\pgfqpoint{-0.048611in}{0.000000in}}{\pgfqpoint{-0.000000in}{0.000000in}}{%
\pgfpathmoveto{\pgfqpoint{-0.000000in}{0.000000in}}%
\pgfpathlineto{\pgfqpoint{-0.048611in}{0.000000in}}%
\pgfusepath{stroke,fill}%
}%
\begin{pgfscope}%
\pgfsys@transformshift{0.800000in}{1.807445in}%
\pgfsys@useobject{currentmarker}{}%
\end{pgfscope}%
\end{pgfscope}%
\begin{pgfscope}%
\definecolor{textcolor}{rgb}{0.000000,0.000000,0.000000}%
\pgfsetstrokecolor{textcolor}%
\pgfsetfillcolor{textcolor}%
\pgftext[x=0.197143in, y=1.754684in, left, base]{\color{textcolor}\sffamily\fontsize{10.000000}{12.000000}\selectfont \ensuremath{-}0.075}%
\end{pgfscope}%
\begin{pgfscope}%
\pgfpathrectangle{\pgfqpoint{0.800000in}{0.528000in}}{\pgfqpoint{4.960000in}{3.696000in}}%
\pgfusepath{clip}%
\pgfsetrectcap%
\pgfsetroundjoin%
\pgfsetlinewidth{0.803000pt}%
\definecolor{currentstroke}{rgb}{0.690196,0.690196,0.690196}%
\pgfsetstrokecolor{currentstroke}%
\pgfsetdash{}{0pt}%
\pgfpathmoveto{\pgfqpoint{0.800000in}{2.241963in}}%
\pgfpathlineto{\pgfqpoint{5.760000in}{2.241963in}}%
\pgfusepath{stroke}%
\end{pgfscope}%
\begin{pgfscope}%
\pgfsetbuttcap%
\pgfsetroundjoin%
\definecolor{currentfill}{rgb}{0.000000,0.000000,0.000000}%
\pgfsetfillcolor{currentfill}%
\pgfsetlinewidth{0.803000pt}%
\definecolor{currentstroke}{rgb}{0.000000,0.000000,0.000000}%
\pgfsetstrokecolor{currentstroke}%
\pgfsetdash{}{0pt}%
\pgfsys@defobject{currentmarker}{\pgfqpoint{-0.048611in}{0.000000in}}{\pgfqpoint{-0.000000in}{0.000000in}}{%
\pgfpathmoveto{\pgfqpoint{-0.000000in}{0.000000in}}%
\pgfpathlineto{\pgfqpoint{-0.048611in}{0.000000in}}%
\pgfusepath{stroke,fill}%
}%
\begin{pgfscope}%
\pgfsys@transformshift{0.800000in}{2.241963in}%
\pgfsys@useobject{currentmarker}{}%
\end{pgfscope}%
\end{pgfscope}%
\begin{pgfscope}%
\definecolor{textcolor}{rgb}{0.000000,0.000000,0.000000}%
\pgfsetstrokecolor{textcolor}%
\pgfsetfillcolor{textcolor}%
\pgftext[x=0.197143in, y=2.189201in, left, base]{\color{textcolor}\sffamily\fontsize{10.000000}{12.000000}\selectfont \ensuremath{-}0.050}%
\end{pgfscope}%
\begin{pgfscope}%
\pgfpathrectangle{\pgfqpoint{0.800000in}{0.528000in}}{\pgfqpoint{4.960000in}{3.696000in}}%
\pgfusepath{clip}%
\pgfsetrectcap%
\pgfsetroundjoin%
\pgfsetlinewidth{0.803000pt}%
\definecolor{currentstroke}{rgb}{0.690196,0.690196,0.690196}%
\pgfsetstrokecolor{currentstroke}%
\pgfsetdash{}{0pt}%
\pgfpathmoveto{\pgfqpoint{0.800000in}{2.676480in}}%
\pgfpathlineto{\pgfqpoint{5.760000in}{2.676480in}}%
\pgfusepath{stroke}%
\end{pgfscope}%
\begin{pgfscope}%
\pgfsetbuttcap%
\pgfsetroundjoin%
\definecolor{currentfill}{rgb}{0.000000,0.000000,0.000000}%
\pgfsetfillcolor{currentfill}%
\pgfsetlinewidth{0.803000pt}%
\definecolor{currentstroke}{rgb}{0.000000,0.000000,0.000000}%
\pgfsetstrokecolor{currentstroke}%
\pgfsetdash{}{0pt}%
\pgfsys@defobject{currentmarker}{\pgfqpoint{-0.048611in}{0.000000in}}{\pgfqpoint{-0.000000in}{0.000000in}}{%
\pgfpathmoveto{\pgfqpoint{-0.000000in}{0.000000in}}%
\pgfpathlineto{\pgfqpoint{-0.048611in}{0.000000in}}%
\pgfusepath{stroke,fill}%
}%
\begin{pgfscope}%
\pgfsys@transformshift{0.800000in}{2.676480in}%
\pgfsys@useobject{currentmarker}{}%
\end{pgfscope}%
\end{pgfscope}%
\begin{pgfscope}%
\definecolor{textcolor}{rgb}{0.000000,0.000000,0.000000}%
\pgfsetstrokecolor{textcolor}%
\pgfsetfillcolor{textcolor}%
\pgftext[x=0.197143in, y=2.623719in, left, base]{\color{textcolor}\sffamily\fontsize{10.000000}{12.000000}\selectfont \ensuremath{-}0.025}%
\end{pgfscope}%
\begin{pgfscope}%
\pgfpathrectangle{\pgfqpoint{0.800000in}{0.528000in}}{\pgfqpoint{4.960000in}{3.696000in}}%
\pgfusepath{clip}%
\pgfsetrectcap%
\pgfsetroundjoin%
\pgfsetlinewidth{0.803000pt}%
\definecolor{currentstroke}{rgb}{0.690196,0.690196,0.690196}%
\pgfsetstrokecolor{currentstroke}%
\pgfsetdash{}{0pt}%
\pgfpathmoveto{\pgfqpoint{0.800000in}{3.110998in}}%
\pgfpathlineto{\pgfqpoint{5.760000in}{3.110998in}}%
\pgfusepath{stroke}%
\end{pgfscope}%
\begin{pgfscope}%
\pgfsetbuttcap%
\pgfsetroundjoin%
\definecolor{currentfill}{rgb}{0.000000,0.000000,0.000000}%
\pgfsetfillcolor{currentfill}%
\pgfsetlinewidth{0.803000pt}%
\definecolor{currentstroke}{rgb}{0.000000,0.000000,0.000000}%
\pgfsetstrokecolor{currentstroke}%
\pgfsetdash{}{0pt}%
\pgfsys@defobject{currentmarker}{\pgfqpoint{-0.048611in}{0.000000in}}{\pgfqpoint{-0.000000in}{0.000000in}}{%
\pgfpathmoveto{\pgfqpoint{-0.000000in}{0.000000in}}%
\pgfpathlineto{\pgfqpoint{-0.048611in}{0.000000in}}%
\pgfusepath{stroke,fill}%
}%
\begin{pgfscope}%
\pgfsys@transformshift{0.800000in}{3.110998in}%
\pgfsys@useobject{currentmarker}{}%
\end{pgfscope}%
\end{pgfscope}%
\begin{pgfscope}%
\definecolor{textcolor}{rgb}{0.000000,0.000000,0.000000}%
\pgfsetstrokecolor{textcolor}%
\pgfsetfillcolor{textcolor}%
\pgftext[x=0.305168in, y=3.058236in, left, base]{\color{textcolor}\sffamily\fontsize{10.000000}{12.000000}\selectfont 0.000}%
\end{pgfscope}%
\begin{pgfscope}%
\pgfpathrectangle{\pgfqpoint{0.800000in}{0.528000in}}{\pgfqpoint{4.960000in}{3.696000in}}%
\pgfusepath{clip}%
\pgfsetrectcap%
\pgfsetroundjoin%
\pgfsetlinewidth{0.803000pt}%
\definecolor{currentstroke}{rgb}{0.690196,0.690196,0.690196}%
\pgfsetstrokecolor{currentstroke}%
\pgfsetdash{}{0pt}%
\pgfpathmoveto{\pgfqpoint{0.800000in}{3.545515in}}%
\pgfpathlineto{\pgfqpoint{5.760000in}{3.545515in}}%
\pgfusepath{stroke}%
\end{pgfscope}%
\begin{pgfscope}%
\pgfsetbuttcap%
\pgfsetroundjoin%
\definecolor{currentfill}{rgb}{0.000000,0.000000,0.000000}%
\pgfsetfillcolor{currentfill}%
\pgfsetlinewidth{0.803000pt}%
\definecolor{currentstroke}{rgb}{0.000000,0.000000,0.000000}%
\pgfsetstrokecolor{currentstroke}%
\pgfsetdash{}{0pt}%
\pgfsys@defobject{currentmarker}{\pgfqpoint{-0.048611in}{0.000000in}}{\pgfqpoint{-0.000000in}{0.000000in}}{%
\pgfpathmoveto{\pgfqpoint{-0.000000in}{0.000000in}}%
\pgfpathlineto{\pgfqpoint{-0.048611in}{0.000000in}}%
\pgfusepath{stroke,fill}%
}%
\begin{pgfscope}%
\pgfsys@transformshift{0.800000in}{3.545515in}%
\pgfsys@useobject{currentmarker}{}%
\end{pgfscope}%
\end{pgfscope}%
\begin{pgfscope}%
\definecolor{textcolor}{rgb}{0.000000,0.000000,0.000000}%
\pgfsetstrokecolor{textcolor}%
\pgfsetfillcolor{textcolor}%
\pgftext[x=0.305168in, y=3.492754in, left, base]{\color{textcolor}\sffamily\fontsize{10.000000}{12.000000}\selectfont 0.025}%
\end{pgfscope}%
\begin{pgfscope}%
\pgfpathrectangle{\pgfqpoint{0.800000in}{0.528000in}}{\pgfqpoint{4.960000in}{3.696000in}}%
\pgfusepath{clip}%
\pgfsetrectcap%
\pgfsetroundjoin%
\pgfsetlinewidth{0.803000pt}%
\definecolor{currentstroke}{rgb}{0.690196,0.690196,0.690196}%
\pgfsetstrokecolor{currentstroke}%
\pgfsetdash{}{0pt}%
\pgfpathmoveto{\pgfqpoint{0.800000in}{3.980033in}}%
\pgfpathlineto{\pgfqpoint{5.760000in}{3.980033in}}%
\pgfusepath{stroke}%
\end{pgfscope}%
\begin{pgfscope}%
\pgfsetbuttcap%
\pgfsetroundjoin%
\definecolor{currentfill}{rgb}{0.000000,0.000000,0.000000}%
\pgfsetfillcolor{currentfill}%
\pgfsetlinewidth{0.803000pt}%
\definecolor{currentstroke}{rgb}{0.000000,0.000000,0.000000}%
\pgfsetstrokecolor{currentstroke}%
\pgfsetdash{}{0pt}%
\pgfsys@defobject{currentmarker}{\pgfqpoint{-0.048611in}{0.000000in}}{\pgfqpoint{-0.000000in}{0.000000in}}{%
\pgfpathmoveto{\pgfqpoint{-0.000000in}{0.000000in}}%
\pgfpathlineto{\pgfqpoint{-0.048611in}{0.000000in}}%
\pgfusepath{stroke,fill}%
}%
\begin{pgfscope}%
\pgfsys@transformshift{0.800000in}{3.980033in}%
\pgfsys@useobject{currentmarker}{}%
\end{pgfscope}%
\end{pgfscope}%
\begin{pgfscope}%
\definecolor{textcolor}{rgb}{0.000000,0.000000,0.000000}%
\pgfsetstrokecolor{textcolor}%
\pgfsetfillcolor{textcolor}%
\pgftext[x=0.305168in, y=3.927272in, left, base]{\color{textcolor}\sffamily\fontsize{10.000000}{12.000000}\selectfont 0.050}%
\end{pgfscope}%
\begin{pgfscope}%
\definecolor{textcolor}{rgb}{0.000000,0.000000,0.000000}%
\pgfsetstrokecolor{textcolor}%
\pgfsetfillcolor{textcolor}%
\pgftext[x=0.141587in,y=2.376000in,,bottom,rotate=90.000000]{\color{textcolor}\sffamily\fontsize{10.000000}{12.000000}\selectfont Computed error [-]}%
\end{pgfscope}%
\begin{pgfscope}%
\pgfpathrectangle{\pgfqpoint{0.800000in}{0.528000in}}{\pgfqpoint{4.960000in}{3.696000in}}%
\pgfusepath{clip}%
\pgfsetrectcap%
\pgfsetroundjoin%
\pgfsetlinewidth{1.505625pt}%
\definecolor{currentstroke}{rgb}{0.121569,0.466667,0.705882}%
\pgfsetstrokecolor{currentstroke}%
\pgfsetdash{}{0pt}%
\pgfpathmoveto{\pgfqpoint{1.025455in}{0.789225in}}%
\pgfpathlineto{\pgfqpoint{1.079822in}{0.773179in}}%
\pgfpathlineto{\pgfqpoint{1.133922in}{0.897203in}}%
\pgfpathlineto{\pgfqpoint{1.188320in}{1.154205in}}%
\pgfpathlineto{\pgfqpoint{1.242522in}{1.425482in}}%
\pgfpathlineto{\pgfqpoint{1.296781in}{1.725810in}}%
\pgfpathlineto{\pgfqpoint{1.351107in}{2.077768in}}%
\pgfpathlineto{\pgfqpoint{1.406915in}{2.434053in}}%
\pgfpathlineto{\pgfqpoint{1.459887in}{2.707703in}}%
\pgfpathlineto{\pgfqpoint{1.514174in}{2.991495in}}%
\pgfpathlineto{\pgfqpoint{1.568068in}{3.139308in}}%
\pgfpathlineto{\pgfqpoint{1.622011in}{3.144253in}}%
\pgfpathlineto{\pgfqpoint{1.678291in}{3.151517in}}%
\pgfpathlineto{\pgfqpoint{1.730672in}{3.126285in}}%
\pgfpathlineto{\pgfqpoint{1.784529in}{3.122421in}}%
\pgfpathlineto{\pgfqpoint{1.840741in}{3.092961in}}%
\pgfpathlineto{\pgfqpoint{1.893182in}{3.071757in}}%
\pgfpathlineto{\pgfqpoint{1.950346in}{3.019666in}}%
\pgfpathlineto{\pgfqpoint{2.004193in}{3.026064in}}%
\pgfpathlineto{\pgfqpoint{2.057899in}{3.026660in}}%
\pgfpathlineto{\pgfqpoint{2.111638in}{3.063325in}}%
\pgfpathlineto{\pgfqpoint{2.166296in}{3.110315in}}%
\pgfpathlineto{\pgfqpoint{2.221139in}{3.097718in}}%
\pgfpathlineto{\pgfqpoint{2.274923in}{3.093128in}}%
\pgfpathlineto{\pgfqpoint{2.329544in}{3.108022in}}%
\pgfpathlineto{\pgfqpoint{2.383912in}{3.106987in}}%
\pgfpathlineto{\pgfqpoint{2.438011in}{3.119579in}}%
\pgfpathlineto{\pgfqpoint{2.492382in}{3.126722in}}%
\pgfpathlineto{\pgfqpoint{2.546771in}{3.134042in}}%
\pgfpathlineto{\pgfqpoint{2.603285in}{3.128934in}}%
\pgfpathlineto{\pgfqpoint{2.656649in}{3.123678in}}%
\pgfpathlineto{\pgfqpoint{2.710414in}{3.156857in}}%
\pgfpathlineto{\pgfqpoint{2.764551in}{3.149127in}}%
\pgfpathlineto{\pgfqpoint{2.818550in}{3.127573in}}%
\pgfpathlineto{\pgfqpoint{2.873018in}{3.106162in}}%
\pgfpathlineto{\pgfqpoint{2.927117in}{3.094865in}}%
\pgfpathlineto{\pgfqpoint{2.981133in}{3.092295in}}%
\pgfpathlineto{\pgfqpoint{3.037500in}{3.090054in}}%
\pgfpathlineto{\pgfqpoint{3.090526in}{3.088892in}}%
\pgfpathlineto{\pgfqpoint{3.144727in}{3.119401in}}%
\pgfpathlineto{\pgfqpoint{3.199853in}{3.121070in}}%
\pgfpathlineto{\pgfqpoint{3.253313in}{3.132723in}}%
\pgfpathlineto{\pgfqpoint{3.307116in}{3.149728in}}%
\pgfpathlineto{\pgfqpoint{3.361269in}{3.138306in}}%
\pgfpathlineto{\pgfqpoint{3.415362in}{3.136284in}}%
\pgfpathlineto{\pgfqpoint{3.469677in}{3.132143in}}%
\pgfpathlineto{\pgfqpoint{3.523499in}{3.126597in}}%
\pgfpathlineto{\pgfqpoint{3.578629in}{3.117931in}}%
\pgfpathlineto{\pgfqpoint{3.632694in}{3.089359in}}%
\pgfpathlineto{\pgfqpoint{3.686957in}{3.079901in}}%
\pgfpathlineto{\pgfqpoint{3.741822in}{3.075977in}}%
\pgfpathlineto{\pgfqpoint{3.795868in}{3.093736in}}%
\pgfpathlineto{\pgfqpoint{3.851520in}{3.195270in}}%
\pgfpathlineto{\pgfqpoint{3.905190in}{3.199238in}}%
\pgfpathlineto{\pgfqpoint{3.958826in}{3.119164in}}%
\pgfpathlineto{\pgfqpoint{4.013358in}{3.083969in}}%
\pgfpathlineto{\pgfqpoint{4.067443in}{3.078009in}}%
\pgfpathlineto{\pgfqpoint{4.121646in}{3.078892in}}%
\pgfpathlineto{\pgfqpoint{4.175763in}{3.066520in}}%
\pgfpathlineto{\pgfqpoint{4.229865in}{3.053167in}}%
\pgfpathlineto{\pgfqpoint{4.284443in}{3.086279in}}%
\pgfpathlineto{\pgfqpoint{4.338643in}{3.098799in}}%
\pgfpathlineto{\pgfqpoint{4.393069in}{3.084974in}}%
\pgfpathlineto{\pgfqpoint{4.446910in}{3.081038in}}%
\pgfpathlineto{\pgfqpoint{4.502405in}{3.078718in}}%
\pgfpathlineto{\pgfqpoint{4.557185in}{3.083557in}}%
\pgfpathlineto{\pgfqpoint{4.610791in}{3.086688in}}%
\pgfpathlineto{\pgfqpoint{4.665075in}{3.137452in}}%
\pgfpathlineto{\pgfqpoint{4.719799in}{3.127957in}}%
\pgfpathlineto{\pgfqpoint{4.773954in}{3.128653in}}%
\pgfpathlineto{\pgfqpoint{4.828060in}{3.152404in}}%
\pgfpathlineto{\pgfqpoint{4.882094in}{3.149208in}}%
\pgfpathlineto{\pgfqpoint{4.936561in}{3.105130in}}%
\pgfpathlineto{\pgfqpoint{4.990765in}{3.093884in}}%
\pgfpathlineto{\pgfqpoint{5.045060in}{3.097176in}}%
\pgfpathlineto{\pgfqpoint{5.100911in}{3.104016in}}%
\pgfpathlineto{\pgfqpoint{5.154168in}{3.108722in}}%
\pgfpathlineto{\pgfqpoint{5.208005in}{3.104060in}}%
\pgfpathlineto{\pgfqpoint{5.262003in}{3.102692in}}%
\pgfpathlineto{\pgfqpoint{5.316720in}{3.106640in}}%
\pgfpathlineto{\pgfqpoint{5.370968in}{3.110341in}}%
\pgfpathlineto{\pgfqpoint{5.424892in}{3.104028in}}%
\pgfpathlineto{\pgfqpoint{5.479256in}{3.104696in}}%
\pgfpathlineto{\pgfqpoint{5.534545in}{3.101239in}}%
\pgfusepath{stroke}%
\end{pgfscope}%
\begin{pgfscope}%
\pgfpathrectangle{\pgfqpoint{0.800000in}{0.528000in}}{\pgfqpoint{4.960000in}{3.696000in}}%
\pgfusepath{clip}%
\pgfsetrectcap%
\pgfsetroundjoin%
\pgfsetlinewidth{1.505625pt}%
\definecolor{currentstroke}{rgb}{1.000000,0.498039,0.054902}%
\pgfsetstrokecolor{currentstroke}%
\pgfsetdash{}{0pt}%
\pgfpathmoveto{\pgfqpoint{1.025455in}{0.812384in}}%
\pgfpathlineto{\pgfqpoint{1.080191in}{0.835085in}}%
\pgfpathlineto{\pgfqpoint{1.134511in}{0.912521in}}%
\pgfpathlineto{\pgfqpoint{1.189704in}{1.171942in}}%
\pgfpathlineto{\pgfqpoint{1.243700in}{1.426806in}}%
\pgfpathlineto{\pgfqpoint{1.297459in}{1.756368in}}%
\pgfpathlineto{\pgfqpoint{1.351906in}{2.100937in}}%
\pgfpathlineto{\pgfqpoint{1.406392in}{2.417854in}}%
\pgfpathlineto{\pgfqpoint{1.460832in}{2.774634in}}%
\pgfpathlineto{\pgfqpoint{1.515300in}{3.048978in}}%
\pgfpathlineto{\pgfqpoint{1.569187in}{3.319560in}}%
\pgfpathlineto{\pgfqpoint{1.623453in}{3.437739in}}%
\pgfpathlineto{\pgfqpoint{1.677782in}{3.468485in}}%
\pgfpathlineto{\pgfqpoint{1.731965in}{3.501470in}}%
\pgfpathlineto{\pgfqpoint{1.788004in}{3.491321in}}%
\pgfpathlineto{\pgfqpoint{1.841917in}{3.363541in}}%
\pgfpathlineto{\pgfqpoint{1.895653in}{3.289638in}}%
\pgfpathlineto{\pgfqpoint{1.950155in}{3.235733in}}%
\pgfpathlineto{\pgfqpoint{2.004236in}{3.213049in}}%
\pgfpathlineto{\pgfqpoint{2.058626in}{3.213289in}}%
\pgfpathlineto{\pgfqpoint{2.112828in}{3.253838in}}%
\pgfpathlineto{\pgfqpoint{2.167037in}{3.265995in}}%
\pgfpathlineto{\pgfqpoint{2.221146in}{3.275832in}}%
\pgfpathlineto{\pgfqpoint{2.275545in}{3.281059in}}%
\pgfpathlineto{\pgfqpoint{2.329935in}{3.306822in}}%
\pgfpathlineto{\pgfqpoint{2.384150in}{3.303114in}}%
\pgfpathlineto{\pgfqpoint{2.440191in}{3.277925in}}%
\pgfpathlineto{\pgfqpoint{2.493932in}{3.306774in}}%
\pgfpathlineto{\pgfqpoint{2.547595in}{3.322503in}}%
\pgfpathlineto{\pgfqpoint{2.601579in}{3.320245in}}%
\pgfpathlineto{\pgfqpoint{2.655631in}{3.331948in}}%
\pgfpathlineto{\pgfqpoint{2.710036in}{3.295387in}}%
\pgfpathlineto{\pgfqpoint{2.764029in}{3.288040in}}%
\pgfpathlineto{\pgfqpoint{2.818210in}{3.325295in}}%
\pgfpathlineto{\pgfqpoint{2.872700in}{3.340937in}}%
\pgfpathlineto{\pgfqpoint{2.927227in}{3.382746in}}%
\pgfpathlineto{\pgfqpoint{2.981657in}{3.362416in}}%
\pgfpathlineto{\pgfqpoint{3.036126in}{3.356893in}}%
\pgfpathlineto{\pgfqpoint{3.092179in}{3.299235in}}%
\pgfpathlineto{\pgfqpoint{3.145406in}{3.254799in}}%
\pgfpathlineto{\pgfqpoint{3.198920in}{3.234151in}}%
\pgfpathlineto{\pgfqpoint{3.253211in}{3.236073in}}%
\pgfpathlineto{\pgfqpoint{3.307156in}{3.236967in}}%
\pgfpathlineto{\pgfqpoint{3.361262in}{3.186869in}}%
\pgfpathlineto{\pgfqpoint{3.415759in}{3.195053in}}%
\pgfpathlineto{\pgfqpoint{3.470553in}{3.196983in}}%
\pgfpathlineto{\pgfqpoint{3.524682in}{3.210870in}}%
\pgfpathlineto{\pgfqpoint{3.578774in}{3.234500in}}%
\pgfpathlineto{\pgfqpoint{3.634763in}{3.239476in}}%
\pgfpathlineto{\pgfqpoint{3.688090in}{3.237368in}}%
\pgfpathlineto{\pgfqpoint{3.741865in}{3.237138in}}%
\pgfpathlineto{\pgfqpoint{3.796063in}{3.247370in}}%
\pgfpathlineto{\pgfqpoint{3.850120in}{3.251631in}}%
\pgfpathlineto{\pgfqpoint{3.904393in}{3.251521in}}%
\pgfpathlineto{\pgfqpoint{3.958371in}{3.241426in}}%
\pgfpathlineto{\pgfqpoint{4.012426in}{3.239044in}}%
\pgfpathlineto{\pgfqpoint{4.066952in}{3.229738in}}%
\pgfpathlineto{\pgfqpoint{4.121207in}{3.292051in}}%
\pgfpathlineto{\pgfqpoint{4.175715in}{3.239773in}}%
\pgfpathlineto{\pgfqpoint{4.231396in}{3.186431in}}%
\pgfpathlineto{\pgfqpoint{4.284967in}{3.155145in}}%
\pgfpathlineto{\pgfqpoint{4.338924in}{3.165341in}}%
\pgfpathlineto{\pgfqpoint{4.392995in}{3.159253in}}%
\pgfpathlineto{\pgfqpoint{4.446941in}{3.184407in}}%
\pgfpathlineto{\pgfqpoint{4.501277in}{3.139581in}}%
\pgfpathlineto{\pgfqpoint{4.555308in}{3.190626in}}%
\pgfpathlineto{\pgfqpoint{4.609611in}{3.211252in}}%
\pgfpathlineto{\pgfqpoint{4.663692in}{3.221078in}}%
\pgfpathlineto{\pgfqpoint{4.718167in}{3.181996in}}%
\pgfpathlineto{\pgfqpoint{4.772549in}{3.184327in}}%
\pgfpathlineto{\pgfqpoint{4.826956in}{3.185211in}}%
\pgfpathlineto{\pgfqpoint{4.882262in}{3.143995in}}%
\pgfpathlineto{\pgfqpoint{4.936365in}{3.126706in}}%
\pgfpathlineto{\pgfqpoint{4.990053in}{3.137376in}}%
\pgfpathlineto{\pgfqpoint{5.044253in}{3.231030in}}%
\pgfpathlineto{\pgfqpoint{5.098153in}{3.233418in}}%
\pgfpathlineto{\pgfqpoint{5.152260in}{3.226543in}}%
\pgfpathlineto{\pgfqpoint{5.206691in}{3.187131in}}%
\pgfpathlineto{\pgfqpoint{5.261099in}{3.200213in}}%
\pgfpathlineto{\pgfqpoint{5.315382in}{3.187906in}}%
\pgfpathlineto{\pgfqpoint{5.369554in}{3.150297in}}%
\pgfpathlineto{\pgfqpoint{5.424024in}{3.134283in}}%
\pgfpathlineto{\pgfqpoint{5.479725in}{3.133413in}}%
\pgfpathlineto{\pgfqpoint{5.533836in}{3.122661in}}%
\pgfusepath{stroke}%
\end{pgfscope}%
\begin{pgfscope}%
\pgfpathrectangle{\pgfqpoint{0.800000in}{0.528000in}}{\pgfqpoint{4.960000in}{3.696000in}}%
\pgfusepath{clip}%
\pgfsetrectcap%
\pgfsetroundjoin%
\pgfsetlinewidth{1.505625pt}%
\definecolor{currentstroke}{rgb}{0.172549,0.627451,0.172549}%
\pgfsetstrokecolor{currentstroke}%
\pgfsetdash{}{0pt}%
\pgfpathmoveto{\pgfqpoint{1.025455in}{0.813974in}}%
\pgfpathlineto{\pgfqpoint{1.079023in}{0.831207in}}%
\pgfpathlineto{\pgfqpoint{1.133077in}{0.905165in}}%
\pgfpathlineto{\pgfqpoint{1.187729in}{1.191852in}}%
\pgfpathlineto{\pgfqpoint{1.242426in}{1.461286in}}%
\pgfpathlineto{\pgfqpoint{1.296480in}{1.818653in}}%
\pgfpathlineto{\pgfqpoint{1.350844in}{2.124056in}}%
\pgfpathlineto{\pgfqpoint{1.404938in}{2.429353in}}%
\pgfpathlineto{\pgfqpoint{1.459203in}{2.769608in}}%
\pgfpathlineto{\pgfqpoint{1.513440in}{3.161092in}}%
\pgfpathlineto{\pgfqpoint{1.569296in}{3.374738in}}%
\pgfpathlineto{\pgfqpoint{1.622804in}{3.528668in}}%
\pgfpathlineto{\pgfqpoint{1.676188in}{3.693247in}}%
\pgfpathlineto{\pgfqpoint{1.731402in}{3.603703in}}%
\pgfpathlineto{\pgfqpoint{1.784618in}{3.597524in}}%
\pgfpathlineto{\pgfqpoint{1.839526in}{3.532791in}}%
\pgfpathlineto{\pgfqpoint{1.893763in}{3.483487in}}%
\pgfpathlineto{\pgfqpoint{1.948606in}{3.415777in}}%
\pgfpathlineto{\pgfqpoint{2.002550in}{3.342524in}}%
\pgfpathlineto{\pgfqpoint{2.056829in}{3.305360in}}%
\pgfpathlineto{\pgfqpoint{2.111081in}{3.288402in}}%
\pgfpathlineto{\pgfqpoint{2.165552in}{3.278113in}}%
\pgfpathlineto{\pgfqpoint{2.221187in}{3.291749in}}%
\pgfpathlineto{\pgfqpoint{2.274918in}{3.333338in}}%
\pgfpathlineto{\pgfqpoint{2.328304in}{3.342305in}}%
\pgfpathlineto{\pgfqpoint{2.382267in}{3.358908in}}%
\pgfpathlineto{\pgfqpoint{2.436537in}{3.347850in}}%
\pgfpathlineto{\pgfqpoint{2.490878in}{3.346061in}}%
\pgfpathlineto{\pgfqpoint{2.545024in}{3.341034in}}%
\pgfpathlineto{\pgfqpoint{2.599118in}{3.335834in}}%
\pgfpathlineto{\pgfqpoint{2.652939in}{3.320186in}}%
\pgfpathlineto{\pgfqpoint{2.707658in}{3.333300in}}%
\pgfpathlineto{\pgfqpoint{2.761830in}{3.308547in}}%
\pgfpathlineto{\pgfqpoint{2.816110in}{3.313266in}}%
\pgfpathlineto{\pgfqpoint{2.871106in}{3.316503in}}%
\pgfpathlineto{\pgfqpoint{2.924769in}{3.293344in}}%
\pgfpathlineto{\pgfqpoint{2.978646in}{3.254332in}}%
\pgfpathlineto{\pgfqpoint{3.032891in}{3.223371in}}%
\pgfpathlineto{\pgfqpoint{3.087098in}{3.236978in}}%
\pgfpathlineto{\pgfqpoint{3.141336in}{3.240596in}}%
\pgfpathlineto{\pgfqpoint{3.195614in}{3.302415in}}%
\pgfpathlineto{\pgfqpoint{3.250636in}{3.298714in}}%
\pgfpathlineto{\pgfqpoint{3.304750in}{3.251529in}}%
\pgfpathlineto{\pgfqpoint{3.358999in}{3.244629in}}%
\pgfpathlineto{\pgfqpoint{3.413137in}{3.227518in}}%
\pgfpathlineto{\pgfqpoint{3.468994in}{3.210895in}}%
\pgfpathlineto{\pgfqpoint{3.522336in}{3.195020in}}%
\pgfpathlineto{\pgfqpoint{3.576091in}{3.183015in}}%
\pgfpathlineto{\pgfqpoint{3.630296in}{3.177561in}}%
\pgfpathlineto{\pgfqpoint{3.684487in}{3.187044in}}%
\pgfpathlineto{\pgfqpoint{3.738575in}{3.209868in}}%
\pgfpathlineto{\pgfqpoint{3.792677in}{3.227409in}}%
\pgfpathlineto{\pgfqpoint{3.846960in}{3.229877in}}%
\pgfpathlineto{\pgfqpoint{3.900932in}{3.243870in}}%
\pgfpathlineto{\pgfqpoint{3.955316in}{3.197314in}}%
\pgfpathlineto{\pgfqpoint{4.009622in}{3.159129in}}%
\pgfpathlineto{\pgfqpoint{4.065801in}{3.151100in}}%
\pgfpathlineto{\pgfqpoint{4.123315in}{3.188502in}}%
\pgfpathlineto{\pgfqpoint{4.174546in}{3.182781in}}%
\pgfpathlineto{\pgfqpoint{4.228628in}{3.187325in}}%
\pgfpathlineto{\pgfqpoint{4.282595in}{3.178334in}}%
\pgfpathlineto{\pgfqpoint{4.336827in}{3.180254in}}%
\pgfpathlineto{\pgfqpoint{4.390621in}{3.175429in}}%
\pgfpathlineto{\pgfqpoint{4.444394in}{3.148981in}}%
\pgfpathlineto{\pgfqpoint{4.498739in}{3.117405in}}%
\pgfpathlineto{\pgfqpoint{4.553085in}{3.123651in}}%
\pgfpathlineto{\pgfqpoint{4.607184in}{3.211150in}}%
\pgfpathlineto{\pgfqpoint{4.661341in}{3.217152in}}%
\pgfpathlineto{\pgfqpoint{4.715660in}{3.174557in}}%
\pgfpathlineto{\pgfqpoint{4.769928in}{3.164794in}}%
\pgfpathlineto{\pgfqpoint{4.823941in}{3.138271in}}%
\pgfpathlineto{\pgfqpoint{4.878008in}{3.113160in}}%
\pgfpathlineto{\pgfqpoint{4.932322in}{3.113509in}}%
\pgfpathlineto{\pgfqpoint{4.986689in}{3.092253in}}%
\pgfpathlineto{\pgfqpoint{5.040889in}{3.104029in}}%
\pgfpathlineto{\pgfqpoint{5.095123in}{3.157762in}}%
\pgfpathlineto{\pgfqpoint{5.150626in}{3.143179in}}%
\pgfpathlineto{\pgfqpoint{5.204186in}{3.160986in}}%
\pgfpathlineto{\pgfqpoint{5.257755in}{3.158822in}}%
\pgfpathlineto{\pgfqpoint{5.312030in}{3.158898in}}%
\pgfpathlineto{\pgfqpoint{5.366183in}{3.161621in}}%
\pgfpathlineto{\pgfqpoint{5.420436in}{3.157644in}}%
\pgfpathlineto{\pgfqpoint{5.474424in}{3.163621in}}%
\pgfpathlineto{\pgfqpoint{5.529039in}{3.161282in}}%
\pgfusepath{stroke}%
\end{pgfscope}%
\begin{pgfscope}%
\pgfpathrectangle{\pgfqpoint{0.800000in}{0.528000in}}{\pgfqpoint{4.960000in}{3.696000in}}%
\pgfusepath{clip}%
\pgfsetrectcap%
\pgfsetroundjoin%
\pgfsetlinewidth{1.505625pt}%
\definecolor{currentstroke}{rgb}{0.839216,0.152941,0.156863}%
\pgfsetstrokecolor{currentstroke}%
\pgfsetdash{}{0pt}%
\pgfpathmoveto{\pgfqpoint{1.025455in}{0.696000in}}%
\pgfpathlineto{\pgfqpoint{1.079881in}{0.709899in}}%
\pgfpathlineto{\pgfqpoint{1.134963in}{0.968306in}}%
\pgfpathlineto{\pgfqpoint{1.188721in}{1.210576in}}%
\pgfpathlineto{\pgfqpoint{1.243225in}{1.442362in}}%
\pgfpathlineto{\pgfqpoint{1.296939in}{1.783461in}}%
\pgfpathlineto{\pgfqpoint{1.350896in}{2.120056in}}%
\pgfpathlineto{\pgfqpoint{1.405288in}{2.457252in}}%
\pgfpathlineto{\pgfqpoint{1.459395in}{2.772313in}}%
\pgfpathlineto{\pgfqpoint{1.513574in}{3.106080in}}%
\pgfpathlineto{\pgfqpoint{1.567839in}{3.392374in}}%
\pgfpathlineto{\pgfqpoint{1.622095in}{3.724144in}}%
\pgfpathlineto{\pgfqpoint{1.676134in}{3.946204in}}%
\pgfpathlineto{\pgfqpoint{1.730698in}{4.056000in}}%
\pgfpathlineto{\pgfqpoint{1.784436in}{3.978835in}}%
\pgfpathlineto{\pgfqpoint{1.838677in}{3.937026in}}%
\pgfpathlineto{\pgfqpoint{1.892956in}{3.919495in}}%
\pgfpathlineto{\pgfqpoint{1.947070in}{3.723286in}}%
\pgfpathlineto{\pgfqpoint{2.001787in}{3.618345in}}%
\pgfpathlineto{\pgfqpoint{2.055849in}{3.573724in}}%
\pgfpathlineto{\pgfqpoint{2.111433in}{3.446844in}}%
\pgfpathlineto{\pgfqpoint{2.164703in}{3.383479in}}%
\pgfpathlineto{\pgfqpoint{2.218534in}{3.366528in}}%
\pgfpathlineto{\pgfqpoint{2.272583in}{3.383714in}}%
\pgfpathlineto{\pgfqpoint{2.326829in}{3.381392in}}%
\pgfpathlineto{\pgfqpoint{2.381213in}{3.414104in}}%
\pgfpathlineto{\pgfqpoint{2.436582in}{3.452642in}}%
\pgfpathlineto{\pgfqpoint{2.490051in}{3.437132in}}%
\pgfpathlineto{\pgfqpoint{2.543806in}{3.418096in}}%
\pgfpathlineto{\pgfqpoint{2.598027in}{3.387956in}}%
\pgfpathlineto{\pgfqpoint{2.653426in}{3.346475in}}%
\pgfpathlineto{\pgfqpoint{2.709604in}{3.317204in}}%
\pgfpathlineto{\pgfqpoint{2.762579in}{3.273479in}}%
\pgfpathlineto{\pgfqpoint{2.816283in}{3.255762in}}%
\pgfpathlineto{\pgfqpoint{2.870528in}{3.230398in}}%
\pgfpathlineto{\pgfqpoint{2.924432in}{3.229058in}}%
\pgfpathlineto{\pgfqpoint{2.978859in}{3.226487in}}%
\pgfpathlineto{\pgfqpoint{3.032975in}{3.232415in}}%
\pgfpathlineto{\pgfqpoint{3.087194in}{3.239137in}}%
\pgfpathlineto{\pgfqpoint{3.141442in}{3.239705in}}%
\pgfpathlineto{\pgfqpoint{3.195694in}{3.238028in}}%
\pgfpathlineto{\pgfqpoint{3.251039in}{3.221172in}}%
\pgfpathlineto{\pgfqpoint{3.304192in}{3.205720in}}%
\pgfpathlineto{\pgfqpoint{3.360493in}{3.176264in}}%
\pgfpathlineto{\pgfqpoint{3.414658in}{3.095552in}}%
\pgfpathlineto{\pgfqpoint{3.468117in}{3.066705in}}%
\pgfpathlineto{\pgfqpoint{3.522313in}{3.098016in}}%
\pgfpathlineto{\pgfqpoint{3.576393in}{3.113716in}}%
\pgfpathlineto{\pgfqpoint{3.630565in}{3.096119in}}%
\pgfpathlineto{\pgfqpoint{3.685084in}{3.117802in}}%
\pgfpathlineto{\pgfqpoint{3.739274in}{3.086488in}}%
\pgfpathlineto{\pgfqpoint{3.793381in}{3.072213in}}%
\pgfpathlineto{\pgfqpoint{3.848141in}{3.088985in}}%
\pgfpathlineto{\pgfqpoint{3.902571in}{3.152214in}}%
\pgfpathlineto{\pgfqpoint{3.956728in}{3.140760in}}%
\pgfpathlineto{\pgfqpoint{4.012889in}{3.136083in}}%
\pgfpathlineto{\pgfqpoint{4.066488in}{3.124901in}}%
\pgfpathlineto{\pgfqpoint{4.120212in}{3.123539in}}%
\pgfpathlineto{\pgfqpoint{4.174560in}{3.123340in}}%
\pgfpathlineto{\pgfqpoint{4.228498in}{3.122392in}}%
\pgfpathlineto{\pgfqpoint{4.282538in}{3.126796in}}%
\pgfpathlineto{\pgfqpoint{4.337534in}{3.127334in}}%
\pgfpathlineto{\pgfqpoint{4.392155in}{3.129768in}}%
\pgfpathlineto{\pgfqpoint{4.445673in}{3.145582in}}%
\pgfpathlineto{\pgfqpoint{4.500420in}{3.152295in}}%
\pgfpathlineto{\pgfqpoint{4.554532in}{3.154427in}}%
\pgfpathlineto{\pgfqpoint{4.608789in}{3.142308in}}%
\pgfpathlineto{\pgfqpoint{4.664770in}{3.138313in}}%
\pgfpathlineto{\pgfqpoint{4.718654in}{3.132013in}}%
\pgfpathlineto{\pgfqpoint{4.772460in}{3.131809in}}%
\pgfpathlineto{\pgfqpoint{4.826649in}{3.131775in}}%
\pgfpathlineto{\pgfqpoint{4.880609in}{3.127052in}}%
\pgfpathlineto{\pgfqpoint{4.935201in}{3.123196in}}%
\pgfpathlineto{\pgfqpoint{4.989247in}{3.111072in}}%
\pgfpathlineto{\pgfqpoint{5.044090in}{3.103478in}}%
\pgfpathlineto{\pgfqpoint{5.098441in}{3.098608in}}%
\pgfpathlineto{\pgfqpoint{5.152543in}{3.099132in}}%
\pgfpathlineto{\pgfqpoint{5.207030in}{3.101370in}}%
\pgfpathlineto{\pgfqpoint{5.261657in}{3.105271in}}%
\pgfpathlineto{\pgfqpoint{5.316759in}{3.104757in}}%
\pgfpathlineto{\pgfqpoint{5.370349in}{3.104346in}}%
\pgfpathlineto{\pgfqpoint{5.424383in}{3.107809in}}%
\pgfpathlineto{\pgfqpoint{5.478327in}{3.106508in}}%
\pgfpathlineto{\pgfqpoint{5.532561in}{3.112051in}}%
\pgfusepath{stroke}%
\end{pgfscope}%
\begin{pgfscope}%
\pgfpathrectangle{\pgfqpoint{0.800000in}{0.528000in}}{\pgfqpoint{4.960000in}{3.696000in}}%
\pgfusepath{clip}%
\pgfsetrectcap%
\pgfsetroundjoin%
\pgfsetlinewidth{1.505625pt}%
\definecolor{currentstroke}{rgb}{0.580392,0.403922,0.741176}%
\pgfsetstrokecolor{currentstroke}%
\pgfsetdash{}{0pt}%
\pgfpathmoveto{\pgfqpoint{1.025455in}{0.809310in}}%
\pgfpathlineto{\pgfqpoint{1.079685in}{0.805001in}}%
\pgfpathlineto{\pgfqpoint{1.134019in}{0.937634in}}%
\pgfpathlineto{\pgfqpoint{1.188338in}{1.178005in}}%
\pgfpathlineto{\pgfqpoint{1.242589in}{1.480803in}}%
\pgfpathlineto{\pgfqpoint{1.298473in}{1.831972in}}%
\pgfpathlineto{\pgfqpoint{1.351996in}{2.166382in}}%
\pgfpathlineto{\pgfqpoint{1.405662in}{2.429428in}}%
\pgfpathlineto{\pgfqpoint{1.461685in}{2.764669in}}%
\pgfpathlineto{\pgfqpoint{1.514283in}{3.076440in}}%
\pgfpathlineto{\pgfqpoint{1.568549in}{3.499521in}}%
\pgfpathlineto{\pgfqpoint{1.622600in}{3.622917in}}%
\pgfpathlineto{\pgfqpoint{1.677494in}{3.763384in}}%
\pgfpathlineto{\pgfqpoint{1.731276in}{3.803958in}}%
\pgfpathlineto{\pgfqpoint{1.785717in}{3.805220in}}%
\pgfpathlineto{\pgfqpoint{1.840215in}{3.785413in}}%
\pgfpathlineto{\pgfqpoint{1.893984in}{3.665695in}}%
\pgfpathlineto{\pgfqpoint{1.950009in}{3.542984in}}%
\pgfpathlineto{\pgfqpoint{2.003228in}{3.487628in}}%
\pgfpathlineto{\pgfqpoint{2.056911in}{3.380151in}}%
\pgfpathlineto{\pgfqpoint{2.110897in}{3.277759in}}%
\pgfpathlineto{\pgfqpoint{2.165119in}{3.177590in}}%
\pgfpathlineto{\pgfqpoint{2.219214in}{3.076948in}}%
\pgfpathlineto{\pgfqpoint{2.273309in}{3.104137in}}%
\pgfpathlineto{\pgfqpoint{2.327612in}{3.131584in}}%
\pgfpathlineto{\pgfqpoint{2.381810in}{3.235111in}}%
\pgfpathlineto{\pgfqpoint{2.436125in}{3.234003in}}%
\pgfpathlineto{\pgfqpoint{2.490818in}{3.278851in}}%
\pgfpathlineto{\pgfqpoint{2.545023in}{3.241729in}}%
\pgfpathlineto{\pgfqpoint{2.600597in}{3.247534in}}%
\pgfpathlineto{\pgfqpoint{2.653910in}{3.146489in}}%
\pgfpathlineto{\pgfqpoint{2.708551in}{3.120525in}}%
\pgfpathlineto{\pgfqpoint{2.762889in}{3.103387in}}%
\pgfpathlineto{\pgfqpoint{2.817000in}{3.094197in}}%
\pgfpathlineto{\pgfqpoint{2.870990in}{3.093693in}}%
\pgfpathlineto{\pgfqpoint{2.925218in}{3.103055in}}%
\pgfpathlineto{\pgfqpoint{2.979819in}{3.112961in}}%
\pgfpathlineto{\pgfqpoint{3.033790in}{3.118041in}}%
\pgfpathlineto{\pgfqpoint{3.088414in}{3.110520in}}%
\pgfpathlineto{\pgfqpoint{3.143284in}{3.121700in}}%
\pgfpathlineto{\pgfqpoint{3.197383in}{3.119229in}}%
\pgfpathlineto{\pgfqpoint{3.253621in}{3.119078in}}%
\pgfpathlineto{\pgfqpoint{3.307062in}{3.094937in}}%
\pgfpathlineto{\pgfqpoint{3.361312in}{3.075135in}}%
\pgfpathlineto{\pgfqpoint{3.415498in}{3.061233in}}%
\pgfpathlineto{\pgfqpoint{3.469795in}{3.166466in}}%
\pgfpathlineto{\pgfqpoint{3.524031in}{3.171635in}}%
\pgfpathlineto{\pgfqpoint{3.578223in}{3.165985in}}%
\pgfpathlineto{\pgfqpoint{3.632373in}{3.082029in}}%
\pgfpathlineto{\pgfqpoint{3.686646in}{3.069949in}}%
\pgfpathlineto{\pgfqpoint{3.742368in}{3.052714in}}%
\pgfpathlineto{\pgfqpoint{3.796417in}{3.111031in}}%
\pgfpathlineto{\pgfqpoint{3.852565in}{3.089424in}}%
\pgfpathlineto{\pgfqpoint{3.906271in}{3.098488in}}%
\pgfpathlineto{\pgfqpoint{3.959733in}{3.113915in}}%
\pgfpathlineto{\pgfqpoint{4.014905in}{3.122123in}}%
\pgfpathlineto{\pgfqpoint{4.070133in}{3.123109in}}%
\pgfpathlineto{\pgfqpoint{4.122995in}{3.064934in}}%
\pgfpathlineto{\pgfqpoint{4.176361in}{3.043633in}}%
\pgfpathlineto{\pgfqpoint{4.230242in}{3.096203in}}%
\pgfpathlineto{\pgfqpoint{4.285143in}{3.091563in}}%
\pgfpathlineto{\pgfqpoint{4.338904in}{3.126419in}}%
\pgfpathlineto{\pgfqpoint{4.393070in}{3.117686in}}%
\pgfpathlineto{\pgfqpoint{4.447851in}{3.116733in}}%
\pgfpathlineto{\pgfqpoint{4.501903in}{3.118392in}}%
\pgfpathlineto{\pgfqpoint{4.555981in}{3.159828in}}%
\pgfpathlineto{\pgfqpoint{4.609946in}{3.124507in}}%
\pgfpathlineto{\pgfqpoint{4.665385in}{3.117500in}}%
\pgfpathlineto{\pgfqpoint{4.719020in}{3.144528in}}%
\pgfpathlineto{\pgfqpoint{4.772972in}{3.106183in}}%
\pgfpathlineto{\pgfqpoint{4.827057in}{3.051013in}}%
\pgfpathlineto{\pgfqpoint{4.881177in}{3.023645in}}%
\pgfpathlineto{\pgfqpoint{4.935645in}{3.024182in}}%
\pgfpathlineto{\pgfqpoint{4.989576in}{3.044705in}}%
\pgfpathlineto{\pgfqpoint{5.043959in}{3.046713in}}%
\pgfpathlineto{\pgfqpoint{5.098584in}{3.066930in}}%
\pgfpathlineto{\pgfqpoint{5.153157in}{3.156725in}}%
\pgfpathlineto{\pgfqpoint{5.207385in}{3.176175in}}%
\pgfpathlineto{\pgfqpoint{5.262338in}{3.175251in}}%
\pgfpathlineto{\pgfqpoint{5.316712in}{3.174529in}}%
\pgfpathlineto{\pgfqpoint{5.370845in}{3.172209in}}%
\pgfpathlineto{\pgfqpoint{5.424692in}{3.198925in}}%
\pgfpathlineto{\pgfqpoint{5.478861in}{3.167180in}}%
\pgfpathlineto{\pgfqpoint{5.533803in}{3.117715in}}%
\pgfusepath{stroke}%
\end{pgfscope}%
\begin{pgfscope}%
\pgfpathrectangle{\pgfqpoint{0.800000in}{0.528000in}}{\pgfqpoint{4.960000in}{3.696000in}}%
\pgfusepath{clip}%
\pgfsetrectcap%
\pgfsetroundjoin%
\pgfsetlinewidth{1.505625pt}%
\definecolor{currentstroke}{rgb}{0.549020,0.337255,0.294118}%
\pgfsetstrokecolor{currentstroke}%
\pgfsetdash{}{0pt}%
\pgfpathmoveto{\pgfqpoint{1.025455in}{0.769797in}}%
\pgfpathlineto{\pgfqpoint{1.080003in}{0.807320in}}%
\pgfpathlineto{\pgfqpoint{1.134006in}{0.973446in}}%
\pgfpathlineto{\pgfqpoint{1.188852in}{1.152385in}}%
\pgfpathlineto{\pgfqpoint{1.242782in}{1.433564in}}%
\pgfpathlineto{\pgfqpoint{1.296875in}{1.775068in}}%
\pgfpathlineto{\pgfqpoint{1.351200in}{2.080304in}}%
\pgfpathlineto{\pgfqpoint{1.405455in}{2.410894in}}%
\pgfpathlineto{\pgfqpoint{1.461112in}{2.701906in}}%
\pgfpathlineto{\pgfqpoint{1.514836in}{3.015133in}}%
\pgfpathlineto{\pgfqpoint{1.568352in}{3.336847in}}%
\pgfpathlineto{\pgfqpoint{1.622757in}{3.557179in}}%
\pgfpathlineto{\pgfqpoint{1.677115in}{3.704555in}}%
\pgfpathlineto{\pgfqpoint{1.731333in}{3.797671in}}%
\pgfpathlineto{\pgfqpoint{1.785498in}{3.725540in}}%
\pgfpathlineto{\pgfqpoint{1.839801in}{3.542152in}}%
\pgfpathlineto{\pgfqpoint{1.893801in}{3.395576in}}%
\pgfpathlineto{\pgfqpoint{1.948199in}{3.315710in}}%
\pgfpathlineto{\pgfqpoint{2.002289in}{3.120495in}}%
\pgfpathlineto{\pgfqpoint{2.056675in}{3.017754in}}%
\pgfpathlineto{\pgfqpoint{2.112439in}{2.957118in}}%
\pgfpathlineto{\pgfqpoint{2.166352in}{2.937952in}}%
\pgfpathlineto{\pgfqpoint{2.220134in}{2.958377in}}%
\pgfpathlineto{\pgfqpoint{2.274763in}{3.049754in}}%
\pgfpathlineto{\pgfqpoint{2.328718in}{3.121430in}}%
\pgfpathlineto{\pgfqpoint{2.384312in}{3.138100in}}%
\pgfpathlineto{\pgfqpoint{2.436920in}{3.137114in}}%
\pgfpathlineto{\pgfqpoint{2.491667in}{3.124604in}}%
\pgfpathlineto{\pgfqpoint{2.546354in}{3.113287in}}%
\pgfpathlineto{\pgfqpoint{2.600473in}{3.180304in}}%
\pgfpathlineto{\pgfqpoint{2.654786in}{3.159582in}}%
\pgfpathlineto{\pgfqpoint{2.710870in}{3.185530in}}%
\pgfpathlineto{\pgfqpoint{2.764412in}{3.099506in}}%
\pgfpathlineto{\pgfqpoint{2.818709in}{3.108295in}}%
\pgfpathlineto{\pgfqpoint{2.872258in}{3.112185in}}%
\pgfpathlineto{\pgfqpoint{2.926385in}{3.060689in}}%
\pgfpathlineto{\pgfqpoint{2.980763in}{3.028215in}}%
\pgfpathlineto{\pgfqpoint{3.035157in}{3.028656in}}%
\pgfpathlineto{\pgfqpoint{3.089043in}{3.033164in}}%
\pgfpathlineto{\pgfqpoint{3.143528in}{3.046057in}}%
\pgfpathlineto{\pgfqpoint{3.197726in}{3.044231in}}%
\pgfpathlineto{\pgfqpoint{3.251783in}{3.085743in}}%
\pgfpathlineto{\pgfqpoint{3.307894in}{3.113126in}}%
\pgfpathlineto{\pgfqpoint{3.361482in}{3.166125in}}%
\pgfpathlineto{\pgfqpoint{3.414922in}{3.196116in}}%
\pgfpathlineto{\pgfqpoint{3.468943in}{3.195617in}}%
\pgfpathlineto{\pgfqpoint{3.524326in}{3.186766in}}%
\pgfpathlineto{\pgfqpoint{3.578969in}{3.251703in}}%
\pgfpathlineto{\pgfqpoint{3.632337in}{3.229395in}}%
\pgfpathlineto{\pgfqpoint{3.686528in}{3.195767in}}%
\pgfpathlineto{\pgfqpoint{3.740532in}{3.087878in}}%
\pgfpathlineto{\pgfqpoint{3.795418in}{3.000855in}}%
\pgfpathlineto{\pgfqpoint{3.849306in}{2.974233in}}%
\pgfpathlineto{\pgfqpoint{3.903592in}{2.977366in}}%
\pgfpathlineto{\pgfqpoint{3.959568in}{3.025234in}}%
\pgfpathlineto{\pgfqpoint{4.013316in}{3.029085in}}%
\pgfpathlineto{\pgfqpoint{4.066662in}{3.039833in}}%
\pgfpathlineto{\pgfqpoint{4.120736in}{3.073986in}}%
\pgfpathlineto{\pgfqpoint{4.174802in}{3.275471in}}%
\pgfpathlineto{\pgfqpoint{4.228802in}{3.307625in}}%
\pgfpathlineto{\pgfqpoint{4.283279in}{3.286793in}}%
\pgfpathlineto{\pgfqpoint{4.339558in}{3.221594in}}%
\pgfpathlineto{\pgfqpoint{4.391161in}{3.147030in}}%
\pgfpathlineto{\pgfqpoint{4.445732in}{3.087790in}}%
\pgfpathlineto{\pgfqpoint{4.500189in}{3.049566in}}%
\pgfpathlineto{\pgfqpoint{4.554679in}{3.038984in}}%
\pgfpathlineto{\pgfqpoint{4.610520in}{3.025197in}}%
\pgfpathlineto{\pgfqpoint{4.664649in}{3.097516in}}%
\pgfpathlineto{\pgfqpoint{4.718277in}{3.112280in}}%
\pgfpathlineto{\pgfqpoint{4.772801in}{3.111049in}}%
\pgfpathlineto{\pgfqpoint{4.827168in}{3.072334in}}%
\pgfpathlineto{\pgfqpoint{4.881548in}{3.044853in}}%
\pgfpathlineto{\pgfqpoint{4.935527in}{3.046634in}}%
\pgfpathlineto{\pgfqpoint{4.989571in}{3.051009in}}%
\pgfpathlineto{\pgfqpoint{5.046423in}{3.107366in}}%
\pgfpathlineto{\pgfqpoint{5.099897in}{3.134016in}}%
\pgfpathlineto{\pgfqpoint{5.153971in}{3.158116in}}%
\pgfpathlineto{\pgfqpoint{5.209470in}{3.200900in}}%
\pgfpathlineto{\pgfqpoint{5.264467in}{3.249549in}}%
\pgfpathlineto{\pgfqpoint{5.320550in}{3.231969in}}%
\pgfpathlineto{\pgfqpoint{5.371851in}{3.175139in}}%
\pgfpathlineto{\pgfqpoint{5.426111in}{3.092320in}}%
\pgfpathlineto{\pgfqpoint{5.480046in}{3.065577in}}%
\pgfpathlineto{\pgfqpoint{5.534454in}{3.052447in}}%
\pgfusepath{stroke}%
\end{pgfscope}%
\begin{pgfscope}%
\pgfsetrectcap%
\pgfsetmiterjoin%
\pgfsetlinewidth{0.803000pt}%
\definecolor{currentstroke}{rgb}{0.000000,0.000000,0.000000}%
\pgfsetstrokecolor{currentstroke}%
\pgfsetdash{}{0pt}%
\pgfpathmoveto{\pgfqpoint{0.800000in}{0.528000in}}%
\pgfpathlineto{\pgfqpoint{0.800000in}{4.224000in}}%
\pgfusepath{stroke}%
\end{pgfscope}%
\begin{pgfscope}%
\pgfsetrectcap%
\pgfsetmiterjoin%
\pgfsetlinewidth{0.803000pt}%
\definecolor{currentstroke}{rgb}{0.000000,0.000000,0.000000}%
\pgfsetstrokecolor{currentstroke}%
\pgfsetdash{}{0pt}%
\pgfpathmoveto{\pgfqpoint{5.760000in}{0.528000in}}%
\pgfpathlineto{\pgfqpoint{5.760000in}{4.224000in}}%
\pgfusepath{stroke}%
\end{pgfscope}%
\begin{pgfscope}%
\pgfsetrectcap%
\pgfsetmiterjoin%
\pgfsetlinewidth{0.803000pt}%
\definecolor{currentstroke}{rgb}{0.000000,0.000000,0.000000}%
\pgfsetstrokecolor{currentstroke}%
\pgfsetdash{}{0pt}%
\pgfpathmoveto{\pgfqpoint{0.800000in}{0.528000in}}%
\pgfpathlineto{\pgfqpoint{5.760000in}{0.528000in}}%
\pgfusepath{stroke}%
\end{pgfscope}%
\begin{pgfscope}%
\pgfsetrectcap%
\pgfsetmiterjoin%
\pgfsetlinewidth{0.803000pt}%
\definecolor{currentstroke}{rgb}{0.000000,0.000000,0.000000}%
\pgfsetstrokecolor{currentstroke}%
\pgfsetdash{}{0pt}%
\pgfpathmoveto{\pgfqpoint{0.800000in}{4.224000in}}%
\pgfpathlineto{\pgfqpoint{5.760000in}{4.224000in}}%
\pgfusepath{stroke}%
\end{pgfscope}%
\begin{pgfscope}%
\definecolor{textcolor}{rgb}{0.000000,0.000000,0.000000}%
\pgfsetstrokecolor{textcolor}%
\pgfsetfillcolor{textcolor}%
\pgftext[x=3.280000in,y=4.307333in,,base]{\color{textcolor}\sffamily\fontsize{12.000000}{14.400000}\selectfont Yaw controller input}%
\end{pgfscope}%
\begin{pgfscope}%
\pgfsetbuttcap%
\pgfsetmiterjoin%
\definecolor{currentfill}{rgb}{1.000000,1.000000,1.000000}%
\pgfsetfillcolor{currentfill}%
\pgfsetfillopacity{0.800000}%
\pgfsetlinewidth{1.003750pt}%
\definecolor{currentstroke}{rgb}{0.800000,0.800000,0.800000}%
\pgfsetstrokecolor{currentstroke}%
\pgfsetstrokeopacity{0.800000}%
\pgfsetdash{}{0pt}%
\pgfpathmoveto{\pgfqpoint{5.041603in}{0.597444in}}%
\pgfpathlineto{\pgfqpoint{5.662778in}{0.597444in}}%
\pgfpathquadraticcurveto{\pgfqpoint{5.690556in}{0.597444in}}{\pgfqpoint{5.690556in}{0.625222in}}%
\pgfpathlineto{\pgfqpoint{5.690556in}{1.834477in}}%
\pgfpathquadraticcurveto{\pgfqpoint{5.690556in}{1.862254in}}{\pgfqpoint{5.662778in}{1.862254in}}%
\pgfpathlineto{\pgfqpoint{5.041603in}{1.862254in}}%
\pgfpathquadraticcurveto{\pgfqpoint{5.013825in}{1.862254in}}{\pgfqpoint{5.013825in}{1.834477in}}%
\pgfpathlineto{\pgfqpoint{5.013825in}{0.625222in}}%
\pgfpathquadraticcurveto{\pgfqpoint{5.013825in}{0.597444in}}{\pgfqpoint{5.041603in}{0.597444in}}%
\pgfpathlineto{\pgfqpoint{5.041603in}{0.597444in}}%
\pgfpathclose%
\pgfusepath{stroke,fill}%
\end{pgfscope}%
\begin{pgfscope}%
\pgfsetrectcap%
\pgfsetroundjoin%
\pgfsetlinewidth{1.505625pt}%
\definecolor{currentstroke}{rgb}{0.121569,0.466667,0.705882}%
\pgfsetstrokecolor{currentstroke}%
\pgfsetdash{}{0pt}%
\pgfpathmoveto{\pgfqpoint{5.069380in}{1.749787in}}%
\pgfpathlineto{\pgfqpoint{5.208269in}{1.749787in}}%
\pgfpathlineto{\pgfqpoint{5.347158in}{1.749787in}}%
\pgfusepath{stroke}%
\end{pgfscope}%
\begin{pgfscope}%
\definecolor{textcolor}{rgb}{0.000000,0.000000,0.000000}%
\pgfsetstrokecolor{textcolor}%
\pgfsetfillcolor{textcolor}%
\pgftext[x=5.458269in,y=1.701176in,left,base]{\color{textcolor}\sffamily\fontsize{10.000000}{12.000000}\selectfont 0}%
\end{pgfscope}%
\begin{pgfscope}%
\pgfsetrectcap%
\pgfsetroundjoin%
\pgfsetlinewidth{1.505625pt}%
\definecolor{currentstroke}{rgb}{1.000000,0.498039,0.054902}%
\pgfsetstrokecolor{currentstroke}%
\pgfsetdash{}{0pt}%
\pgfpathmoveto{\pgfqpoint{5.069380in}{1.545930in}}%
\pgfpathlineto{\pgfqpoint{5.208269in}{1.545930in}}%
\pgfpathlineto{\pgfqpoint{5.347158in}{1.545930in}}%
\pgfusepath{stroke}%
\end{pgfscope}%
\begin{pgfscope}%
\definecolor{textcolor}{rgb}{0.000000,0.000000,0.000000}%
\pgfsetstrokecolor{textcolor}%
\pgfsetfillcolor{textcolor}%
\pgftext[x=5.458269in,y=1.497319in,left,base]{\color{textcolor}\sffamily\fontsize{10.000000}{12.000000}\selectfont 5}%
\end{pgfscope}%
\begin{pgfscope}%
\pgfsetrectcap%
\pgfsetroundjoin%
\pgfsetlinewidth{1.505625pt}%
\definecolor{currentstroke}{rgb}{0.172549,0.627451,0.172549}%
\pgfsetstrokecolor{currentstroke}%
\pgfsetdash{}{0pt}%
\pgfpathmoveto{\pgfqpoint{5.069380in}{1.342073in}}%
\pgfpathlineto{\pgfqpoint{5.208269in}{1.342073in}}%
\pgfpathlineto{\pgfqpoint{5.347158in}{1.342073in}}%
\pgfusepath{stroke}%
\end{pgfscope}%
\begin{pgfscope}%
\definecolor{textcolor}{rgb}{0.000000,0.000000,0.000000}%
\pgfsetstrokecolor{textcolor}%
\pgfsetfillcolor{textcolor}%
\pgftext[x=5.458269in,y=1.293461in,left,base]{\color{textcolor}\sffamily\fontsize{10.000000}{12.000000}\selectfont 10}%
\end{pgfscope}%
\begin{pgfscope}%
\pgfsetrectcap%
\pgfsetroundjoin%
\pgfsetlinewidth{1.505625pt}%
\definecolor{currentstroke}{rgb}{0.839216,0.152941,0.156863}%
\pgfsetstrokecolor{currentstroke}%
\pgfsetdash{}{0pt}%
\pgfpathmoveto{\pgfqpoint{5.069380in}{1.138215in}}%
\pgfpathlineto{\pgfqpoint{5.208269in}{1.138215in}}%
\pgfpathlineto{\pgfqpoint{5.347158in}{1.138215in}}%
\pgfusepath{stroke}%
\end{pgfscope}%
\begin{pgfscope}%
\definecolor{textcolor}{rgb}{0.000000,0.000000,0.000000}%
\pgfsetstrokecolor{textcolor}%
\pgfsetfillcolor{textcolor}%
\pgftext[x=5.458269in,y=1.089604in,left,base]{\color{textcolor}\sffamily\fontsize{10.000000}{12.000000}\selectfont 20}%
\end{pgfscope}%
\begin{pgfscope}%
\pgfsetrectcap%
\pgfsetroundjoin%
\pgfsetlinewidth{1.505625pt}%
\definecolor{currentstroke}{rgb}{0.580392,0.403922,0.741176}%
\pgfsetstrokecolor{currentstroke}%
\pgfsetdash{}{0pt}%
\pgfpathmoveto{\pgfqpoint{5.069380in}{0.934358in}}%
\pgfpathlineto{\pgfqpoint{5.208269in}{0.934358in}}%
\pgfpathlineto{\pgfqpoint{5.347158in}{0.934358in}}%
\pgfusepath{stroke}%
\end{pgfscope}%
\begin{pgfscope}%
\definecolor{textcolor}{rgb}{0.000000,0.000000,0.000000}%
\pgfsetstrokecolor{textcolor}%
\pgfsetfillcolor{textcolor}%
\pgftext[x=5.458269in,y=0.885747in,left,base]{\color{textcolor}\sffamily\fontsize{10.000000}{12.000000}\selectfont 40}%
\end{pgfscope}%
\begin{pgfscope}%
\pgfsetrectcap%
\pgfsetroundjoin%
\pgfsetlinewidth{1.505625pt}%
\definecolor{currentstroke}{rgb}{0.549020,0.337255,0.294118}%
\pgfsetstrokecolor{currentstroke}%
\pgfsetdash{}{0pt}%
\pgfpathmoveto{\pgfqpoint{5.069380in}{0.730501in}}%
\pgfpathlineto{\pgfqpoint{5.208269in}{0.730501in}}%
\pgfpathlineto{\pgfqpoint{5.347158in}{0.730501in}}%
\pgfusepath{stroke}%
\end{pgfscope}%
\begin{pgfscope}%
\definecolor{textcolor}{rgb}{0.000000,0.000000,0.000000}%
\pgfsetstrokecolor{textcolor}%
\pgfsetfillcolor{textcolor}%
\pgftext[x=5.458269in,y=0.681890in,left,base]{\color{textcolor}\sffamily\fontsize{10.000000}{12.000000}\selectfont 80}%
\end{pgfscope}%
\end{pgfpicture}%
\makeatother%
\endgroup%
}
    \end{minipage}
    \begin{minipage}[t]{0.5\linewidth}
        \centering
        \scalebox{0.55}{%% Creator: Matplotlib, PGF backend
%%
%% To include the figure in your LaTeX document, write
%%   \input{<filename>.pgf}
%%
%% Make sure the required packages are loaded in your preamble
%%   \usepackage{pgf}
%%
%% Also ensure that all the required font packages are loaded; for instance,
%% the lmodern package is sometimes necessary when using math font.
%%   \usepackage{lmodern}
%%
%% Figures using additional raster images can only be included by \input if
%% they are in the same directory as the main LaTeX file. For loading figures
%% from other directories you can use the `import` package
%%   \usepackage{import}
%%
%% and then include the figures with
%%   \import{<path to file>}{<filename>.pgf}
%%
%% Matplotlib used the following preamble
%%   \usepackage{fontspec}
%%   \setmainfont{DejaVuSerif.ttf}[Path=\detokenize{/home/lgonz/tfg-aero/tfg-giaa-dronecontrol/venv/lib/python3.8/site-packages/matplotlib/mpl-data/fonts/ttf/}]
%%   \setsansfont{DejaVuSans.ttf}[Path=\detokenize{/home/lgonz/tfg-aero/tfg-giaa-dronecontrol/venv/lib/python3.8/site-packages/matplotlib/mpl-data/fonts/ttf/}]
%%   \setmonofont{DejaVuSansMono.ttf}[Path=\detokenize{/home/lgonz/tfg-aero/tfg-giaa-dronecontrol/venv/lib/python3.8/site-packages/matplotlib/mpl-data/fonts/ttf/}]
%%
\begingroup%
\makeatletter%
\begin{pgfpicture}%
\pgfpathrectangle{\pgfpointorigin}{\pgfqpoint{6.400000in}{4.800000in}}%
\pgfusepath{use as bounding box, clip}%
\begin{pgfscope}%
\pgfsetbuttcap%
\pgfsetmiterjoin%
\definecolor{currentfill}{rgb}{1.000000,1.000000,1.000000}%
\pgfsetfillcolor{currentfill}%
\pgfsetlinewidth{0.000000pt}%
\definecolor{currentstroke}{rgb}{1.000000,1.000000,1.000000}%
\pgfsetstrokecolor{currentstroke}%
\pgfsetdash{}{0pt}%
\pgfpathmoveto{\pgfqpoint{0.000000in}{0.000000in}}%
\pgfpathlineto{\pgfqpoint{6.400000in}{0.000000in}}%
\pgfpathlineto{\pgfqpoint{6.400000in}{4.800000in}}%
\pgfpathlineto{\pgfqpoint{0.000000in}{4.800000in}}%
\pgfpathlineto{\pgfqpoint{0.000000in}{0.000000in}}%
\pgfpathclose%
\pgfusepath{fill}%
\end{pgfscope}%
\begin{pgfscope}%
\pgfsetbuttcap%
\pgfsetmiterjoin%
\definecolor{currentfill}{rgb}{1.000000,1.000000,1.000000}%
\pgfsetfillcolor{currentfill}%
\pgfsetlinewidth{0.000000pt}%
\definecolor{currentstroke}{rgb}{0.000000,0.000000,0.000000}%
\pgfsetstrokecolor{currentstroke}%
\pgfsetstrokeopacity{0.000000}%
\pgfsetdash{}{0pt}%
\pgfpathmoveto{\pgfqpoint{0.800000in}{0.528000in}}%
\pgfpathlineto{\pgfqpoint{5.760000in}{0.528000in}}%
\pgfpathlineto{\pgfqpoint{5.760000in}{4.224000in}}%
\pgfpathlineto{\pgfqpoint{0.800000in}{4.224000in}}%
\pgfpathlineto{\pgfqpoint{0.800000in}{0.528000in}}%
\pgfpathclose%
\pgfusepath{fill}%
\end{pgfscope}%
\begin{pgfscope}%
\pgfpathrectangle{\pgfqpoint{0.800000in}{0.528000in}}{\pgfqpoint{4.960000in}{3.696000in}}%
\pgfusepath{clip}%
\pgfsetrectcap%
\pgfsetroundjoin%
\pgfsetlinewidth{0.803000pt}%
\definecolor{currentstroke}{rgb}{0.690196,0.690196,0.690196}%
\pgfsetstrokecolor{currentstroke}%
\pgfsetdash{}{0pt}%
\pgfpathmoveto{\pgfqpoint{1.025455in}{0.528000in}}%
\pgfpathlineto{\pgfqpoint{1.025455in}{4.224000in}}%
\pgfusepath{stroke}%
\end{pgfscope}%
\begin{pgfscope}%
\pgfsetbuttcap%
\pgfsetroundjoin%
\definecolor{currentfill}{rgb}{0.000000,0.000000,0.000000}%
\pgfsetfillcolor{currentfill}%
\pgfsetlinewidth{0.803000pt}%
\definecolor{currentstroke}{rgb}{0.000000,0.000000,0.000000}%
\pgfsetstrokecolor{currentstroke}%
\pgfsetdash{}{0pt}%
\pgfsys@defobject{currentmarker}{\pgfqpoint{0.000000in}{-0.048611in}}{\pgfqpoint{0.000000in}{0.000000in}}{%
\pgfpathmoveto{\pgfqpoint{0.000000in}{0.000000in}}%
\pgfpathlineto{\pgfqpoint{0.000000in}{-0.048611in}}%
\pgfusepath{stroke,fill}%
}%
\begin{pgfscope}%
\pgfsys@transformshift{1.025455in}{0.528000in}%
\pgfsys@useobject{currentmarker}{}%
\end{pgfscope}%
\end{pgfscope}%
\begin{pgfscope}%
\definecolor{textcolor}{rgb}{0.000000,0.000000,0.000000}%
\pgfsetstrokecolor{textcolor}%
\pgfsetfillcolor{textcolor}%
\pgftext[x=1.025455in,y=0.430778in,,top]{\color{textcolor}\sffamily\fontsize{10.000000}{12.000000}\selectfont 0}%
\end{pgfscope}%
\begin{pgfscope}%
\pgfpathrectangle{\pgfqpoint{0.800000in}{0.528000in}}{\pgfqpoint{4.960000in}{3.696000in}}%
\pgfusepath{clip}%
\pgfsetrectcap%
\pgfsetroundjoin%
\pgfsetlinewidth{0.803000pt}%
\definecolor{currentstroke}{rgb}{0.690196,0.690196,0.690196}%
\pgfsetstrokecolor{currentstroke}%
\pgfsetdash{}{0pt}%
\pgfpathmoveto{\pgfqpoint{1.775826in}{0.528000in}}%
\pgfpathlineto{\pgfqpoint{1.775826in}{4.224000in}}%
\pgfusepath{stroke}%
\end{pgfscope}%
\begin{pgfscope}%
\pgfsetbuttcap%
\pgfsetroundjoin%
\definecolor{currentfill}{rgb}{0.000000,0.000000,0.000000}%
\pgfsetfillcolor{currentfill}%
\pgfsetlinewidth{0.803000pt}%
\definecolor{currentstroke}{rgb}{0.000000,0.000000,0.000000}%
\pgfsetstrokecolor{currentstroke}%
\pgfsetdash{}{0pt}%
\pgfsys@defobject{currentmarker}{\pgfqpoint{0.000000in}{-0.048611in}}{\pgfqpoint{0.000000in}{0.000000in}}{%
\pgfpathmoveto{\pgfqpoint{0.000000in}{0.000000in}}%
\pgfpathlineto{\pgfqpoint{0.000000in}{-0.048611in}}%
\pgfusepath{stroke,fill}%
}%
\begin{pgfscope}%
\pgfsys@transformshift{1.775826in}{0.528000in}%
\pgfsys@useobject{currentmarker}{}%
\end{pgfscope}%
\end{pgfscope}%
\begin{pgfscope}%
\definecolor{textcolor}{rgb}{0.000000,0.000000,0.000000}%
\pgfsetstrokecolor{textcolor}%
\pgfsetfillcolor{textcolor}%
\pgftext[x=1.775826in,y=0.430778in,,top]{\color{textcolor}\sffamily\fontsize{10.000000}{12.000000}\selectfont 5}%
\end{pgfscope}%
\begin{pgfscope}%
\pgfpathrectangle{\pgfqpoint{0.800000in}{0.528000in}}{\pgfqpoint{4.960000in}{3.696000in}}%
\pgfusepath{clip}%
\pgfsetrectcap%
\pgfsetroundjoin%
\pgfsetlinewidth{0.803000pt}%
\definecolor{currentstroke}{rgb}{0.690196,0.690196,0.690196}%
\pgfsetstrokecolor{currentstroke}%
\pgfsetdash{}{0pt}%
\pgfpathmoveto{\pgfqpoint{2.526198in}{0.528000in}}%
\pgfpathlineto{\pgfqpoint{2.526198in}{4.224000in}}%
\pgfusepath{stroke}%
\end{pgfscope}%
\begin{pgfscope}%
\pgfsetbuttcap%
\pgfsetroundjoin%
\definecolor{currentfill}{rgb}{0.000000,0.000000,0.000000}%
\pgfsetfillcolor{currentfill}%
\pgfsetlinewidth{0.803000pt}%
\definecolor{currentstroke}{rgb}{0.000000,0.000000,0.000000}%
\pgfsetstrokecolor{currentstroke}%
\pgfsetdash{}{0pt}%
\pgfsys@defobject{currentmarker}{\pgfqpoint{0.000000in}{-0.048611in}}{\pgfqpoint{0.000000in}{0.000000in}}{%
\pgfpathmoveto{\pgfqpoint{0.000000in}{0.000000in}}%
\pgfpathlineto{\pgfqpoint{0.000000in}{-0.048611in}}%
\pgfusepath{stroke,fill}%
}%
\begin{pgfscope}%
\pgfsys@transformshift{2.526198in}{0.528000in}%
\pgfsys@useobject{currentmarker}{}%
\end{pgfscope}%
\end{pgfscope}%
\begin{pgfscope}%
\definecolor{textcolor}{rgb}{0.000000,0.000000,0.000000}%
\pgfsetstrokecolor{textcolor}%
\pgfsetfillcolor{textcolor}%
\pgftext[x=2.526198in,y=0.430778in,,top]{\color{textcolor}\sffamily\fontsize{10.000000}{12.000000}\selectfont 10}%
\end{pgfscope}%
\begin{pgfscope}%
\pgfpathrectangle{\pgfqpoint{0.800000in}{0.528000in}}{\pgfqpoint{4.960000in}{3.696000in}}%
\pgfusepath{clip}%
\pgfsetrectcap%
\pgfsetroundjoin%
\pgfsetlinewidth{0.803000pt}%
\definecolor{currentstroke}{rgb}{0.690196,0.690196,0.690196}%
\pgfsetstrokecolor{currentstroke}%
\pgfsetdash{}{0pt}%
\pgfpathmoveto{\pgfqpoint{3.276569in}{0.528000in}}%
\pgfpathlineto{\pgfqpoint{3.276569in}{4.224000in}}%
\pgfusepath{stroke}%
\end{pgfscope}%
\begin{pgfscope}%
\pgfsetbuttcap%
\pgfsetroundjoin%
\definecolor{currentfill}{rgb}{0.000000,0.000000,0.000000}%
\pgfsetfillcolor{currentfill}%
\pgfsetlinewidth{0.803000pt}%
\definecolor{currentstroke}{rgb}{0.000000,0.000000,0.000000}%
\pgfsetstrokecolor{currentstroke}%
\pgfsetdash{}{0pt}%
\pgfsys@defobject{currentmarker}{\pgfqpoint{0.000000in}{-0.048611in}}{\pgfqpoint{0.000000in}{0.000000in}}{%
\pgfpathmoveto{\pgfqpoint{0.000000in}{0.000000in}}%
\pgfpathlineto{\pgfqpoint{0.000000in}{-0.048611in}}%
\pgfusepath{stroke,fill}%
}%
\begin{pgfscope}%
\pgfsys@transformshift{3.276569in}{0.528000in}%
\pgfsys@useobject{currentmarker}{}%
\end{pgfscope}%
\end{pgfscope}%
\begin{pgfscope}%
\definecolor{textcolor}{rgb}{0.000000,0.000000,0.000000}%
\pgfsetstrokecolor{textcolor}%
\pgfsetfillcolor{textcolor}%
\pgftext[x=3.276569in,y=0.430778in,,top]{\color{textcolor}\sffamily\fontsize{10.000000}{12.000000}\selectfont 15}%
\end{pgfscope}%
\begin{pgfscope}%
\pgfpathrectangle{\pgfqpoint{0.800000in}{0.528000in}}{\pgfqpoint{4.960000in}{3.696000in}}%
\pgfusepath{clip}%
\pgfsetrectcap%
\pgfsetroundjoin%
\pgfsetlinewidth{0.803000pt}%
\definecolor{currentstroke}{rgb}{0.690196,0.690196,0.690196}%
\pgfsetstrokecolor{currentstroke}%
\pgfsetdash{}{0pt}%
\pgfpathmoveto{\pgfqpoint{4.026940in}{0.528000in}}%
\pgfpathlineto{\pgfqpoint{4.026940in}{4.224000in}}%
\pgfusepath{stroke}%
\end{pgfscope}%
\begin{pgfscope}%
\pgfsetbuttcap%
\pgfsetroundjoin%
\definecolor{currentfill}{rgb}{0.000000,0.000000,0.000000}%
\pgfsetfillcolor{currentfill}%
\pgfsetlinewidth{0.803000pt}%
\definecolor{currentstroke}{rgb}{0.000000,0.000000,0.000000}%
\pgfsetstrokecolor{currentstroke}%
\pgfsetdash{}{0pt}%
\pgfsys@defobject{currentmarker}{\pgfqpoint{0.000000in}{-0.048611in}}{\pgfqpoint{0.000000in}{0.000000in}}{%
\pgfpathmoveto{\pgfqpoint{0.000000in}{0.000000in}}%
\pgfpathlineto{\pgfqpoint{0.000000in}{-0.048611in}}%
\pgfusepath{stroke,fill}%
}%
\begin{pgfscope}%
\pgfsys@transformshift{4.026940in}{0.528000in}%
\pgfsys@useobject{currentmarker}{}%
\end{pgfscope}%
\end{pgfscope}%
\begin{pgfscope}%
\definecolor{textcolor}{rgb}{0.000000,0.000000,0.000000}%
\pgfsetstrokecolor{textcolor}%
\pgfsetfillcolor{textcolor}%
\pgftext[x=4.026940in,y=0.430778in,,top]{\color{textcolor}\sffamily\fontsize{10.000000}{12.000000}\selectfont 20}%
\end{pgfscope}%
\begin{pgfscope}%
\pgfpathrectangle{\pgfqpoint{0.800000in}{0.528000in}}{\pgfqpoint{4.960000in}{3.696000in}}%
\pgfusepath{clip}%
\pgfsetrectcap%
\pgfsetroundjoin%
\pgfsetlinewidth{0.803000pt}%
\definecolor{currentstroke}{rgb}{0.690196,0.690196,0.690196}%
\pgfsetstrokecolor{currentstroke}%
\pgfsetdash{}{0pt}%
\pgfpathmoveto{\pgfqpoint{4.777312in}{0.528000in}}%
\pgfpathlineto{\pgfqpoint{4.777312in}{4.224000in}}%
\pgfusepath{stroke}%
\end{pgfscope}%
\begin{pgfscope}%
\pgfsetbuttcap%
\pgfsetroundjoin%
\definecolor{currentfill}{rgb}{0.000000,0.000000,0.000000}%
\pgfsetfillcolor{currentfill}%
\pgfsetlinewidth{0.803000pt}%
\definecolor{currentstroke}{rgb}{0.000000,0.000000,0.000000}%
\pgfsetstrokecolor{currentstroke}%
\pgfsetdash{}{0pt}%
\pgfsys@defobject{currentmarker}{\pgfqpoint{0.000000in}{-0.048611in}}{\pgfqpoint{0.000000in}{0.000000in}}{%
\pgfpathmoveto{\pgfqpoint{0.000000in}{0.000000in}}%
\pgfpathlineto{\pgfqpoint{0.000000in}{-0.048611in}}%
\pgfusepath{stroke,fill}%
}%
\begin{pgfscope}%
\pgfsys@transformshift{4.777312in}{0.528000in}%
\pgfsys@useobject{currentmarker}{}%
\end{pgfscope}%
\end{pgfscope}%
\begin{pgfscope}%
\definecolor{textcolor}{rgb}{0.000000,0.000000,0.000000}%
\pgfsetstrokecolor{textcolor}%
\pgfsetfillcolor{textcolor}%
\pgftext[x=4.777312in,y=0.430778in,,top]{\color{textcolor}\sffamily\fontsize{10.000000}{12.000000}\selectfont 25}%
\end{pgfscope}%
\begin{pgfscope}%
\pgfpathrectangle{\pgfqpoint{0.800000in}{0.528000in}}{\pgfqpoint{4.960000in}{3.696000in}}%
\pgfusepath{clip}%
\pgfsetrectcap%
\pgfsetroundjoin%
\pgfsetlinewidth{0.803000pt}%
\definecolor{currentstroke}{rgb}{0.690196,0.690196,0.690196}%
\pgfsetstrokecolor{currentstroke}%
\pgfsetdash{}{0pt}%
\pgfpathmoveto{\pgfqpoint{5.527683in}{0.528000in}}%
\pgfpathlineto{\pgfqpoint{5.527683in}{4.224000in}}%
\pgfusepath{stroke}%
\end{pgfscope}%
\begin{pgfscope}%
\pgfsetbuttcap%
\pgfsetroundjoin%
\definecolor{currentfill}{rgb}{0.000000,0.000000,0.000000}%
\pgfsetfillcolor{currentfill}%
\pgfsetlinewidth{0.803000pt}%
\definecolor{currentstroke}{rgb}{0.000000,0.000000,0.000000}%
\pgfsetstrokecolor{currentstroke}%
\pgfsetdash{}{0pt}%
\pgfsys@defobject{currentmarker}{\pgfqpoint{0.000000in}{-0.048611in}}{\pgfqpoint{0.000000in}{0.000000in}}{%
\pgfpathmoveto{\pgfqpoint{0.000000in}{0.000000in}}%
\pgfpathlineto{\pgfqpoint{0.000000in}{-0.048611in}}%
\pgfusepath{stroke,fill}%
}%
\begin{pgfscope}%
\pgfsys@transformshift{5.527683in}{0.528000in}%
\pgfsys@useobject{currentmarker}{}%
\end{pgfscope}%
\end{pgfscope}%
\begin{pgfscope}%
\definecolor{textcolor}{rgb}{0.000000,0.000000,0.000000}%
\pgfsetstrokecolor{textcolor}%
\pgfsetfillcolor{textcolor}%
\pgftext[x=5.527683in,y=0.430778in,,top]{\color{textcolor}\sffamily\fontsize{10.000000}{12.000000}\selectfont 30}%
\end{pgfscope}%
\begin{pgfscope}%
\definecolor{textcolor}{rgb}{0.000000,0.000000,0.000000}%
\pgfsetstrokecolor{textcolor}%
\pgfsetfillcolor{textcolor}%
\pgftext[x=3.280000in,y=0.240809in,,top]{\color{textcolor}\sffamily\fontsize{10.000000}{12.000000}\selectfont time [s]}%
\end{pgfscope}%
\begin{pgfscope}%
\pgfpathrectangle{\pgfqpoint{0.800000in}{0.528000in}}{\pgfqpoint{4.960000in}{3.696000in}}%
\pgfusepath{clip}%
\pgfsetrectcap%
\pgfsetroundjoin%
\pgfsetlinewidth{0.803000pt}%
\definecolor{currentstroke}{rgb}{0.690196,0.690196,0.690196}%
\pgfsetstrokecolor{currentstroke}%
\pgfsetdash{}{0pt}%
\pgfpathmoveto{\pgfqpoint{0.800000in}{0.605677in}}%
\pgfpathlineto{\pgfqpoint{5.760000in}{0.605677in}}%
\pgfusepath{stroke}%
\end{pgfscope}%
\begin{pgfscope}%
\pgfsetbuttcap%
\pgfsetroundjoin%
\definecolor{currentfill}{rgb}{0.000000,0.000000,0.000000}%
\pgfsetfillcolor{currentfill}%
\pgfsetlinewidth{0.803000pt}%
\definecolor{currentstroke}{rgb}{0.000000,0.000000,0.000000}%
\pgfsetstrokecolor{currentstroke}%
\pgfsetdash{}{0pt}%
\pgfsys@defobject{currentmarker}{\pgfqpoint{-0.048611in}{0.000000in}}{\pgfqpoint{-0.000000in}{0.000000in}}{%
\pgfpathmoveto{\pgfqpoint{-0.000000in}{0.000000in}}%
\pgfpathlineto{\pgfqpoint{-0.048611in}{0.000000in}}%
\pgfusepath{stroke,fill}%
}%
\begin{pgfscope}%
\pgfsys@transformshift{0.800000in}{0.605677in}%
\pgfsys@useobject{currentmarker}{}%
\end{pgfscope}%
\end{pgfscope}%
\begin{pgfscope}%
\definecolor{textcolor}{rgb}{0.000000,0.000000,0.000000}%
\pgfsetstrokecolor{textcolor}%
\pgfsetfillcolor{textcolor}%
\pgftext[x=0.506387in, y=0.552915in, left, base]{\color{textcolor}\sffamily\fontsize{10.000000}{12.000000}\selectfont \ensuremath{-}3}%
\end{pgfscope}%
\begin{pgfscope}%
\pgfpathrectangle{\pgfqpoint{0.800000in}{0.528000in}}{\pgfqpoint{4.960000in}{3.696000in}}%
\pgfusepath{clip}%
\pgfsetrectcap%
\pgfsetroundjoin%
\pgfsetlinewidth{0.803000pt}%
\definecolor{currentstroke}{rgb}{0.690196,0.690196,0.690196}%
\pgfsetstrokecolor{currentstroke}%
\pgfsetdash{}{0pt}%
\pgfpathmoveto{\pgfqpoint{0.800000in}{1.036967in}}%
\pgfpathlineto{\pgfqpoint{5.760000in}{1.036967in}}%
\pgfusepath{stroke}%
\end{pgfscope}%
\begin{pgfscope}%
\pgfsetbuttcap%
\pgfsetroundjoin%
\definecolor{currentfill}{rgb}{0.000000,0.000000,0.000000}%
\pgfsetfillcolor{currentfill}%
\pgfsetlinewidth{0.803000pt}%
\definecolor{currentstroke}{rgb}{0.000000,0.000000,0.000000}%
\pgfsetstrokecolor{currentstroke}%
\pgfsetdash{}{0pt}%
\pgfsys@defobject{currentmarker}{\pgfqpoint{-0.048611in}{0.000000in}}{\pgfqpoint{-0.000000in}{0.000000in}}{%
\pgfpathmoveto{\pgfqpoint{-0.000000in}{0.000000in}}%
\pgfpathlineto{\pgfqpoint{-0.048611in}{0.000000in}}%
\pgfusepath{stroke,fill}%
}%
\begin{pgfscope}%
\pgfsys@transformshift{0.800000in}{1.036967in}%
\pgfsys@useobject{currentmarker}{}%
\end{pgfscope}%
\end{pgfscope}%
\begin{pgfscope}%
\definecolor{textcolor}{rgb}{0.000000,0.000000,0.000000}%
\pgfsetstrokecolor{textcolor}%
\pgfsetfillcolor{textcolor}%
\pgftext[x=0.506387in, y=0.984206in, left, base]{\color{textcolor}\sffamily\fontsize{10.000000}{12.000000}\selectfont \ensuremath{-}2}%
\end{pgfscope}%
\begin{pgfscope}%
\pgfpathrectangle{\pgfqpoint{0.800000in}{0.528000in}}{\pgfqpoint{4.960000in}{3.696000in}}%
\pgfusepath{clip}%
\pgfsetrectcap%
\pgfsetroundjoin%
\pgfsetlinewidth{0.803000pt}%
\definecolor{currentstroke}{rgb}{0.690196,0.690196,0.690196}%
\pgfsetstrokecolor{currentstroke}%
\pgfsetdash{}{0pt}%
\pgfpathmoveto{\pgfqpoint{0.800000in}{1.468258in}}%
\pgfpathlineto{\pgfqpoint{5.760000in}{1.468258in}}%
\pgfusepath{stroke}%
\end{pgfscope}%
\begin{pgfscope}%
\pgfsetbuttcap%
\pgfsetroundjoin%
\definecolor{currentfill}{rgb}{0.000000,0.000000,0.000000}%
\pgfsetfillcolor{currentfill}%
\pgfsetlinewidth{0.803000pt}%
\definecolor{currentstroke}{rgb}{0.000000,0.000000,0.000000}%
\pgfsetstrokecolor{currentstroke}%
\pgfsetdash{}{0pt}%
\pgfsys@defobject{currentmarker}{\pgfqpoint{-0.048611in}{0.000000in}}{\pgfqpoint{-0.000000in}{0.000000in}}{%
\pgfpathmoveto{\pgfqpoint{-0.000000in}{0.000000in}}%
\pgfpathlineto{\pgfqpoint{-0.048611in}{0.000000in}}%
\pgfusepath{stroke,fill}%
}%
\begin{pgfscope}%
\pgfsys@transformshift{0.800000in}{1.468258in}%
\pgfsys@useobject{currentmarker}{}%
\end{pgfscope}%
\end{pgfscope}%
\begin{pgfscope}%
\definecolor{textcolor}{rgb}{0.000000,0.000000,0.000000}%
\pgfsetstrokecolor{textcolor}%
\pgfsetfillcolor{textcolor}%
\pgftext[x=0.506387in, y=1.415496in, left, base]{\color{textcolor}\sffamily\fontsize{10.000000}{12.000000}\selectfont \ensuremath{-}1}%
\end{pgfscope}%
\begin{pgfscope}%
\pgfpathrectangle{\pgfqpoint{0.800000in}{0.528000in}}{\pgfqpoint{4.960000in}{3.696000in}}%
\pgfusepath{clip}%
\pgfsetrectcap%
\pgfsetroundjoin%
\pgfsetlinewidth{0.803000pt}%
\definecolor{currentstroke}{rgb}{0.690196,0.690196,0.690196}%
\pgfsetstrokecolor{currentstroke}%
\pgfsetdash{}{0pt}%
\pgfpathmoveto{\pgfqpoint{0.800000in}{1.899548in}}%
\pgfpathlineto{\pgfqpoint{5.760000in}{1.899548in}}%
\pgfusepath{stroke}%
\end{pgfscope}%
\begin{pgfscope}%
\pgfsetbuttcap%
\pgfsetroundjoin%
\definecolor{currentfill}{rgb}{0.000000,0.000000,0.000000}%
\pgfsetfillcolor{currentfill}%
\pgfsetlinewidth{0.803000pt}%
\definecolor{currentstroke}{rgb}{0.000000,0.000000,0.000000}%
\pgfsetstrokecolor{currentstroke}%
\pgfsetdash{}{0pt}%
\pgfsys@defobject{currentmarker}{\pgfqpoint{-0.048611in}{0.000000in}}{\pgfqpoint{-0.000000in}{0.000000in}}{%
\pgfpathmoveto{\pgfqpoint{-0.000000in}{0.000000in}}%
\pgfpathlineto{\pgfqpoint{-0.048611in}{0.000000in}}%
\pgfusepath{stroke,fill}%
}%
\begin{pgfscope}%
\pgfsys@transformshift{0.800000in}{1.899548in}%
\pgfsys@useobject{currentmarker}{}%
\end{pgfscope}%
\end{pgfscope}%
\begin{pgfscope}%
\definecolor{textcolor}{rgb}{0.000000,0.000000,0.000000}%
\pgfsetstrokecolor{textcolor}%
\pgfsetfillcolor{textcolor}%
\pgftext[x=0.614412in, y=1.846786in, left, base]{\color{textcolor}\sffamily\fontsize{10.000000}{12.000000}\selectfont 0}%
\end{pgfscope}%
\begin{pgfscope}%
\pgfpathrectangle{\pgfqpoint{0.800000in}{0.528000in}}{\pgfqpoint{4.960000in}{3.696000in}}%
\pgfusepath{clip}%
\pgfsetrectcap%
\pgfsetroundjoin%
\pgfsetlinewidth{0.803000pt}%
\definecolor{currentstroke}{rgb}{0.690196,0.690196,0.690196}%
\pgfsetstrokecolor{currentstroke}%
\pgfsetdash{}{0pt}%
\pgfpathmoveto{\pgfqpoint{0.800000in}{2.330838in}}%
\pgfpathlineto{\pgfqpoint{5.760000in}{2.330838in}}%
\pgfusepath{stroke}%
\end{pgfscope}%
\begin{pgfscope}%
\pgfsetbuttcap%
\pgfsetroundjoin%
\definecolor{currentfill}{rgb}{0.000000,0.000000,0.000000}%
\pgfsetfillcolor{currentfill}%
\pgfsetlinewidth{0.803000pt}%
\definecolor{currentstroke}{rgb}{0.000000,0.000000,0.000000}%
\pgfsetstrokecolor{currentstroke}%
\pgfsetdash{}{0pt}%
\pgfsys@defobject{currentmarker}{\pgfqpoint{-0.048611in}{0.000000in}}{\pgfqpoint{-0.000000in}{0.000000in}}{%
\pgfpathmoveto{\pgfqpoint{-0.000000in}{0.000000in}}%
\pgfpathlineto{\pgfqpoint{-0.048611in}{0.000000in}}%
\pgfusepath{stroke,fill}%
}%
\begin{pgfscope}%
\pgfsys@transformshift{0.800000in}{2.330838in}%
\pgfsys@useobject{currentmarker}{}%
\end{pgfscope}%
\end{pgfscope}%
\begin{pgfscope}%
\definecolor{textcolor}{rgb}{0.000000,0.000000,0.000000}%
\pgfsetstrokecolor{textcolor}%
\pgfsetfillcolor{textcolor}%
\pgftext[x=0.614412in, y=2.278077in, left, base]{\color{textcolor}\sffamily\fontsize{10.000000}{12.000000}\selectfont 1}%
\end{pgfscope}%
\begin{pgfscope}%
\pgfpathrectangle{\pgfqpoint{0.800000in}{0.528000in}}{\pgfqpoint{4.960000in}{3.696000in}}%
\pgfusepath{clip}%
\pgfsetrectcap%
\pgfsetroundjoin%
\pgfsetlinewidth{0.803000pt}%
\definecolor{currentstroke}{rgb}{0.690196,0.690196,0.690196}%
\pgfsetstrokecolor{currentstroke}%
\pgfsetdash{}{0pt}%
\pgfpathmoveto{\pgfqpoint{0.800000in}{2.762129in}}%
\pgfpathlineto{\pgfqpoint{5.760000in}{2.762129in}}%
\pgfusepath{stroke}%
\end{pgfscope}%
\begin{pgfscope}%
\pgfsetbuttcap%
\pgfsetroundjoin%
\definecolor{currentfill}{rgb}{0.000000,0.000000,0.000000}%
\pgfsetfillcolor{currentfill}%
\pgfsetlinewidth{0.803000pt}%
\definecolor{currentstroke}{rgb}{0.000000,0.000000,0.000000}%
\pgfsetstrokecolor{currentstroke}%
\pgfsetdash{}{0pt}%
\pgfsys@defobject{currentmarker}{\pgfqpoint{-0.048611in}{0.000000in}}{\pgfqpoint{-0.000000in}{0.000000in}}{%
\pgfpathmoveto{\pgfqpoint{-0.000000in}{0.000000in}}%
\pgfpathlineto{\pgfqpoint{-0.048611in}{0.000000in}}%
\pgfusepath{stroke,fill}%
}%
\begin{pgfscope}%
\pgfsys@transformshift{0.800000in}{2.762129in}%
\pgfsys@useobject{currentmarker}{}%
\end{pgfscope}%
\end{pgfscope}%
\begin{pgfscope}%
\definecolor{textcolor}{rgb}{0.000000,0.000000,0.000000}%
\pgfsetstrokecolor{textcolor}%
\pgfsetfillcolor{textcolor}%
\pgftext[x=0.614412in, y=2.709367in, left, base]{\color{textcolor}\sffamily\fontsize{10.000000}{12.000000}\selectfont 2}%
\end{pgfscope}%
\begin{pgfscope}%
\pgfpathrectangle{\pgfqpoint{0.800000in}{0.528000in}}{\pgfqpoint{4.960000in}{3.696000in}}%
\pgfusepath{clip}%
\pgfsetrectcap%
\pgfsetroundjoin%
\pgfsetlinewidth{0.803000pt}%
\definecolor{currentstroke}{rgb}{0.690196,0.690196,0.690196}%
\pgfsetstrokecolor{currentstroke}%
\pgfsetdash{}{0pt}%
\pgfpathmoveto{\pgfqpoint{0.800000in}{3.193419in}}%
\pgfpathlineto{\pgfqpoint{5.760000in}{3.193419in}}%
\pgfusepath{stroke}%
\end{pgfscope}%
\begin{pgfscope}%
\pgfsetbuttcap%
\pgfsetroundjoin%
\definecolor{currentfill}{rgb}{0.000000,0.000000,0.000000}%
\pgfsetfillcolor{currentfill}%
\pgfsetlinewidth{0.803000pt}%
\definecolor{currentstroke}{rgb}{0.000000,0.000000,0.000000}%
\pgfsetstrokecolor{currentstroke}%
\pgfsetdash{}{0pt}%
\pgfsys@defobject{currentmarker}{\pgfqpoint{-0.048611in}{0.000000in}}{\pgfqpoint{-0.000000in}{0.000000in}}{%
\pgfpathmoveto{\pgfqpoint{-0.000000in}{0.000000in}}%
\pgfpathlineto{\pgfqpoint{-0.048611in}{0.000000in}}%
\pgfusepath{stroke,fill}%
}%
\begin{pgfscope}%
\pgfsys@transformshift{0.800000in}{3.193419in}%
\pgfsys@useobject{currentmarker}{}%
\end{pgfscope}%
\end{pgfscope}%
\begin{pgfscope}%
\definecolor{textcolor}{rgb}{0.000000,0.000000,0.000000}%
\pgfsetstrokecolor{textcolor}%
\pgfsetfillcolor{textcolor}%
\pgftext[x=0.614412in, y=3.140658in, left, base]{\color{textcolor}\sffamily\fontsize{10.000000}{12.000000}\selectfont 3}%
\end{pgfscope}%
\begin{pgfscope}%
\pgfpathrectangle{\pgfqpoint{0.800000in}{0.528000in}}{\pgfqpoint{4.960000in}{3.696000in}}%
\pgfusepath{clip}%
\pgfsetrectcap%
\pgfsetroundjoin%
\pgfsetlinewidth{0.803000pt}%
\definecolor{currentstroke}{rgb}{0.690196,0.690196,0.690196}%
\pgfsetstrokecolor{currentstroke}%
\pgfsetdash{}{0pt}%
\pgfpathmoveto{\pgfqpoint{0.800000in}{3.624710in}}%
\pgfpathlineto{\pgfqpoint{5.760000in}{3.624710in}}%
\pgfusepath{stroke}%
\end{pgfscope}%
\begin{pgfscope}%
\pgfsetbuttcap%
\pgfsetroundjoin%
\definecolor{currentfill}{rgb}{0.000000,0.000000,0.000000}%
\pgfsetfillcolor{currentfill}%
\pgfsetlinewidth{0.803000pt}%
\definecolor{currentstroke}{rgb}{0.000000,0.000000,0.000000}%
\pgfsetstrokecolor{currentstroke}%
\pgfsetdash{}{0pt}%
\pgfsys@defobject{currentmarker}{\pgfqpoint{-0.048611in}{0.000000in}}{\pgfqpoint{-0.000000in}{0.000000in}}{%
\pgfpathmoveto{\pgfqpoint{-0.000000in}{0.000000in}}%
\pgfpathlineto{\pgfqpoint{-0.048611in}{0.000000in}}%
\pgfusepath{stroke,fill}%
}%
\begin{pgfscope}%
\pgfsys@transformshift{0.800000in}{3.624710in}%
\pgfsys@useobject{currentmarker}{}%
\end{pgfscope}%
\end{pgfscope}%
\begin{pgfscope}%
\definecolor{textcolor}{rgb}{0.000000,0.000000,0.000000}%
\pgfsetstrokecolor{textcolor}%
\pgfsetfillcolor{textcolor}%
\pgftext[x=0.614412in, y=3.571948in, left, base]{\color{textcolor}\sffamily\fontsize{10.000000}{12.000000}\selectfont 4}%
\end{pgfscope}%
\begin{pgfscope}%
\pgfpathrectangle{\pgfqpoint{0.800000in}{0.528000in}}{\pgfqpoint{4.960000in}{3.696000in}}%
\pgfusepath{clip}%
\pgfsetrectcap%
\pgfsetroundjoin%
\pgfsetlinewidth{0.803000pt}%
\definecolor{currentstroke}{rgb}{0.690196,0.690196,0.690196}%
\pgfsetstrokecolor{currentstroke}%
\pgfsetdash{}{0pt}%
\pgfpathmoveto{\pgfqpoint{0.800000in}{4.056000in}}%
\pgfpathlineto{\pgfqpoint{5.760000in}{4.056000in}}%
\pgfusepath{stroke}%
\end{pgfscope}%
\begin{pgfscope}%
\pgfsetbuttcap%
\pgfsetroundjoin%
\definecolor{currentfill}{rgb}{0.000000,0.000000,0.000000}%
\pgfsetfillcolor{currentfill}%
\pgfsetlinewidth{0.803000pt}%
\definecolor{currentstroke}{rgb}{0.000000,0.000000,0.000000}%
\pgfsetstrokecolor{currentstroke}%
\pgfsetdash{}{0pt}%
\pgfsys@defobject{currentmarker}{\pgfqpoint{-0.048611in}{0.000000in}}{\pgfqpoint{-0.000000in}{0.000000in}}{%
\pgfpathmoveto{\pgfqpoint{-0.000000in}{0.000000in}}%
\pgfpathlineto{\pgfqpoint{-0.048611in}{0.000000in}}%
\pgfusepath{stroke,fill}%
}%
\begin{pgfscope}%
\pgfsys@transformshift{0.800000in}{4.056000in}%
\pgfsys@useobject{currentmarker}{}%
\end{pgfscope}%
\end{pgfscope}%
\begin{pgfscope}%
\definecolor{textcolor}{rgb}{0.000000,0.000000,0.000000}%
\pgfsetstrokecolor{textcolor}%
\pgfsetfillcolor{textcolor}%
\pgftext[x=0.614412in, y=4.003238in, left, base]{\color{textcolor}\sffamily\fontsize{10.000000}{12.000000}\selectfont 5}%
\end{pgfscope}%
\begin{pgfscope}%
\definecolor{textcolor}{rgb}{0.000000,0.000000,0.000000}%
\pgfsetstrokecolor{textcolor}%
\pgfsetfillcolor{textcolor}%
\pgftext[x=0.450832in,y=2.376000in,,bottom,rotate=90.000000]{\color{textcolor}\sffamily\fontsize{10.000000}{12.000000}\selectfont Output velocity [deg/s]}%
\end{pgfscope}%
\begin{pgfscope}%
\pgfpathrectangle{\pgfqpoint{0.800000in}{0.528000in}}{\pgfqpoint{4.960000in}{3.696000in}}%
\pgfusepath{clip}%
\pgfsetrectcap%
\pgfsetroundjoin%
\pgfsetlinewidth{1.505625pt}%
\definecolor{currentstroke}{rgb}{0.121569,0.466667,0.705882}%
\pgfsetstrokecolor{currentstroke}%
\pgfsetdash{}{0pt}%
\pgfpathmoveto{\pgfqpoint{1.025455in}{4.056000in}}%
\pgfpathlineto{\pgfqpoint{1.079822in}{4.056000in}}%
\pgfpathlineto{\pgfqpoint{1.133922in}{4.056000in}}%
\pgfpathlineto{\pgfqpoint{1.188320in}{4.056000in}}%
\pgfpathlineto{\pgfqpoint{1.242522in}{4.056000in}}%
\pgfpathlineto{\pgfqpoint{1.296781in}{4.056000in}}%
\pgfpathlineto{\pgfqpoint{1.351107in}{4.056000in}}%
\pgfpathlineto{\pgfqpoint{1.406915in}{3.341593in}}%
\pgfpathlineto{\pgfqpoint{1.459887in}{2.707920in}}%
\pgfpathlineto{\pgfqpoint{1.514174in}{2.001412in}}%
\pgfpathlineto{\pgfqpoint{1.568068in}{1.727162in}}%
\pgfpathlineto{\pgfqpoint{1.622011in}{1.813614in}}%
\pgfpathlineto{\pgfqpoint{1.678291in}{1.794195in}}%
\pgfpathlineto{\pgfqpoint{1.730672in}{1.879552in}}%
\pgfpathlineto{\pgfqpoint{1.784529in}{1.873876in}}%
\pgfpathlineto{\pgfqpoint{1.840741in}{1.963822in}}%
\pgfpathlineto{\pgfqpoint{1.893182in}{2.011980in}}%
\pgfpathlineto{\pgfqpoint{1.950346in}{2.160117in}}%
\pgfpathlineto{\pgfqpoint{2.004193in}{2.105881in}}%
\pgfpathlineto{\pgfqpoint{2.057899in}{2.108414in}}%
\pgfpathlineto{\pgfqpoint{2.111638in}{1.992437in}}%
\pgfpathlineto{\pgfqpoint{2.166296in}{1.869227in}}%
\pgfpathlineto{\pgfqpoint{2.221139in}{1.941054in}}%
\pgfpathlineto{\pgfqpoint{2.274923in}{1.947068in}}%
\pgfpathlineto{\pgfqpoint{2.329544in}{1.896777in}}%
\pgfpathlineto{\pgfqpoint{2.383912in}{1.910209in}}%
\pgfpathlineto{\pgfqpoint{2.438011in}{1.869588in}}%
\pgfpathlineto{\pgfqpoint{2.492382in}{1.855636in}}%
\pgfpathlineto{\pgfqpoint{2.546771in}{1.837353in}}%
\pgfpathlineto{\pgfqpoint{2.603285in}{1.858406in}}%
\pgfpathlineto{\pgfqpoint{2.656649in}{1.871751in}}%
\pgfpathlineto{\pgfqpoint{2.710414in}{1.762768in}}%
\pgfpathlineto{\pgfqpoint{2.764551in}{1.810251in}}%
\pgfpathlineto{\pgfqpoint{2.818550in}{1.873284in}}%
\pgfpathlineto{\pgfqpoint{2.873018in}{1.926187in}}%
\pgfpathlineto{\pgfqpoint{2.927117in}{1.947357in}}%
\pgfpathlineto{\pgfqpoint{2.981133in}{1.947729in}}%
\pgfpathlineto{\pgfqpoint{3.037500in}{1.953000in}}%
\pgfpathlineto{\pgfqpoint{3.090526in}{1.955217in}}%
\pgfpathlineto{\pgfqpoint{3.144727in}{1.857736in}}%
\pgfpathlineto{\pgfqpoint{3.199853in}{1.873427in}}%
\pgfpathlineto{\pgfqpoint{3.253313in}{1.837520in}}%
\pgfpathlineto{\pgfqpoint{3.307116in}{1.791671in}}%
\pgfpathlineto{\pgfqpoint{3.361269in}{1.839640in}}%
\pgfpathlineto{\pgfqpoint{3.415362in}{1.838195in}}%
\pgfpathlineto{\pgfqpoint{3.469677in}{1.849918in}}%
\pgfpathlineto{\pgfqpoint{3.523499in}{1.864678in}}%
\pgfpathlineto{\pgfqpoint{3.578629in}{1.888199in}}%
\pgfpathlineto{\pgfqpoint{3.632694in}{1.972924in}}%
\pgfpathlineto{\pgfqpoint{3.686957in}{1.983205in}}%
\pgfpathlineto{\pgfqpoint{3.741822in}{1.989113in}}%
\pgfpathlineto{\pgfqpoint{3.795868in}{1.930147in}}%
\pgfpathlineto{\pgfqpoint{3.851520in}{1.622489in}}%
\pgfpathlineto{\pgfqpoint{3.905190in}{1.677833in}}%
\pgfpathlineto{\pgfqpoint{3.958826in}{1.934879in}}%
\pgfpathlineto{\pgfqpoint{4.013358in}{1.990652in}}%
\pgfpathlineto{\pgfqpoint{4.067443in}{1.985512in}}%
\pgfpathlineto{\pgfqpoint{4.121646in}{1.978610in}}%
\pgfpathlineto{\pgfqpoint{4.175763in}{2.018431in}}%
\pgfpathlineto{\pgfqpoint{4.229865in}{2.052243in}}%
\pgfpathlineto{\pgfqpoint{4.284443in}{1.938293in}}%
\pgfpathlineto{\pgfqpoint{4.338643in}{1.921215in}}%
\pgfpathlineto{\pgfqpoint{4.393069in}{1.973585in}}%
\pgfpathlineto{\pgfqpoint{4.446910in}{1.976614in}}%
\pgfpathlineto{\pgfqpoint{4.502405in}{1.981206in}}%
\pgfpathlineto{\pgfqpoint{4.557185in}{1.964351in}}%
\pgfpathlineto{\pgfqpoint{4.610791in}{1.957697in}}%
\pgfpathlineto{\pgfqpoint{4.665075in}{1.799080in}}%
\pgfpathlineto{\pgfqpoint{4.719799in}{1.863925in}}%
\pgfpathlineto{\pgfqpoint{4.773954in}{1.855259in}}%
\pgfpathlineto{\pgfqpoint{4.828060in}{1.780456in}}%
\pgfpathlineto{\pgfqpoint{4.882094in}{1.806933in}}%
\pgfpathlineto{\pgfqpoint{4.936561in}{1.944245in}}%
\pgfpathlineto{\pgfqpoint{4.990765in}{1.949740in}}%
\pgfpathlineto{\pgfqpoint{5.045060in}{1.931588in}}%
\pgfpathlineto{\pgfqpoint{5.100911in}{1.912313in}}%
\pgfpathlineto{\pgfqpoint{5.154168in}{1.901904in}}%
\pgfpathlineto{\pgfqpoint{5.208005in}{1.919990in}}%
\pgfpathlineto{\pgfqpoint{5.262003in}{1.921101in}}%
\pgfpathlineto{\pgfqpoint{5.316720in}{1.907674in}}%
\pgfpathlineto{\pgfqpoint{5.370968in}{1.898639in}}%
\pgfpathlineto{\pgfqpoint{5.424892in}{1.921203in}}%
\pgfpathlineto{\pgfqpoint{5.479256in}{1.914728in}}%
\pgfpathlineto{\pgfqpoint{5.534545in}{1.926093in}}%
\pgfusepath{stroke}%
\end{pgfscope}%
\begin{pgfscope}%
\pgfpathrectangle{\pgfqpoint{0.800000in}{0.528000in}}{\pgfqpoint{4.960000in}{3.696000in}}%
\pgfusepath{clip}%
\pgfsetrectcap%
\pgfsetroundjoin%
\pgfsetlinewidth{1.505625pt}%
\definecolor{currentstroke}{rgb}{1.000000,0.498039,0.054902}%
\pgfsetstrokecolor{currentstroke}%
\pgfsetdash{}{0pt}%
\pgfpathmoveto{\pgfqpoint{1.025455in}{4.056000in}}%
\pgfpathlineto{\pgfqpoint{1.080191in}{4.056000in}}%
\pgfpathlineto{\pgfqpoint{1.134511in}{4.056000in}}%
\pgfpathlineto{\pgfqpoint{1.189704in}{4.056000in}}%
\pgfpathlineto{\pgfqpoint{1.243700in}{4.056000in}}%
\pgfpathlineto{\pgfqpoint{1.297459in}{4.056000in}}%
\pgfpathlineto{\pgfqpoint{1.351906in}{4.056000in}}%
\pgfpathlineto{\pgfqpoint{1.406392in}{4.056000in}}%
\pgfpathlineto{\pgfqpoint{1.460832in}{3.353417in}}%
\pgfpathlineto{\pgfqpoint{1.515300in}{2.675445in}}%
\pgfpathlineto{\pgfqpoint{1.569187in}{1.994722in}}%
\pgfpathlineto{\pgfqpoint{1.623453in}{1.686809in}}%
\pgfpathlineto{\pgfqpoint{1.677782in}{1.594458in}}%
\pgfpathlineto{\pgfqpoint{1.731965in}{1.495118in}}%
\pgfpathlineto{\pgfqpoint{1.788004in}{1.502682in}}%
\pgfpathlineto{\pgfqpoint{1.841917in}{1.808503in}}%
\pgfpathlineto{\pgfqpoint{1.895653in}{1.983953in}}%
\pgfpathlineto{\pgfqpoint{1.950155in}{2.112093in}}%
\pgfpathlineto{\pgfqpoint{2.004236in}{2.163820in}}%
\pgfpathlineto{\pgfqpoint{2.058626in}{2.158624in}}%
\pgfpathlineto{\pgfqpoint{2.112828in}{2.051604in}}%
\pgfpathlineto{\pgfqpoint{2.167037in}{2.014490in}}%
\pgfpathlineto{\pgfqpoint{2.221146in}{1.982706in}}%
\pgfpathlineto{\pgfqpoint{2.275545in}{1.962089in}}%
\pgfpathlineto{\pgfqpoint{2.329935in}{1.889354in}}%
\pgfpathlineto{\pgfqpoint{2.384150in}{1.889944in}}%
\pgfpathlineto{\pgfqpoint{2.440191in}{1.944714in}}%
\pgfpathlineto{\pgfqpoint{2.493932in}{1.864430in}}%
\pgfpathlineto{\pgfqpoint{2.547595in}{1.816015in}}%
\pgfpathlineto{\pgfqpoint{2.601579in}{1.812279in}}%
\pgfpathlineto{\pgfqpoint{2.655631in}{1.773366in}}%
\pgfpathlineto{\pgfqpoint{2.710036in}{1.855797in}}%
\pgfpathlineto{\pgfqpoint{2.764029in}{1.866125in}}%
\pgfpathlineto{\pgfqpoint{2.818210in}{1.764080in}}%
\pgfpathlineto{\pgfqpoint{2.872700in}{1.714906in}}%
\pgfpathlineto{\pgfqpoint{2.927227in}{1.598911in}}%
\pgfpathlineto{\pgfqpoint{2.981657in}{1.638044in}}%
\pgfpathlineto{\pgfqpoint{3.036126in}{1.640676in}}%
\pgfpathlineto{\pgfqpoint{3.092179in}{1.775027in}}%
\pgfpathlineto{\pgfqpoint{3.145406in}{1.878964in}}%
\pgfpathlineto{\pgfqpoint{3.198920in}{1.924752in}}%
\pgfpathlineto{\pgfqpoint{3.253211in}{1.914369in}}%
\pgfpathlineto{\pgfqpoint{3.307156in}{1.906534in}}%
\pgfpathlineto{\pgfqpoint{3.361262in}{2.027455in}}%
\pgfpathlineto{\pgfqpoint{3.415759in}{2.003358in}}%
\pgfpathlineto{\pgfqpoint{3.470553in}{1.994675in}}%
\pgfpathlineto{\pgfqpoint{3.524682in}{1.955745in}}%
\pgfpathlineto{\pgfqpoint{3.578774in}{1.891588in}}%
\pgfpathlineto{\pgfqpoint{3.634763in}{1.873292in}}%
\pgfpathlineto{\pgfqpoint{3.688090in}{1.872951in}}%
\pgfpathlineto{\pgfqpoint{3.741865in}{1.867914in}}%
\pgfpathlineto{\pgfqpoint{3.796063in}{1.836413in}}%
\pgfpathlineto{\pgfqpoint{3.850120in}{1.819555in}}%
\pgfpathlineto{\pgfqpoint{3.904393in}{1.813524in}}%
\pgfpathlineto{\pgfqpoint{3.958371in}{1.832754in}}%
\pgfpathlineto{\pgfqpoint{4.012426in}{1.832942in}}%
\pgfpathlineto{\pgfqpoint{4.066952in}{1.850680in}}%
\pgfpathlineto{\pgfqpoint{4.121207in}{1.687933in}}%
\pgfpathlineto{\pgfqpoint{4.175715in}{1.811855in}}%
\pgfpathlineto{\pgfqpoint{4.231396in}{1.940748in}}%
\pgfpathlineto{\pgfqpoint{4.284967in}{2.016427in}}%
\pgfpathlineto{\pgfqpoint{4.338924in}{1.988701in}}%
\pgfpathlineto{\pgfqpoint{4.392995in}{2.001651in}}%
\pgfpathlineto{\pgfqpoint{4.446941in}{1.935960in}}%
\pgfpathlineto{\pgfqpoint{4.501277in}{2.045908in}}%
\pgfpathlineto{\pgfqpoint{4.555308in}{1.915688in}}%
\pgfpathlineto{\pgfqpoint{4.609611in}{1.860003in}}%
\pgfpathlineto{\pgfqpoint{4.663692in}{1.830699in}}%
\pgfpathlineto{\pgfqpoint{4.718167in}{1.924483in}}%
\pgfpathlineto{\pgfqpoint{4.772549in}{1.915401in}}%
\pgfpathlineto{\pgfqpoint{4.826956in}{1.909869in}}%
\pgfpathlineto{\pgfqpoint{4.882262in}{2.010636in}}%
\pgfpathlineto{\pgfqpoint{4.936365in}{2.052833in}}%
\pgfpathlineto{\pgfqpoint{4.990053in}{2.025186in}}%
\pgfpathlineto{\pgfqpoint{5.044253in}{1.787412in}}%
\pgfpathlineto{\pgfqpoint{5.098153in}{1.776030in}}%
\pgfpathlineto{\pgfqpoint{5.152260in}{1.787921in}}%
\pgfpathlineto{\pgfqpoint{5.206691in}{1.882294in}}%
\pgfpathlineto{\pgfqpoint{5.261099in}{1.845818in}}%
\pgfpathlineto{\pgfqpoint{5.315382in}{1.872906in}}%
\pgfpathlineto{\pgfqpoint{5.369554in}{1.964471in}}%
\pgfpathlineto{\pgfqpoint{5.424024in}{2.003161in}}%
\pgfpathlineto{\pgfqpoint{5.479725in}{2.004287in}}%
\pgfpathlineto{\pgfqpoint{5.533836in}{2.030446in}}%
\pgfusepath{stroke}%
\end{pgfscope}%
\begin{pgfscope}%
\pgfpathrectangle{\pgfqpoint{0.800000in}{0.528000in}}{\pgfqpoint{4.960000in}{3.696000in}}%
\pgfusepath{clip}%
\pgfsetrectcap%
\pgfsetroundjoin%
\pgfsetlinewidth{1.505625pt}%
\definecolor{currentstroke}{rgb}{0.172549,0.627451,0.172549}%
\pgfsetstrokecolor{currentstroke}%
\pgfsetdash{}{0pt}%
\pgfpathmoveto{\pgfqpoint{1.025455in}{4.056000in}}%
\pgfpathlineto{\pgfqpoint{1.079023in}{4.056000in}}%
\pgfpathlineto{\pgfqpoint{1.133077in}{4.056000in}}%
\pgfpathlineto{\pgfqpoint{1.187729in}{4.056000in}}%
\pgfpathlineto{\pgfqpoint{1.242426in}{4.056000in}}%
\pgfpathlineto{\pgfqpoint{1.296480in}{4.056000in}}%
\pgfpathlineto{\pgfqpoint{1.350844in}{4.056000in}}%
\pgfpathlineto{\pgfqpoint{1.404938in}{4.056000in}}%
\pgfpathlineto{\pgfqpoint{1.459203in}{3.647651in}}%
\pgfpathlineto{\pgfqpoint{1.513440in}{2.654069in}}%
\pgfpathlineto{\pgfqpoint{1.569296in}{2.162743in}}%
\pgfpathlineto{\pgfqpoint{1.622804in}{1.761477in}}%
\pgfpathlineto{\pgfqpoint{1.676188in}{1.297855in}}%
\pgfpathlineto{\pgfqpoint{1.731402in}{1.562670in}}%
\pgfpathlineto{\pgfqpoint{1.784618in}{1.507159in}}%
\pgfpathlineto{\pgfqpoint{1.839526in}{1.649285in}}%
\pgfpathlineto{\pgfqpoint{1.893763in}{1.733198in}}%
\pgfpathlineto{\pgfqpoint{1.948606in}{1.879639in}}%
\pgfpathlineto{\pgfqpoint{2.002550in}{2.043059in}}%
\pgfpathlineto{\pgfqpoint{2.056829in}{2.105298in}}%
\pgfpathlineto{\pgfqpoint{2.111081in}{2.124537in}}%
\pgfpathlineto{\pgfqpoint{2.165552in}{2.132713in}}%
\pgfpathlineto{\pgfqpoint{2.221187in}{2.074167in}}%
\pgfpathlineto{\pgfqpoint{2.274918in}{1.941365in}}%
\pgfpathlineto{\pgfqpoint{2.328304in}{1.909983in}}%
\pgfpathlineto{\pgfqpoint{2.382267in}{1.844060in}}%
\pgfpathlineto{\pgfqpoint{2.436537in}{1.859769in}}%
\pgfpathlineto{\pgfqpoint{2.490878in}{1.839907in}}%
\pgfpathlineto{\pgfqpoint{2.545024in}{1.832904in}}%
\pgfpathlineto{\pgfqpoint{2.599118in}{1.825756in}}%
\pgfpathlineto{\pgfqpoint{2.652939in}{1.849594in}}%
\pgfpathlineto{\pgfqpoint{2.707658in}{1.787063in}}%
\pgfpathlineto{\pgfqpoint{2.761830in}{1.843763in}}%
\pgfpathlineto{\pgfqpoint{2.816110in}{1.803772in}}%
\pgfpathlineto{\pgfqpoint{2.871106in}{1.777575in}}%
\pgfpathlineto{\pgfqpoint{2.924769in}{1.827994in}}%
\pgfpathlineto{\pgfqpoint{2.978646in}{1.917480in}}%
\pgfpathlineto{\pgfqpoint{3.032891in}{1.981371in}}%
\pgfpathlineto{\pgfqpoint{3.087098in}{1.921013in}}%
\pgfpathlineto{\pgfqpoint{3.141336in}{1.903846in}}%
\pgfpathlineto{\pgfqpoint{3.195614in}{1.713303in}}%
\pgfpathlineto{\pgfqpoint{3.250636in}{1.727867in}}%
\pgfpathlineto{\pgfqpoint{3.304750in}{1.847362in}}%
\pgfpathlineto{\pgfqpoint{3.358999in}{1.838632in}}%
\pgfpathlineto{\pgfqpoint{3.413137in}{1.874176in}}%
\pgfpathlineto{\pgfqpoint{3.468994in}{1.905855in}}%
\pgfpathlineto{\pgfqpoint{3.522336in}{1.937839in}}%
\pgfpathlineto{\pgfqpoint{3.576091in}{1.959842in}}%
\pgfpathlineto{\pgfqpoint{3.630296in}{1.965126in}}%
\pgfpathlineto{\pgfqpoint{3.684487in}{1.929649in}}%
\pgfpathlineto{\pgfqpoint{3.738575in}{1.859571in}}%
\pgfpathlineto{\pgfqpoint{3.792677in}{1.807451in}}%
\pgfpathlineto{\pgfqpoint{3.846960in}{1.795848in}}%
\pgfpathlineto{\pgfqpoint{3.900932in}{1.745287in}}%
\pgfpathlineto{\pgfqpoint{3.955316in}{1.873818in}}%
\pgfpathlineto{\pgfqpoint{4.009622in}{1.961403in}}%
\pgfpathlineto{\pgfqpoint{4.065801in}{1.967168in}}%
\pgfpathlineto{\pgfqpoint{4.123315in}{1.852218in}}%
\pgfpathlineto{\pgfqpoint{4.174546in}{1.874522in}}%
\pgfpathlineto{\pgfqpoint{4.228628in}{1.852778in}}%
\pgfpathlineto{\pgfqpoint{4.282595in}{1.873746in}}%
\pgfpathlineto{\pgfqpoint{4.336827in}{1.859011in}}%
\pgfpathlineto{\pgfqpoint{4.390621in}{1.867581in}}%
\pgfpathlineto{\pgfqpoint{4.444394in}{1.937321in}}%
\pgfpathlineto{\pgfqpoint{4.498739in}{2.016761in}}%
\pgfpathlineto{\pgfqpoint{4.553085in}{1.987165in}}%
\pgfpathlineto{\pgfqpoint{4.607184in}{1.733108in}}%
\pgfpathlineto{\pgfqpoint{4.661341in}{1.736759in}}%
\pgfpathlineto{\pgfqpoint{4.715660in}{1.853413in}}%
\pgfpathlineto{\pgfqpoint{4.769928in}{1.861561in}}%
\pgfpathlineto{\pgfqpoint{4.823941in}{1.930732in}}%
\pgfpathlineto{\pgfqpoint{4.878008in}{1.992355in}}%
\pgfpathlineto{\pgfqpoint{4.932322in}{1.982496in}}%
\pgfpathlineto{\pgfqpoint{4.986689in}{2.044327in}}%
\pgfpathlineto{\pgfqpoint{5.040889in}{2.004404in}}%
\pgfpathlineto{\pgfqpoint{5.095123in}{1.852473in}}%
\pgfpathlineto{\pgfqpoint{5.150626in}{1.909047in}}%
\pgfpathlineto{\pgfqpoint{5.204186in}{1.849352in}}%
\pgfpathlineto{\pgfqpoint{5.257755in}{1.857427in}}%
\pgfpathlineto{\pgfqpoint{5.312030in}{1.852162in}}%
\pgfpathlineto{\pgfqpoint{5.366183in}{1.839962in}}%
\pgfpathlineto{\pgfqpoint{5.420436in}{1.847948in}}%
\pgfpathlineto{\pgfqpoint{5.474424in}{1.824992in}}%
\pgfpathlineto{\pgfqpoint{5.529039in}{1.829114in}}%
\pgfusepath{stroke}%
\end{pgfscope}%
\begin{pgfscope}%
\pgfpathrectangle{\pgfqpoint{0.800000in}{0.528000in}}{\pgfqpoint{4.960000in}{3.696000in}}%
\pgfusepath{clip}%
\pgfsetrectcap%
\pgfsetroundjoin%
\pgfsetlinewidth{1.505625pt}%
\definecolor{currentstroke}{rgb}{0.839216,0.152941,0.156863}%
\pgfsetstrokecolor{currentstroke}%
\pgfsetdash{}{0pt}%
\pgfpathmoveto{\pgfqpoint{1.025455in}{4.056000in}}%
\pgfpathlineto{\pgfqpoint{1.079881in}{4.056000in}}%
\pgfpathlineto{\pgfqpoint{1.134963in}{4.056000in}}%
\pgfpathlineto{\pgfqpoint{1.188721in}{4.056000in}}%
\pgfpathlineto{\pgfqpoint{1.243225in}{4.056000in}}%
\pgfpathlineto{\pgfqpoint{1.296939in}{4.056000in}}%
\pgfpathlineto{\pgfqpoint{1.350896in}{4.056000in}}%
\pgfpathlineto{\pgfqpoint{1.405288in}{4.056000in}}%
\pgfpathlineto{\pgfqpoint{1.459395in}{4.056000in}}%
\pgfpathlineto{\pgfqpoint{1.513574in}{3.838790in}}%
\pgfpathlineto{\pgfqpoint{1.567839in}{3.110819in}}%
\pgfpathlineto{\pgfqpoint{1.622095in}{2.146293in}}%
\pgfpathlineto{\pgfqpoint{1.676134in}{1.520705in}}%
\pgfpathlineto{\pgfqpoint{1.730698in}{1.155829in}}%
\pgfpathlineto{\pgfqpoint{1.784436in}{1.321496in}}%
\pgfpathlineto{\pgfqpoint{1.838677in}{1.252308in}}%
\pgfpathlineto{\pgfqpoint{1.892956in}{1.134009in}}%
\pgfpathlineto{\pgfqpoint{1.947070in}{1.634318in}}%
\pgfpathlineto{\pgfqpoint{2.001787in}{1.739316in}}%
\pgfpathlineto{\pgfqpoint{2.055849in}{1.726628in}}%
\pgfpathlineto{\pgfqpoint{2.111433in}{2.034011in}}%
\pgfpathlineto{\pgfqpoint{2.164703in}{2.102534in}}%
\pgfpathlineto{\pgfqpoint{2.218534in}{2.066539in}}%
\pgfpathlineto{\pgfqpoint{2.272583in}{1.951580in}}%
\pgfpathlineto{\pgfqpoint{2.326829in}{1.922273in}}%
\pgfpathlineto{\pgfqpoint{2.381213in}{1.762595in}}%
\pgfpathlineto{\pgfqpoint{2.436582in}{1.600889in}}%
\pgfpathlineto{\pgfqpoint{2.490051in}{1.618431in}}%
\pgfpathlineto{\pgfqpoint{2.543806in}{1.613461in}}%
\pgfpathlineto{\pgfqpoint{2.598027in}{1.646107in}}%
\pgfpathlineto{\pgfqpoint{2.653426in}{1.713084in}}%
\pgfpathlineto{\pgfqpoint{2.709604in}{1.738928in}}%
\pgfpathlineto{\pgfqpoint{2.762579in}{1.830296in}}%
\pgfpathlineto{\pgfqpoint{2.816283in}{1.830098in}}%
\pgfpathlineto{\pgfqpoint{2.870528in}{1.876746in}}%
\pgfpathlineto{\pgfqpoint{2.924432in}{1.842539in}}%
\pgfpathlineto{\pgfqpoint{2.978859in}{1.828966in}}%
\pgfpathlineto{\pgfqpoint{3.032975in}{1.786690in}}%
\pgfpathlineto{\pgfqpoint{3.087194in}{1.746497in}}%
\pgfpathlineto{\pgfqpoint{3.141442in}{1.726225in}}%
\pgfpathlineto{\pgfqpoint{3.195694in}{1.709136in}}%
\pgfpathlineto{\pgfqpoint{3.251039in}{1.740992in}}%
\pgfpathlineto{\pgfqpoint{3.304192in}{1.762169in}}%
\pgfpathlineto{\pgfqpoint{3.360493in}{1.831767in}}%
\pgfpathlineto{\pgfqpoint{3.414658in}{2.070824in}}%
\pgfpathlineto{\pgfqpoint{3.468117in}{2.114837in}}%
\pgfpathlineto{\pgfqpoint{3.522313in}{1.997859in}}%
\pgfpathlineto{\pgfqpoint{3.576393in}{1.969120in}}%
\pgfpathlineto{\pgfqpoint{3.630565in}{2.038360in}}%
\pgfpathlineto{\pgfqpoint{3.685084in}{1.956417in}}%
\pgfpathlineto{\pgfqpoint{3.739274in}{2.074846in}}%
\pgfpathlineto{\pgfqpoint{3.793381in}{2.105511in}}%
\pgfpathlineto{\pgfqpoint{3.848141in}{2.046651in}}%
\pgfpathlineto{\pgfqpoint{3.902571in}{1.850478in}}%
\pgfpathlineto{\pgfqpoint{3.956728in}{1.924705in}}%
\pgfpathlineto{\pgfqpoint{4.012889in}{1.926878in}}%
\pgfpathlineto{\pgfqpoint{4.066488in}{1.956830in}}%
\pgfpathlineto{\pgfqpoint{4.120212in}{1.951155in}}%
\pgfpathlineto{\pgfqpoint{4.174560in}{1.948623in}}%
\pgfpathlineto{\pgfqpoint{4.228498in}{1.949463in}}%
\pgfpathlineto{\pgfqpoint{4.282538in}{1.932020in}}%
\pgfpathlineto{\pgfqpoint{4.337534in}{1.930385in}}%
\pgfpathlineto{\pgfqpoint{4.392155in}{1.919661in}}%
\pgfpathlineto{\pgfqpoint{4.445673in}{1.864953in}}%
\pgfpathlineto{\pgfqpoint{4.500420in}{1.847258in}}%
\pgfpathlineto{\pgfqpoint{4.554532in}{1.837294in}}%
\pgfpathlineto{\pgfqpoint{4.608789in}{1.871536in}}%
\pgfpathlineto{\pgfqpoint{4.664770in}{1.870731in}}%
\pgfpathlineto{\pgfqpoint{4.718654in}{1.884314in}}%
\pgfpathlineto{\pgfqpoint{4.772460in}{1.876906in}}%
\pgfpathlineto{\pgfqpoint{4.826649in}{1.873149in}}%
\pgfpathlineto{\pgfqpoint{4.880609in}{1.885240in}}%
\pgfpathlineto{\pgfqpoint{4.935201in}{1.891975in}}%
\pgfpathlineto{\pgfqpoint{4.989247in}{1.927772in}}%
\pgfpathlineto{\pgfqpoint{5.044090in}{1.944784in}}%
\pgfpathlineto{\pgfqpoint{5.098441in}{1.957274in}}%
\pgfpathlineto{\pgfqpoint{5.152543in}{1.954400in}}%
\pgfpathlineto{\pgfqpoint{5.207030in}{1.949412in}}%
\pgfpathlineto{\pgfqpoint{5.261657in}{1.939638in}}%
\pgfpathlineto{\pgfqpoint{5.316759in}{1.945056in}}%
\pgfpathlineto{\pgfqpoint{5.370349in}{1.947192in}}%
\pgfpathlineto{\pgfqpoint{5.424383in}{1.936497in}}%
\pgfpathlineto{\pgfqpoint{5.478327in}{1.943813in}}%
\pgfpathlineto{\pgfqpoint{5.532561in}{1.925165in}}%
\pgfusepath{stroke}%
\end{pgfscope}%
\begin{pgfscope}%
\pgfpathrectangle{\pgfqpoint{0.800000in}{0.528000in}}{\pgfqpoint{4.960000in}{3.696000in}}%
\pgfusepath{clip}%
\pgfsetrectcap%
\pgfsetroundjoin%
\pgfsetlinewidth{1.505625pt}%
\definecolor{currentstroke}{rgb}{0.580392,0.403922,0.741176}%
\pgfsetstrokecolor{currentstroke}%
\pgfsetdash{}{0pt}%
\pgfpathmoveto{\pgfqpoint{1.025455in}{4.056000in}}%
\pgfpathlineto{\pgfqpoint{1.079685in}{4.056000in}}%
\pgfpathlineto{\pgfqpoint{1.134019in}{4.056000in}}%
\pgfpathlineto{\pgfqpoint{1.188338in}{4.056000in}}%
\pgfpathlineto{\pgfqpoint{1.242589in}{4.056000in}}%
\pgfpathlineto{\pgfqpoint{1.298473in}{4.056000in}}%
\pgfpathlineto{\pgfqpoint{1.351996in}{4.056000in}}%
\pgfpathlineto{\pgfqpoint{1.405662in}{4.056000in}}%
\pgfpathlineto{\pgfqpoint{1.461685in}{4.056000in}}%
\pgfpathlineto{\pgfqpoint{1.514283in}{3.700301in}}%
\pgfpathlineto{\pgfqpoint{1.568549in}{2.371788in}}%
\pgfpathlineto{\pgfqpoint{1.622600in}{2.293222in}}%
\pgfpathlineto{\pgfqpoint{1.677494in}{1.687259in}}%
\pgfpathlineto{\pgfqpoint{1.731276in}{1.474482in}}%
\pgfpathlineto{\pgfqpoint{1.785717in}{1.275848in}}%
\pgfpathlineto{\pgfqpoint{1.840215in}{1.110704in}}%
\pgfpathlineto{\pgfqpoint{1.893984in}{1.349272in}}%
\pgfpathlineto{\pgfqpoint{1.950009in}{1.491011in}}%
\pgfpathlineto{\pgfqpoint{2.003228in}{1.410138in}}%
\pgfpathlineto{\pgfqpoint{2.056911in}{1.652914in}}%
\pgfpathlineto{\pgfqpoint{2.110897in}{1.839596in}}%
\pgfpathlineto{\pgfqpoint{2.165119in}{2.060613in}}%
\pgfpathlineto{\pgfqpoint{2.219214in}{2.323507in}}%
\pgfpathlineto{\pgfqpoint{2.273309in}{2.082489in}}%
\pgfpathlineto{\pgfqpoint{2.327612in}{2.006776in}}%
\pgfpathlineto{\pgfqpoint{2.381810in}{1.600769in}}%
\pgfpathlineto{\pgfqpoint{2.436125in}{1.703118in}}%
\pgfpathlineto{\pgfqpoint{2.490818in}{1.468522in}}%
\pgfpathlineto{\pgfqpoint{2.545023in}{1.625853in}}%
\pgfpathlineto{\pgfqpoint{2.600597in}{1.502473in}}%
\pgfpathlineto{\pgfqpoint{2.653910in}{1.889639in}}%
\pgfpathlineto{\pgfqpoint{2.708551in}{1.844850in}}%
\pgfpathlineto{\pgfqpoint{2.762889in}{1.878212in}}%
\pgfpathlineto{\pgfqpoint{2.817000in}{1.896188in}}%
\pgfpathlineto{\pgfqpoint{2.870990in}{1.891666in}}%
\pgfpathlineto{\pgfqpoint{2.925218in}{1.857727in}}%
\pgfpathlineto{\pgfqpoint{2.979819in}{1.831785in}}%
\pgfpathlineto{\pgfqpoint{3.033790in}{1.823168in}}%
\pgfpathlineto{\pgfqpoint{3.088414in}{1.859269in}}%
\pgfpathlineto{\pgfqpoint{3.143284in}{1.802210in}}%
\pgfpathlineto{\pgfqpoint{3.197383in}{1.823973in}}%
\pgfpathlineto{\pgfqpoint{3.253621in}{1.818143in}}%
\pgfpathlineto{\pgfqpoint{3.307062in}{1.917169in}}%
\pgfpathlineto{\pgfqpoint{3.361312in}{1.972713in}}%
\pgfpathlineto{\pgfqpoint{3.415498in}{2.016968in}}%
\pgfpathlineto{\pgfqpoint{3.469795in}{1.572459in}}%
\pgfpathlineto{\pgfqpoint{3.524031in}{1.675135in}}%
\pgfpathlineto{\pgfqpoint{3.578223in}{1.684313in}}%
\pgfpathlineto{\pgfqpoint{3.632373in}{2.010726in}}%
\pgfpathlineto{\pgfqpoint{3.686646in}{1.956539in}}%
\pgfpathlineto{\pgfqpoint{3.742368in}{2.027248in}}%
\pgfpathlineto{\pgfqpoint{3.796417in}{1.779124in}}%
\pgfpathlineto{\pgfqpoint{3.852565in}{1.949779in}}%
\pgfpathlineto{\pgfqpoint{3.906271in}{1.890497in}}%
\pgfpathlineto{\pgfqpoint{3.959733in}{1.842264in}}%
\pgfpathlineto{\pgfqpoint{4.014905in}{1.828246in}}%
\pgfpathlineto{\pgfqpoint{4.070133in}{1.831129in}}%
\pgfpathlineto{\pgfqpoint{4.122995in}{2.074885in}}%
\pgfpathlineto{\pgfqpoint{4.176361in}{2.099284in}}%
\pgfpathlineto{\pgfqpoint{4.230242in}{1.871708in}}%
\pgfpathlineto{\pgfqpoint{4.285143in}{1.969243in}}%
\pgfpathlineto{\pgfqpoint{4.338904in}{1.822685in}}%
\pgfpathlineto{\pgfqpoint{4.393070in}{1.902256in}}%
\pgfpathlineto{\pgfqpoint{4.447851in}{1.891831in}}%
\pgfpathlineto{\pgfqpoint{4.501903in}{1.881489in}}%
\pgfpathlineto{\pgfqpoint{4.555981in}{1.706421in}}%
\pgfpathlineto{\pgfqpoint{4.609946in}{1.895063in}}%
\pgfpathlineto{\pgfqpoint{4.665385in}{1.870730in}}%
\pgfpathlineto{\pgfqpoint{4.719020in}{1.744823in}}%
\pgfpathlineto{\pgfqpoint{4.772972in}{1.932161in}}%
\pgfpathlineto{\pgfqpoint{4.827057in}{2.113549in}}%
\pgfpathlineto{\pgfqpoint{4.881177in}{2.174425in}}%
\pgfpathlineto{\pgfqpoint{4.935645in}{2.165971in}}%
\pgfpathlineto{\pgfqpoint{4.989576in}{2.111081in}}%
\pgfpathlineto{\pgfqpoint{5.043959in}{2.154814in}}%
\pgfpathlineto{\pgfqpoint{5.098584in}{2.095753in}}%
\pgfpathlineto{\pgfqpoint{5.153157in}{1.761444in}}%
\pgfpathlineto{\pgfqpoint{5.207385in}{1.785637in}}%
\pgfpathlineto{\pgfqpoint{5.262338in}{1.792545in}}%
\pgfpathlineto{\pgfqpoint{5.316712in}{1.771226in}}%
\pgfpathlineto{\pgfqpoint{5.370845in}{1.757270in}}%
\pgfpathlineto{\pgfqpoint{5.424692in}{1.619519in}}%
\pgfpathlineto{\pgfqpoint{5.478861in}{1.758763in}}%
\pgfpathlineto{\pgfqpoint{5.533803in}{1.902475in}}%
\pgfusepath{stroke}%
\end{pgfscope}%
\begin{pgfscope}%
\pgfpathrectangle{\pgfqpoint{0.800000in}{0.528000in}}{\pgfqpoint{4.960000in}{3.696000in}}%
\pgfusepath{clip}%
\pgfsetrectcap%
\pgfsetroundjoin%
\pgfsetlinewidth{1.505625pt}%
\definecolor{currentstroke}{rgb}{0.549020,0.337255,0.294118}%
\pgfsetstrokecolor{currentstroke}%
\pgfsetdash{}{0pt}%
\pgfpathmoveto{\pgfqpoint{1.025455in}{4.056000in}}%
\pgfpathlineto{\pgfqpoint{1.080003in}{4.056000in}}%
\pgfpathlineto{\pgfqpoint{1.134006in}{4.056000in}}%
\pgfpathlineto{\pgfqpoint{1.188852in}{4.056000in}}%
\pgfpathlineto{\pgfqpoint{1.242782in}{4.056000in}}%
\pgfpathlineto{\pgfqpoint{1.296875in}{4.056000in}}%
\pgfpathlineto{\pgfqpoint{1.351200in}{4.056000in}}%
\pgfpathlineto{\pgfqpoint{1.405455in}{4.056000in}}%
\pgfpathlineto{\pgfqpoint{1.461112in}{4.056000in}}%
\pgfpathlineto{\pgfqpoint{1.514836in}{3.425405in}}%
\pgfpathlineto{\pgfqpoint{1.568352in}{2.440207in}}%
\pgfpathlineto{\pgfqpoint{1.622757in}{1.864595in}}%
\pgfpathlineto{\pgfqpoint{1.677115in}{1.271507in}}%
\pgfpathlineto{\pgfqpoint{1.731333in}{0.696000in}}%
\pgfpathlineto{\pgfqpoint{1.785498in}{0.888872in}}%
\pgfpathlineto{\pgfqpoint{1.839801in}{1.338932in}}%
\pgfpathlineto{\pgfqpoint{1.893801in}{1.400657in}}%
\pgfpathlineto{\pgfqpoint{1.948199in}{1.265897in}}%
\pgfpathlineto{\pgfqpoint{2.002289in}{2.062439in}}%
\pgfpathlineto{\pgfqpoint{2.056675in}{2.128239in}}%
\pgfpathlineto{\pgfqpoint{2.112439in}{2.272785in}}%
\pgfpathlineto{\pgfqpoint{2.166352in}{2.334737in}}%
\pgfpathlineto{\pgfqpoint{2.220134in}{2.283102in}}%
\pgfpathlineto{\pgfqpoint{2.274763in}{1.908021in}}%
\pgfpathlineto{\pgfqpoint{2.328718in}{1.773996in}}%
\pgfpathlineto{\pgfqpoint{2.384312in}{1.865917in}}%
\pgfpathlineto{\pgfqpoint{2.436920in}{1.897649in}}%
\pgfpathlineto{\pgfqpoint{2.491667in}{1.950083in}}%
\pgfpathlineto{\pgfqpoint{2.546354in}{1.973299in}}%
\pgfpathlineto{\pgfqpoint{2.600473in}{1.542098in}}%
\pgfpathlineto{\pgfqpoint{2.654786in}{1.799905in}}%
\pgfpathlineto{\pgfqpoint{2.710870in}{1.554477in}}%
\pgfpathlineto{\pgfqpoint{2.764412in}{2.084319in}}%
\pgfpathlineto{\pgfqpoint{2.818709in}{1.801019in}}%
\pgfpathlineto{\pgfqpoint{2.872258in}{1.803815in}}%
\pgfpathlineto{\pgfqpoint{2.926385in}{2.120156in}}%
\pgfpathlineto{\pgfqpoint{2.980763in}{2.207524in}}%
\pgfpathlineto{\pgfqpoint{3.035157in}{2.175509in}}%
\pgfpathlineto{\pgfqpoint{3.089043in}{2.208551in}}%
\pgfpathlineto{\pgfqpoint{3.143528in}{2.200574in}}%
\pgfpathlineto{\pgfqpoint{3.197726in}{2.293240in}}%
\pgfpathlineto{\pgfqpoint{3.251783in}{2.088877in}}%
\pgfpathlineto{\pgfqpoint{3.307894in}{2.061047in}}%
\pgfpathlineto{\pgfqpoint{3.361482in}{1.815829in}}%
\pgfpathlineto{\pgfqpoint{3.414922in}{1.744960in}}%
\pgfpathlineto{\pgfqpoint{3.468943in}{1.770706in}}%
\pgfpathlineto{\pgfqpoint{3.524326in}{1.759592in}}%
\pgfpathlineto{\pgfqpoint{3.578969in}{1.295930in}}%
\pgfpathlineto{\pgfqpoint{3.632337in}{1.506986in}}%
\pgfpathlineto{\pgfqpoint{3.686528in}{1.559839in}}%
\pgfpathlineto{\pgfqpoint{3.740532in}{2.049226in}}%
\pgfpathlineto{\pgfqpoint{3.795418in}{2.283730in}}%
\pgfpathlineto{\pgfqpoint{3.849306in}{2.284681in}}%
\pgfpathlineto{\pgfqpoint{3.903592in}{2.290677in}}%
\pgfpathlineto{\pgfqpoint{3.959568in}{2.116620in}}%
\pgfpathlineto{\pgfqpoint{4.013316in}{2.282006in}}%
\pgfpathlineto{\pgfqpoint{4.066662in}{2.286213in}}%
\pgfpathlineto{\pgfqpoint{4.120736in}{2.163870in}}%
\pgfpathlineto{\pgfqpoint{4.174802in}{1.085220in}}%
\pgfpathlineto{\pgfqpoint{4.228802in}{1.331424in}}%
\pgfpathlineto{\pgfqpoint{4.283279in}{1.402093in}}%
\pgfpathlineto{\pgfqpoint{4.339558in}{1.597154in}}%
\pgfpathlineto{\pgfqpoint{4.391161in}{1.800249in}}%
\pgfpathlineto{\pgfqpoint{4.445732in}{1.910471in}}%
\pgfpathlineto{\pgfqpoint{4.500189in}{1.992426in}}%
\pgfpathlineto{\pgfqpoint{4.554679in}{1.994959in}}%
\pgfpathlineto{\pgfqpoint{4.610520in}{2.100401in}}%
\pgfpathlineto{\pgfqpoint{4.664649in}{1.694803in}}%
\pgfpathlineto{\pgfqpoint{4.718277in}{1.815263in}}%
\pgfpathlineto{\pgfqpoint{4.772801in}{1.862655in}}%
\pgfpathlineto{\pgfqpoint{4.827168in}{2.089244in}}%
\pgfpathlineto{\pgfqpoint{4.881548in}{2.174213in}}%
\pgfpathlineto{\pgfqpoint{4.935527in}{2.135557in}}%
\pgfpathlineto{\pgfqpoint{4.989571in}{2.160449in}}%
\pgfpathlineto{\pgfqpoint{5.046423in}{1.887722in}}%
\pgfpathlineto{\pgfqpoint{5.099897in}{1.878741in}}%
\pgfpathlineto{\pgfqpoint{5.153971in}{1.793085in}}%
\pgfpathlineto{\pgfqpoint{5.209470in}{1.572476in}}%
\pgfpathlineto{\pgfqpoint{5.264467in}{1.334029in}}%
\pgfpathlineto{\pgfqpoint{5.320550in}{1.466370in}}%
\pgfpathlineto{\pgfqpoint{5.371851in}{1.682185in}}%
\pgfpathlineto{\pgfqpoint{5.426111in}{1.963446in}}%
\pgfpathlineto{\pgfqpoint{5.480046in}{1.908715in}}%
\pgfpathlineto{\pgfqpoint{5.534454in}{1.945516in}}%
\pgfusepath{stroke}%
\end{pgfscope}%
\begin{pgfscope}%
\pgfsetrectcap%
\pgfsetmiterjoin%
\pgfsetlinewidth{0.803000pt}%
\definecolor{currentstroke}{rgb}{0.000000,0.000000,0.000000}%
\pgfsetstrokecolor{currentstroke}%
\pgfsetdash{}{0pt}%
\pgfpathmoveto{\pgfqpoint{0.800000in}{0.528000in}}%
\pgfpathlineto{\pgfqpoint{0.800000in}{4.224000in}}%
\pgfusepath{stroke}%
\end{pgfscope}%
\begin{pgfscope}%
\pgfsetrectcap%
\pgfsetmiterjoin%
\pgfsetlinewidth{0.803000pt}%
\definecolor{currentstroke}{rgb}{0.000000,0.000000,0.000000}%
\pgfsetstrokecolor{currentstroke}%
\pgfsetdash{}{0pt}%
\pgfpathmoveto{\pgfqpoint{5.760000in}{0.528000in}}%
\pgfpathlineto{\pgfqpoint{5.760000in}{4.224000in}}%
\pgfusepath{stroke}%
\end{pgfscope}%
\begin{pgfscope}%
\pgfsetrectcap%
\pgfsetmiterjoin%
\pgfsetlinewidth{0.803000pt}%
\definecolor{currentstroke}{rgb}{0.000000,0.000000,0.000000}%
\pgfsetstrokecolor{currentstroke}%
\pgfsetdash{}{0pt}%
\pgfpathmoveto{\pgfqpoint{0.800000in}{0.528000in}}%
\pgfpathlineto{\pgfqpoint{5.760000in}{0.528000in}}%
\pgfusepath{stroke}%
\end{pgfscope}%
\begin{pgfscope}%
\pgfsetrectcap%
\pgfsetmiterjoin%
\pgfsetlinewidth{0.803000pt}%
\definecolor{currentstroke}{rgb}{0.000000,0.000000,0.000000}%
\pgfsetstrokecolor{currentstroke}%
\pgfsetdash{}{0pt}%
\pgfpathmoveto{\pgfqpoint{0.800000in}{4.224000in}}%
\pgfpathlineto{\pgfqpoint{5.760000in}{4.224000in}}%
\pgfusepath{stroke}%
\end{pgfscope}%
\begin{pgfscope}%
\definecolor{textcolor}{rgb}{0.000000,0.000000,0.000000}%
\pgfsetstrokecolor{textcolor}%
\pgfsetfillcolor{textcolor}%
\pgftext[x=3.280000in,y=4.307333in,,base]{\color{textcolor}\sffamily\fontsize{12.000000}{14.400000}\selectfont Yaw controller output}%
\end{pgfscope}%
\begin{pgfscope}%
\pgfsetbuttcap%
\pgfsetmiterjoin%
\definecolor{currentfill}{rgb}{1.000000,1.000000,1.000000}%
\pgfsetfillcolor{currentfill}%
\pgfsetfillopacity{0.800000}%
\pgfsetlinewidth{1.003750pt}%
\definecolor{currentstroke}{rgb}{0.800000,0.800000,0.800000}%
\pgfsetstrokecolor{currentstroke}%
\pgfsetstrokeopacity{0.800000}%
\pgfsetdash{}{0pt}%
\pgfpathmoveto{\pgfqpoint{5.041603in}{2.889746in}}%
\pgfpathlineto{\pgfqpoint{5.662778in}{2.889746in}}%
\pgfpathquadraticcurveto{\pgfqpoint{5.690556in}{2.889746in}}{\pgfqpoint{5.690556in}{2.917523in}}%
\pgfpathlineto{\pgfqpoint{5.690556in}{4.126778in}}%
\pgfpathquadraticcurveto{\pgfqpoint{5.690556in}{4.154556in}}{\pgfqpoint{5.662778in}{4.154556in}}%
\pgfpathlineto{\pgfqpoint{5.041603in}{4.154556in}}%
\pgfpathquadraticcurveto{\pgfqpoint{5.013825in}{4.154556in}}{\pgfqpoint{5.013825in}{4.126778in}}%
\pgfpathlineto{\pgfqpoint{5.013825in}{2.917523in}}%
\pgfpathquadraticcurveto{\pgfqpoint{5.013825in}{2.889746in}}{\pgfqpoint{5.041603in}{2.889746in}}%
\pgfpathlineto{\pgfqpoint{5.041603in}{2.889746in}}%
\pgfpathclose%
\pgfusepath{stroke,fill}%
\end{pgfscope}%
\begin{pgfscope}%
\pgfsetrectcap%
\pgfsetroundjoin%
\pgfsetlinewidth{1.505625pt}%
\definecolor{currentstroke}{rgb}{0.121569,0.466667,0.705882}%
\pgfsetstrokecolor{currentstroke}%
\pgfsetdash{}{0pt}%
\pgfpathmoveto{\pgfqpoint{5.069380in}{4.042088in}}%
\pgfpathlineto{\pgfqpoint{5.208269in}{4.042088in}}%
\pgfpathlineto{\pgfqpoint{5.347158in}{4.042088in}}%
\pgfusepath{stroke}%
\end{pgfscope}%
\begin{pgfscope}%
\definecolor{textcolor}{rgb}{0.000000,0.000000,0.000000}%
\pgfsetstrokecolor{textcolor}%
\pgfsetfillcolor{textcolor}%
\pgftext[x=5.458269in,y=3.993477in,left,base]{\color{textcolor}\sffamily\fontsize{10.000000}{12.000000}\selectfont 0}%
\end{pgfscope}%
\begin{pgfscope}%
\pgfsetrectcap%
\pgfsetroundjoin%
\pgfsetlinewidth{1.505625pt}%
\definecolor{currentstroke}{rgb}{1.000000,0.498039,0.054902}%
\pgfsetstrokecolor{currentstroke}%
\pgfsetdash{}{0pt}%
\pgfpathmoveto{\pgfqpoint{5.069380in}{3.838231in}}%
\pgfpathlineto{\pgfqpoint{5.208269in}{3.838231in}}%
\pgfpathlineto{\pgfqpoint{5.347158in}{3.838231in}}%
\pgfusepath{stroke}%
\end{pgfscope}%
\begin{pgfscope}%
\definecolor{textcolor}{rgb}{0.000000,0.000000,0.000000}%
\pgfsetstrokecolor{textcolor}%
\pgfsetfillcolor{textcolor}%
\pgftext[x=5.458269in,y=3.789620in,left,base]{\color{textcolor}\sffamily\fontsize{10.000000}{12.000000}\selectfont 5}%
\end{pgfscope}%
\begin{pgfscope}%
\pgfsetrectcap%
\pgfsetroundjoin%
\pgfsetlinewidth{1.505625pt}%
\definecolor{currentstroke}{rgb}{0.172549,0.627451,0.172549}%
\pgfsetstrokecolor{currentstroke}%
\pgfsetdash{}{0pt}%
\pgfpathmoveto{\pgfqpoint{5.069380in}{3.634374in}}%
\pgfpathlineto{\pgfqpoint{5.208269in}{3.634374in}}%
\pgfpathlineto{\pgfqpoint{5.347158in}{3.634374in}}%
\pgfusepath{stroke}%
\end{pgfscope}%
\begin{pgfscope}%
\definecolor{textcolor}{rgb}{0.000000,0.000000,0.000000}%
\pgfsetstrokecolor{textcolor}%
\pgfsetfillcolor{textcolor}%
\pgftext[x=5.458269in,y=3.585762in,left,base]{\color{textcolor}\sffamily\fontsize{10.000000}{12.000000}\selectfont 10}%
\end{pgfscope}%
\begin{pgfscope}%
\pgfsetrectcap%
\pgfsetroundjoin%
\pgfsetlinewidth{1.505625pt}%
\definecolor{currentstroke}{rgb}{0.839216,0.152941,0.156863}%
\pgfsetstrokecolor{currentstroke}%
\pgfsetdash{}{0pt}%
\pgfpathmoveto{\pgfqpoint{5.069380in}{3.430516in}}%
\pgfpathlineto{\pgfqpoint{5.208269in}{3.430516in}}%
\pgfpathlineto{\pgfqpoint{5.347158in}{3.430516in}}%
\pgfusepath{stroke}%
\end{pgfscope}%
\begin{pgfscope}%
\definecolor{textcolor}{rgb}{0.000000,0.000000,0.000000}%
\pgfsetstrokecolor{textcolor}%
\pgfsetfillcolor{textcolor}%
\pgftext[x=5.458269in,y=3.381905in,left,base]{\color{textcolor}\sffamily\fontsize{10.000000}{12.000000}\selectfont 20}%
\end{pgfscope}%
\begin{pgfscope}%
\pgfsetrectcap%
\pgfsetroundjoin%
\pgfsetlinewidth{1.505625pt}%
\definecolor{currentstroke}{rgb}{0.580392,0.403922,0.741176}%
\pgfsetstrokecolor{currentstroke}%
\pgfsetdash{}{0pt}%
\pgfpathmoveto{\pgfqpoint{5.069380in}{3.226659in}}%
\pgfpathlineto{\pgfqpoint{5.208269in}{3.226659in}}%
\pgfpathlineto{\pgfqpoint{5.347158in}{3.226659in}}%
\pgfusepath{stroke}%
\end{pgfscope}%
\begin{pgfscope}%
\definecolor{textcolor}{rgb}{0.000000,0.000000,0.000000}%
\pgfsetstrokecolor{textcolor}%
\pgfsetfillcolor{textcolor}%
\pgftext[x=5.458269in,y=3.178048in,left,base]{\color{textcolor}\sffamily\fontsize{10.000000}{12.000000}\selectfont 40}%
\end{pgfscope}%
\begin{pgfscope}%
\pgfsetrectcap%
\pgfsetroundjoin%
\pgfsetlinewidth{1.505625pt}%
\definecolor{currentstroke}{rgb}{0.549020,0.337255,0.294118}%
\pgfsetstrokecolor{currentstroke}%
\pgfsetdash{}{0pt}%
\pgfpathmoveto{\pgfqpoint{5.069380in}{3.022802in}}%
\pgfpathlineto{\pgfqpoint{5.208269in}{3.022802in}}%
\pgfpathlineto{\pgfqpoint{5.347158in}{3.022802in}}%
\pgfusepath{stroke}%
\end{pgfscope}%
\begin{pgfscope}%
\definecolor{textcolor}{rgb}{0.000000,0.000000,0.000000}%
\pgfsetstrokecolor{textcolor}%
\pgfsetfillcolor{textcolor}%
\pgftext[x=5.458269in,y=2.974191in,left,base]{\color{textcolor}\sffamily\fontsize{10.000000}{12.000000}\selectfont 80}%
\end{pgfscope}%
\end{pgfpicture}%
\makeatother%
\endgroup%
}
    \end{minipage}
    \caption{Variation of (a) computed error and (b) output velocity for different values of $K_{D}$ and $K_P=100$, $K_I=30$ while the yaw controller is engaged.}
    \label{fig:tune-yaw-der-io}
\end{figure}
\begin{figure}[H]
    \begin{minipage}[t]{0.5\linewidth}
        \centering
        \scalebox{0.55}{%% Creator: Matplotlib, PGF backend
%%
%% To include the figure in your LaTeX document, write
%%   \input{<filename>.pgf}
%%
%% Make sure the required packages are loaded in your preamble
%%   \usepackage{pgf}
%%
%% Also ensure that all the required font packages are loaded; for instance,
%% the lmodern package is sometimes necessary when using math font.
%%   \usepackage{lmodern}
%%
%% Figures using additional raster images can only be included by \input if
%% they are in the same directory as the main LaTeX file. For loading figures
%% from other directories you can use the `import` package
%%   \usepackage{import}
%%
%% and then include the figures with
%%   \import{<path to file>}{<filename>.pgf}
%%
%% Matplotlib used the following preamble
%%   \usepackage{fontspec}
%%   \setmainfont{DejaVuSerif.ttf}[Path=\detokenize{/home/lgonz/tfg-aero/tfg-giaa-dronecontrol/venv/lib/python3.8/site-packages/matplotlib/mpl-data/fonts/ttf/}]
%%   \setsansfont{DejaVuSans.ttf}[Path=\detokenize{/home/lgonz/tfg-aero/tfg-giaa-dronecontrol/venv/lib/python3.8/site-packages/matplotlib/mpl-data/fonts/ttf/}]
%%   \setmonofont{DejaVuSansMono.ttf}[Path=\detokenize{/home/lgonz/tfg-aero/tfg-giaa-dronecontrol/venv/lib/python3.8/site-packages/matplotlib/mpl-data/fonts/ttf/}]
%%
\begingroup%
\makeatletter%
\begin{pgfpicture}%
\pgfpathrectangle{\pgfpointorigin}{\pgfqpoint{6.400000in}{4.800000in}}%
\pgfusepath{use as bounding box, clip}%
\begin{pgfscope}%
\pgfsetbuttcap%
\pgfsetmiterjoin%
\definecolor{currentfill}{rgb}{1.000000,1.000000,1.000000}%
\pgfsetfillcolor{currentfill}%
\pgfsetlinewidth{0.000000pt}%
\definecolor{currentstroke}{rgb}{1.000000,1.000000,1.000000}%
\pgfsetstrokecolor{currentstroke}%
\pgfsetdash{}{0pt}%
\pgfpathmoveto{\pgfqpoint{0.000000in}{0.000000in}}%
\pgfpathlineto{\pgfqpoint{6.400000in}{0.000000in}}%
\pgfpathlineto{\pgfqpoint{6.400000in}{4.800000in}}%
\pgfpathlineto{\pgfqpoint{0.000000in}{4.800000in}}%
\pgfpathlineto{\pgfqpoint{0.000000in}{0.000000in}}%
\pgfpathclose%
\pgfusepath{fill}%
\end{pgfscope}%
\begin{pgfscope}%
\pgfsetbuttcap%
\pgfsetmiterjoin%
\definecolor{currentfill}{rgb}{1.000000,1.000000,1.000000}%
\pgfsetfillcolor{currentfill}%
\pgfsetlinewidth{0.000000pt}%
\definecolor{currentstroke}{rgb}{0.000000,0.000000,0.000000}%
\pgfsetstrokecolor{currentstroke}%
\pgfsetstrokeopacity{0.000000}%
\pgfsetdash{}{0pt}%
\pgfpathmoveto{\pgfqpoint{0.800000in}{0.528000in}}%
\pgfpathlineto{\pgfqpoint{5.760000in}{0.528000in}}%
\pgfpathlineto{\pgfqpoint{5.760000in}{4.224000in}}%
\pgfpathlineto{\pgfqpoint{0.800000in}{4.224000in}}%
\pgfpathlineto{\pgfqpoint{0.800000in}{0.528000in}}%
\pgfpathclose%
\pgfusepath{fill}%
\end{pgfscope}%
\begin{pgfscope}%
\pgfpathrectangle{\pgfqpoint{0.800000in}{0.528000in}}{\pgfqpoint{4.960000in}{3.696000in}}%
\pgfusepath{clip}%
\pgfsetrectcap%
\pgfsetroundjoin%
\pgfsetlinewidth{0.803000pt}%
\definecolor{currentstroke}{rgb}{0.690196,0.690196,0.690196}%
\pgfsetstrokecolor{currentstroke}%
\pgfsetdash{}{0pt}%
\pgfpathmoveto{\pgfqpoint{1.025455in}{0.528000in}}%
\pgfpathlineto{\pgfqpoint{1.025455in}{4.224000in}}%
\pgfusepath{stroke}%
\end{pgfscope}%
\begin{pgfscope}%
\pgfsetbuttcap%
\pgfsetroundjoin%
\definecolor{currentfill}{rgb}{0.000000,0.000000,0.000000}%
\pgfsetfillcolor{currentfill}%
\pgfsetlinewidth{0.803000pt}%
\definecolor{currentstroke}{rgb}{0.000000,0.000000,0.000000}%
\pgfsetstrokecolor{currentstroke}%
\pgfsetdash{}{0pt}%
\pgfsys@defobject{currentmarker}{\pgfqpoint{0.000000in}{-0.048611in}}{\pgfqpoint{0.000000in}{0.000000in}}{%
\pgfpathmoveto{\pgfqpoint{0.000000in}{0.000000in}}%
\pgfpathlineto{\pgfqpoint{0.000000in}{-0.048611in}}%
\pgfusepath{stroke,fill}%
}%
\begin{pgfscope}%
\pgfsys@transformshift{1.025455in}{0.528000in}%
\pgfsys@useobject{currentmarker}{}%
\end{pgfscope}%
\end{pgfscope}%
\begin{pgfscope}%
\definecolor{textcolor}{rgb}{0.000000,0.000000,0.000000}%
\pgfsetstrokecolor{textcolor}%
\pgfsetfillcolor{textcolor}%
\pgftext[x=1.025455in,y=0.430778in,,top]{\color{textcolor}\sffamily\fontsize{10.000000}{12.000000}\selectfont 0}%
\end{pgfscope}%
\begin{pgfscope}%
\pgfpathrectangle{\pgfqpoint{0.800000in}{0.528000in}}{\pgfqpoint{4.960000in}{3.696000in}}%
\pgfusepath{clip}%
\pgfsetrectcap%
\pgfsetroundjoin%
\pgfsetlinewidth{0.803000pt}%
\definecolor{currentstroke}{rgb}{0.690196,0.690196,0.690196}%
\pgfsetstrokecolor{currentstroke}%
\pgfsetdash{}{0pt}%
\pgfpathmoveto{\pgfqpoint{1.775826in}{0.528000in}}%
\pgfpathlineto{\pgfqpoint{1.775826in}{4.224000in}}%
\pgfusepath{stroke}%
\end{pgfscope}%
\begin{pgfscope}%
\pgfsetbuttcap%
\pgfsetroundjoin%
\definecolor{currentfill}{rgb}{0.000000,0.000000,0.000000}%
\pgfsetfillcolor{currentfill}%
\pgfsetlinewidth{0.803000pt}%
\definecolor{currentstroke}{rgb}{0.000000,0.000000,0.000000}%
\pgfsetstrokecolor{currentstroke}%
\pgfsetdash{}{0pt}%
\pgfsys@defobject{currentmarker}{\pgfqpoint{0.000000in}{-0.048611in}}{\pgfqpoint{0.000000in}{0.000000in}}{%
\pgfpathmoveto{\pgfqpoint{0.000000in}{0.000000in}}%
\pgfpathlineto{\pgfqpoint{0.000000in}{-0.048611in}}%
\pgfusepath{stroke,fill}%
}%
\begin{pgfscope}%
\pgfsys@transformshift{1.775826in}{0.528000in}%
\pgfsys@useobject{currentmarker}{}%
\end{pgfscope}%
\end{pgfscope}%
\begin{pgfscope}%
\definecolor{textcolor}{rgb}{0.000000,0.000000,0.000000}%
\pgfsetstrokecolor{textcolor}%
\pgfsetfillcolor{textcolor}%
\pgftext[x=1.775826in,y=0.430778in,,top]{\color{textcolor}\sffamily\fontsize{10.000000}{12.000000}\selectfont 5}%
\end{pgfscope}%
\begin{pgfscope}%
\pgfpathrectangle{\pgfqpoint{0.800000in}{0.528000in}}{\pgfqpoint{4.960000in}{3.696000in}}%
\pgfusepath{clip}%
\pgfsetrectcap%
\pgfsetroundjoin%
\pgfsetlinewidth{0.803000pt}%
\definecolor{currentstroke}{rgb}{0.690196,0.690196,0.690196}%
\pgfsetstrokecolor{currentstroke}%
\pgfsetdash{}{0pt}%
\pgfpathmoveto{\pgfqpoint{2.526198in}{0.528000in}}%
\pgfpathlineto{\pgfqpoint{2.526198in}{4.224000in}}%
\pgfusepath{stroke}%
\end{pgfscope}%
\begin{pgfscope}%
\pgfsetbuttcap%
\pgfsetroundjoin%
\definecolor{currentfill}{rgb}{0.000000,0.000000,0.000000}%
\pgfsetfillcolor{currentfill}%
\pgfsetlinewidth{0.803000pt}%
\definecolor{currentstroke}{rgb}{0.000000,0.000000,0.000000}%
\pgfsetstrokecolor{currentstroke}%
\pgfsetdash{}{0pt}%
\pgfsys@defobject{currentmarker}{\pgfqpoint{0.000000in}{-0.048611in}}{\pgfqpoint{0.000000in}{0.000000in}}{%
\pgfpathmoveto{\pgfqpoint{0.000000in}{0.000000in}}%
\pgfpathlineto{\pgfqpoint{0.000000in}{-0.048611in}}%
\pgfusepath{stroke,fill}%
}%
\begin{pgfscope}%
\pgfsys@transformshift{2.526198in}{0.528000in}%
\pgfsys@useobject{currentmarker}{}%
\end{pgfscope}%
\end{pgfscope}%
\begin{pgfscope}%
\definecolor{textcolor}{rgb}{0.000000,0.000000,0.000000}%
\pgfsetstrokecolor{textcolor}%
\pgfsetfillcolor{textcolor}%
\pgftext[x=2.526198in,y=0.430778in,,top]{\color{textcolor}\sffamily\fontsize{10.000000}{12.000000}\selectfont 10}%
\end{pgfscope}%
\begin{pgfscope}%
\pgfpathrectangle{\pgfqpoint{0.800000in}{0.528000in}}{\pgfqpoint{4.960000in}{3.696000in}}%
\pgfusepath{clip}%
\pgfsetrectcap%
\pgfsetroundjoin%
\pgfsetlinewidth{0.803000pt}%
\definecolor{currentstroke}{rgb}{0.690196,0.690196,0.690196}%
\pgfsetstrokecolor{currentstroke}%
\pgfsetdash{}{0pt}%
\pgfpathmoveto{\pgfqpoint{3.276570in}{0.528000in}}%
\pgfpathlineto{\pgfqpoint{3.276570in}{4.224000in}}%
\pgfusepath{stroke}%
\end{pgfscope}%
\begin{pgfscope}%
\pgfsetbuttcap%
\pgfsetroundjoin%
\definecolor{currentfill}{rgb}{0.000000,0.000000,0.000000}%
\pgfsetfillcolor{currentfill}%
\pgfsetlinewidth{0.803000pt}%
\definecolor{currentstroke}{rgb}{0.000000,0.000000,0.000000}%
\pgfsetstrokecolor{currentstroke}%
\pgfsetdash{}{0pt}%
\pgfsys@defobject{currentmarker}{\pgfqpoint{0.000000in}{-0.048611in}}{\pgfqpoint{0.000000in}{0.000000in}}{%
\pgfpathmoveto{\pgfqpoint{0.000000in}{0.000000in}}%
\pgfpathlineto{\pgfqpoint{0.000000in}{-0.048611in}}%
\pgfusepath{stroke,fill}%
}%
\begin{pgfscope}%
\pgfsys@transformshift{3.276570in}{0.528000in}%
\pgfsys@useobject{currentmarker}{}%
\end{pgfscope}%
\end{pgfscope}%
\begin{pgfscope}%
\definecolor{textcolor}{rgb}{0.000000,0.000000,0.000000}%
\pgfsetstrokecolor{textcolor}%
\pgfsetfillcolor{textcolor}%
\pgftext[x=3.276570in,y=0.430778in,,top]{\color{textcolor}\sffamily\fontsize{10.000000}{12.000000}\selectfont 15}%
\end{pgfscope}%
\begin{pgfscope}%
\pgfpathrectangle{\pgfqpoint{0.800000in}{0.528000in}}{\pgfqpoint{4.960000in}{3.696000in}}%
\pgfusepath{clip}%
\pgfsetrectcap%
\pgfsetroundjoin%
\pgfsetlinewidth{0.803000pt}%
\definecolor{currentstroke}{rgb}{0.690196,0.690196,0.690196}%
\pgfsetstrokecolor{currentstroke}%
\pgfsetdash{}{0pt}%
\pgfpathmoveto{\pgfqpoint{4.026941in}{0.528000in}}%
\pgfpathlineto{\pgfqpoint{4.026941in}{4.224000in}}%
\pgfusepath{stroke}%
\end{pgfscope}%
\begin{pgfscope}%
\pgfsetbuttcap%
\pgfsetroundjoin%
\definecolor{currentfill}{rgb}{0.000000,0.000000,0.000000}%
\pgfsetfillcolor{currentfill}%
\pgfsetlinewidth{0.803000pt}%
\definecolor{currentstroke}{rgb}{0.000000,0.000000,0.000000}%
\pgfsetstrokecolor{currentstroke}%
\pgfsetdash{}{0pt}%
\pgfsys@defobject{currentmarker}{\pgfqpoint{0.000000in}{-0.048611in}}{\pgfqpoint{0.000000in}{0.000000in}}{%
\pgfpathmoveto{\pgfqpoint{0.000000in}{0.000000in}}%
\pgfpathlineto{\pgfqpoint{0.000000in}{-0.048611in}}%
\pgfusepath{stroke,fill}%
}%
\begin{pgfscope}%
\pgfsys@transformshift{4.026941in}{0.528000in}%
\pgfsys@useobject{currentmarker}{}%
\end{pgfscope}%
\end{pgfscope}%
\begin{pgfscope}%
\definecolor{textcolor}{rgb}{0.000000,0.000000,0.000000}%
\pgfsetstrokecolor{textcolor}%
\pgfsetfillcolor{textcolor}%
\pgftext[x=4.026941in,y=0.430778in,,top]{\color{textcolor}\sffamily\fontsize{10.000000}{12.000000}\selectfont 20}%
\end{pgfscope}%
\begin{pgfscope}%
\pgfpathrectangle{\pgfqpoint{0.800000in}{0.528000in}}{\pgfqpoint{4.960000in}{3.696000in}}%
\pgfusepath{clip}%
\pgfsetrectcap%
\pgfsetroundjoin%
\pgfsetlinewidth{0.803000pt}%
\definecolor{currentstroke}{rgb}{0.690196,0.690196,0.690196}%
\pgfsetstrokecolor{currentstroke}%
\pgfsetdash{}{0pt}%
\pgfpathmoveto{\pgfqpoint{4.777313in}{0.528000in}}%
\pgfpathlineto{\pgfqpoint{4.777313in}{4.224000in}}%
\pgfusepath{stroke}%
\end{pgfscope}%
\begin{pgfscope}%
\pgfsetbuttcap%
\pgfsetroundjoin%
\definecolor{currentfill}{rgb}{0.000000,0.000000,0.000000}%
\pgfsetfillcolor{currentfill}%
\pgfsetlinewidth{0.803000pt}%
\definecolor{currentstroke}{rgb}{0.000000,0.000000,0.000000}%
\pgfsetstrokecolor{currentstroke}%
\pgfsetdash{}{0pt}%
\pgfsys@defobject{currentmarker}{\pgfqpoint{0.000000in}{-0.048611in}}{\pgfqpoint{0.000000in}{0.000000in}}{%
\pgfpathmoveto{\pgfqpoint{0.000000in}{0.000000in}}%
\pgfpathlineto{\pgfqpoint{0.000000in}{-0.048611in}}%
\pgfusepath{stroke,fill}%
}%
\begin{pgfscope}%
\pgfsys@transformshift{4.777313in}{0.528000in}%
\pgfsys@useobject{currentmarker}{}%
\end{pgfscope}%
\end{pgfscope}%
\begin{pgfscope}%
\definecolor{textcolor}{rgb}{0.000000,0.000000,0.000000}%
\pgfsetstrokecolor{textcolor}%
\pgfsetfillcolor{textcolor}%
\pgftext[x=4.777313in,y=0.430778in,,top]{\color{textcolor}\sffamily\fontsize{10.000000}{12.000000}\selectfont 25}%
\end{pgfscope}%
\begin{pgfscope}%
\pgfpathrectangle{\pgfqpoint{0.800000in}{0.528000in}}{\pgfqpoint{4.960000in}{3.696000in}}%
\pgfusepath{clip}%
\pgfsetrectcap%
\pgfsetroundjoin%
\pgfsetlinewidth{0.803000pt}%
\definecolor{currentstroke}{rgb}{0.690196,0.690196,0.690196}%
\pgfsetstrokecolor{currentstroke}%
\pgfsetdash{}{0pt}%
\pgfpathmoveto{\pgfqpoint{5.527685in}{0.528000in}}%
\pgfpathlineto{\pgfqpoint{5.527685in}{4.224000in}}%
\pgfusepath{stroke}%
\end{pgfscope}%
\begin{pgfscope}%
\pgfsetbuttcap%
\pgfsetroundjoin%
\definecolor{currentfill}{rgb}{0.000000,0.000000,0.000000}%
\pgfsetfillcolor{currentfill}%
\pgfsetlinewidth{0.803000pt}%
\definecolor{currentstroke}{rgb}{0.000000,0.000000,0.000000}%
\pgfsetstrokecolor{currentstroke}%
\pgfsetdash{}{0pt}%
\pgfsys@defobject{currentmarker}{\pgfqpoint{0.000000in}{-0.048611in}}{\pgfqpoint{0.000000in}{0.000000in}}{%
\pgfpathmoveto{\pgfqpoint{0.000000in}{0.000000in}}%
\pgfpathlineto{\pgfqpoint{0.000000in}{-0.048611in}}%
\pgfusepath{stroke,fill}%
}%
\begin{pgfscope}%
\pgfsys@transformshift{5.527685in}{0.528000in}%
\pgfsys@useobject{currentmarker}{}%
\end{pgfscope}%
\end{pgfscope}%
\begin{pgfscope}%
\definecolor{textcolor}{rgb}{0.000000,0.000000,0.000000}%
\pgfsetstrokecolor{textcolor}%
\pgfsetfillcolor{textcolor}%
\pgftext[x=5.527685in,y=0.430778in,,top]{\color{textcolor}\sffamily\fontsize{10.000000}{12.000000}\selectfont 30}%
\end{pgfscope}%
\begin{pgfscope}%
\definecolor{textcolor}{rgb}{0.000000,0.000000,0.000000}%
\pgfsetstrokecolor{textcolor}%
\pgfsetfillcolor{textcolor}%
\pgftext[x=3.280000in,y=0.240809in,,top]{\color{textcolor}\sffamily\fontsize{10.000000}{12.000000}\selectfont time [s]}%
\end{pgfscope}%
\begin{pgfscope}%
\pgfpathrectangle{\pgfqpoint{0.800000in}{0.528000in}}{\pgfqpoint{4.960000in}{3.696000in}}%
\pgfusepath{clip}%
\pgfsetrectcap%
\pgfsetroundjoin%
\pgfsetlinewidth{0.803000pt}%
\definecolor{currentstroke}{rgb}{0.690196,0.690196,0.690196}%
\pgfsetstrokecolor{currentstroke}%
\pgfsetdash{}{0pt}%
\pgfpathmoveto{\pgfqpoint{0.800000in}{0.789533in}}%
\pgfpathlineto{\pgfqpoint{5.760000in}{0.789533in}}%
\pgfusepath{stroke}%
\end{pgfscope}%
\begin{pgfscope}%
\pgfsetbuttcap%
\pgfsetroundjoin%
\definecolor{currentfill}{rgb}{0.000000,0.000000,0.000000}%
\pgfsetfillcolor{currentfill}%
\pgfsetlinewidth{0.803000pt}%
\definecolor{currentstroke}{rgb}{0.000000,0.000000,0.000000}%
\pgfsetstrokecolor{currentstroke}%
\pgfsetdash{}{0pt}%
\pgfsys@defobject{currentmarker}{\pgfqpoint{-0.048611in}{0.000000in}}{\pgfqpoint{-0.000000in}{0.000000in}}{%
\pgfpathmoveto{\pgfqpoint{-0.000000in}{0.000000in}}%
\pgfpathlineto{\pgfqpoint{-0.048611in}{0.000000in}}%
\pgfusepath{stroke,fill}%
}%
\begin{pgfscope}%
\pgfsys@transformshift{0.800000in}{0.789533in}%
\pgfsys@useobject{currentmarker}{}%
\end{pgfscope}%
\end{pgfscope}%
\begin{pgfscope}%
\definecolor{textcolor}{rgb}{0.000000,0.000000,0.000000}%
\pgfsetstrokecolor{textcolor}%
\pgfsetfillcolor{textcolor}%
\pgftext[x=0.481898in, y=0.736772in, left, base]{\color{textcolor}\sffamily\fontsize{10.000000}{12.000000}\selectfont 7.5}%
\end{pgfscope}%
\begin{pgfscope}%
\pgfpathrectangle{\pgfqpoint{0.800000in}{0.528000in}}{\pgfqpoint{4.960000in}{3.696000in}}%
\pgfusepath{clip}%
\pgfsetrectcap%
\pgfsetroundjoin%
\pgfsetlinewidth{0.803000pt}%
\definecolor{currentstroke}{rgb}{0.690196,0.690196,0.690196}%
\pgfsetstrokecolor{currentstroke}%
\pgfsetdash{}{0pt}%
\pgfpathmoveto{\pgfqpoint{0.800000in}{1.239212in}}%
\pgfpathlineto{\pgfqpoint{5.760000in}{1.239212in}}%
\pgfusepath{stroke}%
\end{pgfscope}%
\begin{pgfscope}%
\pgfsetbuttcap%
\pgfsetroundjoin%
\definecolor{currentfill}{rgb}{0.000000,0.000000,0.000000}%
\pgfsetfillcolor{currentfill}%
\pgfsetlinewidth{0.803000pt}%
\definecolor{currentstroke}{rgb}{0.000000,0.000000,0.000000}%
\pgfsetstrokecolor{currentstroke}%
\pgfsetdash{}{0pt}%
\pgfsys@defobject{currentmarker}{\pgfqpoint{-0.048611in}{0.000000in}}{\pgfqpoint{-0.000000in}{0.000000in}}{%
\pgfpathmoveto{\pgfqpoint{-0.000000in}{0.000000in}}%
\pgfpathlineto{\pgfqpoint{-0.048611in}{0.000000in}}%
\pgfusepath{stroke,fill}%
}%
\begin{pgfscope}%
\pgfsys@transformshift{0.800000in}{1.239212in}%
\pgfsys@useobject{currentmarker}{}%
\end{pgfscope}%
\end{pgfscope}%
\begin{pgfscope}%
\definecolor{textcolor}{rgb}{0.000000,0.000000,0.000000}%
\pgfsetstrokecolor{textcolor}%
\pgfsetfillcolor{textcolor}%
\pgftext[x=0.393533in, y=1.186450in, left, base]{\color{textcolor}\sffamily\fontsize{10.000000}{12.000000}\selectfont 10.0}%
\end{pgfscope}%
\begin{pgfscope}%
\pgfpathrectangle{\pgfqpoint{0.800000in}{0.528000in}}{\pgfqpoint{4.960000in}{3.696000in}}%
\pgfusepath{clip}%
\pgfsetrectcap%
\pgfsetroundjoin%
\pgfsetlinewidth{0.803000pt}%
\definecolor{currentstroke}{rgb}{0.690196,0.690196,0.690196}%
\pgfsetstrokecolor{currentstroke}%
\pgfsetdash{}{0pt}%
\pgfpathmoveto{\pgfqpoint{0.800000in}{1.688891in}}%
\pgfpathlineto{\pgfqpoint{5.760000in}{1.688891in}}%
\pgfusepath{stroke}%
\end{pgfscope}%
\begin{pgfscope}%
\pgfsetbuttcap%
\pgfsetroundjoin%
\definecolor{currentfill}{rgb}{0.000000,0.000000,0.000000}%
\pgfsetfillcolor{currentfill}%
\pgfsetlinewidth{0.803000pt}%
\definecolor{currentstroke}{rgb}{0.000000,0.000000,0.000000}%
\pgfsetstrokecolor{currentstroke}%
\pgfsetdash{}{0pt}%
\pgfsys@defobject{currentmarker}{\pgfqpoint{-0.048611in}{0.000000in}}{\pgfqpoint{-0.000000in}{0.000000in}}{%
\pgfpathmoveto{\pgfqpoint{-0.000000in}{0.000000in}}%
\pgfpathlineto{\pgfqpoint{-0.048611in}{0.000000in}}%
\pgfusepath{stroke,fill}%
}%
\begin{pgfscope}%
\pgfsys@transformshift{0.800000in}{1.688891in}%
\pgfsys@useobject{currentmarker}{}%
\end{pgfscope}%
\end{pgfscope}%
\begin{pgfscope}%
\definecolor{textcolor}{rgb}{0.000000,0.000000,0.000000}%
\pgfsetstrokecolor{textcolor}%
\pgfsetfillcolor{textcolor}%
\pgftext[x=0.393533in, y=1.636129in, left, base]{\color{textcolor}\sffamily\fontsize{10.000000}{12.000000}\selectfont 12.5}%
\end{pgfscope}%
\begin{pgfscope}%
\pgfpathrectangle{\pgfqpoint{0.800000in}{0.528000in}}{\pgfqpoint{4.960000in}{3.696000in}}%
\pgfusepath{clip}%
\pgfsetrectcap%
\pgfsetroundjoin%
\pgfsetlinewidth{0.803000pt}%
\definecolor{currentstroke}{rgb}{0.690196,0.690196,0.690196}%
\pgfsetstrokecolor{currentstroke}%
\pgfsetdash{}{0pt}%
\pgfpathmoveto{\pgfqpoint{0.800000in}{2.138570in}}%
\pgfpathlineto{\pgfqpoint{5.760000in}{2.138570in}}%
\pgfusepath{stroke}%
\end{pgfscope}%
\begin{pgfscope}%
\pgfsetbuttcap%
\pgfsetroundjoin%
\definecolor{currentfill}{rgb}{0.000000,0.000000,0.000000}%
\pgfsetfillcolor{currentfill}%
\pgfsetlinewidth{0.803000pt}%
\definecolor{currentstroke}{rgb}{0.000000,0.000000,0.000000}%
\pgfsetstrokecolor{currentstroke}%
\pgfsetdash{}{0pt}%
\pgfsys@defobject{currentmarker}{\pgfqpoint{-0.048611in}{0.000000in}}{\pgfqpoint{-0.000000in}{0.000000in}}{%
\pgfpathmoveto{\pgfqpoint{-0.000000in}{0.000000in}}%
\pgfpathlineto{\pgfqpoint{-0.048611in}{0.000000in}}%
\pgfusepath{stroke,fill}%
}%
\begin{pgfscope}%
\pgfsys@transformshift{0.800000in}{2.138570in}%
\pgfsys@useobject{currentmarker}{}%
\end{pgfscope}%
\end{pgfscope}%
\begin{pgfscope}%
\definecolor{textcolor}{rgb}{0.000000,0.000000,0.000000}%
\pgfsetstrokecolor{textcolor}%
\pgfsetfillcolor{textcolor}%
\pgftext[x=0.393533in, y=2.085808in, left, base]{\color{textcolor}\sffamily\fontsize{10.000000}{12.000000}\selectfont 15.0}%
\end{pgfscope}%
\begin{pgfscope}%
\pgfpathrectangle{\pgfqpoint{0.800000in}{0.528000in}}{\pgfqpoint{4.960000in}{3.696000in}}%
\pgfusepath{clip}%
\pgfsetrectcap%
\pgfsetroundjoin%
\pgfsetlinewidth{0.803000pt}%
\definecolor{currentstroke}{rgb}{0.690196,0.690196,0.690196}%
\pgfsetstrokecolor{currentstroke}%
\pgfsetdash{}{0pt}%
\pgfpathmoveto{\pgfqpoint{0.800000in}{2.588248in}}%
\pgfpathlineto{\pgfqpoint{5.760000in}{2.588248in}}%
\pgfusepath{stroke}%
\end{pgfscope}%
\begin{pgfscope}%
\pgfsetbuttcap%
\pgfsetroundjoin%
\definecolor{currentfill}{rgb}{0.000000,0.000000,0.000000}%
\pgfsetfillcolor{currentfill}%
\pgfsetlinewidth{0.803000pt}%
\definecolor{currentstroke}{rgb}{0.000000,0.000000,0.000000}%
\pgfsetstrokecolor{currentstroke}%
\pgfsetdash{}{0pt}%
\pgfsys@defobject{currentmarker}{\pgfqpoint{-0.048611in}{0.000000in}}{\pgfqpoint{-0.000000in}{0.000000in}}{%
\pgfpathmoveto{\pgfqpoint{-0.000000in}{0.000000in}}%
\pgfpathlineto{\pgfqpoint{-0.048611in}{0.000000in}}%
\pgfusepath{stroke,fill}%
}%
\begin{pgfscope}%
\pgfsys@transformshift{0.800000in}{2.588248in}%
\pgfsys@useobject{currentmarker}{}%
\end{pgfscope}%
\end{pgfscope}%
\begin{pgfscope}%
\definecolor{textcolor}{rgb}{0.000000,0.000000,0.000000}%
\pgfsetstrokecolor{textcolor}%
\pgfsetfillcolor{textcolor}%
\pgftext[x=0.393533in, y=2.535487in, left, base]{\color{textcolor}\sffamily\fontsize{10.000000}{12.000000}\selectfont 17.5}%
\end{pgfscope}%
\begin{pgfscope}%
\pgfpathrectangle{\pgfqpoint{0.800000in}{0.528000in}}{\pgfqpoint{4.960000in}{3.696000in}}%
\pgfusepath{clip}%
\pgfsetrectcap%
\pgfsetroundjoin%
\pgfsetlinewidth{0.803000pt}%
\definecolor{currentstroke}{rgb}{0.690196,0.690196,0.690196}%
\pgfsetstrokecolor{currentstroke}%
\pgfsetdash{}{0pt}%
\pgfpathmoveto{\pgfqpoint{0.800000in}{3.037927in}}%
\pgfpathlineto{\pgfqpoint{5.760000in}{3.037927in}}%
\pgfusepath{stroke}%
\end{pgfscope}%
\begin{pgfscope}%
\pgfsetbuttcap%
\pgfsetroundjoin%
\definecolor{currentfill}{rgb}{0.000000,0.000000,0.000000}%
\pgfsetfillcolor{currentfill}%
\pgfsetlinewidth{0.803000pt}%
\definecolor{currentstroke}{rgb}{0.000000,0.000000,0.000000}%
\pgfsetstrokecolor{currentstroke}%
\pgfsetdash{}{0pt}%
\pgfsys@defobject{currentmarker}{\pgfqpoint{-0.048611in}{0.000000in}}{\pgfqpoint{-0.000000in}{0.000000in}}{%
\pgfpathmoveto{\pgfqpoint{-0.000000in}{0.000000in}}%
\pgfpathlineto{\pgfqpoint{-0.048611in}{0.000000in}}%
\pgfusepath{stroke,fill}%
}%
\begin{pgfscope}%
\pgfsys@transformshift{0.800000in}{3.037927in}%
\pgfsys@useobject{currentmarker}{}%
\end{pgfscope}%
\end{pgfscope}%
\begin{pgfscope}%
\definecolor{textcolor}{rgb}{0.000000,0.000000,0.000000}%
\pgfsetstrokecolor{textcolor}%
\pgfsetfillcolor{textcolor}%
\pgftext[x=0.393533in, y=2.985166in, left, base]{\color{textcolor}\sffamily\fontsize{10.000000}{12.000000}\selectfont 20.0}%
\end{pgfscope}%
\begin{pgfscope}%
\pgfpathrectangle{\pgfqpoint{0.800000in}{0.528000in}}{\pgfqpoint{4.960000in}{3.696000in}}%
\pgfusepath{clip}%
\pgfsetrectcap%
\pgfsetroundjoin%
\pgfsetlinewidth{0.803000pt}%
\definecolor{currentstroke}{rgb}{0.690196,0.690196,0.690196}%
\pgfsetstrokecolor{currentstroke}%
\pgfsetdash{}{0pt}%
\pgfpathmoveto{\pgfqpoint{0.800000in}{3.487606in}}%
\pgfpathlineto{\pgfqpoint{5.760000in}{3.487606in}}%
\pgfusepath{stroke}%
\end{pgfscope}%
\begin{pgfscope}%
\pgfsetbuttcap%
\pgfsetroundjoin%
\definecolor{currentfill}{rgb}{0.000000,0.000000,0.000000}%
\pgfsetfillcolor{currentfill}%
\pgfsetlinewidth{0.803000pt}%
\definecolor{currentstroke}{rgb}{0.000000,0.000000,0.000000}%
\pgfsetstrokecolor{currentstroke}%
\pgfsetdash{}{0pt}%
\pgfsys@defobject{currentmarker}{\pgfqpoint{-0.048611in}{0.000000in}}{\pgfqpoint{-0.000000in}{0.000000in}}{%
\pgfpathmoveto{\pgfqpoint{-0.000000in}{0.000000in}}%
\pgfpathlineto{\pgfqpoint{-0.048611in}{0.000000in}}%
\pgfusepath{stroke,fill}%
}%
\begin{pgfscope}%
\pgfsys@transformshift{0.800000in}{3.487606in}%
\pgfsys@useobject{currentmarker}{}%
\end{pgfscope}%
\end{pgfscope}%
\begin{pgfscope}%
\definecolor{textcolor}{rgb}{0.000000,0.000000,0.000000}%
\pgfsetstrokecolor{textcolor}%
\pgfsetfillcolor{textcolor}%
\pgftext[x=0.393533in, y=3.434844in, left, base]{\color{textcolor}\sffamily\fontsize{10.000000}{12.000000}\selectfont 22.5}%
\end{pgfscope}%
\begin{pgfscope}%
\pgfpathrectangle{\pgfqpoint{0.800000in}{0.528000in}}{\pgfqpoint{4.960000in}{3.696000in}}%
\pgfusepath{clip}%
\pgfsetrectcap%
\pgfsetroundjoin%
\pgfsetlinewidth{0.803000pt}%
\definecolor{currentstroke}{rgb}{0.690196,0.690196,0.690196}%
\pgfsetstrokecolor{currentstroke}%
\pgfsetdash{}{0pt}%
\pgfpathmoveto{\pgfqpoint{0.800000in}{3.937285in}}%
\pgfpathlineto{\pgfqpoint{5.760000in}{3.937285in}}%
\pgfusepath{stroke}%
\end{pgfscope}%
\begin{pgfscope}%
\pgfsetbuttcap%
\pgfsetroundjoin%
\definecolor{currentfill}{rgb}{0.000000,0.000000,0.000000}%
\pgfsetfillcolor{currentfill}%
\pgfsetlinewidth{0.803000pt}%
\definecolor{currentstroke}{rgb}{0.000000,0.000000,0.000000}%
\pgfsetstrokecolor{currentstroke}%
\pgfsetdash{}{0pt}%
\pgfsys@defobject{currentmarker}{\pgfqpoint{-0.048611in}{0.000000in}}{\pgfqpoint{-0.000000in}{0.000000in}}{%
\pgfpathmoveto{\pgfqpoint{-0.000000in}{0.000000in}}%
\pgfpathlineto{\pgfqpoint{-0.048611in}{0.000000in}}%
\pgfusepath{stroke,fill}%
}%
\begin{pgfscope}%
\pgfsys@transformshift{0.800000in}{3.937285in}%
\pgfsys@useobject{currentmarker}{}%
\end{pgfscope}%
\end{pgfscope}%
\begin{pgfscope}%
\definecolor{textcolor}{rgb}{0.000000,0.000000,0.000000}%
\pgfsetstrokecolor{textcolor}%
\pgfsetfillcolor{textcolor}%
\pgftext[x=0.393533in, y=3.884523in, left, base]{\color{textcolor}\sffamily\fontsize{10.000000}{12.000000}\selectfont 25.0}%
\end{pgfscope}%
\begin{pgfscope}%
\definecolor{textcolor}{rgb}{0.000000,0.000000,0.000000}%
\pgfsetstrokecolor{textcolor}%
\pgfsetfillcolor{textcolor}%
\pgftext[x=0.337977in,y=2.376000in,,bottom,rotate=90.000000]{\color{textcolor}\sffamily\fontsize{10.000000}{12.000000}\selectfont Heading [deg]}%
\end{pgfscope}%
\begin{pgfscope}%
\pgfpathrectangle{\pgfqpoint{0.800000in}{0.528000in}}{\pgfqpoint{4.960000in}{3.696000in}}%
\pgfusepath{clip}%
\pgfsetrectcap%
\pgfsetroundjoin%
\pgfsetlinewidth{1.505625pt}%
\definecolor{currentstroke}{rgb}{0.121569,0.466667,0.705882}%
\pgfsetstrokecolor{currentstroke}%
\pgfsetdash{}{0pt}%
\pgfpathmoveto{\pgfqpoint{1.025455in}{0.697799in}}%
\pgfpathlineto{\pgfqpoint{1.079817in}{0.796728in}}%
\pgfpathlineto{\pgfqpoint{1.133923in}{1.010775in}}%
\pgfpathlineto{\pgfqpoint{1.188290in}{1.296771in}}%
\pgfpathlineto{\pgfqpoint{1.242517in}{1.622338in}}%
\pgfpathlineto{\pgfqpoint{1.296728in}{1.942510in}}%
\pgfpathlineto{\pgfqpoint{1.351077in}{2.286064in}}%
\pgfpathlineto{\pgfqpoint{1.406904in}{2.617028in}}%
\pgfpathlineto{\pgfqpoint{1.459894in}{2.903024in}}%
\pgfpathlineto{\pgfqpoint{1.514166in}{3.133259in}}%
\pgfpathlineto{\pgfqpoint{1.568065in}{3.259169in}}%
\pgfpathlineto{\pgfqpoint{1.621995in}{3.300540in}}%
\pgfpathlineto{\pgfqpoint{1.678250in}{3.287949in}}%
\pgfpathlineto{\pgfqpoint{1.730619in}{3.253773in}}%
\pgfpathlineto{\pgfqpoint{1.784562in}{3.216000in}}%
\pgfpathlineto{\pgfqpoint{1.840736in}{3.176428in}}%
\pgfpathlineto{\pgfqpoint{1.893198in}{3.149448in}}%
\pgfpathlineto{\pgfqpoint{1.950363in}{3.131460in}}%
\pgfpathlineto{\pgfqpoint{2.004186in}{3.129662in}}%
\pgfpathlineto{\pgfqpoint{2.057916in}{3.142253in}}%
\pgfpathlineto{\pgfqpoint{2.111602in}{3.162039in}}%
\pgfpathlineto{\pgfqpoint{2.166266in}{3.180026in}}%
\pgfpathlineto{\pgfqpoint{2.221090in}{3.187221in}}%
\pgfpathlineto{\pgfqpoint{2.274901in}{3.189019in}}%
\pgfpathlineto{\pgfqpoint{2.329511in}{3.189019in}}%
\pgfpathlineto{\pgfqpoint{2.383896in}{3.189019in}}%
\pgfpathlineto{\pgfqpoint{2.437984in}{3.187221in}}%
\pgfpathlineto{\pgfqpoint{2.492379in}{3.181824in}}%
\pgfpathlineto{\pgfqpoint{2.546770in}{3.176428in}}%
\pgfpathlineto{\pgfqpoint{2.603287in}{3.165636in}}%
\pgfpathlineto{\pgfqpoint{2.656638in}{3.156642in}}%
\pgfpathlineto{\pgfqpoint{2.710391in}{3.147649in}}%
\pgfpathlineto{\pgfqpoint{2.764488in}{3.136857in}}%
\pgfpathlineto{\pgfqpoint{2.818536in}{3.120668in}}%
\pgfpathlineto{\pgfqpoint{2.872961in}{3.109876in}}%
\pgfpathlineto{\pgfqpoint{2.927076in}{3.104480in}}%
\pgfpathlineto{\pgfqpoint{2.981116in}{3.106278in}}%
\pgfpathlineto{\pgfqpoint{3.037525in}{3.113473in}}%
\pgfpathlineto{\pgfqpoint{3.090490in}{3.118869in}}%
\pgfpathlineto{\pgfqpoint{3.144707in}{3.127863in}}%
\pgfpathlineto{\pgfqpoint{3.199894in}{3.131460in}}%
\pgfpathlineto{\pgfqpoint{3.253338in}{3.131460in}}%
\pgfpathlineto{\pgfqpoint{3.307116in}{3.127863in}}%
\pgfpathlineto{\pgfqpoint{3.361232in}{3.117071in}}%
\pgfpathlineto{\pgfqpoint{3.415329in}{3.106278in}}%
\pgfpathlineto{\pgfqpoint{3.469664in}{3.095486in}}%
\pgfpathlineto{\pgfqpoint{3.523439in}{3.084694in}}%
\pgfpathlineto{\pgfqpoint{3.578614in}{3.075700in}}%
\pgfpathlineto{\pgfqpoint{3.632675in}{3.070304in}}%
\pgfpathlineto{\pgfqpoint{3.686967in}{3.072103in}}%
\pgfpathlineto{\pgfqpoint{3.741763in}{3.081096in}}%
\pgfpathlineto{\pgfqpoint{3.795833in}{3.093687in}}%
\pgfpathlineto{\pgfqpoint{3.851525in}{3.102681in}}%
\pgfpathlineto{\pgfqpoint{3.905219in}{3.099084in}}%
\pgfpathlineto{\pgfqpoint{3.958767in}{3.077499in}}%
\pgfpathlineto{\pgfqpoint{4.013311in}{3.059512in}}%
\pgfpathlineto{\pgfqpoint{4.067422in}{3.057713in}}%
\pgfpathlineto{\pgfqpoint{4.121571in}{3.064908in}}%
\pgfpathlineto{\pgfqpoint{4.175734in}{3.075700in}}%
\pgfpathlineto{\pgfqpoint{4.229785in}{3.088291in}}%
\pgfpathlineto{\pgfqpoint{4.284440in}{3.106278in}}%
\pgfpathlineto{\pgfqpoint{4.338666in}{3.122467in}}%
\pgfpathlineto{\pgfqpoint{4.393092in}{3.133259in}}%
\pgfpathlineto{\pgfqpoint{4.446864in}{3.144051in}}%
\pgfpathlineto{\pgfqpoint{4.502372in}{3.154844in}}%
\pgfpathlineto{\pgfqpoint{4.557166in}{3.167435in}}%
\pgfpathlineto{\pgfqpoint{4.610727in}{3.178227in}}%
\pgfpathlineto{\pgfqpoint{4.665053in}{3.189019in}}%
\pgfpathlineto{\pgfqpoint{4.719746in}{3.190818in}}%
\pgfpathlineto{\pgfqpoint{4.773889in}{3.183623in}}%
\pgfpathlineto{\pgfqpoint{4.828083in}{3.178227in}}%
\pgfpathlineto{\pgfqpoint{4.882083in}{3.165636in}}%
\pgfpathlineto{\pgfqpoint{4.936499in}{3.153045in}}%
\pgfpathlineto{\pgfqpoint{4.990710in}{3.144051in}}%
\pgfpathlineto{\pgfqpoint{5.045041in}{3.145850in}}%
\pgfpathlineto{\pgfqpoint{5.100981in}{3.147649in}}%
\pgfpathlineto{\pgfqpoint{5.154170in}{3.151246in}}%
\pgfpathlineto{\pgfqpoint{5.207988in}{3.153045in}}%
\pgfpathlineto{\pgfqpoint{5.261945in}{3.156642in}}%
\pgfpathlineto{\pgfqpoint{5.316693in}{3.158441in}}%
\pgfpathlineto{\pgfqpoint{5.370897in}{3.162039in}}%
\pgfpathlineto{\pgfqpoint{5.424825in}{3.163837in}}%
\pgfpathlineto{\pgfqpoint{5.479200in}{3.165636in}}%
\pgfpathlineto{\pgfqpoint{5.534545in}{3.169233in}}%
\pgfusepath{stroke}%
\end{pgfscope}%
\begin{pgfscope}%
\pgfpathrectangle{\pgfqpoint{0.800000in}{0.528000in}}{\pgfqpoint{4.960000in}{3.696000in}}%
\pgfusepath{clip}%
\pgfsetrectcap%
\pgfsetroundjoin%
\pgfsetlinewidth{1.505625pt}%
\definecolor{currentstroke}{rgb}{1.000000,0.498039,0.054902}%
\pgfsetstrokecolor{currentstroke}%
\pgfsetdash{}{0pt}%
\pgfpathmoveto{\pgfqpoint{1.025455in}{0.699597in}}%
\pgfpathlineto{\pgfqpoint{1.080220in}{0.802124in}}%
\pgfpathlineto{\pgfqpoint{1.134585in}{1.017970in}}%
\pgfpathlineto{\pgfqpoint{1.189826in}{1.302167in}}%
\pgfpathlineto{\pgfqpoint{1.243773in}{1.629533in}}%
\pgfpathlineto{\pgfqpoint{1.297489in}{1.967692in}}%
\pgfpathlineto{\pgfqpoint{1.351929in}{2.282467in}}%
\pgfpathlineto{\pgfqpoint{1.406435in}{2.624223in}}%
\pgfpathlineto{\pgfqpoint{1.460887in}{2.958784in}}%
\pgfpathlineto{\pgfqpoint{1.515248in}{3.241182in}}%
\pgfpathlineto{\pgfqpoint{1.569183in}{3.457028in}}%
\pgfpathlineto{\pgfqpoint{1.623482in}{3.579340in}}%
\pgfpathlineto{\pgfqpoint{1.677765in}{3.611717in}}%
\pgfpathlineto{\pgfqpoint{1.731972in}{3.586535in}}%
\pgfpathlineto{\pgfqpoint{1.788130in}{3.519983in}}%
\pgfpathlineto{\pgfqpoint{1.841967in}{3.437242in}}%
\pgfpathlineto{\pgfqpoint{1.895677in}{3.365293in}}%
\pgfpathlineto{\pgfqpoint{1.950222in}{3.314929in}}%
\pgfpathlineto{\pgfqpoint{2.004201in}{3.293345in}}%
\pgfpathlineto{\pgfqpoint{2.058649in}{3.296942in}}%
\pgfpathlineto{\pgfqpoint{2.112810in}{3.316728in}}%
\pgfpathlineto{\pgfqpoint{2.167018in}{3.340111in}}%
\pgfpathlineto{\pgfqpoint{2.221126in}{3.359897in}}%
\pgfpathlineto{\pgfqpoint{2.275560in}{3.374287in}}%
\pgfpathlineto{\pgfqpoint{2.329897in}{3.385079in}}%
\pgfpathlineto{\pgfqpoint{2.384139in}{3.390475in}}%
\pgfpathlineto{\pgfqpoint{2.440283in}{3.386878in}}%
\pgfpathlineto{\pgfqpoint{2.494007in}{3.386878in}}%
\pgfpathlineto{\pgfqpoint{2.547589in}{3.383281in}}%
\pgfpathlineto{\pgfqpoint{2.601548in}{3.376086in}}%
\pgfpathlineto{\pgfqpoint{2.655608in}{3.365293in}}%
\pgfpathlineto{\pgfqpoint{2.710059in}{3.349105in}}%
\pgfpathlineto{\pgfqpoint{2.764074in}{3.334715in}}%
\pgfpathlineto{\pgfqpoint{2.818221in}{3.325722in}}%
\pgfpathlineto{\pgfqpoint{2.872747in}{3.313131in}}%
\pgfpathlineto{\pgfqpoint{2.927224in}{3.293345in}}%
\pgfpathlineto{\pgfqpoint{2.981679in}{3.266364in}}%
\pgfpathlineto{\pgfqpoint{3.036127in}{3.228591in}}%
\pgfpathlineto{\pgfqpoint{3.092208in}{3.189019in}}%
\pgfpathlineto{\pgfqpoint{3.145417in}{3.153045in}}%
\pgfpathlineto{\pgfqpoint{3.198933in}{3.133259in}}%
\pgfpathlineto{\pgfqpoint{3.253261in}{3.124266in}}%
\pgfpathlineto{\pgfqpoint{3.307191in}{3.122467in}}%
\pgfpathlineto{\pgfqpoint{3.361257in}{3.126064in}}%
\pgfpathlineto{\pgfqpoint{3.415719in}{3.133259in}}%
\pgfpathlineto{\pgfqpoint{3.470548in}{3.149448in}}%
\pgfpathlineto{\pgfqpoint{3.524664in}{3.165636in}}%
\pgfpathlineto{\pgfqpoint{3.578813in}{3.183623in}}%
\pgfpathlineto{\pgfqpoint{3.634790in}{3.194415in}}%
\pgfpathlineto{\pgfqpoint{3.688072in}{3.198013in}}%
\pgfpathlineto{\pgfqpoint{3.741894in}{3.196214in}}%
\pgfpathlineto{\pgfqpoint{3.796095in}{3.194415in}}%
\pgfpathlineto{\pgfqpoint{3.850161in}{3.190818in}}%
\pgfpathlineto{\pgfqpoint{3.904446in}{3.181824in}}%
\pgfpathlineto{\pgfqpoint{3.958348in}{3.169233in}}%
\pgfpathlineto{\pgfqpoint{4.012420in}{3.156642in}}%
\pgfpathlineto{\pgfqpoint{4.066944in}{3.145850in}}%
\pgfpathlineto{\pgfqpoint{4.121231in}{3.135058in}}%
\pgfpathlineto{\pgfqpoint{4.175678in}{3.118869in}}%
\pgfpathlineto{\pgfqpoint{4.231433in}{3.100882in}}%
\pgfpathlineto{\pgfqpoint{4.284972in}{3.091889in}}%
\pgfpathlineto{\pgfqpoint{4.338915in}{3.095486in}}%
\pgfpathlineto{\pgfqpoint{4.392959in}{3.104480in}}%
\pgfpathlineto{\pgfqpoint{4.446948in}{3.120668in}}%
\pgfpathlineto{\pgfqpoint{4.501291in}{3.135058in}}%
\pgfpathlineto{\pgfqpoint{4.555287in}{3.149448in}}%
\pgfpathlineto{\pgfqpoint{4.609571in}{3.165636in}}%
\pgfpathlineto{\pgfqpoint{4.663653in}{3.171032in}}%
\pgfpathlineto{\pgfqpoint{4.718127in}{3.169233in}}%
\pgfpathlineto{\pgfqpoint{4.772528in}{3.167435in}}%
\pgfpathlineto{\pgfqpoint{4.826943in}{3.167435in}}%
\pgfpathlineto{\pgfqpoint{4.882287in}{3.167435in}}%
\pgfpathlineto{\pgfqpoint{4.936382in}{3.171032in}}%
\pgfpathlineto{\pgfqpoint{4.990063in}{3.185422in}}%
\pgfpathlineto{\pgfqpoint{5.044235in}{3.203409in}}%
\pgfpathlineto{\pgfqpoint{5.098168in}{3.208805in}}%
\pgfpathlineto{\pgfqpoint{5.152252in}{3.203409in}}%
\pgfpathlineto{\pgfqpoint{5.206756in}{3.190818in}}%
\pgfpathlineto{\pgfqpoint{5.261091in}{3.180026in}}%
\pgfpathlineto{\pgfqpoint{5.315426in}{3.169233in}}%
\pgfpathlineto{\pgfqpoint{5.369543in}{3.160240in}}%
\pgfpathlineto{\pgfqpoint{5.424082in}{3.160240in}}%
\pgfpathlineto{\pgfqpoint{5.479807in}{3.165636in}}%
\pgfpathlineto{\pgfqpoint{5.533857in}{3.178227in}}%
\pgfusepath{stroke}%
\end{pgfscope}%
\begin{pgfscope}%
\pgfpathrectangle{\pgfqpoint{0.800000in}{0.528000in}}{\pgfqpoint{4.960000in}{3.696000in}}%
\pgfusepath{clip}%
\pgfsetrectcap%
\pgfsetroundjoin%
\pgfsetlinewidth{1.505625pt}%
\definecolor{currentstroke}{rgb}{0.172549,0.627451,0.172549}%
\pgfsetstrokecolor{currentstroke}%
\pgfsetdash{}{0pt}%
\pgfpathmoveto{\pgfqpoint{1.025455in}{0.701396in}}%
\pgfpathlineto{\pgfqpoint{1.078966in}{0.798527in}}%
\pgfpathlineto{\pgfqpoint{1.132997in}{1.012574in}}%
\pgfpathlineto{\pgfqpoint{1.187625in}{1.300368in}}%
\pgfpathlineto{\pgfqpoint{1.242351in}{1.627734in}}%
\pgfpathlineto{\pgfqpoint{1.296420in}{1.947906in}}%
\pgfpathlineto{\pgfqpoint{1.350736in}{2.289662in}}%
\pgfpathlineto{\pgfqpoint{1.404847in}{2.618827in}}%
\pgfpathlineto{\pgfqpoint{1.459129in}{2.953388in}}%
\pgfpathlineto{\pgfqpoint{1.513355in}{3.250176in}}%
\pgfpathlineto{\pgfqpoint{1.569261in}{3.484009in}}%
\pgfpathlineto{\pgfqpoint{1.622775in}{3.624308in}}%
\pgfpathlineto{\pgfqpoint{1.676148in}{3.674672in}}%
\pgfpathlineto{\pgfqpoint{1.731322in}{3.645893in}}%
\pgfpathlineto{\pgfqpoint{1.784569in}{3.572146in}}%
\pgfpathlineto{\pgfqpoint{1.839432in}{3.489405in}}%
\pgfpathlineto{\pgfqpoint{1.893669in}{3.403066in}}%
\pgfpathlineto{\pgfqpoint{1.948509in}{3.334715in}}%
\pgfpathlineto{\pgfqpoint{2.002491in}{3.282552in}}%
\pgfpathlineto{\pgfqpoint{2.056724in}{3.257370in}}%
\pgfpathlineto{\pgfqpoint{2.111038in}{3.257370in}}%
\pgfpathlineto{\pgfqpoint{2.165495in}{3.271760in}}%
\pgfpathlineto{\pgfqpoint{2.221152in}{3.296942in}}%
\pgfpathlineto{\pgfqpoint{2.274905in}{3.322124in}}%
\pgfpathlineto{\pgfqpoint{2.328211in}{3.343709in}}%
\pgfpathlineto{\pgfqpoint{2.382187in}{3.354501in}}%
\pgfpathlineto{\pgfqpoint{2.436422in}{3.356300in}}%
\pgfpathlineto{\pgfqpoint{2.490788in}{3.350904in}}%
\pgfpathlineto{\pgfqpoint{2.544984in}{3.343709in}}%
\pgfpathlineto{\pgfqpoint{2.599018in}{3.332916in}}%
\pgfpathlineto{\pgfqpoint{2.652870in}{3.320325in}}%
\pgfpathlineto{\pgfqpoint{2.707582in}{3.309533in}}%
\pgfpathlineto{\pgfqpoint{2.761731in}{3.295143in}}%
\pgfpathlineto{\pgfqpoint{2.816106in}{3.282552in}}%
\pgfpathlineto{\pgfqpoint{2.871064in}{3.268163in}}%
\pgfpathlineto{\pgfqpoint{2.924723in}{3.253773in}}%
\pgfpathlineto{\pgfqpoint{2.978554in}{3.239383in}}%
\pgfpathlineto{\pgfqpoint{3.032795in}{3.232188in}}%
\pgfpathlineto{\pgfqpoint{3.087010in}{3.232188in}}%
\pgfpathlineto{\pgfqpoint{3.141259in}{3.237585in}}%
\pgfpathlineto{\pgfqpoint{3.195494in}{3.241182in}}%
\pgfpathlineto{\pgfqpoint{3.250554in}{3.235786in}}%
\pgfpathlineto{\pgfqpoint{3.304679in}{3.219597in}}%
\pgfpathlineto{\pgfqpoint{3.358947in}{3.201610in}}%
\pgfpathlineto{\pgfqpoint{3.413083in}{3.190818in}}%
\pgfpathlineto{\pgfqpoint{3.468973in}{3.181824in}}%
\pgfpathlineto{\pgfqpoint{3.522247in}{3.176428in}}%
\pgfpathlineto{\pgfqpoint{3.576045in}{3.176428in}}%
\pgfpathlineto{\pgfqpoint{3.630242in}{3.183623in}}%
\pgfpathlineto{\pgfqpoint{3.684458in}{3.192617in}}%
\pgfpathlineto{\pgfqpoint{3.738531in}{3.201610in}}%
\pgfpathlineto{\pgfqpoint{3.792593in}{3.203409in}}%
\pgfpathlineto{\pgfqpoint{3.846892in}{3.199812in}}%
\pgfpathlineto{\pgfqpoint{3.900843in}{3.190818in}}%
\pgfpathlineto{\pgfqpoint{3.955224in}{3.174630in}}%
\pgfpathlineto{\pgfqpoint{4.009566in}{3.160240in}}%
\pgfpathlineto{\pgfqpoint{4.065797in}{3.154844in}}%
\pgfpathlineto{\pgfqpoint{4.123343in}{3.160240in}}%
\pgfpathlineto{\pgfqpoint{4.174497in}{3.163837in}}%
\pgfpathlineto{\pgfqpoint{4.228600in}{3.162039in}}%
\pgfpathlineto{\pgfqpoint{4.282532in}{3.158441in}}%
\pgfpathlineto{\pgfqpoint{4.336773in}{3.154844in}}%
\pgfpathlineto{\pgfqpoint{4.390573in}{3.149448in}}%
\pgfpathlineto{\pgfqpoint{4.444326in}{3.144051in}}%
\pgfpathlineto{\pgfqpoint{4.498712in}{3.144051in}}%
\pgfpathlineto{\pgfqpoint{4.553020in}{3.149448in}}%
\pgfpathlineto{\pgfqpoint{4.607096in}{3.162039in}}%
\pgfpathlineto{\pgfqpoint{4.661293in}{3.163837in}}%
\pgfpathlineto{\pgfqpoint{4.715555in}{3.153045in}}%
\pgfpathlineto{\pgfqpoint{4.769887in}{3.138655in}}%
\pgfpathlineto{\pgfqpoint{4.823880in}{3.126064in}}%
\pgfpathlineto{\pgfqpoint{4.877957in}{3.120668in}}%
\pgfpathlineto{\pgfqpoint{4.932251in}{3.124266in}}%
\pgfpathlineto{\pgfqpoint{4.986649in}{3.135058in}}%
\pgfpathlineto{\pgfqpoint{5.040826in}{3.149448in}}%
\pgfpathlineto{\pgfqpoint{5.095066in}{3.169233in}}%
\pgfpathlineto{\pgfqpoint{5.150645in}{3.178227in}}%
\pgfpathlineto{\pgfqpoint{5.204147in}{3.183623in}}%
\pgfpathlineto{\pgfqpoint{5.257682in}{3.183623in}}%
\pgfpathlineto{\pgfqpoint{5.311952in}{3.180026in}}%
\pgfpathlineto{\pgfqpoint{5.366106in}{3.174630in}}%
\pgfpathlineto{\pgfqpoint{5.420363in}{3.163837in}}%
\pgfpathlineto{\pgfqpoint{5.474391in}{3.156642in}}%
\pgfpathlineto{\pgfqpoint{5.529047in}{3.145850in}}%
\pgfusepath{stroke}%
\end{pgfscope}%
\begin{pgfscope}%
\pgfpathrectangle{\pgfqpoint{0.800000in}{0.528000in}}{\pgfqpoint{4.960000in}{3.696000in}}%
\pgfusepath{clip}%
\pgfsetrectcap%
\pgfsetroundjoin%
\pgfsetlinewidth{1.505625pt}%
\definecolor{currentstroke}{rgb}{0.839216,0.152941,0.156863}%
\pgfsetstrokecolor{currentstroke}%
\pgfsetdash{}{0pt}%
\pgfpathmoveto{\pgfqpoint{1.025455in}{0.696000in}}%
\pgfpathlineto{\pgfqpoint{1.079947in}{0.796728in}}%
\pgfpathlineto{\pgfqpoint{1.135043in}{1.028762in}}%
\pgfpathlineto{\pgfqpoint{1.188759in}{1.300368in}}%
\pgfpathlineto{\pgfqpoint{1.243244in}{1.627734in}}%
\pgfpathlineto{\pgfqpoint{1.296941in}{1.965893in}}%
\pgfpathlineto{\pgfqpoint{1.350943in}{2.282467in}}%
\pgfpathlineto{\pgfqpoint{1.405296in}{2.622424in}}%
\pgfpathlineto{\pgfqpoint{1.459410in}{2.956985in}}%
\pgfpathlineto{\pgfqpoint{1.513550in}{3.260968in}}%
\pgfpathlineto{\pgfqpoint{1.567854in}{3.577542in}}%
\pgfpathlineto{\pgfqpoint{1.622094in}{3.841953in}}%
\pgfpathlineto{\pgfqpoint{1.676183in}{4.002039in}}%
\pgfpathlineto{\pgfqpoint{1.730685in}{4.056000in}}%
\pgfpathlineto{\pgfqpoint{1.784453in}{4.011032in}}%
\pgfpathlineto{\pgfqpoint{1.838746in}{3.915700in}}%
\pgfpathlineto{\pgfqpoint{1.892942in}{3.798784in}}%
\pgfpathlineto{\pgfqpoint{1.947080in}{3.660283in}}%
\pgfpathlineto{\pgfqpoint{2.001826in}{3.545165in}}%
\pgfpathlineto{\pgfqpoint{2.055851in}{3.457028in}}%
\pgfpathlineto{\pgfqpoint{2.111534in}{3.395872in}}%
\pgfpathlineto{\pgfqpoint{2.164734in}{3.359897in}}%
\pgfpathlineto{\pgfqpoint{2.218593in}{3.356300in}}%
\pgfpathlineto{\pgfqpoint{2.272588in}{3.367092in}}%
\pgfpathlineto{\pgfqpoint{2.326831in}{3.383281in}}%
\pgfpathlineto{\pgfqpoint{2.381279in}{3.394073in}}%
\pgfpathlineto{\pgfqpoint{2.436572in}{3.394073in}}%
\pgfpathlineto{\pgfqpoint{2.490157in}{3.377884in}}%
\pgfpathlineto{\pgfqpoint{2.543800in}{3.347306in}}%
\pgfpathlineto{\pgfqpoint{2.598059in}{3.307734in}}%
\pgfpathlineto{\pgfqpoint{2.653457in}{3.268163in}}%
\pgfpathlineto{\pgfqpoint{2.709710in}{3.230390in}}%
\pgfpathlineto{\pgfqpoint{2.762669in}{3.198013in}}%
\pgfpathlineto{\pgfqpoint{2.816340in}{3.176428in}}%
\pgfpathlineto{\pgfqpoint{2.870598in}{3.162039in}}%
\pgfpathlineto{\pgfqpoint{2.924476in}{3.154844in}}%
\pgfpathlineto{\pgfqpoint{2.978863in}{3.149448in}}%
\pgfpathlineto{\pgfqpoint{3.032992in}{3.144051in}}%
\pgfpathlineto{\pgfqpoint{3.087277in}{3.135058in}}%
\pgfpathlineto{\pgfqpoint{3.141448in}{3.120668in}}%
\pgfpathlineto{\pgfqpoint{3.195694in}{3.100882in}}%
\pgfpathlineto{\pgfqpoint{3.251038in}{3.077499in}}%
\pgfpathlineto{\pgfqpoint{3.304234in}{3.052317in}}%
\pgfpathlineto{\pgfqpoint{3.360523in}{3.030732in}}%
\pgfpathlineto{\pgfqpoint{3.414760in}{3.012745in}}%
\pgfpathlineto{\pgfqpoint{3.468180in}{3.010946in}}%
\pgfpathlineto{\pgfqpoint{3.522369in}{3.028934in}}%
\pgfpathlineto{\pgfqpoint{3.576428in}{3.050518in}}%
\pgfpathlineto{\pgfqpoint{3.630607in}{3.070304in}}%
\pgfpathlineto{\pgfqpoint{3.685114in}{3.088291in}}%
\pgfpathlineto{\pgfqpoint{3.739314in}{3.106278in}}%
\pgfpathlineto{\pgfqpoint{3.793434in}{3.126064in}}%
\pgfpathlineto{\pgfqpoint{3.848197in}{3.149448in}}%
\pgfpathlineto{\pgfqpoint{3.902608in}{3.176428in}}%
\pgfpathlineto{\pgfqpoint{3.956727in}{3.192617in}}%
\pgfpathlineto{\pgfqpoint{4.012974in}{3.199812in}}%
\pgfpathlineto{\pgfqpoint{4.066523in}{3.201610in}}%
\pgfpathlineto{\pgfqpoint{4.120231in}{3.208805in}}%
\pgfpathlineto{\pgfqpoint{4.174558in}{3.212403in}}%
\pgfpathlineto{\pgfqpoint{4.228531in}{3.219597in}}%
\pgfpathlineto{\pgfqpoint{4.282541in}{3.226792in}}%
\pgfpathlineto{\pgfqpoint{4.337498in}{3.232188in}}%
\pgfpathlineto{\pgfqpoint{4.392156in}{3.239383in}}%
\pgfpathlineto{\pgfqpoint{4.445690in}{3.242981in}}%
\pgfpathlineto{\pgfqpoint{4.500469in}{3.244779in}}%
\pgfpathlineto{\pgfqpoint{4.554604in}{3.241182in}}%
\pgfpathlineto{\pgfqpoint{4.608823in}{3.235786in}}%
\pgfpathlineto{\pgfqpoint{4.664880in}{3.228591in}}%
\pgfpathlineto{\pgfqpoint{4.718697in}{3.221396in}}%
\pgfpathlineto{\pgfqpoint{4.772489in}{3.217799in}}%
\pgfpathlineto{\pgfqpoint{4.826662in}{3.212403in}}%
\pgfpathlineto{\pgfqpoint{4.880693in}{3.208805in}}%
\pgfpathlineto{\pgfqpoint{4.935212in}{3.207006in}}%
\pgfpathlineto{\pgfqpoint{4.989215in}{3.205208in}}%
\pgfpathlineto{\pgfqpoint{5.044150in}{3.207006in}}%
\pgfpathlineto{\pgfqpoint{5.098445in}{3.208805in}}%
\pgfpathlineto{\pgfqpoint{5.152556in}{3.216000in}}%
\pgfpathlineto{\pgfqpoint{5.207028in}{3.223195in}}%
\pgfpathlineto{\pgfqpoint{5.261721in}{3.232188in}}%
\pgfpathlineto{\pgfqpoint{5.316835in}{3.239383in}}%
\pgfpathlineto{\pgfqpoint{5.370415in}{3.246578in}}%
\pgfpathlineto{\pgfqpoint{5.424401in}{3.253773in}}%
\pgfpathlineto{\pgfqpoint{5.478377in}{3.259169in}}%
\pgfpathlineto{\pgfqpoint{5.532589in}{3.268163in}}%
\pgfusepath{stroke}%
\end{pgfscope}%
\begin{pgfscope}%
\pgfpathrectangle{\pgfqpoint{0.800000in}{0.528000in}}{\pgfqpoint{4.960000in}{3.696000in}}%
\pgfusepath{clip}%
\pgfsetrectcap%
\pgfsetroundjoin%
\pgfsetlinewidth{1.505625pt}%
\definecolor{currentstroke}{rgb}{0.580392,0.403922,0.741176}%
\pgfsetstrokecolor{currentstroke}%
\pgfsetdash{}{0pt}%
\pgfpathmoveto{\pgfqpoint{1.025455in}{0.699597in}}%
\pgfpathlineto{\pgfqpoint{1.079690in}{0.793131in}}%
\pgfpathlineto{\pgfqpoint{1.134013in}{1.014373in}}%
\pgfpathlineto{\pgfqpoint{1.188348in}{1.309362in}}%
\pgfpathlineto{\pgfqpoint{1.242604in}{1.618741in}}%
\pgfpathlineto{\pgfqpoint{1.298592in}{1.949704in}}%
\pgfpathlineto{\pgfqpoint{1.352103in}{2.293259in}}%
\pgfpathlineto{\pgfqpoint{1.405742in}{2.613430in}}%
\pgfpathlineto{\pgfqpoint{1.461786in}{2.964180in}}%
\pgfpathlineto{\pgfqpoint{1.514310in}{3.269961in}}%
\pgfpathlineto{\pgfqpoint{1.568579in}{3.564951in}}%
\pgfpathlineto{\pgfqpoint{1.622582in}{3.793388in}}%
\pgfpathlineto{\pgfqpoint{1.677527in}{3.915700in}}%
\pgfpathlineto{\pgfqpoint{1.731265in}{3.958869in}}%
\pgfpathlineto{\pgfqpoint{1.785722in}{3.931889in}}%
\pgfpathlineto{\pgfqpoint{1.840231in}{3.849148in}}%
\pgfpathlineto{\pgfqpoint{1.894008in}{3.734030in}}%
\pgfpathlineto{\pgfqpoint{1.950051in}{3.602724in}}%
\pgfpathlineto{\pgfqpoint{2.003275in}{3.493002in}}%
\pgfpathlineto{\pgfqpoint{2.056911in}{3.390475in}}%
\pgfpathlineto{\pgfqpoint{2.110893in}{3.307734in}}%
\pgfpathlineto{\pgfqpoint{2.165111in}{3.248377in}}%
\pgfpathlineto{\pgfqpoint{2.219222in}{3.223195in}}%
\pgfpathlineto{\pgfqpoint{2.273343in}{3.235786in}}%
\pgfpathlineto{\pgfqpoint{2.327661in}{3.268163in}}%
\pgfpathlineto{\pgfqpoint{2.381805in}{3.298741in}}%
\pgfpathlineto{\pgfqpoint{2.436143in}{3.304137in}}%
\pgfpathlineto{\pgfqpoint{2.490891in}{3.284351in}}%
\pgfpathlineto{\pgfqpoint{2.545061in}{3.250176in}}%
\pgfpathlineto{\pgfqpoint{2.600632in}{3.201610in}}%
\pgfpathlineto{\pgfqpoint{2.654009in}{3.156642in}}%
\pgfpathlineto{\pgfqpoint{2.708578in}{3.117071in}}%
\pgfpathlineto{\pgfqpoint{2.762920in}{3.095486in}}%
\pgfpathlineto{\pgfqpoint{2.817057in}{3.084694in}}%
\pgfpathlineto{\pgfqpoint{2.871017in}{3.081096in}}%
\pgfpathlineto{\pgfqpoint{2.925244in}{3.079298in}}%
\pgfpathlineto{\pgfqpoint{2.979870in}{3.079298in}}%
\pgfpathlineto{\pgfqpoint{3.033799in}{3.075700in}}%
\pgfpathlineto{\pgfqpoint{3.088444in}{3.070304in}}%
\pgfpathlineto{\pgfqpoint{3.143332in}{3.063109in}}%
\pgfpathlineto{\pgfqpoint{3.197372in}{3.054116in}}%
\pgfpathlineto{\pgfqpoint{3.253701in}{3.045122in}}%
\pgfpathlineto{\pgfqpoint{3.307076in}{3.036128in}}%
\pgfpathlineto{\pgfqpoint{3.361382in}{3.028934in}}%
\pgfpathlineto{\pgfqpoint{3.415547in}{3.028934in}}%
\pgfpathlineto{\pgfqpoint{3.469859in}{3.036128in}}%
\pgfpathlineto{\pgfqpoint{3.524049in}{3.034330in}}%
\pgfpathlineto{\pgfqpoint{3.578230in}{3.012745in}}%
\pgfpathlineto{\pgfqpoint{3.632345in}{2.982167in}}%
\pgfpathlineto{\pgfqpoint{3.686652in}{2.964180in}}%
\pgfpathlineto{\pgfqpoint{3.742434in}{2.964180in}}%
\pgfpathlineto{\pgfqpoint{3.796444in}{2.973173in}}%
\pgfpathlineto{\pgfqpoint{3.852666in}{2.978570in}}%
\pgfpathlineto{\pgfqpoint{3.906402in}{2.980368in}}%
\pgfpathlineto{\pgfqpoint{3.959759in}{2.983966in}}%
\pgfpathlineto{\pgfqpoint{4.014981in}{2.983966in}}%
\pgfpathlineto{\pgfqpoint{4.070187in}{2.978570in}}%
\pgfpathlineto{\pgfqpoint{4.123039in}{2.969576in}}%
\pgfpathlineto{\pgfqpoint{4.176337in}{2.971375in}}%
\pgfpathlineto{\pgfqpoint{4.230234in}{2.989362in}}%
\pgfpathlineto{\pgfqpoint{4.285211in}{3.003752in}}%
\pgfpathlineto{\pgfqpoint{4.338944in}{3.012745in}}%
\pgfpathlineto{\pgfqpoint{4.393095in}{3.018141in}}%
\pgfpathlineto{\pgfqpoint{4.447850in}{3.018141in}}%
\pgfpathlineto{\pgfqpoint{4.501945in}{3.014544in}}%
\pgfpathlineto{\pgfqpoint{4.556050in}{3.012745in}}%
\pgfpathlineto{\pgfqpoint{4.609931in}{3.003752in}}%
\pgfpathlineto{\pgfqpoint{4.665426in}{2.991161in}}%
\pgfpathlineto{\pgfqpoint{4.719055in}{2.983966in}}%
\pgfpathlineto{\pgfqpoint{4.773045in}{2.971375in}}%
\pgfpathlineto{\pgfqpoint{4.827102in}{2.962381in}}%
\pgfpathlineto{\pgfqpoint{4.881257in}{2.967777in}}%
\pgfpathlineto{\pgfqpoint{4.935654in}{2.991161in}}%
\pgfpathlineto{\pgfqpoint{4.989556in}{3.025336in}}%
\pgfpathlineto{\pgfqpoint{5.044014in}{3.063109in}}%
\pgfpathlineto{\pgfqpoint{5.098561in}{3.102681in}}%
\pgfpathlineto{\pgfqpoint{5.153143in}{3.142253in}}%
\pgfpathlineto{\pgfqpoint{5.207414in}{3.160240in}}%
\pgfpathlineto{\pgfqpoint{5.262474in}{3.156642in}}%
\pgfpathlineto{\pgfqpoint{5.316755in}{3.145850in}}%
\pgfpathlineto{\pgfqpoint{5.370890in}{3.129662in}}%
\pgfpathlineto{\pgfqpoint{5.424704in}{3.109876in}}%
\pgfpathlineto{\pgfqpoint{5.478925in}{3.079298in}}%
\pgfpathlineto{\pgfqpoint{5.533836in}{3.046921in}}%
\pgfusepath{stroke}%
\end{pgfscope}%
\begin{pgfscope}%
\pgfpathrectangle{\pgfqpoint{0.800000in}{0.528000in}}{\pgfqpoint{4.960000in}{3.696000in}}%
\pgfusepath{clip}%
\pgfsetrectcap%
\pgfsetroundjoin%
\pgfsetlinewidth{1.505625pt}%
\definecolor{currentstroke}{rgb}{0.549020,0.337255,0.294118}%
\pgfsetstrokecolor{currentstroke}%
\pgfsetdash{}{0pt}%
\pgfpathmoveto{\pgfqpoint{1.025455in}{0.697799in}}%
\pgfpathlineto{\pgfqpoint{1.080023in}{0.793131in}}%
\pgfpathlineto{\pgfqpoint{1.133950in}{1.019769in}}%
\pgfpathlineto{\pgfqpoint{1.188858in}{1.298570in}}%
\pgfpathlineto{\pgfqpoint{1.242748in}{1.625936in}}%
\pgfpathlineto{\pgfqpoint{1.296905in}{1.956899in}}%
\pgfpathlineto{\pgfqpoint{1.351201in}{2.280668in}}%
\pgfpathlineto{\pgfqpoint{1.405479in}{2.620625in}}%
\pgfpathlineto{\pgfqpoint{1.461132in}{2.946193in}}%
\pgfpathlineto{\pgfqpoint{1.514846in}{3.277156in}}%
\pgfpathlineto{\pgfqpoint{1.568345in}{3.559555in}}%
\pgfpathlineto{\pgfqpoint{1.622776in}{3.771803in}}%
\pgfpathlineto{\pgfqpoint{1.677160in}{3.868934in}}%
\pgfpathlineto{\pgfqpoint{1.731329in}{3.867135in}}%
\pgfpathlineto{\pgfqpoint{1.785487in}{3.762809in}}%
\pgfpathlineto{\pgfqpoint{1.839854in}{3.602724in}}%
\pgfpathlineto{\pgfqpoint{1.893803in}{3.444437in}}%
\pgfpathlineto{\pgfqpoint{1.948228in}{3.302338in}}%
\pgfpathlineto{\pgfqpoint{2.002342in}{3.174630in}}%
\pgfpathlineto{\pgfqpoint{2.056665in}{3.097285in}}%
\pgfpathlineto{\pgfqpoint{2.112475in}{3.064908in}}%
\pgfpathlineto{\pgfqpoint{2.166361in}{3.075700in}}%
\pgfpathlineto{\pgfqpoint{2.220153in}{3.115272in}}%
\pgfpathlineto{\pgfqpoint{2.274745in}{3.172831in}}%
\pgfpathlineto{\pgfqpoint{2.328733in}{3.214201in}}%
\pgfpathlineto{\pgfqpoint{2.384336in}{3.232188in}}%
\pgfpathlineto{\pgfqpoint{2.436870in}{3.232188in}}%
\pgfpathlineto{\pgfqpoint{2.491682in}{3.232188in}}%
\pgfpathlineto{\pgfqpoint{2.546362in}{3.230390in}}%
\pgfpathlineto{\pgfqpoint{2.600459in}{3.233987in}}%
\pgfpathlineto{\pgfqpoint{2.654776in}{3.221396in}}%
\pgfpathlineto{\pgfqpoint{2.710928in}{3.196214in}}%
\pgfpathlineto{\pgfqpoint{2.764404in}{3.165636in}}%
\pgfpathlineto{\pgfqpoint{2.818768in}{3.144051in}}%
\pgfpathlineto{\pgfqpoint{2.872296in}{3.136857in}}%
\pgfpathlineto{\pgfqpoint{2.926411in}{3.126064in}}%
\pgfpathlineto{\pgfqpoint{2.980787in}{3.127863in}}%
\pgfpathlineto{\pgfqpoint{3.035124in}{3.149448in}}%
\pgfpathlineto{\pgfqpoint{3.089025in}{3.187221in}}%
\pgfpathlineto{\pgfqpoint{3.143523in}{3.230390in}}%
\pgfpathlineto{\pgfqpoint{3.197738in}{3.275358in}}%
\pgfpathlineto{\pgfqpoint{3.251828in}{3.323923in}}%
\pgfpathlineto{\pgfqpoint{3.307984in}{3.374287in}}%
\pgfpathlineto{\pgfqpoint{3.361490in}{3.412060in}}%
\pgfpathlineto{\pgfqpoint{3.414924in}{3.431846in}}%
\pgfpathlineto{\pgfqpoint{3.469007in}{3.430047in}}%
\pgfpathlineto{\pgfqpoint{3.524324in}{3.413859in}}%
\pgfpathlineto{\pgfqpoint{3.578963in}{3.392274in}}%
\pgfpathlineto{\pgfqpoint{3.632371in}{3.347306in}}%
\pgfpathlineto{\pgfqpoint{3.686555in}{3.284351in}}%
\pgfpathlineto{\pgfqpoint{3.740517in}{3.224994in}}%
\pgfpathlineto{\pgfqpoint{3.795411in}{3.187221in}}%
\pgfpathlineto{\pgfqpoint{3.849298in}{3.192617in}}%
\pgfpathlineto{\pgfqpoint{3.903586in}{3.224994in}}%
\pgfpathlineto{\pgfqpoint{3.959610in}{3.280754in}}%
\pgfpathlineto{\pgfqpoint{4.013334in}{3.331118in}}%
\pgfpathlineto{\pgfqpoint{4.066697in}{3.379683in}}%
\pgfpathlineto{\pgfqpoint{4.120732in}{3.439041in}}%
\pgfpathlineto{\pgfqpoint{4.174850in}{3.496600in}}%
\pgfpathlineto{\pgfqpoint{4.228816in}{3.496600in}}%
\pgfpathlineto{\pgfqpoint{4.283332in}{3.439041in}}%
\pgfpathlineto{\pgfqpoint{4.339525in}{3.363495in}}%
\pgfpathlineto{\pgfqpoint{4.391112in}{3.293345in}}%
\pgfpathlineto{\pgfqpoint{4.445759in}{3.241182in}}%
\pgfpathlineto{\pgfqpoint{4.500206in}{3.210604in}}%
\pgfpathlineto{\pgfqpoint{4.554681in}{3.203409in}}%
\pgfpathlineto{\pgfqpoint{4.610590in}{3.210604in}}%
\pgfpathlineto{\pgfqpoint{4.664701in}{3.226792in}}%
\pgfpathlineto{\pgfqpoint{4.718290in}{3.235786in}}%
\pgfpathlineto{\pgfqpoint{4.772829in}{3.230390in}}%
\pgfpathlineto{\pgfqpoint{4.827177in}{3.223195in}}%
\pgfpathlineto{\pgfqpoint{4.881591in}{3.228591in}}%
\pgfpathlineto{\pgfqpoint{4.935559in}{3.250176in}}%
\pgfpathlineto{\pgfqpoint{4.989568in}{3.278955in}}%
\pgfpathlineto{\pgfqpoint{5.046442in}{3.318527in}}%
\pgfpathlineto{\pgfqpoint{5.099910in}{3.347306in}}%
\pgfpathlineto{\pgfqpoint{5.153948in}{3.361696in}}%
\pgfpathlineto{\pgfqpoint{5.209490in}{3.361696in}}%
\pgfpathlineto{\pgfqpoint{5.264559in}{3.341910in}}%
\pgfpathlineto{\pgfqpoint{5.320562in}{3.291546in}}%
\pgfpathlineto{\pgfqpoint{5.371856in}{3.228591in}}%
\pgfpathlineto{\pgfqpoint{5.426096in}{3.171032in}}%
\pgfpathlineto{\pgfqpoint{5.480045in}{3.135058in}}%
\pgfpathlineto{\pgfqpoint{5.534493in}{3.122467in}}%
\pgfusepath{stroke}%
\end{pgfscope}%
\begin{pgfscope}%
\pgfpathrectangle{\pgfqpoint{0.800000in}{0.528000in}}{\pgfqpoint{4.960000in}{3.696000in}}%
\pgfusepath{clip}%
\pgfsetrectcap%
\pgfsetroundjoin%
\pgfsetlinewidth{1.505625pt}%
\definecolor{currentstroke}{rgb}{0.890196,0.466667,0.760784}%
\pgfsetstrokecolor{currentstroke}%
\pgfsetdash{}{0pt}%
\pgfpathmoveto{\pgfqpoint{1.025455in}{3.108526in}}%
\pgfpathlineto{\pgfqpoint{5.527685in}{3.108526in}}%
\pgfusepath{stroke}%
\end{pgfscope}%
\begin{pgfscope}%
\pgfsetrectcap%
\pgfsetmiterjoin%
\pgfsetlinewidth{0.803000pt}%
\definecolor{currentstroke}{rgb}{0.000000,0.000000,0.000000}%
\pgfsetstrokecolor{currentstroke}%
\pgfsetdash{}{0pt}%
\pgfpathmoveto{\pgfqpoint{0.800000in}{0.528000in}}%
\pgfpathlineto{\pgfqpoint{0.800000in}{4.224000in}}%
\pgfusepath{stroke}%
\end{pgfscope}%
\begin{pgfscope}%
\pgfsetrectcap%
\pgfsetmiterjoin%
\pgfsetlinewidth{0.803000pt}%
\definecolor{currentstroke}{rgb}{0.000000,0.000000,0.000000}%
\pgfsetstrokecolor{currentstroke}%
\pgfsetdash{}{0pt}%
\pgfpathmoveto{\pgfqpoint{5.760000in}{0.528000in}}%
\pgfpathlineto{\pgfqpoint{5.760000in}{4.224000in}}%
\pgfusepath{stroke}%
\end{pgfscope}%
\begin{pgfscope}%
\pgfsetrectcap%
\pgfsetmiterjoin%
\pgfsetlinewidth{0.803000pt}%
\definecolor{currentstroke}{rgb}{0.000000,0.000000,0.000000}%
\pgfsetstrokecolor{currentstroke}%
\pgfsetdash{}{0pt}%
\pgfpathmoveto{\pgfqpoint{0.800000in}{0.528000in}}%
\pgfpathlineto{\pgfqpoint{5.760000in}{0.528000in}}%
\pgfusepath{stroke}%
\end{pgfscope}%
\begin{pgfscope}%
\pgfsetrectcap%
\pgfsetmiterjoin%
\pgfsetlinewidth{0.803000pt}%
\definecolor{currentstroke}{rgb}{0.000000,0.000000,0.000000}%
\pgfsetstrokecolor{currentstroke}%
\pgfsetdash{}{0pt}%
\pgfpathmoveto{\pgfqpoint{0.800000in}{4.224000in}}%
\pgfpathlineto{\pgfqpoint{5.760000in}{4.224000in}}%
\pgfusepath{stroke}%
\end{pgfscope}%
\begin{pgfscope}%
\definecolor{textcolor}{rgb}{0.000000,0.000000,0.000000}%
\pgfsetstrokecolor{textcolor}%
\pgfsetfillcolor{textcolor}%
\pgftext[x=3.280000in,y=4.307333in,,base]{\color{textcolor}\sffamily\fontsize{12.000000}{14.400000}\selectfont Measured yaw position}%
\end{pgfscope}%
\begin{pgfscope}%
\pgfsetbuttcap%
\pgfsetmiterjoin%
\definecolor{currentfill}{rgb}{1.000000,1.000000,1.000000}%
\pgfsetfillcolor{currentfill}%
\pgfsetfillopacity{0.800000}%
\pgfsetlinewidth{1.003750pt}%
\definecolor{currentstroke}{rgb}{0.800000,0.800000,0.800000}%
\pgfsetstrokecolor{currentstroke}%
\pgfsetstrokeopacity{0.800000}%
\pgfsetdash{}{0pt}%
\pgfpathmoveto{\pgfqpoint{4.788646in}{0.597444in}}%
\pgfpathlineto{\pgfqpoint{5.662778in}{0.597444in}}%
\pgfpathquadraticcurveto{\pgfqpoint{5.690556in}{0.597444in}}{\pgfqpoint{5.690556in}{0.625222in}}%
\pgfpathlineto{\pgfqpoint{5.690556in}{2.038334in}}%
\pgfpathquadraticcurveto{\pgfqpoint{5.690556in}{2.066112in}}{\pgfqpoint{5.662778in}{2.066112in}}%
\pgfpathlineto{\pgfqpoint{4.788646in}{2.066112in}}%
\pgfpathquadraticcurveto{\pgfqpoint{4.760868in}{2.066112in}}{\pgfqpoint{4.760868in}{2.038334in}}%
\pgfpathlineto{\pgfqpoint{4.760868in}{0.625222in}}%
\pgfpathquadraticcurveto{\pgfqpoint{4.760868in}{0.597444in}}{\pgfqpoint{4.788646in}{0.597444in}}%
\pgfpathlineto{\pgfqpoint{4.788646in}{0.597444in}}%
\pgfpathclose%
\pgfusepath{stroke,fill}%
\end{pgfscope}%
\begin{pgfscope}%
\pgfsetrectcap%
\pgfsetroundjoin%
\pgfsetlinewidth{1.505625pt}%
\definecolor{currentstroke}{rgb}{0.121569,0.466667,0.705882}%
\pgfsetstrokecolor{currentstroke}%
\pgfsetdash{}{0pt}%
\pgfpathmoveto{\pgfqpoint{4.816424in}{1.953644in}}%
\pgfpathlineto{\pgfqpoint{4.955312in}{1.953644in}}%
\pgfpathlineto{\pgfqpoint{5.094201in}{1.953644in}}%
\pgfusepath{stroke}%
\end{pgfscope}%
\begin{pgfscope}%
\definecolor{textcolor}{rgb}{0.000000,0.000000,0.000000}%
\pgfsetstrokecolor{textcolor}%
\pgfsetfillcolor{textcolor}%
\pgftext[x=5.205312in,y=1.905033in,left,base]{\color{textcolor}\sffamily\fontsize{10.000000}{12.000000}\selectfont 0}%
\end{pgfscope}%
\begin{pgfscope}%
\pgfsetrectcap%
\pgfsetroundjoin%
\pgfsetlinewidth{1.505625pt}%
\definecolor{currentstroke}{rgb}{1.000000,0.498039,0.054902}%
\pgfsetstrokecolor{currentstroke}%
\pgfsetdash{}{0pt}%
\pgfpathmoveto{\pgfqpoint{4.816424in}{1.749787in}}%
\pgfpathlineto{\pgfqpoint{4.955312in}{1.749787in}}%
\pgfpathlineto{\pgfqpoint{5.094201in}{1.749787in}}%
\pgfusepath{stroke}%
\end{pgfscope}%
\begin{pgfscope}%
\definecolor{textcolor}{rgb}{0.000000,0.000000,0.000000}%
\pgfsetstrokecolor{textcolor}%
\pgfsetfillcolor{textcolor}%
\pgftext[x=5.205312in,y=1.701176in,left,base]{\color{textcolor}\sffamily\fontsize{10.000000}{12.000000}\selectfont 5}%
\end{pgfscope}%
\begin{pgfscope}%
\pgfsetrectcap%
\pgfsetroundjoin%
\pgfsetlinewidth{1.505625pt}%
\definecolor{currentstroke}{rgb}{0.172549,0.627451,0.172549}%
\pgfsetstrokecolor{currentstroke}%
\pgfsetdash{}{0pt}%
\pgfpathmoveto{\pgfqpoint{4.816424in}{1.545930in}}%
\pgfpathlineto{\pgfqpoint{4.955312in}{1.545930in}}%
\pgfpathlineto{\pgfqpoint{5.094201in}{1.545930in}}%
\pgfusepath{stroke}%
\end{pgfscope}%
\begin{pgfscope}%
\definecolor{textcolor}{rgb}{0.000000,0.000000,0.000000}%
\pgfsetstrokecolor{textcolor}%
\pgfsetfillcolor{textcolor}%
\pgftext[x=5.205312in,y=1.497319in,left,base]{\color{textcolor}\sffamily\fontsize{10.000000}{12.000000}\selectfont 10}%
\end{pgfscope}%
\begin{pgfscope}%
\pgfsetrectcap%
\pgfsetroundjoin%
\pgfsetlinewidth{1.505625pt}%
\definecolor{currentstroke}{rgb}{0.839216,0.152941,0.156863}%
\pgfsetstrokecolor{currentstroke}%
\pgfsetdash{}{0pt}%
\pgfpathmoveto{\pgfqpoint{4.816424in}{1.342073in}}%
\pgfpathlineto{\pgfqpoint{4.955312in}{1.342073in}}%
\pgfpathlineto{\pgfqpoint{5.094201in}{1.342073in}}%
\pgfusepath{stroke}%
\end{pgfscope}%
\begin{pgfscope}%
\definecolor{textcolor}{rgb}{0.000000,0.000000,0.000000}%
\pgfsetstrokecolor{textcolor}%
\pgfsetfillcolor{textcolor}%
\pgftext[x=5.205312in,y=1.293461in,left,base]{\color{textcolor}\sffamily\fontsize{10.000000}{12.000000}\selectfont 20}%
\end{pgfscope}%
\begin{pgfscope}%
\pgfsetrectcap%
\pgfsetroundjoin%
\pgfsetlinewidth{1.505625pt}%
\definecolor{currentstroke}{rgb}{0.580392,0.403922,0.741176}%
\pgfsetstrokecolor{currentstroke}%
\pgfsetdash{}{0pt}%
\pgfpathmoveto{\pgfqpoint{4.816424in}{1.138215in}}%
\pgfpathlineto{\pgfqpoint{4.955312in}{1.138215in}}%
\pgfpathlineto{\pgfqpoint{5.094201in}{1.138215in}}%
\pgfusepath{stroke}%
\end{pgfscope}%
\begin{pgfscope}%
\definecolor{textcolor}{rgb}{0.000000,0.000000,0.000000}%
\pgfsetstrokecolor{textcolor}%
\pgfsetfillcolor{textcolor}%
\pgftext[x=5.205312in,y=1.089604in,left,base]{\color{textcolor}\sffamily\fontsize{10.000000}{12.000000}\selectfont 40}%
\end{pgfscope}%
\begin{pgfscope}%
\pgfsetrectcap%
\pgfsetroundjoin%
\pgfsetlinewidth{1.505625pt}%
\definecolor{currentstroke}{rgb}{0.549020,0.337255,0.294118}%
\pgfsetstrokecolor{currentstroke}%
\pgfsetdash{}{0pt}%
\pgfpathmoveto{\pgfqpoint{4.816424in}{0.934358in}}%
\pgfpathlineto{\pgfqpoint{4.955312in}{0.934358in}}%
\pgfpathlineto{\pgfqpoint{5.094201in}{0.934358in}}%
\pgfusepath{stroke}%
\end{pgfscope}%
\begin{pgfscope}%
\definecolor{textcolor}{rgb}{0.000000,0.000000,0.000000}%
\pgfsetstrokecolor{textcolor}%
\pgfsetfillcolor{textcolor}%
\pgftext[x=5.205312in,y=0.885747in,left,base]{\color{textcolor}\sffamily\fontsize{10.000000}{12.000000}\selectfont 80}%
\end{pgfscope}%
\begin{pgfscope}%
\pgfsetrectcap%
\pgfsetroundjoin%
\pgfsetlinewidth{1.505625pt}%
\definecolor{currentstroke}{rgb}{0.890196,0.466667,0.760784}%
\pgfsetstrokecolor{currentstroke}%
\pgfsetdash{}{0pt}%
\pgfpathmoveto{\pgfqpoint{4.816424in}{0.730501in}}%
\pgfpathlineto{\pgfqpoint{4.955312in}{0.730501in}}%
\pgfpathlineto{\pgfqpoint{5.094201in}{0.730501in}}%
\pgfusepath{stroke}%
\end{pgfscope}%
\begin{pgfscope}%
\definecolor{textcolor}{rgb}{0.000000,0.000000,0.000000}%
\pgfsetstrokecolor{textcolor}%
\pgfsetfillcolor{textcolor}%
\pgftext[x=5.205312in,y=0.681890in,left,base]{\color{textcolor}\sffamily\fontsize{10.000000}{12.000000}\selectfont Target}%
\end{pgfscope}%
\end{pgfpicture}%
\makeatother%
\endgroup%
}
    \end{minipage}
    \begin{minipage}[t]{0.5\linewidth}
        \centering
        \scalebox{0.55}{%% Creator: Matplotlib, PGF backend
%%
%% To include the figure in your LaTeX document, write
%%   \input{<filename>.pgf}
%%
%% Make sure the required packages are loaded in your preamble
%%   \usepackage{pgf}
%%
%% Also ensure that all the required font packages are loaded; for instance,
%% the lmodern package is sometimes necessary when using math font.
%%   \usepackage{lmodern}
%%
%% Figures using additional raster images can only be included by \input if
%% they are in the same directory as the main LaTeX file. For loading figures
%% from other directories you can use the `import` package
%%   \usepackage{import}
%%
%% and then include the figures with
%%   \import{<path to file>}{<filename>.pgf}
%%
%% Matplotlib used the following preamble
%%   \usepackage{fontspec}
%%   \setmainfont{DejaVuSerif.ttf}[Path=\detokenize{/home/lgonz/tfg-aero/tfg-giaa-dronecontrol/venv/lib/python3.8/site-packages/matplotlib/mpl-data/fonts/ttf/}]
%%   \setsansfont{DejaVuSans.ttf}[Path=\detokenize{/home/lgonz/tfg-aero/tfg-giaa-dronecontrol/venv/lib/python3.8/site-packages/matplotlib/mpl-data/fonts/ttf/}]
%%   \setmonofont{DejaVuSansMono.ttf}[Path=\detokenize{/home/lgonz/tfg-aero/tfg-giaa-dronecontrol/venv/lib/python3.8/site-packages/matplotlib/mpl-data/fonts/ttf/}]
%%
\begingroup%
\makeatletter%
\begin{pgfpicture}%
\pgfpathrectangle{\pgfpointorigin}{\pgfqpoint{6.400000in}{4.800000in}}%
\pgfusepath{use as bounding box, clip}%
\begin{pgfscope}%
\pgfsetbuttcap%
\pgfsetmiterjoin%
\definecolor{currentfill}{rgb}{1.000000,1.000000,1.000000}%
\pgfsetfillcolor{currentfill}%
\pgfsetlinewidth{0.000000pt}%
\definecolor{currentstroke}{rgb}{1.000000,1.000000,1.000000}%
\pgfsetstrokecolor{currentstroke}%
\pgfsetdash{}{0pt}%
\pgfpathmoveto{\pgfqpoint{0.000000in}{0.000000in}}%
\pgfpathlineto{\pgfqpoint{6.400000in}{0.000000in}}%
\pgfpathlineto{\pgfqpoint{6.400000in}{4.800000in}}%
\pgfpathlineto{\pgfqpoint{0.000000in}{4.800000in}}%
\pgfpathlineto{\pgfqpoint{0.000000in}{0.000000in}}%
\pgfpathclose%
\pgfusepath{fill}%
\end{pgfscope}%
\begin{pgfscope}%
\pgfsetbuttcap%
\pgfsetmiterjoin%
\definecolor{currentfill}{rgb}{1.000000,1.000000,1.000000}%
\pgfsetfillcolor{currentfill}%
\pgfsetlinewidth{0.000000pt}%
\definecolor{currentstroke}{rgb}{0.000000,0.000000,0.000000}%
\pgfsetstrokecolor{currentstroke}%
\pgfsetstrokeopacity{0.000000}%
\pgfsetdash{}{0pt}%
\pgfpathmoveto{\pgfqpoint{0.800000in}{0.528000in}}%
\pgfpathlineto{\pgfqpoint{5.760000in}{0.528000in}}%
\pgfpathlineto{\pgfqpoint{5.760000in}{4.224000in}}%
\pgfpathlineto{\pgfqpoint{0.800000in}{4.224000in}}%
\pgfpathlineto{\pgfqpoint{0.800000in}{0.528000in}}%
\pgfpathclose%
\pgfusepath{fill}%
\end{pgfscope}%
\begin{pgfscope}%
\pgfpathrectangle{\pgfqpoint{0.800000in}{0.528000in}}{\pgfqpoint{4.960000in}{3.696000in}}%
\pgfusepath{clip}%
\pgfsetrectcap%
\pgfsetroundjoin%
\pgfsetlinewidth{0.803000pt}%
\definecolor{currentstroke}{rgb}{0.690196,0.690196,0.690196}%
\pgfsetstrokecolor{currentstroke}%
\pgfsetdash{}{0pt}%
\pgfpathmoveto{\pgfqpoint{1.025455in}{0.528000in}}%
\pgfpathlineto{\pgfqpoint{1.025455in}{4.224000in}}%
\pgfusepath{stroke}%
\end{pgfscope}%
\begin{pgfscope}%
\pgfsetbuttcap%
\pgfsetroundjoin%
\definecolor{currentfill}{rgb}{0.000000,0.000000,0.000000}%
\pgfsetfillcolor{currentfill}%
\pgfsetlinewidth{0.803000pt}%
\definecolor{currentstroke}{rgb}{0.000000,0.000000,0.000000}%
\pgfsetstrokecolor{currentstroke}%
\pgfsetdash{}{0pt}%
\pgfsys@defobject{currentmarker}{\pgfqpoint{0.000000in}{-0.048611in}}{\pgfqpoint{0.000000in}{0.000000in}}{%
\pgfpathmoveto{\pgfqpoint{0.000000in}{0.000000in}}%
\pgfpathlineto{\pgfqpoint{0.000000in}{-0.048611in}}%
\pgfusepath{stroke,fill}%
}%
\begin{pgfscope}%
\pgfsys@transformshift{1.025455in}{0.528000in}%
\pgfsys@useobject{currentmarker}{}%
\end{pgfscope}%
\end{pgfscope}%
\begin{pgfscope}%
\definecolor{textcolor}{rgb}{0.000000,0.000000,0.000000}%
\pgfsetstrokecolor{textcolor}%
\pgfsetfillcolor{textcolor}%
\pgftext[x=1.025455in,y=0.430778in,,top]{\color{textcolor}\sffamily\fontsize{10.000000}{12.000000}\selectfont 0}%
\end{pgfscope}%
\begin{pgfscope}%
\pgfpathrectangle{\pgfqpoint{0.800000in}{0.528000in}}{\pgfqpoint{4.960000in}{3.696000in}}%
\pgfusepath{clip}%
\pgfsetrectcap%
\pgfsetroundjoin%
\pgfsetlinewidth{0.803000pt}%
\definecolor{currentstroke}{rgb}{0.690196,0.690196,0.690196}%
\pgfsetstrokecolor{currentstroke}%
\pgfsetdash{}{0pt}%
\pgfpathmoveto{\pgfqpoint{1.775826in}{0.528000in}}%
\pgfpathlineto{\pgfqpoint{1.775826in}{4.224000in}}%
\pgfusepath{stroke}%
\end{pgfscope}%
\begin{pgfscope}%
\pgfsetbuttcap%
\pgfsetroundjoin%
\definecolor{currentfill}{rgb}{0.000000,0.000000,0.000000}%
\pgfsetfillcolor{currentfill}%
\pgfsetlinewidth{0.803000pt}%
\definecolor{currentstroke}{rgb}{0.000000,0.000000,0.000000}%
\pgfsetstrokecolor{currentstroke}%
\pgfsetdash{}{0pt}%
\pgfsys@defobject{currentmarker}{\pgfqpoint{0.000000in}{-0.048611in}}{\pgfqpoint{0.000000in}{0.000000in}}{%
\pgfpathmoveto{\pgfqpoint{0.000000in}{0.000000in}}%
\pgfpathlineto{\pgfqpoint{0.000000in}{-0.048611in}}%
\pgfusepath{stroke,fill}%
}%
\begin{pgfscope}%
\pgfsys@transformshift{1.775826in}{0.528000in}%
\pgfsys@useobject{currentmarker}{}%
\end{pgfscope}%
\end{pgfscope}%
\begin{pgfscope}%
\definecolor{textcolor}{rgb}{0.000000,0.000000,0.000000}%
\pgfsetstrokecolor{textcolor}%
\pgfsetfillcolor{textcolor}%
\pgftext[x=1.775826in,y=0.430778in,,top]{\color{textcolor}\sffamily\fontsize{10.000000}{12.000000}\selectfont 5}%
\end{pgfscope}%
\begin{pgfscope}%
\pgfpathrectangle{\pgfqpoint{0.800000in}{0.528000in}}{\pgfqpoint{4.960000in}{3.696000in}}%
\pgfusepath{clip}%
\pgfsetrectcap%
\pgfsetroundjoin%
\pgfsetlinewidth{0.803000pt}%
\definecolor{currentstroke}{rgb}{0.690196,0.690196,0.690196}%
\pgfsetstrokecolor{currentstroke}%
\pgfsetdash{}{0pt}%
\pgfpathmoveto{\pgfqpoint{2.526198in}{0.528000in}}%
\pgfpathlineto{\pgfqpoint{2.526198in}{4.224000in}}%
\pgfusepath{stroke}%
\end{pgfscope}%
\begin{pgfscope}%
\pgfsetbuttcap%
\pgfsetroundjoin%
\definecolor{currentfill}{rgb}{0.000000,0.000000,0.000000}%
\pgfsetfillcolor{currentfill}%
\pgfsetlinewidth{0.803000pt}%
\definecolor{currentstroke}{rgb}{0.000000,0.000000,0.000000}%
\pgfsetstrokecolor{currentstroke}%
\pgfsetdash{}{0pt}%
\pgfsys@defobject{currentmarker}{\pgfqpoint{0.000000in}{-0.048611in}}{\pgfqpoint{0.000000in}{0.000000in}}{%
\pgfpathmoveto{\pgfqpoint{0.000000in}{0.000000in}}%
\pgfpathlineto{\pgfqpoint{0.000000in}{-0.048611in}}%
\pgfusepath{stroke,fill}%
}%
\begin{pgfscope}%
\pgfsys@transformshift{2.526198in}{0.528000in}%
\pgfsys@useobject{currentmarker}{}%
\end{pgfscope}%
\end{pgfscope}%
\begin{pgfscope}%
\definecolor{textcolor}{rgb}{0.000000,0.000000,0.000000}%
\pgfsetstrokecolor{textcolor}%
\pgfsetfillcolor{textcolor}%
\pgftext[x=2.526198in,y=0.430778in,,top]{\color{textcolor}\sffamily\fontsize{10.000000}{12.000000}\selectfont 10}%
\end{pgfscope}%
\begin{pgfscope}%
\pgfpathrectangle{\pgfqpoint{0.800000in}{0.528000in}}{\pgfqpoint{4.960000in}{3.696000in}}%
\pgfusepath{clip}%
\pgfsetrectcap%
\pgfsetroundjoin%
\pgfsetlinewidth{0.803000pt}%
\definecolor{currentstroke}{rgb}{0.690196,0.690196,0.690196}%
\pgfsetstrokecolor{currentstroke}%
\pgfsetdash{}{0pt}%
\pgfpathmoveto{\pgfqpoint{3.276570in}{0.528000in}}%
\pgfpathlineto{\pgfqpoint{3.276570in}{4.224000in}}%
\pgfusepath{stroke}%
\end{pgfscope}%
\begin{pgfscope}%
\pgfsetbuttcap%
\pgfsetroundjoin%
\definecolor{currentfill}{rgb}{0.000000,0.000000,0.000000}%
\pgfsetfillcolor{currentfill}%
\pgfsetlinewidth{0.803000pt}%
\definecolor{currentstroke}{rgb}{0.000000,0.000000,0.000000}%
\pgfsetstrokecolor{currentstroke}%
\pgfsetdash{}{0pt}%
\pgfsys@defobject{currentmarker}{\pgfqpoint{0.000000in}{-0.048611in}}{\pgfqpoint{0.000000in}{0.000000in}}{%
\pgfpathmoveto{\pgfqpoint{0.000000in}{0.000000in}}%
\pgfpathlineto{\pgfqpoint{0.000000in}{-0.048611in}}%
\pgfusepath{stroke,fill}%
}%
\begin{pgfscope}%
\pgfsys@transformshift{3.276570in}{0.528000in}%
\pgfsys@useobject{currentmarker}{}%
\end{pgfscope}%
\end{pgfscope}%
\begin{pgfscope}%
\definecolor{textcolor}{rgb}{0.000000,0.000000,0.000000}%
\pgfsetstrokecolor{textcolor}%
\pgfsetfillcolor{textcolor}%
\pgftext[x=3.276570in,y=0.430778in,,top]{\color{textcolor}\sffamily\fontsize{10.000000}{12.000000}\selectfont 15}%
\end{pgfscope}%
\begin{pgfscope}%
\pgfpathrectangle{\pgfqpoint{0.800000in}{0.528000in}}{\pgfqpoint{4.960000in}{3.696000in}}%
\pgfusepath{clip}%
\pgfsetrectcap%
\pgfsetroundjoin%
\pgfsetlinewidth{0.803000pt}%
\definecolor{currentstroke}{rgb}{0.690196,0.690196,0.690196}%
\pgfsetstrokecolor{currentstroke}%
\pgfsetdash{}{0pt}%
\pgfpathmoveto{\pgfqpoint{4.026941in}{0.528000in}}%
\pgfpathlineto{\pgfqpoint{4.026941in}{4.224000in}}%
\pgfusepath{stroke}%
\end{pgfscope}%
\begin{pgfscope}%
\pgfsetbuttcap%
\pgfsetroundjoin%
\definecolor{currentfill}{rgb}{0.000000,0.000000,0.000000}%
\pgfsetfillcolor{currentfill}%
\pgfsetlinewidth{0.803000pt}%
\definecolor{currentstroke}{rgb}{0.000000,0.000000,0.000000}%
\pgfsetstrokecolor{currentstroke}%
\pgfsetdash{}{0pt}%
\pgfsys@defobject{currentmarker}{\pgfqpoint{0.000000in}{-0.048611in}}{\pgfqpoint{0.000000in}{0.000000in}}{%
\pgfpathmoveto{\pgfqpoint{0.000000in}{0.000000in}}%
\pgfpathlineto{\pgfqpoint{0.000000in}{-0.048611in}}%
\pgfusepath{stroke,fill}%
}%
\begin{pgfscope}%
\pgfsys@transformshift{4.026941in}{0.528000in}%
\pgfsys@useobject{currentmarker}{}%
\end{pgfscope}%
\end{pgfscope}%
\begin{pgfscope}%
\definecolor{textcolor}{rgb}{0.000000,0.000000,0.000000}%
\pgfsetstrokecolor{textcolor}%
\pgfsetfillcolor{textcolor}%
\pgftext[x=4.026941in,y=0.430778in,,top]{\color{textcolor}\sffamily\fontsize{10.000000}{12.000000}\selectfont 20}%
\end{pgfscope}%
\begin{pgfscope}%
\pgfpathrectangle{\pgfqpoint{0.800000in}{0.528000in}}{\pgfqpoint{4.960000in}{3.696000in}}%
\pgfusepath{clip}%
\pgfsetrectcap%
\pgfsetroundjoin%
\pgfsetlinewidth{0.803000pt}%
\definecolor{currentstroke}{rgb}{0.690196,0.690196,0.690196}%
\pgfsetstrokecolor{currentstroke}%
\pgfsetdash{}{0pt}%
\pgfpathmoveto{\pgfqpoint{4.777313in}{0.528000in}}%
\pgfpathlineto{\pgfqpoint{4.777313in}{4.224000in}}%
\pgfusepath{stroke}%
\end{pgfscope}%
\begin{pgfscope}%
\pgfsetbuttcap%
\pgfsetroundjoin%
\definecolor{currentfill}{rgb}{0.000000,0.000000,0.000000}%
\pgfsetfillcolor{currentfill}%
\pgfsetlinewidth{0.803000pt}%
\definecolor{currentstroke}{rgb}{0.000000,0.000000,0.000000}%
\pgfsetstrokecolor{currentstroke}%
\pgfsetdash{}{0pt}%
\pgfsys@defobject{currentmarker}{\pgfqpoint{0.000000in}{-0.048611in}}{\pgfqpoint{0.000000in}{0.000000in}}{%
\pgfpathmoveto{\pgfqpoint{0.000000in}{0.000000in}}%
\pgfpathlineto{\pgfqpoint{0.000000in}{-0.048611in}}%
\pgfusepath{stroke,fill}%
}%
\begin{pgfscope}%
\pgfsys@transformshift{4.777313in}{0.528000in}%
\pgfsys@useobject{currentmarker}{}%
\end{pgfscope}%
\end{pgfscope}%
\begin{pgfscope}%
\definecolor{textcolor}{rgb}{0.000000,0.000000,0.000000}%
\pgfsetstrokecolor{textcolor}%
\pgfsetfillcolor{textcolor}%
\pgftext[x=4.777313in,y=0.430778in,,top]{\color{textcolor}\sffamily\fontsize{10.000000}{12.000000}\selectfont 25}%
\end{pgfscope}%
\begin{pgfscope}%
\pgfpathrectangle{\pgfqpoint{0.800000in}{0.528000in}}{\pgfqpoint{4.960000in}{3.696000in}}%
\pgfusepath{clip}%
\pgfsetrectcap%
\pgfsetroundjoin%
\pgfsetlinewidth{0.803000pt}%
\definecolor{currentstroke}{rgb}{0.690196,0.690196,0.690196}%
\pgfsetstrokecolor{currentstroke}%
\pgfsetdash{}{0pt}%
\pgfpathmoveto{\pgfqpoint{5.527685in}{0.528000in}}%
\pgfpathlineto{\pgfqpoint{5.527685in}{4.224000in}}%
\pgfusepath{stroke}%
\end{pgfscope}%
\begin{pgfscope}%
\pgfsetbuttcap%
\pgfsetroundjoin%
\definecolor{currentfill}{rgb}{0.000000,0.000000,0.000000}%
\pgfsetfillcolor{currentfill}%
\pgfsetlinewidth{0.803000pt}%
\definecolor{currentstroke}{rgb}{0.000000,0.000000,0.000000}%
\pgfsetstrokecolor{currentstroke}%
\pgfsetdash{}{0pt}%
\pgfsys@defobject{currentmarker}{\pgfqpoint{0.000000in}{-0.048611in}}{\pgfqpoint{0.000000in}{0.000000in}}{%
\pgfpathmoveto{\pgfqpoint{0.000000in}{0.000000in}}%
\pgfpathlineto{\pgfqpoint{0.000000in}{-0.048611in}}%
\pgfusepath{stroke,fill}%
}%
\begin{pgfscope}%
\pgfsys@transformshift{5.527685in}{0.528000in}%
\pgfsys@useobject{currentmarker}{}%
\end{pgfscope}%
\end{pgfscope}%
\begin{pgfscope}%
\definecolor{textcolor}{rgb}{0.000000,0.000000,0.000000}%
\pgfsetstrokecolor{textcolor}%
\pgfsetfillcolor{textcolor}%
\pgftext[x=5.527685in,y=0.430778in,,top]{\color{textcolor}\sffamily\fontsize{10.000000}{12.000000}\selectfont 30}%
\end{pgfscope}%
\begin{pgfscope}%
\definecolor{textcolor}{rgb}{0.000000,0.000000,0.000000}%
\pgfsetstrokecolor{textcolor}%
\pgfsetfillcolor{textcolor}%
\pgftext[x=3.280000in,y=0.240809in,,top]{\color{textcolor}\sffamily\fontsize{10.000000}{12.000000}\selectfont time [s]}%
\end{pgfscope}%
\begin{pgfscope}%
\pgfpathrectangle{\pgfqpoint{0.800000in}{0.528000in}}{\pgfqpoint{4.960000in}{3.696000in}}%
\pgfusepath{clip}%
\pgfsetrectcap%
\pgfsetroundjoin%
\pgfsetlinewidth{0.803000pt}%
\definecolor{currentstroke}{rgb}{0.690196,0.690196,0.690196}%
\pgfsetstrokecolor{currentstroke}%
\pgfsetdash{}{0pt}%
\pgfpathmoveto{\pgfqpoint{0.800000in}{1.066182in}}%
\pgfpathlineto{\pgfqpoint{5.760000in}{1.066182in}}%
\pgfusepath{stroke}%
\end{pgfscope}%
\begin{pgfscope}%
\pgfsetbuttcap%
\pgfsetroundjoin%
\definecolor{currentfill}{rgb}{0.000000,0.000000,0.000000}%
\pgfsetfillcolor{currentfill}%
\pgfsetlinewidth{0.803000pt}%
\definecolor{currentstroke}{rgb}{0.000000,0.000000,0.000000}%
\pgfsetstrokecolor{currentstroke}%
\pgfsetdash{}{0pt}%
\pgfsys@defobject{currentmarker}{\pgfqpoint{-0.048611in}{0.000000in}}{\pgfqpoint{-0.000000in}{0.000000in}}{%
\pgfpathmoveto{\pgfqpoint{-0.000000in}{0.000000in}}%
\pgfpathlineto{\pgfqpoint{-0.048611in}{0.000000in}}%
\pgfusepath{stroke,fill}%
}%
\begin{pgfscope}%
\pgfsys@transformshift{0.800000in}{1.066182in}%
\pgfsys@useobject{currentmarker}{}%
\end{pgfscope}%
\end{pgfscope}%
\begin{pgfscope}%
\definecolor{textcolor}{rgb}{0.000000,0.000000,0.000000}%
\pgfsetstrokecolor{textcolor}%
\pgfsetfillcolor{textcolor}%
\pgftext[x=0.506387in, y=1.013420in, left, base]{\color{textcolor}\sffamily\fontsize{10.000000}{12.000000}\selectfont \ensuremath{-}2}%
\end{pgfscope}%
\begin{pgfscope}%
\pgfpathrectangle{\pgfqpoint{0.800000in}{0.528000in}}{\pgfqpoint{4.960000in}{3.696000in}}%
\pgfusepath{clip}%
\pgfsetrectcap%
\pgfsetroundjoin%
\pgfsetlinewidth{0.803000pt}%
\definecolor{currentstroke}{rgb}{0.690196,0.690196,0.690196}%
\pgfsetstrokecolor{currentstroke}%
\pgfsetdash{}{0pt}%
\pgfpathmoveto{\pgfqpoint{0.800000in}{1.851285in}}%
\pgfpathlineto{\pgfqpoint{5.760000in}{1.851285in}}%
\pgfusepath{stroke}%
\end{pgfscope}%
\begin{pgfscope}%
\pgfsetbuttcap%
\pgfsetroundjoin%
\definecolor{currentfill}{rgb}{0.000000,0.000000,0.000000}%
\pgfsetfillcolor{currentfill}%
\pgfsetlinewidth{0.803000pt}%
\definecolor{currentstroke}{rgb}{0.000000,0.000000,0.000000}%
\pgfsetstrokecolor{currentstroke}%
\pgfsetdash{}{0pt}%
\pgfsys@defobject{currentmarker}{\pgfqpoint{-0.048611in}{0.000000in}}{\pgfqpoint{-0.000000in}{0.000000in}}{%
\pgfpathmoveto{\pgfqpoint{-0.000000in}{0.000000in}}%
\pgfpathlineto{\pgfqpoint{-0.048611in}{0.000000in}}%
\pgfusepath{stroke,fill}%
}%
\begin{pgfscope}%
\pgfsys@transformshift{0.800000in}{1.851285in}%
\pgfsys@useobject{currentmarker}{}%
\end{pgfscope}%
\end{pgfscope}%
\begin{pgfscope}%
\definecolor{textcolor}{rgb}{0.000000,0.000000,0.000000}%
\pgfsetstrokecolor{textcolor}%
\pgfsetfillcolor{textcolor}%
\pgftext[x=0.614412in, y=1.798523in, left, base]{\color{textcolor}\sffamily\fontsize{10.000000}{12.000000}\selectfont 0}%
\end{pgfscope}%
\begin{pgfscope}%
\pgfpathrectangle{\pgfqpoint{0.800000in}{0.528000in}}{\pgfqpoint{4.960000in}{3.696000in}}%
\pgfusepath{clip}%
\pgfsetrectcap%
\pgfsetroundjoin%
\pgfsetlinewidth{0.803000pt}%
\definecolor{currentstroke}{rgb}{0.690196,0.690196,0.690196}%
\pgfsetstrokecolor{currentstroke}%
\pgfsetdash{}{0pt}%
\pgfpathmoveto{\pgfqpoint{0.800000in}{2.636387in}}%
\pgfpathlineto{\pgfqpoint{5.760000in}{2.636387in}}%
\pgfusepath{stroke}%
\end{pgfscope}%
\begin{pgfscope}%
\pgfsetbuttcap%
\pgfsetroundjoin%
\definecolor{currentfill}{rgb}{0.000000,0.000000,0.000000}%
\pgfsetfillcolor{currentfill}%
\pgfsetlinewidth{0.803000pt}%
\definecolor{currentstroke}{rgb}{0.000000,0.000000,0.000000}%
\pgfsetstrokecolor{currentstroke}%
\pgfsetdash{}{0pt}%
\pgfsys@defobject{currentmarker}{\pgfqpoint{-0.048611in}{0.000000in}}{\pgfqpoint{-0.000000in}{0.000000in}}{%
\pgfpathmoveto{\pgfqpoint{-0.000000in}{0.000000in}}%
\pgfpathlineto{\pgfqpoint{-0.048611in}{0.000000in}}%
\pgfusepath{stroke,fill}%
}%
\begin{pgfscope}%
\pgfsys@transformshift{0.800000in}{2.636387in}%
\pgfsys@useobject{currentmarker}{}%
\end{pgfscope}%
\end{pgfscope}%
\begin{pgfscope}%
\definecolor{textcolor}{rgb}{0.000000,0.000000,0.000000}%
\pgfsetstrokecolor{textcolor}%
\pgfsetfillcolor{textcolor}%
\pgftext[x=0.614412in, y=2.583626in, left, base]{\color{textcolor}\sffamily\fontsize{10.000000}{12.000000}\selectfont 2}%
\end{pgfscope}%
\begin{pgfscope}%
\pgfpathrectangle{\pgfqpoint{0.800000in}{0.528000in}}{\pgfqpoint{4.960000in}{3.696000in}}%
\pgfusepath{clip}%
\pgfsetrectcap%
\pgfsetroundjoin%
\pgfsetlinewidth{0.803000pt}%
\definecolor{currentstroke}{rgb}{0.690196,0.690196,0.690196}%
\pgfsetstrokecolor{currentstroke}%
\pgfsetdash{}{0pt}%
\pgfpathmoveto{\pgfqpoint{0.800000in}{3.421490in}}%
\pgfpathlineto{\pgfqpoint{5.760000in}{3.421490in}}%
\pgfusepath{stroke}%
\end{pgfscope}%
\begin{pgfscope}%
\pgfsetbuttcap%
\pgfsetroundjoin%
\definecolor{currentfill}{rgb}{0.000000,0.000000,0.000000}%
\pgfsetfillcolor{currentfill}%
\pgfsetlinewidth{0.803000pt}%
\definecolor{currentstroke}{rgb}{0.000000,0.000000,0.000000}%
\pgfsetstrokecolor{currentstroke}%
\pgfsetdash{}{0pt}%
\pgfsys@defobject{currentmarker}{\pgfqpoint{-0.048611in}{0.000000in}}{\pgfqpoint{-0.000000in}{0.000000in}}{%
\pgfpathmoveto{\pgfqpoint{-0.000000in}{0.000000in}}%
\pgfpathlineto{\pgfqpoint{-0.048611in}{0.000000in}}%
\pgfusepath{stroke,fill}%
}%
\begin{pgfscope}%
\pgfsys@transformshift{0.800000in}{3.421490in}%
\pgfsys@useobject{currentmarker}{}%
\end{pgfscope}%
\end{pgfscope}%
\begin{pgfscope}%
\definecolor{textcolor}{rgb}{0.000000,0.000000,0.000000}%
\pgfsetstrokecolor{textcolor}%
\pgfsetfillcolor{textcolor}%
\pgftext[x=0.614412in, y=3.368728in, left, base]{\color{textcolor}\sffamily\fontsize{10.000000}{12.000000}\selectfont 4}%
\end{pgfscope}%
\begin{pgfscope}%
\pgfpathrectangle{\pgfqpoint{0.800000in}{0.528000in}}{\pgfqpoint{4.960000in}{3.696000in}}%
\pgfusepath{clip}%
\pgfsetrectcap%
\pgfsetroundjoin%
\pgfsetlinewidth{0.803000pt}%
\definecolor{currentstroke}{rgb}{0.690196,0.690196,0.690196}%
\pgfsetstrokecolor{currentstroke}%
\pgfsetdash{}{0pt}%
\pgfpathmoveto{\pgfqpoint{0.800000in}{4.206592in}}%
\pgfpathlineto{\pgfqpoint{5.760000in}{4.206592in}}%
\pgfusepath{stroke}%
\end{pgfscope}%
\begin{pgfscope}%
\pgfsetbuttcap%
\pgfsetroundjoin%
\definecolor{currentfill}{rgb}{0.000000,0.000000,0.000000}%
\pgfsetfillcolor{currentfill}%
\pgfsetlinewidth{0.803000pt}%
\definecolor{currentstroke}{rgb}{0.000000,0.000000,0.000000}%
\pgfsetstrokecolor{currentstroke}%
\pgfsetdash{}{0pt}%
\pgfsys@defobject{currentmarker}{\pgfqpoint{-0.048611in}{0.000000in}}{\pgfqpoint{-0.000000in}{0.000000in}}{%
\pgfpathmoveto{\pgfqpoint{-0.000000in}{0.000000in}}%
\pgfpathlineto{\pgfqpoint{-0.048611in}{0.000000in}}%
\pgfusepath{stroke,fill}%
}%
\begin{pgfscope}%
\pgfsys@transformshift{0.800000in}{4.206592in}%
\pgfsys@useobject{currentmarker}{}%
\end{pgfscope}%
\end{pgfscope}%
\begin{pgfscope}%
\definecolor{textcolor}{rgb}{0.000000,0.000000,0.000000}%
\pgfsetstrokecolor{textcolor}%
\pgfsetfillcolor{textcolor}%
\pgftext[x=0.614412in, y=4.153831in, left, base]{\color{textcolor}\sffamily\fontsize{10.000000}{12.000000}\selectfont 6}%
\end{pgfscope}%
\begin{pgfscope}%
\definecolor{textcolor}{rgb}{0.000000,0.000000,0.000000}%
\pgfsetstrokecolor{textcolor}%
\pgfsetfillcolor{textcolor}%
\pgftext[x=0.450832in,y=2.376000in,,bottom,rotate=90.000000]{\color{textcolor}\sffamily\fontsize{10.000000}{12.000000}\selectfont Velocity [deg/s]}%
\end{pgfscope}%
\begin{pgfscope}%
\pgfpathrectangle{\pgfqpoint{0.800000in}{0.528000in}}{\pgfqpoint{4.960000in}{3.696000in}}%
\pgfusepath{clip}%
\pgfsetrectcap%
\pgfsetroundjoin%
\pgfsetlinewidth{1.505625pt}%
\definecolor{currentstroke}{rgb}{0.121569,0.466667,0.705882}%
\pgfsetstrokecolor{currentstroke}%
\pgfsetdash{}{0pt}%
\pgfpathmoveto{\pgfqpoint{1.025455in}{2.023624in}}%
\pgfpathlineto{\pgfqpoint{1.079817in}{3.048629in}}%
\pgfpathlineto{\pgfqpoint{1.133923in}{3.579715in}}%
\pgfpathlineto{\pgfqpoint{1.188290in}{3.901797in}}%
\pgfpathlineto{\pgfqpoint{1.242517in}{3.983475in}}%
\pgfpathlineto{\pgfqpoint{1.296728in}{4.032663in}}%
\pgfpathlineto{\pgfqpoint{1.351077in}{4.051810in}}%
\pgfpathlineto{\pgfqpoint{1.406904in}{3.965488in}}%
\pgfpathlineto{\pgfqpoint{1.459894in}{3.530408in}}%
\pgfpathlineto{\pgfqpoint{1.514166in}{2.951894in}}%
\pgfpathlineto{\pgfqpoint{1.568065in}{2.329901in}}%
\pgfpathlineto{\pgfqpoint{1.621995in}{1.884705in}}%
\pgfpathlineto{\pgfqpoint{1.678250in}{1.684432in}}%
\pgfpathlineto{\pgfqpoint{1.730619in}{1.609200in}}%
\pgfpathlineto{\pgfqpoint{1.784562in}{1.590544in}}%
\pgfpathlineto{\pgfqpoint{1.840736in}{1.631310in}}%
\pgfpathlineto{\pgfqpoint{1.893198in}{1.723257in}}%
\pgfpathlineto{\pgfqpoint{1.950363in}{1.829371in}}%
\pgfpathlineto{\pgfqpoint{2.004186in}{1.914860in}}%
\pgfpathlineto{\pgfqpoint{2.057916in}{1.970637in}}%
\pgfpathlineto{\pgfqpoint{2.111602in}{2.007248in}}%
\pgfpathlineto{\pgfqpoint{2.166266in}{1.947526in}}%
\pgfpathlineto{\pgfqpoint{2.221090in}{1.888039in}}%
\pgfpathlineto{\pgfqpoint{2.274901in}{1.846749in}}%
\pgfpathlineto{\pgfqpoint{2.329511in}{1.837513in}}%
\pgfpathlineto{\pgfqpoint{2.383896in}{1.878939in}}%
\pgfpathlineto{\pgfqpoint{2.437984in}{1.793388in}}%
\pgfpathlineto{\pgfqpoint{2.492379in}{1.844492in}}%
\pgfpathlineto{\pgfqpoint{2.546770in}{1.820555in}}%
\pgfpathlineto{\pgfqpoint{2.603287in}{1.816562in}}%
\pgfpathlineto{\pgfqpoint{2.656638in}{1.780782in}}%
\pgfpathlineto{\pgfqpoint{2.710391in}{1.848934in}}%
\pgfpathlineto{\pgfqpoint{2.764488in}{1.754400in}}%
\pgfpathlineto{\pgfqpoint{2.818536in}{1.710790in}}%
\pgfpathlineto{\pgfqpoint{2.872961in}{1.795770in}}%
\pgfpathlineto{\pgfqpoint{2.927076in}{1.865388in}}%
\pgfpathlineto{\pgfqpoint{2.981116in}{1.875012in}}%
\pgfpathlineto{\pgfqpoint{3.037525in}{1.897882in}}%
\pgfpathlineto{\pgfqpoint{3.090490in}{1.903296in}}%
\pgfpathlineto{\pgfqpoint{3.144707in}{1.892757in}}%
\pgfpathlineto{\pgfqpoint{3.199894in}{1.858024in}}%
\pgfpathlineto{\pgfqpoint{3.253338in}{1.844800in}}%
\pgfpathlineto{\pgfqpoint{3.307116in}{1.830467in}}%
\pgfpathlineto{\pgfqpoint{3.361232in}{1.786298in}}%
\pgfpathlineto{\pgfqpoint{3.415329in}{1.806345in}}%
\pgfpathlineto{\pgfqpoint{3.469664in}{1.781010in}}%
\pgfpathlineto{\pgfqpoint{3.523439in}{1.806148in}}%
\pgfpathlineto{\pgfqpoint{3.578614in}{1.796738in}}%
\pgfpathlineto{\pgfqpoint{3.632675in}{1.825178in}}%
\pgfpathlineto{\pgfqpoint{3.686967in}{1.900136in}}%
\pgfpathlineto{\pgfqpoint{3.741763in}{1.963903in}}%
\pgfpathlineto{\pgfqpoint{3.795833in}{1.913849in}}%
\pgfpathlineto{\pgfqpoint{3.851525in}{1.902211in}}%
\pgfpathlineto{\pgfqpoint{3.905219in}{1.715870in}}%
\pgfpathlineto{\pgfqpoint{3.958767in}{1.704902in}}%
\pgfpathlineto{\pgfqpoint{4.013311in}{1.803754in}}%
\pgfpathlineto{\pgfqpoint{4.067422in}{1.886600in}}%
\pgfpathlineto{\pgfqpoint{4.121571in}{1.913487in}}%
\pgfpathlineto{\pgfqpoint{4.175734in}{1.927544in}}%
\pgfpathlineto{\pgfqpoint{4.229785in}{1.957664in}}%
\pgfpathlineto{\pgfqpoint{4.284440in}{1.976421in}}%
\pgfpathlineto{\pgfqpoint{4.338666in}{1.947088in}}%
\pgfpathlineto{\pgfqpoint{4.393092in}{1.929324in}}%
\pgfpathlineto{\pgfqpoint{4.446864in}{1.918275in}}%
\pgfpathlineto{\pgfqpoint{4.502372in}{1.896950in}}%
\pgfpathlineto{\pgfqpoint{4.557166in}{1.911671in}}%
\pgfpathlineto{\pgfqpoint{4.610727in}{1.924180in}}%
\pgfpathlineto{\pgfqpoint{4.665053in}{1.856908in}}%
\pgfpathlineto{\pgfqpoint{4.719746in}{1.836170in}}%
\pgfpathlineto{\pgfqpoint{4.773889in}{1.848896in}}%
\pgfpathlineto{\pgfqpoint{4.828083in}{1.784995in}}%
\pgfpathlineto{\pgfqpoint{4.882083in}{1.795491in}}%
\pgfpathlineto{\pgfqpoint{4.936499in}{1.740906in}}%
\pgfpathlineto{\pgfqpoint{4.990710in}{1.822963in}}%
\pgfpathlineto{\pgfqpoint{5.045041in}{1.862400in}}%
\pgfpathlineto{\pgfqpoint{5.100981in}{1.867640in}}%
\pgfpathlineto{\pgfqpoint{5.154170in}{1.869576in}}%
\pgfpathlineto{\pgfqpoint{5.207988in}{1.876581in}}%
\pgfpathlineto{\pgfqpoint{5.261945in}{1.887481in}}%
\pgfpathlineto{\pgfqpoint{5.316693in}{1.871638in}}%
\pgfpathlineto{\pgfqpoint{5.370897in}{1.873277in}}%
\pgfpathlineto{\pgfqpoint{5.424825in}{1.845186in}}%
\pgfpathlineto{\pgfqpoint{5.479200in}{1.837335in}}%
\pgfpathlineto{\pgfqpoint{5.534545in}{1.872109in}}%
\pgfusepath{stroke}%
\end{pgfscope}%
\begin{pgfscope}%
\pgfpathrectangle{\pgfqpoint{0.800000in}{0.528000in}}{\pgfqpoint{4.960000in}{3.696000in}}%
\pgfusepath{clip}%
\pgfsetrectcap%
\pgfsetroundjoin%
\pgfsetlinewidth{1.505625pt}%
\definecolor{currentstroke}{rgb}{1.000000,0.498039,0.054902}%
\pgfsetstrokecolor{currentstroke}%
\pgfsetdash{}{0pt}%
\pgfpathmoveto{\pgfqpoint{1.025455in}{2.002401in}}%
\pgfpathlineto{\pgfqpoint{1.080220in}{3.086123in}}%
\pgfpathlineto{\pgfqpoint{1.134585in}{3.632807in}}%
\pgfpathlineto{\pgfqpoint{1.189826in}{3.880690in}}%
\pgfpathlineto{\pgfqpoint{1.243773in}{4.007798in}}%
\pgfpathlineto{\pgfqpoint{1.297489in}{4.042561in}}%
\pgfpathlineto{\pgfqpoint{1.351929in}{4.038607in}}%
\pgfpathlineto{\pgfqpoint{1.406435in}{3.978211in}}%
\pgfpathlineto{\pgfqpoint{1.460887in}{3.905617in}}%
\pgfpathlineto{\pgfqpoint{1.515248in}{3.534129in}}%
\pgfpathlineto{\pgfqpoint{1.569183in}{2.875245in}}%
\pgfpathlineto{\pgfqpoint{1.623482in}{2.258486in}}%
\pgfpathlineto{\pgfqpoint{1.677765in}{1.818802in}}%
\pgfpathlineto{\pgfqpoint{1.731972in}{1.524261in}}%
\pgfpathlineto{\pgfqpoint{1.788130in}{1.335016in}}%
\pgfpathlineto{\pgfqpoint{1.841967in}{1.296344in}}%
\pgfpathlineto{\pgfqpoint{1.895677in}{1.456581in}}%
\pgfpathlineto{\pgfqpoint{1.950222in}{1.641471in}}%
\pgfpathlineto{\pgfqpoint{2.004201in}{1.793125in}}%
\pgfpathlineto{\pgfqpoint{2.058649in}{1.986941in}}%
\pgfpathlineto{\pgfqpoint{2.112810in}{2.024245in}}%
\pgfpathlineto{\pgfqpoint{2.167018in}{1.988411in}}%
\pgfpathlineto{\pgfqpoint{2.221126in}{1.960710in}}%
\pgfpathlineto{\pgfqpoint{2.275560in}{1.929874in}}%
\pgfpathlineto{\pgfqpoint{2.329897in}{1.907781in}}%
\pgfpathlineto{\pgfqpoint{2.384139in}{1.860532in}}%
\pgfpathlineto{\pgfqpoint{2.440283in}{1.834873in}}%
\pgfpathlineto{\pgfqpoint{2.494007in}{1.860312in}}%
\pgfpathlineto{\pgfqpoint{2.547589in}{1.815169in}}%
\pgfpathlineto{\pgfqpoint{2.601548in}{1.789059in}}%
\pgfpathlineto{\pgfqpoint{2.655608in}{1.735171in}}%
\pgfpathlineto{\pgfqpoint{2.710059in}{1.751212in}}%
\pgfpathlineto{\pgfqpoint{2.764074in}{1.762770in}}%
\pgfpathlineto{\pgfqpoint{2.818221in}{1.805985in}}%
\pgfpathlineto{\pgfqpoint{2.872747in}{1.740636in}}%
\pgfpathlineto{\pgfqpoint{2.927224in}{1.679714in}}%
\pgfpathlineto{\pgfqpoint{2.981679in}{1.607618in}}%
\pgfpathlineto{\pgfqpoint{3.036127in}{1.578093in}}%
\pgfpathlineto{\pgfqpoint{3.092208in}{1.617140in}}%
\pgfpathlineto{\pgfqpoint{3.145417in}{1.647896in}}%
\pgfpathlineto{\pgfqpoint{3.198933in}{1.826963in}}%
\pgfpathlineto{\pgfqpoint{3.253261in}{1.814251in}}%
\pgfpathlineto{\pgfqpoint{3.307191in}{1.834901in}}%
\pgfpathlineto{\pgfqpoint{3.361257in}{1.871827in}}%
\pgfpathlineto{\pgfqpoint{3.415719in}{1.956433in}}%
\pgfpathlineto{\pgfqpoint{3.470548in}{1.933431in}}%
\pgfpathlineto{\pgfqpoint{3.524664in}{1.920064in}}%
\pgfpathlineto{\pgfqpoint{3.578813in}{1.913494in}}%
\pgfpathlineto{\pgfqpoint{3.634790in}{1.877862in}}%
\pgfpathlineto{\pgfqpoint{3.688072in}{1.859891in}}%
\pgfpathlineto{\pgfqpoint{3.741894in}{1.832858in}}%
\pgfpathlineto{\pgfqpoint{3.796095in}{1.817295in}}%
\pgfpathlineto{\pgfqpoint{3.850161in}{1.772846in}}%
\pgfpathlineto{\pgfqpoint{3.904446in}{1.780790in}}%
\pgfpathlineto{\pgfqpoint{3.958348in}{1.759257in}}%
\pgfpathlineto{\pgfqpoint{4.012420in}{1.815043in}}%
\pgfpathlineto{\pgfqpoint{4.066944in}{1.826193in}}%
\pgfpathlineto{\pgfqpoint{4.121231in}{1.785247in}}%
\pgfpathlineto{\pgfqpoint{4.175678in}{1.713210in}}%
\pgfpathlineto{\pgfqpoint{4.231433in}{1.744038in}}%
\pgfpathlineto{\pgfqpoint{4.284972in}{1.863284in}}%
\pgfpathlineto{\pgfqpoint{4.338915in}{1.906945in}}%
\pgfpathlineto{\pgfqpoint{4.392959in}{1.942526in}}%
\pgfpathlineto{\pgfqpoint{4.446948in}{1.942970in}}%
\pgfpathlineto{\pgfqpoint{4.501291in}{1.914121in}}%
\pgfpathlineto{\pgfqpoint{4.555287in}{1.952112in}}%
\pgfpathlineto{\pgfqpoint{4.609571in}{1.902257in}}%
\pgfpathlineto{\pgfqpoint{4.663653in}{1.885471in}}%
\pgfpathlineto{\pgfqpoint{4.718127in}{1.815616in}}%
\pgfpathlineto{\pgfqpoint{4.772528in}{1.835153in}}%
\pgfpathlineto{\pgfqpoint{4.826943in}{1.876646in}}%
\pgfpathlineto{\pgfqpoint{4.882287in}{1.860248in}}%
\pgfpathlineto{\pgfqpoint{4.936382in}{1.931492in}}%
\pgfpathlineto{\pgfqpoint{4.990063in}{1.972480in}}%
\pgfpathlineto{\pgfqpoint{5.044235in}{1.982155in}}%
\pgfpathlineto{\pgfqpoint{5.098168in}{1.831341in}}%
\pgfpathlineto{\pgfqpoint{5.152252in}{1.779606in}}%
\pgfpathlineto{\pgfqpoint{5.206756in}{1.750872in}}%
\pgfpathlineto{\pgfqpoint{5.261091in}{1.770139in}}%
\pgfpathlineto{\pgfqpoint{5.315426in}{1.781217in}}%
\pgfpathlineto{\pgfqpoint{5.369543in}{1.823107in}}%
\pgfpathlineto{\pgfqpoint{5.424082in}{1.885920in}}%
\pgfpathlineto{\pgfqpoint{5.479807in}{1.922560in}}%
\pgfpathlineto{\pgfqpoint{5.533857in}{1.934528in}}%
\pgfusepath{stroke}%
\end{pgfscope}%
\begin{pgfscope}%
\pgfpathrectangle{\pgfqpoint{0.800000in}{0.528000in}}{\pgfqpoint{4.960000in}{3.696000in}}%
\pgfusepath{clip}%
\pgfsetrectcap%
\pgfsetroundjoin%
\pgfsetlinewidth{1.505625pt}%
\definecolor{currentstroke}{rgb}{0.172549,0.627451,0.172549}%
\pgfsetstrokecolor{currentstroke}%
\pgfsetdash{}{0pt}%
\pgfpathmoveto{\pgfqpoint{1.025455in}{1.997090in}}%
\pgfpathlineto{\pgfqpoint{1.078966in}{3.060405in}}%
\pgfpathlineto{\pgfqpoint{1.132997in}{3.600189in}}%
\pgfpathlineto{\pgfqpoint{1.187625in}{3.886292in}}%
\pgfpathlineto{\pgfqpoint{1.242351in}{3.984464in}}%
\pgfpathlineto{\pgfqpoint{1.296420in}{4.013257in}}%
\pgfpathlineto{\pgfqpoint{1.350736in}{4.023083in}}%
\pgfpathlineto{\pgfqpoint{1.404847in}{3.960841in}}%
\pgfpathlineto{\pgfqpoint{1.459129in}{3.950905in}}%
\pgfpathlineto{\pgfqpoint{1.513355in}{3.651746in}}%
\pgfpathlineto{\pgfqpoint{1.569261in}{3.013325in}}%
\pgfpathlineto{\pgfqpoint{1.622775in}{2.403648in}}%
\pgfpathlineto{\pgfqpoint{1.676148in}{1.903469in}}%
\pgfpathlineto{\pgfqpoint{1.731322in}{1.467934in}}%
\pgfpathlineto{\pgfqpoint{1.784569in}{1.321851in}}%
\pgfpathlineto{\pgfqpoint{1.839432in}{1.254106in}}%
\pgfpathlineto{\pgfqpoint{1.893669in}{1.343432in}}%
\pgfpathlineto{\pgfqpoint{1.948509in}{1.410942in}}%
\pgfpathlineto{\pgfqpoint{2.002491in}{1.623115in}}%
\pgfpathlineto{\pgfqpoint{2.056724in}{1.753524in}}%
\pgfpathlineto{\pgfqpoint{2.111038in}{1.885885in}}%
\pgfpathlineto{\pgfqpoint{2.165495in}{1.979040in}}%
\pgfpathlineto{\pgfqpoint{2.221152in}{2.051753in}}%
\pgfpathlineto{\pgfqpoint{2.274905in}{2.012939in}}%
\pgfpathlineto{\pgfqpoint{2.328211in}{1.937050in}}%
\pgfpathlineto{\pgfqpoint{2.382187in}{1.888157in}}%
\pgfpathlineto{\pgfqpoint{2.436422in}{1.863659in}}%
\pgfpathlineto{\pgfqpoint{2.490788in}{1.799760in}}%
\pgfpathlineto{\pgfqpoint{2.544984in}{1.787814in}}%
\pgfpathlineto{\pgfqpoint{2.599018in}{1.797917in}}%
\pgfpathlineto{\pgfqpoint{2.652870in}{1.808743in}}%
\pgfpathlineto{\pgfqpoint{2.707582in}{1.792637in}}%
\pgfpathlineto{\pgfqpoint{2.761731in}{1.749164in}}%
\pgfpathlineto{\pgfqpoint{2.816106in}{1.725627in}}%
\pgfpathlineto{\pgfqpoint{2.871064in}{1.772560in}}%
\pgfpathlineto{\pgfqpoint{2.924723in}{1.738941in}}%
\pgfpathlineto{\pgfqpoint{2.978554in}{1.809329in}}%
\pgfpathlineto{\pgfqpoint{3.032795in}{1.815641in}}%
\pgfpathlineto{\pgfqpoint{3.087010in}{1.886993in}}%
\pgfpathlineto{\pgfqpoint{3.141259in}{1.874007in}}%
\pgfpathlineto{\pgfqpoint{3.195494in}{1.855687in}}%
\pgfpathlineto{\pgfqpoint{3.250554in}{1.755108in}}%
\pgfpathlineto{\pgfqpoint{3.304679in}{1.721925in}}%
\pgfpathlineto{\pgfqpoint{3.358947in}{1.749266in}}%
\pgfpathlineto{\pgfqpoint{3.413083in}{1.735556in}}%
\pgfpathlineto{\pgfqpoint{3.468973in}{1.768830in}}%
\pgfpathlineto{\pgfqpoint{3.522247in}{1.843077in}}%
\pgfpathlineto{\pgfqpoint{3.576045in}{1.865510in}}%
\pgfpathlineto{\pgfqpoint{3.630242in}{1.922436in}}%
\pgfpathlineto{\pgfqpoint{3.684458in}{1.893043in}}%
\pgfpathlineto{\pgfqpoint{3.738531in}{1.921015in}}%
\pgfpathlineto{\pgfqpoint{3.792593in}{1.843641in}}%
\pgfpathlineto{\pgfqpoint{3.846892in}{1.791812in}}%
\pgfpathlineto{\pgfqpoint{3.900843in}{1.762692in}}%
\pgfpathlineto{\pgfqpoint{3.955224in}{1.754802in}}%
\pgfpathlineto{\pgfqpoint{4.009566in}{1.795628in}}%
\pgfpathlineto{\pgfqpoint{4.065797in}{1.895558in}}%
\pgfpathlineto{\pgfqpoint{4.123343in}{1.875083in}}%
\pgfpathlineto{\pgfqpoint{4.174497in}{1.835177in}}%
\pgfpathlineto{\pgfqpoint{4.228600in}{1.843075in}}%
\pgfpathlineto{\pgfqpoint{4.282532in}{1.839359in}}%
\pgfpathlineto{\pgfqpoint{4.336773in}{1.827567in}}%
\pgfpathlineto{\pgfqpoint{4.390573in}{1.828091in}}%
\pgfpathlineto{\pgfqpoint{4.444326in}{1.830378in}}%
\pgfpathlineto{\pgfqpoint{4.498712in}{1.877266in}}%
\pgfpathlineto{\pgfqpoint{4.553020in}{1.911346in}}%
\pgfpathlineto{\pgfqpoint{4.607096in}{1.913793in}}%
\pgfpathlineto{\pgfqpoint{4.661293in}{1.790875in}}%
\pgfpathlineto{\pgfqpoint{4.715555in}{1.753790in}}%
\pgfpathlineto{\pgfqpoint{4.769887in}{1.777308in}}%
\pgfpathlineto{\pgfqpoint{4.823880in}{1.774891in}}%
\pgfpathlineto{\pgfqpoint{4.877957in}{1.815213in}}%
\pgfpathlineto{\pgfqpoint{4.932251in}{1.904060in}}%
\pgfpathlineto{\pgfqpoint{4.986649in}{1.937354in}}%
\pgfpathlineto{\pgfqpoint{5.040826in}{1.964510in}}%
\pgfpathlineto{\pgfqpoint{5.095066in}{1.930681in}}%
\pgfpathlineto{\pgfqpoint{5.150645in}{1.867401in}}%
\pgfpathlineto{\pgfqpoint{5.204147in}{1.843593in}}%
\pgfpathlineto{\pgfqpoint{5.257682in}{1.826820in}}%
\pgfpathlineto{\pgfqpoint{5.311952in}{1.792701in}}%
\pgfpathlineto{\pgfqpoint{5.366106in}{1.808950in}}%
\pgfpathlineto{\pgfqpoint{5.420363in}{1.786076in}}%
\pgfpathlineto{\pgfqpoint{5.474391in}{1.817642in}}%
\pgfpathlineto{\pgfqpoint{5.529047in}{1.809837in}}%
\pgfusepath{stroke}%
\end{pgfscope}%
\begin{pgfscope}%
\pgfpathrectangle{\pgfqpoint{0.800000in}{0.528000in}}{\pgfqpoint{4.960000in}{3.696000in}}%
\pgfusepath{clip}%
\pgfsetrectcap%
\pgfsetroundjoin%
\pgfsetlinewidth{1.505625pt}%
\definecolor{currentstroke}{rgb}{0.839216,0.152941,0.156863}%
\pgfsetstrokecolor{currentstroke}%
\pgfsetdash{}{0pt}%
\pgfpathmoveto{\pgfqpoint{1.025455in}{2.017657in}}%
\pgfpathlineto{\pgfqpoint{1.079947in}{3.063500in}}%
\pgfpathlineto{\pgfqpoint{1.135043in}{3.639359in}}%
\pgfpathlineto{\pgfqpoint{1.188759in}{3.867442in}}%
\pgfpathlineto{\pgfqpoint{1.243244in}{3.988855in}}%
\pgfpathlineto{\pgfqpoint{1.296941in}{3.998932in}}%
\pgfpathlineto{\pgfqpoint{1.350943in}{4.056000in}}%
\pgfpathlineto{\pgfqpoint{1.405296in}{4.002820in}}%
\pgfpathlineto{\pgfqpoint{1.459410in}{3.955439in}}%
\pgfpathlineto{\pgfqpoint{1.513550in}{3.913063in}}%
\pgfpathlineto{\pgfqpoint{1.567854in}{3.728526in}}%
\pgfpathlineto{\pgfqpoint{1.622094in}{3.221129in}}%
\pgfpathlineto{\pgfqpoint{1.676183in}{2.520515in}}%
\pgfpathlineto{\pgfqpoint{1.730685in}{1.790251in}}%
\pgfpathlineto{\pgfqpoint{1.784453in}{1.289602in}}%
\pgfpathlineto{\pgfqpoint{1.838746in}{1.108741in}}%
\pgfpathlineto{\pgfqpoint{1.892942in}{1.009257in}}%
\pgfpathlineto{\pgfqpoint{1.947080in}{0.949649in}}%
\pgfpathlineto{\pgfqpoint{2.001826in}{1.181605in}}%
\pgfpathlineto{\pgfqpoint{2.055851in}{1.384716in}}%
\pgfpathlineto{\pgfqpoint{2.111534in}{1.506279in}}%
\pgfpathlineto{\pgfqpoint{2.164734in}{1.756388in}}%
\pgfpathlineto{\pgfqpoint{2.218593in}{1.876870in}}%
\pgfpathlineto{\pgfqpoint{2.272588in}{1.938320in}}%
\pgfpathlineto{\pgfqpoint{2.326831in}{1.932105in}}%
\pgfpathlineto{\pgfqpoint{2.381279in}{1.883516in}}%
\pgfpathlineto{\pgfqpoint{2.436572in}{1.811692in}}%
\pgfpathlineto{\pgfqpoint{2.490157in}{1.645411in}}%
\pgfpathlineto{\pgfqpoint{2.543800in}{1.630798in}}%
\pgfpathlineto{\pgfqpoint{2.598059in}{1.586574in}}%
\pgfpathlineto{\pgfqpoint{2.653457in}{1.565901in}}%
\pgfpathlineto{\pgfqpoint{2.709710in}{1.624846in}}%
\pgfpathlineto{\pgfqpoint{2.762669in}{1.651108in}}%
\pgfpathlineto{\pgfqpoint{2.816340in}{1.755988in}}%
\pgfpathlineto{\pgfqpoint{2.870598in}{1.758819in}}%
\pgfpathlineto{\pgfqpoint{2.924476in}{1.829068in}}%
\pgfpathlineto{\pgfqpoint{2.978863in}{1.800653in}}%
\pgfpathlineto{\pgfqpoint{3.032992in}{1.837232in}}%
\pgfpathlineto{\pgfqpoint{3.087277in}{1.762724in}}%
\pgfpathlineto{\pgfqpoint{3.141448in}{1.726810in}}%
\pgfpathlineto{\pgfqpoint{3.195694in}{1.687095in}}%
\pgfpathlineto{\pgfqpoint{3.251038in}{1.705923in}}%
\pgfpathlineto{\pgfqpoint{3.304234in}{1.711484in}}%
\pgfpathlineto{\pgfqpoint{3.360523in}{1.726290in}}%
\pgfpathlineto{\pgfqpoint{3.414760in}{1.751441in}}%
\pgfpathlineto{\pgfqpoint{3.468180in}{1.919307in}}%
\pgfpathlineto{\pgfqpoint{3.522369in}{1.996569in}}%
\pgfpathlineto{\pgfqpoint{3.576428in}{1.988335in}}%
\pgfpathlineto{\pgfqpoint{3.630607in}{1.923073in}}%
\pgfpathlineto{\pgfqpoint{3.685114in}{2.002802in}}%
\pgfpathlineto{\pgfqpoint{3.739314in}{1.930659in}}%
\pgfpathlineto{\pgfqpoint{3.793434in}{1.981265in}}%
\pgfpathlineto{\pgfqpoint{3.848197in}{2.078215in}}%
\pgfpathlineto{\pgfqpoint{3.902608in}{1.963936in}}%
\pgfpathlineto{\pgfqpoint{3.956727in}{1.901736in}}%
\pgfpathlineto{\pgfqpoint{4.012974in}{1.895791in}}%
\pgfpathlineto{\pgfqpoint{4.066523in}{1.884905in}}%
\pgfpathlineto{\pgfqpoint{4.120231in}{1.877829in}}%
\pgfpathlineto{\pgfqpoint{4.174558in}{1.903763in}}%
\pgfpathlineto{\pgfqpoint{4.228531in}{1.892308in}}%
\pgfpathlineto{\pgfqpoint{4.282541in}{1.879611in}}%
\pgfpathlineto{\pgfqpoint{4.337498in}{1.905878in}}%
\pgfpathlineto{\pgfqpoint{4.392156in}{1.891345in}}%
\pgfpathlineto{\pgfqpoint{4.445690in}{1.879077in}}%
\pgfpathlineto{\pgfqpoint{4.500469in}{1.840360in}}%
\pgfpathlineto{\pgfqpoint{4.554604in}{1.808654in}}%
\pgfpathlineto{\pgfqpoint{4.608823in}{1.804584in}}%
\pgfpathlineto{\pgfqpoint{4.664880in}{1.750465in}}%
\pgfpathlineto{\pgfqpoint{4.718697in}{1.809782in}}%
\pgfpathlineto{\pgfqpoint{4.772489in}{1.803246in}}%
\pgfpathlineto{\pgfqpoint{4.826662in}{1.813982in}}%
\pgfpathlineto{\pgfqpoint{4.880693in}{1.843970in}}%
\pgfpathlineto{\pgfqpoint{4.935212in}{1.823277in}}%
\pgfpathlineto{\pgfqpoint{4.989215in}{1.836505in}}%
\pgfpathlineto{\pgfqpoint{5.044150in}{1.817054in}}%
\pgfpathlineto{\pgfqpoint{5.098445in}{1.876053in}}%
\pgfpathlineto{\pgfqpoint{5.152556in}{1.863467in}}%
\pgfpathlineto{\pgfqpoint{5.207028in}{1.918989in}}%
\pgfpathlineto{\pgfqpoint{5.261721in}{1.891119in}}%
\pgfpathlineto{\pgfqpoint{5.316835in}{1.906399in}}%
\pgfpathlineto{\pgfqpoint{5.370415in}{1.905027in}}%
\pgfpathlineto{\pgfqpoint{5.424401in}{1.916947in}}%
\pgfpathlineto{\pgfqpoint{5.478377in}{1.908669in}}%
\pgfpathlineto{\pgfqpoint{5.532589in}{1.891863in}}%
\pgfusepath{stroke}%
\end{pgfscope}%
\begin{pgfscope}%
\pgfpathrectangle{\pgfqpoint{0.800000in}{0.528000in}}{\pgfqpoint{4.960000in}{3.696000in}}%
\pgfusepath{clip}%
\pgfsetrectcap%
\pgfsetroundjoin%
\pgfsetlinewidth{1.505625pt}%
\definecolor{currentstroke}{rgb}{0.580392,0.403922,0.741176}%
\pgfsetstrokecolor{currentstroke}%
\pgfsetdash{}{0pt}%
\pgfpathmoveto{\pgfqpoint{1.025455in}{1.935083in}}%
\pgfpathlineto{\pgfqpoint{1.079690in}{3.008134in}}%
\pgfpathlineto{\pgfqpoint{1.134013in}{3.606475in}}%
\pgfpathlineto{\pgfqpoint{1.188348in}{3.923284in}}%
\pgfpathlineto{\pgfqpoint{1.242604in}{3.998538in}}%
\pgfpathlineto{\pgfqpoint{1.298592in}{4.035653in}}%
\pgfpathlineto{\pgfqpoint{1.352103in}{4.037946in}}%
\pgfpathlineto{\pgfqpoint{1.405742in}{3.990097in}}%
\pgfpathlineto{\pgfqpoint{1.461786in}{4.001823in}}%
\pgfpathlineto{\pgfqpoint{1.514310in}{3.908526in}}%
\pgfpathlineto{\pgfqpoint{1.568579in}{3.704083in}}%
\pgfpathlineto{\pgfqpoint{1.622582in}{2.854883in}}%
\pgfpathlineto{\pgfqpoint{1.677527in}{2.372332in}}%
\pgfpathlineto{\pgfqpoint{1.731265in}{1.780748in}}%
\pgfpathlineto{\pgfqpoint{1.785722in}{1.436557in}}%
\pgfpathlineto{\pgfqpoint{1.840231in}{1.191453in}}%
\pgfpathlineto{\pgfqpoint{1.894008in}{1.020239in}}%
\pgfpathlineto{\pgfqpoint{1.950051in}{1.084361in}}%
\pgfpathlineto{\pgfqpoint{2.003275in}{1.165992in}}%
\pgfpathlineto{\pgfqpoint{2.056911in}{1.250875in}}%
\pgfpathlineto{\pgfqpoint{2.110893in}{1.376090in}}%
\pgfpathlineto{\pgfqpoint{2.165111in}{1.577621in}}%
\pgfpathlineto{\pgfqpoint{2.219222in}{1.839444in}}%
\pgfpathlineto{\pgfqpoint{2.273343in}{2.070861in}}%
\pgfpathlineto{\pgfqpoint{2.327661in}{2.070371in}}%
\pgfpathlineto{\pgfqpoint{2.381805in}{1.963344in}}%
\pgfpathlineto{\pgfqpoint{2.436143in}{1.796266in}}%
\pgfpathlineto{\pgfqpoint{2.490891in}{1.695361in}}%
\pgfpathlineto{\pgfqpoint{2.545061in}{1.579260in}}%
\pgfpathlineto{\pgfqpoint{2.600632in}{1.540325in}}%
\pgfpathlineto{\pgfqpoint{2.654009in}{1.532257in}}%
\pgfpathlineto{\pgfqpoint{2.708578in}{1.693458in}}%
\pgfpathlineto{\pgfqpoint{2.762920in}{1.754731in}}%
\pgfpathlineto{\pgfqpoint{2.817057in}{1.792668in}}%
\pgfpathlineto{\pgfqpoint{2.871017in}{1.815388in}}%
\pgfpathlineto{\pgfqpoint{2.925244in}{1.859861in}}%
\pgfpathlineto{\pgfqpoint{2.979870in}{1.825416in}}%
\pgfpathlineto{\pgfqpoint{3.033799in}{1.806024in}}%
\pgfpathlineto{\pgfqpoint{3.088444in}{1.817980in}}%
\pgfpathlineto{\pgfqpoint{3.143332in}{1.807805in}}%
\pgfpathlineto{\pgfqpoint{3.197372in}{1.766557in}}%
\pgfpathlineto{\pgfqpoint{3.253701in}{1.743566in}}%
\pgfpathlineto{\pgfqpoint{3.307076in}{1.797697in}}%
\pgfpathlineto{\pgfqpoint{3.361382in}{1.836354in}}%
\pgfpathlineto{\pgfqpoint{3.415547in}{1.897536in}}%
\pgfpathlineto{\pgfqpoint{3.469859in}{1.896644in}}%
\pgfpathlineto{\pgfqpoint{3.524049in}{1.710499in}}%
\pgfpathlineto{\pgfqpoint{3.578230in}{1.678541in}}%
\pgfpathlineto{\pgfqpoint{3.632345in}{1.665652in}}%
\pgfpathlineto{\pgfqpoint{3.686652in}{1.812203in}}%
\pgfpathlineto{\pgfqpoint{3.742434in}{1.871621in}}%
\pgfpathlineto{\pgfqpoint{3.796444in}{1.927737in}}%
\pgfpathlineto{\pgfqpoint{3.852666in}{1.862244in}}%
\pgfpathlineto{\pgfqpoint{3.906402in}{1.849859in}}%
\pgfpathlineto{\pgfqpoint{3.959759in}{1.868697in}}%
\pgfpathlineto{\pgfqpoint{4.014981in}{1.813955in}}%
\pgfpathlineto{\pgfqpoint{4.070187in}{1.801015in}}%
\pgfpathlineto{\pgfqpoint{4.123039in}{1.779709in}}%
\pgfpathlineto{\pgfqpoint{4.176337in}{1.925449in}}%
\pgfpathlineto{\pgfqpoint{4.230234in}{1.960158in}}%
\pgfpathlineto{\pgfqpoint{4.285211in}{1.905856in}}%
\pgfpathlineto{\pgfqpoint{4.338944in}{1.931040in}}%
\pgfpathlineto{\pgfqpoint{4.393095in}{1.838789in}}%
\pgfpathlineto{\pgfqpoint{4.447850in}{1.881087in}}%
\pgfpathlineto{\pgfqpoint{4.501945in}{1.821864in}}%
\pgfpathlineto{\pgfqpoint{4.556050in}{1.821761in}}%
\pgfpathlineto{\pgfqpoint{4.609931in}{1.733605in}}%
\pgfpathlineto{\pgfqpoint{4.665426in}{1.811511in}}%
\pgfpathlineto{\pgfqpoint{4.719055in}{1.799556in}}%
\pgfpathlineto{\pgfqpoint{4.773045in}{1.765805in}}%
\pgfpathlineto{\pgfqpoint{4.827102in}{1.851397in}}%
\pgfpathlineto{\pgfqpoint{4.881257in}{1.978619in}}%
\pgfpathlineto{\pgfqpoint{4.935654in}{2.037459in}}%
\pgfpathlineto{\pgfqpoint{4.989556in}{2.091450in}}%
\pgfpathlineto{\pgfqpoint{5.044014in}{2.114806in}}%
\pgfpathlineto{\pgfqpoint{5.098561in}{2.099131in}}%
\pgfpathlineto{\pgfqpoint{5.153143in}{2.038788in}}%
\pgfpathlineto{\pgfqpoint{5.207414in}{1.882576in}}%
\pgfpathlineto{\pgfqpoint{5.262474in}{1.754273in}}%
\pgfpathlineto{\pgfqpoint{5.316755in}{1.759560in}}%
\pgfpathlineto{\pgfqpoint{5.370890in}{1.719454in}}%
\pgfpathlineto{\pgfqpoint{5.424704in}{1.685432in}}%
\pgfpathlineto{\pgfqpoint{5.478925in}{1.643685in}}%
\pgfpathlineto{\pgfqpoint{5.533836in}{1.663948in}}%
\pgfusepath{stroke}%
\end{pgfscope}%
\begin{pgfscope}%
\pgfpathrectangle{\pgfqpoint{0.800000in}{0.528000in}}{\pgfqpoint{4.960000in}{3.696000in}}%
\pgfusepath{clip}%
\pgfsetrectcap%
\pgfsetroundjoin%
\pgfsetlinewidth{1.505625pt}%
\definecolor{currentstroke}{rgb}{0.549020,0.337255,0.294118}%
\pgfsetstrokecolor{currentstroke}%
\pgfsetdash{}{0pt}%
\pgfpathmoveto{\pgfqpoint{1.025455in}{2.006145in}}%
\pgfpathlineto{\pgfqpoint{1.080023in}{3.065312in}}%
\pgfpathlineto{\pgfqpoint{1.133950in}{3.622063in}}%
\pgfpathlineto{\pgfqpoint{1.188858in}{3.895572in}}%
\pgfpathlineto{\pgfqpoint{1.242748in}{3.978680in}}%
\pgfpathlineto{\pgfqpoint{1.296905in}{4.039696in}}%
\pgfpathlineto{\pgfqpoint{1.351201in}{4.004138in}}%
\pgfpathlineto{\pgfqpoint{1.405479in}{4.009697in}}%
\pgfpathlineto{\pgfqpoint{1.461132in}{4.001185in}}%
\pgfpathlineto{\pgfqpoint{1.514846in}{3.886749in}}%
\pgfpathlineto{\pgfqpoint{1.568345in}{3.499488in}}%
\pgfpathlineto{\pgfqpoint{1.622776in}{2.747912in}}%
\pgfpathlineto{\pgfqpoint{1.677160in}{2.088324in}}%
\pgfpathlineto{\pgfqpoint{1.731329in}{1.477158in}}%
\pgfpathlineto{\pgfqpoint{1.785487in}{0.891981in}}%
\pgfpathlineto{\pgfqpoint{1.839854in}{0.696000in}}%
\pgfpathlineto{\pgfqpoint{1.893803in}{0.871762in}}%
\pgfpathlineto{\pgfqpoint{1.948228in}{1.026074in}}%
\pgfpathlineto{\pgfqpoint{2.002342in}{1.116145in}}%
\pgfpathlineto{\pgfqpoint{2.056665in}{1.510729in}}%
\pgfpathlineto{\pgfqpoint{2.112475in}{1.834325in}}%
\pgfpathlineto{\pgfqpoint{2.166361in}{2.041289in}}%
\pgfpathlineto{\pgfqpoint{2.220153in}{2.210908in}}%
\pgfpathlineto{\pgfqpoint{2.274745in}{2.196838in}}%
\pgfpathlineto{\pgfqpoint{2.328733in}{2.066839in}}%
\pgfpathlineto{\pgfqpoint{2.384336in}{1.894702in}}%
\pgfpathlineto{\pgfqpoint{2.436870in}{1.854071in}}%
\pgfpathlineto{\pgfqpoint{2.491682in}{1.801165in}}%
\pgfpathlineto{\pgfqpoint{2.546362in}{1.876781in}}%
\pgfpathlineto{\pgfqpoint{2.600459in}{1.845371in}}%
\pgfpathlineto{\pgfqpoint{2.654776in}{1.697975in}}%
\pgfpathlineto{\pgfqpoint{2.710928in}{1.695475in}}%
\pgfpathlineto{\pgfqpoint{2.764404in}{1.642021in}}%
\pgfpathlineto{\pgfqpoint{2.818768in}{1.804985in}}%
\pgfpathlineto{\pgfqpoint{2.872296in}{1.783194in}}%
\pgfpathlineto{\pgfqpoint{2.926411in}{1.794562in}}%
\pgfpathlineto{\pgfqpoint{2.980787in}{1.937852in}}%
\pgfpathlineto{\pgfqpoint{3.035124in}{2.072022in}}%
\pgfpathlineto{\pgfqpoint{3.089025in}{2.096436in}}%
\pgfpathlineto{\pgfqpoint{3.143523in}{2.164521in}}%
\pgfpathlineto{\pgfqpoint{3.197738in}{2.187559in}}%
\pgfpathlineto{\pgfqpoint{3.251828in}{2.205798in}}%
\pgfpathlineto{\pgfqpoint{3.307984in}{2.104113in}}%
\pgfpathlineto{\pgfqpoint{3.361490in}{2.062884in}}%
\pgfpathlineto{\pgfqpoint{3.414924in}{1.888860in}}%
\pgfpathlineto{\pgfqpoint{3.469007in}{1.761940in}}%
\pgfpathlineto{\pgfqpoint{3.524324in}{1.711128in}}%
\pgfpathlineto{\pgfqpoint{3.578963in}{1.696444in}}%
\pgfpathlineto{\pgfqpoint{3.632371in}{1.446714in}}%
\pgfpathlineto{\pgfqpoint{3.686555in}{1.402556in}}%
\pgfpathlineto{\pgfqpoint{3.740517in}{1.497199in}}%
\pgfpathlineto{\pgfqpoint{3.795411in}{1.749161in}}%
\pgfpathlineto{\pgfqpoint{3.849298in}{2.033448in}}%
\pgfpathlineto{\pgfqpoint{3.903586in}{2.128743in}}%
\pgfpathlineto{\pgfqpoint{3.959610in}{2.206403in}}%
\pgfpathlineto{\pgfqpoint{4.013334in}{2.148873in}}%
\pgfpathlineto{\pgfqpoint{4.066697in}{2.204559in}}%
\pgfpathlineto{\pgfqpoint{4.120732in}{2.208451in}}%
\pgfpathlineto{\pgfqpoint{4.174850in}{2.060963in}}%
\pgfpathlineto{\pgfqpoint{4.228816in}{1.573000in}}%
\pgfpathlineto{\pgfqpoint{4.283332in}{1.408043in}}%
\pgfpathlineto{\pgfqpoint{4.339525in}{1.345903in}}%
\pgfpathlineto{\pgfqpoint{4.391112in}{1.419879in}}%
\pgfpathlineto{\pgfqpoint{4.445759in}{1.613178in}}%
\pgfpathlineto{\pgfqpoint{4.500206in}{1.751091in}}%
\pgfpathlineto{\pgfqpoint{4.554681in}{1.906627in}}%
\pgfpathlineto{\pgfqpoint{4.610590in}{1.918107in}}%
\pgfpathlineto{\pgfqpoint{4.664701in}{1.973321in}}%
\pgfpathlineto{\pgfqpoint{4.718290in}{1.813501in}}%
\pgfpathlineto{\pgfqpoint{4.772829in}{1.802744in}}%
\pgfpathlineto{\pgfqpoint{4.827177in}{1.834902in}}%
\pgfpathlineto{\pgfqpoint{4.881591in}{1.912656in}}%
\pgfpathlineto{\pgfqpoint{4.935559in}{2.057335in}}%
\pgfpathlineto{\pgfqpoint{4.989568in}{2.112212in}}%
\pgfpathlineto{\pgfqpoint{5.046442in}{2.082817in}}%
\pgfpathlineto{\pgfqpoint{5.099910in}{1.936386in}}%
\pgfpathlineto{\pgfqpoint{5.153948in}{1.894574in}}%
\pgfpathlineto{\pgfqpoint{5.209490in}{1.793401in}}%
\pgfpathlineto{\pgfqpoint{5.264559in}{1.613848in}}%
\pgfpathlineto{\pgfqpoint{5.320562in}{1.443016in}}%
\pgfpathlineto{\pgfqpoint{5.371856in}{1.446672in}}%
\pgfpathlineto{\pgfqpoint{5.426096in}{1.531297in}}%
\pgfpathlineto{\pgfqpoint{5.480045in}{1.710832in}}%
\pgfpathlineto{\pgfqpoint{5.534493in}{1.807200in}}%
\pgfusepath{stroke}%
\end{pgfscope}%
\begin{pgfscope}%
\pgfsetrectcap%
\pgfsetmiterjoin%
\pgfsetlinewidth{0.803000pt}%
\definecolor{currentstroke}{rgb}{0.000000,0.000000,0.000000}%
\pgfsetstrokecolor{currentstroke}%
\pgfsetdash{}{0pt}%
\pgfpathmoveto{\pgfqpoint{0.800000in}{0.528000in}}%
\pgfpathlineto{\pgfqpoint{0.800000in}{4.224000in}}%
\pgfusepath{stroke}%
\end{pgfscope}%
\begin{pgfscope}%
\pgfsetrectcap%
\pgfsetmiterjoin%
\pgfsetlinewidth{0.803000pt}%
\definecolor{currentstroke}{rgb}{0.000000,0.000000,0.000000}%
\pgfsetstrokecolor{currentstroke}%
\pgfsetdash{}{0pt}%
\pgfpathmoveto{\pgfqpoint{5.760000in}{0.528000in}}%
\pgfpathlineto{\pgfqpoint{5.760000in}{4.224000in}}%
\pgfusepath{stroke}%
\end{pgfscope}%
\begin{pgfscope}%
\pgfsetrectcap%
\pgfsetmiterjoin%
\pgfsetlinewidth{0.803000pt}%
\definecolor{currentstroke}{rgb}{0.000000,0.000000,0.000000}%
\pgfsetstrokecolor{currentstroke}%
\pgfsetdash{}{0pt}%
\pgfpathmoveto{\pgfqpoint{0.800000in}{0.528000in}}%
\pgfpathlineto{\pgfqpoint{5.760000in}{0.528000in}}%
\pgfusepath{stroke}%
\end{pgfscope}%
\begin{pgfscope}%
\pgfsetrectcap%
\pgfsetmiterjoin%
\pgfsetlinewidth{0.803000pt}%
\definecolor{currentstroke}{rgb}{0.000000,0.000000,0.000000}%
\pgfsetstrokecolor{currentstroke}%
\pgfsetdash{}{0pt}%
\pgfpathmoveto{\pgfqpoint{0.800000in}{4.224000in}}%
\pgfpathlineto{\pgfqpoint{5.760000in}{4.224000in}}%
\pgfusepath{stroke}%
\end{pgfscope}%
\begin{pgfscope}%
\definecolor{textcolor}{rgb}{0.000000,0.000000,0.000000}%
\pgfsetstrokecolor{textcolor}%
\pgfsetfillcolor{textcolor}%
\pgftext[x=3.280000in,y=4.307333in,,base]{\color{textcolor}\sffamily\fontsize{12.000000}{14.400000}\selectfont Measured yaw speed}%
\end{pgfscope}%
\begin{pgfscope}%
\pgfsetbuttcap%
\pgfsetmiterjoin%
\definecolor{currentfill}{rgb}{1.000000,1.000000,1.000000}%
\pgfsetfillcolor{currentfill}%
\pgfsetfillopacity{0.800000}%
\pgfsetlinewidth{1.003750pt}%
\definecolor{currentstroke}{rgb}{0.800000,0.800000,0.800000}%
\pgfsetstrokecolor{currentstroke}%
\pgfsetstrokeopacity{0.800000}%
\pgfsetdash{}{0pt}%
\pgfpathmoveto{\pgfqpoint{5.041603in}{2.889746in}}%
\pgfpathlineto{\pgfqpoint{5.662778in}{2.889746in}}%
\pgfpathquadraticcurveto{\pgfqpoint{5.690556in}{2.889746in}}{\pgfqpoint{5.690556in}{2.917523in}}%
\pgfpathlineto{\pgfqpoint{5.690556in}{4.126778in}}%
\pgfpathquadraticcurveto{\pgfqpoint{5.690556in}{4.154556in}}{\pgfqpoint{5.662778in}{4.154556in}}%
\pgfpathlineto{\pgfqpoint{5.041603in}{4.154556in}}%
\pgfpathquadraticcurveto{\pgfqpoint{5.013825in}{4.154556in}}{\pgfqpoint{5.013825in}{4.126778in}}%
\pgfpathlineto{\pgfqpoint{5.013825in}{2.917523in}}%
\pgfpathquadraticcurveto{\pgfqpoint{5.013825in}{2.889746in}}{\pgfqpoint{5.041603in}{2.889746in}}%
\pgfpathlineto{\pgfqpoint{5.041603in}{2.889746in}}%
\pgfpathclose%
\pgfusepath{stroke,fill}%
\end{pgfscope}%
\begin{pgfscope}%
\pgfsetrectcap%
\pgfsetroundjoin%
\pgfsetlinewidth{1.505625pt}%
\definecolor{currentstroke}{rgb}{0.121569,0.466667,0.705882}%
\pgfsetstrokecolor{currentstroke}%
\pgfsetdash{}{0pt}%
\pgfpathmoveto{\pgfqpoint{5.069380in}{4.042088in}}%
\pgfpathlineto{\pgfqpoint{5.208269in}{4.042088in}}%
\pgfpathlineto{\pgfqpoint{5.347158in}{4.042088in}}%
\pgfusepath{stroke}%
\end{pgfscope}%
\begin{pgfscope}%
\definecolor{textcolor}{rgb}{0.000000,0.000000,0.000000}%
\pgfsetstrokecolor{textcolor}%
\pgfsetfillcolor{textcolor}%
\pgftext[x=5.458269in,y=3.993477in,left,base]{\color{textcolor}\sffamily\fontsize{10.000000}{12.000000}\selectfont 0}%
\end{pgfscope}%
\begin{pgfscope}%
\pgfsetrectcap%
\pgfsetroundjoin%
\pgfsetlinewidth{1.505625pt}%
\definecolor{currentstroke}{rgb}{1.000000,0.498039,0.054902}%
\pgfsetstrokecolor{currentstroke}%
\pgfsetdash{}{0pt}%
\pgfpathmoveto{\pgfqpoint{5.069380in}{3.838231in}}%
\pgfpathlineto{\pgfqpoint{5.208269in}{3.838231in}}%
\pgfpathlineto{\pgfqpoint{5.347158in}{3.838231in}}%
\pgfusepath{stroke}%
\end{pgfscope}%
\begin{pgfscope}%
\definecolor{textcolor}{rgb}{0.000000,0.000000,0.000000}%
\pgfsetstrokecolor{textcolor}%
\pgfsetfillcolor{textcolor}%
\pgftext[x=5.458269in,y=3.789620in,left,base]{\color{textcolor}\sffamily\fontsize{10.000000}{12.000000}\selectfont 5}%
\end{pgfscope}%
\begin{pgfscope}%
\pgfsetrectcap%
\pgfsetroundjoin%
\pgfsetlinewidth{1.505625pt}%
\definecolor{currentstroke}{rgb}{0.172549,0.627451,0.172549}%
\pgfsetstrokecolor{currentstroke}%
\pgfsetdash{}{0pt}%
\pgfpathmoveto{\pgfqpoint{5.069380in}{3.634374in}}%
\pgfpathlineto{\pgfqpoint{5.208269in}{3.634374in}}%
\pgfpathlineto{\pgfqpoint{5.347158in}{3.634374in}}%
\pgfusepath{stroke}%
\end{pgfscope}%
\begin{pgfscope}%
\definecolor{textcolor}{rgb}{0.000000,0.000000,0.000000}%
\pgfsetstrokecolor{textcolor}%
\pgfsetfillcolor{textcolor}%
\pgftext[x=5.458269in,y=3.585762in,left,base]{\color{textcolor}\sffamily\fontsize{10.000000}{12.000000}\selectfont 10}%
\end{pgfscope}%
\begin{pgfscope}%
\pgfsetrectcap%
\pgfsetroundjoin%
\pgfsetlinewidth{1.505625pt}%
\definecolor{currentstroke}{rgb}{0.839216,0.152941,0.156863}%
\pgfsetstrokecolor{currentstroke}%
\pgfsetdash{}{0pt}%
\pgfpathmoveto{\pgfqpoint{5.069380in}{3.430516in}}%
\pgfpathlineto{\pgfqpoint{5.208269in}{3.430516in}}%
\pgfpathlineto{\pgfqpoint{5.347158in}{3.430516in}}%
\pgfusepath{stroke}%
\end{pgfscope}%
\begin{pgfscope}%
\definecolor{textcolor}{rgb}{0.000000,0.000000,0.000000}%
\pgfsetstrokecolor{textcolor}%
\pgfsetfillcolor{textcolor}%
\pgftext[x=5.458269in,y=3.381905in,left,base]{\color{textcolor}\sffamily\fontsize{10.000000}{12.000000}\selectfont 20}%
\end{pgfscope}%
\begin{pgfscope}%
\pgfsetrectcap%
\pgfsetroundjoin%
\pgfsetlinewidth{1.505625pt}%
\definecolor{currentstroke}{rgb}{0.580392,0.403922,0.741176}%
\pgfsetstrokecolor{currentstroke}%
\pgfsetdash{}{0pt}%
\pgfpathmoveto{\pgfqpoint{5.069380in}{3.226659in}}%
\pgfpathlineto{\pgfqpoint{5.208269in}{3.226659in}}%
\pgfpathlineto{\pgfqpoint{5.347158in}{3.226659in}}%
\pgfusepath{stroke}%
\end{pgfscope}%
\begin{pgfscope}%
\definecolor{textcolor}{rgb}{0.000000,0.000000,0.000000}%
\pgfsetstrokecolor{textcolor}%
\pgfsetfillcolor{textcolor}%
\pgftext[x=5.458269in,y=3.178048in,left,base]{\color{textcolor}\sffamily\fontsize{10.000000}{12.000000}\selectfont 40}%
\end{pgfscope}%
\begin{pgfscope}%
\pgfsetrectcap%
\pgfsetroundjoin%
\pgfsetlinewidth{1.505625pt}%
\definecolor{currentstroke}{rgb}{0.549020,0.337255,0.294118}%
\pgfsetstrokecolor{currentstroke}%
\pgfsetdash{}{0pt}%
\pgfpathmoveto{\pgfqpoint{5.069380in}{3.022802in}}%
\pgfpathlineto{\pgfqpoint{5.208269in}{3.022802in}}%
\pgfpathlineto{\pgfqpoint{5.347158in}{3.022802in}}%
\pgfusepath{stroke}%
\end{pgfscope}%
\begin{pgfscope}%
\definecolor{textcolor}{rgb}{0.000000,0.000000,0.000000}%
\pgfsetstrokecolor{textcolor}%
\pgfsetfillcolor{textcolor}%
\pgftext[x=5.458269in,y=2.974191in,left,base]{\color{textcolor}\sffamily\fontsize{10.000000}{12.000000}\selectfont 80}%
\end{pgfscope}%
\end{pgfpicture}%
\makeatother%
\endgroup%
}
    \end{minipage}
    \caption{Variation of (a) measured yaw heading and (b) measured yaw velocity for different values of $K_{D}$ and $K_P=100$, $K_I=30$ while the yaw controller is engaged.}
    \label{fig:tune-yaw-der-measures}
\end{figure}

\section{Forward controller}
\label{app:fwd-pid-results}

\begin{figure}[H]
    \begin{minipage}[t]{0.5\linewidth}
        \centering
        \scalebox{0.55}{%% Creator: Matplotlib, PGF backend
%%
%% To include the figure in your LaTeX document, write
%%   \input{<filename>.pgf}
%%
%% Make sure the required packages are loaded in your preamble
%%   \usepackage{pgf}
%%
%% Also ensure that all the required font packages are loaded; for instance,
%% the lmodern package is sometimes necessary when using math font.
%%   \usepackage{lmodern}
%%
%% Figures using additional raster images can only be included by \input if
%% they are in the same directory as the main LaTeX file. For loading figures
%% from other directories you can use the `import` package
%%   \usepackage{import}
%%
%% and then include the figures with
%%   \import{<path to file>}{<filename>.pgf}
%%
%% Matplotlib used the following preamble
%%   \usepackage{fontspec}
%%   \setmainfont{DejaVuSerif.ttf}[Path=\detokenize{/home/lgonz/tfg-aero/tfg-giaa-dronecontrol/venv/lib/python3.8/site-packages/matplotlib/mpl-data/fonts/ttf/}]
%%   \setsansfont{DejaVuSans.ttf}[Path=\detokenize{/home/lgonz/tfg-aero/tfg-giaa-dronecontrol/venv/lib/python3.8/site-packages/matplotlib/mpl-data/fonts/ttf/}]
%%   \setmonofont{DejaVuSansMono.ttf}[Path=\detokenize{/home/lgonz/tfg-aero/tfg-giaa-dronecontrol/venv/lib/python3.8/site-packages/matplotlib/mpl-data/fonts/ttf/}]
%%
\begingroup%
\makeatletter%
\begin{pgfpicture}%
\pgfpathrectangle{\pgfpointorigin}{\pgfqpoint{6.400000in}{4.800000in}}%
\pgfusepath{use as bounding box, clip}%
\begin{pgfscope}%
\pgfsetbuttcap%
\pgfsetmiterjoin%
\definecolor{currentfill}{rgb}{1.000000,1.000000,1.000000}%
\pgfsetfillcolor{currentfill}%
\pgfsetlinewidth{0.000000pt}%
\definecolor{currentstroke}{rgb}{1.000000,1.000000,1.000000}%
\pgfsetstrokecolor{currentstroke}%
\pgfsetdash{}{0pt}%
\pgfpathmoveto{\pgfqpoint{0.000000in}{0.000000in}}%
\pgfpathlineto{\pgfqpoint{6.400000in}{0.000000in}}%
\pgfpathlineto{\pgfqpoint{6.400000in}{4.800000in}}%
\pgfpathlineto{\pgfqpoint{0.000000in}{4.800000in}}%
\pgfpathlineto{\pgfqpoint{0.000000in}{0.000000in}}%
\pgfpathclose%
\pgfusepath{fill}%
\end{pgfscope}%
\begin{pgfscope}%
\pgfsetbuttcap%
\pgfsetmiterjoin%
\definecolor{currentfill}{rgb}{1.000000,1.000000,1.000000}%
\pgfsetfillcolor{currentfill}%
\pgfsetlinewidth{0.000000pt}%
\definecolor{currentstroke}{rgb}{0.000000,0.000000,0.000000}%
\pgfsetstrokecolor{currentstroke}%
\pgfsetstrokeopacity{0.000000}%
\pgfsetdash{}{0pt}%
\pgfpathmoveto{\pgfqpoint{0.800000in}{0.528000in}}%
\pgfpathlineto{\pgfqpoint{5.760000in}{0.528000in}}%
\pgfpathlineto{\pgfqpoint{5.760000in}{4.224000in}}%
\pgfpathlineto{\pgfqpoint{0.800000in}{4.224000in}}%
\pgfpathlineto{\pgfqpoint{0.800000in}{0.528000in}}%
\pgfpathclose%
\pgfusepath{fill}%
\end{pgfscope}%
\begin{pgfscope}%
\pgfpathrectangle{\pgfqpoint{0.800000in}{0.528000in}}{\pgfqpoint{4.960000in}{3.696000in}}%
\pgfusepath{clip}%
\pgfsetrectcap%
\pgfsetroundjoin%
\pgfsetlinewidth{0.803000pt}%
\definecolor{currentstroke}{rgb}{0.690196,0.690196,0.690196}%
\pgfsetstrokecolor{currentstroke}%
\pgfsetdash{}{0pt}%
\pgfpathmoveto{\pgfqpoint{1.025455in}{0.528000in}}%
\pgfpathlineto{\pgfqpoint{1.025455in}{4.224000in}}%
\pgfusepath{stroke}%
\end{pgfscope}%
\begin{pgfscope}%
\pgfsetbuttcap%
\pgfsetroundjoin%
\definecolor{currentfill}{rgb}{0.000000,0.000000,0.000000}%
\pgfsetfillcolor{currentfill}%
\pgfsetlinewidth{0.803000pt}%
\definecolor{currentstroke}{rgb}{0.000000,0.000000,0.000000}%
\pgfsetstrokecolor{currentstroke}%
\pgfsetdash{}{0pt}%
\pgfsys@defobject{currentmarker}{\pgfqpoint{0.000000in}{-0.048611in}}{\pgfqpoint{0.000000in}{0.000000in}}{%
\pgfpathmoveto{\pgfqpoint{0.000000in}{0.000000in}}%
\pgfpathlineto{\pgfqpoint{0.000000in}{-0.048611in}}%
\pgfusepath{stroke,fill}%
}%
\begin{pgfscope}%
\pgfsys@transformshift{1.025455in}{0.528000in}%
\pgfsys@useobject{currentmarker}{}%
\end{pgfscope}%
\end{pgfscope}%
\begin{pgfscope}%
\definecolor{textcolor}{rgb}{0.000000,0.000000,0.000000}%
\pgfsetstrokecolor{textcolor}%
\pgfsetfillcolor{textcolor}%
\pgftext[x=1.025455in,y=0.430778in,,top]{\color{textcolor}\sffamily\fontsize{10.000000}{12.000000}\selectfont 0}%
\end{pgfscope}%
\begin{pgfscope}%
\pgfpathrectangle{\pgfqpoint{0.800000in}{0.528000in}}{\pgfqpoint{4.960000in}{3.696000in}}%
\pgfusepath{clip}%
\pgfsetrectcap%
\pgfsetroundjoin%
\pgfsetlinewidth{0.803000pt}%
\definecolor{currentstroke}{rgb}{0.690196,0.690196,0.690196}%
\pgfsetstrokecolor{currentstroke}%
\pgfsetdash{}{0pt}%
\pgfpathmoveto{\pgfqpoint{1.775829in}{0.528000in}}%
\pgfpathlineto{\pgfqpoint{1.775829in}{4.224000in}}%
\pgfusepath{stroke}%
\end{pgfscope}%
\begin{pgfscope}%
\pgfsetbuttcap%
\pgfsetroundjoin%
\definecolor{currentfill}{rgb}{0.000000,0.000000,0.000000}%
\pgfsetfillcolor{currentfill}%
\pgfsetlinewidth{0.803000pt}%
\definecolor{currentstroke}{rgb}{0.000000,0.000000,0.000000}%
\pgfsetstrokecolor{currentstroke}%
\pgfsetdash{}{0pt}%
\pgfsys@defobject{currentmarker}{\pgfqpoint{0.000000in}{-0.048611in}}{\pgfqpoint{0.000000in}{0.000000in}}{%
\pgfpathmoveto{\pgfqpoint{0.000000in}{0.000000in}}%
\pgfpathlineto{\pgfqpoint{0.000000in}{-0.048611in}}%
\pgfusepath{stroke,fill}%
}%
\begin{pgfscope}%
\pgfsys@transformshift{1.775829in}{0.528000in}%
\pgfsys@useobject{currentmarker}{}%
\end{pgfscope}%
\end{pgfscope}%
\begin{pgfscope}%
\definecolor{textcolor}{rgb}{0.000000,0.000000,0.000000}%
\pgfsetstrokecolor{textcolor}%
\pgfsetfillcolor{textcolor}%
\pgftext[x=1.775829in,y=0.430778in,,top]{\color{textcolor}\sffamily\fontsize{10.000000}{12.000000}\selectfont 5}%
\end{pgfscope}%
\begin{pgfscope}%
\pgfpathrectangle{\pgfqpoint{0.800000in}{0.528000in}}{\pgfqpoint{4.960000in}{3.696000in}}%
\pgfusepath{clip}%
\pgfsetrectcap%
\pgfsetroundjoin%
\pgfsetlinewidth{0.803000pt}%
\definecolor{currentstroke}{rgb}{0.690196,0.690196,0.690196}%
\pgfsetstrokecolor{currentstroke}%
\pgfsetdash{}{0pt}%
\pgfpathmoveto{\pgfqpoint{2.526202in}{0.528000in}}%
\pgfpathlineto{\pgfqpoint{2.526202in}{4.224000in}}%
\pgfusepath{stroke}%
\end{pgfscope}%
\begin{pgfscope}%
\pgfsetbuttcap%
\pgfsetroundjoin%
\definecolor{currentfill}{rgb}{0.000000,0.000000,0.000000}%
\pgfsetfillcolor{currentfill}%
\pgfsetlinewidth{0.803000pt}%
\definecolor{currentstroke}{rgb}{0.000000,0.000000,0.000000}%
\pgfsetstrokecolor{currentstroke}%
\pgfsetdash{}{0pt}%
\pgfsys@defobject{currentmarker}{\pgfqpoint{0.000000in}{-0.048611in}}{\pgfqpoint{0.000000in}{0.000000in}}{%
\pgfpathmoveto{\pgfqpoint{0.000000in}{0.000000in}}%
\pgfpathlineto{\pgfqpoint{0.000000in}{-0.048611in}}%
\pgfusepath{stroke,fill}%
}%
\begin{pgfscope}%
\pgfsys@transformshift{2.526202in}{0.528000in}%
\pgfsys@useobject{currentmarker}{}%
\end{pgfscope}%
\end{pgfscope}%
\begin{pgfscope}%
\definecolor{textcolor}{rgb}{0.000000,0.000000,0.000000}%
\pgfsetstrokecolor{textcolor}%
\pgfsetfillcolor{textcolor}%
\pgftext[x=2.526202in,y=0.430778in,,top]{\color{textcolor}\sffamily\fontsize{10.000000}{12.000000}\selectfont 10}%
\end{pgfscope}%
\begin{pgfscope}%
\pgfpathrectangle{\pgfqpoint{0.800000in}{0.528000in}}{\pgfqpoint{4.960000in}{3.696000in}}%
\pgfusepath{clip}%
\pgfsetrectcap%
\pgfsetroundjoin%
\pgfsetlinewidth{0.803000pt}%
\definecolor{currentstroke}{rgb}{0.690196,0.690196,0.690196}%
\pgfsetstrokecolor{currentstroke}%
\pgfsetdash{}{0pt}%
\pgfpathmoveto{\pgfqpoint{3.276576in}{0.528000in}}%
\pgfpathlineto{\pgfqpoint{3.276576in}{4.224000in}}%
\pgfusepath{stroke}%
\end{pgfscope}%
\begin{pgfscope}%
\pgfsetbuttcap%
\pgfsetroundjoin%
\definecolor{currentfill}{rgb}{0.000000,0.000000,0.000000}%
\pgfsetfillcolor{currentfill}%
\pgfsetlinewidth{0.803000pt}%
\definecolor{currentstroke}{rgb}{0.000000,0.000000,0.000000}%
\pgfsetstrokecolor{currentstroke}%
\pgfsetdash{}{0pt}%
\pgfsys@defobject{currentmarker}{\pgfqpoint{0.000000in}{-0.048611in}}{\pgfqpoint{0.000000in}{0.000000in}}{%
\pgfpathmoveto{\pgfqpoint{0.000000in}{0.000000in}}%
\pgfpathlineto{\pgfqpoint{0.000000in}{-0.048611in}}%
\pgfusepath{stroke,fill}%
}%
\begin{pgfscope}%
\pgfsys@transformshift{3.276576in}{0.528000in}%
\pgfsys@useobject{currentmarker}{}%
\end{pgfscope}%
\end{pgfscope}%
\begin{pgfscope}%
\definecolor{textcolor}{rgb}{0.000000,0.000000,0.000000}%
\pgfsetstrokecolor{textcolor}%
\pgfsetfillcolor{textcolor}%
\pgftext[x=3.276576in,y=0.430778in,,top]{\color{textcolor}\sffamily\fontsize{10.000000}{12.000000}\selectfont 15}%
\end{pgfscope}%
\begin{pgfscope}%
\pgfpathrectangle{\pgfqpoint{0.800000in}{0.528000in}}{\pgfqpoint{4.960000in}{3.696000in}}%
\pgfusepath{clip}%
\pgfsetrectcap%
\pgfsetroundjoin%
\pgfsetlinewidth{0.803000pt}%
\definecolor{currentstroke}{rgb}{0.690196,0.690196,0.690196}%
\pgfsetstrokecolor{currentstroke}%
\pgfsetdash{}{0pt}%
\pgfpathmoveto{\pgfqpoint{4.026950in}{0.528000in}}%
\pgfpathlineto{\pgfqpoint{4.026950in}{4.224000in}}%
\pgfusepath{stroke}%
\end{pgfscope}%
\begin{pgfscope}%
\pgfsetbuttcap%
\pgfsetroundjoin%
\definecolor{currentfill}{rgb}{0.000000,0.000000,0.000000}%
\pgfsetfillcolor{currentfill}%
\pgfsetlinewidth{0.803000pt}%
\definecolor{currentstroke}{rgb}{0.000000,0.000000,0.000000}%
\pgfsetstrokecolor{currentstroke}%
\pgfsetdash{}{0pt}%
\pgfsys@defobject{currentmarker}{\pgfqpoint{0.000000in}{-0.048611in}}{\pgfqpoint{0.000000in}{0.000000in}}{%
\pgfpathmoveto{\pgfqpoint{0.000000in}{0.000000in}}%
\pgfpathlineto{\pgfqpoint{0.000000in}{-0.048611in}}%
\pgfusepath{stroke,fill}%
}%
\begin{pgfscope}%
\pgfsys@transformshift{4.026950in}{0.528000in}%
\pgfsys@useobject{currentmarker}{}%
\end{pgfscope}%
\end{pgfscope}%
\begin{pgfscope}%
\definecolor{textcolor}{rgb}{0.000000,0.000000,0.000000}%
\pgfsetstrokecolor{textcolor}%
\pgfsetfillcolor{textcolor}%
\pgftext[x=4.026950in,y=0.430778in,,top]{\color{textcolor}\sffamily\fontsize{10.000000}{12.000000}\selectfont 20}%
\end{pgfscope}%
\begin{pgfscope}%
\pgfpathrectangle{\pgfqpoint{0.800000in}{0.528000in}}{\pgfqpoint{4.960000in}{3.696000in}}%
\pgfusepath{clip}%
\pgfsetrectcap%
\pgfsetroundjoin%
\pgfsetlinewidth{0.803000pt}%
\definecolor{currentstroke}{rgb}{0.690196,0.690196,0.690196}%
\pgfsetstrokecolor{currentstroke}%
\pgfsetdash{}{0pt}%
\pgfpathmoveto{\pgfqpoint{4.777324in}{0.528000in}}%
\pgfpathlineto{\pgfqpoint{4.777324in}{4.224000in}}%
\pgfusepath{stroke}%
\end{pgfscope}%
\begin{pgfscope}%
\pgfsetbuttcap%
\pgfsetroundjoin%
\definecolor{currentfill}{rgb}{0.000000,0.000000,0.000000}%
\pgfsetfillcolor{currentfill}%
\pgfsetlinewidth{0.803000pt}%
\definecolor{currentstroke}{rgb}{0.000000,0.000000,0.000000}%
\pgfsetstrokecolor{currentstroke}%
\pgfsetdash{}{0pt}%
\pgfsys@defobject{currentmarker}{\pgfqpoint{0.000000in}{-0.048611in}}{\pgfqpoint{0.000000in}{0.000000in}}{%
\pgfpathmoveto{\pgfqpoint{0.000000in}{0.000000in}}%
\pgfpathlineto{\pgfqpoint{0.000000in}{-0.048611in}}%
\pgfusepath{stroke,fill}%
}%
\begin{pgfscope}%
\pgfsys@transformshift{4.777324in}{0.528000in}%
\pgfsys@useobject{currentmarker}{}%
\end{pgfscope}%
\end{pgfscope}%
\begin{pgfscope}%
\definecolor{textcolor}{rgb}{0.000000,0.000000,0.000000}%
\pgfsetstrokecolor{textcolor}%
\pgfsetfillcolor{textcolor}%
\pgftext[x=4.777324in,y=0.430778in,,top]{\color{textcolor}\sffamily\fontsize{10.000000}{12.000000}\selectfont 25}%
\end{pgfscope}%
\begin{pgfscope}%
\pgfpathrectangle{\pgfqpoint{0.800000in}{0.528000in}}{\pgfqpoint{4.960000in}{3.696000in}}%
\pgfusepath{clip}%
\pgfsetrectcap%
\pgfsetroundjoin%
\pgfsetlinewidth{0.803000pt}%
\definecolor{currentstroke}{rgb}{0.690196,0.690196,0.690196}%
\pgfsetstrokecolor{currentstroke}%
\pgfsetdash{}{0pt}%
\pgfpathmoveto{\pgfqpoint{5.527698in}{0.528000in}}%
\pgfpathlineto{\pgfqpoint{5.527698in}{4.224000in}}%
\pgfusepath{stroke}%
\end{pgfscope}%
\begin{pgfscope}%
\pgfsetbuttcap%
\pgfsetroundjoin%
\definecolor{currentfill}{rgb}{0.000000,0.000000,0.000000}%
\pgfsetfillcolor{currentfill}%
\pgfsetlinewidth{0.803000pt}%
\definecolor{currentstroke}{rgb}{0.000000,0.000000,0.000000}%
\pgfsetstrokecolor{currentstroke}%
\pgfsetdash{}{0pt}%
\pgfsys@defobject{currentmarker}{\pgfqpoint{0.000000in}{-0.048611in}}{\pgfqpoint{0.000000in}{0.000000in}}{%
\pgfpathmoveto{\pgfqpoint{0.000000in}{0.000000in}}%
\pgfpathlineto{\pgfqpoint{0.000000in}{-0.048611in}}%
\pgfusepath{stroke,fill}%
}%
\begin{pgfscope}%
\pgfsys@transformshift{5.527698in}{0.528000in}%
\pgfsys@useobject{currentmarker}{}%
\end{pgfscope}%
\end{pgfscope}%
\begin{pgfscope}%
\definecolor{textcolor}{rgb}{0.000000,0.000000,0.000000}%
\pgfsetstrokecolor{textcolor}%
\pgfsetfillcolor{textcolor}%
\pgftext[x=5.527698in,y=0.430778in,,top]{\color{textcolor}\sffamily\fontsize{10.000000}{12.000000}\selectfont 30}%
\end{pgfscope}%
\begin{pgfscope}%
\definecolor{textcolor}{rgb}{0.000000,0.000000,0.000000}%
\pgfsetstrokecolor{textcolor}%
\pgfsetfillcolor{textcolor}%
\pgftext[x=3.280000in,y=0.240809in,,top]{\color{textcolor}\sffamily\fontsize{10.000000}{12.000000}\selectfont time [s]}%
\end{pgfscope}%
\begin{pgfscope}%
\pgfpathrectangle{\pgfqpoint{0.800000in}{0.528000in}}{\pgfqpoint{4.960000in}{3.696000in}}%
\pgfusepath{clip}%
\pgfsetrectcap%
\pgfsetroundjoin%
\pgfsetlinewidth{0.803000pt}%
\definecolor{currentstroke}{rgb}{0.690196,0.690196,0.690196}%
\pgfsetstrokecolor{currentstroke}%
\pgfsetdash{}{0pt}%
\pgfpathmoveto{\pgfqpoint{0.800000in}{0.830417in}}%
\pgfpathlineto{\pgfqpoint{5.760000in}{0.830417in}}%
\pgfusepath{stroke}%
\end{pgfscope}%
\begin{pgfscope}%
\pgfsetbuttcap%
\pgfsetroundjoin%
\definecolor{currentfill}{rgb}{0.000000,0.000000,0.000000}%
\pgfsetfillcolor{currentfill}%
\pgfsetlinewidth{0.803000pt}%
\definecolor{currentstroke}{rgb}{0.000000,0.000000,0.000000}%
\pgfsetstrokecolor{currentstroke}%
\pgfsetdash{}{0pt}%
\pgfsys@defobject{currentmarker}{\pgfqpoint{-0.048611in}{0.000000in}}{\pgfqpoint{-0.000000in}{0.000000in}}{%
\pgfpathmoveto{\pgfqpoint{-0.000000in}{0.000000in}}%
\pgfpathlineto{\pgfqpoint{-0.048611in}{0.000000in}}%
\pgfusepath{stroke,fill}%
}%
\begin{pgfscope}%
\pgfsys@transformshift{0.800000in}{0.830417in}%
\pgfsys@useobject{currentmarker}{}%
\end{pgfscope}%
\end{pgfscope}%
\begin{pgfscope}%
\definecolor{textcolor}{rgb}{0.000000,0.000000,0.000000}%
\pgfsetstrokecolor{textcolor}%
\pgfsetfillcolor{textcolor}%
\pgftext[x=0.285508in, y=0.777656in, left, base]{\color{textcolor}\sffamily\fontsize{10.000000}{12.000000}\selectfont \ensuremath{-}0.20}%
\end{pgfscope}%
\begin{pgfscope}%
\pgfpathrectangle{\pgfqpoint{0.800000in}{0.528000in}}{\pgfqpoint{4.960000in}{3.696000in}}%
\pgfusepath{clip}%
\pgfsetrectcap%
\pgfsetroundjoin%
\pgfsetlinewidth{0.803000pt}%
\definecolor{currentstroke}{rgb}{0.690196,0.690196,0.690196}%
\pgfsetstrokecolor{currentstroke}%
\pgfsetdash{}{0pt}%
\pgfpathmoveto{\pgfqpoint{0.800000in}{1.483413in}}%
\pgfpathlineto{\pgfqpoint{5.760000in}{1.483413in}}%
\pgfusepath{stroke}%
\end{pgfscope}%
\begin{pgfscope}%
\pgfsetbuttcap%
\pgfsetroundjoin%
\definecolor{currentfill}{rgb}{0.000000,0.000000,0.000000}%
\pgfsetfillcolor{currentfill}%
\pgfsetlinewidth{0.803000pt}%
\definecolor{currentstroke}{rgb}{0.000000,0.000000,0.000000}%
\pgfsetstrokecolor{currentstroke}%
\pgfsetdash{}{0pt}%
\pgfsys@defobject{currentmarker}{\pgfqpoint{-0.048611in}{0.000000in}}{\pgfqpoint{-0.000000in}{0.000000in}}{%
\pgfpathmoveto{\pgfqpoint{-0.000000in}{0.000000in}}%
\pgfpathlineto{\pgfqpoint{-0.048611in}{0.000000in}}%
\pgfusepath{stroke,fill}%
}%
\begin{pgfscope}%
\pgfsys@transformshift{0.800000in}{1.483413in}%
\pgfsys@useobject{currentmarker}{}%
\end{pgfscope}%
\end{pgfscope}%
\begin{pgfscope}%
\definecolor{textcolor}{rgb}{0.000000,0.000000,0.000000}%
\pgfsetstrokecolor{textcolor}%
\pgfsetfillcolor{textcolor}%
\pgftext[x=0.285508in, y=1.430651in, left, base]{\color{textcolor}\sffamily\fontsize{10.000000}{12.000000}\selectfont \ensuremath{-}0.15}%
\end{pgfscope}%
\begin{pgfscope}%
\pgfpathrectangle{\pgfqpoint{0.800000in}{0.528000in}}{\pgfqpoint{4.960000in}{3.696000in}}%
\pgfusepath{clip}%
\pgfsetrectcap%
\pgfsetroundjoin%
\pgfsetlinewidth{0.803000pt}%
\definecolor{currentstroke}{rgb}{0.690196,0.690196,0.690196}%
\pgfsetstrokecolor{currentstroke}%
\pgfsetdash{}{0pt}%
\pgfpathmoveto{\pgfqpoint{0.800000in}{2.136408in}}%
\pgfpathlineto{\pgfqpoint{5.760000in}{2.136408in}}%
\pgfusepath{stroke}%
\end{pgfscope}%
\begin{pgfscope}%
\pgfsetbuttcap%
\pgfsetroundjoin%
\definecolor{currentfill}{rgb}{0.000000,0.000000,0.000000}%
\pgfsetfillcolor{currentfill}%
\pgfsetlinewidth{0.803000pt}%
\definecolor{currentstroke}{rgb}{0.000000,0.000000,0.000000}%
\pgfsetstrokecolor{currentstroke}%
\pgfsetdash{}{0pt}%
\pgfsys@defobject{currentmarker}{\pgfqpoint{-0.048611in}{0.000000in}}{\pgfqpoint{-0.000000in}{0.000000in}}{%
\pgfpathmoveto{\pgfqpoint{-0.000000in}{0.000000in}}%
\pgfpathlineto{\pgfqpoint{-0.048611in}{0.000000in}}%
\pgfusepath{stroke,fill}%
}%
\begin{pgfscope}%
\pgfsys@transformshift{0.800000in}{2.136408in}%
\pgfsys@useobject{currentmarker}{}%
\end{pgfscope}%
\end{pgfscope}%
\begin{pgfscope}%
\definecolor{textcolor}{rgb}{0.000000,0.000000,0.000000}%
\pgfsetstrokecolor{textcolor}%
\pgfsetfillcolor{textcolor}%
\pgftext[x=0.285508in, y=2.083647in, left, base]{\color{textcolor}\sffamily\fontsize{10.000000}{12.000000}\selectfont \ensuremath{-}0.10}%
\end{pgfscope}%
\begin{pgfscope}%
\pgfpathrectangle{\pgfqpoint{0.800000in}{0.528000in}}{\pgfqpoint{4.960000in}{3.696000in}}%
\pgfusepath{clip}%
\pgfsetrectcap%
\pgfsetroundjoin%
\pgfsetlinewidth{0.803000pt}%
\definecolor{currentstroke}{rgb}{0.690196,0.690196,0.690196}%
\pgfsetstrokecolor{currentstroke}%
\pgfsetdash{}{0pt}%
\pgfpathmoveto{\pgfqpoint{0.800000in}{2.789404in}}%
\pgfpathlineto{\pgfqpoint{5.760000in}{2.789404in}}%
\pgfusepath{stroke}%
\end{pgfscope}%
\begin{pgfscope}%
\pgfsetbuttcap%
\pgfsetroundjoin%
\definecolor{currentfill}{rgb}{0.000000,0.000000,0.000000}%
\pgfsetfillcolor{currentfill}%
\pgfsetlinewidth{0.803000pt}%
\definecolor{currentstroke}{rgb}{0.000000,0.000000,0.000000}%
\pgfsetstrokecolor{currentstroke}%
\pgfsetdash{}{0pt}%
\pgfsys@defobject{currentmarker}{\pgfqpoint{-0.048611in}{0.000000in}}{\pgfqpoint{-0.000000in}{0.000000in}}{%
\pgfpathmoveto{\pgfqpoint{-0.000000in}{0.000000in}}%
\pgfpathlineto{\pgfqpoint{-0.048611in}{0.000000in}}%
\pgfusepath{stroke,fill}%
}%
\begin{pgfscope}%
\pgfsys@transformshift{0.800000in}{2.789404in}%
\pgfsys@useobject{currentmarker}{}%
\end{pgfscope}%
\end{pgfscope}%
\begin{pgfscope}%
\definecolor{textcolor}{rgb}{0.000000,0.000000,0.000000}%
\pgfsetstrokecolor{textcolor}%
\pgfsetfillcolor{textcolor}%
\pgftext[x=0.285508in, y=2.736642in, left, base]{\color{textcolor}\sffamily\fontsize{10.000000}{12.000000}\selectfont \ensuremath{-}0.05}%
\end{pgfscope}%
\begin{pgfscope}%
\pgfpathrectangle{\pgfqpoint{0.800000in}{0.528000in}}{\pgfqpoint{4.960000in}{3.696000in}}%
\pgfusepath{clip}%
\pgfsetrectcap%
\pgfsetroundjoin%
\pgfsetlinewidth{0.803000pt}%
\definecolor{currentstroke}{rgb}{0.690196,0.690196,0.690196}%
\pgfsetstrokecolor{currentstroke}%
\pgfsetdash{}{0pt}%
\pgfpathmoveto{\pgfqpoint{0.800000in}{3.442399in}}%
\pgfpathlineto{\pgfqpoint{5.760000in}{3.442399in}}%
\pgfusepath{stroke}%
\end{pgfscope}%
\begin{pgfscope}%
\pgfsetbuttcap%
\pgfsetroundjoin%
\definecolor{currentfill}{rgb}{0.000000,0.000000,0.000000}%
\pgfsetfillcolor{currentfill}%
\pgfsetlinewidth{0.803000pt}%
\definecolor{currentstroke}{rgb}{0.000000,0.000000,0.000000}%
\pgfsetstrokecolor{currentstroke}%
\pgfsetdash{}{0pt}%
\pgfsys@defobject{currentmarker}{\pgfqpoint{-0.048611in}{0.000000in}}{\pgfqpoint{-0.000000in}{0.000000in}}{%
\pgfpathmoveto{\pgfqpoint{-0.000000in}{0.000000in}}%
\pgfpathlineto{\pgfqpoint{-0.048611in}{0.000000in}}%
\pgfusepath{stroke,fill}%
}%
\begin{pgfscope}%
\pgfsys@transformshift{0.800000in}{3.442399in}%
\pgfsys@useobject{currentmarker}{}%
\end{pgfscope}%
\end{pgfscope}%
\begin{pgfscope}%
\definecolor{textcolor}{rgb}{0.000000,0.000000,0.000000}%
\pgfsetstrokecolor{textcolor}%
\pgfsetfillcolor{textcolor}%
\pgftext[x=0.393533in, y=3.389638in, left, base]{\color{textcolor}\sffamily\fontsize{10.000000}{12.000000}\selectfont 0.00}%
\end{pgfscope}%
\begin{pgfscope}%
\pgfpathrectangle{\pgfqpoint{0.800000in}{0.528000in}}{\pgfqpoint{4.960000in}{3.696000in}}%
\pgfusepath{clip}%
\pgfsetrectcap%
\pgfsetroundjoin%
\pgfsetlinewidth{0.803000pt}%
\definecolor{currentstroke}{rgb}{0.690196,0.690196,0.690196}%
\pgfsetstrokecolor{currentstroke}%
\pgfsetdash{}{0pt}%
\pgfpathmoveto{\pgfqpoint{0.800000in}{4.095395in}}%
\pgfpathlineto{\pgfqpoint{5.760000in}{4.095395in}}%
\pgfusepath{stroke}%
\end{pgfscope}%
\begin{pgfscope}%
\pgfsetbuttcap%
\pgfsetroundjoin%
\definecolor{currentfill}{rgb}{0.000000,0.000000,0.000000}%
\pgfsetfillcolor{currentfill}%
\pgfsetlinewidth{0.803000pt}%
\definecolor{currentstroke}{rgb}{0.000000,0.000000,0.000000}%
\pgfsetstrokecolor{currentstroke}%
\pgfsetdash{}{0pt}%
\pgfsys@defobject{currentmarker}{\pgfqpoint{-0.048611in}{0.000000in}}{\pgfqpoint{-0.000000in}{0.000000in}}{%
\pgfpathmoveto{\pgfqpoint{-0.000000in}{0.000000in}}%
\pgfpathlineto{\pgfqpoint{-0.048611in}{0.000000in}}%
\pgfusepath{stroke,fill}%
}%
\begin{pgfscope}%
\pgfsys@transformshift{0.800000in}{4.095395in}%
\pgfsys@useobject{currentmarker}{}%
\end{pgfscope}%
\end{pgfscope}%
\begin{pgfscope}%
\definecolor{textcolor}{rgb}{0.000000,0.000000,0.000000}%
\pgfsetstrokecolor{textcolor}%
\pgfsetfillcolor{textcolor}%
\pgftext[x=0.393533in, y=4.042633in, left, base]{\color{textcolor}\sffamily\fontsize{10.000000}{12.000000}\selectfont 0.05}%
\end{pgfscope}%
\begin{pgfscope}%
\definecolor{textcolor}{rgb}{0.000000,0.000000,0.000000}%
\pgfsetstrokecolor{textcolor}%
\pgfsetfillcolor{textcolor}%
\pgftext[x=0.229952in,y=2.376000in,,bottom,rotate=90.000000]{\color{textcolor}\sffamily\fontsize{10.000000}{12.000000}\selectfont Computed error [-]}%
\end{pgfscope}%
\begin{pgfscope}%
\pgfpathrectangle{\pgfqpoint{0.800000in}{0.528000in}}{\pgfqpoint{4.960000in}{3.696000in}}%
\pgfusepath{clip}%
\pgfsetrectcap%
\pgfsetroundjoin%
\pgfsetlinewidth{1.505625pt}%
\definecolor{currentstroke}{rgb}{0.121569,0.466667,0.705882}%
\pgfsetstrokecolor{currentstroke}%
\pgfsetdash{}{0pt}%
\pgfpathmoveto{\pgfqpoint{1.025455in}{0.696000in}}%
\pgfpathlineto{\pgfqpoint{1.079346in}{0.717509in}}%
\pgfpathlineto{\pgfqpoint{1.133733in}{0.880055in}}%
\pgfpathlineto{\pgfqpoint{1.188343in}{0.902226in}}%
\pgfpathlineto{\pgfqpoint{1.243001in}{0.977824in}}%
\pgfpathlineto{\pgfqpoint{1.297692in}{1.415981in}}%
\pgfpathlineto{\pgfqpoint{1.350599in}{2.140815in}}%
\pgfpathlineto{\pgfqpoint{1.404959in}{2.952804in}}%
\pgfpathlineto{\pgfqpoint{1.459957in}{3.168099in}}%
\pgfpathlineto{\pgfqpoint{1.514486in}{3.385960in}}%
\pgfpathlineto{\pgfqpoint{1.569896in}{3.467601in}}%
\pgfpathlineto{\pgfqpoint{1.623211in}{3.434230in}}%
\pgfpathlineto{\pgfqpoint{1.677396in}{3.510477in}}%
\pgfpathlineto{\pgfqpoint{1.731440in}{3.484935in}}%
\pgfpathlineto{\pgfqpoint{1.785614in}{3.413717in}}%
\pgfpathlineto{\pgfqpoint{1.839797in}{3.546456in}}%
\pgfpathlineto{\pgfqpoint{1.894125in}{3.471621in}}%
\pgfpathlineto{\pgfqpoint{1.948443in}{3.548777in}}%
\pgfpathlineto{\pgfqpoint{2.002587in}{3.486713in}}%
\pgfpathlineto{\pgfqpoint{2.056571in}{3.531756in}}%
\pgfpathlineto{\pgfqpoint{2.111067in}{3.513038in}}%
\pgfpathlineto{\pgfqpoint{2.165740in}{3.460136in}}%
\pgfpathlineto{\pgfqpoint{2.219578in}{3.386115in}}%
\pgfpathlineto{\pgfqpoint{2.273625in}{3.381915in}}%
\pgfpathlineto{\pgfqpoint{2.327972in}{3.359004in}}%
\pgfpathlineto{\pgfqpoint{2.382158in}{3.288322in}}%
\pgfpathlineto{\pgfqpoint{2.436201in}{3.235030in}}%
\pgfpathlineto{\pgfqpoint{2.490727in}{3.243621in}}%
\pgfpathlineto{\pgfqpoint{2.545152in}{3.228542in}}%
\pgfpathlineto{\pgfqpoint{2.599168in}{3.364304in}}%
\pgfpathlineto{\pgfqpoint{2.653162in}{3.409516in}}%
\pgfpathlineto{\pgfqpoint{2.708816in}{3.409229in}}%
\pgfpathlineto{\pgfqpoint{2.763068in}{3.445266in}}%
\pgfpathlineto{\pgfqpoint{2.817233in}{3.410488in}}%
\pgfpathlineto{\pgfqpoint{2.871345in}{3.436483in}}%
\pgfpathlineto{\pgfqpoint{2.925498in}{3.381273in}}%
\pgfpathlineto{\pgfqpoint{2.979648in}{3.412039in}}%
\pgfpathlineto{\pgfqpoint{3.033884in}{3.401420in}}%
\pgfpathlineto{\pgfqpoint{3.088277in}{3.468629in}}%
\pgfpathlineto{\pgfqpoint{3.142459in}{3.405399in}}%
\pgfpathlineto{\pgfqpoint{3.196688in}{3.413394in}}%
\pgfpathlineto{\pgfqpoint{3.250942in}{3.404996in}}%
\pgfpathlineto{\pgfqpoint{3.305168in}{3.401400in}}%
\pgfpathlineto{\pgfqpoint{3.360651in}{3.443640in}}%
\pgfpathlineto{\pgfqpoint{3.414236in}{3.410923in}}%
\pgfpathlineto{\pgfqpoint{3.468063in}{3.438273in}}%
\pgfpathlineto{\pgfqpoint{3.523055in}{3.419916in}}%
\pgfpathlineto{\pgfqpoint{3.577067in}{3.437086in}}%
\pgfpathlineto{\pgfqpoint{3.631733in}{3.511086in}}%
\pgfpathlineto{\pgfqpoint{3.686018in}{3.484979in}}%
\pgfpathlineto{\pgfqpoint{3.740366in}{3.470195in}}%
\pgfpathlineto{\pgfqpoint{3.794333in}{3.456821in}}%
\pgfpathlineto{\pgfqpoint{3.848511in}{3.444544in}}%
\pgfpathlineto{\pgfqpoint{3.902528in}{3.419848in}}%
\pgfpathlineto{\pgfqpoint{3.958208in}{3.410977in}}%
\pgfpathlineto{\pgfqpoint{4.011516in}{3.400406in}}%
\pgfpathlineto{\pgfqpoint{4.065429in}{3.397419in}}%
\pgfpathlineto{\pgfqpoint{4.119969in}{3.394635in}}%
\pgfpathlineto{\pgfqpoint{4.174054in}{3.378929in}}%
\pgfpathlineto{\pgfqpoint{4.228102in}{3.376961in}}%
\pgfpathlineto{\pgfqpoint{4.282096in}{3.341317in}}%
\pgfpathlineto{\pgfqpoint{4.336947in}{3.345249in}}%
\pgfpathlineto{\pgfqpoint{4.390525in}{3.345907in}}%
\pgfpathlineto{\pgfqpoint{4.444795in}{3.359748in}}%
\pgfpathlineto{\pgfqpoint{4.499103in}{3.409287in}}%
\pgfpathlineto{\pgfqpoint{4.553310in}{3.430233in}}%
\pgfpathlineto{\pgfqpoint{4.608482in}{3.443836in}}%
\pgfpathlineto{\pgfqpoint{4.662064in}{3.466285in}}%
\pgfpathlineto{\pgfqpoint{4.715946in}{3.480097in}}%
\pgfpathlineto{\pgfqpoint{4.770242in}{3.470940in}}%
\pgfpathlineto{\pgfqpoint{4.824550in}{3.457810in}}%
\pgfpathlineto{\pgfqpoint{4.878651in}{3.458707in}}%
\pgfpathlineto{\pgfqpoint{4.933610in}{3.481666in}}%
\pgfpathlineto{\pgfqpoint{4.987108in}{3.500575in}}%
\pgfpathlineto{\pgfqpoint{5.041786in}{3.489318in}}%
\pgfpathlineto{\pgfqpoint{5.096356in}{3.475845in}}%
\pgfpathlineto{\pgfqpoint{5.150043in}{3.443358in}}%
\pgfpathlineto{\pgfqpoint{5.205548in}{3.440249in}}%
\pgfpathlineto{\pgfqpoint{5.259176in}{3.352451in}}%
\pgfpathlineto{\pgfqpoint{5.314060in}{3.346494in}}%
\pgfpathlineto{\pgfqpoint{5.368522in}{3.345982in}}%
\pgfpathlineto{\pgfqpoint{5.422471in}{3.322482in}}%
\pgfpathlineto{\pgfqpoint{5.476713in}{3.314062in}}%
\pgfpathlineto{\pgfqpoint{5.531850in}{3.317520in}}%
\pgfusepath{stroke}%
\end{pgfscope}%
\begin{pgfscope}%
\pgfpathrectangle{\pgfqpoint{0.800000in}{0.528000in}}{\pgfqpoint{4.960000in}{3.696000in}}%
\pgfusepath{clip}%
\pgfsetrectcap%
\pgfsetroundjoin%
\pgfsetlinewidth{1.505625pt}%
\definecolor{currentstroke}{rgb}{1.000000,0.498039,0.054902}%
\pgfsetstrokecolor{currentstroke}%
\pgfsetdash{}{0pt}%
\pgfpathmoveto{\pgfqpoint{1.025455in}{0.903689in}}%
\pgfpathlineto{\pgfqpoint{1.079097in}{0.983401in}}%
\pgfpathlineto{\pgfqpoint{1.133290in}{0.980752in}}%
\pgfpathlineto{\pgfqpoint{1.187648in}{1.023720in}}%
\pgfpathlineto{\pgfqpoint{1.242800in}{1.386160in}}%
\pgfpathlineto{\pgfqpoint{1.297032in}{1.961676in}}%
\pgfpathlineto{\pgfqpoint{1.351342in}{2.582757in}}%
\pgfpathlineto{\pgfqpoint{1.405625in}{2.923091in}}%
\pgfpathlineto{\pgfqpoint{1.459821in}{3.279534in}}%
\pgfpathlineto{\pgfqpoint{1.514206in}{3.258595in}}%
\pgfpathlineto{\pgfqpoint{1.569696in}{3.373448in}}%
\pgfpathlineto{\pgfqpoint{1.623165in}{3.429897in}}%
\pgfpathlineto{\pgfqpoint{1.676738in}{3.508713in}}%
\pgfpathlineto{\pgfqpoint{1.731575in}{3.553593in}}%
\pgfpathlineto{\pgfqpoint{1.786338in}{3.577241in}}%
\pgfpathlineto{\pgfqpoint{1.840720in}{3.573354in}}%
\pgfpathlineto{\pgfqpoint{1.894971in}{3.482491in}}%
\pgfpathlineto{\pgfqpoint{1.949451in}{3.444085in}}%
\pgfpathlineto{\pgfqpoint{2.003523in}{3.480460in}}%
\pgfpathlineto{\pgfqpoint{2.057638in}{3.459977in}}%
\pgfpathlineto{\pgfqpoint{2.111872in}{3.431935in}}%
\pgfpathlineto{\pgfqpoint{2.167806in}{3.376167in}}%
\pgfpathlineto{\pgfqpoint{2.221402in}{3.301477in}}%
\pgfpathlineto{\pgfqpoint{2.275305in}{3.260438in}}%
\pgfpathlineto{\pgfqpoint{2.329434in}{3.265712in}}%
\pgfpathlineto{\pgfqpoint{2.383566in}{3.301497in}}%
\pgfpathlineto{\pgfqpoint{2.437640in}{3.285874in}}%
\pgfpathlineto{\pgfqpoint{2.491996in}{3.524936in}}%
\pgfpathlineto{\pgfqpoint{2.546218in}{3.434170in}}%
\pgfpathlineto{\pgfqpoint{2.600534in}{3.427705in}}%
\pgfpathlineto{\pgfqpoint{2.654787in}{3.432722in}}%
\pgfpathlineto{\pgfqpoint{2.709027in}{3.435356in}}%
\pgfpathlineto{\pgfqpoint{2.763930in}{3.449159in}}%
\pgfpathlineto{\pgfqpoint{2.817830in}{3.475180in}}%
\pgfpathlineto{\pgfqpoint{2.871752in}{3.465088in}}%
\pgfpathlineto{\pgfqpoint{2.925837in}{3.461382in}}%
\pgfpathlineto{\pgfqpoint{2.980536in}{3.446847in}}%
\pgfpathlineto{\pgfqpoint{3.034703in}{3.474948in}}%
\pgfpathlineto{\pgfqpoint{3.089039in}{3.471862in}}%
\pgfpathlineto{\pgfqpoint{3.143307in}{3.470574in}}%
\pgfpathlineto{\pgfqpoint{3.197327in}{3.458213in}}%
\pgfpathlineto{\pgfqpoint{3.251469in}{3.449046in}}%
\pgfpathlineto{\pgfqpoint{3.305853in}{3.470094in}}%
\pgfpathlineto{\pgfqpoint{3.361722in}{3.472348in}}%
\pgfpathlineto{\pgfqpoint{3.415028in}{3.459751in}}%
\pgfpathlineto{\pgfqpoint{3.469019in}{3.449523in}}%
\pgfpathlineto{\pgfqpoint{3.523110in}{3.400374in}}%
\pgfpathlineto{\pgfqpoint{3.577426in}{3.363754in}}%
\pgfpathlineto{\pgfqpoint{3.631486in}{3.378617in}}%
\pgfpathlineto{\pgfqpoint{3.686172in}{3.344255in}}%
\pgfpathlineto{\pgfqpoint{3.740698in}{3.374424in}}%
\pgfpathlineto{\pgfqpoint{3.794630in}{3.358991in}}%
\pgfpathlineto{\pgfqpoint{3.848700in}{3.377665in}}%
\pgfpathlineto{\pgfqpoint{3.904942in}{3.379886in}}%
\pgfpathlineto{\pgfqpoint{3.958943in}{3.609826in}}%
\pgfpathlineto{\pgfqpoint{4.012177in}{3.585799in}}%
\pgfpathlineto{\pgfqpoint{4.066136in}{3.545914in}}%
\pgfpathlineto{\pgfqpoint{4.120402in}{3.627681in}}%
\pgfpathlineto{\pgfqpoint{4.174524in}{3.596980in}}%
\pgfpathlineto{\pgfqpoint{4.228795in}{3.416024in}}%
\pgfpathlineto{\pgfqpoint{4.283227in}{3.342456in}}%
\pgfpathlineto{\pgfqpoint{4.337654in}{3.354539in}}%
\pgfpathlineto{\pgfqpoint{4.391591in}{3.377705in}}%
\pgfpathlineto{\pgfqpoint{4.445858in}{3.399633in}}%
\pgfpathlineto{\pgfqpoint{4.500443in}{3.407180in}}%
\pgfpathlineto{\pgfqpoint{4.555957in}{3.415861in}}%
\pgfpathlineto{\pgfqpoint{4.609229in}{3.414311in}}%
\pgfpathlineto{\pgfqpoint{4.663051in}{3.415161in}}%
\pgfpathlineto{\pgfqpoint{4.716973in}{3.406361in}}%
\pgfpathlineto{\pgfqpoint{4.771149in}{3.418479in}}%
\pgfpathlineto{\pgfqpoint{4.825460in}{3.421607in}}%
\pgfpathlineto{\pgfqpoint{4.879583in}{3.422554in}}%
\pgfpathlineto{\pgfqpoint{4.935368in}{3.441633in}}%
\pgfpathlineto{\pgfqpoint{4.989369in}{3.439556in}}%
\pgfpathlineto{\pgfqpoint{5.043418in}{3.439290in}}%
\pgfpathlineto{\pgfqpoint{5.097613in}{3.434250in}}%
\pgfpathlineto{\pgfqpoint{5.153554in}{3.433638in}}%
\pgfpathlineto{\pgfqpoint{5.206849in}{3.433414in}}%
\pgfpathlineto{\pgfqpoint{5.260579in}{3.426990in}}%
\pgfpathlineto{\pgfqpoint{5.314728in}{3.423036in}}%
\pgfpathlineto{\pgfqpoint{5.368880in}{3.436584in}}%
\pgfpathlineto{\pgfqpoint{5.423031in}{3.436620in}}%
\pgfpathlineto{\pgfqpoint{5.477563in}{3.443374in}}%
\pgfpathlineto{\pgfqpoint{5.531987in}{3.442148in}}%
\pgfusepath{stroke}%
\end{pgfscope}%
\begin{pgfscope}%
\pgfpathrectangle{\pgfqpoint{0.800000in}{0.528000in}}{\pgfqpoint{4.960000in}{3.696000in}}%
\pgfusepath{clip}%
\pgfsetrectcap%
\pgfsetroundjoin%
\pgfsetlinewidth{1.505625pt}%
\definecolor{currentstroke}{rgb}{0.172549,0.627451,0.172549}%
\pgfsetstrokecolor{currentstroke}%
\pgfsetdash{}{0pt}%
\pgfpathmoveto{\pgfqpoint{1.025455in}{0.837280in}}%
\pgfpathlineto{\pgfqpoint{1.079118in}{0.971734in}}%
\pgfpathlineto{\pgfqpoint{1.133167in}{1.103778in}}%
\pgfpathlineto{\pgfqpoint{1.187559in}{1.039022in}}%
\pgfpathlineto{\pgfqpoint{1.241834in}{1.171276in}}%
\pgfpathlineto{\pgfqpoint{1.298213in}{1.409724in}}%
\pgfpathlineto{\pgfqpoint{1.350148in}{1.963231in}}%
\pgfpathlineto{\pgfqpoint{1.404149in}{2.690349in}}%
\pgfpathlineto{\pgfqpoint{1.458653in}{3.151103in}}%
\pgfpathlineto{\pgfqpoint{1.514270in}{3.549225in}}%
\pgfpathlineto{\pgfqpoint{1.567555in}{3.616784in}}%
\pgfpathlineto{\pgfqpoint{1.622013in}{3.689229in}}%
\pgfpathlineto{\pgfqpoint{1.676086in}{3.713897in}}%
\pgfpathlineto{\pgfqpoint{1.730346in}{3.882629in}}%
\pgfpathlineto{\pgfqpoint{1.784400in}{3.748228in}}%
\pgfpathlineto{\pgfqpoint{1.838929in}{3.615682in}}%
\pgfpathlineto{\pgfqpoint{1.895431in}{3.781532in}}%
\pgfpathlineto{\pgfqpoint{1.947023in}{3.680395in}}%
\pgfpathlineto{\pgfqpoint{2.001357in}{3.594524in}}%
\pgfpathlineto{\pgfqpoint{2.055546in}{3.280918in}}%
\pgfpathlineto{\pgfqpoint{2.111687in}{2.947138in}}%
\pgfpathlineto{\pgfqpoint{2.165154in}{2.733071in}}%
\pgfpathlineto{\pgfqpoint{2.219132in}{2.631164in}}%
\pgfpathlineto{\pgfqpoint{2.273327in}{2.817997in}}%
\pgfpathlineto{\pgfqpoint{2.327792in}{3.474692in}}%
\pgfpathlineto{\pgfqpoint{2.381690in}{3.585005in}}%
\pgfpathlineto{\pgfqpoint{2.435983in}{3.674972in}}%
\pgfpathlineto{\pgfqpoint{2.490271in}{3.774525in}}%
\pgfpathlineto{\pgfqpoint{2.545021in}{3.744526in}}%
\pgfpathlineto{\pgfqpoint{2.599538in}{3.892500in}}%
\pgfpathlineto{\pgfqpoint{2.653746in}{3.745600in}}%
\pgfpathlineto{\pgfqpoint{2.709840in}{3.615588in}}%
\pgfpathlineto{\pgfqpoint{2.762944in}{3.486898in}}%
\pgfpathlineto{\pgfqpoint{2.816737in}{3.495593in}}%
\pgfpathlineto{\pgfqpoint{2.870654in}{3.513661in}}%
\pgfpathlineto{\pgfqpoint{2.924871in}{3.393751in}}%
\pgfpathlineto{\pgfqpoint{2.979336in}{3.417543in}}%
\pgfpathlineto{\pgfqpoint{3.033403in}{3.401548in}}%
\pgfpathlineto{\pgfqpoint{3.087801in}{3.279935in}}%
\pgfpathlineto{\pgfqpoint{3.141910in}{3.226682in}}%
\pgfpathlineto{\pgfqpoint{3.196257in}{3.210501in}}%
\pgfpathlineto{\pgfqpoint{3.250954in}{3.411160in}}%
\pgfpathlineto{\pgfqpoint{3.305517in}{3.367167in}}%
\pgfpathlineto{\pgfqpoint{3.361240in}{3.366219in}}%
\pgfpathlineto{\pgfqpoint{3.414602in}{3.386622in}}%
\pgfpathlineto{\pgfqpoint{3.468834in}{3.453509in}}%
\pgfpathlineto{\pgfqpoint{3.522565in}{3.474687in}}%
\pgfpathlineto{\pgfqpoint{3.576800in}{3.503640in}}%
\pgfpathlineto{\pgfqpoint{3.631129in}{3.480548in}}%
\pgfpathlineto{\pgfqpoint{3.685263in}{3.452206in}}%
\pgfpathlineto{\pgfqpoint{3.739292in}{3.477918in}}%
\pgfpathlineto{\pgfqpoint{3.793550in}{3.467814in}}%
\pgfpathlineto{\pgfqpoint{3.847941in}{3.460465in}}%
\pgfpathlineto{\pgfqpoint{3.902737in}{3.459988in}}%
\pgfpathlineto{\pgfqpoint{3.956780in}{3.443354in}}%
\pgfpathlineto{\pgfqpoint{4.012438in}{3.442499in}}%
\pgfpathlineto{\pgfqpoint{4.065856in}{3.434049in}}%
\pgfpathlineto{\pgfqpoint{4.119873in}{3.429341in}}%
\pgfpathlineto{\pgfqpoint{4.173597in}{3.420656in}}%
\pgfpathlineto{\pgfqpoint{4.227910in}{3.420591in}}%
\pgfpathlineto{\pgfqpoint{4.282610in}{3.417400in}}%
\pgfpathlineto{\pgfqpoint{4.336431in}{3.413875in}}%
\pgfpathlineto{\pgfqpoint{4.390236in}{3.413806in}}%
\pgfpathlineto{\pgfqpoint{4.444628in}{3.413741in}}%
\pgfpathlineto{\pgfqpoint{4.500735in}{3.413657in}}%
\pgfpathlineto{\pgfqpoint{4.554421in}{3.401110in}}%
\pgfpathlineto{\pgfqpoint{4.608131in}{3.388679in}}%
\pgfpathlineto{\pgfqpoint{4.662275in}{3.385882in}}%
\pgfpathlineto{\pgfqpoint{4.716284in}{3.389108in}}%
\pgfpathlineto{\pgfqpoint{4.770614in}{3.387761in}}%
\pgfpathlineto{\pgfqpoint{4.824533in}{3.389012in}}%
\pgfpathlineto{\pgfqpoint{4.879129in}{3.385315in}}%
\pgfpathlineto{\pgfqpoint{4.933628in}{3.409312in}}%
\pgfpathlineto{\pgfqpoint{4.988017in}{3.417989in}}%
\pgfpathlineto{\pgfqpoint{5.042027in}{3.435965in}}%
\pgfpathlineto{\pgfqpoint{5.096457in}{3.439992in}}%
\pgfpathlineto{\pgfqpoint{5.151536in}{3.465095in}}%
\pgfpathlineto{\pgfqpoint{5.205771in}{3.494647in}}%
\pgfpathlineto{\pgfqpoint{5.259761in}{3.507400in}}%
\pgfpathlineto{\pgfqpoint{5.314187in}{3.526889in}}%
\pgfpathlineto{\pgfqpoint{5.367959in}{3.525516in}}%
\pgfpathlineto{\pgfqpoint{5.422056in}{3.532553in}}%
\pgfpathlineto{\pgfqpoint{5.476438in}{3.551464in}}%
\pgfpathlineto{\pgfqpoint{5.530639in}{3.563154in}}%
\pgfusepath{stroke}%
\end{pgfscope}%
\begin{pgfscope}%
\pgfpathrectangle{\pgfqpoint{0.800000in}{0.528000in}}{\pgfqpoint{4.960000in}{3.696000in}}%
\pgfusepath{clip}%
\pgfsetrectcap%
\pgfsetroundjoin%
\pgfsetlinewidth{1.505625pt}%
\definecolor{currentstroke}{rgb}{0.839216,0.152941,0.156863}%
\pgfsetstrokecolor{currentstroke}%
\pgfsetdash{}{0pt}%
\pgfpathmoveto{\pgfqpoint{1.025455in}{0.880754in}}%
\pgfpathlineto{\pgfqpoint{1.079580in}{1.028188in}}%
\pgfpathlineto{\pgfqpoint{1.133762in}{0.932570in}}%
\pgfpathlineto{\pgfqpoint{1.187808in}{0.811463in}}%
\pgfpathlineto{\pgfqpoint{1.241911in}{1.208480in}}%
\pgfpathlineto{\pgfqpoint{1.297793in}{2.060357in}}%
\pgfpathlineto{\pgfqpoint{1.349164in}{2.608039in}}%
\pgfpathlineto{\pgfqpoint{1.403831in}{2.893291in}}%
\pgfpathlineto{\pgfqpoint{1.459885in}{3.153203in}}%
\pgfpathlineto{\pgfqpoint{1.513066in}{3.453023in}}%
\pgfpathlineto{\pgfqpoint{1.569073in}{3.620410in}}%
\pgfpathlineto{\pgfqpoint{1.620912in}{3.681661in}}%
\pgfpathlineto{\pgfqpoint{1.674991in}{3.658144in}}%
\pgfpathlineto{\pgfqpoint{1.729485in}{3.612795in}}%
\pgfpathlineto{\pgfqpoint{1.783597in}{3.487577in}}%
\pgfpathlineto{\pgfqpoint{1.838041in}{3.723528in}}%
\pgfpathlineto{\pgfqpoint{1.892167in}{3.659519in}}%
\pgfpathlineto{\pgfqpoint{1.946590in}{3.646657in}}%
\pgfpathlineto{\pgfqpoint{2.001140in}{3.270925in}}%
\pgfpathlineto{\pgfqpoint{2.060457in}{3.079750in}}%
\pgfpathlineto{\pgfqpoint{2.110573in}{3.220985in}}%
\pgfpathlineto{\pgfqpoint{2.163982in}{3.283281in}}%
\pgfpathlineto{\pgfqpoint{2.219073in}{3.453829in}}%
\pgfpathlineto{\pgfqpoint{2.273282in}{3.401710in}}%
\pgfpathlineto{\pgfqpoint{2.327258in}{3.503464in}}%
\pgfpathlineto{\pgfqpoint{2.381833in}{3.678676in}}%
\pgfpathlineto{\pgfqpoint{2.436028in}{3.595051in}}%
\pgfpathlineto{\pgfqpoint{2.490065in}{3.512267in}}%
\pgfpathlineto{\pgfqpoint{2.544453in}{3.604277in}}%
\pgfpathlineto{\pgfqpoint{2.598580in}{3.552116in}}%
\pgfpathlineto{\pgfqpoint{2.653580in}{3.459119in}}%
\pgfpathlineto{\pgfqpoint{2.709855in}{3.446947in}}%
\pgfpathlineto{\pgfqpoint{2.763088in}{3.363427in}}%
\pgfpathlineto{\pgfqpoint{2.817383in}{3.313333in}}%
\pgfpathlineto{\pgfqpoint{2.871388in}{3.250514in}}%
\pgfpathlineto{\pgfqpoint{2.926073in}{3.177328in}}%
\pgfpathlineto{\pgfqpoint{2.980262in}{3.176073in}}%
\pgfpathlineto{\pgfqpoint{3.034808in}{3.319401in}}%
\pgfpathlineto{\pgfqpoint{3.089346in}{3.368137in}}%
\pgfpathlineto{\pgfqpoint{3.142968in}{3.495927in}}%
\pgfpathlineto{\pgfqpoint{3.197588in}{3.419731in}}%
\pgfpathlineto{\pgfqpoint{3.251671in}{3.566280in}}%
\pgfpathlineto{\pgfqpoint{3.307440in}{3.549702in}}%
\pgfpathlineto{\pgfqpoint{3.361306in}{3.580338in}}%
\pgfpathlineto{\pgfqpoint{3.416297in}{3.559589in}}%
\pgfpathlineto{\pgfqpoint{3.470449in}{3.691321in}}%
\pgfpathlineto{\pgfqpoint{3.523914in}{3.628550in}}%
\pgfpathlineto{\pgfqpoint{3.578473in}{3.491376in}}%
\pgfpathlineto{\pgfqpoint{3.632583in}{3.238744in}}%
\pgfpathlineto{\pgfqpoint{3.687107in}{3.326340in}}%
\pgfpathlineto{\pgfqpoint{3.742604in}{3.379133in}}%
\pgfpathlineto{\pgfqpoint{3.796394in}{3.411585in}}%
\pgfpathlineto{\pgfqpoint{3.850516in}{3.358483in}}%
\pgfpathlineto{\pgfqpoint{3.906056in}{3.354198in}}%
\pgfpathlineto{\pgfqpoint{3.959288in}{3.358317in}}%
\pgfpathlineto{\pgfqpoint{4.013068in}{3.327018in}}%
\pgfpathlineto{\pgfqpoint{4.067296in}{3.411233in}}%
\pgfpathlineto{\pgfqpoint{4.121172in}{3.681557in}}%
\pgfpathlineto{\pgfqpoint{4.175689in}{3.516878in}}%
\pgfpathlineto{\pgfqpoint{4.229933in}{3.511166in}}%
\pgfpathlineto{\pgfqpoint{4.284986in}{3.390552in}}%
\pgfpathlineto{\pgfqpoint{4.338971in}{3.388609in}}%
\pgfpathlineto{\pgfqpoint{4.393209in}{3.457930in}}%
\pgfpathlineto{\pgfqpoint{4.447806in}{3.414049in}}%
\pgfpathlineto{\pgfqpoint{4.502549in}{3.413575in}}%
\pgfpathlineto{\pgfqpoint{4.557534in}{3.399323in}}%
\pgfpathlineto{\pgfqpoint{4.610982in}{3.406253in}}%
\pgfpathlineto{\pgfqpoint{4.665088in}{3.408249in}}%
\pgfpathlineto{\pgfqpoint{4.719343in}{3.409882in}}%
\pgfpathlineto{\pgfqpoint{4.773736in}{3.403085in}}%
\pgfpathlineto{\pgfqpoint{4.827702in}{3.393318in}}%
\pgfpathlineto{\pgfqpoint{4.882646in}{3.370924in}}%
\pgfpathlineto{\pgfqpoint{4.936301in}{3.378027in}}%
\pgfpathlineto{\pgfqpoint{4.990694in}{3.384825in}}%
\pgfpathlineto{\pgfqpoint{5.045010in}{3.721717in}}%
\pgfpathlineto{\pgfqpoint{5.099315in}{3.447166in}}%
\pgfpathlineto{\pgfqpoint{5.155259in}{3.418584in}}%
\pgfpathlineto{\pgfqpoint{5.208444in}{3.467262in}}%
\pgfpathlineto{\pgfqpoint{5.262143in}{3.687379in}}%
\pgfpathlineto{\pgfqpoint{5.316238in}{3.535095in}}%
\pgfpathlineto{\pgfqpoint{5.370459in}{3.360821in}}%
\pgfpathlineto{\pgfqpoint{5.425230in}{3.270913in}}%
\pgfpathlineto{\pgfqpoint{5.479369in}{3.432950in}}%
\pgfpathlineto{\pgfqpoint{5.534545in}{3.395727in}}%
\pgfusepath{stroke}%
\end{pgfscope}%
\begin{pgfscope}%
\pgfpathrectangle{\pgfqpoint{0.800000in}{0.528000in}}{\pgfqpoint{4.960000in}{3.696000in}}%
\pgfusepath{clip}%
\pgfsetrectcap%
\pgfsetroundjoin%
\pgfsetlinewidth{1.505625pt}%
\definecolor{currentstroke}{rgb}{0.580392,0.403922,0.741176}%
\pgfsetstrokecolor{currentstroke}%
\pgfsetdash{}{0pt}%
\pgfpathmoveto{\pgfqpoint{1.025455in}{0.874339in}}%
\pgfpathlineto{\pgfqpoint{1.079775in}{1.136728in}}%
\pgfpathlineto{\pgfqpoint{1.134512in}{1.052685in}}%
\pgfpathlineto{\pgfqpoint{1.188622in}{1.080127in}}%
\pgfpathlineto{\pgfqpoint{1.244257in}{1.299007in}}%
\pgfpathlineto{\pgfqpoint{1.297294in}{1.634485in}}%
\pgfpathlineto{\pgfqpoint{1.350518in}{2.494585in}}%
\pgfpathlineto{\pgfqpoint{1.404521in}{3.093232in}}%
\pgfpathlineto{\pgfqpoint{1.458607in}{3.393955in}}%
\pgfpathlineto{\pgfqpoint{1.513370in}{3.482193in}}%
\pgfpathlineto{\pgfqpoint{1.567196in}{3.658354in}}%
\pgfpathlineto{\pgfqpoint{1.621234in}{3.611856in}}%
\pgfpathlineto{\pgfqpoint{1.675622in}{3.667326in}}%
\pgfpathlineto{\pgfqpoint{1.730254in}{3.889738in}}%
\pgfpathlineto{\pgfqpoint{1.786788in}{3.662529in}}%
\pgfpathlineto{\pgfqpoint{1.840218in}{4.056000in}}%
\pgfpathlineto{\pgfqpoint{1.894067in}{3.587954in}}%
\pgfpathlineto{\pgfqpoint{1.948041in}{2.916721in}}%
\pgfpathlineto{\pgfqpoint{2.001840in}{3.193169in}}%
\pgfpathlineto{\pgfqpoint{2.056502in}{2.861256in}}%
\pgfpathlineto{\pgfqpoint{2.110587in}{2.588086in}}%
\pgfpathlineto{\pgfqpoint{2.164640in}{2.733793in}}%
\pgfpathlineto{\pgfqpoint{2.219082in}{3.606770in}}%
\pgfpathlineto{\pgfqpoint{2.272843in}{3.842894in}}%
\pgfpathlineto{\pgfqpoint{2.327753in}{3.507853in}}%
\pgfpathlineto{\pgfqpoint{2.381989in}{3.724801in}}%
\pgfpathlineto{\pgfqpoint{2.435675in}{3.660886in}}%
\pgfpathlineto{\pgfqpoint{2.489952in}{3.763143in}}%
\pgfpathlineto{\pgfqpoint{2.544170in}{3.752880in}}%
\pgfpathlineto{\pgfqpoint{2.599652in}{3.489444in}}%
\pgfpathlineto{\pgfqpoint{2.653382in}{3.260804in}}%
\pgfpathlineto{\pgfqpoint{2.708423in}{2.777330in}}%
\pgfpathlineto{\pgfqpoint{2.762914in}{2.632703in}}%
\pgfpathlineto{\pgfqpoint{2.817087in}{2.863988in}}%
\pgfpathlineto{\pgfqpoint{2.871525in}{3.059806in}}%
\pgfpathlineto{\pgfqpoint{2.925766in}{3.508542in}}%
\pgfpathlineto{\pgfqpoint{2.979861in}{3.501704in}}%
\pgfpathlineto{\pgfqpoint{3.034913in}{3.603294in}}%
\pgfpathlineto{\pgfqpoint{3.089363in}{3.540563in}}%
\pgfpathlineto{\pgfqpoint{3.144439in}{3.631905in}}%
\pgfpathlineto{\pgfqpoint{3.198438in}{3.658379in}}%
\pgfpathlineto{\pgfqpoint{3.252573in}{3.710528in}}%
\pgfpathlineto{\pgfqpoint{3.306794in}{3.769562in}}%
\pgfpathlineto{\pgfqpoint{3.361211in}{3.728668in}}%
\pgfpathlineto{\pgfqpoint{3.416834in}{3.394889in}}%
\pgfpathlineto{\pgfqpoint{3.470506in}{3.061397in}}%
\pgfpathlineto{\pgfqpoint{3.524067in}{2.791068in}}%
\pgfpathlineto{\pgfqpoint{3.578493in}{3.146374in}}%
\pgfpathlineto{\pgfqpoint{3.633218in}{3.149361in}}%
\pgfpathlineto{\pgfqpoint{3.687655in}{3.477240in}}%
\pgfpathlineto{\pgfqpoint{3.742105in}{3.583530in}}%
\pgfpathlineto{\pgfqpoint{3.796018in}{3.560652in}}%
\pgfpathlineto{\pgfqpoint{3.850763in}{3.531661in}}%
\pgfpathlineto{\pgfqpoint{3.905906in}{3.660292in}}%
\pgfpathlineto{\pgfqpoint{3.959752in}{3.582212in}}%
\pgfpathlineto{\pgfqpoint{4.014000in}{3.556430in}}%
\pgfpathlineto{\pgfqpoint{4.068363in}{3.394923in}}%
\pgfpathlineto{\pgfqpoint{4.122621in}{3.354803in}}%
\pgfpathlineto{\pgfqpoint{4.176880in}{3.413967in}}%
\pgfpathlineto{\pgfqpoint{4.230930in}{3.444668in}}%
\pgfpathlineto{\pgfqpoint{4.285264in}{3.419118in}}%
\pgfpathlineto{\pgfqpoint{4.339573in}{3.434911in}}%
\pgfpathlineto{\pgfqpoint{4.393537in}{3.408564in}}%
\pgfpathlineto{\pgfqpoint{4.447656in}{3.416512in}}%
\pgfpathlineto{\pgfqpoint{4.502359in}{3.417436in}}%
\pgfpathlineto{\pgfqpoint{4.556439in}{3.436081in}}%
\pgfpathlineto{\pgfqpoint{4.611062in}{3.418314in}}%
\pgfpathlineto{\pgfqpoint{4.665374in}{3.417371in}}%
\pgfpathlineto{\pgfqpoint{4.719555in}{3.444276in}}%
\pgfpathlineto{\pgfqpoint{4.774083in}{3.439645in}}%
\pgfpathlineto{\pgfqpoint{4.828390in}{3.444517in}}%
\pgfpathlineto{\pgfqpoint{4.882381in}{3.444754in}}%
\pgfpathlineto{\pgfqpoint{4.936806in}{3.455405in}}%
\pgfpathlineto{\pgfqpoint{4.990444in}{3.481399in}}%
\pgfpathlineto{\pgfqpoint{5.044526in}{3.810289in}}%
\pgfpathlineto{\pgfqpoint{5.098797in}{3.466967in}}%
\pgfpathlineto{\pgfqpoint{5.152893in}{3.249578in}}%
\pgfpathlineto{\pgfqpoint{5.207293in}{3.360554in}}%
\pgfpathlineto{\pgfqpoint{5.261630in}{3.429521in}}%
\pgfpathlineto{\pgfqpoint{5.316071in}{3.327375in}}%
\pgfpathlineto{\pgfqpoint{5.370174in}{3.395555in}}%
\pgfpathlineto{\pgfqpoint{5.424466in}{3.414084in}}%
\pgfpathlineto{\pgfqpoint{5.478616in}{3.407032in}}%
\pgfpathlineto{\pgfqpoint{5.532690in}{3.731776in}}%
\pgfusepath{stroke}%
\end{pgfscope}%
\begin{pgfscope}%
\pgfsetrectcap%
\pgfsetmiterjoin%
\pgfsetlinewidth{0.803000pt}%
\definecolor{currentstroke}{rgb}{0.000000,0.000000,0.000000}%
\pgfsetstrokecolor{currentstroke}%
\pgfsetdash{}{0pt}%
\pgfpathmoveto{\pgfqpoint{0.800000in}{0.528000in}}%
\pgfpathlineto{\pgfqpoint{0.800000in}{4.224000in}}%
\pgfusepath{stroke}%
\end{pgfscope}%
\begin{pgfscope}%
\pgfsetrectcap%
\pgfsetmiterjoin%
\pgfsetlinewidth{0.803000pt}%
\definecolor{currentstroke}{rgb}{0.000000,0.000000,0.000000}%
\pgfsetstrokecolor{currentstroke}%
\pgfsetdash{}{0pt}%
\pgfpathmoveto{\pgfqpoint{5.760000in}{0.528000in}}%
\pgfpathlineto{\pgfqpoint{5.760000in}{4.224000in}}%
\pgfusepath{stroke}%
\end{pgfscope}%
\begin{pgfscope}%
\pgfsetrectcap%
\pgfsetmiterjoin%
\pgfsetlinewidth{0.803000pt}%
\definecolor{currentstroke}{rgb}{0.000000,0.000000,0.000000}%
\pgfsetstrokecolor{currentstroke}%
\pgfsetdash{}{0pt}%
\pgfpathmoveto{\pgfqpoint{0.800000in}{0.528000in}}%
\pgfpathlineto{\pgfqpoint{5.760000in}{0.528000in}}%
\pgfusepath{stroke}%
\end{pgfscope}%
\begin{pgfscope}%
\pgfsetrectcap%
\pgfsetmiterjoin%
\pgfsetlinewidth{0.803000pt}%
\definecolor{currentstroke}{rgb}{0.000000,0.000000,0.000000}%
\pgfsetstrokecolor{currentstroke}%
\pgfsetdash{}{0pt}%
\pgfpathmoveto{\pgfqpoint{0.800000in}{4.224000in}}%
\pgfpathlineto{\pgfqpoint{5.760000in}{4.224000in}}%
\pgfusepath{stroke}%
\end{pgfscope}%
\begin{pgfscope}%
\definecolor{textcolor}{rgb}{0.000000,0.000000,0.000000}%
\pgfsetstrokecolor{textcolor}%
\pgfsetfillcolor{textcolor}%
\pgftext[x=3.280000in,y=4.307333in,,base]{\color{textcolor}\sffamily\fontsize{12.000000}{14.400000}\selectfont Forward controller input}%
\end{pgfscope}%
\begin{pgfscope}%
\pgfsetbuttcap%
\pgfsetmiterjoin%
\definecolor{currentfill}{rgb}{1.000000,1.000000,1.000000}%
\pgfsetfillcolor{currentfill}%
\pgfsetfillopacity{0.800000}%
\pgfsetlinewidth{1.003750pt}%
\definecolor{currentstroke}{rgb}{0.800000,0.800000,0.800000}%
\pgfsetstrokecolor{currentstroke}%
\pgfsetstrokeopacity{0.800000}%
\pgfsetdash{}{0pt}%
\pgfpathmoveto{\pgfqpoint{5.041603in}{0.597444in}}%
\pgfpathlineto{\pgfqpoint{5.662778in}{0.597444in}}%
\pgfpathquadraticcurveto{\pgfqpoint{5.690556in}{0.597444in}}{\pgfqpoint{5.690556in}{0.625222in}}%
\pgfpathlineto{\pgfqpoint{5.690556in}{1.630619in}}%
\pgfpathquadraticcurveto{\pgfqpoint{5.690556in}{1.658397in}}{\pgfqpoint{5.662778in}{1.658397in}}%
\pgfpathlineto{\pgfqpoint{5.041603in}{1.658397in}}%
\pgfpathquadraticcurveto{\pgfqpoint{5.013825in}{1.658397in}}{\pgfqpoint{5.013825in}{1.630619in}}%
\pgfpathlineto{\pgfqpoint{5.013825in}{0.625222in}}%
\pgfpathquadraticcurveto{\pgfqpoint{5.013825in}{0.597444in}}{\pgfqpoint{5.041603in}{0.597444in}}%
\pgfpathlineto{\pgfqpoint{5.041603in}{0.597444in}}%
\pgfpathclose%
\pgfusepath{stroke,fill}%
\end{pgfscope}%
\begin{pgfscope}%
\pgfsetrectcap%
\pgfsetroundjoin%
\pgfsetlinewidth{1.505625pt}%
\definecolor{currentstroke}{rgb}{0.121569,0.466667,0.705882}%
\pgfsetstrokecolor{currentstroke}%
\pgfsetdash{}{0pt}%
\pgfpathmoveto{\pgfqpoint{5.069380in}{1.545930in}}%
\pgfpathlineto{\pgfqpoint{5.208269in}{1.545930in}}%
\pgfpathlineto{\pgfqpoint{5.347158in}{1.545930in}}%
\pgfusepath{stroke}%
\end{pgfscope}%
\begin{pgfscope}%
\definecolor{textcolor}{rgb}{0.000000,0.000000,0.000000}%
\pgfsetstrokecolor{textcolor}%
\pgfsetfillcolor{textcolor}%
\pgftext[x=5.458269in,y=1.497319in,left,base]{\color{textcolor}\sffamily\fontsize{10.000000}{12.000000}\selectfont 2}%
\end{pgfscope}%
\begin{pgfscope}%
\pgfsetrectcap%
\pgfsetroundjoin%
\pgfsetlinewidth{1.505625pt}%
\definecolor{currentstroke}{rgb}{1.000000,0.498039,0.054902}%
\pgfsetstrokecolor{currentstroke}%
\pgfsetdash{}{0pt}%
\pgfpathmoveto{\pgfqpoint{5.069380in}{1.342073in}}%
\pgfpathlineto{\pgfqpoint{5.208269in}{1.342073in}}%
\pgfpathlineto{\pgfqpoint{5.347158in}{1.342073in}}%
\pgfusepath{stroke}%
\end{pgfscope}%
\begin{pgfscope}%
\definecolor{textcolor}{rgb}{0.000000,0.000000,0.000000}%
\pgfsetstrokecolor{textcolor}%
\pgfsetfillcolor{textcolor}%
\pgftext[x=5.458269in,y=1.293461in,left,base]{\color{textcolor}\sffamily\fontsize{10.000000}{12.000000}\selectfont 4}%
\end{pgfscope}%
\begin{pgfscope}%
\pgfsetrectcap%
\pgfsetroundjoin%
\pgfsetlinewidth{1.505625pt}%
\definecolor{currentstroke}{rgb}{0.172549,0.627451,0.172549}%
\pgfsetstrokecolor{currentstroke}%
\pgfsetdash{}{0pt}%
\pgfpathmoveto{\pgfqpoint{5.069380in}{1.138215in}}%
\pgfpathlineto{\pgfqpoint{5.208269in}{1.138215in}}%
\pgfpathlineto{\pgfqpoint{5.347158in}{1.138215in}}%
\pgfusepath{stroke}%
\end{pgfscope}%
\begin{pgfscope}%
\definecolor{textcolor}{rgb}{0.000000,0.000000,0.000000}%
\pgfsetstrokecolor{textcolor}%
\pgfsetfillcolor{textcolor}%
\pgftext[x=5.458269in,y=1.089604in,left,base]{\color{textcolor}\sffamily\fontsize{10.000000}{12.000000}\selectfont 6}%
\end{pgfscope}%
\begin{pgfscope}%
\pgfsetrectcap%
\pgfsetroundjoin%
\pgfsetlinewidth{1.505625pt}%
\definecolor{currentstroke}{rgb}{0.839216,0.152941,0.156863}%
\pgfsetstrokecolor{currentstroke}%
\pgfsetdash{}{0pt}%
\pgfpathmoveto{\pgfqpoint{5.069380in}{0.934358in}}%
\pgfpathlineto{\pgfqpoint{5.208269in}{0.934358in}}%
\pgfpathlineto{\pgfqpoint{5.347158in}{0.934358in}}%
\pgfusepath{stroke}%
\end{pgfscope}%
\begin{pgfscope}%
\definecolor{textcolor}{rgb}{0.000000,0.000000,0.000000}%
\pgfsetstrokecolor{textcolor}%
\pgfsetfillcolor{textcolor}%
\pgftext[x=5.458269in,y=0.885747in,left,base]{\color{textcolor}\sffamily\fontsize{10.000000}{12.000000}\selectfont 8}%
\end{pgfscope}%
\begin{pgfscope}%
\pgfsetrectcap%
\pgfsetroundjoin%
\pgfsetlinewidth{1.505625pt}%
\definecolor{currentstroke}{rgb}{0.580392,0.403922,0.741176}%
\pgfsetstrokecolor{currentstroke}%
\pgfsetdash{}{0pt}%
\pgfpathmoveto{\pgfqpoint{5.069380in}{0.730501in}}%
\pgfpathlineto{\pgfqpoint{5.208269in}{0.730501in}}%
\pgfpathlineto{\pgfqpoint{5.347158in}{0.730501in}}%
\pgfusepath{stroke}%
\end{pgfscope}%
\begin{pgfscope}%
\definecolor{textcolor}{rgb}{0.000000,0.000000,0.000000}%
\pgfsetstrokecolor{textcolor}%
\pgfsetfillcolor{textcolor}%
\pgftext[x=5.458269in,y=0.681890in,left,base]{\color{textcolor}\sffamily\fontsize{10.000000}{12.000000}\selectfont 10}%
\end{pgfscope}%
\end{pgfpicture}%
\makeatother%
\endgroup%
}
    \end{minipage}
    \begin{minipage}[t]{0.5\linewidth}
        \centering
        \scalebox{0.55}{%% Creator: Matplotlib, PGF backend
%%
%% To include the figure in your LaTeX document, write
%%   \input{<filename>.pgf}
%%
%% Make sure the required packages are loaded in your preamble
%%   \usepackage{pgf}
%%
%% Also ensure that all the required font packages are loaded; for instance,
%% the lmodern package is sometimes necessary when using math font.
%%   \usepackage{lmodern}
%%
%% Figures using additional raster images can only be included by \input if
%% they are in the same directory as the main LaTeX file. For loading figures
%% from other directories you can use the `import` package
%%   \usepackage{import}
%%
%% and then include the figures with
%%   \import{<path to file>}{<filename>.pgf}
%%
%% Matplotlib used the following preamble
%%   \usepackage{fontspec}
%%   \setmainfont{DejaVuSerif.ttf}[Path=\detokenize{/home/lgonz/tfg-aero/tfg-giaa-dronecontrol/venv/lib/python3.8/site-packages/matplotlib/mpl-data/fonts/ttf/}]
%%   \setsansfont{DejaVuSans.ttf}[Path=\detokenize{/home/lgonz/tfg-aero/tfg-giaa-dronecontrol/venv/lib/python3.8/site-packages/matplotlib/mpl-data/fonts/ttf/}]
%%   \setmonofont{DejaVuSansMono.ttf}[Path=\detokenize{/home/lgonz/tfg-aero/tfg-giaa-dronecontrol/venv/lib/python3.8/site-packages/matplotlib/mpl-data/fonts/ttf/}]
%%
\begingroup%
\makeatletter%
\begin{pgfpicture}%
\pgfpathrectangle{\pgfpointorigin}{\pgfqpoint{6.400000in}{4.800000in}}%
\pgfusepath{use as bounding box, clip}%
\begin{pgfscope}%
\pgfsetbuttcap%
\pgfsetmiterjoin%
\definecolor{currentfill}{rgb}{1.000000,1.000000,1.000000}%
\pgfsetfillcolor{currentfill}%
\pgfsetlinewidth{0.000000pt}%
\definecolor{currentstroke}{rgb}{1.000000,1.000000,1.000000}%
\pgfsetstrokecolor{currentstroke}%
\pgfsetdash{}{0pt}%
\pgfpathmoveto{\pgfqpoint{0.000000in}{0.000000in}}%
\pgfpathlineto{\pgfqpoint{6.400000in}{0.000000in}}%
\pgfpathlineto{\pgfqpoint{6.400000in}{4.800000in}}%
\pgfpathlineto{\pgfqpoint{0.000000in}{4.800000in}}%
\pgfpathlineto{\pgfqpoint{0.000000in}{0.000000in}}%
\pgfpathclose%
\pgfusepath{fill}%
\end{pgfscope}%
\begin{pgfscope}%
\pgfsetbuttcap%
\pgfsetmiterjoin%
\definecolor{currentfill}{rgb}{1.000000,1.000000,1.000000}%
\pgfsetfillcolor{currentfill}%
\pgfsetlinewidth{0.000000pt}%
\definecolor{currentstroke}{rgb}{0.000000,0.000000,0.000000}%
\pgfsetstrokecolor{currentstroke}%
\pgfsetstrokeopacity{0.000000}%
\pgfsetdash{}{0pt}%
\pgfpathmoveto{\pgfqpoint{0.800000in}{0.528000in}}%
\pgfpathlineto{\pgfqpoint{5.760000in}{0.528000in}}%
\pgfpathlineto{\pgfqpoint{5.760000in}{4.224000in}}%
\pgfpathlineto{\pgfqpoint{0.800000in}{4.224000in}}%
\pgfpathlineto{\pgfqpoint{0.800000in}{0.528000in}}%
\pgfpathclose%
\pgfusepath{fill}%
\end{pgfscope}%
\begin{pgfscope}%
\pgfpathrectangle{\pgfqpoint{0.800000in}{0.528000in}}{\pgfqpoint{4.960000in}{3.696000in}}%
\pgfusepath{clip}%
\pgfsetrectcap%
\pgfsetroundjoin%
\pgfsetlinewidth{0.803000pt}%
\definecolor{currentstroke}{rgb}{0.690196,0.690196,0.690196}%
\pgfsetstrokecolor{currentstroke}%
\pgfsetdash{}{0pt}%
\pgfpathmoveto{\pgfqpoint{1.025455in}{0.528000in}}%
\pgfpathlineto{\pgfqpoint{1.025455in}{4.224000in}}%
\pgfusepath{stroke}%
\end{pgfscope}%
\begin{pgfscope}%
\pgfsetbuttcap%
\pgfsetroundjoin%
\definecolor{currentfill}{rgb}{0.000000,0.000000,0.000000}%
\pgfsetfillcolor{currentfill}%
\pgfsetlinewidth{0.803000pt}%
\definecolor{currentstroke}{rgb}{0.000000,0.000000,0.000000}%
\pgfsetstrokecolor{currentstroke}%
\pgfsetdash{}{0pt}%
\pgfsys@defobject{currentmarker}{\pgfqpoint{0.000000in}{-0.048611in}}{\pgfqpoint{0.000000in}{0.000000in}}{%
\pgfpathmoveto{\pgfqpoint{0.000000in}{0.000000in}}%
\pgfpathlineto{\pgfqpoint{0.000000in}{-0.048611in}}%
\pgfusepath{stroke,fill}%
}%
\begin{pgfscope}%
\pgfsys@transformshift{1.025455in}{0.528000in}%
\pgfsys@useobject{currentmarker}{}%
\end{pgfscope}%
\end{pgfscope}%
\begin{pgfscope}%
\definecolor{textcolor}{rgb}{0.000000,0.000000,0.000000}%
\pgfsetstrokecolor{textcolor}%
\pgfsetfillcolor{textcolor}%
\pgftext[x=1.025455in,y=0.430778in,,top]{\color{textcolor}\sffamily\fontsize{10.000000}{12.000000}\selectfont 0}%
\end{pgfscope}%
\begin{pgfscope}%
\pgfpathrectangle{\pgfqpoint{0.800000in}{0.528000in}}{\pgfqpoint{4.960000in}{3.696000in}}%
\pgfusepath{clip}%
\pgfsetrectcap%
\pgfsetroundjoin%
\pgfsetlinewidth{0.803000pt}%
\definecolor{currentstroke}{rgb}{0.690196,0.690196,0.690196}%
\pgfsetstrokecolor{currentstroke}%
\pgfsetdash{}{0pt}%
\pgfpathmoveto{\pgfqpoint{1.775829in}{0.528000in}}%
\pgfpathlineto{\pgfqpoint{1.775829in}{4.224000in}}%
\pgfusepath{stroke}%
\end{pgfscope}%
\begin{pgfscope}%
\pgfsetbuttcap%
\pgfsetroundjoin%
\definecolor{currentfill}{rgb}{0.000000,0.000000,0.000000}%
\pgfsetfillcolor{currentfill}%
\pgfsetlinewidth{0.803000pt}%
\definecolor{currentstroke}{rgb}{0.000000,0.000000,0.000000}%
\pgfsetstrokecolor{currentstroke}%
\pgfsetdash{}{0pt}%
\pgfsys@defobject{currentmarker}{\pgfqpoint{0.000000in}{-0.048611in}}{\pgfqpoint{0.000000in}{0.000000in}}{%
\pgfpathmoveto{\pgfqpoint{0.000000in}{0.000000in}}%
\pgfpathlineto{\pgfqpoint{0.000000in}{-0.048611in}}%
\pgfusepath{stroke,fill}%
}%
\begin{pgfscope}%
\pgfsys@transformshift{1.775829in}{0.528000in}%
\pgfsys@useobject{currentmarker}{}%
\end{pgfscope}%
\end{pgfscope}%
\begin{pgfscope}%
\definecolor{textcolor}{rgb}{0.000000,0.000000,0.000000}%
\pgfsetstrokecolor{textcolor}%
\pgfsetfillcolor{textcolor}%
\pgftext[x=1.775829in,y=0.430778in,,top]{\color{textcolor}\sffamily\fontsize{10.000000}{12.000000}\selectfont 5}%
\end{pgfscope}%
\begin{pgfscope}%
\pgfpathrectangle{\pgfqpoint{0.800000in}{0.528000in}}{\pgfqpoint{4.960000in}{3.696000in}}%
\pgfusepath{clip}%
\pgfsetrectcap%
\pgfsetroundjoin%
\pgfsetlinewidth{0.803000pt}%
\definecolor{currentstroke}{rgb}{0.690196,0.690196,0.690196}%
\pgfsetstrokecolor{currentstroke}%
\pgfsetdash{}{0pt}%
\pgfpathmoveto{\pgfqpoint{2.526202in}{0.528000in}}%
\pgfpathlineto{\pgfqpoint{2.526202in}{4.224000in}}%
\pgfusepath{stroke}%
\end{pgfscope}%
\begin{pgfscope}%
\pgfsetbuttcap%
\pgfsetroundjoin%
\definecolor{currentfill}{rgb}{0.000000,0.000000,0.000000}%
\pgfsetfillcolor{currentfill}%
\pgfsetlinewidth{0.803000pt}%
\definecolor{currentstroke}{rgb}{0.000000,0.000000,0.000000}%
\pgfsetstrokecolor{currentstroke}%
\pgfsetdash{}{0pt}%
\pgfsys@defobject{currentmarker}{\pgfqpoint{0.000000in}{-0.048611in}}{\pgfqpoint{0.000000in}{0.000000in}}{%
\pgfpathmoveto{\pgfqpoint{0.000000in}{0.000000in}}%
\pgfpathlineto{\pgfqpoint{0.000000in}{-0.048611in}}%
\pgfusepath{stroke,fill}%
}%
\begin{pgfscope}%
\pgfsys@transformshift{2.526202in}{0.528000in}%
\pgfsys@useobject{currentmarker}{}%
\end{pgfscope}%
\end{pgfscope}%
\begin{pgfscope}%
\definecolor{textcolor}{rgb}{0.000000,0.000000,0.000000}%
\pgfsetstrokecolor{textcolor}%
\pgfsetfillcolor{textcolor}%
\pgftext[x=2.526202in,y=0.430778in,,top]{\color{textcolor}\sffamily\fontsize{10.000000}{12.000000}\selectfont 10}%
\end{pgfscope}%
\begin{pgfscope}%
\pgfpathrectangle{\pgfqpoint{0.800000in}{0.528000in}}{\pgfqpoint{4.960000in}{3.696000in}}%
\pgfusepath{clip}%
\pgfsetrectcap%
\pgfsetroundjoin%
\pgfsetlinewidth{0.803000pt}%
\definecolor{currentstroke}{rgb}{0.690196,0.690196,0.690196}%
\pgfsetstrokecolor{currentstroke}%
\pgfsetdash{}{0pt}%
\pgfpathmoveto{\pgfqpoint{3.276576in}{0.528000in}}%
\pgfpathlineto{\pgfqpoint{3.276576in}{4.224000in}}%
\pgfusepath{stroke}%
\end{pgfscope}%
\begin{pgfscope}%
\pgfsetbuttcap%
\pgfsetroundjoin%
\definecolor{currentfill}{rgb}{0.000000,0.000000,0.000000}%
\pgfsetfillcolor{currentfill}%
\pgfsetlinewidth{0.803000pt}%
\definecolor{currentstroke}{rgb}{0.000000,0.000000,0.000000}%
\pgfsetstrokecolor{currentstroke}%
\pgfsetdash{}{0pt}%
\pgfsys@defobject{currentmarker}{\pgfqpoint{0.000000in}{-0.048611in}}{\pgfqpoint{0.000000in}{0.000000in}}{%
\pgfpathmoveto{\pgfqpoint{0.000000in}{0.000000in}}%
\pgfpathlineto{\pgfqpoint{0.000000in}{-0.048611in}}%
\pgfusepath{stroke,fill}%
}%
\begin{pgfscope}%
\pgfsys@transformshift{3.276576in}{0.528000in}%
\pgfsys@useobject{currentmarker}{}%
\end{pgfscope}%
\end{pgfscope}%
\begin{pgfscope}%
\definecolor{textcolor}{rgb}{0.000000,0.000000,0.000000}%
\pgfsetstrokecolor{textcolor}%
\pgfsetfillcolor{textcolor}%
\pgftext[x=3.276576in,y=0.430778in,,top]{\color{textcolor}\sffamily\fontsize{10.000000}{12.000000}\selectfont 15}%
\end{pgfscope}%
\begin{pgfscope}%
\pgfpathrectangle{\pgfqpoint{0.800000in}{0.528000in}}{\pgfqpoint{4.960000in}{3.696000in}}%
\pgfusepath{clip}%
\pgfsetrectcap%
\pgfsetroundjoin%
\pgfsetlinewidth{0.803000pt}%
\definecolor{currentstroke}{rgb}{0.690196,0.690196,0.690196}%
\pgfsetstrokecolor{currentstroke}%
\pgfsetdash{}{0pt}%
\pgfpathmoveto{\pgfqpoint{4.026950in}{0.528000in}}%
\pgfpathlineto{\pgfqpoint{4.026950in}{4.224000in}}%
\pgfusepath{stroke}%
\end{pgfscope}%
\begin{pgfscope}%
\pgfsetbuttcap%
\pgfsetroundjoin%
\definecolor{currentfill}{rgb}{0.000000,0.000000,0.000000}%
\pgfsetfillcolor{currentfill}%
\pgfsetlinewidth{0.803000pt}%
\definecolor{currentstroke}{rgb}{0.000000,0.000000,0.000000}%
\pgfsetstrokecolor{currentstroke}%
\pgfsetdash{}{0pt}%
\pgfsys@defobject{currentmarker}{\pgfqpoint{0.000000in}{-0.048611in}}{\pgfqpoint{0.000000in}{0.000000in}}{%
\pgfpathmoveto{\pgfqpoint{0.000000in}{0.000000in}}%
\pgfpathlineto{\pgfqpoint{0.000000in}{-0.048611in}}%
\pgfusepath{stroke,fill}%
}%
\begin{pgfscope}%
\pgfsys@transformshift{4.026950in}{0.528000in}%
\pgfsys@useobject{currentmarker}{}%
\end{pgfscope}%
\end{pgfscope}%
\begin{pgfscope}%
\definecolor{textcolor}{rgb}{0.000000,0.000000,0.000000}%
\pgfsetstrokecolor{textcolor}%
\pgfsetfillcolor{textcolor}%
\pgftext[x=4.026950in,y=0.430778in,,top]{\color{textcolor}\sffamily\fontsize{10.000000}{12.000000}\selectfont 20}%
\end{pgfscope}%
\begin{pgfscope}%
\pgfpathrectangle{\pgfqpoint{0.800000in}{0.528000in}}{\pgfqpoint{4.960000in}{3.696000in}}%
\pgfusepath{clip}%
\pgfsetrectcap%
\pgfsetroundjoin%
\pgfsetlinewidth{0.803000pt}%
\definecolor{currentstroke}{rgb}{0.690196,0.690196,0.690196}%
\pgfsetstrokecolor{currentstroke}%
\pgfsetdash{}{0pt}%
\pgfpathmoveto{\pgfqpoint{4.777324in}{0.528000in}}%
\pgfpathlineto{\pgfqpoint{4.777324in}{4.224000in}}%
\pgfusepath{stroke}%
\end{pgfscope}%
\begin{pgfscope}%
\pgfsetbuttcap%
\pgfsetroundjoin%
\definecolor{currentfill}{rgb}{0.000000,0.000000,0.000000}%
\pgfsetfillcolor{currentfill}%
\pgfsetlinewidth{0.803000pt}%
\definecolor{currentstroke}{rgb}{0.000000,0.000000,0.000000}%
\pgfsetstrokecolor{currentstroke}%
\pgfsetdash{}{0pt}%
\pgfsys@defobject{currentmarker}{\pgfqpoint{0.000000in}{-0.048611in}}{\pgfqpoint{0.000000in}{0.000000in}}{%
\pgfpathmoveto{\pgfqpoint{0.000000in}{0.000000in}}%
\pgfpathlineto{\pgfqpoint{0.000000in}{-0.048611in}}%
\pgfusepath{stroke,fill}%
}%
\begin{pgfscope}%
\pgfsys@transformshift{4.777324in}{0.528000in}%
\pgfsys@useobject{currentmarker}{}%
\end{pgfscope}%
\end{pgfscope}%
\begin{pgfscope}%
\definecolor{textcolor}{rgb}{0.000000,0.000000,0.000000}%
\pgfsetstrokecolor{textcolor}%
\pgfsetfillcolor{textcolor}%
\pgftext[x=4.777324in,y=0.430778in,,top]{\color{textcolor}\sffamily\fontsize{10.000000}{12.000000}\selectfont 25}%
\end{pgfscope}%
\begin{pgfscope}%
\pgfpathrectangle{\pgfqpoint{0.800000in}{0.528000in}}{\pgfqpoint{4.960000in}{3.696000in}}%
\pgfusepath{clip}%
\pgfsetrectcap%
\pgfsetroundjoin%
\pgfsetlinewidth{0.803000pt}%
\definecolor{currentstroke}{rgb}{0.690196,0.690196,0.690196}%
\pgfsetstrokecolor{currentstroke}%
\pgfsetdash{}{0pt}%
\pgfpathmoveto{\pgfqpoint{5.527698in}{0.528000in}}%
\pgfpathlineto{\pgfqpoint{5.527698in}{4.224000in}}%
\pgfusepath{stroke}%
\end{pgfscope}%
\begin{pgfscope}%
\pgfsetbuttcap%
\pgfsetroundjoin%
\definecolor{currentfill}{rgb}{0.000000,0.000000,0.000000}%
\pgfsetfillcolor{currentfill}%
\pgfsetlinewidth{0.803000pt}%
\definecolor{currentstroke}{rgb}{0.000000,0.000000,0.000000}%
\pgfsetstrokecolor{currentstroke}%
\pgfsetdash{}{0pt}%
\pgfsys@defobject{currentmarker}{\pgfqpoint{0.000000in}{-0.048611in}}{\pgfqpoint{0.000000in}{0.000000in}}{%
\pgfpathmoveto{\pgfqpoint{0.000000in}{0.000000in}}%
\pgfpathlineto{\pgfqpoint{0.000000in}{-0.048611in}}%
\pgfusepath{stroke,fill}%
}%
\begin{pgfscope}%
\pgfsys@transformshift{5.527698in}{0.528000in}%
\pgfsys@useobject{currentmarker}{}%
\end{pgfscope}%
\end{pgfscope}%
\begin{pgfscope}%
\definecolor{textcolor}{rgb}{0.000000,0.000000,0.000000}%
\pgfsetstrokecolor{textcolor}%
\pgfsetfillcolor{textcolor}%
\pgftext[x=5.527698in,y=0.430778in,,top]{\color{textcolor}\sffamily\fontsize{10.000000}{12.000000}\selectfont 30}%
\end{pgfscope}%
\begin{pgfscope}%
\definecolor{textcolor}{rgb}{0.000000,0.000000,0.000000}%
\pgfsetstrokecolor{textcolor}%
\pgfsetfillcolor{textcolor}%
\pgftext[x=3.280000in,y=0.240809in,,top]{\color{textcolor}\sffamily\fontsize{10.000000}{12.000000}\selectfont time [s]}%
\end{pgfscope}%
\begin{pgfscope}%
\pgfpathrectangle{\pgfqpoint{0.800000in}{0.528000in}}{\pgfqpoint{4.960000in}{3.696000in}}%
\pgfusepath{clip}%
\pgfsetrectcap%
\pgfsetroundjoin%
\pgfsetlinewidth{0.803000pt}%
\definecolor{currentstroke}{rgb}{0.690196,0.690196,0.690196}%
\pgfsetstrokecolor{currentstroke}%
\pgfsetdash{}{0pt}%
\pgfpathmoveto{\pgfqpoint{0.800000in}{0.696000in}}%
\pgfpathlineto{\pgfqpoint{5.760000in}{0.696000in}}%
\pgfusepath{stroke}%
\end{pgfscope}%
\begin{pgfscope}%
\pgfsetbuttcap%
\pgfsetroundjoin%
\definecolor{currentfill}{rgb}{0.000000,0.000000,0.000000}%
\pgfsetfillcolor{currentfill}%
\pgfsetlinewidth{0.803000pt}%
\definecolor{currentstroke}{rgb}{0.000000,0.000000,0.000000}%
\pgfsetstrokecolor{currentstroke}%
\pgfsetdash{}{0pt}%
\pgfsys@defobject{currentmarker}{\pgfqpoint{-0.048611in}{0.000000in}}{\pgfqpoint{-0.000000in}{0.000000in}}{%
\pgfpathmoveto{\pgfqpoint{-0.000000in}{0.000000in}}%
\pgfpathlineto{\pgfqpoint{-0.048611in}{0.000000in}}%
\pgfusepath{stroke,fill}%
}%
\begin{pgfscope}%
\pgfsys@transformshift{0.800000in}{0.696000in}%
\pgfsys@useobject{currentmarker}{}%
\end{pgfscope}%
\end{pgfscope}%
\begin{pgfscope}%
\definecolor{textcolor}{rgb}{0.000000,0.000000,0.000000}%
\pgfsetstrokecolor{textcolor}%
\pgfsetfillcolor{textcolor}%
\pgftext[x=0.373873in, y=0.643238in, left, base]{\color{textcolor}\sffamily\fontsize{10.000000}{12.000000}\selectfont \ensuremath{-}0.4}%
\end{pgfscope}%
\begin{pgfscope}%
\pgfpathrectangle{\pgfqpoint{0.800000in}{0.528000in}}{\pgfqpoint{4.960000in}{3.696000in}}%
\pgfusepath{clip}%
\pgfsetrectcap%
\pgfsetroundjoin%
\pgfsetlinewidth{0.803000pt}%
\definecolor{currentstroke}{rgb}{0.690196,0.690196,0.690196}%
\pgfsetstrokecolor{currentstroke}%
\pgfsetdash{}{0pt}%
\pgfpathmoveto{\pgfqpoint{0.800000in}{1.116000in}}%
\pgfpathlineto{\pgfqpoint{5.760000in}{1.116000in}}%
\pgfusepath{stroke}%
\end{pgfscope}%
\begin{pgfscope}%
\pgfsetbuttcap%
\pgfsetroundjoin%
\definecolor{currentfill}{rgb}{0.000000,0.000000,0.000000}%
\pgfsetfillcolor{currentfill}%
\pgfsetlinewidth{0.803000pt}%
\definecolor{currentstroke}{rgb}{0.000000,0.000000,0.000000}%
\pgfsetstrokecolor{currentstroke}%
\pgfsetdash{}{0pt}%
\pgfsys@defobject{currentmarker}{\pgfqpoint{-0.048611in}{0.000000in}}{\pgfqpoint{-0.000000in}{0.000000in}}{%
\pgfpathmoveto{\pgfqpoint{-0.000000in}{0.000000in}}%
\pgfpathlineto{\pgfqpoint{-0.048611in}{0.000000in}}%
\pgfusepath{stroke,fill}%
}%
\begin{pgfscope}%
\pgfsys@transformshift{0.800000in}{1.116000in}%
\pgfsys@useobject{currentmarker}{}%
\end{pgfscope}%
\end{pgfscope}%
\begin{pgfscope}%
\definecolor{textcolor}{rgb}{0.000000,0.000000,0.000000}%
\pgfsetstrokecolor{textcolor}%
\pgfsetfillcolor{textcolor}%
\pgftext[x=0.373873in, y=1.063238in, left, base]{\color{textcolor}\sffamily\fontsize{10.000000}{12.000000}\selectfont \ensuremath{-}0.3}%
\end{pgfscope}%
\begin{pgfscope}%
\pgfpathrectangle{\pgfqpoint{0.800000in}{0.528000in}}{\pgfqpoint{4.960000in}{3.696000in}}%
\pgfusepath{clip}%
\pgfsetrectcap%
\pgfsetroundjoin%
\pgfsetlinewidth{0.803000pt}%
\definecolor{currentstroke}{rgb}{0.690196,0.690196,0.690196}%
\pgfsetstrokecolor{currentstroke}%
\pgfsetdash{}{0pt}%
\pgfpathmoveto{\pgfqpoint{0.800000in}{1.536000in}}%
\pgfpathlineto{\pgfqpoint{5.760000in}{1.536000in}}%
\pgfusepath{stroke}%
\end{pgfscope}%
\begin{pgfscope}%
\pgfsetbuttcap%
\pgfsetroundjoin%
\definecolor{currentfill}{rgb}{0.000000,0.000000,0.000000}%
\pgfsetfillcolor{currentfill}%
\pgfsetlinewidth{0.803000pt}%
\definecolor{currentstroke}{rgb}{0.000000,0.000000,0.000000}%
\pgfsetstrokecolor{currentstroke}%
\pgfsetdash{}{0pt}%
\pgfsys@defobject{currentmarker}{\pgfqpoint{-0.048611in}{0.000000in}}{\pgfqpoint{-0.000000in}{0.000000in}}{%
\pgfpathmoveto{\pgfqpoint{-0.000000in}{0.000000in}}%
\pgfpathlineto{\pgfqpoint{-0.048611in}{0.000000in}}%
\pgfusepath{stroke,fill}%
}%
\begin{pgfscope}%
\pgfsys@transformshift{0.800000in}{1.536000in}%
\pgfsys@useobject{currentmarker}{}%
\end{pgfscope}%
\end{pgfscope}%
\begin{pgfscope}%
\definecolor{textcolor}{rgb}{0.000000,0.000000,0.000000}%
\pgfsetstrokecolor{textcolor}%
\pgfsetfillcolor{textcolor}%
\pgftext[x=0.373873in, y=1.483238in, left, base]{\color{textcolor}\sffamily\fontsize{10.000000}{12.000000}\selectfont \ensuremath{-}0.2}%
\end{pgfscope}%
\begin{pgfscope}%
\pgfpathrectangle{\pgfqpoint{0.800000in}{0.528000in}}{\pgfqpoint{4.960000in}{3.696000in}}%
\pgfusepath{clip}%
\pgfsetrectcap%
\pgfsetroundjoin%
\pgfsetlinewidth{0.803000pt}%
\definecolor{currentstroke}{rgb}{0.690196,0.690196,0.690196}%
\pgfsetstrokecolor{currentstroke}%
\pgfsetdash{}{0pt}%
\pgfpathmoveto{\pgfqpoint{0.800000in}{1.956000in}}%
\pgfpathlineto{\pgfqpoint{5.760000in}{1.956000in}}%
\pgfusepath{stroke}%
\end{pgfscope}%
\begin{pgfscope}%
\pgfsetbuttcap%
\pgfsetroundjoin%
\definecolor{currentfill}{rgb}{0.000000,0.000000,0.000000}%
\pgfsetfillcolor{currentfill}%
\pgfsetlinewidth{0.803000pt}%
\definecolor{currentstroke}{rgb}{0.000000,0.000000,0.000000}%
\pgfsetstrokecolor{currentstroke}%
\pgfsetdash{}{0pt}%
\pgfsys@defobject{currentmarker}{\pgfqpoint{-0.048611in}{0.000000in}}{\pgfqpoint{-0.000000in}{0.000000in}}{%
\pgfpathmoveto{\pgfqpoint{-0.000000in}{0.000000in}}%
\pgfpathlineto{\pgfqpoint{-0.048611in}{0.000000in}}%
\pgfusepath{stroke,fill}%
}%
\begin{pgfscope}%
\pgfsys@transformshift{0.800000in}{1.956000in}%
\pgfsys@useobject{currentmarker}{}%
\end{pgfscope}%
\end{pgfscope}%
\begin{pgfscope}%
\definecolor{textcolor}{rgb}{0.000000,0.000000,0.000000}%
\pgfsetstrokecolor{textcolor}%
\pgfsetfillcolor{textcolor}%
\pgftext[x=0.373873in, y=1.903238in, left, base]{\color{textcolor}\sffamily\fontsize{10.000000}{12.000000}\selectfont \ensuremath{-}0.1}%
\end{pgfscope}%
\begin{pgfscope}%
\pgfpathrectangle{\pgfqpoint{0.800000in}{0.528000in}}{\pgfqpoint{4.960000in}{3.696000in}}%
\pgfusepath{clip}%
\pgfsetrectcap%
\pgfsetroundjoin%
\pgfsetlinewidth{0.803000pt}%
\definecolor{currentstroke}{rgb}{0.690196,0.690196,0.690196}%
\pgfsetstrokecolor{currentstroke}%
\pgfsetdash{}{0pt}%
\pgfpathmoveto{\pgfqpoint{0.800000in}{2.376000in}}%
\pgfpathlineto{\pgfqpoint{5.760000in}{2.376000in}}%
\pgfusepath{stroke}%
\end{pgfscope}%
\begin{pgfscope}%
\pgfsetbuttcap%
\pgfsetroundjoin%
\definecolor{currentfill}{rgb}{0.000000,0.000000,0.000000}%
\pgfsetfillcolor{currentfill}%
\pgfsetlinewidth{0.803000pt}%
\definecolor{currentstroke}{rgb}{0.000000,0.000000,0.000000}%
\pgfsetstrokecolor{currentstroke}%
\pgfsetdash{}{0pt}%
\pgfsys@defobject{currentmarker}{\pgfqpoint{-0.048611in}{0.000000in}}{\pgfqpoint{-0.000000in}{0.000000in}}{%
\pgfpathmoveto{\pgfqpoint{-0.000000in}{0.000000in}}%
\pgfpathlineto{\pgfqpoint{-0.048611in}{0.000000in}}%
\pgfusepath{stroke,fill}%
}%
\begin{pgfscope}%
\pgfsys@transformshift{0.800000in}{2.376000in}%
\pgfsys@useobject{currentmarker}{}%
\end{pgfscope}%
\end{pgfscope}%
\begin{pgfscope}%
\definecolor{textcolor}{rgb}{0.000000,0.000000,0.000000}%
\pgfsetstrokecolor{textcolor}%
\pgfsetfillcolor{textcolor}%
\pgftext[x=0.481898in, y=2.323238in, left, base]{\color{textcolor}\sffamily\fontsize{10.000000}{12.000000}\selectfont 0.0}%
\end{pgfscope}%
\begin{pgfscope}%
\pgfpathrectangle{\pgfqpoint{0.800000in}{0.528000in}}{\pgfqpoint{4.960000in}{3.696000in}}%
\pgfusepath{clip}%
\pgfsetrectcap%
\pgfsetroundjoin%
\pgfsetlinewidth{0.803000pt}%
\definecolor{currentstroke}{rgb}{0.690196,0.690196,0.690196}%
\pgfsetstrokecolor{currentstroke}%
\pgfsetdash{}{0pt}%
\pgfpathmoveto{\pgfqpoint{0.800000in}{2.796000in}}%
\pgfpathlineto{\pgfqpoint{5.760000in}{2.796000in}}%
\pgfusepath{stroke}%
\end{pgfscope}%
\begin{pgfscope}%
\pgfsetbuttcap%
\pgfsetroundjoin%
\definecolor{currentfill}{rgb}{0.000000,0.000000,0.000000}%
\pgfsetfillcolor{currentfill}%
\pgfsetlinewidth{0.803000pt}%
\definecolor{currentstroke}{rgb}{0.000000,0.000000,0.000000}%
\pgfsetstrokecolor{currentstroke}%
\pgfsetdash{}{0pt}%
\pgfsys@defobject{currentmarker}{\pgfqpoint{-0.048611in}{0.000000in}}{\pgfqpoint{-0.000000in}{0.000000in}}{%
\pgfpathmoveto{\pgfqpoint{-0.000000in}{0.000000in}}%
\pgfpathlineto{\pgfqpoint{-0.048611in}{0.000000in}}%
\pgfusepath{stroke,fill}%
}%
\begin{pgfscope}%
\pgfsys@transformshift{0.800000in}{2.796000in}%
\pgfsys@useobject{currentmarker}{}%
\end{pgfscope}%
\end{pgfscope}%
\begin{pgfscope}%
\definecolor{textcolor}{rgb}{0.000000,0.000000,0.000000}%
\pgfsetstrokecolor{textcolor}%
\pgfsetfillcolor{textcolor}%
\pgftext[x=0.481898in, y=2.743238in, left, base]{\color{textcolor}\sffamily\fontsize{10.000000}{12.000000}\selectfont 0.1}%
\end{pgfscope}%
\begin{pgfscope}%
\pgfpathrectangle{\pgfqpoint{0.800000in}{0.528000in}}{\pgfqpoint{4.960000in}{3.696000in}}%
\pgfusepath{clip}%
\pgfsetrectcap%
\pgfsetroundjoin%
\pgfsetlinewidth{0.803000pt}%
\definecolor{currentstroke}{rgb}{0.690196,0.690196,0.690196}%
\pgfsetstrokecolor{currentstroke}%
\pgfsetdash{}{0pt}%
\pgfpathmoveto{\pgfqpoint{0.800000in}{3.216000in}}%
\pgfpathlineto{\pgfqpoint{5.760000in}{3.216000in}}%
\pgfusepath{stroke}%
\end{pgfscope}%
\begin{pgfscope}%
\pgfsetbuttcap%
\pgfsetroundjoin%
\definecolor{currentfill}{rgb}{0.000000,0.000000,0.000000}%
\pgfsetfillcolor{currentfill}%
\pgfsetlinewidth{0.803000pt}%
\definecolor{currentstroke}{rgb}{0.000000,0.000000,0.000000}%
\pgfsetstrokecolor{currentstroke}%
\pgfsetdash{}{0pt}%
\pgfsys@defobject{currentmarker}{\pgfqpoint{-0.048611in}{0.000000in}}{\pgfqpoint{-0.000000in}{0.000000in}}{%
\pgfpathmoveto{\pgfqpoint{-0.000000in}{0.000000in}}%
\pgfpathlineto{\pgfqpoint{-0.048611in}{0.000000in}}%
\pgfusepath{stroke,fill}%
}%
\begin{pgfscope}%
\pgfsys@transformshift{0.800000in}{3.216000in}%
\pgfsys@useobject{currentmarker}{}%
\end{pgfscope}%
\end{pgfscope}%
\begin{pgfscope}%
\definecolor{textcolor}{rgb}{0.000000,0.000000,0.000000}%
\pgfsetstrokecolor{textcolor}%
\pgfsetfillcolor{textcolor}%
\pgftext[x=0.481898in, y=3.163238in, left, base]{\color{textcolor}\sffamily\fontsize{10.000000}{12.000000}\selectfont 0.2}%
\end{pgfscope}%
\begin{pgfscope}%
\pgfpathrectangle{\pgfqpoint{0.800000in}{0.528000in}}{\pgfqpoint{4.960000in}{3.696000in}}%
\pgfusepath{clip}%
\pgfsetrectcap%
\pgfsetroundjoin%
\pgfsetlinewidth{0.803000pt}%
\definecolor{currentstroke}{rgb}{0.690196,0.690196,0.690196}%
\pgfsetstrokecolor{currentstroke}%
\pgfsetdash{}{0pt}%
\pgfpathmoveto{\pgfqpoint{0.800000in}{3.636000in}}%
\pgfpathlineto{\pgfqpoint{5.760000in}{3.636000in}}%
\pgfusepath{stroke}%
\end{pgfscope}%
\begin{pgfscope}%
\pgfsetbuttcap%
\pgfsetroundjoin%
\definecolor{currentfill}{rgb}{0.000000,0.000000,0.000000}%
\pgfsetfillcolor{currentfill}%
\pgfsetlinewidth{0.803000pt}%
\definecolor{currentstroke}{rgb}{0.000000,0.000000,0.000000}%
\pgfsetstrokecolor{currentstroke}%
\pgfsetdash{}{0pt}%
\pgfsys@defobject{currentmarker}{\pgfqpoint{-0.048611in}{0.000000in}}{\pgfqpoint{-0.000000in}{0.000000in}}{%
\pgfpathmoveto{\pgfqpoint{-0.000000in}{0.000000in}}%
\pgfpathlineto{\pgfqpoint{-0.048611in}{0.000000in}}%
\pgfusepath{stroke,fill}%
}%
\begin{pgfscope}%
\pgfsys@transformshift{0.800000in}{3.636000in}%
\pgfsys@useobject{currentmarker}{}%
\end{pgfscope}%
\end{pgfscope}%
\begin{pgfscope}%
\definecolor{textcolor}{rgb}{0.000000,0.000000,0.000000}%
\pgfsetstrokecolor{textcolor}%
\pgfsetfillcolor{textcolor}%
\pgftext[x=0.481898in, y=3.583238in, left, base]{\color{textcolor}\sffamily\fontsize{10.000000}{12.000000}\selectfont 0.3}%
\end{pgfscope}%
\begin{pgfscope}%
\pgfpathrectangle{\pgfqpoint{0.800000in}{0.528000in}}{\pgfqpoint{4.960000in}{3.696000in}}%
\pgfusepath{clip}%
\pgfsetrectcap%
\pgfsetroundjoin%
\pgfsetlinewidth{0.803000pt}%
\definecolor{currentstroke}{rgb}{0.690196,0.690196,0.690196}%
\pgfsetstrokecolor{currentstroke}%
\pgfsetdash{}{0pt}%
\pgfpathmoveto{\pgfqpoint{0.800000in}{4.056000in}}%
\pgfpathlineto{\pgfqpoint{5.760000in}{4.056000in}}%
\pgfusepath{stroke}%
\end{pgfscope}%
\begin{pgfscope}%
\pgfsetbuttcap%
\pgfsetroundjoin%
\definecolor{currentfill}{rgb}{0.000000,0.000000,0.000000}%
\pgfsetfillcolor{currentfill}%
\pgfsetlinewidth{0.803000pt}%
\definecolor{currentstroke}{rgb}{0.000000,0.000000,0.000000}%
\pgfsetstrokecolor{currentstroke}%
\pgfsetdash{}{0pt}%
\pgfsys@defobject{currentmarker}{\pgfqpoint{-0.048611in}{0.000000in}}{\pgfqpoint{-0.000000in}{0.000000in}}{%
\pgfpathmoveto{\pgfqpoint{-0.000000in}{0.000000in}}%
\pgfpathlineto{\pgfqpoint{-0.048611in}{0.000000in}}%
\pgfusepath{stroke,fill}%
}%
\begin{pgfscope}%
\pgfsys@transformshift{0.800000in}{4.056000in}%
\pgfsys@useobject{currentmarker}{}%
\end{pgfscope}%
\end{pgfscope}%
\begin{pgfscope}%
\definecolor{textcolor}{rgb}{0.000000,0.000000,0.000000}%
\pgfsetstrokecolor{textcolor}%
\pgfsetfillcolor{textcolor}%
\pgftext[x=0.481898in, y=4.003238in, left, base]{\color{textcolor}\sffamily\fontsize{10.000000}{12.000000}\selectfont 0.4}%
\end{pgfscope}%
\begin{pgfscope}%
\definecolor{textcolor}{rgb}{0.000000,0.000000,0.000000}%
\pgfsetstrokecolor{textcolor}%
\pgfsetfillcolor{textcolor}%
\pgftext[x=0.318318in,y=2.376000in,,bottom,rotate=90.000000]{\color{textcolor}\sffamily\fontsize{10.000000}{12.000000}\selectfont  [m/s]}%
\end{pgfscope}%
\begin{pgfscope}%
\pgfpathrectangle{\pgfqpoint{0.800000in}{0.528000in}}{\pgfqpoint{4.960000in}{3.696000in}}%
\pgfusepath{clip}%
\pgfsetrectcap%
\pgfsetroundjoin%
\pgfsetlinewidth{1.505625pt}%
\definecolor{currentstroke}{rgb}{0.121569,0.466667,0.705882}%
\pgfsetstrokecolor{currentstroke}%
\pgfsetdash{}{0pt}%
\pgfpathmoveto{\pgfqpoint{1.025455in}{0.696000in}}%
\pgfpathlineto{\pgfqpoint{1.079346in}{0.696000in}}%
\pgfpathlineto{\pgfqpoint{1.133733in}{0.727926in}}%
\pgfpathlineto{\pgfqpoint{1.188343in}{0.742187in}}%
\pgfpathlineto{\pgfqpoint{1.243001in}{0.790811in}}%
\pgfpathlineto{\pgfqpoint{1.297692in}{1.072629in}}%
\pgfpathlineto{\pgfqpoint{1.350599in}{1.538835in}}%
\pgfpathlineto{\pgfqpoint{1.404959in}{2.061098in}}%
\pgfpathlineto{\pgfqpoint{1.459957in}{2.199573in}}%
\pgfpathlineto{\pgfqpoint{1.514486in}{2.339699in}}%
\pgfpathlineto{\pgfqpoint{1.569896in}{2.392210in}}%
\pgfpathlineto{\pgfqpoint{1.623211in}{2.370746in}}%
\pgfpathlineto{\pgfqpoint{1.677396in}{2.419787in}}%
\pgfpathlineto{\pgfqpoint{1.731440in}{2.403359in}}%
\pgfpathlineto{\pgfqpoint{1.785614in}{2.357552in}}%
\pgfpathlineto{\pgfqpoint{1.839797in}{2.442928in}}%
\pgfpathlineto{\pgfqpoint{1.894125in}{2.394795in}}%
\pgfpathlineto{\pgfqpoint{1.948443in}{2.444421in}}%
\pgfpathlineto{\pgfqpoint{2.002587in}{2.404502in}}%
\pgfpathlineto{\pgfqpoint{2.056571in}{2.433474in}}%
\pgfpathlineto{\pgfqpoint{2.111067in}{2.421434in}}%
\pgfpathlineto{\pgfqpoint{2.165740in}{2.387408in}}%
\pgfpathlineto{\pgfqpoint{2.219578in}{2.339799in}}%
\pgfpathlineto{\pgfqpoint{2.273625in}{2.337097in}}%
\pgfpathlineto{\pgfqpoint{2.327972in}{2.322361in}}%
\pgfpathlineto{\pgfqpoint{2.382158in}{2.276899in}}%
\pgfpathlineto{\pgfqpoint{2.436201in}{2.242623in}}%
\pgfpathlineto{\pgfqpoint{2.490727in}{2.248148in}}%
\pgfpathlineto{\pgfqpoint{2.545152in}{2.238449in}}%
\pgfpathlineto{\pgfqpoint{2.599168in}{2.325770in}}%
\pgfpathlineto{\pgfqpoint{2.653162in}{2.354850in}}%
\pgfpathlineto{\pgfqpoint{2.708816in}{2.354665in}}%
\pgfpathlineto{\pgfqpoint{2.763068in}{2.377844in}}%
\pgfpathlineto{\pgfqpoint{2.817233in}{2.355475in}}%
\pgfpathlineto{\pgfqpoint{2.871345in}{2.372195in}}%
\pgfpathlineto{\pgfqpoint{2.925498in}{2.336684in}}%
\pgfpathlineto{\pgfqpoint{2.979648in}{2.356473in}}%
\pgfpathlineto{\pgfqpoint{3.033884in}{2.349643in}}%
\pgfpathlineto{\pgfqpoint{3.088277in}{2.392871in}}%
\pgfpathlineto{\pgfqpoint{3.142459in}{2.352202in}}%
\pgfpathlineto{\pgfqpoint{3.196688in}{2.357344in}}%
\pgfpathlineto{\pgfqpoint{3.250942in}{2.351943in}}%
\pgfpathlineto{\pgfqpoint{3.305168in}{2.349630in}}%
\pgfpathlineto{\pgfqpoint{3.360651in}{2.376798in}}%
\pgfpathlineto{\pgfqpoint{3.414236in}{2.355755in}}%
\pgfpathlineto{\pgfqpoint{3.468063in}{2.373346in}}%
\pgfpathlineto{\pgfqpoint{3.523055in}{2.361539in}}%
\pgfpathlineto{\pgfqpoint{3.577067in}{2.372582in}}%
\pgfpathlineto{\pgfqpoint{3.631733in}{2.420179in}}%
\pgfpathlineto{\pgfqpoint{3.686018in}{2.403387in}}%
\pgfpathlineto{\pgfqpoint{3.740366in}{2.393878in}}%
\pgfpathlineto{\pgfqpoint{3.794333in}{2.385276in}}%
\pgfpathlineto{\pgfqpoint{3.848511in}{2.377379in}}%
\pgfpathlineto{\pgfqpoint{3.902528in}{2.361496in}}%
\pgfpathlineto{\pgfqpoint{3.958208in}{2.355790in}}%
\pgfpathlineto{\pgfqpoint{4.011516in}{2.348990in}}%
\pgfpathlineto{\pgfqpoint{4.065429in}{2.347069in}}%
\pgfpathlineto{\pgfqpoint{4.119969in}{2.345278in}}%
\pgfpathlineto{\pgfqpoint{4.174054in}{2.335177in}}%
\pgfpathlineto{\pgfqpoint{4.228102in}{2.333911in}}%
\pgfpathlineto{\pgfqpoint{4.282096in}{2.310985in}}%
\pgfpathlineto{\pgfqpoint{4.336947in}{2.313514in}}%
\pgfpathlineto{\pgfqpoint{4.390525in}{2.313937in}}%
\pgfpathlineto{\pgfqpoint{4.444795in}{2.322840in}}%
\pgfpathlineto{\pgfqpoint{4.499103in}{2.354703in}}%
\pgfpathlineto{\pgfqpoint{4.553310in}{2.368175in}}%
\pgfpathlineto{\pgfqpoint{4.608482in}{2.376924in}}%
\pgfpathlineto{\pgfqpoint{4.662064in}{2.391363in}}%
\pgfpathlineto{\pgfqpoint{4.715946in}{2.400247in}}%
\pgfpathlineto{\pgfqpoint{4.770242in}{2.394357in}}%
\pgfpathlineto{\pgfqpoint{4.824550in}{2.385912in}}%
\pgfpathlineto{\pgfqpoint{4.878651in}{2.386489in}}%
\pgfpathlineto{\pgfqpoint{4.933610in}{2.401256in}}%
\pgfpathlineto{\pgfqpoint{4.987108in}{2.413418in}}%
\pgfpathlineto{\pgfqpoint{5.041786in}{2.406178in}}%
\pgfpathlineto{\pgfqpoint{5.096356in}{2.397512in}}%
\pgfpathlineto{\pgfqpoint{5.150043in}{2.376617in}}%
\pgfpathlineto{\pgfqpoint{5.205548in}{2.374617in}}%
\pgfpathlineto{\pgfqpoint{5.259176in}{2.318146in}}%
\pgfpathlineto{\pgfqpoint{5.314060in}{2.314315in}}%
\pgfpathlineto{\pgfqpoint{5.368522in}{2.313985in}}%
\pgfpathlineto{\pgfqpoint{5.422471in}{2.298871in}}%
\pgfpathlineto{\pgfqpoint{5.476713in}{2.293455in}}%
\pgfpathlineto{\pgfqpoint{5.531850in}{2.295679in}}%
\pgfusepath{stroke}%
\end{pgfscope}%
\begin{pgfscope}%
\pgfpathrectangle{\pgfqpoint{0.800000in}{0.528000in}}{\pgfqpoint{4.960000in}{3.696000in}}%
\pgfusepath{clip}%
\pgfsetrectcap%
\pgfsetroundjoin%
\pgfsetlinewidth{1.505625pt}%
\definecolor{currentstroke}{rgb}{1.000000,0.498039,0.054902}%
\pgfsetstrokecolor{currentstroke}%
\pgfsetdash{}{0pt}%
\pgfpathmoveto{\pgfqpoint{1.025455in}{0.696000in}}%
\pgfpathlineto{\pgfqpoint{1.079097in}{0.696000in}}%
\pgfpathlineto{\pgfqpoint{1.133290in}{0.696000in}}%
\pgfpathlineto{\pgfqpoint{1.187648in}{0.696000in}}%
\pgfpathlineto{\pgfqpoint{1.242800in}{0.696000in}}%
\pgfpathlineto{\pgfqpoint{1.297032in}{0.696000in}}%
\pgfpathlineto{\pgfqpoint{1.351342in}{1.270173in}}%
\pgfpathlineto{\pgfqpoint{1.405625in}{1.707973in}}%
\pgfpathlineto{\pgfqpoint{1.459821in}{2.166494in}}%
\pgfpathlineto{\pgfqpoint{1.514206in}{2.139559in}}%
\pgfpathlineto{\pgfqpoint{1.569696in}{2.287303in}}%
\pgfpathlineto{\pgfqpoint{1.623165in}{2.359917in}}%
\pgfpathlineto{\pgfqpoint{1.676738in}{2.461304in}}%
\pgfpathlineto{\pgfqpoint{1.731575in}{2.519038in}}%
\pgfpathlineto{\pgfqpoint{1.786338in}{2.549458in}}%
\pgfpathlineto{\pgfqpoint{1.840720in}{2.544457in}}%
\pgfpathlineto{\pgfqpoint{1.894971in}{2.427574in}}%
\pgfpathlineto{\pgfqpoint{1.949451in}{2.378169in}}%
\pgfpathlineto{\pgfqpoint{2.003523in}{2.424961in}}%
\pgfpathlineto{\pgfqpoint{2.057638in}{2.398612in}}%
\pgfpathlineto{\pgfqpoint{2.111872in}{2.362539in}}%
\pgfpathlineto{\pgfqpoint{2.167806in}{2.290800in}}%
\pgfpathlineto{\pgfqpoint{2.221402in}{2.194721in}}%
\pgfpathlineto{\pgfqpoint{2.275305in}{2.141929in}}%
\pgfpathlineto{\pgfqpoint{2.329434in}{2.148714in}}%
\pgfpathlineto{\pgfqpoint{2.383566in}{2.194747in}}%
\pgfpathlineto{\pgfqpoint{2.437640in}{2.174650in}}%
\pgfpathlineto{\pgfqpoint{2.491996in}{2.482173in}}%
\pgfpathlineto{\pgfqpoint{2.546218in}{2.365414in}}%
\pgfpathlineto{\pgfqpoint{2.600534in}{2.357098in}}%
\pgfpathlineto{\pgfqpoint{2.654787in}{2.363552in}}%
\pgfpathlineto{\pgfqpoint{2.709027in}{2.366941in}}%
\pgfpathlineto{\pgfqpoint{2.763930in}{2.384696in}}%
\pgfpathlineto{\pgfqpoint{2.817830in}{2.418168in}}%
\pgfpathlineto{\pgfqpoint{2.871752in}{2.405186in}}%
\pgfpathlineto{\pgfqpoint{2.925837in}{2.400419in}}%
\pgfpathlineto{\pgfqpoint{2.980536in}{2.381722in}}%
\pgfpathlineto{\pgfqpoint{3.034703in}{2.417870in}}%
\pgfpathlineto{\pgfqpoint{3.089039in}{2.413900in}}%
\pgfpathlineto{\pgfqpoint{3.143307in}{2.412243in}}%
\pgfpathlineto{\pgfqpoint{3.197327in}{2.396343in}}%
\pgfpathlineto{\pgfqpoint{3.251469in}{2.384550in}}%
\pgfpathlineto{\pgfqpoint{3.305853in}{2.411627in}}%
\pgfpathlineto{\pgfqpoint{3.361722in}{2.414526in}}%
\pgfpathlineto{\pgfqpoint{3.415028in}{2.398321in}}%
\pgfpathlineto{\pgfqpoint{3.469019in}{2.385164in}}%
\pgfpathlineto{\pgfqpoint{3.523110in}{2.321939in}}%
\pgfpathlineto{\pgfqpoint{3.577426in}{2.274832in}}%
\pgfpathlineto{\pgfqpoint{3.631486in}{2.293952in}}%
\pgfpathlineto{\pgfqpoint{3.686172in}{2.249749in}}%
\pgfpathlineto{\pgfqpoint{3.740698in}{2.288558in}}%
\pgfpathlineto{\pgfqpoint{3.794630in}{2.268706in}}%
\pgfpathlineto{\pgfqpoint{3.848700in}{2.292728in}}%
\pgfpathlineto{\pgfqpoint{3.904942in}{2.295584in}}%
\pgfpathlineto{\pgfqpoint{3.958943in}{2.591375in}}%
\pgfpathlineto{\pgfqpoint{4.012177in}{2.560467in}}%
\pgfpathlineto{\pgfqpoint{4.066136in}{2.509159in}}%
\pgfpathlineto{\pgfqpoint{4.120402in}{2.614343in}}%
\pgfpathlineto{\pgfqpoint{4.174524in}{2.574850in}}%
\pgfpathlineto{\pgfqpoint{4.228795in}{2.342072in}}%
\pgfpathlineto{\pgfqpoint{4.283227in}{2.247435in}}%
\pgfpathlineto{\pgfqpoint{4.337654in}{2.262979in}}%
\pgfpathlineto{\pgfqpoint{4.391591in}{2.292778in}}%
\pgfpathlineto{\pgfqpoint{4.445858in}{2.320987in}}%
\pgfpathlineto{\pgfqpoint{4.500443in}{2.330695in}}%
\pgfpathlineto{\pgfqpoint{4.555957in}{2.341861in}}%
\pgfpathlineto{\pgfqpoint{4.609229in}{2.339868in}}%
\pgfpathlineto{\pgfqpoint{4.663051in}{2.340961in}}%
\pgfpathlineto{\pgfqpoint{4.716973in}{2.329641in}}%
\pgfpathlineto{\pgfqpoint{4.771149in}{2.345229in}}%
\pgfpathlineto{\pgfqpoint{4.825460in}{2.349254in}}%
\pgfpathlineto{\pgfqpoint{4.879583in}{2.350472in}}%
\pgfpathlineto{\pgfqpoint{4.935368in}{2.375014in}}%
\pgfpathlineto{\pgfqpoint{4.989369in}{2.372343in}}%
\pgfpathlineto{\pgfqpoint{5.043418in}{2.372001in}}%
\pgfpathlineto{\pgfqpoint{5.097613in}{2.365518in}}%
\pgfpathlineto{\pgfqpoint{5.153554in}{2.364730in}}%
\pgfpathlineto{\pgfqpoint{5.206849in}{2.364441in}}%
\pgfpathlineto{\pgfqpoint{5.260579in}{2.356178in}}%
\pgfpathlineto{\pgfqpoint{5.314728in}{2.351092in}}%
\pgfpathlineto{\pgfqpoint{5.368880in}{2.368519in}}%
\pgfpathlineto{\pgfqpoint{5.423031in}{2.368566in}}%
\pgfpathlineto{\pgfqpoint{5.477563in}{2.377254in}}%
\pgfpathlineto{\pgfqpoint{5.531987in}{2.375676in}}%
\pgfusepath{stroke}%
\end{pgfscope}%
\begin{pgfscope}%
\pgfpathrectangle{\pgfqpoint{0.800000in}{0.528000in}}{\pgfqpoint{4.960000in}{3.696000in}}%
\pgfusepath{clip}%
\pgfsetrectcap%
\pgfsetroundjoin%
\pgfsetlinewidth{1.505625pt}%
\definecolor{currentstroke}{rgb}{0.172549,0.627451,0.172549}%
\pgfsetstrokecolor{currentstroke}%
\pgfsetdash{}{0pt}%
\pgfpathmoveto{\pgfqpoint{1.025455in}{0.696000in}}%
\pgfpathlineto{\pgfqpoint{1.079118in}{0.696000in}}%
\pgfpathlineto{\pgfqpoint{1.133167in}{0.696000in}}%
\pgfpathlineto{\pgfqpoint{1.187559in}{0.696000in}}%
\pgfpathlineto{\pgfqpoint{1.241834in}{0.696000in}}%
\pgfpathlineto{\pgfqpoint{1.298213in}{0.696000in}}%
\pgfpathlineto{\pgfqpoint{1.350148in}{0.696000in}}%
\pgfpathlineto{\pgfqpoint{1.404149in}{0.924868in}}%
\pgfpathlineto{\pgfqpoint{1.458653in}{1.813923in}}%
\pgfpathlineto{\pgfqpoint{1.514270in}{2.582129in}}%
\pgfpathlineto{\pgfqpoint{1.567555in}{2.712487in}}%
\pgfpathlineto{\pgfqpoint{1.622013in}{2.852276in}}%
\pgfpathlineto{\pgfqpoint{1.676086in}{2.899874in}}%
\pgfpathlineto{\pgfqpoint{1.730346in}{3.225454in}}%
\pgfpathlineto{\pgfqpoint{1.784400in}{2.966118in}}%
\pgfpathlineto{\pgfqpoint{1.838929in}{2.710362in}}%
\pgfpathlineto{\pgfqpoint{1.895431in}{3.030380in}}%
\pgfpathlineto{\pgfqpoint{1.947023in}{2.835230in}}%
\pgfpathlineto{\pgfqpoint{2.001357in}{2.669535in}}%
\pgfpathlineto{\pgfqpoint{2.055546in}{2.064411in}}%
\pgfpathlineto{\pgfqpoint{2.111687in}{1.420360in}}%
\pgfpathlineto{\pgfqpoint{2.165154in}{1.007302in}}%
\pgfpathlineto{\pgfqpoint{2.219132in}{0.810665in}}%
\pgfpathlineto{\pgfqpoint{2.273327in}{1.171173in}}%
\pgfpathlineto{\pgfqpoint{2.327792in}{2.438312in}}%
\pgfpathlineto{\pgfqpoint{2.381690in}{2.651168in}}%
\pgfpathlineto{\pgfqpoint{2.435983in}{2.824765in}}%
\pgfpathlineto{\pgfqpoint{2.490271in}{3.016860in}}%
\pgfpathlineto{\pgfqpoint{2.545021in}{2.958975in}}%
\pgfpathlineto{\pgfqpoint{2.599538in}{3.244502in}}%
\pgfpathlineto{\pgfqpoint{2.653746in}{2.961048in}}%
\pgfpathlineto{\pgfqpoint{2.709840in}{2.710180in}}%
\pgfpathlineto{\pgfqpoint{2.762944in}{2.461864in}}%
\pgfpathlineto{\pgfqpoint{2.816737in}{2.478641in}}%
\pgfpathlineto{\pgfqpoint{2.870654in}{2.513504in}}%
\pgfpathlineto{\pgfqpoint{2.924871in}{2.282130in}}%
\pgfpathlineto{\pgfqpoint{2.979336in}{2.328038in}}%
\pgfpathlineto{\pgfqpoint{3.033403in}{2.297175in}}%
\pgfpathlineto{\pgfqpoint{3.087801in}{2.062515in}}%
\pgfpathlineto{\pgfqpoint{3.141910in}{1.959759in}}%
\pgfpathlineto{\pgfqpoint{3.196257in}{1.928537in}}%
\pgfpathlineto{\pgfqpoint{3.250954in}{2.315722in}}%
\pgfpathlineto{\pgfqpoint{3.305517in}{2.230834in}}%
\pgfpathlineto{\pgfqpoint{3.361240in}{2.229005in}}%
\pgfpathlineto{\pgfqpoint{3.414602in}{2.268374in}}%
\pgfpathlineto{\pgfqpoint{3.468834in}{2.397438in}}%
\pgfpathlineto{\pgfqpoint{3.522565in}{2.438303in}}%
\pgfpathlineto{\pgfqpoint{3.576800in}{2.494168in}}%
\pgfpathlineto{\pgfqpoint{3.631129in}{2.449611in}}%
\pgfpathlineto{\pgfqpoint{3.685263in}{2.394923in}}%
\pgfpathlineto{\pgfqpoint{3.739292in}{2.444536in}}%
\pgfpathlineto{\pgfqpoint{3.793550in}{2.425040in}}%
\pgfpathlineto{\pgfqpoint{3.847941in}{2.410859in}}%
\pgfpathlineto{\pgfqpoint{3.902737in}{2.409939in}}%
\pgfpathlineto{\pgfqpoint{3.956780in}{2.377843in}}%
\pgfpathlineto{\pgfqpoint{4.012438in}{2.376192in}}%
\pgfpathlineto{\pgfqpoint{4.065856in}{2.359887in}}%
\pgfpathlineto{\pgfqpoint{4.119873in}{2.350803in}}%
\pgfpathlineto{\pgfqpoint{4.173597in}{2.334046in}}%
\pgfpathlineto{\pgfqpoint{4.227910in}{2.333920in}}%
\pgfpathlineto{\pgfqpoint{4.282610in}{2.327762in}}%
\pgfpathlineto{\pgfqpoint{4.336431in}{2.320960in}}%
\pgfpathlineto{\pgfqpoint{4.390236in}{2.320828in}}%
\pgfpathlineto{\pgfqpoint{4.444628in}{2.320702in}}%
\pgfpathlineto{\pgfqpoint{4.500735in}{2.320540in}}%
\pgfpathlineto{\pgfqpoint{4.554421in}{2.296330in}}%
\pgfpathlineto{\pgfqpoint{4.608131in}{2.272343in}}%
\pgfpathlineto{\pgfqpoint{4.662275in}{2.266946in}}%
\pgfpathlineto{\pgfqpoint{4.716284in}{2.273172in}}%
\pgfpathlineto{\pgfqpoint{4.770614in}{2.270573in}}%
\pgfpathlineto{\pgfqpoint{4.824533in}{2.272986in}}%
\pgfpathlineto{\pgfqpoint{4.879129in}{2.265852in}}%
\pgfpathlineto{\pgfqpoint{4.933628in}{2.312157in}}%
\pgfpathlineto{\pgfqpoint{4.988017in}{2.328900in}}%
\pgfpathlineto{\pgfqpoint{5.042027in}{2.363584in}}%
\pgfpathlineto{\pgfqpoint{5.096457in}{2.371356in}}%
\pgfpathlineto{\pgfqpoint{5.151536in}{2.419793in}}%
\pgfpathlineto{\pgfqpoint{5.205771in}{2.476816in}}%
\pgfpathlineto{\pgfqpoint{5.259761in}{2.501424in}}%
\pgfpathlineto{\pgfqpoint{5.314187in}{2.539029in}}%
\pgfpathlineto{\pgfqpoint{5.367959in}{2.536379in}}%
\pgfpathlineto{\pgfqpoint{5.422056in}{2.549959in}}%
\pgfpathlineto{\pgfqpoint{5.476438in}{2.586449in}}%
\pgfpathlineto{\pgfqpoint{5.530639in}{2.609005in}}%
\pgfusepath{stroke}%
\end{pgfscope}%
\begin{pgfscope}%
\pgfpathrectangle{\pgfqpoint{0.800000in}{0.528000in}}{\pgfqpoint{4.960000in}{3.696000in}}%
\pgfusepath{clip}%
\pgfsetrectcap%
\pgfsetroundjoin%
\pgfsetlinewidth{1.505625pt}%
\definecolor{currentstroke}{rgb}{0.839216,0.152941,0.156863}%
\pgfsetstrokecolor{currentstroke}%
\pgfsetdash{}{0pt}%
\pgfpathmoveto{\pgfqpoint{1.025455in}{0.696000in}}%
\pgfpathlineto{\pgfqpoint{1.079580in}{0.696000in}}%
\pgfpathlineto{\pgfqpoint{1.133762in}{0.696000in}}%
\pgfpathlineto{\pgfqpoint{1.187808in}{0.696000in}}%
\pgfpathlineto{\pgfqpoint{1.241911in}{0.696000in}}%
\pgfpathlineto{\pgfqpoint{1.297793in}{0.696000in}}%
\pgfpathlineto{\pgfqpoint{1.349164in}{0.696000in}}%
\pgfpathlineto{\pgfqpoint{1.403831in}{0.963277in}}%
\pgfpathlineto{\pgfqpoint{1.459885in}{1.631969in}}%
\pgfpathlineto{\pgfqpoint{1.513066in}{2.403331in}}%
\pgfpathlineto{\pgfqpoint{1.569073in}{2.833979in}}%
\pgfpathlineto{\pgfqpoint{1.620912in}{2.991563in}}%
\pgfpathlineto{\pgfqpoint{1.674991in}{2.931061in}}%
\pgfpathlineto{\pgfqpoint{1.729485in}{2.814387in}}%
\pgfpathlineto{\pgfqpoint{1.783597in}{2.492233in}}%
\pgfpathlineto{\pgfqpoint{1.838041in}{3.099277in}}%
\pgfpathlineto{\pgfqpoint{1.892167in}{2.934598in}}%
\pgfpathlineto{\pgfqpoint{1.946590in}{2.901505in}}%
\pgfpathlineto{\pgfqpoint{2.001140in}{1.934838in}}%
\pgfpathlineto{\pgfqpoint{2.060457in}{1.442991in}}%
\pgfpathlineto{\pgfqpoint{2.110573in}{1.806355in}}%
\pgfpathlineto{\pgfqpoint{2.163982in}{1.966628in}}%
\pgfpathlineto{\pgfqpoint{2.219073in}{2.405405in}}%
\pgfpathlineto{\pgfqpoint{2.273282in}{2.271318in}}%
\pgfpathlineto{\pgfqpoint{2.327258in}{2.533105in}}%
\pgfpathlineto{\pgfqpoint{2.381833in}{2.983884in}}%
\pgfpathlineto{\pgfqpoint{2.436028in}{2.768737in}}%
\pgfpathlineto{\pgfqpoint{2.490065in}{2.555753in}}%
\pgfpathlineto{\pgfqpoint{2.544453in}{2.792472in}}%
\pgfpathlineto{\pgfqpoint{2.598580in}{2.658274in}}%
\pgfpathlineto{\pgfqpoint{2.653580in}{2.419015in}}%
\pgfpathlineto{\pgfqpoint{2.709855in}{2.387701in}}%
\pgfpathlineto{\pgfqpoint{2.763088in}{2.172823in}}%
\pgfpathlineto{\pgfqpoint{2.817383in}{2.043943in}}%
\pgfpathlineto{\pgfqpoint{2.871388in}{1.882326in}}%
\pgfpathlineto{\pgfqpoint{2.926073in}{1.694036in}}%
\pgfpathlineto{\pgfqpoint{2.980262in}{1.690808in}}%
\pgfpathlineto{\pgfqpoint{3.034808in}{2.059555in}}%
\pgfpathlineto{\pgfqpoint{3.089346in}{2.184942in}}%
\pgfpathlineto{\pgfqpoint{3.142968in}{2.513713in}}%
\pgfpathlineto{\pgfqpoint{3.197588in}{2.317679in}}%
\pgfpathlineto{\pgfqpoint{3.251671in}{2.694715in}}%
\pgfpathlineto{\pgfqpoint{3.307440in}{2.652064in}}%
\pgfpathlineto{\pgfqpoint{3.361306in}{2.730884in}}%
\pgfpathlineto{\pgfqpoint{3.416297in}{2.677502in}}%
\pgfpathlineto{\pgfqpoint{3.470449in}{3.016417in}}%
\pgfpathlineto{\pgfqpoint{3.523914in}{2.854921in}}%
\pgfpathlineto{\pgfqpoint{3.578473in}{2.502007in}}%
\pgfpathlineto{\pgfqpoint{3.632583in}{1.852045in}}%
\pgfpathlineto{\pgfqpoint{3.687107in}{2.077409in}}%
\pgfpathlineto{\pgfqpoint{3.742604in}{2.213231in}}%
\pgfpathlineto{\pgfqpoint{3.796394in}{2.296723in}}%
\pgfpathlineto{\pgfqpoint{3.850516in}{2.160105in}}%
\pgfpathlineto{\pgfqpoint{3.906056in}{2.149081in}}%
\pgfpathlineto{\pgfqpoint{3.959288in}{2.159677in}}%
\pgfpathlineto{\pgfqpoint{4.013068in}{2.079153in}}%
\pgfpathlineto{\pgfqpoint{4.067296in}{2.295817in}}%
\pgfpathlineto{\pgfqpoint{4.121172in}{2.991296in}}%
\pgfpathlineto{\pgfqpoint{4.175689in}{2.567616in}}%
\pgfpathlineto{\pgfqpoint{4.229933in}{2.552920in}}%
\pgfpathlineto{\pgfqpoint{4.284986in}{2.242609in}}%
\pgfpathlineto{\pgfqpoint{4.338971in}{2.237610in}}%
\pgfpathlineto{\pgfqpoint{4.393209in}{2.415958in}}%
\pgfpathlineto{\pgfqpoint{4.447806in}{2.303063in}}%
\pgfpathlineto{\pgfqpoint{4.502549in}{2.301842in}}%
\pgfpathlineto{\pgfqpoint{4.557534in}{2.265175in}}%
\pgfpathlineto{\pgfqpoint{4.610982in}{2.283005in}}%
\pgfpathlineto{\pgfqpoint{4.665088in}{2.288141in}}%
\pgfpathlineto{\pgfqpoint{4.719343in}{2.292342in}}%
\pgfpathlineto{\pgfqpoint{4.773736in}{2.274853in}}%
\pgfpathlineto{\pgfqpoint{4.827702in}{2.249727in}}%
\pgfpathlineto{\pgfqpoint{4.882646in}{2.192111in}}%
\pgfpathlineto{\pgfqpoint{4.936301in}{2.210386in}}%
\pgfpathlineto{\pgfqpoint{4.990694in}{2.227877in}}%
\pgfpathlineto{\pgfqpoint{5.045010in}{3.094617in}}%
\pgfpathlineto{\pgfqpoint{5.099315in}{2.388263in}}%
\pgfpathlineto{\pgfqpoint{5.155259in}{2.314731in}}%
\pgfpathlineto{\pgfqpoint{5.208444in}{2.439967in}}%
\pgfpathlineto{\pgfqpoint{5.262143in}{3.006273in}}%
\pgfpathlineto{\pgfqpoint{5.316238in}{2.614484in}}%
\pgfpathlineto{\pgfqpoint{5.370459in}{2.166120in}}%
\pgfpathlineto{\pgfqpoint{5.425230in}{1.934806in}}%
\pgfpathlineto{\pgfqpoint{5.479369in}{2.351690in}}%
\pgfpathlineto{\pgfqpoint{5.534545in}{2.255923in}}%
\pgfusepath{stroke}%
\end{pgfscope}%
\begin{pgfscope}%
\pgfpathrectangle{\pgfqpoint{0.800000in}{0.528000in}}{\pgfqpoint{4.960000in}{3.696000in}}%
\pgfusepath{clip}%
\pgfsetrectcap%
\pgfsetroundjoin%
\pgfsetlinewidth{1.505625pt}%
\definecolor{currentstroke}{rgb}{0.580392,0.403922,0.741176}%
\pgfsetstrokecolor{currentstroke}%
\pgfsetdash{}{0pt}%
\pgfpathmoveto{\pgfqpoint{1.025455in}{0.696000in}}%
\pgfpathlineto{\pgfqpoint{1.079775in}{0.696000in}}%
\pgfpathlineto{\pgfqpoint{1.134512in}{0.696000in}}%
\pgfpathlineto{\pgfqpoint{1.188622in}{0.696000in}}%
\pgfpathlineto{\pgfqpoint{1.244257in}{0.696000in}}%
\pgfpathlineto{\pgfqpoint{1.297294in}{0.696000in}}%
\pgfpathlineto{\pgfqpoint{1.350518in}{0.696000in}}%
\pgfpathlineto{\pgfqpoint{1.404521in}{1.253097in}}%
\pgfpathlineto{\pgfqpoint{1.458607in}{2.220208in}}%
\pgfpathlineto{\pgfqpoint{1.513370in}{2.503976in}}%
\pgfpathlineto{\pgfqpoint{1.567196in}{3.070499in}}%
\pgfpathlineto{\pgfqpoint{1.621234in}{2.920965in}}%
\pgfpathlineto{\pgfqpoint{1.675622in}{3.099353in}}%
\pgfpathlineto{\pgfqpoint{1.730254in}{3.814620in}}%
\pgfpathlineto{\pgfqpoint{1.786788in}{3.083927in}}%
\pgfpathlineto{\pgfqpoint{1.840218in}{4.056000in}}%
\pgfpathlineto{\pgfqpoint{1.894067in}{2.844097in}}%
\pgfpathlineto{\pgfqpoint{1.948041in}{0.696000in}}%
\pgfpathlineto{\pgfqpoint{2.001840in}{1.574488in}}%
\pgfpathlineto{\pgfqpoint{2.056502in}{0.696000in}}%
\pgfpathlineto{\pgfqpoint{2.110587in}{0.696000in}}%
\pgfpathlineto{\pgfqpoint{2.164640in}{0.696000in}}%
\pgfpathlineto{\pgfqpoint{2.219082in}{2.904608in}}%
\pgfpathlineto{\pgfqpoint{2.272843in}{3.663972in}}%
\pgfpathlineto{\pgfqpoint{2.327753in}{2.586497in}}%
\pgfpathlineto{\pgfqpoint{2.381989in}{3.284191in}}%
\pgfpathlineto{\pgfqpoint{2.435675in}{3.078643in}}%
\pgfpathlineto{\pgfqpoint{2.489952in}{3.407496in}}%
\pgfpathlineto{\pgfqpoint{2.544170in}{3.374490in}}%
\pgfpathlineto{\pgfqpoint{2.599652in}{2.527293in}}%
\pgfpathlineto{\pgfqpoint{2.653382in}{1.792001in}}%
\pgfpathlineto{\pgfqpoint{2.708423in}{0.696000in}}%
\pgfpathlineto{\pgfqpoint{2.762914in}{0.696000in}}%
\pgfpathlineto{\pgfqpoint{2.817087in}{0.696000in}}%
\pgfpathlineto{\pgfqpoint{2.871525in}{1.145599in}}%
\pgfpathlineto{\pgfqpoint{2.925766in}{2.588711in}}%
\pgfpathlineto{\pgfqpoint{2.979861in}{2.566722in}}%
\pgfpathlineto{\pgfqpoint{3.034913in}{2.893430in}}%
\pgfpathlineto{\pgfqpoint{3.089363in}{2.691691in}}%
\pgfpathlineto{\pgfqpoint{3.144439in}{2.985442in}}%
\pgfpathlineto{\pgfqpoint{3.198438in}{3.070580in}}%
\pgfpathlineto{\pgfqpoint{3.252573in}{3.238288in}}%
\pgfpathlineto{\pgfqpoint{3.306794in}{3.428140in}}%
\pgfpathlineto{\pgfqpoint{3.361211in}{3.296628in}}%
\pgfpathlineto{\pgfqpoint{3.416834in}{2.223209in}}%
\pgfpathlineto{\pgfqpoint{3.470506in}{1.150718in}}%
\pgfpathlineto{\pgfqpoint{3.524067in}{0.696000in}}%
\pgfpathlineto{\pgfqpoint{3.578493in}{1.423999in}}%
\pgfpathlineto{\pgfqpoint{3.633218in}{1.433603in}}%
\pgfpathlineto{\pgfqpoint{3.687655in}{2.488048in}}%
\pgfpathlineto{\pgfqpoint{3.742105in}{2.829870in}}%
\pgfpathlineto{\pgfqpoint{3.796018in}{2.756294in}}%
\pgfpathlineto{\pgfqpoint{3.850763in}{2.663062in}}%
\pgfpathlineto{\pgfqpoint{3.905906in}{3.076732in}}%
\pgfpathlineto{\pgfqpoint{3.959752in}{2.825631in}}%
\pgfpathlineto{\pgfqpoint{4.014000in}{2.742718in}}%
\pgfpathlineto{\pgfqpoint{4.068363in}{2.223320in}}%
\pgfpathlineto{\pgfqpoint{4.122621in}{2.094295in}}%
\pgfpathlineto{\pgfqpoint{4.176880in}{2.284564in}}%
\pgfpathlineto{\pgfqpoint{4.230930in}{2.383296in}}%
\pgfpathlineto{\pgfqpoint{4.285264in}{2.301128in}}%
\pgfpathlineto{\pgfqpoint{4.339573in}{2.351918in}}%
\pgfpathlineto{\pgfqpoint{4.393537in}{2.267188in}}%
\pgfpathlineto{\pgfqpoint{4.447656in}{2.292747in}}%
\pgfpathlineto{\pgfqpoint{4.502359in}{2.295721in}}%
\pgfpathlineto{\pgfqpoint{4.556439in}{2.355682in}}%
\pgfpathlineto{\pgfqpoint{4.611062in}{2.298545in}}%
\pgfpathlineto{\pgfqpoint{4.665374in}{2.295511in}}%
\pgfpathlineto{\pgfqpoint{4.719555in}{2.382037in}}%
\pgfpathlineto{\pgfqpoint{4.774083in}{2.367143in}}%
\pgfpathlineto{\pgfqpoint{4.828390in}{2.382812in}}%
\pgfpathlineto{\pgfqpoint{4.882381in}{2.383572in}}%
\pgfpathlineto{\pgfqpoint{4.936806in}{2.417825in}}%
\pgfpathlineto{\pgfqpoint{4.990444in}{2.501422in}}%
\pgfpathlineto{\pgfqpoint{5.044526in}{3.559115in}}%
\pgfpathlineto{\pgfqpoint{5.098797in}{2.455009in}}%
\pgfpathlineto{\pgfqpoint{5.152893in}{1.755898in}}%
\pgfpathlineto{\pgfqpoint{5.207293in}{2.112791in}}%
\pgfpathlineto{\pgfqpoint{5.261630in}{2.334585in}}%
\pgfpathlineto{\pgfqpoint{5.316071in}{2.006089in}}%
\pgfpathlineto{\pgfqpoint{5.370174in}{2.225351in}}%
\pgfpathlineto{\pgfqpoint{5.424466in}{2.284940in}}%
\pgfpathlineto{\pgfqpoint{5.478616in}{2.262262in}}%
\pgfpathlineto{\pgfqpoint{5.532690in}{3.306622in}}%
\pgfusepath{stroke}%
\end{pgfscope}%
\begin{pgfscope}%
\pgfsetrectcap%
\pgfsetmiterjoin%
\pgfsetlinewidth{0.803000pt}%
\definecolor{currentstroke}{rgb}{0.000000,0.000000,0.000000}%
\pgfsetstrokecolor{currentstroke}%
\pgfsetdash{}{0pt}%
\pgfpathmoveto{\pgfqpoint{0.800000in}{0.528000in}}%
\pgfpathlineto{\pgfqpoint{0.800000in}{4.224000in}}%
\pgfusepath{stroke}%
\end{pgfscope}%
\begin{pgfscope}%
\pgfsetrectcap%
\pgfsetmiterjoin%
\pgfsetlinewidth{0.803000pt}%
\definecolor{currentstroke}{rgb}{0.000000,0.000000,0.000000}%
\pgfsetstrokecolor{currentstroke}%
\pgfsetdash{}{0pt}%
\pgfpathmoveto{\pgfqpoint{5.760000in}{0.528000in}}%
\pgfpathlineto{\pgfqpoint{5.760000in}{4.224000in}}%
\pgfusepath{stroke}%
\end{pgfscope}%
\begin{pgfscope}%
\pgfsetrectcap%
\pgfsetmiterjoin%
\pgfsetlinewidth{0.803000pt}%
\definecolor{currentstroke}{rgb}{0.000000,0.000000,0.000000}%
\pgfsetstrokecolor{currentstroke}%
\pgfsetdash{}{0pt}%
\pgfpathmoveto{\pgfqpoint{0.800000in}{0.528000in}}%
\pgfpathlineto{\pgfqpoint{5.760000in}{0.528000in}}%
\pgfusepath{stroke}%
\end{pgfscope}%
\begin{pgfscope}%
\pgfsetrectcap%
\pgfsetmiterjoin%
\pgfsetlinewidth{0.803000pt}%
\definecolor{currentstroke}{rgb}{0.000000,0.000000,0.000000}%
\pgfsetstrokecolor{currentstroke}%
\pgfsetdash{}{0pt}%
\pgfpathmoveto{\pgfqpoint{0.800000in}{4.224000in}}%
\pgfpathlineto{\pgfqpoint{5.760000in}{4.224000in}}%
\pgfusepath{stroke}%
\end{pgfscope}%
\begin{pgfscope}%
\definecolor{textcolor}{rgb}{0.000000,0.000000,0.000000}%
\pgfsetstrokecolor{textcolor}%
\pgfsetfillcolor{textcolor}%
\pgftext[x=3.280000in,y=4.307333in,,base]{\color{textcolor}\sffamily\fontsize{12.000000}{14.400000}\selectfont Forward controller output}%
\end{pgfscope}%
\begin{pgfscope}%
\pgfsetbuttcap%
\pgfsetmiterjoin%
\definecolor{currentfill}{rgb}{1.000000,1.000000,1.000000}%
\pgfsetfillcolor{currentfill}%
\pgfsetfillopacity{0.800000}%
\pgfsetlinewidth{1.003750pt}%
\definecolor{currentstroke}{rgb}{0.800000,0.800000,0.800000}%
\pgfsetstrokecolor{currentstroke}%
\pgfsetstrokeopacity{0.800000}%
\pgfsetdash{}{0pt}%
\pgfpathmoveto{\pgfqpoint{0.897222in}{3.093603in}}%
\pgfpathlineto{\pgfqpoint{1.518397in}{3.093603in}}%
\pgfpathquadraticcurveto{\pgfqpoint{1.546175in}{3.093603in}}{\pgfqpoint{1.546175in}{3.121381in}}%
\pgfpathlineto{\pgfqpoint{1.546175in}{4.126778in}}%
\pgfpathquadraticcurveto{\pgfqpoint{1.546175in}{4.154556in}}{\pgfqpoint{1.518397in}{4.154556in}}%
\pgfpathlineto{\pgfqpoint{0.897222in}{4.154556in}}%
\pgfpathquadraticcurveto{\pgfqpoint{0.869444in}{4.154556in}}{\pgfqpoint{0.869444in}{4.126778in}}%
\pgfpathlineto{\pgfqpoint{0.869444in}{3.121381in}}%
\pgfpathquadraticcurveto{\pgfqpoint{0.869444in}{3.093603in}}{\pgfqpoint{0.897222in}{3.093603in}}%
\pgfpathlineto{\pgfqpoint{0.897222in}{3.093603in}}%
\pgfpathclose%
\pgfusepath{stroke,fill}%
\end{pgfscope}%
\begin{pgfscope}%
\pgfsetrectcap%
\pgfsetroundjoin%
\pgfsetlinewidth{1.505625pt}%
\definecolor{currentstroke}{rgb}{0.121569,0.466667,0.705882}%
\pgfsetstrokecolor{currentstroke}%
\pgfsetdash{}{0pt}%
\pgfpathmoveto{\pgfqpoint{0.925000in}{4.042088in}}%
\pgfpathlineto{\pgfqpoint{1.063889in}{4.042088in}}%
\pgfpathlineto{\pgfqpoint{1.202778in}{4.042088in}}%
\pgfusepath{stroke}%
\end{pgfscope}%
\begin{pgfscope}%
\definecolor{textcolor}{rgb}{0.000000,0.000000,0.000000}%
\pgfsetstrokecolor{textcolor}%
\pgfsetfillcolor{textcolor}%
\pgftext[x=1.313889in,y=3.993477in,left,base]{\color{textcolor}\sffamily\fontsize{10.000000}{12.000000}\selectfont 2}%
\end{pgfscope}%
\begin{pgfscope}%
\pgfsetrectcap%
\pgfsetroundjoin%
\pgfsetlinewidth{1.505625pt}%
\definecolor{currentstroke}{rgb}{1.000000,0.498039,0.054902}%
\pgfsetstrokecolor{currentstroke}%
\pgfsetdash{}{0pt}%
\pgfpathmoveto{\pgfqpoint{0.925000in}{3.838231in}}%
\pgfpathlineto{\pgfqpoint{1.063889in}{3.838231in}}%
\pgfpathlineto{\pgfqpoint{1.202778in}{3.838231in}}%
\pgfusepath{stroke}%
\end{pgfscope}%
\begin{pgfscope}%
\definecolor{textcolor}{rgb}{0.000000,0.000000,0.000000}%
\pgfsetstrokecolor{textcolor}%
\pgfsetfillcolor{textcolor}%
\pgftext[x=1.313889in,y=3.789620in,left,base]{\color{textcolor}\sffamily\fontsize{10.000000}{12.000000}\selectfont 4}%
\end{pgfscope}%
\begin{pgfscope}%
\pgfsetrectcap%
\pgfsetroundjoin%
\pgfsetlinewidth{1.505625pt}%
\definecolor{currentstroke}{rgb}{0.172549,0.627451,0.172549}%
\pgfsetstrokecolor{currentstroke}%
\pgfsetdash{}{0pt}%
\pgfpathmoveto{\pgfqpoint{0.925000in}{3.634374in}}%
\pgfpathlineto{\pgfqpoint{1.063889in}{3.634374in}}%
\pgfpathlineto{\pgfqpoint{1.202778in}{3.634374in}}%
\pgfusepath{stroke}%
\end{pgfscope}%
\begin{pgfscope}%
\definecolor{textcolor}{rgb}{0.000000,0.000000,0.000000}%
\pgfsetstrokecolor{textcolor}%
\pgfsetfillcolor{textcolor}%
\pgftext[x=1.313889in,y=3.585762in,left,base]{\color{textcolor}\sffamily\fontsize{10.000000}{12.000000}\selectfont 6}%
\end{pgfscope}%
\begin{pgfscope}%
\pgfsetrectcap%
\pgfsetroundjoin%
\pgfsetlinewidth{1.505625pt}%
\definecolor{currentstroke}{rgb}{0.839216,0.152941,0.156863}%
\pgfsetstrokecolor{currentstroke}%
\pgfsetdash{}{0pt}%
\pgfpathmoveto{\pgfqpoint{0.925000in}{3.430516in}}%
\pgfpathlineto{\pgfqpoint{1.063889in}{3.430516in}}%
\pgfpathlineto{\pgfqpoint{1.202778in}{3.430516in}}%
\pgfusepath{stroke}%
\end{pgfscope}%
\begin{pgfscope}%
\definecolor{textcolor}{rgb}{0.000000,0.000000,0.000000}%
\pgfsetstrokecolor{textcolor}%
\pgfsetfillcolor{textcolor}%
\pgftext[x=1.313889in,y=3.381905in,left,base]{\color{textcolor}\sffamily\fontsize{10.000000}{12.000000}\selectfont 8}%
\end{pgfscope}%
\begin{pgfscope}%
\pgfsetrectcap%
\pgfsetroundjoin%
\pgfsetlinewidth{1.505625pt}%
\definecolor{currentstroke}{rgb}{0.580392,0.403922,0.741176}%
\pgfsetstrokecolor{currentstroke}%
\pgfsetdash{}{0pt}%
\pgfpathmoveto{\pgfqpoint{0.925000in}{3.226659in}}%
\pgfpathlineto{\pgfqpoint{1.063889in}{3.226659in}}%
\pgfpathlineto{\pgfqpoint{1.202778in}{3.226659in}}%
\pgfusepath{stroke}%
\end{pgfscope}%
\begin{pgfscope}%
\definecolor{textcolor}{rgb}{0.000000,0.000000,0.000000}%
\pgfsetstrokecolor{textcolor}%
\pgfsetfillcolor{textcolor}%
\pgftext[x=1.313889in,y=3.178048in,left,base]{\color{textcolor}\sffamily\fontsize{10.000000}{12.000000}\selectfont 10}%
\end{pgfscope}%
\end{pgfpicture}%
\makeatother%
\endgroup%
}
    \end{minipage}
    \caption{Variation of (a) computed error and (b) output velocity for different values of $K_{P}$ and $K_I=0$, $K_D=0$ while the forward controller is engaged.}
    \label{fig:tune-fwd-prop-io}
\end{figure}
\begin{figure}[H]
    \begin{minipage}[t]{0.5\linewidth}
        \centering
        \scalebox{0.55}{%% Creator: Matplotlib, PGF backend
%%
%% To include the figure in your LaTeX document, write
%%   \input{<filename>.pgf}
%%
%% Make sure the required packages are loaded in your preamble
%%   \usepackage{pgf}
%%
%% Also ensure that all the required font packages are loaded; for instance,
%% the lmodern package is sometimes necessary when using math font.
%%   \usepackage{lmodern}
%%
%% Figures using additional raster images can only be included by \input if
%% they are in the same directory as the main LaTeX file. For loading figures
%% from other directories you can use the `import` package
%%   \usepackage{import}
%%
%% and then include the figures with
%%   \import{<path to file>}{<filename>.pgf}
%%
%% Matplotlib used the following preamble
%%   \usepackage{fontspec}
%%   \setmainfont{DejaVuSerif.ttf}[Path=\detokenize{/home/lgonz/tfg-aero/tfg-giaa-dronecontrol/venv/lib/python3.8/site-packages/matplotlib/mpl-data/fonts/ttf/}]
%%   \setsansfont{DejaVuSans.ttf}[Path=\detokenize{/home/lgonz/tfg-aero/tfg-giaa-dronecontrol/venv/lib/python3.8/site-packages/matplotlib/mpl-data/fonts/ttf/}]
%%   \setmonofont{DejaVuSansMono.ttf}[Path=\detokenize{/home/lgonz/tfg-aero/tfg-giaa-dronecontrol/venv/lib/python3.8/site-packages/matplotlib/mpl-data/fonts/ttf/}]
%%
\begingroup%
\makeatletter%
\begin{pgfpicture}%
\pgfpathrectangle{\pgfpointorigin}{\pgfqpoint{6.400000in}{4.800000in}}%
\pgfusepath{use as bounding box, clip}%
\begin{pgfscope}%
\pgfsetbuttcap%
\pgfsetmiterjoin%
\definecolor{currentfill}{rgb}{1.000000,1.000000,1.000000}%
\pgfsetfillcolor{currentfill}%
\pgfsetlinewidth{0.000000pt}%
\definecolor{currentstroke}{rgb}{1.000000,1.000000,1.000000}%
\pgfsetstrokecolor{currentstroke}%
\pgfsetdash{}{0pt}%
\pgfpathmoveto{\pgfqpoint{0.000000in}{0.000000in}}%
\pgfpathlineto{\pgfqpoint{6.400000in}{0.000000in}}%
\pgfpathlineto{\pgfqpoint{6.400000in}{4.800000in}}%
\pgfpathlineto{\pgfqpoint{0.000000in}{4.800000in}}%
\pgfpathlineto{\pgfqpoint{0.000000in}{0.000000in}}%
\pgfpathclose%
\pgfusepath{fill}%
\end{pgfscope}%
\begin{pgfscope}%
\pgfsetbuttcap%
\pgfsetmiterjoin%
\definecolor{currentfill}{rgb}{1.000000,1.000000,1.000000}%
\pgfsetfillcolor{currentfill}%
\pgfsetlinewidth{0.000000pt}%
\definecolor{currentstroke}{rgb}{0.000000,0.000000,0.000000}%
\pgfsetstrokecolor{currentstroke}%
\pgfsetstrokeopacity{0.000000}%
\pgfsetdash{}{0pt}%
\pgfpathmoveto{\pgfqpoint{0.800000in}{0.528000in}}%
\pgfpathlineto{\pgfqpoint{5.760000in}{0.528000in}}%
\pgfpathlineto{\pgfqpoint{5.760000in}{4.224000in}}%
\pgfpathlineto{\pgfqpoint{0.800000in}{4.224000in}}%
\pgfpathlineto{\pgfqpoint{0.800000in}{0.528000in}}%
\pgfpathclose%
\pgfusepath{fill}%
\end{pgfscope}%
\begin{pgfscope}%
\pgfpathrectangle{\pgfqpoint{0.800000in}{0.528000in}}{\pgfqpoint{4.960000in}{3.696000in}}%
\pgfusepath{clip}%
\pgfsetrectcap%
\pgfsetroundjoin%
\pgfsetlinewidth{0.803000pt}%
\definecolor{currentstroke}{rgb}{0.690196,0.690196,0.690196}%
\pgfsetstrokecolor{currentstroke}%
\pgfsetdash{}{0pt}%
\pgfpathmoveto{\pgfqpoint{1.025455in}{0.528000in}}%
\pgfpathlineto{\pgfqpoint{1.025455in}{4.224000in}}%
\pgfusepath{stroke}%
\end{pgfscope}%
\begin{pgfscope}%
\pgfsetbuttcap%
\pgfsetroundjoin%
\definecolor{currentfill}{rgb}{0.000000,0.000000,0.000000}%
\pgfsetfillcolor{currentfill}%
\pgfsetlinewidth{0.803000pt}%
\definecolor{currentstroke}{rgb}{0.000000,0.000000,0.000000}%
\pgfsetstrokecolor{currentstroke}%
\pgfsetdash{}{0pt}%
\pgfsys@defobject{currentmarker}{\pgfqpoint{0.000000in}{-0.048611in}}{\pgfqpoint{0.000000in}{0.000000in}}{%
\pgfpathmoveto{\pgfqpoint{0.000000in}{0.000000in}}%
\pgfpathlineto{\pgfqpoint{0.000000in}{-0.048611in}}%
\pgfusepath{stroke,fill}%
}%
\begin{pgfscope}%
\pgfsys@transformshift{1.025455in}{0.528000in}%
\pgfsys@useobject{currentmarker}{}%
\end{pgfscope}%
\end{pgfscope}%
\begin{pgfscope}%
\definecolor{textcolor}{rgb}{0.000000,0.000000,0.000000}%
\pgfsetstrokecolor{textcolor}%
\pgfsetfillcolor{textcolor}%
\pgftext[x=1.025455in,y=0.430778in,,top]{\color{textcolor}\sffamily\fontsize{10.000000}{12.000000}\selectfont 0}%
\end{pgfscope}%
\begin{pgfscope}%
\pgfpathrectangle{\pgfqpoint{0.800000in}{0.528000in}}{\pgfqpoint{4.960000in}{3.696000in}}%
\pgfusepath{clip}%
\pgfsetrectcap%
\pgfsetroundjoin%
\pgfsetlinewidth{0.803000pt}%
\definecolor{currentstroke}{rgb}{0.690196,0.690196,0.690196}%
\pgfsetstrokecolor{currentstroke}%
\pgfsetdash{}{0pt}%
\pgfpathmoveto{\pgfqpoint{1.775840in}{0.528000in}}%
\pgfpathlineto{\pgfqpoint{1.775840in}{4.224000in}}%
\pgfusepath{stroke}%
\end{pgfscope}%
\begin{pgfscope}%
\pgfsetbuttcap%
\pgfsetroundjoin%
\definecolor{currentfill}{rgb}{0.000000,0.000000,0.000000}%
\pgfsetfillcolor{currentfill}%
\pgfsetlinewidth{0.803000pt}%
\definecolor{currentstroke}{rgb}{0.000000,0.000000,0.000000}%
\pgfsetstrokecolor{currentstroke}%
\pgfsetdash{}{0pt}%
\pgfsys@defobject{currentmarker}{\pgfqpoint{0.000000in}{-0.048611in}}{\pgfqpoint{0.000000in}{0.000000in}}{%
\pgfpathmoveto{\pgfqpoint{0.000000in}{0.000000in}}%
\pgfpathlineto{\pgfqpoint{0.000000in}{-0.048611in}}%
\pgfusepath{stroke,fill}%
}%
\begin{pgfscope}%
\pgfsys@transformshift{1.775840in}{0.528000in}%
\pgfsys@useobject{currentmarker}{}%
\end{pgfscope}%
\end{pgfscope}%
\begin{pgfscope}%
\definecolor{textcolor}{rgb}{0.000000,0.000000,0.000000}%
\pgfsetstrokecolor{textcolor}%
\pgfsetfillcolor{textcolor}%
\pgftext[x=1.775840in,y=0.430778in,,top]{\color{textcolor}\sffamily\fontsize{10.000000}{12.000000}\selectfont 5}%
\end{pgfscope}%
\begin{pgfscope}%
\pgfpathrectangle{\pgfqpoint{0.800000in}{0.528000in}}{\pgfqpoint{4.960000in}{3.696000in}}%
\pgfusepath{clip}%
\pgfsetrectcap%
\pgfsetroundjoin%
\pgfsetlinewidth{0.803000pt}%
\definecolor{currentstroke}{rgb}{0.690196,0.690196,0.690196}%
\pgfsetstrokecolor{currentstroke}%
\pgfsetdash{}{0pt}%
\pgfpathmoveto{\pgfqpoint{2.526225in}{0.528000in}}%
\pgfpathlineto{\pgfqpoint{2.526225in}{4.224000in}}%
\pgfusepath{stroke}%
\end{pgfscope}%
\begin{pgfscope}%
\pgfsetbuttcap%
\pgfsetroundjoin%
\definecolor{currentfill}{rgb}{0.000000,0.000000,0.000000}%
\pgfsetfillcolor{currentfill}%
\pgfsetlinewidth{0.803000pt}%
\definecolor{currentstroke}{rgb}{0.000000,0.000000,0.000000}%
\pgfsetstrokecolor{currentstroke}%
\pgfsetdash{}{0pt}%
\pgfsys@defobject{currentmarker}{\pgfqpoint{0.000000in}{-0.048611in}}{\pgfqpoint{0.000000in}{0.000000in}}{%
\pgfpathmoveto{\pgfqpoint{0.000000in}{0.000000in}}%
\pgfpathlineto{\pgfqpoint{0.000000in}{-0.048611in}}%
\pgfusepath{stroke,fill}%
}%
\begin{pgfscope}%
\pgfsys@transformshift{2.526225in}{0.528000in}%
\pgfsys@useobject{currentmarker}{}%
\end{pgfscope}%
\end{pgfscope}%
\begin{pgfscope}%
\definecolor{textcolor}{rgb}{0.000000,0.000000,0.000000}%
\pgfsetstrokecolor{textcolor}%
\pgfsetfillcolor{textcolor}%
\pgftext[x=2.526225in,y=0.430778in,,top]{\color{textcolor}\sffamily\fontsize{10.000000}{12.000000}\selectfont 10}%
\end{pgfscope}%
\begin{pgfscope}%
\pgfpathrectangle{\pgfqpoint{0.800000in}{0.528000in}}{\pgfqpoint{4.960000in}{3.696000in}}%
\pgfusepath{clip}%
\pgfsetrectcap%
\pgfsetroundjoin%
\pgfsetlinewidth{0.803000pt}%
\definecolor{currentstroke}{rgb}{0.690196,0.690196,0.690196}%
\pgfsetstrokecolor{currentstroke}%
\pgfsetdash{}{0pt}%
\pgfpathmoveto{\pgfqpoint{3.276611in}{0.528000in}}%
\pgfpathlineto{\pgfqpoint{3.276611in}{4.224000in}}%
\pgfusepath{stroke}%
\end{pgfscope}%
\begin{pgfscope}%
\pgfsetbuttcap%
\pgfsetroundjoin%
\definecolor{currentfill}{rgb}{0.000000,0.000000,0.000000}%
\pgfsetfillcolor{currentfill}%
\pgfsetlinewidth{0.803000pt}%
\definecolor{currentstroke}{rgb}{0.000000,0.000000,0.000000}%
\pgfsetstrokecolor{currentstroke}%
\pgfsetdash{}{0pt}%
\pgfsys@defobject{currentmarker}{\pgfqpoint{0.000000in}{-0.048611in}}{\pgfqpoint{0.000000in}{0.000000in}}{%
\pgfpathmoveto{\pgfqpoint{0.000000in}{0.000000in}}%
\pgfpathlineto{\pgfqpoint{0.000000in}{-0.048611in}}%
\pgfusepath{stroke,fill}%
}%
\begin{pgfscope}%
\pgfsys@transformshift{3.276611in}{0.528000in}%
\pgfsys@useobject{currentmarker}{}%
\end{pgfscope}%
\end{pgfscope}%
\begin{pgfscope}%
\definecolor{textcolor}{rgb}{0.000000,0.000000,0.000000}%
\pgfsetstrokecolor{textcolor}%
\pgfsetfillcolor{textcolor}%
\pgftext[x=3.276611in,y=0.430778in,,top]{\color{textcolor}\sffamily\fontsize{10.000000}{12.000000}\selectfont 15}%
\end{pgfscope}%
\begin{pgfscope}%
\pgfpathrectangle{\pgfqpoint{0.800000in}{0.528000in}}{\pgfqpoint{4.960000in}{3.696000in}}%
\pgfusepath{clip}%
\pgfsetrectcap%
\pgfsetroundjoin%
\pgfsetlinewidth{0.803000pt}%
\definecolor{currentstroke}{rgb}{0.690196,0.690196,0.690196}%
\pgfsetstrokecolor{currentstroke}%
\pgfsetdash{}{0pt}%
\pgfpathmoveto{\pgfqpoint{4.026996in}{0.528000in}}%
\pgfpathlineto{\pgfqpoint{4.026996in}{4.224000in}}%
\pgfusepath{stroke}%
\end{pgfscope}%
\begin{pgfscope}%
\pgfsetbuttcap%
\pgfsetroundjoin%
\definecolor{currentfill}{rgb}{0.000000,0.000000,0.000000}%
\pgfsetfillcolor{currentfill}%
\pgfsetlinewidth{0.803000pt}%
\definecolor{currentstroke}{rgb}{0.000000,0.000000,0.000000}%
\pgfsetstrokecolor{currentstroke}%
\pgfsetdash{}{0pt}%
\pgfsys@defobject{currentmarker}{\pgfqpoint{0.000000in}{-0.048611in}}{\pgfqpoint{0.000000in}{0.000000in}}{%
\pgfpathmoveto{\pgfqpoint{0.000000in}{0.000000in}}%
\pgfpathlineto{\pgfqpoint{0.000000in}{-0.048611in}}%
\pgfusepath{stroke,fill}%
}%
\begin{pgfscope}%
\pgfsys@transformshift{4.026996in}{0.528000in}%
\pgfsys@useobject{currentmarker}{}%
\end{pgfscope}%
\end{pgfscope}%
\begin{pgfscope}%
\definecolor{textcolor}{rgb}{0.000000,0.000000,0.000000}%
\pgfsetstrokecolor{textcolor}%
\pgfsetfillcolor{textcolor}%
\pgftext[x=4.026996in,y=0.430778in,,top]{\color{textcolor}\sffamily\fontsize{10.000000}{12.000000}\selectfont 20}%
\end{pgfscope}%
\begin{pgfscope}%
\pgfpathrectangle{\pgfqpoint{0.800000in}{0.528000in}}{\pgfqpoint{4.960000in}{3.696000in}}%
\pgfusepath{clip}%
\pgfsetrectcap%
\pgfsetroundjoin%
\pgfsetlinewidth{0.803000pt}%
\definecolor{currentstroke}{rgb}{0.690196,0.690196,0.690196}%
\pgfsetstrokecolor{currentstroke}%
\pgfsetdash{}{0pt}%
\pgfpathmoveto{\pgfqpoint{4.777382in}{0.528000in}}%
\pgfpathlineto{\pgfqpoint{4.777382in}{4.224000in}}%
\pgfusepath{stroke}%
\end{pgfscope}%
\begin{pgfscope}%
\pgfsetbuttcap%
\pgfsetroundjoin%
\definecolor{currentfill}{rgb}{0.000000,0.000000,0.000000}%
\pgfsetfillcolor{currentfill}%
\pgfsetlinewidth{0.803000pt}%
\definecolor{currentstroke}{rgb}{0.000000,0.000000,0.000000}%
\pgfsetstrokecolor{currentstroke}%
\pgfsetdash{}{0pt}%
\pgfsys@defobject{currentmarker}{\pgfqpoint{0.000000in}{-0.048611in}}{\pgfqpoint{0.000000in}{0.000000in}}{%
\pgfpathmoveto{\pgfqpoint{0.000000in}{0.000000in}}%
\pgfpathlineto{\pgfqpoint{0.000000in}{-0.048611in}}%
\pgfusepath{stroke,fill}%
}%
\begin{pgfscope}%
\pgfsys@transformshift{4.777382in}{0.528000in}%
\pgfsys@useobject{currentmarker}{}%
\end{pgfscope}%
\end{pgfscope}%
\begin{pgfscope}%
\definecolor{textcolor}{rgb}{0.000000,0.000000,0.000000}%
\pgfsetstrokecolor{textcolor}%
\pgfsetfillcolor{textcolor}%
\pgftext[x=4.777382in,y=0.430778in,,top]{\color{textcolor}\sffamily\fontsize{10.000000}{12.000000}\selectfont 25}%
\end{pgfscope}%
\begin{pgfscope}%
\pgfpathrectangle{\pgfqpoint{0.800000in}{0.528000in}}{\pgfqpoint{4.960000in}{3.696000in}}%
\pgfusepath{clip}%
\pgfsetrectcap%
\pgfsetroundjoin%
\pgfsetlinewidth{0.803000pt}%
\definecolor{currentstroke}{rgb}{0.690196,0.690196,0.690196}%
\pgfsetstrokecolor{currentstroke}%
\pgfsetdash{}{0pt}%
\pgfpathmoveto{\pgfqpoint{5.527767in}{0.528000in}}%
\pgfpathlineto{\pgfqpoint{5.527767in}{4.224000in}}%
\pgfusepath{stroke}%
\end{pgfscope}%
\begin{pgfscope}%
\pgfsetbuttcap%
\pgfsetroundjoin%
\definecolor{currentfill}{rgb}{0.000000,0.000000,0.000000}%
\pgfsetfillcolor{currentfill}%
\pgfsetlinewidth{0.803000pt}%
\definecolor{currentstroke}{rgb}{0.000000,0.000000,0.000000}%
\pgfsetstrokecolor{currentstroke}%
\pgfsetdash{}{0pt}%
\pgfsys@defobject{currentmarker}{\pgfqpoint{0.000000in}{-0.048611in}}{\pgfqpoint{0.000000in}{0.000000in}}{%
\pgfpathmoveto{\pgfqpoint{0.000000in}{0.000000in}}%
\pgfpathlineto{\pgfqpoint{0.000000in}{-0.048611in}}%
\pgfusepath{stroke,fill}%
}%
\begin{pgfscope}%
\pgfsys@transformshift{5.527767in}{0.528000in}%
\pgfsys@useobject{currentmarker}{}%
\end{pgfscope}%
\end{pgfscope}%
\begin{pgfscope}%
\definecolor{textcolor}{rgb}{0.000000,0.000000,0.000000}%
\pgfsetstrokecolor{textcolor}%
\pgfsetfillcolor{textcolor}%
\pgftext[x=5.527767in,y=0.430778in,,top]{\color{textcolor}\sffamily\fontsize{10.000000}{12.000000}\selectfont 30}%
\end{pgfscope}%
\begin{pgfscope}%
\definecolor{textcolor}{rgb}{0.000000,0.000000,0.000000}%
\pgfsetstrokecolor{textcolor}%
\pgfsetfillcolor{textcolor}%
\pgftext[x=3.280000in,y=0.240809in,,top]{\color{textcolor}\sffamily\fontsize{10.000000}{12.000000}\selectfont time [s]}%
\end{pgfscope}%
\begin{pgfscope}%
\pgfpathrectangle{\pgfqpoint{0.800000in}{0.528000in}}{\pgfqpoint{4.960000in}{3.696000in}}%
\pgfusepath{clip}%
\pgfsetrectcap%
\pgfsetroundjoin%
\pgfsetlinewidth{0.803000pt}%
\definecolor{currentstroke}{rgb}{0.690196,0.690196,0.690196}%
\pgfsetstrokecolor{currentstroke}%
\pgfsetdash{}{0pt}%
\pgfpathmoveto{\pgfqpoint{0.800000in}{0.891460in}}%
\pgfpathlineto{\pgfqpoint{5.760000in}{0.891460in}}%
\pgfusepath{stroke}%
\end{pgfscope}%
\begin{pgfscope}%
\pgfsetbuttcap%
\pgfsetroundjoin%
\definecolor{currentfill}{rgb}{0.000000,0.000000,0.000000}%
\pgfsetfillcolor{currentfill}%
\pgfsetlinewidth{0.803000pt}%
\definecolor{currentstroke}{rgb}{0.000000,0.000000,0.000000}%
\pgfsetstrokecolor{currentstroke}%
\pgfsetdash{}{0pt}%
\pgfsys@defobject{currentmarker}{\pgfqpoint{-0.048611in}{0.000000in}}{\pgfqpoint{-0.000000in}{0.000000in}}{%
\pgfpathmoveto{\pgfqpoint{-0.000000in}{0.000000in}}%
\pgfpathlineto{\pgfqpoint{-0.048611in}{0.000000in}}%
\pgfusepath{stroke,fill}%
}%
\begin{pgfscope}%
\pgfsys@transformshift{0.800000in}{0.891460in}%
\pgfsys@useobject{currentmarker}{}%
\end{pgfscope}%
\end{pgfscope}%
\begin{pgfscope}%
\definecolor{textcolor}{rgb}{0.000000,0.000000,0.000000}%
\pgfsetstrokecolor{textcolor}%
\pgfsetfillcolor{textcolor}%
\pgftext[x=0.373873in, y=0.838699in, left, base]{\color{textcolor}\sffamily\fontsize{10.000000}{12.000000}\selectfont \ensuremath{-}1.2}%
\end{pgfscope}%
\begin{pgfscope}%
\pgfpathrectangle{\pgfqpoint{0.800000in}{0.528000in}}{\pgfqpoint{4.960000in}{3.696000in}}%
\pgfusepath{clip}%
\pgfsetrectcap%
\pgfsetroundjoin%
\pgfsetlinewidth{0.803000pt}%
\definecolor{currentstroke}{rgb}{0.690196,0.690196,0.690196}%
\pgfsetstrokecolor{currentstroke}%
\pgfsetdash{}{0pt}%
\pgfpathmoveto{\pgfqpoint{0.800000in}{1.398641in}}%
\pgfpathlineto{\pgfqpoint{5.760000in}{1.398641in}}%
\pgfusepath{stroke}%
\end{pgfscope}%
\begin{pgfscope}%
\pgfsetbuttcap%
\pgfsetroundjoin%
\definecolor{currentfill}{rgb}{0.000000,0.000000,0.000000}%
\pgfsetfillcolor{currentfill}%
\pgfsetlinewidth{0.803000pt}%
\definecolor{currentstroke}{rgb}{0.000000,0.000000,0.000000}%
\pgfsetstrokecolor{currentstroke}%
\pgfsetdash{}{0pt}%
\pgfsys@defobject{currentmarker}{\pgfqpoint{-0.048611in}{0.000000in}}{\pgfqpoint{-0.000000in}{0.000000in}}{%
\pgfpathmoveto{\pgfqpoint{-0.000000in}{0.000000in}}%
\pgfpathlineto{\pgfqpoint{-0.048611in}{0.000000in}}%
\pgfusepath{stroke,fill}%
}%
\begin{pgfscope}%
\pgfsys@transformshift{0.800000in}{1.398641in}%
\pgfsys@useobject{currentmarker}{}%
\end{pgfscope}%
\end{pgfscope}%
\begin{pgfscope}%
\definecolor{textcolor}{rgb}{0.000000,0.000000,0.000000}%
\pgfsetstrokecolor{textcolor}%
\pgfsetfillcolor{textcolor}%
\pgftext[x=0.373873in, y=1.345880in, left, base]{\color{textcolor}\sffamily\fontsize{10.000000}{12.000000}\selectfont \ensuremath{-}1.0}%
\end{pgfscope}%
\begin{pgfscope}%
\pgfpathrectangle{\pgfqpoint{0.800000in}{0.528000in}}{\pgfqpoint{4.960000in}{3.696000in}}%
\pgfusepath{clip}%
\pgfsetrectcap%
\pgfsetroundjoin%
\pgfsetlinewidth{0.803000pt}%
\definecolor{currentstroke}{rgb}{0.690196,0.690196,0.690196}%
\pgfsetstrokecolor{currentstroke}%
\pgfsetdash{}{0pt}%
\pgfpathmoveto{\pgfqpoint{0.800000in}{1.905822in}}%
\pgfpathlineto{\pgfqpoint{5.760000in}{1.905822in}}%
\pgfusepath{stroke}%
\end{pgfscope}%
\begin{pgfscope}%
\pgfsetbuttcap%
\pgfsetroundjoin%
\definecolor{currentfill}{rgb}{0.000000,0.000000,0.000000}%
\pgfsetfillcolor{currentfill}%
\pgfsetlinewidth{0.803000pt}%
\definecolor{currentstroke}{rgb}{0.000000,0.000000,0.000000}%
\pgfsetstrokecolor{currentstroke}%
\pgfsetdash{}{0pt}%
\pgfsys@defobject{currentmarker}{\pgfqpoint{-0.048611in}{0.000000in}}{\pgfqpoint{-0.000000in}{0.000000in}}{%
\pgfpathmoveto{\pgfqpoint{-0.000000in}{0.000000in}}%
\pgfpathlineto{\pgfqpoint{-0.048611in}{0.000000in}}%
\pgfusepath{stroke,fill}%
}%
\begin{pgfscope}%
\pgfsys@transformshift{0.800000in}{1.905822in}%
\pgfsys@useobject{currentmarker}{}%
\end{pgfscope}%
\end{pgfscope}%
\begin{pgfscope}%
\definecolor{textcolor}{rgb}{0.000000,0.000000,0.000000}%
\pgfsetstrokecolor{textcolor}%
\pgfsetfillcolor{textcolor}%
\pgftext[x=0.373873in, y=1.853061in, left, base]{\color{textcolor}\sffamily\fontsize{10.000000}{12.000000}\selectfont \ensuremath{-}0.8}%
\end{pgfscope}%
\begin{pgfscope}%
\pgfpathrectangle{\pgfqpoint{0.800000in}{0.528000in}}{\pgfqpoint{4.960000in}{3.696000in}}%
\pgfusepath{clip}%
\pgfsetrectcap%
\pgfsetroundjoin%
\pgfsetlinewidth{0.803000pt}%
\definecolor{currentstroke}{rgb}{0.690196,0.690196,0.690196}%
\pgfsetstrokecolor{currentstroke}%
\pgfsetdash{}{0pt}%
\pgfpathmoveto{\pgfqpoint{0.800000in}{2.413003in}}%
\pgfpathlineto{\pgfqpoint{5.760000in}{2.413003in}}%
\pgfusepath{stroke}%
\end{pgfscope}%
\begin{pgfscope}%
\pgfsetbuttcap%
\pgfsetroundjoin%
\definecolor{currentfill}{rgb}{0.000000,0.000000,0.000000}%
\pgfsetfillcolor{currentfill}%
\pgfsetlinewidth{0.803000pt}%
\definecolor{currentstroke}{rgb}{0.000000,0.000000,0.000000}%
\pgfsetstrokecolor{currentstroke}%
\pgfsetdash{}{0pt}%
\pgfsys@defobject{currentmarker}{\pgfqpoint{-0.048611in}{0.000000in}}{\pgfqpoint{-0.000000in}{0.000000in}}{%
\pgfpathmoveto{\pgfqpoint{-0.000000in}{0.000000in}}%
\pgfpathlineto{\pgfqpoint{-0.048611in}{0.000000in}}%
\pgfusepath{stroke,fill}%
}%
\begin{pgfscope}%
\pgfsys@transformshift{0.800000in}{2.413003in}%
\pgfsys@useobject{currentmarker}{}%
\end{pgfscope}%
\end{pgfscope}%
\begin{pgfscope}%
\definecolor{textcolor}{rgb}{0.000000,0.000000,0.000000}%
\pgfsetstrokecolor{textcolor}%
\pgfsetfillcolor{textcolor}%
\pgftext[x=0.373873in, y=2.360242in, left, base]{\color{textcolor}\sffamily\fontsize{10.000000}{12.000000}\selectfont \ensuremath{-}0.6}%
\end{pgfscope}%
\begin{pgfscope}%
\pgfpathrectangle{\pgfqpoint{0.800000in}{0.528000in}}{\pgfqpoint{4.960000in}{3.696000in}}%
\pgfusepath{clip}%
\pgfsetrectcap%
\pgfsetroundjoin%
\pgfsetlinewidth{0.803000pt}%
\definecolor{currentstroke}{rgb}{0.690196,0.690196,0.690196}%
\pgfsetstrokecolor{currentstroke}%
\pgfsetdash{}{0pt}%
\pgfpathmoveto{\pgfqpoint{0.800000in}{2.920184in}}%
\pgfpathlineto{\pgfqpoint{5.760000in}{2.920184in}}%
\pgfusepath{stroke}%
\end{pgfscope}%
\begin{pgfscope}%
\pgfsetbuttcap%
\pgfsetroundjoin%
\definecolor{currentfill}{rgb}{0.000000,0.000000,0.000000}%
\pgfsetfillcolor{currentfill}%
\pgfsetlinewidth{0.803000pt}%
\definecolor{currentstroke}{rgb}{0.000000,0.000000,0.000000}%
\pgfsetstrokecolor{currentstroke}%
\pgfsetdash{}{0pt}%
\pgfsys@defobject{currentmarker}{\pgfqpoint{-0.048611in}{0.000000in}}{\pgfqpoint{-0.000000in}{0.000000in}}{%
\pgfpathmoveto{\pgfqpoint{-0.000000in}{0.000000in}}%
\pgfpathlineto{\pgfqpoint{-0.048611in}{0.000000in}}%
\pgfusepath{stroke,fill}%
}%
\begin{pgfscope}%
\pgfsys@transformshift{0.800000in}{2.920184in}%
\pgfsys@useobject{currentmarker}{}%
\end{pgfscope}%
\end{pgfscope}%
\begin{pgfscope}%
\definecolor{textcolor}{rgb}{0.000000,0.000000,0.000000}%
\pgfsetstrokecolor{textcolor}%
\pgfsetfillcolor{textcolor}%
\pgftext[x=0.373873in, y=2.867422in, left, base]{\color{textcolor}\sffamily\fontsize{10.000000}{12.000000}\selectfont \ensuremath{-}0.4}%
\end{pgfscope}%
\begin{pgfscope}%
\pgfpathrectangle{\pgfqpoint{0.800000in}{0.528000in}}{\pgfqpoint{4.960000in}{3.696000in}}%
\pgfusepath{clip}%
\pgfsetrectcap%
\pgfsetroundjoin%
\pgfsetlinewidth{0.803000pt}%
\definecolor{currentstroke}{rgb}{0.690196,0.690196,0.690196}%
\pgfsetstrokecolor{currentstroke}%
\pgfsetdash{}{0pt}%
\pgfpathmoveto{\pgfqpoint{0.800000in}{3.427365in}}%
\pgfpathlineto{\pgfqpoint{5.760000in}{3.427365in}}%
\pgfusepath{stroke}%
\end{pgfscope}%
\begin{pgfscope}%
\pgfsetbuttcap%
\pgfsetroundjoin%
\definecolor{currentfill}{rgb}{0.000000,0.000000,0.000000}%
\pgfsetfillcolor{currentfill}%
\pgfsetlinewidth{0.803000pt}%
\definecolor{currentstroke}{rgb}{0.000000,0.000000,0.000000}%
\pgfsetstrokecolor{currentstroke}%
\pgfsetdash{}{0pt}%
\pgfsys@defobject{currentmarker}{\pgfqpoint{-0.048611in}{0.000000in}}{\pgfqpoint{-0.000000in}{0.000000in}}{%
\pgfpathmoveto{\pgfqpoint{-0.000000in}{0.000000in}}%
\pgfpathlineto{\pgfqpoint{-0.048611in}{0.000000in}}%
\pgfusepath{stroke,fill}%
}%
\begin{pgfscope}%
\pgfsys@transformshift{0.800000in}{3.427365in}%
\pgfsys@useobject{currentmarker}{}%
\end{pgfscope}%
\end{pgfscope}%
\begin{pgfscope}%
\definecolor{textcolor}{rgb}{0.000000,0.000000,0.000000}%
\pgfsetstrokecolor{textcolor}%
\pgfsetfillcolor{textcolor}%
\pgftext[x=0.373873in, y=3.374603in, left, base]{\color{textcolor}\sffamily\fontsize{10.000000}{12.000000}\selectfont \ensuremath{-}0.2}%
\end{pgfscope}%
\begin{pgfscope}%
\pgfpathrectangle{\pgfqpoint{0.800000in}{0.528000in}}{\pgfqpoint{4.960000in}{3.696000in}}%
\pgfusepath{clip}%
\pgfsetrectcap%
\pgfsetroundjoin%
\pgfsetlinewidth{0.803000pt}%
\definecolor{currentstroke}{rgb}{0.690196,0.690196,0.690196}%
\pgfsetstrokecolor{currentstroke}%
\pgfsetdash{}{0pt}%
\pgfpathmoveto{\pgfqpoint{0.800000in}{3.934546in}}%
\pgfpathlineto{\pgfqpoint{5.760000in}{3.934546in}}%
\pgfusepath{stroke}%
\end{pgfscope}%
\begin{pgfscope}%
\pgfsetbuttcap%
\pgfsetroundjoin%
\definecolor{currentfill}{rgb}{0.000000,0.000000,0.000000}%
\pgfsetfillcolor{currentfill}%
\pgfsetlinewidth{0.803000pt}%
\definecolor{currentstroke}{rgb}{0.000000,0.000000,0.000000}%
\pgfsetstrokecolor{currentstroke}%
\pgfsetdash{}{0pt}%
\pgfsys@defobject{currentmarker}{\pgfqpoint{-0.048611in}{0.000000in}}{\pgfqpoint{-0.000000in}{0.000000in}}{%
\pgfpathmoveto{\pgfqpoint{-0.000000in}{0.000000in}}%
\pgfpathlineto{\pgfqpoint{-0.048611in}{0.000000in}}%
\pgfusepath{stroke,fill}%
}%
\begin{pgfscope}%
\pgfsys@transformshift{0.800000in}{3.934546in}%
\pgfsys@useobject{currentmarker}{}%
\end{pgfscope}%
\end{pgfscope}%
\begin{pgfscope}%
\definecolor{textcolor}{rgb}{0.000000,0.000000,0.000000}%
\pgfsetstrokecolor{textcolor}%
\pgfsetfillcolor{textcolor}%
\pgftext[x=0.481898in, y=3.881784in, left, base]{\color{textcolor}\sffamily\fontsize{10.000000}{12.000000}\selectfont 0.0}%
\end{pgfscope}%
\begin{pgfscope}%
\definecolor{textcolor}{rgb}{0.000000,0.000000,0.000000}%
\pgfsetstrokecolor{textcolor}%
\pgfsetfillcolor{textcolor}%
\pgftext[x=0.318318in,y=2.376000in,,bottom,rotate=90.000000]{\color{textcolor}\sffamily\fontsize{10.000000}{12.000000}\selectfont Forward movement [m]}%
\end{pgfscope}%
\begin{pgfscope}%
\pgfpathrectangle{\pgfqpoint{0.800000in}{0.528000in}}{\pgfqpoint{4.960000in}{3.696000in}}%
\pgfusepath{clip}%
\pgfsetrectcap%
\pgfsetroundjoin%
\pgfsetlinewidth{1.505625pt}%
\definecolor{currentstroke}{rgb}{0.121569,0.466667,0.705882}%
\pgfsetstrokecolor{currentstroke}%
\pgfsetdash{}{0pt}%
\pgfpathmoveto{\pgfqpoint{1.025455in}{4.056000in}}%
\pgfpathlineto{\pgfqpoint{1.079273in}{4.023710in}}%
\pgfpathlineto{\pgfqpoint{1.133704in}{3.877922in}}%
\pgfpathlineto{\pgfqpoint{1.188241in}{3.664573in}}%
\pgfpathlineto{\pgfqpoint{1.242959in}{3.358881in}}%
\pgfpathlineto{\pgfqpoint{1.297637in}{2.999128in}}%
\pgfpathlineto{\pgfqpoint{1.350578in}{2.646373in}}%
\pgfpathlineto{\pgfqpoint{1.404924in}{2.275286in}}%
\pgfpathlineto{\pgfqpoint{1.459872in}{1.958626in}}%
\pgfpathlineto{\pgfqpoint{1.514467in}{1.721217in}}%
\pgfpathlineto{\pgfqpoint{1.569921in}{1.560916in}}%
\pgfpathlineto{\pgfqpoint{1.623153in}{1.480048in}}%
\pgfpathlineto{\pgfqpoint{1.677358in}{1.447828in}}%
\pgfpathlineto{\pgfqpoint{1.731351in}{1.452807in}}%
\pgfpathlineto{\pgfqpoint{1.785542in}{1.482576in}}%
\pgfpathlineto{\pgfqpoint{1.839730in}{1.524713in}}%
\pgfpathlineto{\pgfqpoint{1.894065in}{1.572186in}}%
\pgfpathlineto{\pgfqpoint{1.948406in}{1.625988in}}%
\pgfpathlineto{\pgfqpoint{2.002513in}{1.680738in}}%
\pgfpathlineto{\pgfqpoint{2.056496in}{1.728744in}}%
\pgfpathlineto{\pgfqpoint{2.111032in}{1.770746in}}%
\pgfpathlineto{\pgfqpoint{2.165703in}{1.808683in}}%
\pgfpathlineto{\pgfqpoint{2.219523in}{1.843392in}}%
\pgfpathlineto{\pgfqpoint{2.273591in}{1.869551in}}%
\pgfpathlineto{\pgfqpoint{2.327939in}{1.888128in}}%
\pgfpathlineto{\pgfqpoint{2.382101in}{1.897339in}}%
\pgfpathlineto{\pgfqpoint{2.436110in}{1.899022in}}%
\pgfpathlineto{\pgfqpoint{2.490742in}{1.893959in}}%
\pgfpathlineto{\pgfqpoint{2.545161in}{1.879399in}}%
\pgfpathlineto{\pgfqpoint{2.599102in}{1.857546in}}%
\pgfpathlineto{\pgfqpoint{2.653239in}{1.826757in}}%
\pgfpathlineto{\pgfqpoint{2.708846in}{1.797730in}}%
\pgfpathlineto{\pgfqpoint{2.763044in}{1.768886in}}%
\pgfpathlineto{\pgfqpoint{2.817183in}{1.744112in}}%
\pgfpathlineto{\pgfqpoint{2.871310in}{1.722131in}}%
\pgfpathlineto{\pgfqpoint{2.925477in}{1.703075in}}%
\pgfpathlineto{\pgfqpoint{2.979597in}{1.690609in}}%
\pgfpathlineto{\pgfqpoint{3.033866in}{1.677274in}}%
\pgfpathlineto{\pgfqpoint{3.088241in}{1.665047in}}%
\pgfpathlineto{\pgfqpoint{3.142387in}{1.656547in}}%
\pgfpathlineto{\pgfqpoint{3.196681in}{1.654249in}}%
\pgfpathlineto{\pgfqpoint{3.250944in}{1.649939in}}%
\pgfpathlineto{\pgfqpoint{3.305156in}{1.645747in}}%
\pgfpathlineto{\pgfqpoint{3.360635in}{1.638947in}}%
\pgfpathlineto{\pgfqpoint{3.414213in}{1.635418in}}%
\pgfpathlineto{\pgfqpoint{3.468043in}{1.633076in}}%
\pgfpathlineto{\pgfqpoint{3.523044in}{1.630298in}}%
\pgfpathlineto{\pgfqpoint{3.577034in}{1.627920in}}%
\pgfpathlineto{\pgfqpoint{3.631704in}{1.626638in}}%
\pgfpathlineto{\pgfqpoint{3.685975in}{1.628484in}}%
\pgfpathlineto{\pgfqpoint{3.740407in}{1.631302in}}%
\pgfpathlineto{\pgfqpoint{3.794289in}{1.631547in}}%
\pgfpathlineto{\pgfqpoint{3.848486in}{1.631735in}}%
\pgfpathlineto{\pgfqpoint{3.902462in}{1.634321in}}%
\pgfpathlineto{\pgfqpoint{3.958235in}{1.636709in}}%
\pgfpathlineto{\pgfqpoint{4.011499in}{1.638095in}}%
\pgfpathlineto{\pgfqpoint{4.065422in}{1.641072in}}%
\pgfpathlineto{\pgfqpoint{4.119979in}{1.647027in}}%
\pgfpathlineto{\pgfqpoint{4.174041in}{1.651199in}}%
\pgfpathlineto{\pgfqpoint{4.228153in}{1.653224in}}%
\pgfpathlineto{\pgfqpoint{4.282055in}{1.656473in}}%
\pgfpathlineto{\pgfqpoint{4.336916in}{1.659348in}}%
\pgfpathlineto{\pgfqpoint{4.390452in}{1.659949in}}%
\pgfpathlineto{\pgfqpoint{4.444801in}{1.652893in}}%
\pgfpathlineto{\pgfqpoint{4.499110in}{1.643453in}}%
\pgfpathlineto{\pgfqpoint{4.553316in}{1.628117in}}%
\pgfpathlineto{\pgfqpoint{4.608476in}{1.610710in}}%
\pgfpathlineto{\pgfqpoint{4.662088in}{1.595489in}}%
\pgfpathlineto{\pgfqpoint{4.715954in}{1.586502in}}%
\pgfpathlineto{\pgfqpoint{4.770217in}{1.579061in}}%
\pgfpathlineto{\pgfqpoint{4.824576in}{1.576325in}}%
\pgfpathlineto{\pgfqpoint{4.878672in}{1.575183in}}%
\pgfpathlineto{\pgfqpoint{4.933642in}{1.574020in}}%
\pgfpathlineto{\pgfqpoint{4.987098in}{1.573378in}}%
\pgfpathlineto{\pgfqpoint{5.041817in}{1.573420in}}%
\pgfpathlineto{\pgfqpoint{5.096378in}{1.574586in}}%
\pgfpathlineto{\pgfqpoint{5.150047in}{1.574029in}}%
\pgfpathlineto{\pgfqpoint{5.205611in}{1.574594in}}%
\pgfpathlineto{\pgfqpoint{5.259204in}{1.580015in}}%
\pgfpathlineto{\pgfqpoint{5.314074in}{1.587280in}}%
\pgfpathlineto{\pgfqpoint{5.368521in}{1.588984in}}%
\pgfpathlineto{\pgfqpoint{5.422499in}{1.586852in}}%
\pgfpathlineto{\pgfqpoint{5.476748in}{1.586154in}}%
\pgfpathlineto{\pgfqpoint{5.531898in}{1.584185in}}%
\pgfusepath{stroke}%
\end{pgfscope}%
\begin{pgfscope}%
\pgfpathrectangle{\pgfqpoint{0.800000in}{0.528000in}}{\pgfqpoint{4.960000in}{3.696000in}}%
\pgfusepath{clip}%
\pgfsetrectcap%
\pgfsetroundjoin%
\pgfsetlinewidth{1.505625pt}%
\definecolor{currentstroke}{rgb}{1.000000,0.498039,0.054902}%
\pgfsetstrokecolor{currentstroke}%
\pgfsetdash{}{0pt}%
\pgfpathmoveto{\pgfqpoint{1.025455in}{3.855413in}}%
\pgfpathlineto{\pgfqpoint{1.079078in}{3.829820in}}%
\pgfpathlineto{\pgfqpoint{1.133304in}{3.700607in}}%
\pgfpathlineto{\pgfqpoint{1.187626in}{3.496258in}}%
\pgfpathlineto{\pgfqpoint{1.242811in}{3.213468in}}%
\pgfpathlineto{\pgfqpoint{1.297094in}{2.892889in}}%
\pgfpathlineto{\pgfqpoint{1.351331in}{2.524661in}}%
\pgfpathlineto{\pgfqpoint{1.405602in}{2.134220in}}%
\pgfpathlineto{\pgfqpoint{1.459783in}{1.794740in}}%
\pgfpathlineto{\pgfqpoint{1.514233in}{1.500598in}}%
\pgfpathlineto{\pgfqpoint{1.569789in}{1.294131in}}%
\pgfpathlineto{\pgfqpoint{1.623167in}{1.157578in}}%
\pgfpathlineto{\pgfqpoint{1.676786in}{1.077810in}}%
\pgfpathlineto{\pgfqpoint{1.731557in}{1.052818in}}%
\pgfpathlineto{\pgfqpoint{1.786339in}{1.070022in}}%
\pgfpathlineto{\pgfqpoint{1.840721in}{1.114166in}}%
\pgfpathlineto{\pgfqpoint{1.894997in}{1.184301in}}%
\pgfpathlineto{\pgfqpoint{1.949447in}{1.264000in}}%
\pgfpathlineto{\pgfqpoint{2.003494in}{1.340098in}}%
\pgfpathlineto{\pgfqpoint{2.057647in}{1.411494in}}%
\pgfpathlineto{\pgfqpoint{2.111846in}{1.470876in}}%
\pgfpathlineto{\pgfqpoint{2.167836in}{1.519357in}}%
\pgfpathlineto{\pgfqpoint{2.221397in}{1.551612in}}%
\pgfpathlineto{\pgfqpoint{2.275275in}{1.564046in}}%
\pgfpathlineto{\pgfqpoint{2.329386in}{1.561021in}}%
\pgfpathlineto{\pgfqpoint{2.383558in}{1.542273in}}%
\pgfpathlineto{\pgfqpoint{2.437599in}{1.508891in}}%
\pgfpathlineto{\pgfqpoint{2.492066in}{1.466767in}}%
\pgfpathlineto{\pgfqpoint{2.546254in}{1.422077in}}%
\pgfpathlineto{\pgfqpoint{2.600569in}{1.388560in}}%
\pgfpathlineto{\pgfqpoint{2.654829in}{1.360396in}}%
\pgfpathlineto{\pgfqpoint{2.709010in}{1.342336in}}%
\pgfpathlineto{\pgfqpoint{2.764001in}{1.327358in}}%
\pgfpathlineto{\pgfqpoint{2.817829in}{1.317362in}}%
\pgfpathlineto{\pgfqpoint{2.871719in}{1.310271in}}%
\pgfpathlineto{\pgfqpoint{2.925842in}{1.307546in}}%
\pgfpathlineto{\pgfqpoint{2.980575in}{1.305168in}}%
\pgfpathlineto{\pgfqpoint{3.034757in}{1.302406in}}%
\pgfpathlineto{\pgfqpoint{3.089131in}{1.300306in}}%
\pgfpathlineto{\pgfqpoint{3.143395in}{1.305976in}}%
\pgfpathlineto{\pgfqpoint{3.197336in}{1.315293in}}%
\pgfpathlineto{\pgfqpoint{3.251525in}{1.323689in}}%
\pgfpathlineto{\pgfqpoint{3.305852in}{1.336429in}}%
\pgfpathlineto{\pgfqpoint{3.361730in}{1.355847in}}%
\pgfpathlineto{\pgfqpoint{3.415076in}{1.375761in}}%
\pgfpathlineto{\pgfqpoint{3.469024in}{1.395301in}}%
\pgfpathlineto{\pgfqpoint{3.523121in}{1.414676in}}%
\pgfpathlineto{\pgfqpoint{3.577469in}{1.426446in}}%
\pgfpathlineto{\pgfqpoint{3.631472in}{1.431293in}}%
\pgfpathlineto{\pgfqpoint{3.686281in}{1.428835in}}%
\pgfpathlineto{\pgfqpoint{3.740754in}{1.423290in}}%
\pgfpathlineto{\pgfqpoint{3.794691in}{1.412444in}}%
\pgfpathlineto{\pgfqpoint{3.848692in}{1.396050in}}%
\pgfpathlineto{\pgfqpoint{3.904960in}{1.375996in}}%
\pgfpathlineto{\pgfqpoint{3.959033in}{1.350880in}}%
\pgfpathlineto{\pgfqpoint{4.012218in}{1.332509in}}%
\pgfpathlineto{\pgfqpoint{4.066193in}{1.330456in}}%
\pgfpathlineto{\pgfqpoint{4.120416in}{1.336205in}}%
\pgfpathlineto{\pgfqpoint{4.174555in}{1.350412in}}%
\pgfpathlineto{\pgfqpoint{4.228812in}{1.377929in}}%
\pgfpathlineto{\pgfqpoint{4.283367in}{1.410085in}}%
\pgfpathlineto{\pgfqpoint{4.337657in}{1.432547in}}%
\pgfpathlineto{\pgfqpoint{4.391641in}{1.444358in}}%
\pgfpathlineto{\pgfqpoint{4.445915in}{1.445629in}}%
\pgfpathlineto{\pgfqpoint{4.500461in}{1.446377in}}%
\pgfpathlineto{\pgfqpoint{4.556032in}{1.445618in}}%
\pgfpathlineto{\pgfqpoint{4.609325in}{1.440162in}}%
\pgfpathlineto{\pgfqpoint{4.663128in}{1.434118in}}%
\pgfpathlineto{\pgfqpoint{4.716968in}{1.425521in}}%
\pgfpathlineto{\pgfqpoint{4.771155in}{1.415263in}}%
\pgfpathlineto{\pgfqpoint{4.825515in}{1.404745in}}%
\pgfpathlineto{\pgfqpoint{4.879641in}{1.394095in}}%
\pgfpathlineto{\pgfqpoint{4.935444in}{1.385312in}}%
\pgfpathlineto{\pgfqpoint{4.989412in}{1.374605in}}%
\pgfpathlineto{\pgfqpoint{5.043430in}{1.369688in}}%
\pgfpathlineto{\pgfqpoint{5.097637in}{1.369157in}}%
\pgfpathlineto{\pgfqpoint{5.153695in}{1.363538in}}%
\pgfpathlineto{\pgfqpoint{5.206955in}{1.357637in}}%
\pgfpathlineto{\pgfqpoint{5.260652in}{1.354109in}}%
\pgfpathlineto{\pgfqpoint{5.314831in}{1.350024in}}%
\pgfpathlineto{\pgfqpoint{5.368949in}{1.343077in}}%
\pgfpathlineto{\pgfqpoint{5.423057in}{1.331291in}}%
\pgfpathlineto{\pgfqpoint{5.477650in}{1.321567in}}%
\pgfpathlineto{\pgfqpoint{5.532121in}{1.309524in}}%
\pgfusepath{stroke}%
\end{pgfscope}%
\begin{pgfscope}%
\pgfpathrectangle{\pgfqpoint{0.800000in}{0.528000in}}{\pgfqpoint{4.960000in}{3.696000in}}%
\pgfusepath{clip}%
\pgfsetrectcap%
\pgfsetroundjoin%
\pgfsetlinewidth{1.505625pt}%
\definecolor{currentstroke}{rgb}{0.172549,0.627451,0.172549}%
\pgfsetstrokecolor{currentstroke}%
\pgfsetdash{}{0pt}%
\pgfpathmoveto{\pgfqpoint{1.025455in}{3.983346in}}%
\pgfpathlineto{\pgfqpoint{1.078995in}{3.951332in}}%
\pgfpathlineto{\pgfqpoint{1.133065in}{3.818235in}}%
\pgfpathlineto{\pgfqpoint{1.187429in}{3.613696in}}%
\pgfpathlineto{\pgfqpoint{1.241803in}{3.322457in}}%
\pgfpathlineto{\pgfqpoint{1.298094in}{2.952427in}}%
\pgfpathlineto{\pgfqpoint{1.350082in}{2.592553in}}%
\pgfpathlineto{\pgfqpoint{1.404095in}{2.202416in}}%
\pgfpathlineto{\pgfqpoint{1.458600in}{1.788205in}}%
\pgfpathlineto{\pgfqpoint{1.514272in}{1.416541in}}%
\pgfpathlineto{\pgfqpoint{1.567502in}{1.136169in}}%
\pgfpathlineto{\pgfqpoint{1.621959in}{0.953521in}}%
\pgfpathlineto{\pgfqpoint{1.675980in}{0.879699in}}%
\pgfpathlineto{\pgfqpoint{1.730265in}{0.892917in}}%
\pgfpathlineto{\pgfqpoint{1.784313in}{0.980968in}}%
\pgfpathlineto{\pgfqpoint{1.838817in}{1.135507in}}%
\pgfpathlineto{\pgfqpoint{1.895334in}{1.310100in}}%
\pgfpathlineto{\pgfqpoint{1.946939in}{1.486330in}}%
\pgfpathlineto{\pgfqpoint{2.001327in}{1.680269in}}%
\pgfpathlineto{\pgfqpoint{2.055487in}{1.846081in}}%
\pgfpathlineto{\pgfqpoint{2.111692in}{1.995365in}}%
\pgfpathlineto{\pgfqpoint{2.165126in}{2.059419in}}%
\pgfpathlineto{\pgfqpoint{2.219094in}{2.026883in}}%
\pgfpathlineto{\pgfqpoint{2.273245in}{1.904557in}}%
\pgfpathlineto{\pgfqpoint{2.327784in}{1.684535in}}%
\pgfpathlineto{\pgfqpoint{2.381593in}{1.439364in}}%
\pgfpathlineto{\pgfqpoint{2.435903in}{1.259810in}}%
\pgfpathlineto{\pgfqpoint{2.490233in}{1.125725in}}%
\pgfpathlineto{\pgfqpoint{2.544968in}{1.069312in}}%
\pgfpathlineto{\pgfqpoint{2.599464in}{1.084251in}}%
\pgfpathlineto{\pgfqpoint{2.653678in}{1.147299in}}%
\pgfpathlineto{\pgfqpoint{2.709852in}{1.275195in}}%
\pgfpathlineto{\pgfqpoint{2.762915in}{1.410051in}}%
\pgfpathlineto{\pgfqpoint{2.816687in}{1.536904in}}%
\pgfpathlineto{\pgfqpoint{2.870599in}{1.653401in}}%
\pgfpathlineto{\pgfqpoint{2.924839in}{1.749273in}}%
\pgfpathlineto{\pgfqpoint{2.979277in}{1.815990in}}%
\pgfpathlineto{\pgfqpoint{3.033328in}{1.858818in}}%
\pgfpathlineto{\pgfqpoint{3.087848in}{1.877910in}}%
\pgfpathlineto{\pgfqpoint{3.141810in}{1.875168in}}%
\pgfpathlineto{\pgfqpoint{3.196201in}{1.847684in}}%
\pgfpathlineto{\pgfqpoint{3.250942in}{1.793432in}}%
\pgfpathlineto{\pgfqpoint{3.305458in}{1.721989in}}%
\pgfpathlineto{\pgfqpoint{3.361260in}{1.657182in}}%
\pgfpathlineto{\pgfqpoint{3.414539in}{1.601010in}}%
\pgfpathlineto{\pgfqpoint{3.468771in}{1.546252in}}%
\pgfpathlineto{\pgfqpoint{3.522518in}{1.500616in}}%
\pgfpathlineto{\pgfqpoint{3.576810in}{1.470465in}}%
\pgfpathlineto{\pgfqpoint{3.631051in}{1.451547in}}%
\pgfpathlineto{\pgfqpoint{3.685184in}{1.447839in}}%
\pgfpathlineto{\pgfqpoint{3.739229in}{1.448257in}}%
\pgfpathlineto{\pgfqpoint{3.793518in}{1.452560in}}%
\pgfpathlineto{\pgfqpoint{3.847915in}{1.462457in}}%
\pgfpathlineto{\pgfqpoint{3.902706in}{1.474458in}}%
\pgfpathlineto{\pgfqpoint{3.956786in}{1.490810in}}%
\pgfpathlineto{\pgfqpoint{4.012470in}{1.507614in}}%
\pgfpathlineto{\pgfqpoint{4.065851in}{1.523812in}}%
\pgfpathlineto{\pgfqpoint{4.119832in}{1.541646in}}%
\pgfpathlineto{\pgfqpoint{4.173605in}{1.557958in}}%
\pgfpathlineto{\pgfqpoint{4.227840in}{1.568720in}}%
\pgfpathlineto{\pgfqpoint{4.282617in}{1.574750in}}%
\pgfpathlineto{\pgfqpoint{4.336399in}{1.576187in}}%
\pgfpathlineto{\pgfqpoint{4.390228in}{1.577675in}}%
\pgfpathlineto{\pgfqpoint{4.444621in}{1.578325in}}%
\pgfpathlineto{\pgfqpoint{4.500778in}{1.578430in}}%
\pgfpathlineto{\pgfqpoint{4.554452in}{1.571518in}}%
\pgfpathlineto{\pgfqpoint{4.608115in}{1.560982in}}%
\pgfpathlineto{\pgfqpoint{4.662279in}{1.541981in}}%
\pgfpathlineto{\pgfqpoint{4.716283in}{1.520006in}}%
\pgfpathlineto{\pgfqpoint{4.770576in}{1.497114in}}%
\pgfpathlineto{\pgfqpoint{4.824470in}{1.468400in}}%
\pgfpathlineto{\pgfqpoint{4.879132in}{1.438455in}}%
\pgfpathlineto{\pgfqpoint{4.933561in}{1.410757in}}%
\pgfpathlineto{\pgfqpoint{4.988047in}{1.382466in}}%
\pgfpathlineto{\pgfqpoint{5.041969in}{1.358712in}}%
\pgfpathlineto{\pgfqpoint{5.096449in}{1.339672in}}%
\pgfpathlineto{\pgfqpoint{5.151561in}{1.322781in}}%
\pgfpathlineto{\pgfqpoint{5.205746in}{1.312707in}}%
\pgfpathlineto{\pgfqpoint{5.259769in}{1.309194in}}%
\pgfpathlineto{\pgfqpoint{5.314179in}{1.311585in}}%
\pgfpathlineto{\pgfqpoint{5.367952in}{1.317398in}}%
\pgfpathlineto{\pgfqpoint{5.422075in}{1.333793in}}%
\pgfpathlineto{\pgfqpoint{5.476458in}{1.358140in}}%
\pgfpathlineto{\pgfqpoint{5.530606in}{1.393435in}}%
\pgfusepath{stroke}%
\end{pgfscope}%
\begin{pgfscope}%
\pgfpathrectangle{\pgfqpoint{0.800000in}{0.528000in}}{\pgfqpoint{4.960000in}{3.696000in}}%
\pgfusepath{clip}%
\pgfsetrectcap%
\pgfsetroundjoin%
\pgfsetlinewidth{1.505625pt}%
\definecolor{currentstroke}{rgb}{0.839216,0.152941,0.156863}%
\pgfsetstrokecolor{currentstroke}%
\pgfsetdash{}{0pt}%
\pgfpathmoveto{\pgfqpoint{1.025455in}{3.877528in}}%
\pgfpathlineto{\pgfqpoint{1.079449in}{3.845411in}}%
\pgfpathlineto{\pgfqpoint{1.133671in}{3.701532in}}%
\pgfpathlineto{\pgfqpoint{1.187708in}{3.471321in}}%
\pgfpathlineto{\pgfqpoint{1.241820in}{3.191670in}}%
\pgfpathlineto{\pgfqpoint{1.297771in}{2.833605in}}%
\pgfpathlineto{\pgfqpoint{1.349036in}{2.461111in}}%
\pgfpathlineto{\pgfqpoint{1.403750in}{2.049250in}}%
\pgfpathlineto{\pgfqpoint{1.459845in}{1.644284in}}%
\pgfpathlineto{\pgfqpoint{1.513008in}{1.259017in}}%
\pgfpathlineto{\pgfqpoint{1.568951in}{0.954245in}}%
\pgfpathlineto{\pgfqpoint{1.620777in}{0.775091in}}%
\pgfpathlineto{\pgfqpoint{1.674852in}{0.696000in}}%
\pgfpathlineto{\pgfqpoint{1.729429in}{0.711794in}}%
\pgfpathlineto{\pgfqpoint{1.783499in}{0.795412in}}%
\pgfpathlineto{\pgfqpoint{1.837989in}{0.914525in}}%
\pgfpathlineto{\pgfqpoint{1.892076in}{1.050051in}}%
\pgfpathlineto{\pgfqpoint{1.946461in}{1.224717in}}%
\pgfpathlineto{\pgfqpoint{2.001040in}{1.403521in}}%
\pgfpathlineto{\pgfqpoint{2.060426in}{1.573696in}}%
\pgfpathlineto{\pgfqpoint{2.110491in}{1.652020in}}%
\pgfpathlineto{\pgfqpoint{2.163879in}{1.662452in}}%
\pgfpathlineto{\pgfqpoint{2.218970in}{1.625808in}}%
\pgfpathlineto{\pgfqpoint{2.273175in}{1.566501in}}%
\pgfpathlineto{\pgfqpoint{2.327142in}{1.504175in}}%
\pgfpathlineto{\pgfqpoint{2.381749in}{1.447297in}}%
\pgfpathlineto{\pgfqpoint{2.436005in}{1.421250in}}%
\pgfpathlineto{\pgfqpoint{2.489927in}{1.434928in}}%
\pgfpathlineto{\pgfqpoint{2.544383in}{1.465904in}}%
\pgfpathlineto{\pgfqpoint{2.598477in}{1.507824in}}%
\pgfpathlineto{\pgfqpoint{2.653470in}{1.569952in}}%
\pgfpathlineto{\pgfqpoint{2.709771in}{1.634718in}}%
\pgfpathlineto{\pgfqpoint{2.763024in}{1.679807in}}%
\pgfpathlineto{\pgfqpoint{2.817258in}{1.714289in}}%
\pgfpathlineto{\pgfqpoint{2.871287in}{1.722494in}}%
\pgfpathlineto{\pgfqpoint{2.925948in}{1.704201in}}%
\pgfpathlineto{\pgfqpoint{2.980167in}{1.651672in}}%
\pgfpathlineto{\pgfqpoint{3.034717in}{1.557851in}}%
\pgfpathlineto{\pgfqpoint{3.089282in}{1.448346in}}%
\pgfpathlineto{\pgfqpoint{3.142899in}{1.340108in}}%
\pgfpathlineto{\pgfqpoint{3.197547in}{1.247465in}}%
\pgfpathlineto{\pgfqpoint{3.251610in}{1.188115in}}%
\pgfpathlineto{\pgfqpoint{3.307448in}{1.148938in}}%
\pgfpathlineto{\pgfqpoint{3.361216in}{1.145594in}}%
\pgfpathlineto{\pgfqpoint{3.416253in}{1.169079in}}%
\pgfpathlineto{\pgfqpoint{3.470414in}{1.218600in}}%
\pgfpathlineto{\pgfqpoint{3.523818in}{1.286868in}}%
\pgfpathlineto{\pgfqpoint{3.578408in}{1.392416in}}%
\pgfpathlineto{\pgfqpoint{3.632512in}{1.499634in}}%
\pgfpathlineto{\pgfqpoint{3.687109in}{1.586618in}}%
\pgfpathlineto{\pgfqpoint{3.742489in}{1.625195in}}%
\pgfpathlineto{\pgfqpoint{3.796298in}{1.633042in}}%
\pgfpathlineto{\pgfqpoint{3.850408in}{1.623332in}}%
\pgfpathlineto{\pgfqpoint{3.906021in}{1.602368in}}%
\pgfpathlineto{\pgfqpoint{3.959208in}{1.562413in}}%
\pgfpathlineto{\pgfqpoint{4.013012in}{1.515216in}}%
\pgfpathlineto{\pgfqpoint{4.067220in}{1.459163in}}%
\pgfpathlineto{\pgfqpoint{4.121085in}{1.396632in}}%
\pgfpathlineto{\pgfqpoint{4.175647in}{1.360126in}}%
\pgfpathlineto{\pgfqpoint{4.229930in}{1.363813in}}%
\pgfpathlineto{\pgfqpoint{4.284935in}{1.376534in}}%
\pgfpathlineto{\pgfqpoint{4.338929in}{1.398346in}}%
\pgfpathlineto{\pgfqpoint{4.393155in}{1.407871in}}%
\pgfpathlineto{\pgfqpoint{4.447737in}{1.413580in}}%
\pgfpathlineto{\pgfqpoint{4.502472in}{1.419450in}}%
\pgfpathlineto{\pgfqpoint{4.557531in}{1.419424in}}%
\pgfpathlineto{\pgfqpoint{4.610978in}{1.419311in}}%
\pgfpathlineto{\pgfqpoint{4.665003in}{1.418511in}}%
\pgfpathlineto{\pgfqpoint{4.719328in}{1.411390in}}%
\pgfpathlineto{\pgfqpoint{4.773656in}{1.401412in}}%
\pgfpathlineto{\pgfqpoint{4.827662in}{1.385326in}}%
\pgfpathlineto{\pgfqpoint{4.882569in}{1.365792in}}%
\pgfpathlineto{\pgfqpoint{4.936220in}{1.339671in}}%
\pgfpathlineto{\pgfqpoint{4.990679in}{1.303999in}}%
\pgfpathlineto{\pgfqpoint{5.044968in}{1.264875in}}%
\pgfpathlineto{\pgfqpoint{5.099273in}{1.239516in}}%
\pgfpathlineto{\pgfqpoint{5.155269in}{1.252157in}}%
\pgfpathlineto{\pgfqpoint{5.208387in}{1.257578in}}%
\pgfpathlineto{\pgfqpoint{5.262080in}{1.265292in}}%
\pgfpathlineto{\pgfqpoint{5.316239in}{1.287905in}}%
\pgfpathlineto{\pgfqpoint{5.370386in}{1.344982in}}%
\pgfpathlineto{\pgfqpoint{5.425186in}{1.398606in}}%
\pgfpathlineto{\pgfqpoint{5.479363in}{1.425733in}}%
\pgfpathlineto{\pgfqpoint{5.534545in}{1.430305in}}%
\pgfusepath{stroke}%
\end{pgfscope}%
\begin{pgfscope}%
\pgfpathrectangle{\pgfqpoint{0.800000in}{0.528000in}}{\pgfqpoint{4.960000in}{3.696000in}}%
\pgfusepath{clip}%
\pgfsetrectcap%
\pgfsetroundjoin%
\pgfsetlinewidth{1.505625pt}%
\definecolor{currentstroke}{rgb}{0.580392,0.403922,0.741176}%
\pgfsetstrokecolor{currentstroke}%
\pgfsetdash{}{0pt}%
\pgfpathmoveto{\pgfqpoint{1.025455in}{3.835727in}}%
\pgfpathlineto{\pgfqpoint{1.079790in}{3.797291in}}%
\pgfpathlineto{\pgfqpoint{1.134498in}{3.666256in}}%
\pgfpathlineto{\pgfqpoint{1.188594in}{3.442836in}}%
\pgfpathlineto{\pgfqpoint{1.244278in}{3.144399in}}%
\pgfpathlineto{\pgfqpoint{1.297288in}{2.790446in}}%
\pgfpathlineto{\pgfqpoint{1.350501in}{2.433168in}}%
\pgfpathlineto{\pgfqpoint{1.404513in}{2.046765in}}%
\pgfpathlineto{\pgfqpoint{1.458652in}{1.649987in}}%
\pgfpathlineto{\pgfqpoint{1.513394in}{1.295302in}}%
\pgfpathlineto{\pgfqpoint{1.567273in}{1.043502in}}%
\pgfpathlineto{\pgfqpoint{1.621257in}{0.912405in}}%
\pgfpathlineto{\pgfqpoint{1.675608in}{0.892088in}}%
\pgfpathlineto{\pgfqpoint{1.730279in}{0.956486in}}%
\pgfpathlineto{\pgfqpoint{1.786921in}{1.111357in}}%
\pgfpathlineto{\pgfqpoint{1.840293in}{1.349171in}}%
\pgfpathlineto{\pgfqpoint{1.894141in}{1.603829in}}%
\pgfpathlineto{\pgfqpoint{1.948068in}{1.925491in}}%
\pgfpathlineto{\pgfqpoint{2.001878in}{2.155355in}}%
\pgfpathlineto{\pgfqpoint{2.056555in}{2.244176in}}%
\pgfpathlineto{\pgfqpoint{2.110643in}{2.229319in}}%
\pgfpathlineto{\pgfqpoint{2.164736in}{2.085018in}}%
\pgfpathlineto{\pgfqpoint{2.219139in}{1.835513in}}%
\pgfpathlineto{\pgfqpoint{2.272873in}{1.555970in}}%
\pgfpathlineto{\pgfqpoint{2.327797in}{1.361018in}}%
\pgfpathlineto{\pgfqpoint{2.382023in}{1.271258in}}%
\pgfpathlineto{\pgfqpoint{2.435738in}{1.243516in}}%
\pgfpathlineto{\pgfqpoint{2.490004in}{1.296752in}}%
\pgfpathlineto{\pgfqpoint{2.544230in}{1.412091in}}%
\pgfpathlineto{\pgfqpoint{2.599676in}{1.582730in}}%
\pgfpathlineto{\pgfqpoint{2.653395in}{1.771596in}}%
\pgfpathlineto{\pgfqpoint{2.708467in}{1.930717in}}%
\pgfpathlineto{\pgfqpoint{2.762948in}{1.993321in}}%
\pgfpathlineto{\pgfqpoint{2.817143in}{1.921726in}}%
\pgfpathlineto{\pgfqpoint{2.871575in}{1.750484in}}%
\pgfpathlineto{\pgfqpoint{2.925797in}{1.485511in}}%
\pgfpathlineto{\pgfqpoint{2.979903in}{1.215376in}}%
\pgfpathlineto{\pgfqpoint{3.034975in}{1.008584in}}%
\pgfpathlineto{\pgfqpoint{3.089420in}{0.872163in}}%
\pgfpathlineto{\pgfqpoint{3.144511in}{0.818757in}}%
\pgfpathlineto{\pgfqpoint{3.198501in}{0.823211in}}%
\pgfpathlineto{\pgfqpoint{3.252625in}{0.887202in}}%
\pgfpathlineto{\pgfqpoint{3.306828in}{0.996975in}}%
\pgfpathlineto{\pgfqpoint{3.361301in}{1.158720in}}%
\pgfpathlineto{\pgfqpoint{3.416921in}{1.368391in}}%
\pgfpathlineto{\pgfqpoint{3.470533in}{1.581588in}}%
\pgfpathlineto{\pgfqpoint{3.524128in}{1.712223in}}%
\pgfpathlineto{\pgfqpoint{3.578579in}{1.725137in}}%
\pgfpathlineto{\pgfqpoint{3.633289in}{1.623950in}}%
\pgfpathlineto{\pgfqpoint{3.687682in}{1.469821in}}%
\pgfpathlineto{\pgfqpoint{3.742176in}{1.283193in}}%
\pgfpathlineto{\pgfqpoint{3.796098in}{1.139089in}}%
\pgfpathlineto{\pgfqpoint{3.850810in}{1.055222in}}%
\pgfpathlineto{\pgfqpoint{3.906014in}{1.024567in}}%
\pgfpathlineto{\pgfqpoint{3.959825in}{1.038992in}}%
\pgfpathlineto{\pgfqpoint{4.014103in}{1.103751in}}%
\pgfpathlineto{\pgfqpoint{4.068466in}{1.194784in}}%
\pgfpathlineto{\pgfqpoint{4.122727in}{1.283585in}}%
\pgfpathlineto{\pgfqpoint{4.176945in}{1.352206in}}%
\pgfpathlineto{\pgfqpoint{4.231036in}{1.389545in}}%
\pgfpathlineto{\pgfqpoint{4.285328in}{1.407220in}}%
\pgfpathlineto{\pgfqpoint{4.339621in}{1.414030in}}%
\pgfpathlineto{\pgfqpoint{4.393594in}{1.413058in}}%
\pgfpathlineto{\pgfqpoint{4.447734in}{1.410400in}}%
\pgfpathlineto{\pgfqpoint{4.502459in}{1.399510in}}%
\pgfpathlineto{\pgfqpoint{4.556477in}{1.381887in}}%
\pgfpathlineto{\pgfqpoint{4.611209in}{1.365720in}}%
\pgfpathlineto{\pgfqpoint{4.665429in}{1.349418in}}%
\pgfpathlineto{\pgfqpoint{4.719602in}{1.328723in}}%
\pgfpathlineto{\pgfqpoint{4.774113in}{1.309123in}}%
\pgfpathlineto{\pgfqpoint{4.828488in}{1.292938in}}%
\pgfpathlineto{\pgfqpoint{4.882448in}{1.279844in}}%
\pgfpathlineto{\pgfqpoint{4.936854in}{1.269790in}}%
\pgfpathlineto{\pgfqpoint{4.990493in}{1.264878in}}%
\pgfpathlineto{\pgfqpoint{5.044596in}{1.263971in}}%
\pgfpathlineto{\pgfqpoint{5.098879in}{1.284959in}}%
\pgfpathlineto{\pgfqpoint{5.152934in}{1.357355in}}%
\pgfpathlineto{\pgfqpoint{5.207352in}{1.404536in}}%
\pgfpathlineto{\pgfqpoint{5.261714in}{1.411119in}}%
\pgfpathlineto{\pgfqpoint{5.316133in}{1.407708in}}%
\pgfpathlineto{\pgfqpoint{5.370212in}{1.390350in}}%
\pgfpathlineto{\pgfqpoint{5.424514in}{1.354369in}}%
\pgfpathlineto{\pgfqpoint{5.478630in}{1.310797in}}%
\pgfpathlineto{\pgfqpoint{5.532785in}{1.267284in}}%
\pgfusepath{stroke}%
\end{pgfscope}%
\begin{pgfscope}%
\pgfpathrectangle{\pgfqpoint{0.800000in}{0.528000in}}{\pgfqpoint{4.960000in}{3.696000in}}%
\pgfusepath{clip}%
\pgfsetrectcap%
\pgfsetroundjoin%
\pgfsetlinewidth{1.505625pt}%
\definecolor{currentstroke}{rgb}{0.549020,0.337255,0.294118}%
\pgfsetstrokecolor{currentstroke}%
\pgfsetdash{}{0pt}%
\pgfpathmoveto{\pgfqpoint{1.025455in}{1.398641in}}%
\pgfpathlineto{\pgfqpoint{5.527767in}{1.398641in}}%
\pgfusepath{stroke}%
\end{pgfscope}%
\begin{pgfscope}%
\pgfsetrectcap%
\pgfsetmiterjoin%
\pgfsetlinewidth{0.803000pt}%
\definecolor{currentstroke}{rgb}{0.000000,0.000000,0.000000}%
\pgfsetstrokecolor{currentstroke}%
\pgfsetdash{}{0pt}%
\pgfpathmoveto{\pgfqpoint{0.800000in}{0.528000in}}%
\pgfpathlineto{\pgfqpoint{0.800000in}{4.224000in}}%
\pgfusepath{stroke}%
\end{pgfscope}%
\begin{pgfscope}%
\pgfsetrectcap%
\pgfsetmiterjoin%
\pgfsetlinewidth{0.803000pt}%
\definecolor{currentstroke}{rgb}{0.000000,0.000000,0.000000}%
\pgfsetstrokecolor{currentstroke}%
\pgfsetdash{}{0pt}%
\pgfpathmoveto{\pgfqpoint{5.760000in}{0.528000in}}%
\pgfpathlineto{\pgfqpoint{5.760000in}{4.224000in}}%
\pgfusepath{stroke}%
\end{pgfscope}%
\begin{pgfscope}%
\pgfsetrectcap%
\pgfsetmiterjoin%
\pgfsetlinewidth{0.803000pt}%
\definecolor{currentstroke}{rgb}{0.000000,0.000000,0.000000}%
\pgfsetstrokecolor{currentstroke}%
\pgfsetdash{}{0pt}%
\pgfpathmoveto{\pgfqpoint{0.800000in}{0.528000in}}%
\pgfpathlineto{\pgfqpoint{5.760000in}{0.528000in}}%
\pgfusepath{stroke}%
\end{pgfscope}%
\begin{pgfscope}%
\pgfsetrectcap%
\pgfsetmiterjoin%
\pgfsetlinewidth{0.803000pt}%
\definecolor{currentstroke}{rgb}{0.000000,0.000000,0.000000}%
\pgfsetstrokecolor{currentstroke}%
\pgfsetdash{}{0pt}%
\pgfpathmoveto{\pgfqpoint{0.800000in}{4.224000in}}%
\pgfpathlineto{\pgfqpoint{5.760000in}{4.224000in}}%
\pgfusepath{stroke}%
\end{pgfscope}%
\begin{pgfscope}%
\definecolor{textcolor}{rgb}{0.000000,0.000000,0.000000}%
\pgfsetstrokecolor{textcolor}%
\pgfsetfillcolor{textcolor}%
\pgftext[x=3.280000in,y=4.307333in,,base]{\color{textcolor}\sffamily\fontsize{12.000000}{14.400000}\selectfont Measured forward position}%
\end{pgfscope}%
\begin{pgfscope}%
\pgfsetbuttcap%
\pgfsetmiterjoin%
\definecolor{currentfill}{rgb}{1.000000,1.000000,1.000000}%
\pgfsetfillcolor{currentfill}%
\pgfsetfillopacity{0.800000}%
\pgfsetlinewidth{1.003750pt}%
\definecolor{currentstroke}{rgb}{0.800000,0.800000,0.800000}%
\pgfsetstrokecolor{currentstroke}%
\pgfsetstrokeopacity{0.800000}%
\pgfsetdash{}{0pt}%
\pgfpathmoveto{\pgfqpoint{4.788646in}{2.889746in}}%
\pgfpathlineto{\pgfqpoint{5.662778in}{2.889746in}}%
\pgfpathquadraticcurveto{\pgfqpoint{5.690556in}{2.889746in}}{\pgfqpoint{5.690556in}{2.917523in}}%
\pgfpathlineto{\pgfqpoint{5.690556in}{4.126778in}}%
\pgfpathquadraticcurveto{\pgfqpoint{5.690556in}{4.154556in}}{\pgfqpoint{5.662778in}{4.154556in}}%
\pgfpathlineto{\pgfqpoint{4.788646in}{4.154556in}}%
\pgfpathquadraticcurveto{\pgfqpoint{4.760868in}{4.154556in}}{\pgfqpoint{4.760868in}{4.126778in}}%
\pgfpathlineto{\pgfqpoint{4.760868in}{2.917523in}}%
\pgfpathquadraticcurveto{\pgfqpoint{4.760868in}{2.889746in}}{\pgfqpoint{4.788646in}{2.889746in}}%
\pgfpathlineto{\pgfqpoint{4.788646in}{2.889746in}}%
\pgfpathclose%
\pgfusepath{stroke,fill}%
\end{pgfscope}%
\begin{pgfscope}%
\pgfsetrectcap%
\pgfsetroundjoin%
\pgfsetlinewidth{1.505625pt}%
\definecolor{currentstroke}{rgb}{0.121569,0.466667,0.705882}%
\pgfsetstrokecolor{currentstroke}%
\pgfsetdash{}{0pt}%
\pgfpathmoveto{\pgfqpoint{4.816424in}{4.042088in}}%
\pgfpathlineto{\pgfqpoint{4.955312in}{4.042088in}}%
\pgfpathlineto{\pgfqpoint{5.094201in}{4.042088in}}%
\pgfusepath{stroke}%
\end{pgfscope}%
\begin{pgfscope}%
\definecolor{textcolor}{rgb}{0.000000,0.000000,0.000000}%
\pgfsetstrokecolor{textcolor}%
\pgfsetfillcolor{textcolor}%
\pgftext[x=5.205312in,y=3.993477in,left,base]{\color{textcolor}\sffamily\fontsize{10.000000}{12.000000}\selectfont 2}%
\end{pgfscope}%
\begin{pgfscope}%
\pgfsetrectcap%
\pgfsetroundjoin%
\pgfsetlinewidth{1.505625pt}%
\definecolor{currentstroke}{rgb}{1.000000,0.498039,0.054902}%
\pgfsetstrokecolor{currentstroke}%
\pgfsetdash{}{0pt}%
\pgfpathmoveto{\pgfqpoint{4.816424in}{3.838231in}}%
\pgfpathlineto{\pgfqpoint{4.955312in}{3.838231in}}%
\pgfpathlineto{\pgfqpoint{5.094201in}{3.838231in}}%
\pgfusepath{stroke}%
\end{pgfscope}%
\begin{pgfscope}%
\definecolor{textcolor}{rgb}{0.000000,0.000000,0.000000}%
\pgfsetstrokecolor{textcolor}%
\pgfsetfillcolor{textcolor}%
\pgftext[x=5.205312in,y=3.789620in,left,base]{\color{textcolor}\sffamily\fontsize{10.000000}{12.000000}\selectfont 4}%
\end{pgfscope}%
\begin{pgfscope}%
\pgfsetrectcap%
\pgfsetroundjoin%
\pgfsetlinewidth{1.505625pt}%
\definecolor{currentstroke}{rgb}{0.172549,0.627451,0.172549}%
\pgfsetstrokecolor{currentstroke}%
\pgfsetdash{}{0pt}%
\pgfpathmoveto{\pgfqpoint{4.816424in}{3.634374in}}%
\pgfpathlineto{\pgfqpoint{4.955312in}{3.634374in}}%
\pgfpathlineto{\pgfqpoint{5.094201in}{3.634374in}}%
\pgfusepath{stroke}%
\end{pgfscope}%
\begin{pgfscope}%
\definecolor{textcolor}{rgb}{0.000000,0.000000,0.000000}%
\pgfsetstrokecolor{textcolor}%
\pgfsetfillcolor{textcolor}%
\pgftext[x=5.205312in,y=3.585762in,left,base]{\color{textcolor}\sffamily\fontsize{10.000000}{12.000000}\selectfont 6}%
\end{pgfscope}%
\begin{pgfscope}%
\pgfsetrectcap%
\pgfsetroundjoin%
\pgfsetlinewidth{1.505625pt}%
\definecolor{currentstroke}{rgb}{0.839216,0.152941,0.156863}%
\pgfsetstrokecolor{currentstroke}%
\pgfsetdash{}{0pt}%
\pgfpathmoveto{\pgfqpoint{4.816424in}{3.430516in}}%
\pgfpathlineto{\pgfqpoint{4.955312in}{3.430516in}}%
\pgfpathlineto{\pgfqpoint{5.094201in}{3.430516in}}%
\pgfusepath{stroke}%
\end{pgfscope}%
\begin{pgfscope}%
\definecolor{textcolor}{rgb}{0.000000,0.000000,0.000000}%
\pgfsetstrokecolor{textcolor}%
\pgfsetfillcolor{textcolor}%
\pgftext[x=5.205312in,y=3.381905in,left,base]{\color{textcolor}\sffamily\fontsize{10.000000}{12.000000}\selectfont 8}%
\end{pgfscope}%
\begin{pgfscope}%
\pgfsetrectcap%
\pgfsetroundjoin%
\pgfsetlinewidth{1.505625pt}%
\definecolor{currentstroke}{rgb}{0.580392,0.403922,0.741176}%
\pgfsetstrokecolor{currentstroke}%
\pgfsetdash{}{0pt}%
\pgfpathmoveto{\pgfqpoint{4.816424in}{3.226659in}}%
\pgfpathlineto{\pgfqpoint{4.955312in}{3.226659in}}%
\pgfpathlineto{\pgfqpoint{5.094201in}{3.226659in}}%
\pgfusepath{stroke}%
\end{pgfscope}%
\begin{pgfscope}%
\definecolor{textcolor}{rgb}{0.000000,0.000000,0.000000}%
\pgfsetstrokecolor{textcolor}%
\pgfsetfillcolor{textcolor}%
\pgftext[x=5.205312in,y=3.178048in,left,base]{\color{textcolor}\sffamily\fontsize{10.000000}{12.000000}\selectfont 10}%
\end{pgfscope}%
\begin{pgfscope}%
\pgfsetrectcap%
\pgfsetroundjoin%
\pgfsetlinewidth{1.505625pt}%
\definecolor{currentstroke}{rgb}{0.549020,0.337255,0.294118}%
\pgfsetstrokecolor{currentstroke}%
\pgfsetdash{}{0pt}%
\pgfpathmoveto{\pgfqpoint{4.816424in}{3.022802in}}%
\pgfpathlineto{\pgfqpoint{4.955312in}{3.022802in}}%
\pgfpathlineto{\pgfqpoint{5.094201in}{3.022802in}}%
\pgfusepath{stroke}%
\end{pgfscope}%
\begin{pgfscope}%
\definecolor{textcolor}{rgb}{0.000000,0.000000,0.000000}%
\pgfsetstrokecolor{textcolor}%
\pgfsetfillcolor{textcolor}%
\pgftext[x=5.205312in,y=2.974191in,left,base]{\color{textcolor}\sffamily\fontsize{10.000000}{12.000000}\selectfont Target}%
\end{pgfscope}%
\end{pgfpicture}%
\makeatother%
\endgroup%
}
    \end{minipage}
    \begin{minipage}[t]{0.5\linewidth}
        \centering
        \scalebox{0.55}{%% Creator: Matplotlib, PGF backend
%%
%% To include the figure in your LaTeX document, write
%%   \input{<filename>.pgf}
%%
%% Make sure the required packages are loaded in your preamble
%%   \usepackage{pgf}
%%
%% Also ensure that all the required font packages are loaded; for instance,
%% the lmodern package is sometimes necessary when using math font.
%%   \usepackage{lmodern}
%%
%% Figures using additional raster images can only be included by \input if
%% they are in the same directory as the main LaTeX file. For loading figures
%% from other directories you can use the `import` package
%%   \usepackage{import}
%%
%% and then include the figures with
%%   \import{<path to file>}{<filename>.pgf}
%%
%% Matplotlib used the following preamble
%%   \usepackage{fontspec}
%%   \setmainfont{DejaVuSerif.ttf}[Path=\detokenize{/home/lgonz/tfg-aero/tfg-giaa-dronecontrol/venv/lib/python3.8/site-packages/matplotlib/mpl-data/fonts/ttf/}]
%%   \setsansfont{DejaVuSans.ttf}[Path=\detokenize{/home/lgonz/tfg-aero/tfg-giaa-dronecontrol/venv/lib/python3.8/site-packages/matplotlib/mpl-data/fonts/ttf/}]
%%   \setmonofont{DejaVuSansMono.ttf}[Path=\detokenize{/home/lgonz/tfg-aero/tfg-giaa-dronecontrol/venv/lib/python3.8/site-packages/matplotlib/mpl-data/fonts/ttf/}]
%%
\begingroup%
\makeatletter%
\begin{pgfpicture}%
\pgfpathrectangle{\pgfpointorigin}{\pgfqpoint{6.400000in}{4.800000in}}%
\pgfusepath{use as bounding box, clip}%
\begin{pgfscope}%
\pgfsetbuttcap%
\pgfsetmiterjoin%
\definecolor{currentfill}{rgb}{1.000000,1.000000,1.000000}%
\pgfsetfillcolor{currentfill}%
\pgfsetlinewidth{0.000000pt}%
\definecolor{currentstroke}{rgb}{1.000000,1.000000,1.000000}%
\pgfsetstrokecolor{currentstroke}%
\pgfsetdash{}{0pt}%
\pgfpathmoveto{\pgfqpoint{0.000000in}{0.000000in}}%
\pgfpathlineto{\pgfqpoint{6.400000in}{0.000000in}}%
\pgfpathlineto{\pgfqpoint{6.400000in}{4.800000in}}%
\pgfpathlineto{\pgfqpoint{0.000000in}{4.800000in}}%
\pgfpathlineto{\pgfqpoint{0.000000in}{0.000000in}}%
\pgfpathclose%
\pgfusepath{fill}%
\end{pgfscope}%
\begin{pgfscope}%
\pgfsetbuttcap%
\pgfsetmiterjoin%
\definecolor{currentfill}{rgb}{1.000000,1.000000,1.000000}%
\pgfsetfillcolor{currentfill}%
\pgfsetlinewidth{0.000000pt}%
\definecolor{currentstroke}{rgb}{0.000000,0.000000,0.000000}%
\pgfsetstrokecolor{currentstroke}%
\pgfsetstrokeopacity{0.000000}%
\pgfsetdash{}{0pt}%
\pgfpathmoveto{\pgfqpoint{0.800000in}{0.528000in}}%
\pgfpathlineto{\pgfqpoint{5.760000in}{0.528000in}}%
\pgfpathlineto{\pgfqpoint{5.760000in}{4.224000in}}%
\pgfpathlineto{\pgfqpoint{0.800000in}{4.224000in}}%
\pgfpathlineto{\pgfqpoint{0.800000in}{0.528000in}}%
\pgfpathclose%
\pgfusepath{fill}%
\end{pgfscope}%
\begin{pgfscope}%
\pgfpathrectangle{\pgfqpoint{0.800000in}{0.528000in}}{\pgfqpoint{4.960000in}{3.696000in}}%
\pgfusepath{clip}%
\pgfsetrectcap%
\pgfsetroundjoin%
\pgfsetlinewidth{0.803000pt}%
\definecolor{currentstroke}{rgb}{0.690196,0.690196,0.690196}%
\pgfsetstrokecolor{currentstroke}%
\pgfsetdash{}{0pt}%
\pgfpathmoveto{\pgfqpoint{1.025455in}{0.528000in}}%
\pgfpathlineto{\pgfqpoint{1.025455in}{4.224000in}}%
\pgfusepath{stroke}%
\end{pgfscope}%
\begin{pgfscope}%
\pgfsetbuttcap%
\pgfsetroundjoin%
\definecolor{currentfill}{rgb}{0.000000,0.000000,0.000000}%
\pgfsetfillcolor{currentfill}%
\pgfsetlinewidth{0.803000pt}%
\definecolor{currentstroke}{rgb}{0.000000,0.000000,0.000000}%
\pgfsetstrokecolor{currentstroke}%
\pgfsetdash{}{0pt}%
\pgfsys@defobject{currentmarker}{\pgfqpoint{0.000000in}{-0.048611in}}{\pgfqpoint{0.000000in}{0.000000in}}{%
\pgfpathmoveto{\pgfqpoint{0.000000in}{0.000000in}}%
\pgfpathlineto{\pgfqpoint{0.000000in}{-0.048611in}}%
\pgfusepath{stroke,fill}%
}%
\begin{pgfscope}%
\pgfsys@transformshift{1.025455in}{0.528000in}%
\pgfsys@useobject{currentmarker}{}%
\end{pgfscope}%
\end{pgfscope}%
\begin{pgfscope}%
\definecolor{textcolor}{rgb}{0.000000,0.000000,0.000000}%
\pgfsetstrokecolor{textcolor}%
\pgfsetfillcolor{textcolor}%
\pgftext[x=1.025455in,y=0.430778in,,top]{\color{textcolor}\sffamily\fontsize{10.000000}{12.000000}\selectfont 0}%
\end{pgfscope}%
\begin{pgfscope}%
\pgfpathrectangle{\pgfqpoint{0.800000in}{0.528000in}}{\pgfqpoint{4.960000in}{3.696000in}}%
\pgfusepath{clip}%
\pgfsetrectcap%
\pgfsetroundjoin%
\pgfsetlinewidth{0.803000pt}%
\definecolor{currentstroke}{rgb}{0.690196,0.690196,0.690196}%
\pgfsetstrokecolor{currentstroke}%
\pgfsetdash{}{0pt}%
\pgfpathmoveto{\pgfqpoint{1.775840in}{0.528000in}}%
\pgfpathlineto{\pgfqpoint{1.775840in}{4.224000in}}%
\pgfusepath{stroke}%
\end{pgfscope}%
\begin{pgfscope}%
\pgfsetbuttcap%
\pgfsetroundjoin%
\definecolor{currentfill}{rgb}{0.000000,0.000000,0.000000}%
\pgfsetfillcolor{currentfill}%
\pgfsetlinewidth{0.803000pt}%
\definecolor{currentstroke}{rgb}{0.000000,0.000000,0.000000}%
\pgfsetstrokecolor{currentstroke}%
\pgfsetdash{}{0pt}%
\pgfsys@defobject{currentmarker}{\pgfqpoint{0.000000in}{-0.048611in}}{\pgfqpoint{0.000000in}{0.000000in}}{%
\pgfpathmoveto{\pgfqpoint{0.000000in}{0.000000in}}%
\pgfpathlineto{\pgfqpoint{0.000000in}{-0.048611in}}%
\pgfusepath{stroke,fill}%
}%
\begin{pgfscope}%
\pgfsys@transformshift{1.775840in}{0.528000in}%
\pgfsys@useobject{currentmarker}{}%
\end{pgfscope}%
\end{pgfscope}%
\begin{pgfscope}%
\definecolor{textcolor}{rgb}{0.000000,0.000000,0.000000}%
\pgfsetstrokecolor{textcolor}%
\pgfsetfillcolor{textcolor}%
\pgftext[x=1.775840in,y=0.430778in,,top]{\color{textcolor}\sffamily\fontsize{10.000000}{12.000000}\selectfont 5}%
\end{pgfscope}%
\begin{pgfscope}%
\pgfpathrectangle{\pgfqpoint{0.800000in}{0.528000in}}{\pgfqpoint{4.960000in}{3.696000in}}%
\pgfusepath{clip}%
\pgfsetrectcap%
\pgfsetroundjoin%
\pgfsetlinewidth{0.803000pt}%
\definecolor{currentstroke}{rgb}{0.690196,0.690196,0.690196}%
\pgfsetstrokecolor{currentstroke}%
\pgfsetdash{}{0pt}%
\pgfpathmoveto{\pgfqpoint{2.526225in}{0.528000in}}%
\pgfpathlineto{\pgfqpoint{2.526225in}{4.224000in}}%
\pgfusepath{stroke}%
\end{pgfscope}%
\begin{pgfscope}%
\pgfsetbuttcap%
\pgfsetroundjoin%
\definecolor{currentfill}{rgb}{0.000000,0.000000,0.000000}%
\pgfsetfillcolor{currentfill}%
\pgfsetlinewidth{0.803000pt}%
\definecolor{currentstroke}{rgb}{0.000000,0.000000,0.000000}%
\pgfsetstrokecolor{currentstroke}%
\pgfsetdash{}{0pt}%
\pgfsys@defobject{currentmarker}{\pgfqpoint{0.000000in}{-0.048611in}}{\pgfqpoint{0.000000in}{0.000000in}}{%
\pgfpathmoveto{\pgfqpoint{0.000000in}{0.000000in}}%
\pgfpathlineto{\pgfqpoint{0.000000in}{-0.048611in}}%
\pgfusepath{stroke,fill}%
}%
\begin{pgfscope}%
\pgfsys@transformshift{2.526225in}{0.528000in}%
\pgfsys@useobject{currentmarker}{}%
\end{pgfscope}%
\end{pgfscope}%
\begin{pgfscope}%
\definecolor{textcolor}{rgb}{0.000000,0.000000,0.000000}%
\pgfsetstrokecolor{textcolor}%
\pgfsetfillcolor{textcolor}%
\pgftext[x=2.526225in,y=0.430778in,,top]{\color{textcolor}\sffamily\fontsize{10.000000}{12.000000}\selectfont 10}%
\end{pgfscope}%
\begin{pgfscope}%
\pgfpathrectangle{\pgfqpoint{0.800000in}{0.528000in}}{\pgfqpoint{4.960000in}{3.696000in}}%
\pgfusepath{clip}%
\pgfsetrectcap%
\pgfsetroundjoin%
\pgfsetlinewidth{0.803000pt}%
\definecolor{currentstroke}{rgb}{0.690196,0.690196,0.690196}%
\pgfsetstrokecolor{currentstroke}%
\pgfsetdash{}{0pt}%
\pgfpathmoveto{\pgfqpoint{3.276611in}{0.528000in}}%
\pgfpathlineto{\pgfqpoint{3.276611in}{4.224000in}}%
\pgfusepath{stroke}%
\end{pgfscope}%
\begin{pgfscope}%
\pgfsetbuttcap%
\pgfsetroundjoin%
\definecolor{currentfill}{rgb}{0.000000,0.000000,0.000000}%
\pgfsetfillcolor{currentfill}%
\pgfsetlinewidth{0.803000pt}%
\definecolor{currentstroke}{rgb}{0.000000,0.000000,0.000000}%
\pgfsetstrokecolor{currentstroke}%
\pgfsetdash{}{0pt}%
\pgfsys@defobject{currentmarker}{\pgfqpoint{0.000000in}{-0.048611in}}{\pgfqpoint{0.000000in}{0.000000in}}{%
\pgfpathmoveto{\pgfqpoint{0.000000in}{0.000000in}}%
\pgfpathlineto{\pgfqpoint{0.000000in}{-0.048611in}}%
\pgfusepath{stroke,fill}%
}%
\begin{pgfscope}%
\pgfsys@transformshift{3.276611in}{0.528000in}%
\pgfsys@useobject{currentmarker}{}%
\end{pgfscope}%
\end{pgfscope}%
\begin{pgfscope}%
\definecolor{textcolor}{rgb}{0.000000,0.000000,0.000000}%
\pgfsetstrokecolor{textcolor}%
\pgfsetfillcolor{textcolor}%
\pgftext[x=3.276611in,y=0.430778in,,top]{\color{textcolor}\sffamily\fontsize{10.000000}{12.000000}\selectfont 15}%
\end{pgfscope}%
\begin{pgfscope}%
\pgfpathrectangle{\pgfqpoint{0.800000in}{0.528000in}}{\pgfqpoint{4.960000in}{3.696000in}}%
\pgfusepath{clip}%
\pgfsetrectcap%
\pgfsetroundjoin%
\pgfsetlinewidth{0.803000pt}%
\definecolor{currentstroke}{rgb}{0.690196,0.690196,0.690196}%
\pgfsetstrokecolor{currentstroke}%
\pgfsetdash{}{0pt}%
\pgfpathmoveto{\pgfqpoint{4.026996in}{0.528000in}}%
\pgfpathlineto{\pgfqpoint{4.026996in}{4.224000in}}%
\pgfusepath{stroke}%
\end{pgfscope}%
\begin{pgfscope}%
\pgfsetbuttcap%
\pgfsetroundjoin%
\definecolor{currentfill}{rgb}{0.000000,0.000000,0.000000}%
\pgfsetfillcolor{currentfill}%
\pgfsetlinewidth{0.803000pt}%
\definecolor{currentstroke}{rgb}{0.000000,0.000000,0.000000}%
\pgfsetstrokecolor{currentstroke}%
\pgfsetdash{}{0pt}%
\pgfsys@defobject{currentmarker}{\pgfqpoint{0.000000in}{-0.048611in}}{\pgfqpoint{0.000000in}{0.000000in}}{%
\pgfpathmoveto{\pgfqpoint{0.000000in}{0.000000in}}%
\pgfpathlineto{\pgfqpoint{0.000000in}{-0.048611in}}%
\pgfusepath{stroke,fill}%
}%
\begin{pgfscope}%
\pgfsys@transformshift{4.026996in}{0.528000in}%
\pgfsys@useobject{currentmarker}{}%
\end{pgfscope}%
\end{pgfscope}%
\begin{pgfscope}%
\definecolor{textcolor}{rgb}{0.000000,0.000000,0.000000}%
\pgfsetstrokecolor{textcolor}%
\pgfsetfillcolor{textcolor}%
\pgftext[x=4.026996in,y=0.430778in,,top]{\color{textcolor}\sffamily\fontsize{10.000000}{12.000000}\selectfont 20}%
\end{pgfscope}%
\begin{pgfscope}%
\pgfpathrectangle{\pgfqpoint{0.800000in}{0.528000in}}{\pgfqpoint{4.960000in}{3.696000in}}%
\pgfusepath{clip}%
\pgfsetrectcap%
\pgfsetroundjoin%
\pgfsetlinewidth{0.803000pt}%
\definecolor{currentstroke}{rgb}{0.690196,0.690196,0.690196}%
\pgfsetstrokecolor{currentstroke}%
\pgfsetdash{}{0pt}%
\pgfpathmoveto{\pgfqpoint{4.777382in}{0.528000in}}%
\pgfpathlineto{\pgfqpoint{4.777382in}{4.224000in}}%
\pgfusepath{stroke}%
\end{pgfscope}%
\begin{pgfscope}%
\pgfsetbuttcap%
\pgfsetroundjoin%
\definecolor{currentfill}{rgb}{0.000000,0.000000,0.000000}%
\pgfsetfillcolor{currentfill}%
\pgfsetlinewidth{0.803000pt}%
\definecolor{currentstroke}{rgb}{0.000000,0.000000,0.000000}%
\pgfsetstrokecolor{currentstroke}%
\pgfsetdash{}{0pt}%
\pgfsys@defobject{currentmarker}{\pgfqpoint{0.000000in}{-0.048611in}}{\pgfqpoint{0.000000in}{0.000000in}}{%
\pgfpathmoveto{\pgfqpoint{0.000000in}{0.000000in}}%
\pgfpathlineto{\pgfqpoint{0.000000in}{-0.048611in}}%
\pgfusepath{stroke,fill}%
}%
\begin{pgfscope}%
\pgfsys@transformshift{4.777382in}{0.528000in}%
\pgfsys@useobject{currentmarker}{}%
\end{pgfscope}%
\end{pgfscope}%
\begin{pgfscope}%
\definecolor{textcolor}{rgb}{0.000000,0.000000,0.000000}%
\pgfsetstrokecolor{textcolor}%
\pgfsetfillcolor{textcolor}%
\pgftext[x=4.777382in,y=0.430778in,,top]{\color{textcolor}\sffamily\fontsize{10.000000}{12.000000}\selectfont 25}%
\end{pgfscope}%
\begin{pgfscope}%
\pgfpathrectangle{\pgfqpoint{0.800000in}{0.528000in}}{\pgfqpoint{4.960000in}{3.696000in}}%
\pgfusepath{clip}%
\pgfsetrectcap%
\pgfsetroundjoin%
\pgfsetlinewidth{0.803000pt}%
\definecolor{currentstroke}{rgb}{0.690196,0.690196,0.690196}%
\pgfsetstrokecolor{currentstroke}%
\pgfsetdash{}{0pt}%
\pgfpathmoveto{\pgfqpoint{5.527767in}{0.528000in}}%
\pgfpathlineto{\pgfqpoint{5.527767in}{4.224000in}}%
\pgfusepath{stroke}%
\end{pgfscope}%
\begin{pgfscope}%
\pgfsetbuttcap%
\pgfsetroundjoin%
\definecolor{currentfill}{rgb}{0.000000,0.000000,0.000000}%
\pgfsetfillcolor{currentfill}%
\pgfsetlinewidth{0.803000pt}%
\definecolor{currentstroke}{rgb}{0.000000,0.000000,0.000000}%
\pgfsetstrokecolor{currentstroke}%
\pgfsetdash{}{0pt}%
\pgfsys@defobject{currentmarker}{\pgfqpoint{0.000000in}{-0.048611in}}{\pgfqpoint{0.000000in}{0.000000in}}{%
\pgfpathmoveto{\pgfqpoint{0.000000in}{0.000000in}}%
\pgfpathlineto{\pgfqpoint{0.000000in}{-0.048611in}}%
\pgfusepath{stroke,fill}%
}%
\begin{pgfscope}%
\pgfsys@transformshift{5.527767in}{0.528000in}%
\pgfsys@useobject{currentmarker}{}%
\end{pgfscope}%
\end{pgfscope}%
\begin{pgfscope}%
\definecolor{textcolor}{rgb}{0.000000,0.000000,0.000000}%
\pgfsetstrokecolor{textcolor}%
\pgfsetfillcolor{textcolor}%
\pgftext[x=5.527767in,y=0.430778in,,top]{\color{textcolor}\sffamily\fontsize{10.000000}{12.000000}\selectfont 30}%
\end{pgfscope}%
\begin{pgfscope}%
\definecolor{textcolor}{rgb}{0.000000,0.000000,0.000000}%
\pgfsetstrokecolor{textcolor}%
\pgfsetfillcolor{textcolor}%
\pgftext[x=3.280000in,y=0.240809in,,top]{\color{textcolor}\sffamily\fontsize{10.000000}{12.000000}\selectfont time [s]}%
\end{pgfscope}%
\begin{pgfscope}%
\pgfpathrectangle{\pgfqpoint{0.800000in}{0.528000in}}{\pgfqpoint{4.960000in}{3.696000in}}%
\pgfusepath{clip}%
\pgfsetrectcap%
\pgfsetroundjoin%
\pgfsetlinewidth{0.803000pt}%
\definecolor{currentstroke}{rgb}{0.690196,0.690196,0.690196}%
\pgfsetstrokecolor{currentstroke}%
\pgfsetdash{}{0pt}%
\pgfpathmoveto{\pgfqpoint{0.800000in}{0.696000in}}%
\pgfpathlineto{\pgfqpoint{5.760000in}{0.696000in}}%
\pgfusepath{stroke}%
\end{pgfscope}%
\begin{pgfscope}%
\pgfsetbuttcap%
\pgfsetroundjoin%
\definecolor{currentfill}{rgb}{0.000000,0.000000,0.000000}%
\pgfsetfillcolor{currentfill}%
\pgfsetlinewidth{0.803000pt}%
\definecolor{currentstroke}{rgb}{0.000000,0.000000,0.000000}%
\pgfsetstrokecolor{currentstroke}%
\pgfsetdash{}{0pt}%
\pgfsys@defobject{currentmarker}{\pgfqpoint{-0.048611in}{0.000000in}}{\pgfqpoint{-0.000000in}{0.000000in}}{%
\pgfpathmoveto{\pgfqpoint{-0.000000in}{0.000000in}}%
\pgfpathlineto{\pgfqpoint{-0.048611in}{0.000000in}}%
\pgfusepath{stroke,fill}%
}%
\begin{pgfscope}%
\pgfsys@transformshift{0.800000in}{0.696000in}%
\pgfsys@useobject{currentmarker}{}%
\end{pgfscope}%
\end{pgfscope}%
\begin{pgfscope}%
\definecolor{textcolor}{rgb}{0.000000,0.000000,0.000000}%
\pgfsetstrokecolor{textcolor}%
\pgfsetfillcolor{textcolor}%
\pgftext[x=0.481898in, y=0.643238in, left, base]{\color{textcolor}\sffamily\fontsize{10.000000}{12.000000}\selectfont 0.0}%
\end{pgfscope}%
\begin{pgfscope}%
\pgfpathrectangle{\pgfqpoint{0.800000in}{0.528000in}}{\pgfqpoint{4.960000in}{3.696000in}}%
\pgfusepath{clip}%
\pgfsetrectcap%
\pgfsetroundjoin%
\pgfsetlinewidth{0.803000pt}%
\definecolor{currentstroke}{rgb}{0.690196,0.690196,0.690196}%
\pgfsetstrokecolor{currentstroke}%
\pgfsetdash{}{0pt}%
\pgfpathmoveto{\pgfqpoint{0.800000in}{1.438100in}}%
\pgfpathlineto{\pgfqpoint{5.760000in}{1.438100in}}%
\pgfusepath{stroke}%
\end{pgfscope}%
\begin{pgfscope}%
\pgfsetbuttcap%
\pgfsetroundjoin%
\definecolor{currentfill}{rgb}{0.000000,0.000000,0.000000}%
\pgfsetfillcolor{currentfill}%
\pgfsetlinewidth{0.803000pt}%
\definecolor{currentstroke}{rgb}{0.000000,0.000000,0.000000}%
\pgfsetstrokecolor{currentstroke}%
\pgfsetdash{}{0pt}%
\pgfsys@defobject{currentmarker}{\pgfqpoint{-0.048611in}{0.000000in}}{\pgfqpoint{-0.000000in}{0.000000in}}{%
\pgfpathmoveto{\pgfqpoint{-0.000000in}{0.000000in}}%
\pgfpathlineto{\pgfqpoint{-0.048611in}{0.000000in}}%
\pgfusepath{stroke,fill}%
}%
\begin{pgfscope}%
\pgfsys@transformshift{0.800000in}{1.438100in}%
\pgfsys@useobject{currentmarker}{}%
\end{pgfscope}%
\end{pgfscope}%
\begin{pgfscope}%
\definecolor{textcolor}{rgb}{0.000000,0.000000,0.000000}%
\pgfsetstrokecolor{textcolor}%
\pgfsetfillcolor{textcolor}%
\pgftext[x=0.481898in, y=1.385338in, left, base]{\color{textcolor}\sffamily\fontsize{10.000000}{12.000000}\selectfont 0.1}%
\end{pgfscope}%
\begin{pgfscope}%
\pgfpathrectangle{\pgfqpoint{0.800000in}{0.528000in}}{\pgfqpoint{4.960000in}{3.696000in}}%
\pgfusepath{clip}%
\pgfsetrectcap%
\pgfsetroundjoin%
\pgfsetlinewidth{0.803000pt}%
\definecolor{currentstroke}{rgb}{0.690196,0.690196,0.690196}%
\pgfsetstrokecolor{currentstroke}%
\pgfsetdash{}{0pt}%
\pgfpathmoveto{\pgfqpoint{0.800000in}{2.180200in}}%
\pgfpathlineto{\pgfqpoint{5.760000in}{2.180200in}}%
\pgfusepath{stroke}%
\end{pgfscope}%
\begin{pgfscope}%
\pgfsetbuttcap%
\pgfsetroundjoin%
\definecolor{currentfill}{rgb}{0.000000,0.000000,0.000000}%
\pgfsetfillcolor{currentfill}%
\pgfsetlinewidth{0.803000pt}%
\definecolor{currentstroke}{rgb}{0.000000,0.000000,0.000000}%
\pgfsetstrokecolor{currentstroke}%
\pgfsetdash{}{0pt}%
\pgfsys@defobject{currentmarker}{\pgfqpoint{-0.048611in}{0.000000in}}{\pgfqpoint{-0.000000in}{0.000000in}}{%
\pgfpathmoveto{\pgfqpoint{-0.000000in}{0.000000in}}%
\pgfpathlineto{\pgfqpoint{-0.048611in}{0.000000in}}%
\pgfusepath{stroke,fill}%
}%
\begin{pgfscope}%
\pgfsys@transformshift{0.800000in}{2.180200in}%
\pgfsys@useobject{currentmarker}{}%
\end{pgfscope}%
\end{pgfscope}%
\begin{pgfscope}%
\definecolor{textcolor}{rgb}{0.000000,0.000000,0.000000}%
\pgfsetstrokecolor{textcolor}%
\pgfsetfillcolor{textcolor}%
\pgftext[x=0.481898in, y=2.127438in, left, base]{\color{textcolor}\sffamily\fontsize{10.000000}{12.000000}\selectfont 0.2}%
\end{pgfscope}%
\begin{pgfscope}%
\pgfpathrectangle{\pgfqpoint{0.800000in}{0.528000in}}{\pgfqpoint{4.960000in}{3.696000in}}%
\pgfusepath{clip}%
\pgfsetrectcap%
\pgfsetroundjoin%
\pgfsetlinewidth{0.803000pt}%
\definecolor{currentstroke}{rgb}{0.690196,0.690196,0.690196}%
\pgfsetstrokecolor{currentstroke}%
\pgfsetdash{}{0pt}%
\pgfpathmoveto{\pgfqpoint{0.800000in}{2.922300in}}%
\pgfpathlineto{\pgfqpoint{5.760000in}{2.922300in}}%
\pgfusepath{stroke}%
\end{pgfscope}%
\begin{pgfscope}%
\pgfsetbuttcap%
\pgfsetroundjoin%
\definecolor{currentfill}{rgb}{0.000000,0.000000,0.000000}%
\pgfsetfillcolor{currentfill}%
\pgfsetlinewidth{0.803000pt}%
\definecolor{currentstroke}{rgb}{0.000000,0.000000,0.000000}%
\pgfsetstrokecolor{currentstroke}%
\pgfsetdash{}{0pt}%
\pgfsys@defobject{currentmarker}{\pgfqpoint{-0.048611in}{0.000000in}}{\pgfqpoint{-0.000000in}{0.000000in}}{%
\pgfpathmoveto{\pgfqpoint{-0.000000in}{0.000000in}}%
\pgfpathlineto{\pgfqpoint{-0.048611in}{0.000000in}}%
\pgfusepath{stroke,fill}%
}%
\begin{pgfscope}%
\pgfsys@transformshift{0.800000in}{2.922300in}%
\pgfsys@useobject{currentmarker}{}%
\end{pgfscope}%
\end{pgfscope}%
\begin{pgfscope}%
\definecolor{textcolor}{rgb}{0.000000,0.000000,0.000000}%
\pgfsetstrokecolor{textcolor}%
\pgfsetfillcolor{textcolor}%
\pgftext[x=0.481898in, y=2.869538in, left, base]{\color{textcolor}\sffamily\fontsize{10.000000}{12.000000}\selectfont 0.3}%
\end{pgfscope}%
\begin{pgfscope}%
\pgfpathrectangle{\pgfqpoint{0.800000in}{0.528000in}}{\pgfqpoint{4.960000in}{3.696000in}}%
\pgfusepath{clip}%
\pgfsetrectcap%
\pgfsetroundjoin%
\pgfsetlinewidth{0.803000pt}%
\definecolor{currentstroke}{rgb}{0.690196,0.690196,0.690196}%
\pgfsetstrokecolor{currentstroke}%
\pgfsetdash{}{0pt}%
\pgfpathmoveto{\pgfqpoint{0.800000in}{3.664400in}}%
\pgfpathlineto{\pgfqpoint{5.760000in}{3.664400in}}%
\pgfusepath{stroke}%
\end{pgfscope}%
\begin{pgfscope}%
\pgfsetbuttcap%
\pgfsetroundjoin%
\definecolor{currentfill}{rgb}{0.000000,0.000000,0.000000}%
\pgfsetfillcolor{currentfill}%
\pgfsetlinewidth{0.803000pt}%
\definecolor{currentstroke}{rgb}{0.000000,0.000000,0.000000}%
\pgfsetstrokecolor{currentstroke}%
\pgfsetdash{}{0pt}%
\pgfsys@defobject{currentmarker}{\pgfqpoint{-0.048611in}{0.000000in}}{\pgfqpoint{-0.000000in}{0.000000in}}{%
\pgfpathmoveto{\pgfqpoint{-0.000000in}{0.000000in}}%
\pgfpathlineto{\pgfqpoint{-0.048611in}{0.000000in}}%
\pgfusepath{stroke,fill}%
}%
\begin{pgfscope}%
\pgfsys@transformshift{0.800000in}{3.664400in}%
\pgfsys@useobject{currentmarker}{}%
\end{pgfscope}%
\end{pgfscope}%
\begin{pgfscope}%
\definecolor{textcolor}{rgb}{0.000000,0.000000,0.000000}%
\pgfsetstrokecolor{textcolor}%
\pgfsetfillcolor{textcolor}%
\pgftext[x=0.481898in, y=3.611638in, left, base]{\color{textcolor}\sffamily\fontsize{10.000000}{12.000000}\selectfont 0.4}%
\end{pgfscope}%
\begin{pgfscope}%
\definecolor{textcolor}{rgb}{0.000000,0.000000,0.000000}%
\pgfsetstrokecolor{textcolor}%
\pgfsetfillcolor{textcolor}%
\pgftext[x=0.426343in,y=2.376000in,,bottom,rotate=90.000000]{\color{textcolor}\sffamily\fontsize{10.000000}{12.000000}\selectfont Velocity [m/s]}%
\end{pgfscope}%
\begin{pgfscope}%
\pgfpathrectangle{\pgfqpoint{0.800000in}{0.528000in}}{\pgfqpoint{4.960000in}{3.696000in}}%
\pgfusepath{clip}%
\pgfsetrectcap%
\pgfsetroundjoin%
\pgfsetlinewidth{1.505625pt}%
\definecolor{currentstroke}{rgb}{0.121569,0.466667,0.705882}%
\pgfsetstrokecolor{currentstroke}%
\pgfsetdash{}{0pt}%
\pgfpathmoveto{\pgfqpoint{1.025455in}{0.696000in}}%
\pgfpathlineto{\pgfqpoint{1.079273in}{1.745488in}}%
\pgfpathlineto{\pgfqpoint{1.133704in}{2.638261in}}%
\pgfpathlineto{\pgfqpoint{1.188241in}{3.155028in}}%
\pgfpathlineto{\pgfqpoint{1.242959in}{3.524754in}}%
\pgfpathlineto{\pgfqpoint{1.297637in}{3.746744in}}%
\pgfpathlineto{\pgfqpoint{1.350578in}{3.672736in}}%
\pgfpathlineto{\pgfqpoint{1.404924in}{3.228942in}}%
\pgfpathlineto{\pgfqpoint{1.459872in}{2.557177in}}%
\pgfpathlineto{\pgfqpoint{1.514467in}{1.966270in}}%
\pgfpathlineto{\pgfqpoint{1.569921in}{1.452796in}}%
\pgfpathlineto{\pgfqpoint{1.623153in}{1.095633in}}%
\pgfpathlineto{\pgfqpoint{1.677358in}{0.861939in}}%
\pgfpathlineto{\pgfqpoint{1.731351in}{0.861939in}}%
\pgfpathlineto{\pgfqpoint{1.785542in}{0.861939in}}%
\pgfpathlineto{\pgfqpoint{1.839730in}{0.861939in}}%
\pgfpathlineto{\pgfqpoint{1.894065in}{0.918630in}}%
\pgfpathlineto{\pgfqpoint{1.948406in}{0.918630in}}%
\pgfpathlineto{\pgfqpoint{2.002513in}{0.992840in}}%
\pgfpathlineto{\pgfqpoint{2.056496in}{0.918630in}}%
\pgfpathlineto{\pgfqpoint{2.111032in}{0.918630in}}%
\pgfpathlineto{\pgfqpoint{2.165703in}{0.844420in}}%
\pgfpathlineto{\pgfqpoint{2.219523in}{0.844420in}}%
\pgfpathlineto{\pgfqpoint{2.273591in}{0.770210in}}%
\pgfpathlineto{\pgfqpoint{2.327939in}{0.770210in}}%
\pgfpathlineto{\pgfqpoint{2.382101in}{0.770210in}}%
\pgfpathlineto{\pgfqpoint{2.436110in}{0.770210in}}%
\pgfpathlineto{\pgfqpoint{2.490742in}{0.770210in}}%
\pgfpathlineto{\pgfqpoint{2.545161in}{0.844420in}}%
\pgfpathlineto{\pgfqpoint{2.599102in}{0.844420in}}%
\pgfpathlineto{\pgfqpoint{2.653239in}{0.844420in}}%
\pgfpathlineto{\pgfqpoint{2.708846in}{0.770210in}}%
\pgfpathlineto{\pgfqpoint{2.763044in}{0.770210in}}%
\pgfpathlineto{\pgfqpoint{2.817183in}{0.770210in}}%
\pgfpathlineto{\pgfqpoint{2.871310in}{0.770210in}}%
\pgfpathlineto{\pgfqpoint{2.925477in}{0.696000in}}%
\pgfpathlineto{\pgfqpoint{2.979597in}{0.696000in}}%
\pgfpathlineto{\pgfqpoint{3.033866in}{0.696000in}}%
\pgfpathlineto{\pgfqpoint{3.088241in}{0.696000in}}%
\pgfpathlineto{\pgfqpoint{3.142387in}{0.696000in}}%
\pgfpathlineto{\pgfqpoint{3.196681in}{0.696000in}}%
\pgfpathlineto{\pgfqpoint{3.250944in}{0.696000in}}%
\pgfpathlineto{\pgfqpoint{3.305156in}{0.696000in}}%
\pgfpathlineto{\pgfqpoint{3.360635in}{0.696000in}}%
\pgfpathlineto{\pgfqpoint{3.414213in}{0.696000in}}%
\pgfpathlineto{\pgfqpoint{3.468043in}{0.696000in}}%
\pgfpathlineto{\pgfqpoint{3.523044in}{0.696000in}}%
\pgfpathlineto{\pgfqpoint{3.577034in}{0.696000in}}%
\pgfpathlineto{\pgfqpoint{3.631704in}{0.696000in}}%
\pgfpathlineto{\pgfqpoint{3.685975in}{0.696000in}}%
\pgfpathlineto{\pgfqpoint{3.740407in}{0.696000in}}%
\pgfpathlineto{\pgfqpoint{3.794289in}{0.696000in}}%
\pgfpathlineto{\pgfqpoint{3.848486in}{0.696000in}}%
\pgfpathlineto{\pgfqpoint{3.902462in}{0.696000in}}%
\pgfpathlineto{\pgfqpoint{3.958235in}{0.696000in}}%
\pgfpathlineto{\pgfqpoint{4.011499in}{0.696000in}}%
\pgfpathlineto{\pgfqpoint{4.065422in}{0.696000in}}%
\pgfpathlineto{\pgfqpoint{4.119979in}{0.696000in}}%
\pgfpathlineto{\pgfqpoint{4.174041in}{0.696000in}}%
\pgfpathlineto{\pgfqpoint{4.228153in}{0.696000in}}%
\pgfpathlineto{\pgfqpoint{4.282055in}{0.696000in}}%
\pgfpathlineto{\pgfqpoint{4.336916in}{0.696000in}}%
\pgfpathlineto{\pgfqpoint{4.390452in}{0.696000in}}%
\pgfpathlineto{\pgfqpoint{4.444801in}{0.770210in}}%
\pgfpathlineto{\pgfqpoint{4.499110in}{0.770210in}}%
\pgfpathlineto{\pgfqpoint{4.553316in}{0.770210in}}%
\pgfpathlineto{\pgfqpoint{4.608476in}{0.770210in}}%
\pgfpathlineto{\pgfqpoint{4.662088in}{0.696000in}}%
\pgfpathlineto{\pgfqpoint{4.715954in}{0.696000in}}%
\pgfpathlineto{\pgfqpoint{4.770217in}{0.696000in}}%
\pgfpathlineto{\pgfqpoint{4.824576in}{0.696000in}}%
\pgfpathlineto{\pgfqpoint{4.878672in}{0.696000in}}%
\pgfpathlineto{\pgfqpoint{4.933642in}{0.696000in}}%
\pgfpathlineto{\pgfqpoint{4.987098in}{0.696000in}}%
\pgfpathlineto{\pgfqpoint{5.041817in}{0.696000in}}%
\pgfpathlineto{\pgfqpoint{5.096378in}{0.696000in}}%
\pgfpathlineto{\pgfqpoint{5.150047in}{0.696000in}}%
\pgfpathlineto{\pgfqpoint{5.205611in}{0.696000in}}%
\pgfpathlineto{\pgfqpoint{5.259204in}{0.696000in}}%
\pgfpathlineto{\pgfqpoint{5.314074in}{0.696000in}}%
\pgfpathlineto{\pgfqpoint{5.368521in}{0.696000in}}%
\pgfpathlineto{\pgfqpoint{5.422499in}{0.696000in}}%
\pgfpathlineto{\pgfqpoint{5.476748in}{0.696000in}}%
\pgfpathlineto{\pgfqpoint{5.531898in}{0.770210in}}%
\pgfusepath{stroke}%
\end{pgfscope}%
\begin{pgfscope}%
\pgfpathrectangle{\pgfqpoint{0.800000in}{0.528000in}}{\pgfqpoint{4.960000in}{3.696000in}}%
\pgfusepath{clip}%
\pgfsetrectcap%
\pgfsetroundjoin%
\pgfsetlinewidth{1.505625pt}%
\definecolor{currentstroke}{rgb}{1.000000,0.498039,0.054902}%
\pgfsetstrokecolor{currentstroke}%
\pgfsetdash{}{0pt}%
\pgfpathmoveto{\pgfqpoint{1.025455in}{0.696000in}}%
\pgfpathlineto{\pgfqpoint{1.079078in}{1.663580in}}%
\pgfpathlineto{\pgfqpoint{1.133304in}{2.638261in}}%
\pgfpathlineto{\pgfqpoint{1.187626in}{3.155028in}}%
\pgfpathlineto{\pgfqpoint{1.242811in}{3.524754in}}%
\pgfpathlineto{\pgfqpoint{1.297094in}{3.753055in}}%
\pgfpathlineto{\pgfqpoint{1.351331in}{3.900806in}}%
\pgfpathlineto{\pgfqpoint{1.405602in}{3.605372in}}%
\pgfpathlineto{\pgfqpoint{1.459783in}{3.081132in}}%
\pgfpathlineto{\pgfqpoint{1.514233in}{2.343729in}}%
\pgfpathlineto{\pgfqpoint{1.569789in}{1.819001in}}%
\pgfpathlineto{\pgfqpoint{1.623167in}{1.452796in}}%
\pgfpathlineto{\pgfqpoint{1.676786in}{1.095633in}}%
\pgfpathlineto{\pgfqpoint{1.731557in}{0.770210in}}%
\pgfpathlineto{\pgfqpoint{1.786339in}{0.861939in}}%
\pgfpathlineto{\pgfqpoint{1.840721in}{1.001976in}}%
\pgfpathlineto{\pgfqpoint{1.894997in}{1.141260in}}%
\pgfpathlineto{\pgfqpoint{1.949447in}{1.141260in}}%
\pgfpathlineto{\pgfqpoint{2.003494in}{1.067050in}}%
\pgfpathlineto{\pgfqpoint{2.057647in}{0.992840in}}%
\pgfpathlineto{\pgfqpoint{2.111846in}{0.992840in}}%
\pgfpathlineto{\pgfqpoint{2.167836in}{0.918630in}}%
\pgfpathlineto{\pgfqpoint{2.221397in}{0.770210in}}%
\pgfpathlineto{\pgfqpoint{2.275275in}{0.696000in}}%
\pgfpathlineto{\pgfqpoint{2.329386in}{0.844420in}}%
\pgfpathlineto{\pgfqpoint{2.383558in}{0.918630in}}%
\pgfpathlineto{\pgfqpoint{2.437599in}{0.992840in}}%
\pgfpathlineto{\pgfqpoint{2.492066in}{0.992840in}}%
\pgfpathlineto{\pgfqpoint{2.546254in}{0.844420in}}%
\pgfpathlineto{\pgfqpoint{2.600569in}{0.770210in}}%
\pgfpathlineto{\pgfqpoint{2.654829in}{0.770210in}}%
\pgfpathlineto{\pgfqpoint{2.709010in}{0.696000in}}%
\pgfpathlineto{\pgfqpoint{2.764001in}{0.696000in}}%
\pgfpathlineto{\pgfqpoint{2.817829in}{0.696000in}}%
\pgfpathlineto{\pgfqpoint{2.871719in}{0.696000in}}%
\pgfpathlineto{\pgfqpoint{2.925842in}{0.696000in}}%
\pgfpathlineto{\pgfqpoint{2.980575in}{0.696000in}}%
\pgfpathlineto{\pgfqpoint{3.034757in}{0.696000in}}%
\pgfpathlineto{\pgfqpoint{3.089131in}{0.696000in}}%
\pgfpathlineto{\pgfqpoint{3.143395in}{0.770210in}}%
\pgfpathlineto{\pgfqpoint{3.197336in}{0.770210in}}%
\pgfpathlineto{\pgfqpoint{3.251525in}{0.696000in}}%
\pgfpathlineto{\pgfqpoint{3.305852in}{0.770210in}}%
\pgfpathlineto{\pgfqpoint{3.361730in}{0.770210in}}%
\pgfpathlineto{\pgfqpoint{3.415076in}{0.770210in}}%
\pgfpathlineto{\pgfqpoint{3.469024in}{0.770210in}}%
\pgfpathlineto{\pgfqpoint{3.523121in}{0.770210in}}%
\pgfpathlineto{\pgfqpoint{3.577469in}{0.696000in}}%
\pgfpathlineto{\pgfqpoint{3.631472in}{0.696000in}}%
\pgfpathlineto{\pgfqpoint{3.686281in}{0.770210in}}%
\pgfpathlineto{\pgfqpoint{3.740754in}{0.770210in}}%
\pgfpathlineto{\pgfqpoint{3.794691in}{0.844420in}}%
\pgfpathlineto{\pgfqpoint{3.848692in}{0.844420in}}%
\pgfpathlineto{\pgfqpoint{3.904960in}{0.844420in}}%
\pgfpathlineto{\pgfqpoint{3.959033in}{0.844420in}}%
\pgfpathlineto{\pgfqpoint{4.012218in}{0.696000in}}%
\pgfpathlineto{\pgfqpoint{4.066193in}{0.770210in}}%
\pgfpathlineto{\pgfqpoint{4.120416in}{0.844420in}}%
\pgfpathlineto{\pgfqpoint{4.174555in}{0.918630in}}%
\pgfpathlineto{\pgfqpoint{4.228812in}{0.992840in}}%
\pgfpathlineto{\pgfqpoint{4.283367in}{0.844420in}}%
\pgfpathlineto{\pgfqpoint{4.337657in}{0.696000in}}%
\pgfpathlineto{\pgfqpoint{4.391641in}{0.696000in}}%
\pgfpathlineto{\pgfqpoint{4.445915in}{0.770210in}}%
\pgfpathlineto{\pgfqpoint{4.500461in}{0.770210in}}%
\pgfpathlineto{\pgfqpoint{4.556032in}{0.770210in}}%
\pgfpathlineto{\pgfqpoint{4.609325in}{0.770210in}}%
\pgfpathlineto{\pgfqpoint{4.663128in}{0.696000in}}%
\pgfpathlineto{\pgfqpoint{4.716968in}{0.770210in}}%
\pgfpathlineto{\pgfqpoint{4.771155in}{0.770210in}}%
\pgfpathlineto{\pgfqpoint{4.825515in}{0.770210in}}%
\pgfpathlineto{\pgfqpoint{4.879641in}{0.770210in}}%
\pgfpathlineto{\pgfqpoint{4.935444in}{0.696000in}}%
\pgfpathlineto{\pgfqpoint{4.989412in}{0.696000in}}%
\pgfpathlineto{\pgfqpoint{5.043430in}{0.696000in}}%
\pgfpathlineto{\pgfqpoint{5.097637in}{0.696000in}}%
\pgfpathlineto{\pgfqpoint{5.153695in}{0.696000in}}%
\pgfpathlineto{\pgfqpoint{5.206955in}{0.696000in}}%
\pgfpathlineto{\pgfqpoint{5.260652in}{0.696000in}}%
\pgfpathlineto{\pgfqpoint{5.314831in}{0.696000in}}%
\pgfpathlineto{\pgfqpoint{5.368949in}{0.696000in}}%
\pgfpathlineto{\pgfqpoint{5.423057in}{0.696000in}}%
\pgfpathlineto{\pgfqpoint{5.477650in}{0.696000in}}%
\pgfpathlineto{\pgfqpoint{5.532121in}{0.696000in}}%
\pgfusepath{stroke}%
\end{pgfscope}%
\begin{pgfscope}%
\pgfpathrectangle{\pgfqpoint{0.800000in}{0.528000in}}{\pgfqpoint{4.960000in}{3.696000in}}%
\pgfusepath{clip}%
\pgfsetrectcap%
\pgfsetroundjoin%
\pgfsetlinewidth{1.505625pt}%
\definecolor{currentstroke}{rgb}{0.172549,0.627451,0.172549}%
\pgfsetstrokecolor{currentstroke}%
\pgfsetdash{}{0pt}%
\pgfpathmoveto{\pgfqpoint{1.025455in}{0.696000in}}%
\pgfpathlineto{\pgfqpoint{1.078995in}{1.737587in}}%
\pgfpathlineto{\pgfqpoint{1.133065in}{2.638261in}}%
\pgfpathlineto{\pgfqpoint{1.187429in}{3.155028in}}%
\pgfpathlineto{\pgfqpoint{1.241803in}{3.524754in}}%
\pgfpathlineto{\pgfqpoint{1.298094in}{3.826923in}}%
\pgfpathlineto{\pgfqpoint{1.350082in}{3.974704in}}%
\pgfpathlineto{\pgfqpoint{1.404095in}{4.056000in}}%
\pgfpathlineto{\pgfqpoint{1.458600in}{3.908530in}}%
\pgfpathlineto{\pgfqpoint{1.514272in}{3.162854in}}%
\pgfpathlineto{\pgfqpoint{1.567502in}{2.196804in}}%
\pgfpathlineto{\pgfqpoint{1.621959in}{1.307951in}}%
\pgfpathlineto{\pgfqpoint{1.675980in}{0.800949in}}%
\pgfpathlineto{\pgfqpoint{1.730265in}{1.220744in}}%
\pgfpathlineto{\pgfqpoint{1.784313in}{1.734940in}}%
\pgfpathlineto{\pgfqpoint{1.838817in}{1.957570in}}%
\pgfpathlineto{\pgfqpoint{1.895334in}{1.883360in}}%
\pgfpathlineto{\pgfqpoint{1.946939in}{2.031780in}}%
\pgfpathlineto{\pgfqpoint{2.001327in}{1.959751in}}%
\pgfpathlineto{\pgfqpoint{2.055487in}{1.811621in}}%
\pgfpathlineto{\pgfqpoint{2.111692in}{1.220744in}}%
\pgfpathlineto{\pgfqpoint{2.165126in}{0.918630in}}%
\pgfpathlineto{\pgfqpoint{2.219094in}{1.809150in}}%
\pgfpathlineto{\pgfqpoint{2.273245in}{2.552733in}}%
\pgfpathlineto{\pgfqpoint{2.327784in}{2.849369in}}%
\pgfpathlineto{\pgfqpoint{2.381593in}{2.180200in}}%
\pgfpathlineto{\pgfqpoint{2.435903in}{1.512310in}}%
\pgfpathlineto{\pgfqpoint{2.490233in}{0.918630in}}%
\pgfpathlineto{\pgfqpoint{2.544968in}{1.001976in}}%
\pgfpathlineto{\pgfqpoint{2.599464in}{1.368000in}}%
\pgfpathlineto{\pgfqpoint{2.653678in}{1.737587in}}%
\pgfpathlineto{\pgfqpoint{2.709852in}{1.885677in}}%
\pgfpathlineto{\pgfqpoint{2.762915in}{1.734940in}}%
\pgfpathlineto{\pgfqpoint{2.816687in}{1.512310in}}%
\pgfpathlineto{\pgfqpoint{2.870599in}{1.289680in}}%
\pgfpathlineto{\pgfqpoint{2.924839in}{1.215470in}}%
\pgfpathlineto{\pgfqpoint{2.979277in}{0.992840in}}%
\pgfpathlineto{\pgfqpoint{3.033328in}{0.844420in}}%
\pgfpathlineto{\pgfqpoint{3.087848in}{0.696000in}}%
\pgfpathlineto{\pgfqpoint{3.141810in}{0.844420in}}%
\pgfpathlineto{\pgfqpoint{3.196201in}{1.141260in}}%
\pgfpathlineto{\pgfqpoint{3.250942in}{1.289680in}}%
\pgfpathlineto{\pgfqpoint{3.305458in}{1.215470in}}%
\pgfpathlineto{\pgfqpoint{3.361260in}{1.067050in}}%
\pgfpathlineto{\pgfqpoint{3.414539in}{0.992840in}}%
\pgfpathlineto{\pgfqpoint{3.468771in}{0.992840in}}%
\pgfpathlineto{\pgfqpoint{3.522518in}{0.844420in}}%
\pgfpathlineto{\pgfqpoint{3.576810in}{0.770210in}}%
\pgfpathlineto{\pgfqpoint{3.631051in}{0.696000in}}%
\pgfpathlineto{\pgfqpoint{3.685184in}{0.770210in}}%
\pgfpathlineto{\pgfqpoint{3.739229in}{0.770210in}}%
\pgfpathlineto{\pgfqpoint{3.793518in}{0.770210in}}%
\pgfpathlineto{\pgfqpoint{3.847915in}{0.770210in}}%
\pgfpathlineto{\pgfqpoint{3.902706in}{0.770210in}}%
\pgfpathlineto{\pgfqpoint{3.956786in}{0.770210in}}%
\pgfpathlineto{\pgfqpoint{4.012470in}{0.770210in}}%
\pgfpathlineto{\pgfqpoint{4.065851in}{0.770210in}}%
\pgfpathlineto{\pgfqpoint{4.119832in}{0.696000in}}%
\pgfpathlineto{\pgfqpoint{4.173605in}{0.696000in}}%
\pgfpathlineto{\pgfqpoint{4.227840in}{0.696000in}}%
\pgfpathlineto{\pgfqpoint{4.282617in}{0.696000in}}%
\pgfpathlineto{\pgfqpoint{4.336399in}{0.696000in}}%
\pgfpathlineto{\pgfqpoint{4.390228in}{0.696000in}}%
\pgfpathlineto{\pgfqpoint{4.444621in}{0.696000in}}%
\pgfpathlineto{\pgfqpoint{4.500778in}{0.696000in}}%
\pgfpathlineto{\pgfqpoint{4.554452in}{0.770210in}}%
\pgfpathlineto{\pgfqpoint{4.608115in}{0.770210in}}%
\pgfpathlineto{\pgfqpoint{4.662279in}{0.844420in}}%
\pgfpathlineto{\pgfqpoint{4.716283in}{0.844420in}}%
\pgfpathlineto{\pgfqpoint{4.770576in}{0.844420in}}%
\pgfpathlineto{\pgfqpoint{4.824470in}{0.844420in}}%
\pgfpathlineto{\pgfqpoint{4.879132in}{0.844420in}}%
\pgfpathlineto{\pgfqpoint{4.933561in}{0.844420in}}%
\pgfpathlineto{\pgfqpoint{4.988047in}{0.844420in}}%
\pgfpathlineto{\pgfqpoint{5.041969in}{0.770210in}}%
\pgfpathlineto{\pgfqpoint{5.096449in}{0.770210in}}%
\pgfpathlineto{\pgfqpoint{5.151561in}{0.696000in}}%
\pgfpathlineto{\pgfqpoint{5.205746in}{0.696000in}}%
\pgfpathlineto{\pgfqpoint{5.259769in}{0.770210in}}%
\pgfpathlineto{\pgfqpoint{5.314179in}{0.770210in}}%
\pgfpathlineto{\pgfqpoint{5.367952in}{0.844420in}}%
\pgfpathlineto{\pgfqpoint{5.422075in}{0.918630in}}%
\pgfpathlineto{\pgfqpoint{5.476458in}{0.918630in}}%
\pgfpathlineto{\pgfqpoint{5.530606in}{0.992840in}}%
\pgfusepath{stroke}%
\end{pgfscope}%
\begin{pgfscope}%
\pgfpathrectangle{\pgfqpoint{0.800000in}{0.528000in}}{\pgfqpoint{4.960000in}{3.696000in}}%
\pgfusepath{clip}%
\pgfsetrectcap%
\pgfsetroundjoin%
\pgfsetlinewidth{1.505625pt}%
\definecolor{currentstroke}{rgb}{0.839216,0.152941,0.156863}%
\pgfsetstrokecolor{currentstroke}%
\pgfsetdash{}{0pt}%
\pgfpathmoveto{\pgfqpoint{1.025455in}{0.696000in}}%
\pgfpathlineto{\pgfqpoint{1.079449in}{1.737587in}}%
\pgfpathlineto{\pgfqpoint{1.133671in}{2.638261in}}%
\pgfpathlineto{\pgfqpoint{1.187708in}{3.155028in}}%
\pgfpathlineto{\pgfqpoint{1.241820in}{3.524754in}}%
\pgfpathlineto{\pgfqpoint{1.297771in}{3.820760in}}%
\pgfpathlineto{\pgfqpoint{1.349036in}{3.900806in}}%
\pgfpathlineto{\pgfqpoint{1.403750in}{3.982254in}}%
\pgfpathlineto{\pgfqpoint{1.459845in}{3.908530in}}%
\pgfpathlineto{\pgfqpoint{1.513008in}{3.310257in}}%
\pgfpathlineto{\pgfqpoint{1.568951in}{2.270232in}}%
\pgfpathlineto{\pgfqpoint{1.620777in}{1.307951in}}%
\pgfpathlineto{\pgfqpoint{1.674852in}{0.800949in}}%
\pgfpathlineto{\pgfqpoint{1.729429in}{1.294300in}}%
\pgfpathlineto{\pgfqpoint{1.783499in}{1.515676in}}%
\pgfpathlineto{\pgfqpoint{1.837989in}{1.515676in}}%
\pgfpathlineto{\pgfqpoint{1.892076in}{1.809150in}}%
\pgfpathlineto{\pgfqpoint{1.946461in}{1.957570in}}%
\pgfpathlineto{\pgfqpoint{2.001040in}{1.959751in}}%
\pgfpathlineto{\pgfqpoint{2.060426in}{1.289680in}}%
\pgfpathlineto{\pgfqpoint{2.110491in}{0.770210in}}%
\pgfpathlineto{\pgfqpoint{2.163879in}{1.215470in}}%
\pgfpathlineto{\pgfqpoint{2.218970in}{1.289680in}}%
\pgfpathlineto{\pgfqpoint{2.273175in}{1.147402in}}%
\pgfpathlineto{\pgfqpoint{2.327142in}{1.074398in}}%
\pgfpathlineto{\pgfqpoint{2.381749in}{0.861939in}}%
\pgfpathlineto{\pgfqpoint{2.436005in}{0.930673in}}%
\pgfpathlineto{\pgfqpoint{2.489927in}{1.147402in}}%
\pgfpathlineto{\pgfqpoint{2.544383in}{1.147402in}}%
\pgfpathlineto{\pgfqpoint{2.598477in}{1.220744in}}%
\pgfpathlineto{\pgfqpoint{2.653470in}{1.294300in}}%
\pgfpathlineto{\pgfqpoint{2.709771in}{1.067050in}}%
\pgfpathlineto{\pgfqpoint{2.763024in}{0.992840in}}%
\pgfpathlineto{\pgfqpoint{2.817258in}{0.696000in}}%
\pgfpathlineto{\pgfqpoint{2.871287in}{0.844420in}}%
\pgfpathlineto{\pgfqpoint{2.925948in}{1.141260in}}%
\pgfpathlineto{\pgfqpoint{2.980167in}{1.441801in}}%
\pgfpathlineto{\pgfqpoint{3.034717in}{1.663580in}}%
\pgfpathlineto{\pgfqpoint{3.089282in}{1.586520in}}%
\pgfpathlineto{\pgfqpoint{3.142899in}{1.438100in}}%
\pgfpathlineto{\pgfqpoint{3.197547in}{1.067050in}}%
\pgfpathlineto{\pgfqpoint{3.251610in}{0.918630in}}%
\pgfpathlineto{\pgfqpoint{3.307448in}{0.696000in}}%
\pgfpathlineto{\pgfqpoint{3.361216in}{0.918630in}}%
\pgfpathlineto{\pgfqpoint{3.416253in}{1.067050in}}%
\pgfpathlineto{\pgfqpoint{3.470414in}{1.215470in}}%
\pgfpathlineto{\pgfqpoint{3.523818in}{1.438100in}}%
\pgfpathlineto{\pgfqpoint{3.578408in}{1.586520in}}%
\pgfpathlineto{\pgfqpoint{3.632512in}{1.438100in}}%
\pgfpathlineto{\pgfqpoint{3.687109in}{0.844420in}}%
\pgfpathlineto{\pgfqpoint{3.742489in}{0.770210in}}%
\pgfpathlineto{\pgfqpoint{3.796298in}{0.844420in}}%
\pgfpathlineto{\pgfqpoint{3.850408in}{0.844420in}}%
\pgfpathlineto{\pgfqpoint{3.906021in}{0.918630in}}%
\pgfpathlineto{\pgfqpoint{3.959208in}{1.067050in}}%
\pgfpathlineto{\pgfqpoint{4.013012in}{1.067050in}}%
\pgfpathlineto{\pgfqpoint{4.067220in}{1.141260in}}%
\pgfpathlineto{\pgfqpoint{4.121085in}{1.067050in}}%
\pgfpathlineto{\pgfqpoint{4.175647in}{0.770210in}}%
\pgfpathlineto{\pgfqpoint{4.229930in}{0.918630in}}%
\pgfpathlineto{\pgfqpoint{4.284935in}{0.918630in}}%
\pgfpathlineto{\pgfqpoint{4.338929in}{0.770210in}}%
\pgfpathlineto{\pgfqpoint{4.393155in}{0.696000in}}%
\pgfpathlineto{\pgfqpoint{4.447737in}{0.696000in}}%
\pgfpathlineto{\pgfqpoint{4.502472in}{0.696000in}}%
\pgfpathlineto{\pgfqpoint{4.557531in}{0.696000in}}%
\pgfpathlineto{\pgfqpoint{4.610978in}{0.770210in}}%
\pgfpathlineto{\pgfqpoint{4.665003in}{0.770210in}}%
\pgfpathlineto{\pgfqpoint{4.719328in}{0.770210in}}%
\pgfpathlineto{\pgfqpoint{4.773656in}{0.770210in}}%
\pgfpathlineto{\pgfqpoint{4.827662in}{0.844420in}}%
\pgfpathlineto{\pgfqpoint{4.882569in}{0.844420in}}%
\pgfpathlineto{\pgfqpoint{4.936220in}{0.918630in}}%
\pgfpathlineto{\pgfqpoint{4.990679in}{0.918630in}}%
\pgfpathlineto{\pgfqpoint{5.044968in}{0.918630in}}%
\pgfpathlineto{\pgfqpoint{5.099273in}{0.844420in}}%
\pgfpathlineto{\pgfqpoint{5.155269in}{0.918630in}}%
\pgfpathlineto{\pgfqpoint{5.208387in}{0.770210in}}%
\pgfpathlineto{\pgfqpoint{5.262080in}{0.770210in}}%
\pgfpathlineto{\pgfqpoint{5.316239in}{1.141260in}}%
\pgfpathlineto{\pgfqpoint{5.370386in}{1.220744in}}%
\pgfpathlineto{\pgfqpoint{5.425186in}{0.918630in}}%
\pgfpathlineto{\pgfqpoint{5.479363in}{0.770210in}}%
\pgfpathlineto{\pgfqpoint{5.534545in}{0.770210in}}%
\pgfusepath{stroke}%
\end{pgfscope}%
\begin{pgfscope}%
\pgfpathrectangle{\pgfqpoint{0.800000in}{0.528000in}}{\pgfqpoint{4.960000in}{3.696000in}}%
\pgfusepath{clip}%
\pgfsetrectcap%
\pgfsetroundjoin%
\pgfsetlinewidth{1.505625pt}%
\definecolor{currentstroke}{rgb}{0.580392,0.403922,0.741176}%
\pgfsetstrokecolor{currentstroke}%
\pgfsetdash{}{0pt}%
\pgfpathmoveto{\pgfqpoint{1.025455in}{0.696000in}}%
\pgfpathlineto{\pgfqpoint{1.079790in}{1.737587in}}%
\pgfpathlineto{\pgfqpoint{1.134498in}{2.638261in}}%
\pgfpathlineto{\pgfqpoint{1.188594in}{3.155028in}}%
\pgfpathlineto{\pgfqpoint{1.244278in}{3.531560in}}%
\pgfpathlineto{\pgfqpoint{1.297288in}{3.753055in}}%
\pgfpathlineto{\pgfqpoint{1.350501in}{3.900806in}}%
\pgfpathlineto{\pgfqpoint{1.404513in}{3.982254in}}%
\pgfpathlineto{\pgfqpoint{1.458652in}{3.679204in}}%
\pgfpathlineto{\pgfqpoint{1.513394in}{2.785772in}}%
\pgfpathlineto{\pgfqpoint{1.567273in}{1.819001in}}%
\pgfpathlineto{\pgfqpoint{1.621257in}{0.930673in}}%
\pgfpathlineto{\pgfqpoint{1.675608in}{1.074398in}}%
\pgfpathlineto{\pgfqpoint{1.730279in}{1.586520in}}%
\pgfpathlineto{\pgfqpoint{1.786921in}{2.477040in}}%
\pgfpathlineto{\pgfqpoint{1.840293in}{2.551250in}}%
\pgfpathlineto{\pgfqpoint{1.894141in}{3.149423in}}%
\pgfpathlineto{\pgfqpoint{1.948068in}{2.853201in}}%
\pgfpathlineto{\pgfqpoint{2.001878in}{1.141260in}}%
\pgfpathlineto{\pgfqpoint{2.056555in}{1.067050in}}%
\pgfpathlineto{\pgfqpoint{2.110643in}{2.033840in}}%
\pgfpathlineto{\pgfqpoint{2.164736in}{2.775204in}}%
\pgfpathlineto{\pgfqpoint{2.219139in}{3.371679in}}%
\pgfpathlineto{\pgfqpoint{2.272873in}{2.328620in}}%
\pgfpathlineto{\pgfqpoint{2.327797in}{0.905898in}}%
\pgfpathlineto{\pgfqpoint{2.382023in}{0.800949in}}%
\pgfpathlineto{\pgfqpoint{2.435738in}{1.294300in}}%
\pgfpathlineto{\pgfqpoint{2.490004in}{1.663580in}}%
\pgfpathlineto{\pgfqpoint{2.544230in}{2.107941in}}%
\pgfpathlineto{\pgfqpoint{2.599676in}{2.404442in}}%
\pgfpathlineto{\pgfqpoint{2.653395in}{2.033840in}}%
\pgfpathlineto{\pgfqpoint{2.708467in}{1.215470in}}%
\pgfpathlineto{\pgfqpoint{2.762948in}{1.380182in}}%
\pgfpathlineto{\pgfqpoint{2.817143in}{2.417288in}}%
\pgfpathlineto{\pgfqpoint{2.871575in}{3.007257in}}%
\pgfpathlineto{\pgfqpoint{2.925797in}{3.155028in}}%
\pgfpathlineto{\pgfqpoint{2.979903in}{2.254410in}}%
\pgfpathlineto{\pgfqpoint{3.034975in}{1.512310in}}%
\pgfpathlineto{\pgfqpoint{3.089420in}{0.918630in}}%
\pgfpathlineto{\pgfqpoint{3.144511in}{0.770210in}}%
\pgfpathlineto{\pgfqpoint{3.198501in}{1.215470in}}%
\pgfpathlineto{\pgfqpoint{3.252625in}{1.586520in}}%
\pgfpathlineto{\pgfqpoint{3.306828in}{1.883360in}}%
\pgfpathlineto{\pgfqpoint{3.361301in}{2.330305in}}%
\pgfpathlineto{\pgfqpoint{3.416921in}{2.478585in}}%
\pgfpathlineto{\pgfqpoint{3.470533in}{1.883360in}}%
\pgfpathlineto{\pgfqpoint{3.524128in}{0.770210in}}%
\pgfpathlineto{\pgfqpoint{3.578579in}{1.892600in}}%
\pgfpathlineto{\pgfqpoint{3.633289in}{2.187602in}}%
\pgfpathlineto{\pgfqpoint{3.687682in}{2.404442in}}%
\pgfpathlineto{\pgfqpoint{3.742176in}{1.883360in}}%
\pgfpathlineto{\pgfqpoint{3.796098in}{1.141260in}}%
\pgfpathlineto{\pgfqpoint{3.850810in}{0.770210in}}%
\pgfpathlineto{\pgfqpoint{3.906014in}{0.844420in}}%
\pgfpathlineto{\pgfqpoint{3.959825in}{1.220744in}}%
\pgfpathlineto{\pgfqpoint{4.014103in}{1.441801in}}%
\pgfpathlineto{\pgfqpoint{4.068466in}{1.441801in}}%
\pgfpathlineto{\pgfqpoint{4.122727in}{1.215470in}}%
\pgfpathlineto{\pgfqpoint{4.176945in}{0.844420in}}%
\pgfpathlineto{\pgfqpoint{4.231036in}{0.696000in}}%
\pgfpathlineto{\pgfqpoint{4.285328in}{0.696000in}}%
\pgfpathlineto{\pgfqpoint{4.339621in}{0.696000in}}%
\pgfpathlineto{\pgfqpoint{4.393594in}{0.696000in}}%
\pgfpathlineto{\pgfqpoint{4.447734in}{0.770210in}}%
\pgfpathlineto{\pgfqpoint{4.502459in}{0.770210in}}%
\pgfpathlineto{\pgfqpoint{4.556477in}{0.844420in}}%
\pgfpathlineto{\pgfqpoint{4.611209in}{0.770210in}}%
\pgfpathlineto{\pgfqpoint{4.665429in}{0.770210in}}%
\pgfpathlineto{\pgfqpoint{4.719602in}{0.844420in}}%
\pgfpathlineto{\pgfqpoint{4.774113in}{0.770210in}}%
\pgfpathlineto{\pgfqpoint{4.828488in}{0.770210in}}%
\pgfpathlineto{\pgfqpoint{4.882448in}{0.696000in}}%
\pgfpathlineto{\pgfqpoint{4.936854in}{0.696000in}}%
\pgfpathlineto{\pgfqpoint{4.990493in}{0.696000in}}%
\pgfpathlineto{\pgfqpoint{5.044596in}{0.770210in}}%
\pgfpathlineto{\pgfqpoint{5.098879in}{1.441801in}}%
\pgfpathlineto{\pgfqpoint{5.152934in}{1.363890in}}%
\pgfpathlineto{\pgfqpoint{5.207352in}{0.696000in}}%
\pgfpathlineto{\pgfqpoint{5.261714in}{0.844420in}}%
\pgfpathlineto{\pgfqpoint{5.316133in}{0.844420in}}%
\pgfpathlineto{\pgfqpoint{5.370212in}{0.992840in}}%
\pgfpathlineto{\pgfqpoint{5.424514in}{1.067050in}}%
\pgfpathlineto{\pgfqpoint{5.478630in}{0.992840in}}%
\pgfpathlineto{\pgfqpoint{5.532785in}{0.918630in}}%
\pgfusepath{stroke}%
\end{pgfscope}%
\begin{pgfscope}%
\pgfsetrectcap%
\pgfsetmiterjoin%
\pgfsetlinewidth{0.803000pt}%
\definecolor{currentstroke}{rgb}{0.000000,0.000000,0.000000}%
\pgfsetstrokecolor{currentstroke}%
\pgfsetdash{}{0pt}%
\pgfpathmoveto{\pgfqpoint{0.800000in}{0.528000in}}%
\pgfpathlineto{\pgfqpoint{0.800000in}{4.224000in}}%
\pgfusepath{stroke}%
\end{pgfscope}%
\begin{pgfscope}%
\pgfsetrectcap%
\pgfsetmiterjoin%
\pgfsetlinewidth{0.803000pt}%
\definecolor{currentstroke}{rgb}{0.000000,0.000000,0.000000}%
\pgfsetstrokecolor{currentstroke}%
\pgfsetdash{}{0pt}%
\pgfpathmoveto{\pgfqpoint{5.760000in}{0.528000in}}%
\pgfpathlineto{\pgfqpoint{5.760000in}{4.224000in}}%
\pgfusepath{stroke}%
\end{pgfscope}%
\begin{pgfscope}%
\pgfsetrectcap%
\pgfsetmiterjoin%
\pgfsetlinewidth{0.803000pt}%
\definecolor{currentstroke}{rgb}{0.000000,0.000000,0.000000}%
\pgfsetstrokecolor{currentstroke}%
\pgfsetdash{}{0pt}%
\pgfpathmoveto{\pgfqpoint{0.800000in}{0.528000in}}%
\pgfpathlineto{\pgfqpoint{5.760000in}{0.528000in}}%
\pgfusepath{stroke}%
\end{pgfscope}%
\begin{pgfscope}%
\pgfsetrectcap%
\pgfsetmiterjoin%
\pgfsetlinewidth{0.803000pt}%
\definecolor{currentstroke}{rgb}{0.000000,0.000000,0.000000}%
\pgfsetstrokecolor{currentstroke}%
\pgfsetdash{}{0pt}%
\pgfpathmoveto{\pgfqpoint{0.800000in}{4.224000in}}%
\pgfpathlineto{\pgfqpoint{5.760000in}{4.224000in}}%
\pgfusepath{stroke}%
\end{pgfscope}%
\begin{pgfscope}%
\definecolor{textcolor}{rgb}{0.000000,0.000000,0.000000}%
\pgfsetstrokecolor{textcolor}%
\pgfsetfillcolor{textcolor}%
\pgftext[x=3.280000in,y=4.307333in,,base]{\color{textcolor}\sffamily\fontsize{12.000000}{14.400000}\selectfont Measured ground speed}%
\end{pgfscope}%
\begin{pgfscope}%
\pgfsetbuttcap%
\pgfsetmiterjoin%
\definecolor{currentfill}{rgb}{1.000000,1.000000,1.000000}%
\pgfsetfillcolor{currentfill}%
\pgfsetfillopacity{0.800000}%
\pgfsetlinewidth{1.003750pt}%
\definecolor{currentstroke}{rgb}{0.800000,0.800000,0.800000}%
\pgfsetstrokecolor{currentstroke}%
\pgfsetstrokeopacity{0.800000}%
\pgfsetdash{}{0pt}%
\pgfpathmoveto{\pgfqpoint{5.041603in}{3.093603in}}%
\pgfpathlineto{\pgfqpoint{5.662778in}{3.093603in}}%
\pgfpathquadraticcurveto{\pgfqpoint{5.690556in}{3.093603in}}{\pgfqpoint{5.690556in}{3.121381in}}%
\pgfpathlineto{\pgfqpoint{5.690556in}{4.126778in}}%
\pgfpathquadraticcurveto{\pgfqpoint{5.690556in}{4.154556in}}{\pgfqpoint{5.662778in}{4.154556in}}%
\pgfpathlineto{\pgfqpoint{5.041603in}{4.154556in}}%
\pgfpathquadraticcurveto{\pgfqpoint{5.013825in}{4.154556in}}{\pgfqpoint{5.013825in}{4.126778in}}%
\pgfpathlineto{\pgfqpoint{5.013825in}{3.121381in}}%
\pgfpathquadraticcurveto{\pgfqpoint{5.013825in}{3.093603in}}{\pgfqpoint{5.041603in}{3.093603in}}%
\pgfpathlineto{\pgfqpoint{5.041603in}{3.093603in}}%
\pgfpathclose%
\pgfusepath{stroke,fill}%
\end{pgfscope}%
\begin{pgfscope}%
\pgfsetrectcap%
\pgfsetroundjoin%
\pgfsetlinewidth{1.505625pt}%
\definecolor{currentstroke}{rgb}{0.121569,0.466667,0.705882}%
\pgfsetstrokecolor{currentstroke}%
\pgfsetdash{}{0pt}%
\pgfpathmoveto{\pgfqpoint{5.069380in}{4.042088in}}%
\pgfpathlineto{\pgfqpoint{5.208269in}{4.042088in}}%
\pgfpathlineto{\pgfqpoint{5.347158in}{4.042088in}}%
\pgfusepath{stroke}%
\end{pgfscope}%
\begin{pgfscope}%
\definecolor{textcolor}{rgb}{0.000000,0.000000,0.000000}%
\pgfsetstrokecolor{textcolor}%
\pgfsetfillcolor{textcolor}%
\pgftext[x=5.458269in,y=3.993477in,left,base]{\color{textcolor}\sffamily\fontsize{10.000000}{12.000000}\selectfont 2}%
\end{pgfscope}%
\begin{pgfscope}%
\pgfsetrectcap%
\pgfsetroundjoin%
\pgfsetlinewidth{1.505625pt}%
\definecolor{currentstroke}{rgb}{1.000000,0.498039,0.054902}%
\pgfsetstrokecolor{currentstroke}%
\pgfsetdash{}{0pt}%
\pgfpathmoveto{\pgfqpoint{5.069380in}{3.838231in}}%
\pgfpathlineto{\pgfqpoint{5.208269in}{3.838231in}}%
\pgfpathlineto{\pgfqpoint{5.347158in}{3.838231in}}%
\pgfusepath{stroke}%
\end{pgfscope}%
\begin{pgfscope}%
\definecolor{textcolor}{rgb}{0.000000,0.000000,0.000000}%
\pgfsetstrokecolor{textcolor}%
\pgfsetfillcolor{textcolor}%
\pgftext[x=5.458269in,y=3.789620in,left,base]{\color{textcolor}\sffamily\fontsize{10.000000}{12.000000}\selectfont 4}%
\end{pgfscope}%
\begin{pgfscope}%
\pgfsetrectcap%
\pgfsetroundjoin%
\pgfsetlinewidth{1.505625pt}%
\definecolor{currentstroke}{rgb}{0.172549,0.627451,0.172549}%
\pgfsetstrokecolor{currentstroke}%
\pgfsetdash{}{0pt}%
\pgfpathmoveto{\pgfqpoint{5.069380in}{3.634374in}}%
\pgfpathlineto{\pgfqpoint{5.208269in}{3.634374in}}%
\pgfpathlineto{\pgfqpoint{5.347158in}{3.634374in}}%
\pgfusepath{stroke}%
\end{pgfscope}%
\begin{pgfscope}%
\definecolor{textcolor}{rgb}{0.000000,0.000000,0.000000}%
\pgfsetstrokecolor{textcolor}%
\pgfsetfillcolor{textcolor}%
\pgftext[x=5.458269in,y=3.585762in,left,base]{\color{textcolor}\sffamily\fontsize{10.000000}{12.000000}\selectfont 6}%
\end{pgfscope}%
\begin{pgfscope}%
\pgfsetrectcap%
\pgfsetroundjoin%
\pgfsetlinewidth{1.505625pt}%
\definecolor{currentstroke}{rgb}{0.839216,0.152941,0.156863}%
\pgfsetstrokecolor{currentstroke}%
\pgfsetdash{}{0pt}%
\pgfpathmoveto{\pgfqpoint{5.069380in}{3.430516in}}%
\pgfpathlineto{\pgfqpoint{5.208269in}{3.430516in}}%
\pgfpathlineto{\pgfqpoint{5.347158in}{3.430516in}}%
\pgfusepath{stroke}%
\end{pgfscope}%
\begin{pgfscope}%
\definecolor{textcolor}{rgb}{0.000000,0.000000,0.000000}%
\pgfsetstrokecolor{textcolor}%
\pgfsetfillcolor{textcolor}%
\pgftext[x=5.458269in,y=3.381905in,left,base]{\color{textcolor}\sffamily\fontsize{10.000000}{12.000000}\selectfont 8}%
\end{pgfscope}%
\begin{pgfscope}%
\pgfsetrectcap%
\pgfsetroundjoin%
\pgfsetlinewidth{1.505625pt}%
\definecolor{currentstroke}{rgb}{0.580392,0.403922,0.741176}%
\pgfsetstrokecolor{currentstroke}%
\pgfsetdash{}{0pt}%
\pgfpathmoveto{\pgfqpoint{5.069380in}{3.226659in}}%
\pgfpathlineto{\pgfqpoint{5.208269in}{3.226659in}}%
\pgfpathlineto{\pgfqpoint{5.347158in}{3.226659in}}%
\pgfusepath{stroke}%
\end{pgfscope}%
\begin{pgfscope}%
\definecolor{textcolor}{rgb}{0.000000,0.000000,0.000000}%
\pgfsetstrokecolor{textcolor}%
\pgfsetfillcolor{textcolor}%
\pgftext[x=5.458269in,y=3.178048in,left,base]{\color{textcolor}\sffamily\fontsize{10.000000}{12.000000}\selectfont 10}%
\end{pgfscope}%
\end{pgfpicture}%
\makeatother%
\endgroup%
}
    \end{minipage}
    \caption{Variation of (a) measured forward position and (b) measured absolute ground velocity for different values of $K_{P}$ and $K_I=0$, $K_D=0$ while the forward controller is engaged.}
    \label{fig:tune-fwd-prop-measures}
\end{figure}

\begin{figure}[H]
    \begin{minipage}[t]{0.5\linewidth}
        \centering
        \scalebox{0.55}{%% Creator: Matplotlib, PGF backend
%%
%% To include the figure in your LaTeX document, write
%%   \input{<filename>.pgf}
%%
%% Make sure the required packages are loaded in your preamble
%%   \usepackage{pgf}
%%
%% Also ensure that all the required font packages are loaded; for instance,
%% the lmodern package is sometimes necessary when using math font.
%%   \usepackage{lmodern}
%%
%% Figures using additional raster images can only be included by \input if
%% they are in the same directory as the main LaTeX file. For loading figures
%% from other directories you can use the `import` package
%%   \usepackage{import}
%%
%% and then include the figures with
%%   \import{<path to file>}{<filename>.pgf}
%%
%% Matplotlib used the following preamble
%%   \usepackage{fontspec}
%%   \setmainfont{DejaVuSerif.ttf}[Path=\detokenize{/home/lgonz/tfg-aero/tfg-giaa-dronecontrol/venv/lib/python3.8/site-packages/matplotlib/mpl-data/fonts/ttf/}]
%%   \setsansfont{DejaVuSans.ttf}[Path=\detokenize{/home/lgonz/tfg-aero/tfg-giaa-dronecontrol/venv/lib/python3.8/site-packages/matplotlib/mpl-data/fonts/ttf/}]
%%   \setmonofont{DejaVuSansMono.ttf}[Path=\detokenize{/home/lgonz/tfg-aero/tfg-giaa-dronecontrol/venv/lib/python3.8/site-packages/matplotlib/mpl-data/fonts/ttf/}]
%%
\begingroup%
\makeatletter%
\begin{pgfpicture}%
\pgfpathrectangle{\pgfpointorigin}{\pgfqpoint{6.400000in}{4.800000in}}%
\pgfusepath{use as bounding box, clip}%
\begin{pgfscope}%
\pgfsetbuttcap%
\pgfsetmiterjoin%
\definecolor{currentfill}{rgb}{1.000000,1.000000,1.000000}%
\pgfsetfillcolor{currentfill}%
\pgfsetlinewidth{0.000000pt}%
\definecolor{currentstroke}{rgb}{1.000000,1.000000,1.000000}%
\pgfsetstrokecolor{currentstroke}%
\pgfsetdash{}{0pt}%
\pgfpathmoveto{\pgfqpoint{0.000000in}{0.000000in}}%
\pgfpathlineto{\pgfqpoint{6.400000in}{0.000000in}}%
\pgfpathlineto{\pgfqpoint{6.400000in}{4.800000in}}%
\pgfpathlineto{\pgfqpoint{0.000000in}{4.800000in}}%
\pgfpathlineto{\pgfqpoint{0.000000in}{0.000000in}}%
\pgfpathclose%
\pgfusepath{fill}%
\end{pgfscope}%
\begin{pgfscope}%
\pgfsetbuttcap%
\pgfsetmiterjoin%
\definecolor{currentfill}{rgb}{1.000000,1.000000,1.000000}%
\pgfsetfillcolor{currentfill}%
\pgfsetlinewidth{0.000000pt}%
\definecolor{currentstroke}{rgb}{0.000000,0.000000,0.000000}%
\pgfsetstrokecolor{currentstroke}%
\pgfsetstrokeopacity{0.000000}%
\pgfsetdash{}{0pt}%
\pgfpathmoveto{\pgfqpoint{0.800000in}{0.528000in}}%
\pgfpathlineto{\pgfqpoint{5.760000in}{0.528000in}}%
\pgfpathlineto{\pgfqpoint{5.760000in}{4.224000in}}%
\pgfpathlineto{\pgfqpoint{0.800000in}{4.224000in}}%
\pgfpathlineto{\pgfqpoint{0.800000in}{0.528000in}}%
\pgfpathclose%
\pgfusepath{fill}%
\end{pgfscope}%
\begin{pgfscope}%
\pgfpathrectangle{\pgfqpoint{0.800000in}{0.528000in}}{\pgfqpoint{4.960000in}{3.696000in}}%
\pgfusepath{clip}%
\pgfsetrectcap%
\pgfsetroundjoin%
\pgfsetlinewidth{0.803000pt}%
\definecolor{currentstroke}{rgb}{0.690196,0.690196,0.690196}%
\pgfsetstrokecolor{currentstroke}%
\pgfsetdash{}{0pt}%
\pgfpathmoveto{\pgfqpoint{1.025455in}{0.528000in}}%
\pgfpathlineto{\pgfqpoint{1.025455in}{4.224000in}}%
\pgfusepath{stroke}%
\end{pgfscope}%
\begin{pgfscope}%
\pgfsetbuttcap%
\pgfsetroundjoin%
\definecolor{currentfill}{rgb}{0.000000,0.000000,0.000000}%
\pgfsetfillcolor{currentfill}%
\pgfsetlinewidth{0.803000pt}%
\definecolor{currentstroke}{rgb}{0.000000,0.000000,0.000000}%
\pgfsetstrokecolor{currentstroke}%
\pgfsetdash{}{0pt}%
\pgfsys@defobject{currentmarker}{\pgfqpoint{0.000000in}{-0.048611in}}{\pgfqpoint{0.000000in}{0.000000in}}{%
\pgfpathmoveto{\pgfqpoint{0.000000in}{0.000000in}}%
\pgfpathlineto{\pgfqpoint{0.000000in}{-0.048611in}}%
\pgfusepath{stroke,fill}%
}%
\begin{pgfscope}%
\pgfsys@transformshift{1.025455in}{0.528000in}%
\pgfsys@useobject{currentmarker}{}%
\end{pgfscope}%
\end{pgfscope}%
\begin{pgfscope}%
\definecolor{textcolor}{rgb}{0.000000,0.000000,0.000000}%
\pgfsetstrokecolor{textcolor}%
\pgfsetfillcolor{textcolor}%
\pgftext[x=1.025455in,y=0.430778in,,top]{\color{textcolor}\sffamily\fontsize{10.000000}{12.000000}\selectfont 0}%
\end{pgfscope}%
\begin{pgfscope}%
\pgfpathrectangle{\pgfqpoint{0.800000in}{0.528000in}}{\pgfqpoint{4.960000in}{3.696000in}}%
\pgfusepath{clip}%
\pgfsetrectcap%
\pgfsetroundjoin%
\pgfsetlinewidth{0.803000pt}%
\definecolor{currentstroke}{rgb}{0.690196,0.690196,0.690196}%
\pgfsetstrokecolor{currentstroke}%
\pgfsetdash{}{0pt}%
\pgfpathmoveto{\pgfqpoint{1.775563in}{0.528000in}}%
\pgfpathlineto{\pgfqpoint{1.775563in}{4.224000in}}%
\pgfusepath{stroke}%
\end{pgfscope}%
\begin{pgfscope}%
\pgfsetbuttcap%
\pgfsetroundjoin%
\definecolor{currentfill}{rgb}{0.000000,0.000000,0.000000}%
\pgfsetfillcolor{currentfill}%
\pgfsetlinewidth{0.803000pt}%
\definecolor{currentstroke}{rgb}{0.000000,0.000000,0.000000}%
\pgfsetstrokecolor{currentstroke}%
\pgfsetdash{}{0pt}%
\pgfsys@defobject{currentmarker}{\pgfqpoint{0.000000in}{-0.048611in}}{\pgfqpoint{0.000000in}{0.000000in}}{%
\pgfpathmoveto{\pgfqpoint{0.000000in}{0.000000in}}%
\pgfpathlineto{\pgfqpoint{0.000000in}{-0.048611in}}%
\pgfusepath{stroke,fill}%
}%
\begin{pgfscope}%
\pgfsys@transformshift{1.775563in}{0.528000in}%
\pgfsys@useobject{currentmarker}{}%
\end{pgfscope}%
\end{pgfscope}%
\begin{pgfscope}%
\definecolor{textcolor}{rgb}{0.000000,0.000000,0.000000}%
\pgfsetstrokecolor{textcolor}%
\pgfsetfillcolor{textcolor}%
\pgftext[x=1.775563in,y=0.430778in,,top]{\color{textcolor}\sffamily\fontsize{10.000000}{12.000000}\selectfont 5}%
\end{pgfscope}%
\begin{pgfscope}%
\pgfpathrectangle{\pgfqpoint{0.800000in}{0.528000in}}{\pgfqpoint{4.960000in}{3.696000in}}%
\pgfusepath{clip}%
\pgfsetrectcap%
\pgfsetroundjoin%
\pgfsetlinewidth{0.803000pt}%
\definecolor{currentstroke}{rgb}{0.690196,0.690196,0.690196}%
\pgfsetstrokecolor{currentstroke}%
\pgfsetdash{}{0pt}%
\pgfpathmoveto{\pgfqpoint{2.525672in}{0.528000in}}%
\pgfpathlineto{\pgfqpoint{2.525672in}{4.224000in}}%
\pgfusepath{stroke}%
\end{pgfscope}%
\begin{pgfscope}%
\pgfsetbuttcap%
\pgfsetroundjoin%
\definecolor{currentfill}{rgb}{0.000000,0.000000,0.000000}%
\pgfsetfillcolor{currentfill}%
\pgfsetlinewidth{0.803000pt}%
\definecolor{currentstroke}{rgb}{0.000000,0.000000,0.000000}%
\pgfsetstrokecolor{currentstroke}%
\pgfsetdash{}{0pt}%
\pgfsys@defobject{currentmarker}{\pgfqpoint{0.000000in}{-0.048611in}}{\pgfqpoint{0.000000in}{0.000000in}}{%
\pgfpathmoveto{\pgfqpoint{0.000000in}{0.000000in}}%
\pgfpathlineto{\pgfqpoint{0.000000in}{-0.048611in}}%
\pgfusepath{stroke,fill}%
}%
\begin{pgfscope}%
\pgfsys@transformshift{2.525672in}{0.528000in}%
\pgfsys@useobject{currentmarker}{}%
\end{pgfscope}%
\end{pgfscope}%
\begin{pgfscope}%
\definecolor{textcolor}{rgb}{0.000000,0.000000,0.000000}%
\pgfsetstrokecolor{textcolor}%
\pgfsetfillcolor{textcolor}%
\pgftext[x=2.525672in,y=0.430778in,,top]{\color{textcolor}\sffamily\fontsize{10.000000}{12.000000}\selectfont 10}%
\end{pgfscope}%
\begin{pgfscope}%
\pgfpathrectangle{\pgfqpoint{0.800000in}{0.528000in}}{\pgfqpoint{4.960000in}{3.696000in}}%
\pgfusepath{clip}%
\pgfsetrectcap%
\pgfsetroundjoin%
\pgfsetlinewidth{0.803000pt}%
\definecolor{currentstroke}{rgb}{0.690196,0.690196,0.690196}%
\pgfsetstrokecolor{currentstroke}%
\pgfsetdash{}{0pt}%
\pgfpathmoveto{\pgfqpoint{3.275781in}{0.528000in}}%
\pgfpathlineto{\pgfqpoint{3.275781in}{4.224000in}}%
\pgfusepath{stroke}%
\end{pgfscope}%
\begin{pgfscope}%
\pgfsetbuttcap%
\pgfsetroundjoin%
\definecolor{currentfill}{rgb}{0.000000,0.000000,0.000000}%
\pgfsetfillcolor{currentfill}%
\pgfsetlinewidth{0.803000pt}%
\definecolor{currentstroke}{rgb}{0.000000,0.000000,0.000000}%
\pgfsetstrokecolor{currentstroke}%
\pgfsetdash{}{0pt}%
\pgfsys@defobject{currentmarker}{\pgfqpoint{0.000000in}{-0.048611in}}{\pgfqpoint{0.000000in}{0.000000in}}{%
\pgfpathmoveto{\pgfqpoint{0.000000in}{0.000000in}}%
\pgfpathlineto{\pgfqpoint{0.000000in}{-0.048611in}}%
\pgfusepath{stroke,fill}%
}%
\begin{pgfscope}%
\pgfsys@transformshift{3.275781in}{0.528000in}%
\pgfsys@useobject{currentmarker}{}%
\end{pgfscope}%
\end{pgfscope}%
\begin{pgfscope}%
\definecolor{textcolor}{rgb}{0.000000,0.000000,0.000000}%
\pgfsetstrokecolor{textcolor}%
\pgfsetfillcolor{textcolor}%
\pgftext[x=3.275781in,y=0.430778in,,top]{\color{textcolor}\sffamily\fontsize{10.000000}{12.000000}\selectfont 15}%
\end{pgfscope}%
\begin{pgfscope}%
\pgfpathrectangle{\pgfqpoint{0.800000in}{0.528000in}}{\pgfqpoint{4.960000in}{3.696000in}}%
\pgfusepath{clip}%
\pgfsetrectcap%
\pgfsetroundjoin%
\pgfsetlinewidth{0.803000pt}%
\definecolor{currentstroke}{rgb}{0.690196,0.690196,0.690196}%
\pgfsetstrokecolor{currentstroke}%
\pgfsetdash{}{0pt}%
\pgfpathmoveto{\pgfqpoint{4.025890in}{0.528000in}}%
\pgfpathlineto{\pgfqpoint{4.025890in}{4.224000in}}%
\pgfusepath{stroke}%
\end{pgfscope}%
\begin{pgfscope}%
\pgfsetbuttcap%
\pgfsetroundjoin%
\definecolor{currentfill}{rgb}{0.000000,0.000000,0.000000}%
\pgfsetfillcolor{currentfill}%
\pgfsetlinewidth{0.803000pt}%
\definecolor{currentstroke}{rgb}{0.000000,0.000000,0.000000}%
\pgfsetstrokecolor{currentstroke}%
\pgfsetdash{}{0pt}%
\pgfsys@defobject{currentmarker}{\pgfqpoint{0.000000in}{-0.048611in}}{\pgfqpoint{0.000000in}{0.000000in}}{%
\pgfpathmoveto{\pgfqpoint{0.000000in}{0.000000in}}%
\pgfpathlineto{\pgfqpoint{0.000000in}{-0.048611in}}%
\pgfusepath{stroke,fill}%
}%
\begin{pgfscope}%
\pgfsys@transformshift{4.025890in}{0.528000in}%
\pgfsys@useobject{currentmarker}{}%
\end{pgfscope}%
\end{pgfscope}%
\begin{pgfscope}%
\definecolor{textcolor}{rgb}{0.000000,0.000000,0.000000}%
\pgfsetstrokecolor{textcolor}%
\pgfsetfillcolor{textcolor}%
\pgftext[x=4.025890in,y=0.430778in,,top]{\color{textcolor}\sffamily\fontsize{10.000000}{12.000000}\selectfont 20}%
\end{pgfscope}%
\begin{pgfscope}%
\pgfpathrectangle{\pgfqpoint{0.800000in}{0.528000in}}{\pgfqpoint{4.960000in}{3.696000in}}%
\pgfusepath{clip}%
\pgfsetrectcap%
\pgfsetroundjoin%
\pgfsetlinewidth{0.803000pt}%
\definecolor{currentstroke}{rgb}{0.690196,0.690196,0.690196}%
\pgfsetstrokecolor{currentstroke}%
\pgfsetdash{}{0pt}%
\pgfpathmoveto{\pgfqpoint{4.775998in}{0.528000in}}%
\pgfpathlineto{\pgfqpoint{4.775998in}{4.224000in}}%
\pgfusepath{stroke}%
\end{pgfscope}%
\begin{pgfscope}%
\pgfsetbuttcap%
\pgfsetroundjoin%
\definecolor{currentfill}{rgb}{0.000000,0.000000,0.000000}%
\pgfsetfillcolor{currentfill}%
\pgfsetlinewidth{0.803000pt}%
\definecolor{currentstroke}{rgb}{0.000000,0.000000,0.000000}%
\pgfsetstrokecolor{currentstroke}%
\pgfsetdash{}{0pt}%
\pgfsys@defobject{currentmarker}{\pgfqpoint{0.000000in}{-0.048611in}}{\pgfqpoint{0.000000in}{0.000000in}}{%
\pgfpathmoveto{\pgfqpoint{0.000000in}{0.000000in}}%
\pgfpathlineto{\pgfqpoint{0.000000in}{-0.048611in}}%
\pgfusepath{stroke,fill}%
}%
\begin{pgfscope}%
\pgfsys@transformshift{4.775998in}{0.528000in}%
\pgfsys@useobject{currentmarker}{}%
\end{pgfscope}%
\end{pgfscope}%
\begin{pgfscope}%
\definecolor{textcolor}{rgb}{0.000000,0.000000,0.000000}%
\pgfsetstrokecolor{textcolor}%
\pgfsetfillcolor{textcolor}%
\pgftext[x=4.775998in,y=0.430778in,,top]{\color{textcolor}\sffamily\fontsize{10.000000}{12.000000}\selectfont 25}%
\end{pgfscope}%
\begin{pgfscope}%
\pgfpathrectangle{\pgfqpoint{0.800000in}{0.528000in}}{\pgfqpoint{4.960000in}{3.696000in}}%
\pgfusepath{clip}%
\pgfsetrectcap%
\pgfsetroundjoin%
\pgfsetlinewidth{0.803000pt}%
\definecolor{currentstroke}{rgb}{0.690196,0.690196,0.690196}%
\pgfsetstrokecolor{currentstroke}%
\pgfsetdash{}{0pt}%
\pgfpathmoveto{\pgfqpoint{5.526107in}{0.528000in}}%
\pgfpathlineto{\pgfqpoint{5.526107in}{4.224000in}}%
\pgfusepath{stroke}%
\end{pgfscope}%
\begin{pgfscope}%
\pgfsetbuttcap%
\pgfsetroundjoin%
\definecolor{currentfill}{rgb}{0.000000,0.000000,0.000000}%
\pgfsetfillcolor{currentfill}%
\pgfsetlinewidth{0.803000pt}%
\definecolor{currentstroke}{rgb}{0.000000,0.000000,0.000000}%
\pgfsetstrokecolor{currentstroke}%
\pgfsetdash{}{0pt}%
\pgfsys@defobject{currentmarker}{\pgfqpoint{0.000000in}{-0.048611in}}{\pgfqpoint{0.000000in}{0.000000in}}{%
\pgfpathmoveto{\pgfqpoint{0.000000in}{0.000000in}}%
\pgfpathlineto{\pgfqpoint{0.000000in}{-0.048611in}}%
\pgfusepath{stroke,fill}%
}%
\begin{pgfscope}%
\pgfsys@transformshift{5.526107in}{0.528000in}%
\pgfsys@useobject{currentmarker}{}%
\end{pgfscope}%
\end{pgfscope}%
\begin{pgfscope}%
\definecolor{textcolor}{rgb}{0.000000,0.000000,0.000000}%
\pgfsetstrokecolor{textcolor}%
\pgfsetfillcolor{textcolor}%
\pgftext[x=5.526107in,y=0.430778in,,top]{\color{textcolor}\sffamily\fontsize{10.000000}{12.000000}\selectfont 30}%
\end{pgfscope}%
\begin{pgfscope}%
\definecolor{textcolor}{rgb}{0.000000,0.000000,0.000000}%
\pgfsetstrokecolor{textcolor}%
\pgfsetfillcolor{textcolor}%
\pgftext[x=3.280000in,y=0.240809in,,top]{\color{textcolor}\sffamily\fontsize{10.000000}{12.000000}\selectfont time [s]}%
\end{pgfscope}%
\begin{pgfscope}%
\pgfpathrectangle{\pgfqpoint{0.800000in}{0.528000in}}{\pgfqpoint{4.960000in}{3.696000in}}%
\pgfusepath{clip}%
\pgfsetrectcap%
\pgfsetroundjoin%
\pgfsetlinewidth{0.803000pt}%
\definecolor{currentstroke}{rgb}{0.690196,0.690196,0.690196}%
\pgfsetstrokecolor{currentstroke}%
\pgfsetdash{}{0pt}%
\pgfpathmoveto{\pgfqpoint{0.800000in}{0.739045in}}%
\pgfpathlineto{\pgfqpoint{5.760000in}{0.739045in}}%
\pgfusepath{stroke}%
\end{pgfscope}%
\begin{pgfscope}%
\pgfsetbuttcap%
\pgfsetroundjoin%
\definecolor{currentfill}{rgb}{0.000000,0.000000,0.000000}%
\pgfsetfillcolor{currentfill}%
\pgfsetlinewidth{0.803000pt}%
\definecolor{currentstroke}{rgb}{0.000000,0.000000,0.000000}%
\pgfsetstrokecolor{currentstroke}%
\pgfsetdash{}{0pt}%
\pgfsys@defobject{currentmarker}{\pgfqpoint{-0.048611in}{0.000000in}}{\pgfqpoint{-0.000000in}{0.000000in}}{%
\pgfpathmoveto{\pgfqpoint{-0.000000in}{0.000000in}}%
\pgfpathlineto{\pgfqpoint{-0.048611in}{0.000000in}}%
\pgfusepath{stroke,fill}%
}%
\begin{pgfscope}%
\pgfsys@transformshift{0.800000in}{0.739045in}%
\pgfsys@useobject{currentmarker}{}%
\end{pgfscope}%
\end{pgfscope}%
\begin{pgfscope}%
\definecolor{textcolor}{rgb}{0.000000,0.000000,0.000000}%
\pgfsetstrokecolor{textcolor}%
\pgfsetfillcolor{textcolor}%
\pgftext[x=0.285508in, y=0.686283in, left, base]{\color{textcolor}\sffamily\fontsize{10.000000}{12.000000}\selectfont \ensuremath{-}0.20}%
\end{pgfscope}%
\begin{pgfscope}%
\pgfpathrectangle{\pgfqpoint{0.800000in}{0.528000in}}{\pgfqpoint{4.960000in}{3.696000in}}%
\pgfusepath{clip}%
\pgfsetrectcap%
\pgfsetroundjoin%
\pgfsetlinewidth{0.803000pt}%
\definecolor{currentstroke}{rgb}{0.690196,0.690196,0.690196}%
\pgfsetstrokecolor{currentstroke}%
\pgfsetdash{}{0pt}%
\pgfpathmoveto{\pgfqpoint{0.800000in}{1.326297in}}%
\pgfpathlineto{\pgfqpoint{5.760000in}{1.326297in}}%
\pgfusepath{stroke}%
\end{pgfscope}%
\begin{pgfscope}%
\pgfsetbuttcap%
\pgfsetroundjoin%
\definecolor{currentfill}{rgb}{0.000000,0.000000,0.000000}%
\pgfsetfillcolor{currentfill}%
\pgfsetlinewidth{0.803000pt}%
\definecolor{currentstroke}{rgb}{0.000000,0.000000,0.000000}%
\pgfsetstrokecolor{currentstroke}%
\pgfsetdash{}{0pt}%
\pgfsys@defobject{currentmarker}{\pgfqpoint{-0.048611in}{0.000000in}}{\pgfqpoint{-0.000000in}{0.000000in}}{%
\pgfpathmoveto{\pgfqpoint{-0.000000in}{0.000000in}}%
\pgfpathlineto{\pgfqpoint{-0.048611in}{0.000000in}}%
\pgfusepath{stroke,fill}%
}%
\begin{pgfscope}%
\pgfsys@transformshift{0.800000in}{1.326297in}%
\pgfsys@useobject{currentmarker}{}%
\end{pgfscope}%
\end{pgfscope}%
\begin{pgfscope}%
\definecolor{textcolor}{rgb}{0.000000,0.000000,0.000000}%
\pgfsetstrokecolor{textcolor}%
\pgfsetfillcolor{textcolor}%
\pgftext[x=0.285508in, y=1.273536in, left, base]{\color{textcolor}\sffamily\fontsize{10.000000}{12.000000}\selectfont \ensuremath{-}0.15}%
\end{pgfscope}%
\begin{pgfscope}%
\pgfpathrectangle{\pgfqpoint{0.800000in}{0.528000in}}{\pgfqpoint{4.960000in}{3.696000in}}%
\pgfusepath{clip}%
\pgfsetrectcap%
\pgfsetroundjoin%
\pgfsetlinewidth{0.803000pt}%
\definecolor{currentstroke}{rgb}{0.690196,0.690196,0.690196}%
\pgfsetstrokecolor{currentstroke}%
\pgfsetdash{}{0pt}%
\pgfpathmoveto{\pgfqpoint{0.800000in}{1.913550in}}%
\pgfpathlineto{\pgfqpoint{5.760000in}{1.913550in}}%
\pgfusepath{stroke}%
\end{pgfscope}%
\begin{pgfscope}%
\pgfsetbuttcap%
\pgfsetroundjoin%
\definecolor{currentfill}{rgb}{0.000000,0.000000,0.000000}%
\pgfsetfillcolor{currentfill}%
\pgfsetlinewidth{0.803000pt}%
\definecolor{currentstroke}{rgb}{0.000000,0.000000,0.000000}%
\pgfsetstrokecolor{currentstroke}%
\pgfsetdash{}{0pt}%
\pgfsys@defobject{currentmarker}{\pgfqpoint{-0.048611in}{0.000000in}}{\pgfqpoint{-0.000000in}{0.000000in}}{%
\pgfpathmoveto{\pgfqpoint{-0.000000in}{0.000000in}}%
\pgfpathlineto{\pgfqpoint{-0.048611in}{0.000000in}}%
\pgfusepath{stroke,fill}%
}%
\begin{pgfscope}%
\pgfsys@transformshift{0.800000in}{1.913550in}%
\pgfsys@useobject{currentmarker}{}%
\end{pgfscope}%
\end{pgfscope}%
\begin{pgfscope}%
\definecolor{textcolor}{rgb}{0.000000,0.000000,0.000000}%
\pgfsetstrokecolor{textcolor}%
\pgfsetfillcolor{textcolor}%
\pgftext[x=0.285508in, y=1.860788in, left, base]{\color{textcolor}\sffamily\fontsize{10.000000}{12.000000}\selectfont \ensuremath{-}0.10}%
\end{pgfscope}%
\begin{pgfscope}%
\pgfpathrectangle{\pgfqpoint{0.800000in}{0.528000in}}{\pgfqpoint{4.960000in}{3.696000in}}%
\pgfusepath{clip}%
\pgfsetrectcap%
\pgfsetroundjoin%
\pgfsetlinewidth{0.803000pt}%
\definecolor{currentstroke}{rgb}{0.690196,0.690196,0.690196}%
\pgfsetstrokecolor{currentstroke}%
\pgfsetdash{}{0pt}%
\pgfpathmoveto{\pgfqpoint{0.800000in}{2.500802in}}%
\pgfpathlineto{\pgfqpoint{5.760000in}{2.500802in}}%
\pgfusepath{stroke}%
\end{pgfscope}%
\begin{pgfscope}%
\pgfsetbuttcap%
\pgfsetroundjoin%
\definecolor{currentfill}{rgb}{0.000000,0.000000,0.000000}%
\pgfsetfillcolor{currentfill}%
\pgfsetlinewidth{0.803000pt}%
\definecolor{currentstroke}{rgb}{0.000000,0.000000,0.000000}%
\pgfsetstrokecolor{currentstroke}%
\pgfsetdash{}{0pt}%
\pgfsys@defobject{currentmarker}{\pgfqpoint{-0.048611in}{0.000000in}}{\pgfqpoint{-0.000000in}{0.000000in}}{%
\pgfpathmoveto{\pgfqpoint{-0.000000in}{0.000000in}}%
\pgfpathlineto{\pgfqpoint{-0.048611in}{0.000000in}}%
\pgfusepath{stroke,fill}%
}%
\begin{pgfscope}%
\pgfsys@transformshift{0.800000in}{2.500802in}%
\pgfsys@useobject{currentmarker}{}%
\end{pgfscope}%
\end{pgfscope}%
\begin{pgfscope}%
\definecolor{textcolor}{rgb}{0.000000,0.000000,0.000000}%
\pgfsetstrokecolor{textcolor}%
\pgfsetfillcolor{textcolor}%
\pgftext[x=0.285508in, y=2.448041in, left, base]{\color{textcolor}\sffamily\fontsize{10.000000}{12.000000}\selectfont \ensuremath{-}0.05}%
\end{pgfscope}%
\begin{pgfscope}%
\pgfpathrectangle{\pgfqpoint{0.800000in}{0.528000in}}{\pgfqpoint{4.960000in}{3.696000in}}%
\pgfusepath{clip}%
\pgfsetrectcap%
\pgfsetroundjoin%
\pgfsetlinewidth{0.803000pt}%
\definecolor{currentstroke}{rgb}{0.690196,0.690196,0.690196}%
\pgfsetstrokecolor{currentstroke}%
\pgfsetdash{}{0pt}%
\pgfpathmoveto{\pgfqpoint{0.800000in}{3.088055in}}%
\pgfpathlineto{\pgfqpoint{5.760000in}{3.088055in}}%
\pgfusepath{stroke}%
\end{pgfscope}%
\begin{pgfscope}%
\pgfsetbuttcap%
\pgfsetroundjoin%
\definecolor{currentfill}{rgb}{0.000000,0.000000,0.000000}%
\pgfsetfillcolor{currentfill}%
\pgfsetlinewidth{0.803000pt}%
\definecolor{currentstroke}{rgb}{0.000000,0.000000,0.000000}%
\pgfsetstrokecolor{currentstroke}%
\pgfsetdash{}{0pt}%
\pgfsys@defobject{currentmarker}{\pgfqpoint{-0.048611in}{0.000000in}}{\pgfqpoint{-0.000000in}{0.000000in}}{%
\pgfpathmoveto{\pgfqpoint{-0.000000in}{0.000000in}}%
\pgfpathlineto{\pgfqpoint{-0.048611in}{0.000000in}}%
\pgfusepath{stroke,fill}%
}%
\begin{pgfscope}%
\pgfsys@transformshift{0.800000in}{3.088055in}%
\pgfsys@useobject{currentmarker}{}%
\end{pgfscope}%
\end{pgfscope}%
\begin{pgfscope}%
\definecolor{textcolor}{rgb}{0.000000,0.000000,0.000000}%
\pgfsetstrokecolor{textcolor}%
\pgfsetfillcolor{textcolor}%
\pgftext[x=0.393533in, y=3.035293in, left, base]{\color{textcolor}\sffamily\fontsize{10.000000}{12.000000}\selectfont 0.00}%
\end{pgfscope}%
\begin{pgfscope}%
\pgfpathrectangle{\pgfqpoint{0.800000in}{0.528000in}}{\pgfqpoint{4.960000in}{3.696000in}}%
\pgfusepath{clip}%
\pgfsetrectcap%
\pgfsetroundjoin%
\pgfsetlinewidth{0.803000pt}%
\definecolor{currentstroke}{rgb}{0.690196,0.690196,0.690196}%
\pgfsetstrokecolor{currentstroke}%
\pgfsetdash{}{0pt}%
\pgfpathmoveto{\pgfqpoint{0.800000in}{3.675307in}}%
\pgfpathlineto{\pgfqpoint{5.760000in}{3.675307in}}%
\pgfusepath{stroke}%
\end{pgfscope}%
\begin{pgfscope}%
\pgfsetbuttcap%
\pgfsetroundjoin%
\definecolor{currentfill}{rgb}{0.000000,0.000000,0.000000}%
\pgfsetfillcolor{currentfill}%
\pgfsetlinewidth{0.803000pt}%
\definecolor{currentstroke}{rgb}{0.000000,0.000000,0.000000}%
\pgfsetstrokecolor{currentstroke}%
\pgfsetdash{}{0pt}%
\pgfsys@defobject{currentmarker}{\pgfqpoint{-0.048611in}{0.000000in}}{\pgfqpoint{-0.000000in}{0.000000in}}{%
\pgfpathmoveto{\pgfqpoint{-0.000000in}{0.000000in}}%
\pgfpathlineto{\pgfqpoint{-0.048611in}{0.000000in}}%
\pgfusepath{stroke,fill}%
}%
\begin{pgfscope}%
\pgfsys@transformshift{0.800000in}{3.675307in}%
\pgfsys@useobject{currentmarker}{}%
\end{pgfscope}%
\end{pgfscope}%
\begin{pgfscope}%
\definecolor{textcolor}{rgb}{0.000000,0.000000,0.000000}%
\pgfsetstrokecolor{textcolor}%
\pgfsetfillcolor{textcolor}%
\pgftext[x=0.393533in, y=3.622546in, left, base]{\color{textcolor}\sffamily\fontsize{10.000000}{12.000000}\selectfont 0.05}%
\end{pgfscope}%
\begin{pgfscope}%
\definecolor{textcolor}{rgb}{0.000000,0.000000,0.000000}%
\pgfsetstrokecolor{textcolor}%
\pgfsetfillcolor{textcolor}%
\pgftext[x=0.229952in,y=2.376000in,,bottom,rotate=90.000000]{\color{textcolor}\sffamily\fontsize{10.000000}{12.000000}\selectfont Computed error [-]}%
\end{pgfscope}%
\begin{pgfscope}%
\pgfpathrectangle{\pgfqpoint{0.800000in}{0.528000in}}{\pgfqpoint{4.960000in}{3.696000in}}%
\pgfusepath{clip}%
\pgfsetrectcap%
\pgfsetroundjoin%
\pgfsetlinewidth{1.505625pt}%
\definecolor{currentstroke}{rgb}{0.121569,0.466667,0.705882}%
\pgfsetstrokecolor{currentstroke}%
\pgfsetdash{}{0pt}%
\pgfpathmoveto{\pgfqpoint{1.025455in}{0.704964in}}%
\pgfpathlineto{\pgfqpoint{1.079665in}{0.707619in}}%
\pgfpathlineto{\pgfqpoint{1.134113in}{0.696000in}}%
\pgfpathlineto{\pgfqpoint{1.188117in}{0.882687in}}%
\pgfpathlineto{\pgfqpoint{1.242157in}{1.326460in}}%
\pgfpathlineto{\pgfqpoint{1.296248in}{1.867642in}}%
\pgfpathlineto{\pgfqpoint{1.350915in}{2.324845in}}%
\pgfpathlineto{\pgfqpoint{1.403640in}{2.607582in}}%
\pgfpathlineto{\pgfqpoint{1.457878in}{2.779926in}}%
\pgfpathlineto{\pgfqpoint{1.512411in}{3.096449in}}%
\pgfpathlineto{\pgfqpoint{1.567710in}{3.240306in}}%
\pgfpathlineto{\pgfqpoint{1.621592in}{3.391136in}}%
\pgfpathlineto{\pgfqpoint{1.675033in}{3.487918in}}%
\pgfpathlineto{\pgfqpoint{1.729231in}{3.605609in}}%
\pgfpathlineto{\pgfqpoint{1.784021in}{3.745263in}}%
\pgfpathlineto{\pgfqpoint{1.837979in}{3.859923in}}%
\pgfpathlineto{\pgfqpoint{1.891832in}{3.909563in}}%
\pgfpathlineto{\pgfqpoint{1.946166in}{3.903791in}}%
\pgfpathlineto{\pgfqpoint{2.000280in}{3.910723in}}%
\pgfpathlineto{\pgfqpoint{2.054665in}{3.853117in}}%
\pgfpathlineto{\pgfqpoint{2.109029in}{3.817285in}}%
\pgfpathlineto{\pgfqpoint{2.164495in}{3.830459in}}%
\pgfpathlineto{\pgfqpoint{2.217998in}{3.854784in}}%
\pgfpathlineto{\pgfqpoint{2.271973in}{3.854208in}}%
\pgfpathlineto{\pgfqpoint{2.325948in}{3.857892in}}%
\pgfpathlineto{\pgfqpoint{2.380122in}{3.777583in}}%
\pgfpathlineto{\pgfqpoint{2.434309in}{3.616419in}}%
\pgfpathlineto{\pgfqpoint{2.488488in}{3.497289in}}%
\pgfpathlineto{\pgfqpoint{2.543866in}{3.352264in}}%
\pgfpathlineto{\pgfqpoint{2.597224in}{3.336088in}}%
\pgfpathlineto{\pgfqpoint{2.651231in}{3.306109in}}%
\pgfpathlineto{\pgfqpoint{2.707093in}{3.309190in}}%
\pgfpathlineto{\pgfqpoint{2.760681in}{3.324247in}}%
\pgfpathlineto{\pgfqpoint{2.814881in}{3.304663in}}%
\pgfpathlineto{\pgfqpoint{2.869226in}{3.314168in}}%
\pgfpathlineto{\pgfqpoint{2.923559in}{3.299734in}}%
\pgfpathlineto{\pgfqpoint{2.977685in}{3.311186in}}%
\pgfpathlineto{\pgfqpoint{3.032178in}{3.328935in}}%
\pgfpathlineto{\pgfqpoint{3.086584in}{3.290954in}}%
\pgfpathlineto{\pgfqpoint{3.141612in}{3.237593in}}%
\pgfpathlineto{\pgfqpoint{3.195303in}{3.253576in}}%
\pgfpathlineto{\pgfqpoint{3.250049in}{3.215093in}}%
\pgfpathlineto{\pgfqpoint{3.304703in}{3.184376in}}%
\pgfpathlineto{\pgfqpoint{3.359114in}{2.959906in}}%
\pgfpathlineto{\pgfqpoint{3.414741in}{2.878726in}}%
\pgfpathlineto{\pgfqpoint{3.466608in}{2.923160in}}%
\pgfpathlineto{\pgfqpoint{3.522268in}{2.959145in}}%
\pgfpathlineto{\pgfqpoint{3.577082in}{2.985106in}}%
\pgfpathlineto{\pgfqpoint{3.632050in}{2.996693in}}%
\pgfpathlineto{\pgfqpoint{3.689200in}{3.004273in}}%
\pgfpathlineto{\pgfqpoint{3.741909in}{2.990059in}}%
\pgfpathlineto{\pgfqpoint{3.795389in}{2.985123in}}%
\pgfpathlineto{\pgfqpoint{3.849246in}{3.255381in}}%
\pgfpathlineto{\pgfqpoint{3.902955in}{3.084146in}}%
\pgfpathlineto{\pgfqpoint{3.957232in}{3.198016in}}%
\pgfpathlineto{\pgfqpoint{4.011673in}{3.150233in}}%
\pgfpathlineto{\pgfqpoint{4.066015in}{3.163611in}}%
\pgfpathlineto{\pgfqpoint{4.119961in}{3.147443in}}%
\pgfpathlineto{\pgfqpoint{4.174047in}{3.098067in}}%
\pgfpathlineto{\pgfqpoint{4.228304in}{2.858330in}}%
\pgfpathlineto{\pgfqpoint{4.282576in}{2.861353in}}%
\pgfpathlineto{\pgfqpoint{4.336784in}{2.929360in}}%
\pgfpathlineto{\pgfqpoint{4.391348in}{2.920872in}}%
\pgfpathlineto{\pgfqpoint{4.445504in}{2.914004in}}%
\pgfpathlineto{\pgfqpoint{4.499131in}{2.908963in}}%
\pgfpathlineto{\pgfqpoint{4.553874in}{2.912783in}}%
\pgfpathlineto{\pgfqpoint{4.608189in}{3.191013in}}%
\pgfpathlineto{\pgfqpoint{4.663418in}{3.039498in}}%
\pgfpathlineto{\pgfqpoint{4.717330in}{3.046443in}}%
\pgfpathlineto{\pgfqpoint{4.770855in}{3.074770in}}%
\pgfpathlineto{\pgfqpoint{4.825033in}{3.101320in}}%
\pgfpathlineto{\pgfqpoint{4.879404in}{3.157061in}}%
\pgfpathlineto{\pgfqpoint{4.933681in}{3.177838in}}%
\pgfpathlineto{\pgfqpoint{4.987637in}{3.153900in}}%
\pgfpathlineto{\pgfqpoint{5.041976in}{3.125376in}}%
\pgfpathlineto{\pgfqpoint{5.095774in}{3.118050in}}%
\pgfpathlineto{\pgfqpoint{5.150594in}{3.125954in}}%
\pgfpathlineto{\pgfqpoint{5.204911in}{3.107283in}}%
\pgfpathlineto{\pgfqpoint{5.260666in}{3.109746in}}%
\pgfpathlineto{\pgfqpoint{5.313665in}{3.135809in}}%
\pgfpathlineto{\pgfqpoint{5.367797in}{3.148505in}}%
\pgfpathlineto{\pgfqpoint{5.421994in}{3.143660in}}%
\pgfpathlineto{\pgfqpoint{5.475988in}{3.139032in}}%
\pgfpathlineto{\pgfqpoint{5.530255in}{3.137536in}}%
\pgfusepath{stroke}%
\end{pgfscope}%
\begin{pgfscope}%
\pgfpathrectangle{\pgfqpoint{0.800000in}{0.528000in}}{\pgfqpoint{4.960000in}{3.696000in}}%
\pgfusepath{clip}%
\pgfsetrectcap%
\pgfsetroundjoin%
\pgfsetlinewidth{1.505625pt}%
\definecolor{currentstroke}{rgb}{1.000000,0.498039,0.054902}%
\pgfsetstrokecolor{currentstroke}%
\pgfsetdash{}{0pt}%
\pgfpathmoveto{\pgfqpoint{1.025455in}{0.738706in}}%
\pgfpathlineto{\pgfqpoint{1.078263in}{1.025426in}}%
\pgfpathlineto{\pgfqpoint{1.133010in}{0.933320in}}%
\pgfpathlineto{\pgfqpoint{1.187483in}{0.831365in}}%
\pgfpathlineto{\pgfqpoint{1.242258in}{1.037644in}}%
\pgfpathlineto{\pgfqpoint{1.296643in}{1.486826in}}%
\pgfpathlineto{\pgfqpoint{1.350939in}{2.135995in}}%
\pgfpathlineto{\pgfqpoint{1.403901in}{2.669911in}}%
\pgfpathlineto{\pgfqpoint{1.459510in}{2.945084in}}%
\pgfpathlineto{\pgfqpoint{1.513139in}{3.286102in}}%
\pgfpathlineto{\pgfqpoint{1.567644in}{3.252576in}}%
\pgfpathlineto{\pgfqpoint{1.621618in}{2.988347in}}%
\pgfpathlineto{\pgfqpoint{1.675729in}{2.976586in}}%
\pgfpathlineto{\pgfqpoint{1.729895in}{3.071882in}}%
\pgfpathlineto{\pgfqpoint{1.784179in}{3.119149in}}%
\pgfpathlineto{\pgfqpoint{1.838904in}{3.174644in}}%
\pgfpathlineto{\pgfqpoint{1.893129in}{3.124404in}}%
\pgfpathlineto{\pgfqpoint{1.947745in}{3.182020in}}%
\pgfpathlineto{\pgfqpoint{2.003392in}{3.122274in}}%
\pgfpathlineto{\pgfqpoint{2.056611in}{3.172802in}}%
\pgfpathlineto{\pgfqpoint{2.110648in}{3.086503in}}%
\pgfpathlineto{\pgfqpoint{2.164726in}{2.936684in}}%
\pgfpathlineto{\pgfqpoint{2.218626in}{2.956152in}}%
\pgfpathlineto{\pgfqpoint{2.272733in}{2.918588in}}%
\pgfpathlineto{\pgfqpoint{2.327130in}{2.902112in}}%
\pgfpathlineto{\pgfqpoint{2.381217in}{2.920565in}}%
\pgfpathlineto{\pgfqpoint{2.435587in}{2.935005in}}%
\pgfpathlineto{\pgfqpoint{2.489898in}{2.957817in}}%
\pgfpathlineto{\pgfqpoint{2.545040in}{3.035025in}}%
\pgfpathlineto{\pgfqpoint{2.600426in}{3.054884in}}%
\pgfpathlineto{\pgfqpoint{2.653686in}{3.035866in}}%
\pgfpathlineto{\pgfqpoint{2.707725in}{3.024864in}}%
\pgfpathlineto{\pgfqpoint{2.761988in}{3.050461in}}%
\pgfpathlineto{\pgfqpoint{2.816260in}{3.116042in}}%
\pgfpathlineto{\pgfqpoint{2.870425in}{3.171149in}}%
\pgfpathlineto{\pgfqpoint{2.924372in}{3.148116in}}%
\pgfpathlineto{\pgfqpoint{2.978748in}{3.143630in}}%
\pgfpathlineto{\pgfqpoint{3.033313in}{3.142102in}}%
\pgfpathlineto{\pgfqpoint{3.087307in}{3.127465in}}%
\pgfpathlineto{\pgfqpoint{3.142796in}{3.056160in}}%
\pgfpathlineto{\pgfqpoint{3.196681in}{2.988896in}}%
\pgfpathlineto{\pgfqpoint{3.250528in}{3.030083in}}%
\pgfpathlineto{\pgfqpoint{3.304817in}{3.024438in}}%
\pgfpathlineto{\pgfqpoint{3.359166in}{3.025979in}}%
\pgfpathlineto{\pgfqpoint{3.412991in}{3.084520in}}%
\pgfpathlineto{\pgfqpoint{3.467110in}{3.090588in}}%
\pgfpathlineto{\pgfqpoint{3.521390in}{3.120416in}}%
\pgfpathlineto{\pgfqpoint{3.575431in}{3.129467in}}%
\pgfpathlineto{\pgfqpoint{3.630336in}{3.115781in}}%
\pgfpathlineto{\pgfqpoint{3.686150in}{3.131325in}}%
\pgfpathlineto{\pgfqpoint{3.739878in}{3.122569in}}%
\pgfpathlineto{\pgfqpoint{3.794228in}{3.115872in}}%
\pgfpathlineto{\pgfqpoint{3.848984in}{3.121227in}}%
\pgfpathlineto{\pgfqpoint{3.903165in}{3.102267in}}%
\pgfpathlineto{\pgfqpoint{3.957354in}{3.096878in}}%
\pgfpathlineto{\pgfqpoint{4.011160in}{3.091832in}}%
\pgfpathlineto{\pgfqpoint{4.065633in}{3.068554in}}%
\pgfpathlineto{\pgfqpoint{4.120192in}{3.068625in}}%
\pgfpathlineto{\pgfqpoint{4.174830in}{3.058345in}}%
\pgfpathlineto{\pgfqpoint{4.230258in}{3.052211in}}%
\pgfpathlineto{\pgfqpoint{4.283868in}{3.045209in}}%
\pgfpathlineto{\pgfqpoint{4.337679in}{3.028341in}}%
\pgfpathlineto{\pgfqpoint{4.391628in}{3.035390in}}%
\pgfpathlineto{\pgfqpoint{4.445754in}{3.029304in}}%
\pgfpathlineto{\pgfqpoint{4.500264in}{3.029569in}}%
\pgfpathlineto{\pgfqpoint{4.554089in}{3.067522in}}%
\pgfpathlineto{\pgfqpoint{4.608092in}{3.068762in}}%
\pgfpathlineto{\pgfqpoint{4.662545in}{3.075570in}}%
\pgfpathlineto{\pgfqpoint{4.716756in}{3.089677in}}%
\pgfpathlineto{\pgfqpoint{4.770986in}{3.091916in}}%
\pgfpathlineto{\pgfqpoint{4.826964in}{3.113218in}}%
\pgfpathlineto{\pgfqpoint{4.880439in}{3.108553in}}%
\pgfpathlineto{\pgfqpoint{4.934083in}{3.106946in}}%
\pgfpathlineto{\pgfqpoint{4.987905in}{3.104829in}}%
\pgfpathlineto{\pgfqpoint{5.042036in}{3.103460in}}%
\pgfpathlineto{\pgfqpoint{5.096206in}{3.082070in}}%
\pgfpathlineto{\pgfqpoint{5.150197in}{3.054612in}}%
\pgfpathlineto{\pgfqpoint{5.204296in}{3.044502in}}%
\pgfpathlineto{\pgfqpoint{5.258850in}{3.052415in}}%
\pgfpathlineto{\pgfqpoint{5.312880in}{3.037935in}}%
\pgfpathlineto{\pgfqpoint{5.367388in}{3.045006in}}%
\pgfpathlineto{\pgfqpoint{5.421613in}{3.045729in}}%
\pgfpathlineto{\pgfqpoint{5.477603in}{3.051516in}}%
\pgfpathlineto{\pgfqpoint{5.530472in}{3.080729in}}%
\pgfusepath{stroke}%
\end{pgfscope}%
\begin{pgfscope}%
\pgfpathrectangle{\pgfqpoint{0.800000in}{0.528000in}}{\pgfqpoint{4.960000in}{3.696000in}}%
\pgfusepath{clip}%
\pgfsetrectcap%
\pgfsetroundjoin%
\pgfsetlinewidth{1.505625pt}%
\definecolor{currentstroke}{rgb}{0.172549,0.627451,0.172549}%
\pgfsetstrokecolor{currentstroke}%
\pgfsetdash{}{0pt}%
\pgfpathmoveto{\pgfqpoint{1.025455in}{0.810325in}}%
\pgfpathlineto{\pgfqpoint{1.079614in}{1.269140in}}%
\pgfpathlineto{\pgfqpoint{1.133861in}{1.133177in}}%
\pgfpathlineto{\pgfqpoint{1.187787in}{1.185561in}}%
\pgfpathlineto{\pgfqpoint{1.241874in}{1.275697in}}%
\pgfpathlineto{\pgfqpoint{1.296152in}{1.953510in}}%
\pgfpathlineto{\pgfqpoint{1.350782in}{2.423157in}}%
\pgfpathlineto{\pgfqpoint{1.404285in}{2.776166in}}%
\pgfpathlineto{\pgfqpoint{1.458732in}{2.850548in}}%
\pgfpathlineto{\pgfqpoint{1.512087in}{3.048514in}}%
\pgfpathlineto{\pgfqpoint{1.566065in}{3.239502in}}%
\pgfpathlineto{\pgfqpoint{1.620418in}{3.353095in}}%
\pgfpathlineto{\pgfqpoint{1.674617in}{3.338844in}}%
\pgfpathlineto{\pgfqpoint{1.729149in}{3.378106in}}%
\pgfpathlineto{\pgfqpoint{1.783586in}{3.462432in}}%
\pgfpathlineto{\pgfqpoint{1.838516in}{3.533714in}}%
\pgfpathlineto{\pgfqpoint{1.892359in}{3.515371in}}%
\pgfpathlineto{\pgfqpoint{1.948823in}{3.532615in}}%
\pgfpathlineto{\pgfqpoint{2.001942in}{3.557044in}}%
\pgfpathlineto{\pgfqpoint{2.056759in}{3.585480in}}%
\pgfpathlineto{\pgfqpoint{2.109781in}{3.594675in}}%
\pgfpathlineto{\pgfqpoint{2.163628in}{3.574777in}}%
\pgfpathlineto{\pgfqpoint{2.217534in}{3.538267in}}%
\pgfpathlineto{\pgfqpoint{2.271782in}{3.557364in}}%
\pgfpathlineto{\pgfqpoint{2.326253in}{3.580846in}}%
\pgfpathlineto{\pgfqpoint{2.380419in}{3.554098in}}%
\pgfpathlineto{\pgfqpoint{2.434688in}{3.552365in}}%
\pgfpathlineto{\pgfqpoint{2.488799in}{3.549108in}}%
\pgfpathlineto{\pgfqpoint{2.544077in}{3.541481in}}%
\pgfpathlineto{\pgfqpoint{2.597819in}{3.524293in}}%
\pgfpathlineto{\pgfqpoint{2.651738in}{3.476264in}}%
\pgfpathlineto{\pgfqpoint{2.706232in}{3.380200in}}%
\pgfpathlineto{\pgfqpoint{2.760092in}{3.450644in}}%
\pgfpathlineto{\pgfqpoint{2.814267in}{3.351444in}}%
\pgfpathlineto{\pgfqpoint{2.868473in}{3.431407in}}%
\pgfpathlineto{\pgfqpoint{2.923125in}{3.327817in}}%
\pgfpathlineto{\pgfqpoint{2.976992in}{3.357733in}}%
\pgfpathlineto{\pgfqpoint{3.031409in}{3.358169in}}%
\pgfpathlineto{\pgfqpoint{3.085485in}{3.350819in}}%
\pgfpathlineto{\pgfqpoint{3.140750in}{3.357884in}}%
\pgfpathlineto{\pgfqpoint{3.194496in}{3.325922in}}%
\pgfpathlineto{\pgfqpoint{3.249577in}{3.333509in}}%
\pgfpathlineto{\pgfqpoint{3.304646in}{3.348863in}}%
\pgfpathlineto{\pgfqpoint{3.358086in}{3.072856in}}%
\pgfpathlineto{\pgfqpoint{3.413304in}{3.043948in}}%
\pgfpathlineto{\pgfqpoint{3.468099in}{3.075903in}}%
\pgfpathlineto{\pgfqpoint{3.522732in}{3.074482in}}%
\pgfpathlineto{\pgfqpoint{3.575691in}{3.011674in}}%
\pgfpathlineto{\pgfqpoint{3.629541in}{2.923438in}}%
\pgfpathlineto{\pgfqpoint{3.685026in}{2.885737in}}%
\pgfpathlineto{\pgfqpoint{3.737242in}{2.904139in}}%
\pgfpathlineto{\pgfqpoint{3.791852in}{2.958775in}}%
\pgfpathlineto{\pgfqpoint{3.845706in}{2.882507in}}%
\pgfpathlineto{\pgfqpoint{3.899946in}{2.932770in}}%
\pgfpathlineto{\pgfqpoint{3.954002in}{2.913074in}}%
\pgfpathlineto{\pgfqpoint{4.008251in}{2.943097in}}%
\pgfpathlineto{\pgfqpoint{4.062421in}{2.939644in}}%
\pgfpathlineto{\pgfqpoint{4.118106in}{2.966805in}}%
\pgfpathlineto{\pgfqpoint{4.171682in}{2.970052in}}%
\pgfpathlineto{\pgfqpoint{4.225578in}{2.967209in}}%
\pgfpathlineto{\pgfqpoint{4.279671in}{2.958583in}}%
\pgfpathlineto{\pgfqpoint{4.333831in}{2.970753in}}%
\pgfpathlineto{\pgfqpoint{4.387845in}{3.000429in}}%
\pgfpathlineto{\pgfqpoint{4.442429in}{3.056373in}}%
\pgfpathlineto{\pgfqpoint{4.496550in}{3.345076in}}%
\pgfpathlineto{\pgfqpoint{4.550875in}{3.075679in}}%
\pgfpathlineto{\pgfqpoint{4.605211in}{3.272946in}}%
\pgfpathlineto{\pgfqpoint{4.659385in}{3.338881in}}%
\pgfpathlineto{\pgfqpoint{4.713258in}{3.261171in}}%
\pgfpathlineto{\pgfqpoint{4.768334in}{3.163506in}}%
\pgfpathlineto{\pgfqpoint{4.822833in}{3.124361in}}%
\pgfpathlineto{\pgfqpoint{4.876677in}{3.071699in}}%
\pgfpathlineto{\pgfqpoint{4.931343in}{3.015188in}}%
\pgfpathlineto{\pgfqpoint{4.985117in}{3.007643in}}%
\pgfpathlineto{\pgfqpoint{5.039326in}{3.015400in}}%
\pgfpathlineto{\pgfqpoint{5.093879in}{2.972577in}}%
\pgfpathlineto{\pgfqpoint{5.148077in}{2.876364in}}%
\pgfpathlineto{\pgfqpoint{5.202190in}{2.890062in}}%
\pgfpathlineto{\pgfqpoint{5.257082in}{2.938369in}}%
\pgfpathlineto{\pgfqpoint{5.311203in}{2.985615in}}%
\pgfpathlineto{\pgfqpoint{5.367165in}{2.956788in}}%
\pgfpathlineto{\pgfqpoint{5.420100in}{2.965090in}}%
\pgfpathlineto{\pgfqpoint{5.473931in}{2.978130in}}%
\pgfpathlineto{\pgfqpoint{5.527868in}{3.014094in}}%
\pgfusepath{stroke}%
\end{pgfscope}%
\begin{pgfscope}%
\pgfpathrectangle{\pgfqpoint{0.800000in}{0.528000in}}{\pgfqpoint{4.960000in}{3.696000in}}%
\pgfusepath{clip}%
\pgfsetrectcap%
\pgfsetroundjoin%
\pgfsetlinewidth{1.505625pt}%
\definecolor{currentstroke}{rgb}{0.839216,0.152941,0.156863}%
\pgfsetstrokecolor{currentstroke}%
\pgfsetdash{}{0pt}%
\pgfpathmoveto{\pgfqpoint{1.025455in}{0.913941in}}%
\pgfpathlineto{\pgfqpoint{1.079597in}{1.075825in}}%
\pgfpathlineto{\pgfqpoint{1.133658in}{1.052554in}}%
\pgfpathlineto{\pgfqpoint{1.189329in}{1.088935in}}%
\pgfpathlineto{\pgfqpoint{1.242761in}{1.230768in}}%
\pgfpathlineto{\pgfqpoint{1.296697in}{1.566851in}}%
\pgfpathlineto{\pgfqpoint{1.349591in}{2.171420in}}%
\pgfpathlineto{\pgfqpoint{1.403833in}{2.761059in}}%
\pgfpathlineto{\pgfqpoint{1.458008in}{3.422123in}}%
\pgfpathlineto{\pgfqpoint{1.512487in}{3.189810in}}%
\pgfpathlineto{\pgfqpoint{1.567559in}{3.091952in}}%
\pgfpathlineto{\pgfqpoint{1.620413in}{3.200236in}}%
\pgfpathlineto{\pgfqpoint{1.674637in}{3.412277in}}%
\pgfpathlineto{\pgfqpoint{1.728965in}{3.448717in}}%
\pgfpathlineto{\pgfqpoint{1.785241in}{3.610631in}}%
\pgfpathlineto{\pgfqpoint{1.839465in}{3.559546in}}%
\pgfpathlineto{\pgfqpoint{1.893575in}{3.503192in}}%
\pgfpathlineto{\pgfqpoint{1.948524in}{3.596277in}}%
\pgfpathlineto{\pgfqpoint{2.002701in}{3.656041in}}%
\pgfpathlineto{\pgfqpoint{2.057640in}{3.603118in}}%
\pgfpathlineto{\pgfqpoint{2.112272in}{3.509386in}}%
\pgfpathlineto{\pgfqpoint{2.166513in}{3.629864in}}%
\pgfpathlineto{\pgfqpoint{2.220857in}{3.635733in}}%
\pgfpathlineto{\pgfqpoint{2.275238in}{3.581404in}}%
\pgfpathlineto{\pgfqpoint{2.329903in}{3.655509in}}%
\pgfpathlineto{\pgfqpoint{2.385513in}{3.542829in}}%
\pgfpathlineto{\pgfqpoint{2.438712in}{3.381722in}}%
\pgfpathlineto{\pgfqpoint{2.492760in}{3.324973in}}%
\pgfpathlineto{\pgfqpoint{2.546778in}{3.369958in}}%
\pgfpathlineto{\pgfqpoint{2.600950in}{3.359104in}}%
\pgfpathlineto{\pgfqpoint{2.655146in}{3.322150in}}%
\pgfpathlineto{\pgfqpoint{2.709202in}{3.087739in}}%
\pgfpathlineto{\pgfqpoint{2.763444in}{2.675423in}}%
\pgfpathlineto{\pgfqpoint{2.817717in}{2.578902in}}%
\pgfpathlineto{\pgfqpoint{2.872018in}{2.977454in}}%
\pgfpathlineto{\pgfqpoint{2.926586in}{3.082050in}}%
\pgfpathlineto{\pgfqpoint{2.980826in}{3.108798in}}%
\pgfpathlineto{\pgfqpoint{3.036130in}{3.047124in}}%
\pgfpathlineto{\pgfqpoint{3.090610in}{3.128674in}}%
\pgfpathlineto{\pgfqpoint{3.145048in}{3.056397in}}%
\pgfpathlineto{\pgfqpoint{3.198567in}{2.893418in}}%
\pgfpathlineto{\pgfqpoint{3.252938in}{2.754277in}}%
\pgfpathlineto{\pgfqpoint{3.306816in}{2.774255in}}%
\pgfpathlineto{\pgfqpoint{3.360919in}{3.006527in}}%
\pgfpathlineto{\pgfqpoint{3.415295in}{3.032235in}}%
\pgfpathlineto{\pgfqpoint{3.473498in}{2.987180in}}%
\pgfpathlineto{\pgfqpoint{3.524214in}{3.113846in}}%
\pgfpathlineto{\pgfqpoint{3.579195in}{3.070643in}}%
\pgfpathlineto{\pgfqpoint{3.632793in}{2.902492in}}%
\pgfpathlineto{\pgfqpoint{3.686740in}{2.999362in}}%
\pgfpathlineto{\pgfqpoint{3.740860in}{3.002461in}}%
\pgfpathlineto{\pgfqpoint{3.794946in}{3.089781in}}%
\pgfpathlineto{\pgfqpoint{3.849077in}{3.107368in}}%
\pgfpathlineto{\pgfqpoint{3.903216in}{3.115312in}}%
\pgfpathlineto{\pgfqpoint{3.957840in}{3.156620in}}%
\pgfpathlineto{\pgfqpoint{4.012092in}{3.168728in}}%
\pgfpathlineto{\pgfqpoint{4.066140in}{3.159072in}}%
\pgfpathlineto{\pgfqpoint{4.122746in}{3.133847in}}%
\pgfpathlineto{\pgfqpoint{4.175990in}{3.094440in}}%
\pgfpathlineto{\pgfqpoint{4.229469in}{3.081983in}}%
\pgfpathlineto{\pgfqpoint{4.284198in}{3.074428in}}%
\pgfpathlineto{\pgfqpoint{4.338336in}{3.069164in}}%
\pgfpathlineto{\pgfqpoint{4.393002in}{3.069940in}}%
\pgfpathlineto{\pgfqpoint{4.447361in}{3.105274in}}%
\pgfpathlineto{\pgfqpoint{4.501391in}{3.098823in}}%
\pgfpathlineto{\pgfqpoint{4.555394in}{3.119451in}}%
\pgfpathlineto{\pgfqpoint{4.609840in}{3.119535in}}%
\pgfpathlineto{\pgfqpoint{4.664142in}{3.122986in}}%
\pgfpathlineto{\pgfqpoint{4.720532in}{3.131527in}}%
\pgfpathlineto{\pgfqpoint{4.773672in}{3.130973in}}%
\pgfpathlineto{\pgfqpoint{4.827513in}{3.138149in}}%
\pgfpathlineto{\pgfqpoint{4.881178in}{3.135033in}}%
\pgfpathlineto{\pgfqpoint{4.935728in}{3.126213in}}%
\pgfpathlineto{\pgfqpoint{4.989451in}{3.145517in}}%
\pgfpathlineto{\pgfqpoint{5.043670in}{3.135264in}}%
\pgfpathlineto{\pgfqpoint{5.097979in}{3.048418in}}%
\pgfpathlineto{\pgfqpoint{5.153158in}{2.975031in}}%
\pgfpathlineto{\pgfqpoint{5.206993in}{2.920854in}}%
\pgfpathlineto{\pgfqpoint{5.261266in}{2.967137in}}%
\pgfpathlineto{\pgfqpoint{5.315995in}{2.916146in}}%
\pgfpathlineto{\pgfqpoint{5.372973in}{2.960887in}}%
\pgfpathlineto{\pgfqpoint{5.426296in}{3.104239in}}%
\pgfpathlineto{\pgfqpoint{5.480510in}{3.032730in}}%
\pgfpathlineto{\pgfqpoint{5.534545in}{3.160240in}}%
\pgfusepath{stroke}%
\end{pgfscope}%
\begin{pgfscope}%
\pgfpathrectangle{\pgfqpoint{0.800000in}{0.528000in}}{\pgfqpoint{4.960000in}{3.696000in}}%
\pgfusepath{clip}%
\pgfsetrectcap%
\pgfsetroundjoin%
\pgfsetlinewidth{1.505625pt}%
\definecolor{currentstroke}{rgb}{0.580392,0.403922,0.741176}%
\pgfsetstrokecolor{currentstroke}%
\pgfsetdash{}{0pt}%
\pgfpathmoveto{\pgfqpoint{1.025455in}{0.800503in}}%
\pgfpathlineto{\pgfqpoint{1.079734in}{1.148063in}}%
\pgfpathlineto{\pgfqpoint{1.133911in}{0.976727in}}%
\pgfpathlineto{\pgfqpoint{1.187729in}{0.931810in}}%
\pgfpathlineto{\pgfqpoint{1.241896in}{1.143400in}}%
\pgfpathlineto{\pgfqpoint{1.296431in}{1.612889in}}%
\pgfpathlineto{\pgfqpoint{1.350693in}{2.054978in}}%
\pgfpathlineto{\pgfqpoint{1.405563in}{2.619765in}}%
\pgfpathlineto{\pgfqpoint{1.460378in}{2.855296in}}%
\pgfpathlineto{\pgfqpoint{1.513982in}{2.993563in}}%
\pgfpathlineto{\pgfqpoint{1.567494in}{3.301701in}}%
\pgfpathlineto{\pgfqpoint{1.621280in}{3.201429in}}%
\pgfpathlineto{\pgfqpoint{1.675519in}{3.149983in}}%
\pgfpathlineto{\pgfqpoint{1.729632in}{3.392689in}}%
\pgfpathlineto{\pgfqpoint{1.783675in}{3.482277in}}%
\pgfpathlineto{\pgfqpoint{1.837906in}{3.459913in}}%
\pgfpathlineto{\pgfqpoint{1.892207in}{3.457475in}}%
\pgfpathlineto{\pgfqpoint{1.946343in}{3.474742in}}%
\pgfpathlineto{\pgfqpoint{2.000663in}{3.596867in}}%
\pgfpathlineto{\pgfqpoint{2.056339in}{3.695700in}}%
\pgfpathlineto{\pgfqpoint{2.109632in}{3.643404in}}%
\pgfpathlineto{\pgfqpoint{2.163520in}{3.384477in}}%
\pgfpathlineto{\pgfqpoint{2.217785in}{3.991031in}}%
\pgfpathlineto{\pgfqpoint{2.271766in}{3.653972in}}%
\pgfpathlineto{\pgfqpoint{2.326053in}{4.056000in}}%
\pgfpathlineto{\pgfqpoint{2.379877in}{3.479367in}}%
\pgfpathlineto{\pgfqpoint{2.434881in}{3.435407in}}%
\pgfpathlineto{\pgfqpoint{2.489085in}{3.075721in}}%
\pgfpathlineto{\pgfqpoint{2.543496in}{3.211363in}}%
\pgfpathlineto{\pgfqpoint{2.597852in}{3.015016in}}%
\pgfpathlineto{\pgfqpoint{2.651899in}{3.037441in}}%
\pgfpathlineto{\pgfqpoint{2.707181in}{2.902484in}}%
\pgfpathlineto{\pgfqpoint{2.760801in}{2.903496in}}%
\pgfpathlineto{\pgfqpoint{2.814883in}{3.006884in}}%
\pgfpathlineto{\pgfqpoint{2.869363in}{2.836284in}}%
\pgfpathlineto{\pgfqpoint{2.923599in}{2.504239in}}%
\pgfpathlineto{\pgfqpoint{2.978074in}{2.241651in}}%
\pgfpathlineto{\pgfqpoint{3.032476in}{2.415217in}}%
\pgfpathlineto{\pgfqpoint{3.087002in}{2.703975in}}%
\pgfpathlineto{\pgfqpoint{3.141179in}{2.994802in}}%
\pgfpathlineto{\pgfqpoint{3.196674in}{2.944932in}}%
\pgfpathlineto{\pgfqpoint{3.250056in}{3.098962in}}%
\pgfpathlineto{\pgfqpoint{3.303754in}{3.109072in}}%
\pgfpathlineto{\pgfqpoint{3.357787in}{3.058217in}}%
\pgfpathlineto{\pgfqpoint{3.412098in}{2.867021in}}%
\pgfpathlineto{\pgfqpoint{3.466296in}{2.958334in}}%
\pgfpathlineto{\pgfqpoint{3.520829in}{2.872527in}}%
\pgfpathlineto{\pgfqpoint{3.575296in}{3.031312in}}%
\pgfpathlineto{\pgfqpoint{3.629615in}{3.131193in}}%
\pgfpathlineto{\pgfqpoint{3.683892in}{3.155779in}}%
\pgfpathlineto{\pgfqpoint{3.738111in}{3.283710in}}%
\pgfpathlineto{\pgfqpoint{3.792485in}{3.225129in}}%
\pgfpathlineto{\pgfqpoint{3.847561in}{3.269291in}}%
\pgfpathlineto{\pgfqpoint{3.900802in}{3.240500in}}%
\pgfpathlineto{\pgfqpoint{3.954925in}{3.283734in}}%
\pgfpathlineto{\pgfqpoint{4.009527in}{3.207004in}}%
\pgfpathlineto{\pgfqpoint{4.063958in}{3.184616in}}%
\pgfpathlineto{\pgfqpoint{4.118390in}{3.160487in}}%
\pgfpathlineto{\pgfqpoint{4.172406in}{3.167798in}}%
\pgfpathlineto{\pgfqpoint{4.226651in}{3.166986in}}%
\pgfpathlineto{\pgfqpoint{4.280748in}{3.142436in}}%
\pgfpathlineto{\pgfqpoint{4.335006in}{3.102004in}}%
\pgfpathlineto{\pgfqpoint{4.389535in}{3.116289in}}%
\pgfpathlineto{\pgfqpoint{4.445238in}{3.146498in}}%
\pgfpathlineto{\pgfqpoint{4.498454in}{3.135723in}}%
\pgfpathlineto{\pgfqpoint{4.552568in}{3.099125in}}%
\pgfpathlineto{\pgfqpoint{4.606635in}{3.066436in}}%
\pgfpathlineto{\pgfqpoint{4.660753in}{3.056768in}}%
\pgfpathlineto{\pgfqpoint{4.714982in}{2.829419in}}%
\pgfpathlineto{\pgfqpoint{4.768961in}{2.366474in}}%
\pgfpathlineto{\pgfqpoint{4.823155in}{2.569690in}}%
\pgfpathlineto{\pgfqpoint{4.877477in}{3.015034in}}%
\pgfpathlineto{\pgfqpoint{4.931730in}{2.917492in}}%
\pgfpathlineto{\pgfqpoint{4.986017in}{2.952067in}}%
\pgfpathlineto{\pgfqpoint{5.040162in}{3.251083in}}%
\pgfpathlineto{\pgfqpoint{5.095447in}{3.182793in}}%
\pgfpathlineto{\pgfqpoint{5.149088in}{3.533637in}}%
\pgfpathlineto{\pgfqpoint{5.203368in}{3.228438in}}%
\pgfpathlineto{\pgfqpoint{5.257185in}{3.354792in}}%
\pgfpathlineto{\pgfqpoint{5.311111in}{3.295614in}}%
\pgfpathlineto{\pgfqpoint{5.365843in}{3.222108in}}%
\pgfpathlineto{\pgfqpoint{5.419849in}{3.291767in}}%
\pgfpathlineto{\pgfqpoint{5.473831in}{3.204173in}}%
\pgfpathlineto{\pgfqpoint{5.528143in}{3.332583in}}%
\pgfusepath{stroke}%
\end{pgfscope}%
\begin{pgfscope}%
\pgfsetrectcap%
\pgfsetmiterjoin%
\pgfsetlinewidth{0.803000pt}%
\definecolor{currentstroke}{rgb}{0.000000,0.000000,0.000000}%
\pgfsetstrokecolor{currentstroke}%
\pgfsetdash{}{0pt}%
\pgfpathmoveto{\pgfqpoint{0.800000in}{0.528000in}}%
\pgfpathlineto{\pgfqpoint{0.800000in}{4.224000in}}%
\pgfusepath{stroke}%
\end{pgfscope}%
\begin{pgfscope}%
\pgfsetrectcap%
\pgfsetmiterjoin%
\pgfsetlinewidth{0.803000pt}%
\definecolor{currentstroke}{rgb}{0.000000,0.000000,0.000000}%
\pgfsetstrokecolor{currentstroke}%
\pgfsetdash{}{0pt}%
\pgfpathmoveto{\pgfqpoint{5.760000in}{0.528000in}}%
\pgfpathlineto{\pgfqpoint{5.760000in}{4.224000in}}%
\pgfusepath{stroke}%
\end{pgfscope}%
\begin{pgfscope}%
\pgfsetrectcap%
\pgfsetmiterjoin%
\pgfsetlinewidth{0.803000pt}%
\definecolor{currentstroke}{rgb}{0.000000,0.000000,0.000000}%
\pgfsetstrokecolor{currentstroke}%
\pgfsetdash{}{0pt}%
\pgfpathmoveto{\pgfqpoint{0.800000in}{0.528000in}}%
\pgfpathlineto{\pgfqpoint{5.760000in}{0.528000in}}%
\pgfusepath{stroke}%
\end{pgfscope}%
\begin{pgfscope}%
\pgfsetrectcap%
\pgfsetmiterjoin%
\pgfsetlinewidth{0.803000pt}%
\definecolor{currentstroke}{rgb}{0.000000,0.000000,0.000000}%
\pgfsetstrokecolor{currentstroke}%
\pgfsetdash{}{0pt}%
\pgfpathmoveto{\pgfqpoint{0.800000in}{4.224000in}}%
\pgfpathlineto{\pgfqpoint{5.760000in}{4.224000in}}%
\pgfusepath{stroke}%
\end{pgfscope}%
\begin{pgfscope}%
\definecolor{textcolor}{rgb}{0.000000,0.000000,0.000000}%
\pgfsetstrokecolor{textcolor}%
\pgfsetfillcolor{textcolor}%
\pgftext[x=3.280000in,y=4.307333in,,base]{\color{textcolor}\sffamily\fontsize{12.000000}{14.400000}\selectfont Forward controller input}%
\end{pgfscope}%
\begin{pgfscope}%
\pgfsetbuttcap%
\pgfsetmiterjoin%
\definecolor{currentfill}{rgb}{1.000000,1.000000,1.000000}%
\pgfsetfillcolor{currentfill}%
\pgfsetfillopacity{0.800000}%
\pgfsetlinewidth{1.003750pt}%
\definecolor{currentstroke}{rgb}{0.800000,0.800000,0.800000}%
\pgfsetstrokecolor{currentstroke}%
\pgfsetstrokeopacity{0.800000}%
\pgfsetdash{}{0pt}%
\pgfpathmoveto{\pgfqpoint{5.129968in}{0.597444in}}%
\pgfpathlineto{\pgfqpoint{5.662778in}{0.597444in}}%
\pgfpathquadraticcurveto{\pgfqpoint{5.690556in}{0.597444in}}{\pgfqpoint{5.690556in}{0.625222in}}%
\pgfpathlineto{\pgfqpoint{5.690556in}{1.630619in}}%
\pgfpathquadraticcurveto{\pgfqpoint{5.690556in}{1.658397in}}{\pgfqpoint{5.662778in}{1.658397in}}%
\pgfpathlineto{\pgfqpoint{5.129968in}{1.658397in}}%
\pgfpathquadraticcurveto{\pgfqpoint{5.102190in}{1.658397in}}{\pgfqpoint{5.102190in}{1.630619in}}%
\pgfpathlineto{\pgfqpoint{5.102190in}{0.625222in}}%
\pgfpathquadraticcurveto{\pgfqpoint{5.102190in}{0.597444in}}{\pgfqpoint{5.129968in}{0.597444in}}%
\pgfpathlineto{\pgfqpoint{5.129968in}{0.597444in}}%
\pgfpathclose%
\pgfusepath{stroke,fill}%
\end{pgfscope}%
\begin{pgfscope}%
\pgfsetrectcap%
\pgfsetroundjoin%
\pgfsetlinewidth{1.505625pt}%
\definecolor{currentstroke}{rgb}{0.121569,0.466667,0.705882}%
\pgfsetstrokecolor{currentstroke}%
\pgfsetdash{}{0pt}%
\pgfpathmoveto{\pgfqpoint{5.157746in}{1.545930in}}%
\pgfpathlineto{\pgfqpoint{5.296635in}{1.545930in}}%
\pgfpathlineto{\pgfqpoint{5.435524in}{1.545930in}}%
\pgfusepath{stroke}%
\end{pgfscope}%
\begin{pgfscope}%
\definecolor{textcolor}{rgb}{0.000000,0.000000,0.000000}%
\pgfsetstrokecolor{textcolor}%
\pgfsetfillcolor{textcolor}%
\pgftext[x=5.546635in,y=1.497319in,left,base]{\color{textcolor}\sffamily\fontsize{10.000000}{12.000000}\selectfont 0}%
\end{pgfscope}%
\begin{pgfscope}%
\pgfsetrectcap%
\pgfsetroundjoin%
\pgfsetlinewidth{1.505625pt}%
\definecolor{currentstroke}{rgb}{1.000000,0.498039,0.054902}%
\pgfsetstrokecolor{currentstroke}%
\pgfsetdash{}{0pt}%
\pgfpathmoveto{\pgfqpoint{5.157746in}{1.342073in}}%
\pgfpathlineto{\pgfqpoint{5.296635in}{1.342073in}}%
\pgfpathlineto{\pgfqpoint{5.435524in}{1.342073in}}%
\pgfusepath{stroke}%
\end{pgfscope}%
\begin{pgfscope}%
\definecolor{textcolor}{rgb}{0.000000,0.000000,0.000000}%
\pgfsetstrokecolor{textcolor}%
\pgfsetfillcolor{textcolor}%
\pgftext[x=5.546635in,y=1.293461in,left,base]{\color{textcolor}\sffamily\fontsize{10.000000}{12.000000}\selectfont 1}%
\end{pgfscope}%
\begin{pgfscope}%
\pgfsetrectcap%
\pgfsetroundjoin%
\pgfsetlinewidth{1.505625pt}%
\definecolor{currentstroke}{rgb}{0.172549,0.627451,0.172549}%
\pgfsetstrokecolor{currentstroke}%
\pgfsetdash{}{0pt}%
\pgfpathmoveto{\pgfqpoint{5.157746in}{1.138215in}}%
\pgfpathlineto{\pgfqpoint{5.296635in}{1.138215in}}%
\pgfpathlineto{\pgfqpoint{5.435524in}{1.138215in}}%
\pgfusepath{stroke}%
\end{pgfscope}%
\begin{pgfscope}%
\definecolor{textcolor}{rgb}{0.000000,0.000000,0.000000}%
\pgfsetstrokecolor{textcolor}%
\pgfsetfillcolor{textcolor}%
\pgftext[x=5.546635in,y=1.089604in,left,base]{\color{textcolor}\sffamily\fontsize{10.000000}{12.000000}\selectfont 2}%
\end{pgfscope}%
\begin{pgfscope}%
\pgfsetrectcap%
\pgfsetroundjoin%
\pgfsetlinewidth{1.505625pt}%
\definecolor{currentstroke}{rgb}{0.839216,0.152941,0.156863}%
\pgfsetstrokecolor{currentstroke}%
\pgfsetdash{}{0pt}%
\pgfpathmoveto{\pgfqpoint{5.157746in}{0.934358in}}%
\pgfpathlineto{\pgfqpoint{5.296635in}{0.934358in}}%
\pgfpathlineto{\pgfqpoint{5.435524in}{0.934358in}}%
\pgfusepath{stroke}%
\end{pgfscope}%
\begin{pgfscope}%
\definecolor{textcolor}{rgb}{0.000000,0.000000,0.000000}%
\pgfsetstrokecolor{textcolor}%
\pgfsetfillcolor{textcolor}%
\pgftext[x=5.546635in,y=0.885747in,left,base]{\color{textcolor}\sffamily\fontsize{10.000000}{12.000000}\selectfont 3}%
\end{pgfscope}%
\begin{pgfscope}%
\pgfsetrectcap%
\pgfsetroundjoin%
\pgfsetlinewidth{1.505625pt}%
\definecolor{currentstroke}{rgb}{0.580392,0.403922,0.741176}%
\pgfsetstrokecolor{currentstroke}%
\pgfsetdash{}{0pt}%
\pgfpathmoveto{\pgfqpoint{5.157746in}{0.730501in}}%
\pgfpathlineto{\pgfqpoint{5.296635in}{0.730501in}}%
\pgfpathlineto{\pgfqpoint{5.435524in}{0.730501in}}%
\pgfusepath{stroke}%
\end{pgfscope}%
\begin{pgfscope}%
\definecolor{textcolor}{rgb}{0.000000,0.000000,0.000000}%
\pgfsetstrokecolor{textcolor}%
\pgfsetfillcolor{textcolor}%
\pgftext[x=5.546635in,y=0.681890in,left,base]{\color{textcolor}\sffamily\fontsize{10.000000}{12.000000}\selectfont 4}%
\end{pgfscope}%
\end{pgfpicture}%
\makeatother%
\endgroup%
}
    \end{minipage}
    \begin{minipage}[t]{0.5\linewidth}
        \centering
        \scalebox{0.55}{%% Creator: Matplotlib, PGF backend
%%
%% To include the figure in your LaTeX document, write
%%   \input{<filename>.pgf}
%%
%% Make sure the required packages are loaded in your preamble
%%   \usepackage{pgf}
%%
%% Also ensure that all the required font packages are loaded; for instance,
%% the lmodern package is sometimes necessary when using math font.
%%   \usepackage{lmodern}
%%
%% Figures using additional raster images can only be included by \input if
%% they are in the same directory as the main LaTeX file. For loading figures
%% from other directories you can use the `import` package
%%   \usepackage{import}
%%
%% and then include the figures with
%%   \import{<path to file>}{<filename>.pgf}
%%
%% Matplotlib used the following preamble
%%   \usepackage{fontspec}
%%   \setmainfont{DejaVuSerif.ttf}[Path=\detokenize{/home/lgonz/tfg-aero/tfg-giaa-dronecontrol/venv/lib/python3.8/site-packages/matplotlib/mpl-data/fonts/ttf/}]
%%   \setsansfont{DejaVuSans.ttf}[Path=\detokenize{/home/lgonz/tfg-aero/tfg-giaa-dronecontrol/venv/lib/python3.8/site-packages/matplotlib/mpl-data/fonts/ttf/}]
%%   \setmonofont{DejaVuSansMono.ttf}[Path=\detokenize{/home/lgonz/tfg-aero/tfg-giaa-dronecontrol/venv/lib/python3.8/site-packages/matplotlib/mpl-data/fonts/ttf/}]
%%
\begingroup%
\makeatletter%
\begin{pgfpicture}%
\pgfpathrectangle{\pgfpointorigin}{\pgfqpoint{6.400000in}{4.800000in}}%
\pgfusepath{use as bounding box, clip}%
\begin{pgfscope}%
\pgfsetbuttcap%
\pgfsetmiterjoin%
\definecolor{currentfill}{rgb}{1.000000,1.000000,1.000000}%
\pgfsetfillcolor{currentfill}%
\pgfsetlinewidth{0.000000pt}%
\definecolor{currentstroke}{rgb}{1.000000,1.000000,1.000000}%
\pgfsetstrokecolor{currentstroke}%
\pgfsetdash{}{0pt}%
\pgfpathmoveto{\pgfqpoint{0.000000in}{0.000000in}}%
\pgfpathlineto{\pgfqpoint{6.400000in}{0.000000in}}%
\pgfpathlineto{\pgfqpoint{6.400000in}{4.800000in}}%
\pgfpathlineto{\pgfqpoint{0.000000in}{4.800000in}}%
\pgfpathlineto{\pgfqpoint{0.000000in}{0.000000in}}%
\pgfpathclose%
\pgfusepath{fill}%
\end{pgfscope}%
\begin{pgfscope}%
\pgfsetbuttcap%
\pgfsetmiterjoin%
\definecolor{currentfill}{rgb}{1.000000,1.000000,1.000000}%
\pgfsetfillcolor{currentfill}%
\pgfsetlinewidth{0.000000pt}%
\definecolor{currentstroke}{rgb}{0.000000,0.000000,0.000000}%
\pgfsetstrokecolor{currentstroke}%
\pgfsetstrokeopacity{0.000000}%
\pgfsetdash{}{0pt}%
\pgfpathmoveto{\pgfqpoint{0.800000in}{0.528000in}}%
\pgfpathlineto{\pgfqpoint{5.760000in}{0.528000in}}%
\pgfpathlineto{\pgfqpoint{5.760000in}{4.224000in}}%
\pgfpathlineto{\pgfqpoint{0.800000in}{4.224000in}}%
\pgfpathlineto{\pgfqpoint{0.800000in}{0.528000in}}%
\pgfpathclose%
\pgfusepath{fill}%
\end{pgfscope}%
\begin{pgfscope}%
\pgfpathrectangle{\pgfqpoint{0.800000in}{0.528000in}}{\pgfqpoint{4.960000in}{3.696000in}}%
\pgfusepath{clip}%
\pgfsetrectcap%
\pgfsetroundjoin%
\pgfsetlinewidth{0.803000pt}%
\definecolor{currentstroke}{rgb}{0.690196,0.690196,0.690196}%
\pgfsetstrokecolor{currentstroke}%
\pgfsetdash{}{0pt}%
\pgfpathmoveto{\pgfqpoint{1.025455in}{0.528000in}}%
\pgfpathlineto{\pgfqpoint{1.025455in}{4.224000in}}%
\pgfusepath{stroke}%
\end{pgfscope}%
\begin{pgfscope}%
\pgfsetbuttcap%
\pgfsetroundjoin%
\definecolor{currentfill}{rgb}{0.000000,0.000000,0.000000}%
\pgfsetfillcolor{currentfill}%
\pgfsetlinewidth{0.803000pt}%
\definecolor{currentstroke}{rgb}{0.000000,0.000000,0.000000}%
\pgfsetstrokecolor{currentstroke}%
\pgfsetdash{}{0pt}%
\pgfsys@defobject{currentmarker}{\pgfqpoint{0.000000in}{-0.048611in}}{\pgfqpoint{0.000000in}{0.000000in}}{%
\pgfpathmoveto{\pgfqpoint{0.000000in}{0.000000in}}%
\pgfpathlineto{\pgfqpoint{0.000000in}{-0.048611in}}%
\pgfusepath{stroke,fill}%
}%
\begin{pgfscope}%
\pgfsys@transformshift{1.025455in}{0.528000in}%
\pgfsys@useobject{currentmarker}{}%
\end{pgfscope}%
\end{pgfscope}%
\begin{pgfscope}%
\definecolor{textcolor}{rgb}{0.000000,0.000000,0.000000}%
\pgfsetstrokecolor{textcolor}%
\pgfsetfillcolor{textcolor}%
\pgftext[x=1.025455in,y=0.430778in,,top]{\color{textcolor}\sffamily\fontsize{10.000000}{12.000000}\selectfont 0}%
\end{pgfscope}%
\begin{pgfscope}%
\pgfpathrectangle{\pgfqpoint{0.800000in}{0.528000in}}{\pgfqpoint{4.960000in}{3.696000in}}%
\pgfusepath{clip}%
\pgfsetrectcap%
\pgfsetroundjoin%
\pgfsetlinewidth{0.803000pt}%
\definecolor{currentstroke}{rgb}{0.690196,0.690196,0.690196}%
\pgfsetstrokecolor{currentstroke}%
\pgfsetdash{}{0pt}%
\pgfpathmoveto{\pgfqpoint{1.775563in}{0.528000in}}%
\pgfpathlineto{\pgfqpoint{1.775563in}{4.224000in}}%
\pgfusepath{stroke}%
\end{pgfscope}%
\begin{pgfscope}%
\pgfsetbuttcap%
\pgfsetroundjoin%
\definecolor{currentfill}{rgb}{0.000000,0.000000,0.000000}%
\pgfsetfillcolor{currentfill}%
\pgfsetlinewidth{0.803000pt}%
\definecolor{currentstroke}{rgb}{0.000000,0.000000,0.000000}%
\pgfsetstrokecolor{currentstroke}%
\pgfsetdash{}{0pt}%
\pgfsys@defobject{currentmarker}{\pgfqpoint{0.000000in}{-0.048611in}}{\pgfqpoint{0.000000in}{0.000000in}}{%
\pgfpathmoveto{\pgfqpoint{0.000000in}{0.000000in}}%
\pgfpathlineto{\pgfqpoint{0.000000in}{-0.048611in}}%
\pgfusepath{stroke,fill}%
}%
\begin{pgfscope}%
\pgfsys@transformshift{1.775563in}{0.528000in}%
\pgfsys@useobject{currentmarker}{}%
\end{pgfscope}%
\end{pgfscope}%
\begin{pgfscope}%
\definecolor{textcolor}{rgb}{0.000000,0.000000,0.000000}%
\pgfsetstrokecolor{textcolor}%
\pgfsetfillcolor{textcolor}%
\pgftext[x=1.775563in,y=0.430778in,,top]{\color{textcolor}\sffamily\fontsize{10.000000}{12.000000}\selectfont 5}%
\end{pgfscope}%
\begin{pgfscope}%
\pgfpathrectangle{\pgfqpoint{0.800000in}{0.528000in}}{\pgfqpoint{4.960000in}{3.696000in}}%
\pgfusepath{clip}%
\pgfsetrectcap%
\pgfsetroundjoin%
\pgfsetlinewidth{0.803000pt}%
\definecolor{currentstroke}{rgb}{0.690196,0.690196,0.690196}%
\pgfsetstrokecolor{currentstroke}%
\pgfsetdash{}{0pt}%
\pgfpathmoveto{\pgfqpoint{2.525672in}{0.528000in}}%
\pgfpathlineto{\pgfqpoint{2.525672in}{4.224000in}}%
\pgfusepath{stroke}%
\end{pgfscope}%
\begin{pgfscope}%
\pgfsetbuttcap%
\pgfsetroundjoin%
\definecolor{currentfill}{rgb}{0.000000,0.000000,0.000000}%
\pgfsetfillcolor{currentfill}%
\pgfsetlinewidth{0.803000pt}%
\definecolor{currentstroke}{rgb}{0.000000,0.000000,0.000000}%
\pgfsetstrokecolor{currentstroke}%
\pgfsetdash{}{0pt}%
\pgfsys@defobject{currentmarker}{\pgfqpoint{0.000000in}{-0.048611in}}{\pgfqpoint{0.000000in}{0.000000in}}{%
\pgfpathmoveto{\pgfqpoint{0.000000in}{0.000000in}}%
\pgfpathlineto{\pgfqpoint{0.000000in}{-0.048611in}}%
\pgfusepath{stroke,fill}%
}%
\begin{pgfscope}%
\pgfsys@transformshift{2.525672in}{0.528000in}%
\pgfsys@useobject{currentmarker}{}%
\end{pgfscope}%
\end{pgfscope}%
\begin{pgfscope}%
\definecolor{textcolor}{rgb}{0.000000,0.000000,0.000000}%
\pgfsetstrokecolor{textcolor}%
\pgfsetfillcolor{textcolor}%
\pgftext[x=2.525672in,y=0.430778in,,top]{\color{textcolor}\sffamily\fontsize{10.000000}{12.000000}\selectfont 10}%
\end{pgfscope}%
\begin{pgfscope}%
\pgfpathrectangle{\pgfqpoint{0.800000in}{0.528000in}}{\pgfqpoint{4.960000in}{3.696000in}}%
\pgfusepath{clip}%
\pgfsetrectcap%
\pgfsetroundjoin%
\pgfsetlinewidth{0.803000pt}%
\definecolor{currentstroke}{rgb}{0.690196,0.690196,0.690196}%
\pgfsetstrokecolor{currentstroke}%
\pgfsetdash{}{0pt}%
\pgfpathmoveto{\pgfqpoint{3.275781in}{0.528000in}}%
\pgfpathlineto{\pgfqpoint{3.275781in}{4.224000in}}%
\pgfusepath{stroke}%
\end{pgfscope}%
\begin{pgfscope}%
\pgfsetbuttcap%
\pgfsetroundjoin%
\definecolor{currentfill}{rgb}{0.000000,0.000000,0.000000}%
\pgfsetfillcolor{currentfill}%
\pgfsetlinewidth{0.803000pt}%
\definecolor{currentstroke}{rgb}{0.000000,0.000000,0.000000}%
\pgfsetstrokecolor{currentstroke}%
\pgfsetdash{}{0pt}%
\pgfsys@defobject{currentmarker}{\pgfqpoint{0.000000in}{-0.048611in}}{\pgfqpoint{0.000000in}{0.000000in}}{%
\pgfpathmoveto{\pgfqpoint{0.000000in}{0.000000in}}%
\pgfpathlineto{\pgfqpoint{0.000000in}{-0.048611in}}%
\pgfusepath{stroke,fill}%
}%
\begin{pgfscope}%
\pgfsys@transformshift{3.275781in}{0.528000in}%
\pgfsys@useobject{currentmarker}{}%
\end{pgfscope}%
\end{pgfscope}%
\begin{pgfscope}%
\definecolor{textcolor}{rgb}{0.000000,0.000000,0.000000}%
\pgfsetstrokecolor{textcolor}%
\pgfsetfillcolor{textcolor}%
\pgftext[x=3.275781in,y=0.430778in,,top]{\color{textcolor}\sffamily\fontsize{10.000000}{12.000000}\selectfont 15}%
\end{pgfscope}%
\begin{pgfscope}%
\pgfpathrectangle{\pgfqpoint{0.800000in}{0.528000in}}{\pgfqpoint{4.960000in}{3.696000in}}%
\pgfusepath{clip}%
\pgfsetrectcap%
\pgfsetroundjoin%
\pgfsetlinewidth{0.803000pt}%
\definecolor{currentstroke}{rgb}{0.690196,0.690196,0.690196}%
\pgfsetstrokecolor{currentstroke}%
\pgfsetdash{}{0pt}%
\pgfpathmoveto{\pgfqpoint{4.025890in}{0.528000in}}%
\pgfpathlineto{\pgfqpoint{4.025890in}{4.224000in}}%
\pgfusepath{stroke}%
\end{pgfscope}%
\begin{pgfscope}%
\pgfsetbuttcap%
\pgfsetroundjoin%
\definecolor{currentfill}{rgb}{0.000000,0.000000,0.000000}%
\pgfsetfillcolor{currentfill}%
\pgfsetlinewidth{0.803000pt}%
\definecolor{currentstroke}{rgb}{0.000000,0.000000,0.000000}%
\pgfsetstrokecolor{currentstroke}%
\pgfsetdash{}{0pt}%
\pgfsys@defobject{currentmarker}{\pgfqpoint{0.000000in}{-0.048611in}}{\pgfqpoint{0.000000in}{0.000000in}}{%
\pgfpathmoveto{\pgfqpoint{0.000000in}{0.000000in}}%
\pgfpathlineto{\pgfqpoint{0.000000in}{-0.048611in}}%
\pgfusepath{stroke,fill}%
}%
\begin{pgfscope}%
\pgfsys@transformshift{4.025890in}{0.528000in}%
\pgfsys@useobject{currentmarker}{}%
\end{pgfscope}%
\end{pgfscope}%
\begin{pgfscope}%
\definecolor{textcolor}{rgb}{0.000000,0.000000,0.000000}%
\pgfsetstrokecolor{textcolor}%
\pgfsetfillcolor{textcolor}%
\pgftext[x=4.025890in,y=0.430778in,,top]{\color{textcolor}\sffamily\fontsize{10.000000}{12.000000}\selectfont 20}%
\end{pgfscope}%
\begin{pgfscope}%
\pgfpathrectangle{\pgfqpoint{0.800000in}{0.528000in}}{\pgfqpoint{4.960000in}{3.696000in}}%
\pgfusepath{clip}%
\pgfsetrectcap%
\pgfsetroundjoin%
\pgfsetlinewidth{0.803000pt}%
\definecolor{currentstroke}{rgb}{0.690196,0.690196,0.690196}%
\pgfsetstrokecolor{currentstroke}%
\pgfsetdash{}{0pt}%
\pgfpathmoveto{\pgfqpoint{4.775998in}{0.528000in}}%
\pgfpathlineto{\pgfqpoint{4.775998in}{4.224000in}}%
\pgfusepath{stroke}%
\end{pgfscope}%
\begin{pgfscope}%
\pgfsetbuttcap%
\pgfsetroundjoin%
\definecolor{currentfill}{rgb}{0.000000,0.000000,0.000000}%
\pgfsetfillcolor{currentfill}%
\pgfsetlinewidth{0.803000pt}%
\definecolor{currentstroke}{rgb}{0.000000,0.000000,0.000000}%
\pgfsetstrokecolor{currentstroke}%
\pgfsetdash{}{0pt}%
\pgfsys@defobject{currentmarker}{\pgfqpoint{0.000000in}{-0.048611in}}{\pgfqpoint{0.000000in}{0.000000in}}{%
\pgfpathmoveto{\pgfqpoint{0.000000in}{0.000000in}}%
\pgfpathlineto{\pgfqpoint{0.000000in}{-0.048611in}}%
\pgfusepath{stroke,fill}%
}%
\begin{pgfscope}%
\pgfsys@transformshift{4.775998in}{0.528000in}%
\pgfsys@useobject{currentmarker}{}%
\end{pgfscope}%
\end{pgfscope}%
\begin{pgfscope}%
\definecolor{textcolor}{rgb}{0.000000,0.000000,0.000000}%
\pgfsetstrokecolor{textcolor}%
\pgfsetfillcolor{textcolor}%
\pgftext[x=4.775998in,y=0.430778in,,top]{\color{textcolor}\sffamily\fontsize{10.000000}{12.000000}\selectfont 25}%
\end{pgfscope}%
\begin{pgfscope}%
\pgfpathrectangle{\pgfqpoint{0.800000in}{0.528000in}}{\pgfqpoint{4.960000in}{3.696000in}}%
\pgfusepath{clip}%
\pgfsetrectcap%
\pgfsetroundjoin%
\pgfsetlinewidth{0.803000pt}%
\definecolor{currentstroke}{rgb}{0.690196,0.690196,0.690196}%
\pgfsetstrokecolor{currentstroke}%
\pgfsetdash{}{0pt}%
\pgfpathmoveto{\pgfqpoint{5.526107in}{0.528000in}}%
\pgfpathlineto{\pgfqpoint{5.526107in}{4.224000in}}%
\pgfusepath{stroke}%
\end{pgfscope}%
\begin{pgfscope}%
\pgfsetbuttcap%
\pgfsetroundjoin%
\definecolor{currentfill}{rgb}{0.000000,0.000000,0.000000}%
\pgfsetfillcolor{currentfill}%
\pgfsetlinewidth{0.803000pt}%
\definecolor{currentstroke}{rgb}{0.000000,0.000000,0.000000}%
\pgfsetstrokecolor{currentstroke}%
\pgfsetdash{}{0pt}%
\pgfsys@defobject{currentmarker}{\pgfqpoint{0.000000in}{-0.048611in}}{\pgfqpoint{0.000000in}{0.000000in}}{%
\pgfpathmoveto{\pgfqpoint{0.000000in}{0.000000in}}%
\pgfpathlineto{\pgfqpoint{0.000000in}{-0.048611in}}%
\pgfusepath{stroke,fill}%
}%
\begin{pgfscope}%
\pgfsys@transformshift{5.526107in}{0.528000in}%
\pgfsys@useobject{currentmarker}{}%
\end{pgfscope}%
\end{pgfscope}%
\begin{pgfscope}%
\definecolor{textcolor}{rgb}{0.000000,0.000000,0.000000}%
\pgfsetstrokecolor{textcolor}%
\pgfsetfillcolor{textcolor}%
\pgftext[x=5.526107in,y=0.430778in,,top]{\color{textcolor}\sffamily\fontsize{10.000000}{12.000000}\selectfont 30}%
\end{pgfscope}%
\begin{pgfscope}%
\definecolor{textcolor}{rgb}{0.000000,0.000000,0.000000}%
\pgfsetstrokecolor{textcolor}%
\pgfsetfillcolor{textcolor}%
\pgftext[x=3.280000in,y=0.240809in,,top]{\color{textcolor}\sffamily\fontsize{10.000000}{12.000000}\selectfont time [s]}%
\end{pgfscope}%
\begin{pgfscope}%
\pgfpathrectangle{\pgfqpoint{0.800000in}{0.528000in}}{\pgfqpoint{4.960000in}{3.696000in}}%
\pgfusepath{clip}%
\pgfsetrectcap%
\pgfsetroundjoin%
\pgfsetlinewidth{0.803000pt}%
\definecolor{currentstroke}{rgb}{0.690196,0.690196,0.690196}%
\pgfsetstrokecolor{currentstroke}%
\pgfsetdash{}{0pt}%
\pgfpathmoveto{\pgfqpoint{0.800000in}{0.696000in}}%
\pgfpathlineto{\pgfqpoint{5.760000in}{0.696000in}}%
\pgfusepath{stroke}%
\end{pgfscope}%
\begin{pgfscope}%
\pgfsetbuttcap%
\pgfsetroundjoin%
\definecolor{currentfill}{rgb}{0.000000,0.000000,0.000000}%
\pgfsetfillcolor{currentfill}%
\pgfsetlinewidth{0.803000pt}%
\definecolor{currentstroke}{rgb}{0.000000,0.000000,0.000000}%
\pgfsetstrokecolor{currentstroke}%
\pgfsetdash{}{0pt}%
\pgfsys@defobject{currentmarker}{\pgfqpoint{-0.048611in}{0.000000in}}{\pgfqpoint{-0.000000in}{0.000000in}}{%
\pgfpathmoveto{\pgfqpoint{-0.000000in}{0.000000in}}%
\pgfpathlineto{\pgfqpoint{-0.048611in}{0.000000in}}%
\pgfusepath{stroke,fill}%
}%
\begin{pgfscope}%
\pgfsys@transformshift{0.800000in}{0.696000in}%
\pgfsys@useobject{currentmarker}{}%
\end{pgfscope}%
\end{pgfscope}%
\begin{pgfscope}%
\definecolor{textcolor}{rgb}{0.000000,0.000000,0.000000}%
\pgfsetstrokecolor{textcolor}%
\pgfsetfillcolor{textcolor}%
\pgftext[x=0.373873in, y=0.643238in, left, base]{\color{textcolor}\sffamily\fontsize{10.000000}{12.000000}\selectfont \ensuremath{-}0.4}%
\end{pgfscope}%
\begin{pgfscope}%
\pgfpathrectangle{\pgfqpoint{0.800000in}{0.528000in}}{\pgfqpoint{4.960000in}{3.696000in}}%
\pgfusepath{clip}%
\pgfsetrectcap%
\pgfsetroundjoin%
\pgfsetlinewidth{0.803000pt}%
\definecolor{currentstroke}{rgb}{0.690196,0.690196,0.690196}%
\pgfsetstrokecolor{currentstroke}%
\pgfsetdash{}{0pt}%
\pgfpathmoveto{\pgfqpoint{0.800000in}{1.116000in}}%
\pgfpathlineto{\pgfqpoint{5.760000in}{1.116000in}}%
\pgfusepath{stroke}%
\end{pgfscope}%
\begin{pgfscope}%
\pgfsetbuttcap%
\pgfsetroundjoin%
\definecolor{currentfill}{rgb}{0.000000,0.000000,0.000000}%
\pgfsetfillcolor{currentfill}%
\pgfsetlinewidth{0.803000pt}%
\definecolor{currentstroke}{rgb}{0.000000,0.000000,0.000000}%
\pgfsetstrokecolor{currentstroke}%
\pgfsetdash{}{0pt}%
\pgfsys@defobject{currentmarker}{\pgfqpoint{-0.048611in}{0.000000in}}{\pgfqpoint{-0.000000in}{0.000000in}}{%
\pgfpathmoveto{\pgfqpoint{-0.000000in}{0.000000in}}%
\pgfpathlineto{\pgfqpoint{-0.048611in}{0.000000in}}%
\pgfusepath{stroke,fill}%
}%
\begin{pgfscope}%
\pgfsys@transformshift{0.800000in}{1.116000in}%
\pgfsys@useobject{currentmarker}{}%
\end{pgfscope}%
\end{pgfscope}%
\begin{pgfscope}%
\definecolor{textcolor}{rgb}{0.000000,0.000000,0.000000}%
\pgfsetstrokecolor{textcolor}%
\pgfsetfillcolor{textcolor}%
\pgftext[x=0.373873in, y=1.063238in, left, base]{\color{textcolor}\sffamily\fontsize{10.000000}{12.000000}\selectfont \ensuremath{-}0.3}%
\end{pgfscope}%
\begin{pgfscope}%
\pgfpathrectangle{\pgfqpoint{0.800000in}{0.528000in}}{\pgfqpoint{4.960000in}{3.696000in}}%
\pgfusepath{clip}%
\pgfsetrectcap%
\pgfsetroundjoin%
\pgfsetlinewidth{0.803000pt}%
\definecolor{currentstroke}{rgb}{0.690196,0.690196,0.690196}%
\pgfsetstrokecolor{currentstroke}%
\pgfsetdash{}{0pt}%
\pgfpathmoveto{\pgfqpoint{0.800000in}{1.536000in}}%
\pgfpathlineto{\pgfqpoint{5.760000in}{1.536000in}}%
\pgfusepath{stroke}%
\end{pgfscope}%
\begin{pgfscope}%
\pgfsetbuttcap%
\pgfsetroundjoin%
\definecolor{currentfill}{rgb}{0.000000,0.000000,0.000000}%
\pgfsetfillcolor{currentfill}%
\pgfsetlinewidth{0.803000pt}%
\definecolor{currentstroke}{rgb}{0.000000,0.000000,0.000000}%
\pgfsetstrokecolor{currentstroke}%
\pgfsetdash{}{0pt}%
\pgfsys@defobject{currentmarker}{\pgfqpoint{-0.048611in}{0.000000in}}{\pgfqpoint{-0.000000in}{0.000000in}}{%
\pgfpathmoveto{\pgfqpoint{-0.000000in}{0.000000in}}%
\pgfpathlineto{\pgfqpoint{-0.048611in}{0.000000in}}%
\pgfusepath{stroke,fill}%
}%
\begin{pgfscope}%
\pgfsys@transformshift{0.800000in}{1.536000in}%
\pgfsys@useobject{currentmarker}{}%
\end{pgfscope}%
\end{pgfscope}%
\begin{pgfscope}%
\definecolor{textcolor}{rgb}{0.000000,0.000000,0.000000}%
\pgfsetstrokecolor{textcolor}%
\pgfsetfillcolor{textcolor}%
\pgftext[x=0.373873in, y=1.483238in, left, base]{\color{textcolor}\sffamily\fontsize{10.000000}{12.000000}\selectfont \ensuremath{-}0.2}%
\end{pgfscope}%
\begin{pgfscope}%
\pgfpathrectangle{\pgfqpoint{0.800000in}{0.528000in}}{\pgfqpoint{4.960000in}{3.696000in}}%
\pgfusepath{clip}%
\pgfsetrectcap%
\pgfsetroundjoin%
\pgfsetlinewidth{0.803000pt}%
\definecolor{currentstroke}{rgb}{0.690196,0.690196,0.690196}%
\pgfsetstrokecolor{currentstroke}%
\pgfsetdash{}{0pt}%
\pgfpathmoveto{\pgfqpoint{0.800000in}{1.956000in}}%
\pgfpathlineto{\pgfqpoint{5.760000in}{1.956000in}}%
\pgfusepath{stroke}%
\end{pgfscope}%
\begin{pgfscope}%
\pgfsetbuttcap%
\pgfsetroundjoin%
\definecolor{currentfill}{rgb}{0.000000,0.000000,0.000000}%
\pgfsetfillcolor{currentfill}%
\pgfsetlinewidth{0.803000pt}%
\definecolor{currentstroke}{rgb}{0.000000,0.000000,0.000000}%
\pgfsetstrokecolor{currentstroke}%
\pgfsetdash{}{0pt}%
\pgfsys@defobject{currentmarker}{\pgfqpoint{-0.048611in}{0.000000in}}{\pgfqpoint{-0.000000in}{0.000000in}}{%
\pgfpathmoveto{\pgfqpoint{-0.000000in}{0.000000in}}%
\pgfpathlineto{\pgfqpoint{-0.048611in}{0.000000in}}%
\pgfusepath{stroke,fill}%
}%
\begin{pgfscope}%
\pgfsys@transformshift{0.800000in}{1.956000in}%
\pgfsys@useobject{currentmarker}{}%
\end{pgfscope}%
\end{pgfscope}%
\begin{pgfscope}%
\definecolor{textcolor}{rgb}{0.000000,0.000000,0.000000}%
\pgfsetstrokecolor{textcolor}%
\pgfsetfillcolor{textcolor}%
\pgftext[x=0.373873in, y=1.903238in, left, base]{\color{textcolor}\sffamily\fontsize{10.000000}{12.000000}\selectfont \ensuremath{-}0.1}%
\end{pgfscope}%
\begin{pgfscope}%
\pgfpathrectangle{\pgfqpoint{0.800000in}{0.528000in}}{\pgfqpoint{4.960000in}{3.696000in}}%
\pgfusepath{clip}%
\pgfsetrectcap%
\pgfsetroundjoin%
\pgfsetlinewidth{0.803000pt}%
\definecolor{currentstroke}{rgb}{0.690196,0.690196,0.690196}%
\pgfsetstrokecolor{currentstroke}%
\pgfsetdash{}{0pt}%
\pgfpathmoveto{\pgfqpoint{0.800000in}{2.376000in}}%
\pgfpathlineto{\pgfqpoint{5.760000in}{2.376000in}}%
\pgfusepath{stroke}%
\end{pgfscope}%
\begin{pgfscope}%
\pgfsetbuttcap%
\pgfsetroundjoin%
\definecolor{currentfill}{rgb}{0.000000,0.000000,0.000000}%
\pgfsetfillcolor{currentfill}%
\pgfsetlinewidth{0.803000pt}%
\definecolor{currentstroke}{rgb}{0.000000,0.000000,0.000000}%
\pgfsetstrokecolor{currentstroke}%
\pgfsetdash{}{0pt}%
\pgfsys@defobject{currentmarker}{\pgfqpoint{-0.048611in}{0.000000in}}{\pgfqpoint{-0.000000in}{0.000000in}}{%
\pgfpathmoveto{\pgfqpoint{-0.000000in}{0.000000in}}%
\pgfpathlineto{\pgfqpoint{-0.048611in}{0.000000in}}%
\pgfusepath{stroke,fill}%
}%
\begin{pgfscope}%
\pgfsys@transformshift{0.800000in}{2.376000in}%
\pgfsys@useobject{currentmarker}{}%
\end{pgfscope}%
\end{pgfscope}%
\begin{pgfscope}%
\definecolor{textcolor}{rgb}{0.000000,0.000000,0.000000}%
\pgfsetstrokecolor{textcolor}%
\pgfsetfillcolor{textcolor}%
\pgftext[x=0.481898in, y=2.323238in, left, base]{\color{textcolor}\sffamily\fontsize{10.000000}{12.000000}\selectfont 0.0}%
\end{pgfscope}%
\begin{pgfscope}%
\pgfpathrectangle{\pgfqpoint{0.800000in}{0.528000in}}{\pgfqpoint{4.960000in}{3.696000in}}%
\pgfusepath{clip}%
\pgfsetrectcap%
\pgfsetroundjoin%
\pgfsetlinewidth{0.803000pt}%
\definecolor{currentstroke}{rgb}{0.690196,0.690196,0.690196}%
\pgfsetstrokecolor{currentstroke}%
\pgfsetdash{}{0pt}%
\pgfpathmoveto{\pgfqpoint{0.800000in}{2.796000in}}%
\pgfpathlineto{\pgfqpoint{5.760000in}{2.796000in}}%
\pgfusepath{stroke}%
\end{pgfscope}%
\begin{pgfscope}%
\pgfsetbuttcap%
\pgfsetroundjoin%
\definecolor{currentfill}{rgb}{0.000000,0.000000,0.000000}%
\pgfsetfillcolor{currentfill}%
\pgfsetlinewidth{0.803000pt}%
\definecolor{currentstroke}{rgb}{0.000000,0.000000,0.000000}%
\pgfsetstrokecolor{currentstroke}%
\pgfsetdash{}{0pt}%
\pgfsys@defobject{currentmarker}{\pgfqpoint{-0.048611in}{0.000000in}}{\pgfqpoint{-0.000000in}{0.000000in}}{%
\pgfpathmoveto{\pgfqpoint{-0.000000in}{0.000000in}}%
\pgfpathlineto{\pgfqpoint{-0.048611in}{0.000000in}}%
\pgfusepath{stroke,fill}%
}%
\begin{pgfscope}%
\pgfsys@transformshift{0.800000in}{2.796000in}%
\pgfsys@useobject{currentmarker}{}%
\end{pgfscope}%
\end{pgfscope}%
\begin{pgfscope}%
\definecolor{textcolor}{rgb}{0.000000,0.000000,0.000000}%
\pgfsetstrokecolor{textcolor}%
\pgfsetfillcolor{textcolor}%
\pgftext[x=0.481898in, y=2.743238in, left, base]{\color{textcolor}\sffamily\fontsize{10.000000}{12.000000}\selectfont 0.1}%
\end{pgfscope}%
\begin{pgfscope}%
\pgfpathrectangle{\pgfqpoint{0.800000in}{0.528000in}}{\pgfqpoint{4.960000in}{3.696000in}}%
\pgfusepath{clip}%
\pgfsetrectcap%
\pgfsetroundjoin%
\pgfsetlinewidth{0.803000pt}%
\definecolor{currentstroke}{rgb}{0.690196,0.690196,0.690196}%
\pgfsetstrokecolor{currentstroke}%
\pgfsetdash{}{0pt}%
\pgfpathmoveto{\pgfqpoint{0.800000in}{3.216000in}}%
\pgfpathlineto{\pgfqpoint{5.760000in}{3.216000in}}%
\pgfusepath{stroke}%
\end{pgfscope}%
\begin{pgfscope}%
\pgfsetbuttcap%
\pgfsetroundjoin%
\definecolor{currentfill}{rgb}{0.000000,0.000000,0.000000}%
\pgfsetfillcolor{currentfill}%
\pgfsetlinewidth{0.803000pt}%
\definecolor{currentstroke}{rgb}{0.000000,0.000000,0.000000}%
\pgfsetstrokecolor{currentstroke}%
\pgfsetdash{}{0pt}%
\pgfsys@defobject{currentmarker}{\pgfqpoint{-0.048611in}{0.000000in}}{\pgfqpoint{-0.000000in}{0.000000in}}{%
\pgfpathmoveto{\pgfqpoint{-0.000000in}{0.000000in}}%
\pgfpathlineto{\pgfqpoint{-0.048611in}{0.000000in}}%
\pgfusepath{stroke,fill}%
}%
\begin{pgfscope}%
\pgfsys@transformshift{0.800000in}{3.216000in}%
\pgfsys@useobject{currentmarker}{}%
\end{pgfscope}%
\end{pgfscope}%
\begin{pgfscope}%
\definecolor{textcolor}{rgb}{0.000000,0.000000,0.000000}%
\pgfsetstrokecolor{textcolor}%
\pgfsetfillcolor{textcolor}%
\pgftext[x=0.481898in, y=3.163238in, left, base]{\color{textcolor}\sffamily\fontsize{10.000000}{12.000000}\selectfont 0.2}%
\end{pgfscope}%
\begin{pgfscope}%
\pgfpathrectangle{\pgfqpoint{0.800000in}{0.528000in}}{\pgfqpoint{4.960000in}{3.696000in}}%
\pgfusepath{clip}%
\pgfsetrectcap%
\pgfsetroundjoin%
\pgfsetlinewidth{0.803000pt}%
\definecolor{currentstroke}{rgb}{0.690196,0.690196,0.690196}%
\pgfsetstrokecolor{currentstroke}%
\pgfsetdash{}{0pt}%
\pgfpathmoveto{\pgfqpoint{0.800000in}{3.636000in}}%
\pgfpathlineto{\pgfqpoint{5.760000in}{3.636000in}}%
\pgfusepath{stroke}%
\end{pgfscope}%
\begin{pgfscope}%
\pgfsetbuttcap%
\pgfsetroundjoin%
\definecolor{currentfill}{rgb}{0.000000,0.000000,0.000000}%
\pgfsetfillcolor{currentfill}%
\pgfsetlinewidth{0.803000pt}%
\definecolor{currentstroke}{rgb}{0.000000,0.000000,0.000000}%
\pgfsetstrokecolor{currentstroke}%
\pgfsetdash{}{0pt}%
\pgfsys@defobject{currentmarker}{\pgfqpoint{-0.048611in}{0.000000in}}{\pgfqpoint{-0.000000in}{0.000000in}}{%
\pgfpathmoveto{\pgfqpoint{-0.000000in}{0.000000in}}%
\pgfpathlineto{\pgfqpoint{-0.048611in}{0.000000in}}%
\pgfusepath{stroke,fill}%
}%
\begin{pgfscope}%
\pgfsys@transformshift{0.800000in}{3.636000in}%
\pgfsys@useobject{currentmarker}{}%
\end{pgfscope}%
\end{pgfscope}%
\begin{pgfscope}%
\definecolor{textcolor}{rgb}{0.000000,0.000000,0.000000}%
\pgfsetstrokecolor{textcolor}%
\pgfsetfillcolor{textcolor}%
\pgftext[x=0.481898in, y=3.583238in, left, base]{\color{textcolor}\sffamily\fontsize{10.000000}{12.000000}\selectfont 0.3}%
\end{pgfscope}%
\begin{pgfscope}%
\pgfpathrectangle{\pgfqpoint{0.800000in}{0.528000in}}{\pgfqpoint{4.960000in}{3.696000in}}%
\pgfusepath{clip}%
\pgfsetrectcap%
\pgfsetroundjoin%
\pgfsetlinewidth{0.803000pt}%
\definecolor{currentstroke}{rgb}{0.690196,0.690196,0.690196}%
\pgfsetstrokecolor{currentstroke}%
\pgfsetdash{}{0pt}%
\pgfpathmoveto{\pgfqpoint{0.800000in}{4.056000in}}%
\pgfpathlineto{\pgfqpoint{5.760000in}{4.056000in}}%
\pgfusepath{stroke}%
\end{pgfscope}%
\begin{pgfscope}%
\pgfsetbuttcap%
\pgfsetroundjoin%
\definecolor{currentfill}{rgb}{0.000000,0.000000,0.000000}%
\pgfsetfillcolor{currentfill}%
\pgfsetlinewidth{0.803000pt}%
\definecolor{currentstroke}{rgb}{0.000000,0.000000,0.000000}%
\pgfsetstrokecolor{currentstroke}%
\pgfsetdash{}{0pt}%
\pgfsys@defobject{currentmarker}{\pgfqpoint{-0.048611in}{0.000000in}}{\pgfqpoint{-0.000000in}{0.000000in}}{%
\pgfpathmoveto{\pgfqpoint{-0.000000in}{0.000000in}}%
\pgfpathlineto{\pgfqpoint{-0.048611in}{0.000000in}}%
\pgfusepath{stroke,fill}%
}%
\begin{pgfscope}%
\pgfsys@transformshift{0.800000in}{4.056000in}%
\pgfsys@useobject{currentmarker}{}%
\end{pgfscope}%
\end{pgfscope}%
\begin{pgfscope}%
\definecolor{textcolor}{rgb}{0.000000,0.000000,0.000000}%
\pgfsetstrokecolor{textcolor}%
\pgfsetfillcolor{textcolor}%
\pgftext[x=0.481898in, y=4.003238in, left, base]{\color{textcolor}\sffamily\fontsize{10.000000}{12.000000}\selectfont 0.4}%
\end{pgfscope}%
\begin{pgfscope}%
\definecolor{textcolor}{rgb}{0.000000,0.000000,0.000000}%
\pgfsetstrokecolor{textcolor}%
\pgfsetfillcolor{textcolor}%
\pgftext[x=0.318318in,y=2.376000in,,bottom,rotate=90.000000]{\color{textcolor}\sffamily\fontsize{10.000000}{12.000000}\selectfont  [m/s]}%
\end{pgfscope}%
\begin{pgfscope}%
\pgfpathrectangle{\pgfqpoint{0.800000in}{0.528000in}}{\pgfqpoint{4.960000in}{3.696000in}}%
\pgfusepath{clip}%
\pgfsetrectcap%
\pgfsetroundjoin%
\pgfsetlinewidth{1.505625pt}%
\definecolor{currentstroke}{rgb}{0.121569,0.466667,0.705882}%
\pgfsetstrokecolor{currentstroke}%
\pgfsetdash{}{0pt}%
\pgfpathmoveto{\pgfqpoint{1.025455in}{0.696000in}}%
\pgfpathlineto{\pgfqpoint{1.079665in}{0.696000in}}%
\pgfpathlineto{\pgfqpoint{1.134113in}{0.696000in}}%
\pgfpathlineto{\pgfqpoint{1.188117in}{0.696000in}}%
\pgfpathlineto{\pgfqpoint{1.242157in}{0.696000in}}%
\pgfpathlineto{\pgfqpoint{1.296248in}{0.696000in}}%
\pgfpathlineto{\pgfqpoint{1.350915in}{0.696000in}}%
\pgfpathlineto{\pgfqpoint{1.403640in}{0.696000in}}%
\pgfpathlineto{\pgfqpoint{1.457878in}{0.696000in}}%
\pgfpathlineto{\pgfqpoint{1.512411in}{0.709097in}}%
\pgfpathlineto{\pgfqpoint{1.567710in}{0.934938in}}%
\pgfpathlineto{\pgfqpoint{1.621592in}{1.189611in}}%
\pgfpathlineto{\pgfqpoint{1.675033in}{1.378982in}}%
\pgfpathlineto{\pgfqpoint{1.729231in}{1.614188in}}%
\pgfpathlineto{\pgfqpoint{1.784021in}{1.899778in}}%
\pgfpathlineto{\pgfqpoint{1.837979in}{2.163062in}}%
\pgfpathlineto{\pgfqpoint{1.891832in}{2.339520in}}%
\pgfpathlineto{\pgfqpoint{1.946166in}{2.436913in}}%
\pgfpathlineto{\pgfqpoint{2.000280in}{2.552944in}}%
\pgfpathlineto{\pgfqpoint{2.054665in}{2.569722in}}%
\pgfpathlineto{\pgfqpoint{2.109029in}{2.612965in}}%
\pgfpathlineto{\pgfqpoint{2.164495in}{2.729963in}}%
\pgfpathlineto{\pgfqpoint{2.217998in}{2.862540in}}%
\pgfpathlineto{\pgfqpoint{2.271973in}{2.960287in}}%
\pgfpathlineto{\pgfqpoint{2.325948in}{3.064600in}}%
\pgfpathlineto{\pgfqpoint{2.380122in}{3.038767in}}%
\pgfpathlineto{\pgfqpoint{2.434309in}{2.876485in}}%
\pgfpathlineto{\pgfqpoint{2.488488in}{2.758932in}}%
\pgfpathlineto{\pgfqpoint{2.543866in}{2.586365in}}%
\pgfpathlineto{\pgfqpoint{2.597224in}{2.594774in}}%
\pgfpathlineto{\pgfqpoint{2.651231in}{2.579962in}}%
\pgfpathlineto{\pgfqpoint{2.707093in}{2.613815in}}%
\pgfpathlineto{\pgfqpoint{2.760681in}{2.665522in}}%
\pgfpathlineto{\pgfqpoint{2.814881in}{2.665494in}}%
\pgfpathlineto{\pgfqpoint{2.869226in}{2.708381in}}%
\pgfpathlineto{\pgfqpoint{2.923559in}{2.715149in}}%
\pgfpathlineto{\pgfqpoint{2.977685in}{2.760317in}}%
\pgfpathlineto{\pgfqpoint{3.032178in}{2.816993in}}%
\pgfpathlineto{\pgfqpoint{3.086584in}{2.788978in}}%
\pgfpathlineto{\pgfqpoint{3.141612in}{2.732266in}}%
\pgfpathlineto{\pgfqpoint{3.195303in}{2.776310in}}%
\pgfpathlineto{\pgfqpoint{3.250049in}{2.737843in}}%
\pgfpathlineto{\pgfqpoint{3.304703in}{2.706453in}}%
\pgfpathlineto{\pgfqpoint{3.359114in}{2.368754in}}%
\pgfpathlineto{\pgfqpoint{3.414741in}{2.224878in}}%
\pgfpathlineto{\pgfqpoint{3.466608in}{2.268050in}}%
\pgfpathlineto{\pgfqpoint{3.522268in}{2.302420in}}%
\pgfpathlineto{\pgfqpoint{3.577082in}{2.326103in}}%
\pgfpathlineto{\pgfqpoint{3.632050in}{2.330708in}}%
\pgfpathlineto{\pgfqpoint{3.689200in}{2.330137in}}%
\pgfpathlineto{\pgfqpoint{3.741909in}{2.297493in}}%
\pgfpathlineto{\pgfqpoint{3.795389in}{2.277311in}}%
\pgfpathlineto{\pgfqpoint{3.849246in}{2.685366in}}%
\pgfpathlineto{\pgfqpoint{3.902955in}{2.439932in}}%
\pgfpathlineto{\pgfqpoint{3.957232in}{2.617037in}}%
\pgfpathlineto{\pgfqpoint{4.011673in}{2.556758in}}%
\pgfpathlineto{\pgfqpoint{4.066015in}{2.585680in}}%
\pgfpathlineto{\pgfqpoint{4.119961in}{2.570190in}}%
\pgfpathlineto{\pgfqpoint{4.174047in}{2.500854in}}%
\pgfpathlineto{\pgfqpoint{4.228304in}{2.128227in}}%
\pgfpathlineto{\pgfqpoint{4.282576in}{2.103223in}}%
\pgfpathlineto{\pgfqpoint{4.336784in}{2.179995in}}%
\pgfpathlineto{\pgfqpoint{4.391348in}{2.146110in}}%
\pgfpathlineto{\pgfqpoint{4.445504in}{2.113819in}}%
\pgfpathlineto{\pgfqpoint{4.499131in}{2.083715in}}%
\pgfpathlineto{\pgfqpoint{4.553874in}{2.066308in}}%
\pgfpathlineto{\pgfqpoint{4.608189in}{2.477615in}}%
\pgfpathlineto{\pgfqpoint{4.663418in}{2.254498in}}%
\pgfpathlineto{\pgfqpoint{4.717330in}{2.259084in}}%
\pgfpathlineto{\pgfqpoint{4.770855in}{2.297907in}}%
\pgfpathlineto{\pgfqpoint{4.825033in}{2.337597in}}%
\pgfpathlineto{\pgfqpoint{4.879404in}{2.426272in}}%
\pgfpathlineto{\pgfqpoint{4.933681in}{2.467607in}}%
\pgfpathlineto{\pgfqpoint{4.987637in}{2.441835in}}%
\pgfpathlineto{\pgfqpoint{5.041976in}{2.405868in}}%
\pgfpathlineto{\pgfqpoint{5.095774in}{2.399236in}}%
\pgfpathlineto{\pgfqpoint{5.150594in}{2.415494in}}%
\pgfpathlineto{\pgfqpoint{5.204911in}{2.391277in}}%
\pgfpathlineto{\pgfqpoint{5.260666in}{2.397682in}}%
\pgfpathlineto{\pgfqpoint{5.313665in}{2.440996in}}%
\pgfpathlineto{\pgfqpoint{5.367797in}{2.466956in}}%
\pgfpathlineto{\pgfqpoint{5.421994in}{2.467208in}}%
\pgfpathlineto{\pgfqpoint{5.475988in}{2.467150in}}%
\pgfpathlineto{\pgfqpoint{5.530255in}{2.471411in}}%
\pgfusepath{stroke}%
\end{pgfscope}%
\begin{pgfscope}%
\pgfpathrectangle{\pgfqpoint{0.800000in}{0.528000in}}{\pgfqpoint{4.960000in}{3.696000in}}%
\pgfusepath{clip}%
\pgfsetrectcap%
\pgfsetroundjoin%
\pgfsetlinewidth{1.505625pt}%
\definecolor{currentstroke}{rgb}{1.000000,0.498039,0.054902}%
\pgfsetstrokecolor{currentstroke}%
\pgfsetdash{}{0pt}%
\pgfpathmoveto{\pgfqpoint{1.025455in}{0.696000in}}%
\pgfpathlineto{\pgfqpoint{1.078263in}{0.696000in}}%
\pgfpathlineto{\pgfqpoint{1.133010in}{0.696000in}}%
\pgfpathlineto{\pgfqpoint{1.187483in}{0.696000in}}%
\pgfpathlineto{\pgfqpoint{1.242258in}{0.696000in}}%
\pgfpathlineto{\pgfqpoint{1.296643in}{0.696000in}}%
\pgfpathlineto{\pgfqpoint{1.350939in}{1.655587in}}%
\pgfpathlineto{\pgfqpoint{1.403901in}{2.318726in}}%
\pgfpathlineto{\pgfqpoint{1.459510in}{2.436960in}}%
\pgfpathlineto{\pgfqpoint{1.513139in}{3.000423in}}%
\pgfpathlineto{\pgfqpoint{1.567644in}{2.578330in}}%
\pgfpathlineto{\pgfqpoint{1.621618in}{1.970747in}}%
\pgfpathlineto{\pgfqpoint{1.675729in}{2.204895in}}%
\pgfpathlineto{\pgfqpoint{1.729895in}{2.447252in}}%
\pgfpathlineto{\pgfqpoint{1.784179in}{2.467189in}}%
\pgfpathlineto{\pgfqpoint{1.838904in}{2.554258in}}%
\pgfpathlineto{\pgfqpoint{1.893129in}{2.378287in}}%
\pgfpathlineto{\pgfqpoint{1.947745in}{2.567001in}}%
\pgfpathlineto{\pgfqpoint{2.003392in}{2.367348in}}%
\pgfpathlineto{\pgfqpoint{2.056611in}{2.548155in}}%
\pgfpathlineto{\pgfqpoint{2.110648in}{2.288103in}}%
\pgfpathlineto{\pgfqpoint{2.164726in}{2.010855in}}%
\pgfpathlineto{\pgfqpoint{2.218626in}{2.206705in}}%
\pgfpathlineto{\pgfqpoint{2.272733in}{2.096350in}}%
\pgfpathlineto{\pgfqpoint{2.327130in}{2.093779in}}%
\pgfpathlineto{\pgfqpoint{2.381217in}{2.154728in}}%
\pgfpathlineto{\pgfqpoint{2.435587in}{2.171326in}}%
\pgfpathlineto{\pgfqpoint{2.489898in}{2.212242in}}%
\pgfpathlineto{\pgfqpoint{2.545040in}{2.375263in}}%
\pgfpathlineto{\pgfqpoint{2.600426in}{2.347789in}}%
\pgfpathlineto{\pgfqpoint{2.653686in}{2.282193in}}%
\pgfpathlineto{\pgfqpoint{2.707725in}{2.274691in}}%
\pgfpathlineto{\pgfqpoint{2.761988in}{2.347533in}}%
\pgfpathlineto{\pgfqpoint{2.816260in}{2.480858in}}%
\pgfpathlineto{\pgfqpoint{2.870425in}{2.549439in}}%
\pgfpathlineto{\pgfqpoint{2.924372in}{2.439006in}}%
\pgfpathlineto{\pgfqpoint{2.978748in}{2.451067in}}%
\pgfpathlineto{\pgfqpoint{3.033313in}{2.451807in}}%
\pgfpathlineto{\pgfqpoint{3.087307in}{2.417828in}}%
\pgfpathlineto{\pgfqpoint{3.142796in}{2.261441in}}%
\pgfpathlineto{\pgfqpoint{3.196681in}{2.167195in}}%
\pgfpathlineto{\pgfqpoint{3.250528in}{2.334111in}}%
\pgfpathlineto{\pgfqpoint{3.304817in}{2.279424in}}%
\pgfpathlineto{\pgfqpoint{3.359166in}{2.288729in}}%
\pgfpathlineto{\pgfqpoint{3.412991in}{2.429293in}}%
\pgfpathlineto{\pgfqpoint{3.467110in}{2.385639in}}%
\pgfpathlineto{\pgfqpoint{3.521390in}{2.451769in}}%
\pgfpathlineto{\pgfqpoint{3.575431in}{2.444220in}}%
\pgfpathlineto{\pgfqpoint{3.630336in}{2.402287in}}%
\pgfpathlineto{\pgfqpoint{3.686150in}{2.452834in}}%
\pgfpathlineto{\pgfqpoint{3.739878in}{2.416626in}}%
\pgfpathlineto{\pgfqpoint{3.794228in}{2.409179in}}%
\pgfpathlineto{\pgfqpoint{3.848984in}{2.428695in}}%
\pgfpathlineto{\pgfqpoint{3.903165in}{2.377556in}}%
\pgfpathlineto{\pgfqpoint{3.957354in}{2.383285in}}%
\pgfpathlineto{\pgfqpoint{4.011160in}{2.376371in}}%
\pgfpathlineto{\pgfqpoint{4.065633in}{2.325181in}}%
\pgfpathlineto{\pgfqpoint{4.120192in}{2.348279in}}%
\pgfpathlineto{\pgfqpoint{4.174830in}{2.323410in}}%
\pgfpathlineto{\pgfqpoint{4.230258in}{2.318792in}}%
\pgfpathlineto{\pgfqpoint{4.283868in}{2.307707in}}%
\pgfpathlineto{\pgfqpoint{4.337679in}{2.273770in}}%
\pgfpathlineto{\pgfqpoint{4.391628in}{2.307677in}}%
\pgfpathlineto{\pgfqpoint{4.445754in}{2.285932in}}%
\pgfpathlineto{\pgfqpoint{4.500264in}{2.292603in}}%
\pgfpathlineto{\pgfqpoint{4.554089in}{2.384457in}}%
\pgfpathlineto{\pgfqpoint{4.608092in}{2.349637in}}%
\pgfpathlineto{\pgfqpoint{4.662545in}{2.364849in}}%
\pgfpathlineto{\pgfqpoint{4.716756in}{2.392280in}}%
\pgfpathlineto{\pgfqpoint{4.770986in}{2.383739in}}%
\pgfpathlineto{\pgfqpoint{4.826964in}{2.432408in}}%
\pgfpathlineto{\pgfqpoint{4.880439in}{2.400641in}}%
\pgfpathlineto{\pgfqpoint{4.934083in}{2.401415in}}%
\pgfpathlineto{\pgfqpoint{4.987905in}{2.397884in}}%
\pgfpathlineto{\pgfqpoint{5.042036in}{2.396678in}}%
\pgfpathlineto{\pgfqpoint{5.096206in}{2.346256in}}%
\pgfpathlineto{\pgfqpoint{5.150197in}{2.300881in}}%
\pgfpathlineto{\pgfqpoint{5.204296in}{2.303677in}}%
\pgfpathlineto{\pgfqpoint{5.258850in}{2.332802in}}%
\pgfpathlineto{\pgfqpoint{5.312880in}{2.289932in}}%
\pgfpathlineto{\pgfqpoint{5.367388in}{2.321383in}}%
\pgfpathlineto{\pgfqpoint{5.421613in}{2.316173in}}%
\pgfpathlineto{\pgfqpoint{5.477603in}{2.329279in}}%
\pgfpathlineto{\pgfqpoint{5.530472in}{2.395165in}}%
\pgfusepath{stroke}%
\end{pgfscope}%
\begin{pgfscope}%
\pgfpathrectangle{\pgfqpoint{0.800000in}{0.528000in}}{\pgfqpoint{4.960000in}{3.696000in}}%
\pgfusepath{clip}%
\pgfsetrectcap%
\pgfsetroundjoin%
\pgfsetlinewidth{1.505625pt}%
\definecolor{currentstroke}{rgb}{0.172549,0.627451,0.172549}%
\pgfsetstrokecolor{currentstroke}%
\pgfsetdash{}{0pt}%
\pgfpathmoveto{\pgfqpoint{1.025455in}{0.696000in}}%
\pgfpathlineto{\pgfqpoint{1.079614in}{0.696000in}}%
\pgfpathlineto{\pgfqpoint{1.133861in}{0.696000in}}%
\pgfpathlineto{\pgfqpoint{1.187787in}{0.696000in}}%
\pgfpathlineto{\pgfqpoint{1.241874in}{0.696000in}}%
\pgfpathlineto{\pgfqpoint{1.296152in}{0.980159in}}%
\pgfpathlineto{\pgfqpoint{1.350782in}{1.147879in}}%
\pgfpathlineto{\pgfqpoint{1.404285in}{1.398572in}}%
\pgfpathlineto{\pgfqpoint{1.458732in}{0.912798in}}%
\pgfpathlineto{\pgfqpoint{1.512087in}{1.442466in}}%
\pgfpathlineto{\pgfqpoint{1.566065in}{1.716668in}}%
\pgfpathlineto{\pgfqpoint{1.620418in}{1.758090in}}%
\pgfpathlineto{\pgfqpoint{1.674617in}{1.517654in}}%
\pgfpathlineto{\pgfqpoint{1.729149in}{1.716981in}}%
\pgfpathlineto{\pgfqpoint{1.783586in}{1.975133in}}%
\pgfpathlineto{\pgfqpoint{1.838516in}{2.108476in}}%
\pgfpathlineto{\pgfqpoint{1.892359in}{1.961292in}}%
\pgfpathlineto{\pgfqpoint{1.948823in}{2.115109in}}%
\pgfpathlineto{\pgfqpoint{2.001942in}{2.226012in}}%
\pgfpathlineto{\pgfqpoint{2.056759in}{2.337998in}}%
\pgfpathlineto{\pgfqpoint{2.109781in}{2.378126in}}%
\pgfpathlineto{\pgfqpoint{2.163628in}{2.353883in}}%
\pgfpathlineto{\pgfqpoint{2.217534in}{2.326483in}}%
\pgfpathlineto{\pgfqpoint{2.271782in}{2.524928in}}%
\pgfpathlineto{\pgfqpoint{2.326253in}{2.630980in}}%
\pgfpathlineto{\pgfqpoint{2.380419in}{2.553664in}}%
\pgfpathlineto{\pgfqpoint{2.434688in}{2.660795in}}%
\pgfpathlineto{\pgfqpoint{2.488799in}{2.712576in}}%
\pgfpathlineto{\pgfqpoint{2.544077in}{2.753063in}}%
\pgfpathlineto{\pgfqpoint{2.597819in}{2.764847in}}%
\pgfpathlineto{\pgfqpoint{2.651738in}{2.684786in}}%
\pgfpathlineto{\pgfqpoint{2.706232in}{2.491752in}}%
\pgfpathlineto{\pgfqpoint{2.760092in}{2.968541in}}%
\pgfpathlineto{\pgfqpoint{2.814267in}{2.523860in}}%
\pgfpathlineto{\pgfqpoint{2.868473in}{3.037348in}}%
\pgfpathlineto{\pgfqpoint{2.923125in}{2.558757in}}%
\pgfpathlineto{\pgfqpoint{2.976992in}{2.899134in}}%
\pgfpathlineto{\pgfqpoint{3.031409in}{2.876066in}}%
\pgfpathlineto{\pgfqpoint{3.085485in}{2.883980in}}%
\pgfpathlineto{\pgfqpoint{3.140750in}{2.957930in}}%
\pgfpathlineto{\pgfqpoint{3.194496in}{2.865162in}}%
\pgfpathlineto{\pgfqpoint{3.249577in}{2.986828in}}%
\pgfpathlineto{\pgfqpoint{3.304646in}{3.058161in}}%
\pgfpathlineto{\pgfqpoint{3.358086in}{2.077359in}}%
\pgfpathlineto{\pgfqpoint{3.413304in}{2.528185in}}%
\pgfpathlineto{\pgfqpoint{3.468099in}{2.691048in}}%
\pgfpathlineto{\pgfqpoint{3.522732in}{2.621885in}}%
\pgfpathlineto{\pgfqpoint{3.575691in}{2.397946in}}%
\pgfpathlineto{\pgfqpoint{3.629541in}{2.202045in}}%
\pgfpathlineto{\pgfqpoint{3.685026in}{2.224263in}}%
\pgfpathlineto{\pgfqpoint{3.737242in}{2.338414in}}%
\pgfpathlineto{\pgfqpoint{3.791852in}{2.469269in}}%
\pgfpathlineto{\pgfqpoint{3.845706in}{2.074491in}}%
\pgfpathlineto{\pgfqpoint{3.899946in}{2.377689in}}%
\pgfpathlineto{\pgfqpoint{3.954002in}{2.188447in}}%
\pgfpathlineto{\pgfqpoint{4.008251in}{2.311124in}}%
\pgfpathlineto{\pgfqpoint{4.062421in}{2.220802in}}%
\pgfpathlineto{\pgfqpoint{4.118106in}{2.302731in}}%
\pgfpathlineto{\pgfqpoint{4.171682in}{2.246477in}}%
\pgfpathlineto{\pgfqpoint{4.225578in}{2.214720in}}%
\pgfpathlineto{\pgfqpoint{4.279671in}{2.174239in}}%
\pgfpathlineto{\pgfqpoint{4.333831in}{2.217723in}}%
\pgfpathlineto{\pgfqpoint{4.387845in}{2.283729in}}%
\pgfpathlineto{\pgfqpoint{4.442429in}{2.410648in}}%
\pgfpathlineto{\pgfqpoint{4.496550in}{3.319144in}}%
\pgfpathlineto{\pgfqpoint{4.550875in}{1.827768in}}%
\pgfpathlineto{\pgfqpoint{4.605211in}{3.055498in}}%
\pgfpathlineto{\pgfqpoint{4.659385in}{2.923253in}}%
\pgfpathlineto{\pgfqpoint{4.713258in}{2.548972in}}%
\pgfpathlineto{\pgfqpoint{4.768334in}{2.383682in}}%
\pgfpathlineto{\pgfqpoint{4.822833in}{2.445605in}}%
\pgfpathlineto{\pgfqpoint{4.876677in}{2.340307in}}%
\pgfpathlineto{\pgfqpoint{4.931343in}{2.243998in}}%
\pgfpathlineto{\pgfqpoint{4.985117in}{2.318765in}}%
\pgfpathlineto{\pgfqpoint{5.039326in}{2.350880in}}%
\pgfpathlineto{\pgfqpoint{5.093879in}{2.175030in}}%
\pgfpathlineto{\pgfqpoint{5.148077in}{1.903816in}}%
\pgfpathlineto{\pgfqpoint{5.202190in}{2.115502in}}%
\pgfpathlineto{\pgfqpoint{5.257082in}{2.232276in}}%
\pgfpathlineto{\pgfqpoint{5.311203in}{2.285886in}}%
\pgfpathlineto{\pgfqpoint{5.367165in}{2.078206in}}%
\pgfpathlineto{\pgfqpoint{5.420100in}{2.146661in}}%
\pgfpathlineto{\pgfqpoint{5.473931in}{2.160375in}}%
\pgfpathlineto{\pgfqpoint{5.527868in}{2.247856in}}%
\pgfusepath{stroke}%
\end{pgfscope}%
\begin{pgfscope}%
\pgfpathrectangle{\pgfqpoint{0.800000in}{0.528000in}}{\pgfqpoint{4.960000in}{3.696000in}}%
\pgfusepath{clip}%
\pgfsetrectcap%
\pgfsetroundjoin%
\pgfsetlinewidth{1.505625pt}%
\definecolor{currentstroke}{rgb}{0.839216,0.152941,0.156863}%
\pgfsetstrokecolor{currentstroke}%
\pgfsetdash{}{0pt}%
\pgfpathmoveto{\pgfqpoint{1.025455in}{0.696000in}}%
\pgfpathlineto{\pgfqpoint{1.079597in}{0.696000in}}%
\pgfpathlineto{\pgfqpoint{1.133658in}{0.696000in}}%
\pgfpathlineto{\pgfqpoint{1.189329in}{0.696000in}}%
\pgfpathlineto{\pgfqpoint{1.242761in}{0.696000in}}%
\pgfpathlineto{\pgfqpoint{1.296697in}{0.696000in}}%
\pgfpathlineto{\pgfqpoint{1.349591in}{1.224384in}}%
\pgfpathlineto{\pgfqpoint{1.403833in}{1.977817in}}%
\pgfpathlineto{\pgfqpoint{1.458008in}{3.223999in}}%
\pgfpathlineto{\pgfqpoint{1.512487in}{0.696000in}}%
\pgfpathlineto{\pgfqpoint{1.567559in}{0.696000in}}%
\pgfpathlineto{\pgfqpoint{1.620413in}{1.328190in}}%
\pgfpathlineto{\pgfqpoint{1.674637in}{2.014929in}}%
\pgfpathlineto{\pgfqpoint{1.728965in}{1.639054in}}%
\pgfpathlineto{\pgfqpoint{1.785241in}{2.365961in}}%
\pgfpathlineto{\pgfqpoint{1.839465in}{1.800091in}}%
\pgfpathlineto{\pgfqpoint{1.893575in}{1.810572in}}%
\pgfpathlineto{\pgfqpoint{1.948524in}{2.517109in}}%
\pgfpathlineto{\pgfqpoint{2.002701in}{2.654198in}}%
\pgfpathlineto{\pgfqpoint{2.057640in}{2.380822in}}%
\pgfpathlineto{\pgfqpoint{2.112272in}{2.235386in}}%
\pgfpathlineto{\pgfqpoint{2.166513in}{3.181427in}}%
\pgfpathlineto{\pgfqpoint{2.220857in}{2.991615in}}%
\pgfpathlineto{\pgfqpoint{2.275238in}{2.863635in}}%
\pgfpathlineto{\pgfqpoint{2.329903in}{3.496475in}}%
\pgfpathlineto{\pgfqpoint{2.385513in}{2.911573in}}%
\pgfpathlineto{\pgfqpoint{2.438712in}{2.594331in}}%
\pgfpathlineto{\pgfqpoint{2.492760in}{2.892606in}}%
\pgfpathlineto{\pgfqpoint{2.546778in}{3.332561in}}%
\pgfpathlineto{\pgfqpoint{2.600950in}{3.220757in}}%
\pgfpathlineto{\pgfqpoint{2.655146in}{3.150892in}}%
\pgfpathlineto{\pgfqpoint{2.709202in}{2.227331in}}%
\pgfpathlineto{\pgfqpoint{2.763444in}{1.005386in}}%
\pgfpathlineto{\pgfqpoint{2.817717in}{1.672751in}}%
\pgfpathlineto{\pgfqpoint{2.872018in}{3.681716in}}%
\pgfpathlineto{\pgfqpoint{2.926586in}{2.956971in}}%
\pgfpathlineto{\pgfqpoint{2.980826in}{2.771477in}}%
\pgfpathlineto{\pgfqpoint{3.036130in}{2.413614in}}%
\pgfpathlineto{\pgfqpoint{3.090610in}{2.961204in}}%
\pgfpathlineto{\pgfqpoint{3.145048in}{2.395011in}}%
\pgfpathlineto{\pgfqpoint{3.198567in}{1.835801in}}%
\pgfpathlineto{\pgfqpoint{3.252938in}{1.628504in}}%
\pgfpathlineto{\pgfqpoint{3.306816in}{2.048025in}}%
\pgfpathlineto{\pgfqpoint{3.360919in}{2.990503in}}%
\pgfpathlineto{\pgfqpoint{3.415295in}{2.397954in}}%
\pgfpathlineto{\pgfqpoint{3.473498in}{2.104836in}}%
\pgfpathlineto{\pgfqpoint{3.524214in}{2.818799in}}%
\pgfpathlineto{\pgfqpoint{3.579195in}{2.224017in}}%
\pgfpathlineto{\pgfqpoint{3.632793in}{1.557628in}}%
\pgfpathlineto{\pgfqpoint{3.686740in}{2.467287in}}%
\pgfpathlineto{\pgfqpoint{3.740860in}{2.169865in}}%
\pgfpathlineto{\pgfqpoint{3.794946in}{2.545826in}}%
\pgfpathlineto{\pgfqpoint{3.849077in}{2.368422in}}%
\pgfpathlineto{\pgfqpoint{3.903216in}{2.358144in}}%
\pgfpathlineto{\pgfqpoint{3.957840in}{2.533185in}}%
\pgfpathlineto{\pgfqpoint{4.012092in}{2.485575in}}%
\pgfpathlineto{\pgfqpoint{4.066140in}{2.425386in}}%
\pgfpathlineto{\pgfqpoint{4.122746in}{2.358698in}}%
\pgfpathlineto{\pgfqpoint{4.175990in}{2.256552in}}%
\pgfpathlineto{\pgfqpoint{4.229469in}{2.318819in}}%
\pgfpathlineto{\pgfqpoint{4.284198in}{2.319724in}}%
\pgfpathlineto{\pgfqpoint{4.338336in}{2.313889in}}%
\pgfpathlineto{\pgfqpoint{4.393002in}{2.328212in}}%
\pgfpathlineto{\pgfqpoint{4.447361in}{2.485547in}}%
\pgfpathlineto{\pgfqpoint{4.501391in}{2.355258in}}%
\pgfpathlineto{\pgfqpoint{4.555394in}{2.473540in}}%
\pgfpathlineto{\pgfqpoint{4.609840in}{2.420608in}}%
\pgfpathlineto{\pgfqpoint{4.664142in}{2.444565in}}%
\pgfpathlineto{\pgfqpoint{4.720532in}{2.482613in}}%
\pgfpathlineto{\pgfqpoint{4.773672in}{2.466643in}}%
\pgfpathlineto{\pgfqpoint{4.827513in}{2.512892in}}%
\pgfpathlineto{\pgfqpoint{4.881178in}{2.489659in}}%
\pgfpathlineto{\pgfqpoint{4.935728in}{2.470288in}}%
\pgfpathlineto{\pgfqpoint{4.989451in}{2.596470in}}%
\pgfpathlineto{\pgfqpoint{5.043670in}{2.505739in}}%
\pgfpathlineto{\pgfqpoint{5.097979in}{2.144333in}}%
\pgfpathlineto{\pgfqpoint{5.153158in}{2.052942in}}%
\pgfpathlineto{\pgfqpoint{5.206993in}{1.984611in}}%
\pgfpathlineto{\pgfqpoint{5.261266in}{2.318748in}}%
\pgfpathlineto{\pgfqpoint{5.315995in}{1.913756in}}%
\pgfpathlineto{\pgfqpoint{5.372973in}{2.219543in}}%
\pgfpathlineto{\pgfqpoint{5.426296in}{2.734988in}}%
\pgfpathlineto{\pgfqpoint{5.480510in}{1.973457in}}%
\pgfpathlineto{\pgfqpoint{5.534545in}{2.766510in}}%
\pgfusepath{stroke}%
\end{pgfscope}%
\begin{pgfscope}%
\pgfpathrectangle{\pgfqpoint{0.800000in}{0.528000in}}{\pgfqpoint{4.960000in}{3.696000in}}%
\pgfusepath{clip}%
\pgfsetrectcap%
\pgfsetroundjoin%
\pgfsetlinewidth{1.505625pt}%
\definecolor{currentstroke}{rgb}{0.580392,0.403922,0.741176}%
\pgfsetstrokecolor{currentstroke}%
\pgfsetdash{}{0pt}%
\pgfpathmoveto{\pgfqpoint{1.025455in}{0.696000in}}%
\pgfpathlineto{\pgfqpoint{1.079734in}{0.696000in}}%
\pgfpathlineto{\pgfqpoint{1.133911in}{0.696000in}}%
\pgfpathlineto{\pgfqpoint{1.187729in}{0.696000in}}%
\pgfpathlineto{\pgfqpoint{1.241896in}{0.696000in}}%
\pgfpathlineto{\pgfqpoint{1.296431in}{0.696000in}}%
\pgfpathlineto{\pgfqpoint{1.350693in}{0.966617in}}%
\pgfpathlineto{\pgfqpoint{1.405563in}{2.235000in}}%
\pgfpathlineto{\pgfqpoint{1.460378in}{1.285125in}}%
\pgfpathlineto{\pgfqpoint{1.513982in}{1.114346in}}%
\pgfpathlineto{\pgfqpoint{1.567494in}{2.319018in}}%
\pgfpathlineto{\pgfqpoint{1.621280in}{0.696000in}}%
\pgfpathlineto{\pgfqpoint{1.675519in}{0.730423in}}%
\pgfpathlineto{\pgfqpoint{1.729632in}{2.361485in}}%
\pgfpathlineto{\pgfqpoint{1.783675in}{2.035229in}}%
\pgfpathlineto{\pgfqpoint{1.837906in}{1.703223in}}%
\pgfpathlineto{\pgfqpoint{1.892207in}{1.922044in}}%
\pgfpathlineto{\pgfqpoint{1.946343in}{2.174515in}}%
\pgfpathlineto{\pgfqpoint{2.000663in}{2.960856in}}%
\pgfpathlineto{\pgfqpoint{2.056339in}{3.242618in}}%
\pgfpathlineto{\pgfqpoint{2.109632in}{2.787952in}}%
\pgfpathlineto{\pgfqpoint{2.163520in}{1.711308in}}%
\pgfpathlineto{\pgfqpoint{2.217785in}{4.056000in}}%
\pgfpathlineto{\pgfqpoint{2.271766in}{2.356821in}}%
\pgfpathlineto{\pgfqpoint{2.326053in}{4.056000in}}%
\pgfpathlineto{\pgfqpoint{2.379877in}{1.674364in}}%
\pgfpathlineto{\pgfqpoint{2.434881in}{3.875578in}}%
\pgfpathlineto{\pgfqpoint{2.489085in}{2.103819in}}%
\pgfpathlineto{\pgfqpoint{2.543496in}{4.056000in}}%
\pgfpathlineto{\pgfqpoint{2.597852in}{2.685406in}}%
\pgfpathlineto{\pgfqpoint{2.651899in}{3.562115in}}%
\pgfpathlineto{\pgfqpoint{2.707181in}{2.682808in}}%
\pgfpathlineto{\pgfqpoint{2.760801in}{3.141410in}}%
\pgfpathlineto{\pgfqpoint{2.814883in}{3.664087in}}%
\pgfpathlineto{\pgfqpoint{2.869363in}{2.239770in}}%
\pgfpathlineto{\pgfqpoint{2.923599in}{0.896601in}}%
\pgfpathlineto{\pgfqpoint{2.978074in}{0.696000in}}%
\pgfpathlineto{\pgfqpoint{3.032476in}{2.176206in}}%
\pgfpathlineto{\pgfqpoint{3.087002in}{2.891254in}}%
\pgfpathlineto{\pgfqpoint{3.141179in}{3.286659in}}%
\pgfpathlineto{\pgfqpoint{3.196674in}{1.813741in}}%
\pgfpathlineto{\pgfqpoint{3.250056in}{2.850246in}}%
\pgfpathlineto{\pgfqpoint{3.303754in}{2.294002in}}%
\pgfpathlineto{\pgfqpoint{3.357787in}{1.967358in}}%
\pgfpathlineto{\pgfqpoint{3.412098in}{1.054556in}}%
\pgfpathlineto{\pgfqpoint{3.466296in}{2.251879in}}%
\pgfpathlineto{\pgfqpoint{3.520829in}{1.345892in}}%
\pgfpathlineto{\pgfqpoint{3.575296in}{2.514150in}}%
\pgfpathlineto{\pgfqpoint{3.629615in}{2.442791in}}%
\pgfpathlineto{\pgfqpoint{3.683892in}{2.206855in}}%
\pgfpathlineto{\pgfqpoint{3.738111in}{2.874828in}}%
\pgfpathlineto{\pgfqpoint{3.792485in}{2.106814in}}%
\pgfpathlineto{\pgfqpoint{3.847561in}{2.644619in}}%
\pgfpathlineto{\pgfqpoint{3.900802in}{2.373371in}}%
\pgfpathlineto{\pgfqpoint{3.954925in}{2.798407in}}%
\pgfpathlineto{\pgfqpoint{4.009527in}{2.262125in}}%
\pgfpathlineto{\pgfqpoint{4.063958in}{2.480976in}}%
\pgfpathlineto{\pgfqpoint{4.118390in}{2.467795in}}%
\pgfpathlineto{\pgfqpoint{4.172406in}{2.633224in}}%
\pgfpathlineto{\pgfqpoint{4.226651in}{2.630420in}}%
\pgfpathlineto{\pgfqpoint{4.280748in}{2.522170in}}%
\pgfpathlineto{\pgfqpoint{4.335006in}{2.407226in}}%
\pgfpathlineto{\pgfqpoint{4.389535in}{2.654789in}}%
\pgfpathlineto{\pgfqpoint{4.445238in}{2.781443in}}%
\pgfpathlineto{\pgfqpoint{4.498454in}{2.624345in}}%
\pgfpathlineto{\pgfqpoint{4.552568in}{2.474597in}}%
\pgfpathlineto{\pgfqpoint{4.606635in}{2.434871in}}%
\pgfpathlineto{\pgfqpoint{4.660753in}{2.500337in}}%
\pgfpathlineto{\pgfqpoint{4.714982in}{1.213534in}}%
\pgfpathlineto{\pgfqpoint{4.768961in}{0.696000in}}%
\pgfpathlineto{\pgfqpoint{4.823155in}{2.066908in}}%
\pgfpathlineto{\pgfqpoint{4.877477in}{3.630181in}}%
\pgfpathlineto{\pgfqpoint{4.931730in}{1.279397in}}%
\pgfpathlineto{\pgfqpoint{4.986017in}{1.798541in}}%
\pgfpathlineto{\pgfqpoint{5.040162in}{3.337790in}}%
\pgfpathlineto{\pgfqpoint{5.095447in}{1.827406in}}%
\pgfpathlineto{\pgfqpoint{5.149088in}{4.056000in}}%
\pgfpathlineto{\pgfqpoint{5.203368in}{1.176609in}}%
\pgfpathlineto{\pgfqpoint{5.257185in}{3.170381in}}%
\pgfpathlineto{\pgfqpoint{5.311111in}{2.426465in}}%
\pgfpathlineto{\pgfqpoint{5.365843in}{2.321082in}}%
\pgfpathlineto{\pgfqpoint{5.419849in}{3.064376in}}%
\pgfpathlineto{\pgfqpoint{5.473831in}{2.358920in}}%
\pgfpathlineto{\pgfqpoint{5.528143in}{3.493114in}}%
\pgfusepath{stroke}%
\end{pgfscope}%
\begin{pgfscope}%
\pgfsetrectcap%
\pgfsetmiterjoin%
\pgfsetlinewidth{0.803000pt}%
\definecolor{currentstroke}{rgb}{0.000000,0.000000,0.000000}%
\pgfsetstrokecolor{currentstroke}%
\pgfsetdash{}{0pt}%
\pgfpathmoveto{\pgfqpoint{0.800000in}{0.528000in}}%
\pgfpathlineto{\pgfqpoint{0.800000in}{4.224000in}}%
\pgfusepath{stroke}%
\end{pgfscope}%
\begin{pgfscope}%
\pgfsetrectcap%
\pgfsetmiterjoin%
\pgfsetlinewidth{0.803000pt}%
\definecolor{currentstroke}{rgb}{0.000000,0.000000,0.000000}%
\pgfsetstrokecolor{currentstroke}%
\pgfsetdash{}{0pt}%
\pgfpathmoveto{\pgfqpoint{5.760000in}{0.528000in}}%
\pgfpathlineto{\pgfqpoint{5.760000in}{4.224000in}}%
\pgfusepath{stroke}%
\end{pgfscope}%
\begin{pgfscope}%
\pgfsetrectcap%
\pgfsetmiterjoin%
\pgfsetlinewidth{0.803000pt}%
\definecolor{currentstroke}{rgb}{0.000000,0.000000,0.000000}%
\pgfsetstrokecolor{currentstroke}%
\pgfsetdash{}{0pt}%
\pgfpathmoveto{\pgfqpoint{0.800000in}{0.528000in}}%
\pgfpathlineto{\pgfqpoint{5.760000in}{0.528000in}}%
\pgfusepath{stroke}%
\end{pgfscope}%
\begin{pgfscope}%
\pgfsetrectcap%
\pgfsetmiterjoin%
\pgfsetlinewidth{0.803000pt}%
\definecolor{currentstroke}{rgb}{0.000000,0.000000,0.000000}%
\pgfsetstrokecolor{currentstroke}%
\pgfsetdash{}{0pt}%
\pgfpathmoveto{\pgfqpoint{0.800000in}{4.224000in}}%
\pgfpathlineto{\pgfqpoint{5.760000in}{4.224000in}}%
\pgfusepath{stroke}%
\end{pgfscope}%
\begin{pgfscope}%
\definecolor{textcolor}{rgb}{0.000000,0.000000,0.000000}%
\pgfsetstrokecolor{textcolor}%
\pgfsetfillcolor{textcolor}%
\pgftext[x=3.280000in,y=4.307333in,,base]{\color{textcolor}\sffamily\fontsize{12.000000}{14.400000}\selectfont Forward controller output}%
\end{pgfscope}%
\begin{pgfscope}%
\pgfsetbuttcap%
\pgfsetmiterjoin%
\definecolor{currentfill}{rgb}{1.000000,1.000000,1.000000}%
\pgfsetfillcolor{currentfill}%
\pgfsetfillopacity{0.800000}%
\pgfsetlinewidth{1.003750pt}%
\definecolor{currentstroke}{rgb}{0.800000,0.800000,0.800000}%
\pgfsetstrokecolor{currentstroke}%
\pgfsetstrokeopacity{0.800000}%
\pgfsetdash{}{0pt}%
\pgfpathmoveto{\pgfqpoint{0.897222in}{3.093603in}}%
\pgfpathlineto{\pgfqpoint{1.430032in}{3.093603in}}%
\pgfpathquadraticcurveto{\pgfqpoint{1.457810in}{3.093603in}}{\pgfqpoint{1.457810in}{3.121381in}}%
\pgfpathlineto{\pgfqpoint{1.457810in}{4.126778in}}%
\pgfpathquadraticcurveto{\pgfqpoint{1.457810in}{4.154556in}}{\pgfqpoint{1.430032in}{4.154556in}}%
\pgfpathlineto{\pgfqpoint{0.897222in}{4.154556in}}%
\pgfpathquadraticcurveto{\pgfqpoint{0.869444in}{4.154556in}}{\pgfqpoint{0.869444in}{4.126778in}}%
\pgfpathlineto{\pgfqpoint{0.869444in}{3.121381in}}%
\pgfpathquadraticcurveto{\pgfqpoint{0.869444in}{3.093603in}}{\pgfqpoint{0.897222in}{3.093603in}}%
\pgfpathlineto{\pgfqpoint{0.897222in}{3.093603in}}%
\pgfpathclose%
\pgfusepath{stroke,fill}%
\end{pgfscope}%
\begin{pgfscope}%
\pgfsetrectcap%
\pgfsetroundjoin%
\pgfsetlinewidth{1.505625pt}%
\definecolor{currentstroke}{rgb}{0.121569,0.466667,0.705882}%
\pgfsetstrokecolor{currentstroke}%
\pgfsetdash{}{0pt}%
\pgfpathmoveto{\pgfqpoint{0.925000in}{4.042088in}}%
\pgfpathlineto{\pgfqpoint{1.063889in}{4.042088in}}%
\pgfpathlineto{\pgfqpoint{1.202778in}{4.042088in}}%
\pgfusepath{stroke}%
\end{pgfscope}%
\begin{pgfscope}%
\definecolor{textcolor}{rgb}{0.000000,0.000000,0.000000}%
\pgfsetstrokecolor{textcolor}%
\pgfsetfillcolor{textcolor}%
\pgftext[x=1.313889in,y=3.993477in,left,base]{\color{textcolor}\sffamily\fontsize{10.000000}{12.000000}\selectfont 0}%
\end{pgfscope}%
\begin{pgfscope}%
\pgfsetrectcap%
\pgfsetroundjoin%
\pgfsetlinewidth{1.505625pt}%
\definecolor{currentstroke}{rgb}{1.000000,0.498039,0.054902}%
\pgfsetstrokecolor{currentstroke}%
\pgfsetdash{}{0pt}%
\pgfpathmoveto{\pgfqpoint{0.925000in}{3.838231in}}%
\pgfpathlineto{\pgfqpoint{1.063889in}{3.838231in}}%
\pgfpathlineto{\pgfqpoint{1.202778in}{3.838231in}}%
\pgfusepath{stroke}%
\end{pgfscope}%
\begin{pgfscope}%
\definecolor{textcolor}{rgb}{0.000000,0.000000,0.000000}%
\pgfsetstrokecolor{textcolor}%
\pgfsetfillcolor{textcolor}%
\pgftext[x=1.313889in,y=3.789620in,left,base]{\color{textcolor}\sffamily\fontsize{10.000000}{12.000000}\selectfont 1}%
\end{pgfscope}%
\begin{pgfscope}%
\pgfsetrectcap%
\pgfsetroundjoin%
\pgfsetlinewidth{1.505625pt}%
\definecolor{currentstroke}{rgb}{0.172549,0.627451,0.172549}%
\pgfsetstrokecolor{currentstroke}%
\pgfsetdash{}{0pt}%
\pgfpathmoveto{\pgfqpoint{0.925000in}{3.634374in}}%
\pgfpathlineto{\pgfqpoint{1.063889in}{3.634374in}}%
\pgfpathlineto{\pgfqpoint{1.202778in}{3.634374in}}%
\pgfusepath{stroke}%
\end{pgfscope}%
\begin{pgfscope}%
\definecolor{textcolor}{rgb}{0.000000,0.000000,0.000000}%
\pgfsetstrokecolor{textcolor}%
\pgfsetfillcolor{textcolor}%
\pgftext[x=1.313889in,y=3.585762in,left,base]{\color{textcolor}\sffamily\fontsize{10.000000}{12.000000}\selectfont 2}%
\end{pgfscope}%
\begin{pgfscope}%
\pgfsetrectcap%
\pgfsetroundjoin%
\pgfsetlinewidth{1.505625pt}%
\definecolor{currentstroke}{rgb}{0.839216,0.152941,0.156863}%
\pgfsetstrokecolor{currentstroke}%
\pgfsetdash{}{0pt}%
\pgfpathmoveto{\pgfqpoint{0.925000in}{3.430516in}}%
\pgfpathlineto{\pgfqpoint{1.063889in}{3.430516in}}%
\pgfpathlineto{\pgfqpoint{1.202778in}{3.430516in}}%
\pgfusepath{stroke}%
\end{pgfscope}%
\begin{pgfscope}%
\definecolor{textcolor}{rgb}{0.000000,0.000000,0.000000}%
\pgfsetstrokecolor{textcolor}%
\pgfsetfillcolor{textcolor}%
\pgftext[x=1.313889in,y=3.381905in,left,base]{\color{textcolor}\sffamily\fontsize{10.000000}{12.000000}\selectfont 3}%
\end{pgfscope}%
\begin{pgfscope}%
\pgfsetrectcap%
\pgfsetroundjoin%
\pgfsetlinewidth{1.505625pt}%
\definecolor{currentstroke}{rgb}{0.580392,0.403922,0.741176}%
\pgfsetstrokecolor{currentstroke}%
\pgfsetdash{}{0pt}%
\pgfpathmoveto{\pgfqpoint{0.925000in}{3.226659in}}%
\pgfpathlineto{\pgfqpoint{1.063889in}{3.226659in}}%
\pgfpathlineto{\pgfqpoint{1.202778in}{3.226659in}}%
\pgfusepath{stroke}%
\end{pgfscope}%
\begin{pgfscope}%
\definecolor{textcolor}{rgb}{0.000000,0.000000,0.000000}%
\pgfsetstrokecolor{textcolor}%
\pgfsetfillcolor{textcolor}%
\pgftext[x=1.313889in,y=3.178048in,left,base]{\color{textcolor}\sffamily\fontsize{10.000000}{12.000000}\selectfont 4}%
\end{pgfscope}%
\end{pgfpicture}%
\makeatother%
\endgroup%
}
    \end{minipage}
    \caption{Variation of (a) computed error and (b) output velocity for different values of $K_{I}$ and $K_P=4$, $K_D=0$ while the forward controller is engaged.}
    \label{fig:tune-fwd-int-io}
\end{figure}
\begin{figure}[H]
    \begin{minipage}[t]{0.5\linewidth}
        \centering
        \scalebox{0.55}{%% Creator: Matplotlib, PGF backend
%%
%% To include the figure in your LaTeX document, write
%%   \input{<filename>.pgf}
%%
%% Make sure the required packages are loaded in your preamble
%%   \usepackage{pgf}
%%
%% Also ensure that all the required font packages are loaded; for instance,
%% the lmodern package is sometimes necessary when using math font.
%%   \usepackage{lmodern}
%%
%% Figures using additional raster images can only be included by \input if
%% they are in the same directory as the main LaTeX file. For loading figures
%% from other directories you can use the `import` package
%%   \usepackage{import}
%%
%% and then include the figures with
%%   \import{<path to file>}{<filename>.pgf}
%%
%% Matplotlib used the following preamble
%%   \usepackage{fontspec}
%%   \setmainfont{DejaVuSerif.ttf}[Path=\detokenize{/home/lgonz/tfg-aero/tfg-giaa-dronecontrol/venv/lib/python3.8/site-packages/matplotlib/mpl-data/fonts/ttf/}]
%%   \setsansfont{DejaVuSans.ttf}[Path=\detokenize{/home/lgonz/tfg-aero/tfg-giaa-dronecontrol/venv/lib/python3.8/site-packages/matplotlib/mpl-data/fonts/ttf/}]
%%   \setmonofont{DejaVuSansMono.ttf}[Path=\detokenize{/home/lgonz/tfg-aero/tfg-giaa-dronecontrol/venv/lib/python3.8/site-packages/matplotlib/mpl-data/fonts/ttf/}]
%%
\begingroup%
\makeatletter%
\begin{pgfpicture}%
\pgfpathrectangle{\pgfpointorigin}{\pgfqpoint{6.400000in}{4.800000in}}%
\pgfusepath{use as bounding box, clip}%
\begin{pgfscope}%
\pgfsetbuttcap%
\pgfsetmiterjoin%
\definecolor{currentfill}{rgb}{1.000000,1.000000,1.000000}%
\pgfsetfillcolor{currentfill}%
\pgfsetlinewidth{0.000000pt}%
\definecolor{currentstroke}{rgb}{1.000000,1.000000,1.000000}%
\pgfsetstrokecolor{currentstroke}%
\pgfsetdash{}{0pt}%
\pgfpathmoveto{\pgfqpoint{0.000000in}{0.000000in}}%
\pgfpathlineto{\pgfqpoint{6.400000in}{0.000000in}}%
\pgfpathlineto{\pgfqpoint{6.400000in}{4.800000in}}%
\pgfpathlineto{\pgfqpoint{0.000000in}{4.800000in}}%
\pgfpathlineto{\pgfqpoint{0.000000in}{0.000000in}}%
\pgfpathclose%
\pgfusepath{fill}%
\end{pgfscope}%
\begin{pgfscope}%
\pgfsetbuttcap%
\pgfsetmiterjoin%
\definecolor{currentfill}{rgb}{1.000000,1.000000,1.000000}%
\pgfsetfillcolor{currentfill}%
\pgfsetlinewidth{0.000000pt}%
\definecolor{currentstroke}{rgb}{0.000000,0.000000,0.000000}%
\pgfsetstrokecolor{currentstroke}%
\pgfsetstrokeopacity{0.000000}%
\pgfsetdash{}{0pt}%
\pgfpathmoveto{\pgfqpoint{0.800000in}{0.528000in}}%
\pgfpathlineto{\pgfqpoint{5.760000in}{0.528000in}}%
\pgfpathlineto{\pgfqpoint{5.760000in}{4.224000in}}%
\pgfpathlineto{\pgfqpoint{0.800000in}{4.224000in}}%
\pgfpathlineto{\pgfqpoint{0.800000in}{0.528000in}}%
\pgfpathclose%
\pgfusepath{fill}%
\end{pgfscope}%
\begin{pgfscope}%
\pgfpathrectangle{\pgfqpoint{0.800000in}{0.528000in}}{\pgfqpoint{4.960000in}{3.696000in}}%
\pgfusepath{clip}%
\pgfsetrectcap%
\pgfsetroundjoin%
\pgfsetlinewidth{0.803000pt}%
\definecolor{currentstroke}{rgb}{0.690196,0.690196,0.690196}%
\pgfsetstrokecolor{currentstroke}%
\pgfsetdash{}{0pt}%
\pgfpathmoveto{\pgfqpoint{1.025455in}{0.528000in}}%
\pgfpathlineto{\pgfqpoint{1.025455in}{4.224000in}}%
\pgfusepath{stroke}%
\end{pgfscope}%
\begin{pgfscope}%
\pgfsetbuttcap%
\pgfsetroundjoin%
\definecolor{currentfill}{rgb}{0.000000,0.000000,0.000000}%
\pgfsetfillcolor{currentfill}%
\pgfsetlinewidth{0.803000pt}%
\definecolor{currentstroke}{rgb}{0.000000,0.000000,0.000000}%
\pgfsetstrokecolor{currentstroke}%
\pgfsetdash{}{0pt}%
\pgfsys@defobject{currentmarker}{\pgfqpoint{0.000000in}{-0.048611in}}{\pgfqpoint{0.000000in}{0.000000in}}{%
\pgfpathmoveto{\pgfqpoint{0.000000in}{0.000000in}}%
\pgfpathlineto{\pgfqpoint{0.000000in}{-0.048611in}}%
\pgfusepath{stroke,fill}%
}%
\begin{pgfscope}%
\pgfsys@transformshift{1.025455in}{0.528000in}%
\pgfsys@useobject{currentmarker}{}%
\end{pgfscope}%
\end{pgfscope}%
\begin{pgfscope}%
\definecolor{textcolor}{rgb}{0.000000,0.000000,0.000000}%
\pgfsetstrokecolor{textcolor}%
\pgfsetfillcolor{textcolor}%
\pgftext[x=1.025455in,y=0.430778in,,top]{\color{textcolor}\sffamily\fontsize{10.000000}{12.000000}\selectfont 0}%
\end{pgfscope}%
\begin{pgfscope}%
\pgfpathrectangle{\pgfqpoint{0.800000in}{0.528000in}}{\pgfqpoint{4.960000in}{3.696000in}}%
\pgfusepath{clip}%
\pgfsetrectcap%
\pgfsetroundjoin%
\pgfsetlinewidth{0.803000pt}%
\definecolor{currentstroke}{rgb}{0.690196,0.690196,0.690196}%
\pgfsetstrokecolor{currentstroke}%
\pgfsetdash{}{0pt}%
\pgfpathmoveto{\pgfqpoint{1.775560in}{0.528000in}}%
\pgfpathlineto{\pgfqpoint{1.775560in}{4.224000in}}%
\pgfusepath{stroke}%
\end{pgfscope}%
\begin{pgfscope}%
\pgfsetbuttcap%
\pgfsetroundjoin%
\definecolor{currentfill}{rgb}{0.000000,0.000000,0.000000}%
\pgfsetfillcolor{currentfill}%
\pgfsetlinewidth{0.803000pt}%
\definecolor{currentstroke}{rgb}{0.000000,0.000000,0.000000}%
\pgfsetstrokecolor{currentstroke}%
\pgfsetdash{}{0pt}%
\pgfsys@defobject{currentmarker}{\pgfqpoint{0.000000in}{-0.048611in}}{\pgfqpoint{0.000000in}{0.000000in}}{%
\pgfpathmoveto{\pgfqpoint{0.000000in}{0.000000in}}%
\pgfpathlineto{\pgfqpoint{0.000000in}{-0.048611in}}%
\pgfusepath{stroke,fill}%
}%
\begin{pgfscope}%
\pgfsys@transformshift{1.775560in}{0.528000in}%
\pgfsys@useobject{currentmarker}{}%
\end{pgfscope}%
\end{pgfscope}%
\begin{pgfscope}%
\definecolor{textcolor}{rgb}{0.000000,0.000000,0.000000}%
\pgfsetstrokecolor{textcolor}%
\pgfsetfillcolor{textcolor}%
\pgftext[x=1.775560in,y=0.430778in,,top]{\color{textcolor}\sffamily\fontsize{10.000000}{12.000000}\selectfont 5}%
\end{pgfscope}%
\begin{pgfscope}%
\pgfpathrectangle{\pgfqpoint{0.800000in}{0.528000in}}{\pgfqpoint{4.960000in}{3.696000in}}%
\pgfusepath{clip}%
\pgfsetrectcap%
\pgfsetroundjoin%
\pgfsetlinewidth{0.803000pt}%
\definecolor{currentstroke}{rgb}{0.690196,0.690196,0.690196}%
\pgfsetstrokecolor{currentstroke}%
\pgfsetdash{}{0pt}%
\pgfpathmoveto{\pgfqpoint{2.525665in}{0.528000in}}%
\pgfpathlineto{\pgfqpoint{2.525665in}{4.224000in}}%
\pgfusepath{stroke}%
\end{pgfscope}%
\begin{pgfscope}%
\pgfsetbuttcap%
\pgfsetroundjoin%
\definecolor{currentfill}{rgb}{0.000000,0.000000,0.000000}%
\pgfsetfillcolor{currentfill}%
\pgfsetlinewidth{0.803000pt}%
\definecolor{currentstroke}{rgb}{0.000000,0.000000,0.000000}%
\pgfsetstrokecolor{currentstroke}%
\pgfsetdash{}{0pt}%
\pgfsys@defobject{currentmarker}{\pgfqpoint{0.000000in}{-0.048611in}}{\pgfqpoint{0.000000in}{0.000000in}}{%
\pgfpathmoveto{\pgfqpoint{0.000000in}{0.000000in}}%
\pgfpathlineto{\pgfqpoint{0.000000in}{-0.048611in}}%
\pgfusepath{stroke,fill}%
}%
\begin{pgfscope}%
\pgfsys@transformshift{2.525665in}{0.528000in}%
\pgfsys@useobject{currentmarker}{}%
\end{pgfscope}%
\end{pgfscope}%
\begin{pgfscope}%
\definecolor{textcolor}{rgb}{0.000000,0.000000,0.000000}%
\pgfsetstrokecolor{textcolor}%
\pgfsetfillcolor{textcolor}%
\pgftext[x=2.525665in,y=0.430778in,,top]{\color{textcolor}\sffamily\fontsize{10.000000}{12.000000}\selectfont 10}%
\end{pgfscope}%
\begin{pgfscope}%
\pgfpathrectangle{\pgfqpoint{0.800000in}{0.528000in}}{\pgfqpoint{4.960000in}{3.696000in}}%
\pgfusepath{clip}%
\pgfsetrectcap%
\pgfsetroundjoin%
\pgfsetlinewidth{0.803000pt}%
\definecolor{currentstroke}{rgb}{0.690196,0.690196,0.690196}%
\pgfsetstrokecolor{currentstroke}%
\pgfsetdash{}{0pt}%
\pgfpathmoveto{\pgfqpoint{3.275770in}{0.528000in}}%
\pgfpathlineto{\pgfqpoint{3.275770in}{4.224000in}}%
\pgfusepath{stroke}%
\end{pgfscope}%
\begin{pgfscope}%
\pgfsetbuttcap%
\pgfsetroundjoin%
\definecolor{currentfill}{rgb}{0.000000,0.000000,0.000000}%
\pgfsetfillcolor{currentfill}%
\pgfsetlinewidth{0.803000pt}%
\definecolor{currentstroke}{rgb}{0.000000,0.000000,0.000000}%
\pgfsetstrokecolor{currentstroke}%
\pgfsetdash{}{0pt}%
\pgfsys@defobject{currentmarker}{\pgfqpoint{0.000000in}{-0.048611in}}{\pgfqpoint{0.000000in}{0.000000in}}{%
\pgfpathmoveto{\pgfqpoint{0.000000in}{0.000000in}}%
\pgfpathlineto{\pgfqpoint{0.000000in}{-0.048611in}}%
\pgfusepath{stroke,fill}%
}%
\begin{pgfscope}%
\pgfsys@transformshift{3.275770in}{0.528000in}%
\pgfsys@useobject{currentmarker}{}%
\end{pgfscope}%
\end{pgfscope}%
\begin{pgfscope}%
\definecolor{textcolor}{rgb}{0.000000,0.000000,0.000000}%
\pgfsetstrokecolor{textcolor}%
\pgfsetfillcolor{textcolor}%
\pgftext[x=3.275770in,y=0.430778in,,top]{\color{textcolor}\sffamily\fontsize{10.000000}{12.000000}\selectfont 15}%
\end{pgfscope}%
\begin{pgfscope}%
\pgfpathrectangle{\pgfqpoint{0.800000in}{0.528000in}}{\pgfqpoint{4.960000in}{3.696000in}}%
\pgfusepath{clip}%
\pgfsetrectcap%
\pgfsetroundjoin%
\pgfsetlinewidth{0.803000pt}%
\definecolor{currentstroke}{rgb}{0.690196,0.690196,0.690196}%
\pgfsetstrokecolor{currentstroke}%
\pgfsetdash{}{0pt}%
\pgfpathmoveto{\pgfqpoint{4.025875in}{0.528000in}}%
\pgfpathlineto{\pgfqpoint{4.025875in}{4.224000in}}%
\pgfusepath{stroke}%
\end{pgfscope}%
\begin{pgfscope}%
\pgfsetbuttcap%
\pgfsetroundjoin%
\definecolor{currentfill}{rgb}{0.000000,0.000000,0.000000}%
\pgfsetfillcolor{currentfill}%
\pgfsetlinewidth{0.803000pt}%
\definecolor{currentstroke}{rgb}{0.000000,0.000000,0.000000}%
\pgfsetstrokecolor{currentstroke}%
\pgfsetdash{}{0pt}%
\pgfsys@defobject{currentmarker}{\pgfqpoint{0.000000in}{-0.048611in}}{\pgfqpoint{0.000000in}{0.000000in}}{%
\pgfpathmoveto{\pgfqpoint{0.000000in}{0.000000in}}%
\pgfpathlineto{\pgfqpoint{0.000000in}{-0.048611in}}%
\pgfusepath{stroke,fill}%
}%
\begin{pgfscope}%
\pgfsys@transformshift{4.025875in}{0.528000in}%
\pgfsys@useobject{currentmarker}{}%
\end{pgfscope}%
\end{pgfscope}%
\begin{pgfscope}%
\definecolor{textcolor}{rgb}{0.000000,0.000000,0.000000}%
\pgfsetstrokecolor{textcolor}%
\pgfsetfillcolor{textcolor}%
\pgftext[x=4.025875in,y=0.430778in,,top]{\color{textcolor}\sffamily\fontsize{10.000000}{12.000000}\selectfont 20}%
\end{pgfscope}%
\begin{pgfscope}%
\pgfpathrectangle{\pgfqpoint{0.800000in}{0.528000in}}{\pgfqpoint{4.960000in}{3.696000in}}%
\pgfusepath{clip}%
\pgfsetrectcap%
\pgfsetroundjoin%
\pgfsetlinewidth{0.803000pt}%
\definecolor{currentstroke}{rgb}{0.690196,0.690196,0.690196}%
\pgfsetstrokecolor{currentstroke}%
\pgfsetdash{}{0pt}%
\pgfpathmoveto{\pgfqpoint{4.775981in}{0.528000in}}%
\pgfpathlineto{\pgfqpoint{4.775981in}{4.224000in}}%
\pgfusepath{stroke}%
\end{pgfscope}%
\begin{pgfscope}%
\pgfsetbuttcap%
\pgfsetroundjoin%
\definecolor{currentfill}{rgb}{0.000000,0.000000,0.000000}%
\pgfsetfillcolor{currentfill}%
\pgfsetlinewidth{0.803000pt}%
\definecolor{currentstroke}{rgb}{0.000000,0.000000,0.000000}%
\pgfsetstrokecolor{currentstroke}%
\pgfsetdash{}{0pt}%
\pgfsys@defobject{currentmarker}{\pgfqpoint{0.000000in}{-0.048611in}}{\pgfqpoint{0.000000in}{0.000000in}}{%
\pgfpathmoveto{\pgfqpoint{0.000000in}{0.000000in}}%
\pgfpathlineto{\pgfqpoint{0.000000in}{-0.048611in}}%
\pgfusepath{stroke,fill}%
}%
\begin{pgfscope}%
\pgfsys@transformshift{4.775981in}{0.528000in}%
\pgfsys@useobject{currentmarker}{}%
\end{pgfscope}%
\end{pgfscope}%
\begin{pgfscope}%
\definecolor{textcolor}{rgb}{0.000000,0.000000,0.000000}%
\pgfsetstrokecolor{textcolor}%
\pgfsetfillcolor{textcolor}%
\pgftext[x=4.775981in,y=0.430778in,,top]{\color{textcolor}\sffamily\fontsize{10.000000}{12.000000}\selectfont 25}%
\end{pgfscope}%
\begin{pgfscope}%
\pgfpathrectangle{\pgfqpoint{0.800000in}{0.528000in}}{\pgfqpoint{4.960000in}{3.696000in}}%
\pgfusepath{clip}%
\pgfsetrectcap%
\pgfsetroundjoin%
\pgfsetlinewidth{0.803000pt}%
\definecolor{currentstroke}{rgb}{0.690196,0.690196,0.690196}%
\pgfsetstrokecolor{currentstroke}%
\pgfsetdash{}{0pt}%
\pgfpathmoveto{\pgfqpoint{5.526086in}{0.528000in}}%
\pgfpathlineto{\pgfqpoint{5.526086in}{4.224000in}}%
\pgfusepath{stroke}%
\end{pgfscope}%
\begin{pgfscope}%
\pgfsetbuttcap%
\pgfsetroundjoin%
\definecolor{currentfill}{rgb}{0.000000,0.000000,0.000000}%
\pgfsetfillcolor{currentfill}%
\pgfsetlinewidth{0.803000pt}%
\definecolor{currentstroke}{rgb}{0.000000,0.000000,0.000000}%
\pgfsetstrokecolor{currentstroke}%
\pgfsetdash{}{0pt}%
\pgfsys@defobject{currentmarker}{\pgfqpoint{0.000000in}{-0.048611in}}{\pgfqpoint{0.000000in}{0.000000in}}{%
\pgfpathmoveto{\pgfqpoint{0.000000in}{0.000000in}}%
\pgfpathlineto{\pgfqpoint{0.000000in}{-0.048611in}}%
\pgfusepath{stroke,fill}%
}%
\begin{pgfscope}%
\pgfsys@transformshift{5.526086in}{0.528000in}%
\pgfsys@useobject{currentmarker}{}%
\end{pgfscope}%
\end{pgfscope}%
\begin{pgfscope}%
\definecolor{textcolor}{rgb}{0.000000,0.000000,0.000000}%
\pgfsetstrokecolor{textcolor}%
\pgfsetfillcolor{textcolor}%
\pgftext[x=5.526086in,y=0.430778in,,top]{\color{textcolor}\sffamily\fontsize{10.000000}{12.000000}\selectfont 30}%
\end{pgfscope}%
\begin{pgfscope}%
\definecolor{textcolor}{rgb}{0.000000,0.000000,0.000000}%
\pgfsetstrokecolor{textcolor}%
\pgfsetfillcolor{textcolor}%
\pgftext[x=3.280000in,y=0.240809in,,top]{\color{textcolor}\sffamily\fontsize{10.000000}{12.000000}\selectfont time [s]}%
\end{pgfscope}%
\begin{pgfscope}%
\pgfpathrectangle{\pgfqpoint{0.800000in}{0.528000in}}{\pgfqpoint{4.960000in}{3.696000in}}%
\pgfusepath{clip}%
\pgfsetrectcap%
\pgfsetroundjoin%
\pgfsetlinewidth{0.803000pt}%
\definecolor{currentstroke}{rgb}{0.690196,0.690196,0.690196}%
\pgfsetstrokecolor{currentstroke}%
\pgfsetdash{}{0pt}%
\pgfpathmoveto{\pgfqpoint{0.800000in}{0.536767in}}%
\pgfpathlineto{\pgfqpoint{5.760000in}{0.536767in}}%
\pgfusepath{stroke}%
\end{pgfscope}%
\begin{pgfscope}%
\pgfsetbuttcap%
\pgfsetroundjoin%
\definecolor{currentfill}{rgb}{0.000000,0.000000,0.000000}%
\pgfsetfillcolor{currentfill}%
\pgfsetlinewidth{0.803000pt}%
\definecolor{currentstroke}{rgb}{0.000000,0.000000,0.000000}%
\pgfsetstrokecolor{currentstroke}%
\pgfsetdash{}{0pt}%
\pgfsys@defobject{currentmarker}{\pgfqpoint{-0.048611in}{0.000000in}}{\pgfqpoint{-0.000000in}{0.000000in}}{%
\pgfpathmoveto{\pgfqpoint{-0.000000in}{0.000000in}}%
\pgfpathlineto{\pgfqpoint{-0.048611in}{0.000000in}}%
\pgfusepath{stroke,fill}%
}%
\begin{pgfscope}%
\pgfsys@transformshift{0.800000in}{0.536767in}%
\pgfsys@useobject{currentmarker}{}%
\end{pgfscope}%
\end{pgfscope}%
\begin{pgfscope}%
\definecolor{textcolor}{rgb}{0.000000,0.000000,0.000000}%
\pgfsetstrokecolor{textcolor}%
\pgfsetfillcolor{textcolor}%
\pgftext[x=0.285508in, y=0.484006in, left, base]{\color{textcolor}\sffamily\fontsize{10.000000}{12.000000}\selectfont \ensuremath{-}2.00}%
\end{pgfscope}%
\begin{pgfscope}%
\pgfpathrectangle{\pgfqpoint{0.800000in}{0.528000in}}{\pgfqpoint{4.960000in}{3.696000in}}%
\pgfusepath{clip}%
\pgfsetrectcap%
\pgfsetroundjoin%
\pgfsetlinewidth{0.803000pt}%
\definecolor{currentstroke}{rgb}{0.690196,0.690196,0.690196}%
\pgfsetstrokecolor{currentstroke}%
\pgfsetdash{}{0pt}%
\pgfpathmoveto{\pgfqpoint{0.800000in}{0.973078in}}%
\pgfpathlineto{\pgfqpoint{5.760000in}{0.973078in}}%
\pgfusepath{stroke}%
\end{pgfscope}%
\begin{pgfscope}%
\pgfsetbuttcap%
\pgfsetroundjoin%
\definecolor{currentfill}{rgb}{0.000000,0.000000,0.000000}%
\pgfsetfillcolor{currentfill}%
\pgfsetlinewidth{0.803000pt}%
\definecolor{currentstroke}{rgb}{0.000000,0.000000,0.000000}%
\pgfsetstrokecolor{currentstroke}%
\pgfsetdash{}{0pt}%
\pgfsys@defobject{currentmarker}{\pgfqpoint{-0.048611in}{0.000000in}}{\pgfqpoint{-0.000000in}{0.000000in}}{%
\pgfpathmoveto{\pgfqpoint{-0.000000in}{0.000000in}}%
\pgfpathlineto{\pgfqpoint{-0.048611in}{0.000000in}}%
\pgfusepath{stroke,fill}%
}%
\begin{pgfscope}%
\pgfsys@transformshift{0.800000in}{0.973078in}%
\pgfsys@useobject{currentmarker}{}%
\end{pgfscope}%
\end{pgfscope}%
\begin{pgfscope}%
\definecolor{textcolor}{rgb}{0.000000,0.000000,0.000000}%
\pgfsetstrokecolor{textcolor}%
\pgfsetfillcolor{textcolor}%
\pgftext[x=0.285508in, y=0.920317in, left, base]{\color{textcolor}\sffamily\fontsize{10.000000}{12.000000}\selectfont \ensuremath{-}1.75}%
\end{pgfscope}%
\begin{pgfscope}%
\pgfpathrectangle{\pgfqpoint{0.800000in}{0.528000in}}{\pgfqpoint{4.960000in}{3.696000in}}%
\pgfusepath{clip}%
\pgfsetrectcap%
\pgfsetroundjoin%
\pgfsetlinewidth{0.803000pt}%
\definecolor{currentstroke}{rgb}{0.690196,0.690196,0.690196}%
\pgfsetstrokecolor{currentstroke}%
\pgfsetdash{}{0pt}%
\pgfpathmoveto{\pgfqpoint{0.800000in}{1.409390in}}%
\pgfpathlineto{\pgfqpoint{5.760000in}{1.409390in}}%
\pgfusepath{stroke}%
\end{pgfscope}%
\begin{pgfscope}%
\pgfsetbuttcap%
\pgfsetroundjoin%
\definecolor{currentfill}{rgb}{0.000000,0.000000,0.000000}%
\pgfsetfillcolor{currentfill}%
\pgfsetlinewidth{0.803000pt}%
\definecolor{currentstroke}{rgb}{0.000000,0.000000,0.000000}%
\pgfsetstrokecolor{currentstroke}%
\pgfsetdash{}{0pt}%
\pgfsys@defobject{currentmarker}{\pgfqpoint{-0.048611in}{0.000000in}}{\pgfqpoint{-0.000000in}{0.000000in}}{%
\pgfpathmoveto{\pgfqpoint{-0.000000in}{0.000000in}}%
\pgfpathlineto{\pgfqpoint{-0.048611in}{0.000000in}}%
\pgfusepath{stroke,fill}%
}%
\begin{pgfscope}%
\pgfsys@transformshift{0.800000in}{1.409390in}%
\pgfsys@useobject{currentmarker}{}%
\end{pgfscope}%
\end{pgfscope}%
\begin{pgfscope}%
\definecolor{textcolor}{rgb}{0.000000,0.000000,0.000000}%
\pgfsetstrokecolor{textcolor}%
\pgfsetfillcolor{textcolor}%
\pgftext[x=0.285508in, y=1.356628in, left, base]{\color{textcolor}\sffamily\fontsize{10.000000}{12.000000}\selectfont \ensuremath{-}1.50}%
\end{pgfscope}%
\begin{pgfscope}%
\pgfpathrectangle{\pgfqpoint{0.800000in}{0.528000in}}{\pgfqpoint{4.960000in}{3.696000in}}%
\pgfusepath{clip}%
\pgfsetrectcap%
\pgfsetroundjoin%
\pgfsetlinewidth{0.803000pt}%
\definecolor{currentstroke}{rgb}{0.690196,0.690196,0.690196}%
\pgfsetstrokecolor{currentstroke}%
\pgfsetdash{}{0pt}%
\pgfpathmoveto{\pgfqpoint{0.800000in}{1.845701in}}%
\pgfpathlineto{\pgfqpoint{5.760000in}{1.845701in}}%
\pgfusepath{stroke}%
\end{pgfscope}%
\begin{pgfscope}%
\pgfsetbuttcap%
\pgfsetroundjoin%
\definecolor{currentfill}{rgb}{0.000000,0.000000,0.000000}%
\pgfsetfillcolor{currentfill}%
\pgfsetlinewidth{0.803000pt}%
\definecolor{currentstroke}{rgb}{0.000000,0.000000,0.000000}%
\pgfsetstrokecolor{currentstroke}%
\pgfsetdash{}{0pt}%
\pgfsys@defobject{currentmarker}{\pgfqpoint{-0.048611in}{0.000000in}}{\pgfqpoint{-0.000000in}{0.000000in}}{%
\pgfpathmoveto{\pgfqpoint{-0.000000in}{0.000000in}}%
\pgfpathlineto{\pgfqpoint{-0.048611in}{0.000000in}}%
\pgfusepath{stroke,fill}%
}%
\begin{pgfscope}%
\pgfsys@transformshift{0.800000in}{1.845701in}%
\pgfsys@useobject{currentmarker}{}%
\end{pgfscope}%
\end{pgfscope}%
\begin{pgfscope}%
\definecolor{textcolor}{rgb}{0.000000,0.000000,0.000000}%
\pgfsetstrokecolor{textcolor}%
\pgfsetfillcolor{textcolor}%
\pgftext[x=0.285508in, y=1.792939in, left, base]{\color{textcolor}\sffamily\fontsize{10.000000}{12.000000}\selectfont \ensuremath{-}1.25}%
\end{pgfscope}%
\begin{pgfscope}%
\pgfpathrectangle{\pgfqpoint{0.800000in}{0.528000in}}{\pgfqpoint{4.960000in}{3.696000in}}%
\pgfusepath{clip}%
\pgfsetrectcap%
\pgfsetroundjoin%
\pgfsetlinewidth{0.803000pt}%
\definecolor{currentstroke}{rgb}{0.690196,0.690196,0.690196}%
\pgfsetstrokecolor{currentstroke}%
\pgfsetdash{}{0pt}%
\pgfpathmoveto{\pgfqpoint{0.800000in}{2.282012in}}%
\pgfpathlineto{\pgfqpoint{5.760000in}{2.282012in}}%
\pgfusepath{stroke}%
\end{pgfscope}%
\begin{pgfscope}%
\pgfsetbuttcap%
\pgfsetroundjoin%
\definecolor{currentfill}{rgb}{0.000000,0.000000,0.000000}%
\pgfsetfillcolor{currentfill}%
\pgfsetlinewidth{0.803000pt}%
\definecolor{currentstroke}{rgb}{0.000000,0.000000,0.000000}%
\pgfsetstrokecolor{currentstroke}%
\pgfsetdash{}{0pt}%
\pgfsys@defobject{currentmarker}{\pgfqpoint{-0.048611in}{0.000000in}}{\pgfqpoint{-0.000000in}{0.000000in}}{%
\pgfpathmoveto{\pgfqpoint{-0.000000in}{0.000000in}}%
\pgfpathlineto{\pgfqpoint{-0.048611in}{0.000000in}}%
\pgfusepath{stroke,fill}%
}%
\begin{pgfscope}%
\pgfsys@transformshift{0.800000in}{2.282012in}%
\pgfsys@useobject{currentmarker}{}%
\end{pgfscope}%
\end{pgfscope}%
\begin{pgfscope}%
\definecolor{textcolor}{rgb}{0.000000,0.000000,0.000000}%
\pgfsetstrokecolor{textcolor}%
\pgfsetfillcolor{textcolor}%
\pgftext[x=0.285508in, y=2.229250in, left, base]{\color{textcolor}\sffamily\fontsize{10.000000}{12.000000}\selectfont \ensuremath{-}1.00}%
\end{pgfscope}%
\begin{pgfscope}%
\pgfpathrectangle{\pgfqpoint{0.800000in}{0.528000in}}{\pgfqpoint{4.960000in}{3.696000in}}%
\pgfusepath{clip}%
\pgfsetrectcap%
\pgfsetroundjoin%
\pgfsetlinewidth{0.803000pt}%
\definecolor{currentstroke}{rgb}{0.690196,0.690196,0.690196}%
\pgfsetstrokecolor{currentstroke}%
\pgfsetdash{}{0pt}%
\pgfpathmoveto{\pgfqpoint{0.800000in}{2.718323in}}%
\pgfpathlineto{\pgfqpoint{5.760000in}{2.718323in}}%
\pgfusepath{stroke}%
\end{pgfscope}%
\begin{pgfscope}%
\pgfsetbuttcap%
\pgfsetroundjoin%
\definecolor{currentfill}{rgb}{0.000000,0.000000,0.000000}%
\pgfsetfillcolor{currentfill}%
\pgfsetlinewidth{0.803000pt}%
\definecolor{currentstroke}{rgb}{0.000000,0.000000,0.000000}%
\pgfsetstrokecolor{currentstroke}%
\pgfsetdash{}{0pt}%
\pgfsys@defobject{currentmarker}{\pgfqpoint{-0.048611in}{0.000000in}}{\pgfqpoint{-0.000000in}{0.000000in}}{%
\pgfpathmoveto{\pgfqpoint{-0.000000in}{0.000000in}}%
\pgfpathlineto{\pgfqpoint{-0.048611in}{0.000000in}}%
\pgfusepath{stroke,fill}%
}%
\begin{pgfscope}%
\pgfsys@transformshift{0.800000in}{2.718323in}%
\pgfsys@useobject{currentmarker}{}%
\end{pgfscope}%
\end{pgfscope}%
\begin{pgfscope}%
\definecolor{textcolor}{rgb}{0.000000,0.000000,0.000000}%
\pgfsetstrokecolor{textcolor}%
\pgfsetfillcolor{textcolor}%
\pgftext[x=0.285508in, y=2.665561in, left, base]{\color{textcolor}\sffamily\fontsize{10.000000}{12.000000}\selectfont \ensuremath{-}0.75}%
\end{pgfscope}%
\begin{pgfscope}%
\pgfpathrectangle{\pgfqpoint{0.800000in}{0.528000in}}{\pgfqpoint{4.960000in}{3.696000in}}%
\pgfusepath{clip}%
\pgfsetrectcap%
\pgfsetroundjoin%
\pgfsetlinewidth{0.803000pt}%
\definecolor{currentstroke}{rgb}{0.690196,0.690196,0.690196}%
\pgfsetstrokecolor{currentstroke}%
\pgfsetdash{}{0pt}%
\pgfpathmoveto{\pgfqpoint{0.800000in}{3.154634in}}%
\pgfpathlineto{\pgfqpoint{5.760000in}{3.154634in}}%
\pgfusepath{stroke}%
\end{pgfscope}%
\begin{pgfscope}%
\pgfsetbuttcap%
\pgfsetroundjoin%
\definecolor{currentfill}{rgb}{0.000000,0.000000,0.000000}%
\pgfsetfillcolor{currentfill}%
\pgfsetlinewidth{0.803000pt}%
\definecolor{currentstroke}{rgb}{0.000000,0.000000,0.000000}%
\pgfsetstrokecolor{currentstroke}%
\pgfsetdash{}{0pt}%
\pgfsys@defobject{currentmarker}{\pgfqpoint{-0.048611in}{0.000000in}}{\pgfqpoint{-0.000000in}{0.000000in}}{%
\pgfpathmoveto{\pgfqpoint{-0.000000in}{0.000000in}}%
\pgfpathlineto{\pgfqpoint{-0.048611in}{0.000000in}}%
\pgfusepath{stroke,fill}%
}%
\begin{pgfscope}%
\pgfsys@transformshift{0.800000in}{3.154634in}%
\pgfsys@useobject{currentmarker}{}%
\end{pgfscope}%
\end{pgfscope}%
\begin{pgfscope}%
\definecolor{textcolor}{rgb}{0.000000,0.000000,0.000000}%
\pgfsetstrokecolor{textcolor}%
\pgfsetfillcolor{textcolor}%
\pgftext[x=0.285508in, y=3.101873in, left, base]{\color{textcolor}\sffamily\fontsize{10.000000}{12.000000}\selectfont \ensuremath{-}0.50}%
\end{pgfscope}%
\begin{pgfscope}%
\pgfpathrectangle{\pgfqpoint{0.800000in}{0.528000in}}{\pgfqpoint{4.960000in}{3.696000in}}%
\pgfusepath{clip}%
\pgfsetrectcap%
\pgfsetroundjoin%
\pgfsetlinewidth{0.803000pt}%
\definecolor{currentstroke}{rgb}{0.690196,0.690196,0.690196}%
\pgfsetstrokecolor{currentstroke}%
\pgfsetdash{}{0pt}%
\pgfpathmoveto{\pgfqpoint{0.800000in}{3.590945in}}%
\pgfpathlineto{\pgfqpoint{5.760000in}{3.590945in}}%
\pgfusepath{stroke}%
\end{pgfscope}%
\begin{pgfscope}%
\pgfsetbuttcap%
\pgfsetroundjoin%
\definecolor{currentfill}{rgb}{0.000000,0.000000,0.000000}%
\pgfsetfillcolor{currentfill}%
\pgfsetlinewidth{0.803000pt}%
\definecolor{currentstroke}{rgb}{0.000000,0.000000,0.000000}%
\pgfsetstrokecolor{currentstroke}%
\pgfsetdash{}{0pt}%
\pgfsys@defobject{currentmarker}{\pgfqpoint{-0.048611in}{0.000000in}}{\pgfqpoint{-0.000000in}{0.000000in}}{%
\pgfpathmoveto{\pgfqpoint{-0.000000in}{0.000000in}}%
\pgfpathlineto{\pgfqpoint{-0.048611in}{0.000000in}}%
\pgfusepath{stroke,fill}%
}%
\begin{pgfscope}%
\pgfsys@transformshift{0.800000in}{3.590945in}%
\pgfsys@useobject{currentmarker}{}%
\end{pgfscope}%
\end{pgfscope}%
\begin{pgfscope}%
\definecolor{textcolor}{rgb}{0.000000,0.000000,0.000000}%
\pgfsetstrokecolor{textcolor}%
\pgfsetfillcolor{textcolor}%
\pgftext[x=0.285508in, y=3.538184in, left, base]{\color{textcolor}\sffamily\fontsize{10.000000}{12.000000}\selectfont \ensuremath{-}0.25}%
\end{pgfscope}%
\begin{pgfscope}%
\pgfpathrectangle{\pgfqpoint{0.800000in}{0.528000in}}{\pgfqpoint{4.960000in}{3.696000in}}%
\pgfusepath{clip}%
\pgfsetrectcap%
\pgfsetroundjoin%
\pgfsetlinewidth{0.803000pt}%
\definecolor{currentstroke}{rgb}{0.690196,0.690196,0.690196}%
\pgfsetstrokecolor{currentstroke}%
\pgfsetdash{}{0pt}%
\pgfpathmoveto{\pgfqpoint{0.800000in}{4.027256in}}%
\pgfpathlineto{\pgfqpoint{5.760000in}{4.027256in}}%
\pgfusepath{stroke}%
\end{pgfscope}%
\begin{pgfscope}%
\pgfsetbuttcap%
\pgfsetroundjoin%
\definecolor{currentfill}{rgb}{0.000000,0.000000,0.000000}%
\pgfsetfillcolor{currentfill}%
\pgfsetlinewidth{0.803000pt}%
\definecolor{currentstroke}{rgb}{0.000000,0.000000,0.000000}%
\pgfsetstrokecolor{currentstroke}%
\pgfsetdash{}{0pt}%
\pgfsys@defobject{currentmarker}{\pgfqpoint{-0.048611in}{0.000000in}}{\pgfqpoint{-0.000000in}{0.000000in}}{%
\pgfpathmoveto{\pgfqpoint{-0.000000in}{0.000000in}}%
\pgfpathlineto{\pgfqpoint{-0.048611in}{0.000000in}}%
\pgfusepath{stroke,fill}%
}%
\begin{pgfscope}%
\pgfsys@transformshift{0.800000in}{4.027256in}%
\pgfsys@useobject{currentmarker}{}%
\end{pgfscope}%
\end{pgfscope}%
\begin{pgfscope}%
\definecolor{textcolor}{rgb}{0.000000,0.000000,0.000000}%
\pgfsetstrokecolor{textcolor}%
\pgfsetfillcolor{textcolor}%
\pgftext[x=0.393533in, y=3.974495in, left, base]{\color{textcolor}\sffamily\fontsize{10.000000}{12.000000}\selectfont 0.00}%
\end{pgfscope}%
\begin{pgfscope}%
\definecolor{textcolor}{rgb}{0.000000,0.000000,0.000000}%
\pgfsetstrokecolor{textcolor}%
\pgfsetfillcolor{textcolor}%
\pgftext[x=0.229952in,y=2.376000in,,bottom,rotate=90.000000]{\color{textcolor}\sffamily\fontsize{10.000000}{12.000000}\selectfont Forward movement [m]}%
\end{pgfscope}%
\begin{pgfscope}%
\pgfpathrectangle{\pgfqpoint{0.800000in}{0.528000in}}{\pgfqpoint{4.960000in}{3.696000in}}%
\pgfusepath{clip}%
\pgfsetrectcap%
\pgfsetroundjoin%
\pgfsetlinewidth{1.505625pt}%
\definecolor{currentstroke}{rgb}{0.121569,0.466667,0.705882}%
\pgfsetstrokecolor{currentstroke}%
\pgfsetdash{}{0pt}%
\pgfpathmoveto{\pgfqpoint{1.025455in}{4.056000in}}%
\pgfpathlineto{\pgfqpoint{1.079621in}{4.031944in}}%
\pgfpathlineto{\pgfqpoint{1.134067in}{3.931918in}}%
\pgfpathlineto{\pgfqpoint{1.188105in}{3.777879in}}%
\pgfpathlineto{\pgfqpoint{1.242094in}{3.579811in}}%
\pgfpathlineto{\pgfqpoint{1.296178in}{3.333168in}}%
\pgfpathlineto{\pgfqpoint{1.350915in}{3.069278in}}%
\pgfpathlineto{\pgfqpoint{1.403595in}{2.799724in}}%
\pgfpathlineto{\pgfqpoint{1.457824in}{2.509893in}}%
\pgfpathlineto{\pgfqpoint{1.512339in}{2.244067in}}%
\pgfpathlineto{\pgfqpoint{1.567679in}{1.960712in}}%
\pgfpathlineto{\pgfqpoint{1.621568in}{1.686382in}}%
\pgfpathlineto{\pgfqpoint{1.674970in}{1.451058in}}%
\pgfpathlineto{\pgfqpoint{1.729154in}{1.223965in}}%
\pgfpathlineto{\pgfqpoint{1.783989in}{1.031877in}}%
\pgfpathlineto{\pgfqpoint{1.837894in}{0.886915in}}%
\pgfpathlineto{\pgfqpoint{1.891785in}{0.781619in}}%
\pgfpathlineto{\pgfqpoint{1.946107in}{0.718652in}}%
\pgfpathlineto{\pgfqpoint{2.000204in}{0.696000in}}%
\pgfpathlineto{\pgfqpoint{2.054649in}{0.702779in}}%
\pgfpathlineto{\pgfqpoint{2.108960in}{0.733290in}}%
\pgfpathlineto{\pgfqpoint{2.164485in}{0.781667in}}%
\pgfpathlineto{\pgfqpoint{2.217989in}{0.842250in}}%
\pgfpathlineto{\pgfqpoint{2.271947in}{0.919468in}}%
\pgfpathlineto{\pgfqpoint{2.325870in}{1.012293in}}%
\pgfpathlineto{\pgfqpoint{2.380043in}{1.112158in}}%
\pgfpathlineto{\pgfqpoint{2.434222in}{1.224842in}}%
\pgfpathlineto{\pgfqpoint{2.488415in}{1.343899in}}%
\pgfpathlineto{\pgfqpoint{2.543838in}{1.455803in}}%
\pgfpathlineto{\pgfqpoint{2.597142in}{1.547706in}}%
\pgfpathlineto{\pgfqpoint{2.651160in}{1.624263in}}%
\pgfpathlineto{\pgfqpoint{2.707046in}{1.687844in}}%
\pgfpathlineto{\pgfqpoint{2.760613in}{1.736076in}}%
\pgfpathlineto{\pgfqpoint{2.814827in}{1.777327in}}%
\pgfpathlineto{\pgfqpoint{2.869156in}{1.813376in}}%
\pgfpathlineto{\pgfqpoint{2.923530in}{1.844457in}}%
\pgfpathlineto{\pgfqpoint{2.977612in}{1.880342in}}%
\pgfpathlineto{\pgfqpoint{3.032110in}{1.919424in}}%
\pgfpathlineto{\pgfqpoint{3.086546in}{1.966032in}}%
\pgfpathlineto{\pgfqpoint{3.141581in}{2.019533in}}%
\pgfpathlineto{\pgfqpoint{3.195204in}{2.073031in}}%
\pgfpathlineto{\pgfqpoint{3.250008in}{2.133960in}}%
\pgfpathlineto{\pgfqpoint{3.304642in}{2.196580in}}%
\pgfpathlineto{\pgfqpoint{3.359094in}{2.254144in}}%
\pgfpathlineto{\pgfqpoint{3.414749in}{2.311058in}}%
\pgfpathlineto{\pgfqpoint{3.466566in}{2.342723in}}%
\pgfpathlineto{\pgfqpoint{3.522217in}{2.362697in}}%
\pgfpathlineto{\pgfqpoint{3.577031in}{2.366255in}}%
\pgfpathlineto{\pgfqpoint{3.632084in}{2.363108in}}%
\pgfpathlineto{\pgfqpoint{3.689284in}{2.353407in}}%
\pgfpathlineto{\pgfqpoint{3.741898in}{2.340763in}}%
\pgfpathlineto{\pgfqpoint{3.795342in}{2.322801in}}%
\pgfpathlineto{\pgfqpoint{3.849219in}{2.304141in}}%
\pgfpathlineto{\pgfqpoint{3.902878in}{2.289081in}}%
\pgfpathlineto{\pgfqpoint{3.957147in}{2.286477in}}%
\pgfpathlineto{\pgfqpoint{4.011641in}{2.285480in}}%
\pgfpathlineto{\pgfqpoint{4.065985in}{2.293929in}}%
\pgfpathlineto{\pgfqpoint{4.119943in}{2.309230in}}%
\pgfpathlineto{\pgfqpoint{4.174014in}{2.332920in}}%
\pgfpathlineto{\pgfqpoint{4.228240in}{2.361091in}}%
\pgfpathlineto{\pgfqpoint{4.282506in}{2.383243in}}%
\pgfpathlineto{\pgfqpoint{4.336707in}{2.390608in}}%
\pgfpathlineto{\pgfqpoint{4.391324in}{2.388041in}}%
\pgfpathlineto{\pgfqpoint{4.445418in}{2.377688in}}%
\pgfpathlineto{\pgfqpoint{4.499016in}{2.358818in}}%
\pgfpathlineto{\pgfqpoint{4.553792in}{2.328796in}}%
\pgfpathlineto{\pgfqpoint{4.608095in}{2.288993in}}%
\pgfpathlineto{\pgfqpoint{4.663392in}{2.248529in}}%
\pgfpathlineto{\pgfqpoint{4.717249in}{2.219033in}}%
\pgfpathlineto{\pgfqpoint{4.770790in}{2.187103in}}%
\pgfpathlineto{\pgfqpoint{4.824973in}{2.160739in}}%
\pgfpathlineto{\pgfqpoint{4.879363in}{2.136883in}}%
\pgfpathlineto{\pgfqpoint{4.933609in}{2.117244in}}%
\pgfpathlineto{\pgfqpoint{4.987553in}{2.106464in}}%
\pgfpathlineto{\pgfqpoint{5.041976in}{2.102053in}}%
\pgfpathlineto{\pgfqpoint{5.095720in}{2.101172in}}%
\pgfpathlineto{\pgfqpoint{5.150542in}{2.101434in}}%
\pgfpathlineto{\pgfqpoint{5.204867in}{2.105299in}}%
\pgfpathlineto{\pgfqpoint{5.260652in}{2.111335in}}%
\pgfpathlineto{\pgfqpoint{5.313598in}{2.117425in}}%
\pgfpathlineto{\pgfqpoint{5.367731in}{2.124508in}}%
\pgfpathlineto{\pgfqpoint{5.421938in}{2.135542in}}%
\pgfpathlineto{\pgfqpoint{5.475898in}{2.150770in}}%
\pgfpathlineto{\pgfqpoint{5.530179in}{2.168131in}}%
\pgfusepath{stroke}%
\end{pgfscope}%
\begin{pgfscope}%
\pgfpathrectangle{\pgfqpoint{0.800000in}{0.528000in}}{\pgfqpoint{4.960000in}{3.696000in}}%
\pgfusepath{clip}%
\pgfsetrectcap%
\pgfsetroundjoin%
\pgfsetlinewidth{1.505625pt}%
\definecolor{currentstroke}{rgb}{1.000000,0.498039,0.054902}%
\pgfsetstrokecolor{currentstroke}%
\pgfsetdash{}{0pt}%
\pgfpathmoveto{\pgfqpoint{1.025455in}{4.050235in}}%
\pgfpathlineto{\pgfqpoint{1.078219in}{4.032720in}}%
\pgfpathlineto{\pgfqpoint{1.132936in}{3.943715in}}%
\pgfpathlineto{\pgfqpoint{1.187392in}{3.803884in}}%
\pgfpathlineto{\pgfqpoint{1.242205in}{3.619490in}}%
\pgfpathlineto{\pgfqpoint{1.296628in}{3.383713in}}%
\pgfpathlineto{\pgfqpoint{1.350891in}{3.126977in}}%
\pgfpathlineto{\pgfqpoint{1.403861in}{2.874531in}}%
\pgfpathlineto{\pgfqpoint{1.459530in}{2.653097in}}%
\pgfpathlineto{\pgfqpoint{1.513111in}{2.495972in}}%
\pgfpathlineto{\pgfqpoint{1.567568in}{2.409549in}}%
\pgfpathlineto{\pgfqpoint{1.621564in}{2.390007in}}%
\pgfpathlineto{\pgfqpoint{1.675632in}{2.397139in}}%
\pgfpathlineto{\pgfqpoint{1.729840in}{2.407991in}}%
\pgfpathlineto{\pgfqpoint{1.784194in}{2.430155in}}%
\pgfpathlineto{\pgfqpoint{1.838904in}{2.462214in}}%
\pgfpathlineto{\pgfqpoint{1.893050in}{2.499315in}}%
\pgfpathlineto{\pgfqpoint{1.947643in}{2.539226in}}%
\pgfpathlineto{\pgfqpoint{2.003377in}{2.582252in}}%
\pgfpathlineto{\pgfqpoint{2.056559in}{2.620308in}}%
\pgfpathlineto{\pgfqpoint{2.110632in}{2.652588in}}%
\pgfpathlineto{\pgfqpoint{2.164642in}{2.681031in}}%
\pgfpathlineto{\pgfqpoint{2.218548in}{2.693233in}}%
\pgfpathlineto{\pgfqpoint{2.272691in}{2.687367in}}%
\pgfpathlineto{\pgfqpoint{2.327036in}{2.673392in}}%
\pgfpathlineto{\pgfqpoint{2.381198in}{2.645713in}}%
\pgfpathlineto{\pgfqpoint{2.435498in}{2.608071in}}%
\pgfpathlineto{\pgfqpoint{2.489813in}{2.569516in}}%
\pgfpathlineto{\pgfqpoint{2.544985in}{2.525686in}}%
\pgfpathlineto{\pgfqpoint{2.600407in}{2.486430in}}%
\pgfpathlineto{\pgfqpoint{2.653635in}{2.455974in}}%
\pgfpathlineto{\pgfqpoint{2.707681in}{2.432301in}}%
\pgfpathlineto{\pgfqpoint{2.761906in}{2.411311in}}%
\pgfpathlineto{\pgfqpoint{2.816153in}{2.395532in}}%
\pgfpathlineto{\pgfqpoint{2.870336in}{2.387298in}}%
\pgfpathlineto{\pgfqpoint{2.924320in}{2.388034in}}%
\pgfpathlineto{\pgfqpoint{2.978679in}{2.395561in}}%
\pgfpathlineto{\pgfqpoint{3.033237in}{2.403476in}}%
\pgfpathlineto{\pgfqpoint{3.087250in}{2.418115in}}%
\pgfpathlineto{\pgfqpoint{3.142788in}{2.436162in}}%
\pgfpathlineto{\pgfqpoint{3.196597in}{2.450859in}}%
\pgfpathlineto{\pgfqpoint{3.250471in}{2.454368in}}%
\pgfpathlineto{\pgfqpoint{3.304787in}{2.451137in}}%
\pgfpathlineto{\pgfqpoint{3.359088in}{2.448085in}}%
\pgfpathlineto{\pgfqpoint{3.412967in}{2.441899in}}%
\pgfpathlineto{\pgfqpoint{3.467023in}{2.431668in}}%
\pgfpathlineto{\pgfqpoint{3.521323in}{2.426368in}}%
\pgfpathlineto{\pgfqpoint{3.575366in}{2.423856in}}%
\pgfpathlineto{\pgfqpoint{3.630264in}{2.425337in}}%
\pgfpathlineto{\pgfqpoint{3.686099in}{2.426910in}}%
\pgfpathlineto{\pgfqpoint{3.739827in}{2.430316in}}%
\pgfpathlineto{\pgfqpoint{3.794152in}{2.436091in}}%
\pgfpathlineto{\pgfqpoint{3.848946in}{2.441933in}}%
\pgfpathlineto{\pgfqpoint{3.903098in}{2.447075in}}%
\pgfpathlineto{\pgfqpoint{3.957325in}{2.453764in}}%
\pgfpathlineto{\pgfqpoint{4.011195in}{2.458881in}}%
\pgfpathlineto{\pgfqpoint{4.065648in}{2.466091in}}%
\pgfpathlineto{\pgfqpoint{4.120089in}{2.473432in}}%
\pgfpathlineto{\pgfqpoint{4.174735in}{2.478666in}}%
\pgfpathlineto{\pgfqpoint{4.230250in}{2.482183in}}%
\pgfpathlineto{\pgfqpoint{4.283857in}{2.483187in}}%
\pgfpathlineto{\pgfqpoint{4.337616in}{2.485018in}}%
\pgfpathlineto{\pgfqpoint{4.391557in}{2.485485in}}%
\pgfpathlineto{\pgfqpoint{4.445715in}{2.482250in}}%
\pgfpathlineto{\pgfqpoint{4.500165in}{2.476063in}}%
\pgfpathlineto{\pgfqpoint{4.554023in}{2.465130in}}%
\pgfpathlineto{\pgfqpoint{4.608007in}{2.451363in}}%
\pgfpathlineto{\pgfqpoint{4.662517in}{2.439509in}}%
\pgfpathlineto{\pgfqpoint{4.716704in}{2.429690in}}%
\pgfpathlineto{\pgfqpoint{4.770918in}{2.423457in}}%
\pgfpathlineto{\pgfqpoint{4.826939in}{2.418252in}}%
\pgfpathlineto{\pgfqpoint{4.880405in}{2.415235in}}%
\pgfpathlineto{\pgfqpoint{4.933976in}{2.414575in}}%
\pgfpathlineto{\pgfqpoint{4.987795in}{2.411991in}}%
\pgfpathlineto{\pgfqpoint{5.042006in}{2.408875in}}%
\pgfpathlineto{\pgfqpoint{5.096184in}{2.406965in}}%
\pgfpathlineto{\pgfqpoint{5.150144in}{2.408071in}}%
\pgfpathlineto{\pgfqpoint{5.204225in}{2.411285in}}%
\pgfpathlineto{\pgfqpoint{5.258772in}{2.411211in}}%
\pgfpathlineto{\pgfqpoint{5.312795in}{2.409441in}}%
\pgfpathlineto{\pgfqpoint{5.367331in}{2.409246in}}%
\pgfpathlineto{\pgfqpoint{5.421537in}{2.407771in}}%
\pgfpathlineto{\pgfqpoint{5.477599in}{2.405321in}}%
\pgfpathlineto{\pgfqpoint{5.530378in}{2.399496in}}%
\pgfusepath{stroke}%
\end{pgfscope}%
\begin{pgfscope}%
\pgfpathrectangle{\pgfqpoint{0.800000in}{0.528000in}}{\pgfqpoint{4.960000in}{3.696000in}}%
\pgfusepath{clip}%
\pgfsetrectcap%
\pgfsetroundjoin%
\pgfsetlinewidth{1.505625pt}%
\definecolor{currentstroke}{rgb}{0.172549,0.627451,0.172549}%
\pgfsetstrokecolor{currentstroke}%
\pgfsetdash{}{0pt}%
\pgfpathmoveto{\pgfqpoint{1.025455in}{3.929401in}}%
\pgfpathlineto{\pgfqpoint{1.079547in}{3.911541in}}%
\pgfpathlineto{\pgfqpoint{1.133794in}{3.825542in}}%
\pgfpathlineto{\pgfqpoint{1.187710in}{3.676861in}}%
\pgfpathlineto{\pgfqpoint{1.241823in}{3.473929in}}%
\pgfpathlineto{\pgfqpoint{1.296054in}{3.251985in}}%
\pgfpathlineto{\pgfqpoint{1.350774in}{2.987930in}}%
\pgfpathlineto{\pgfqpoint{1.404251in}{2.725121in}}%
\pgfpathlineto{\pgfqpoint{1.458677in}{2.494324in}}%
\pgfpathlineto{\pgfqpoint{1.512039in}{2.260659in}}%
\pgfpathlineto{\pgfqpoint{1.565969in}{2.048847in}}%
\pgfpathlineto{\pgfqpoint{1.620412in}{1.861955in}}%
\pgfpathlineto{\pgfqpoint{1.674518in}{1.704693in}}%
\pgfpathlineto{\pgfqpoint{1.729185in}{1.574354in}}%
\pgfpathlineto{\pgfqpoint{1.783493in}{1.470077in}}%
\pgfpathlineto{\pgfqpoint{1.838449in}{1.378849in}}%
\pgfpathlineto{\pgfqpoint{1.892271in}{1.311785in}}%
\pgfpathlineto{\pgfqpoint{1.948822in}{1.263405in}}%
\pgfpathlineto{\pgfqpoint{2.001892in}{1.227996in}}%
\pgfpathlineto{\pgfqpoint{2.056770in}{1.201325in}}%
\pgfpathlineto{\pgfqpoint{2.109781in}{1.187520in}}%
\pgfpathlineto{\pgfqpoint{2.163546in}{1.185832in}}%
\pgfpathlineto{\pgfqpoint{2.217499in}{1.189195in}}%
\pgfpathlineto{\pgfqpoint{2.271750in}{1.196902in}}%
\pgfpathlineto{\pgfqpoint{2.326221in}{1.209719in}}%
\pgfpathlineto{\pgfqpoint{2.380311in}{1.231860in}}%
\pgfpathlineto{\pgfqpoint{2.434635in}{1.261518in}}%
\pgfpathlineto{\pgfqpoint{2.488716in}{1.295426in}}%
\pgfpathlineto{\pgfqpoint{2.544057in}{1.338051in}}%
\pgfpathlineto{\pgfqpoint{2.597737in}{1.389808in}}%
\pgfpathlineto{\pgfqpoint{2.651697in}{1.445161in}}%
\pgfpathlineto{\pgfqpoint{2.706178in}{1.506770in}}%
\pgfpathlineto{\pgfqpoint{2.760045in}{1.562271in}}%
\pgfpathlineto{\pgfqpoint{2.814157in}{1.615001in}}%
\pgfpathlineto{\pgfqpoint{2.868425in}{1.676694in}}%
\pgfpathlineto{\pgfqpoint{2.923075in}{1.731866in}}%
\pgfpathlineto{\pgfqpoint{2.976936in}{1.793273in}}%
\pgfpathlineto{\pgfqpoint{3.031307in}{1.851111in}}%
\pgfpathlineto{\pgfqpoint{3.085412in}{1.916860in}}%
\pgfpathlineto{\pgfqpoint{3.140703in}{1.984135in}}%
\pgfpathlineto{\pgfqpoint{3.194425in}{2.054550in}}%
\pgfpathlineto{\pgfqpoint{3.249562in}{2.135170in}}%
\pgfpathlineto{\pgfqpoint{3.304606in}{2.217504in}}%
\pgfpathlineto{\pgfqpoint{3.357992in}{2.308964in}}%
\pgfpathlineto{\pgfqpoint{3.413253in}{2.391277in}}%
\pgfpathlineto{\pgfqpoint{3.468038in}{2.439402in}}%
\pgfpathlineto{\pgfqpoint{3.522678in}{2.482431in}}%
\pgfpathlineto{\pgfqpoint{3.575616in}{2.520342in}}%
\pgfpathlineto{\pgfqpoint{3.629520in}{2.551151in}}%
\pgfpathlineto{\pgfqpoint{3.684969in}{2.567958in}}%
\pgfpathlineto{\pgfqpoint{3.737150in}{2.569769in}}%
\pgfpathlineto{\pgfqpoint{3.791816in}{2.563754in}}%
\pgfpathlineto{\pgfqpoint{3.845655in}{2.556574in}}%
\pgfpathlineto{\pgfqpoint{3.899891in}{2.548202in}}%
\pgfpathlineto{\pgfqpoint{3.953880in}{2.528292in}}%
\pgfpathlineto{\pgfqpoint{4.008212in}{2.508869in}}%
\pgfpathlineto{\pgfqpoint{4.062359in}{2.486612in}}%
\pgfpathlineto{\pgfqpoint{4.118065in}{2.462837in}}%
\pgfpathlineto{\pgfqpoint{4.171607in}{2.436033in}}%
\pgfpathlineto{\pgfqpoint{4.225472in}{2.413214in}}%
\pgfpathlineto{\pgfqpoint{4.279618in}{2.386180in}}%
\pgfpathlineto{\pgfqpoint{4.333765in}{2.357459in}}%
\pgfpathlineto{\pgfqpoint{4.387769in}{2.328527in}}%
\pgfpathlineto{\pgfqpoint{4.442344in}{2.296984in}}%
\pgfpathlineto{\pgfqpoint{4.496438in}{2.270769in}}%
\pgfpathlineto{\pgfqpoint{4.550833in}{2.267174in}}%
\pgfpathlineto{\pgfqpoint{4.605202in}{2.288194in}}%
\pgfpathlineto{\pgfqpoint{4.659356in}{2.295745in}}%
\pgfpathlineto{\pgfqpoint{4.713153in}{2.335772in}}%
\pgfpathlineto{\pgfqpoint{4.768252in}{2.387644in}}%
\pgfpathlineto{\pgfqpoint{4.822749in}{2.438836in}}%
\pgfpathlineto{\pgfqpoint{4.876558in}{2.483277in}}%
\pgfpathlineto{\pgfqpoint{4.931230in}{2.516663in}}%
\pgfpathlineto{\pgfqpoint{4.985009in}{2.538281in}}%
\pgfpathlineto{\pgfqpoint{5.039275in}{2.547822in}}%
\pgfpathlineto{\pgfqpoint{5.093785in}{2.551609in}}%
\pgfpathlineto{\pgfqpoint{5.147994in}{2.551686in}}%
\pgfpathlineto{\pgfqpoint{5.202131in}{2.535833in}}%
\pgfpathlineto{\pgfqpoint{5.257035in}{2.501267in}}%
\pgfpathlineto{\pgfqpoint{5.311110in}{2.462391in}}%
\pgfpathlineto{\pgfqpoint{5.367126in}{2.418496in}}%
\pgfpathlineto{\pgfqpoint{5.420063in}{2.378414in}}%
\pgfpathlineto{\pgfqpoint{5.473871in}{2.334347in}}%
\pgfpathlineto{\pgfqpoint{5.527798in}{2.288454in}}%
\pgfusepath{stroke}%
\end{pgfscope}%
\begin{pgfscope}%
\pgfpathrectangle{\pgfqpoint{0.800000in}{0.528000in}}{\pgfqpoint{4.960000in}{3.696000in}}%
\pgfusepath{clip}%
\pgfsetrectcap%
\pgfsetroundjoin%
\pgfsetlinewidth{1.505625pt}%
\definecolor{currentstroke}{rgb}{0.839216,0.152941,0.156863}%
\pgfsetstrokecolor{currentstroke}%
\pgfsetdash{}{0pt}%
\pgfpathmoveto{\pgfqpoint{1.025455in}{3.936855in}}%
\pgfpathlineto{\pgfqpoint{1.079566in}{3.909877in}}%
\pgfpathlineto{\pgfqpoint{1.133651in}{3.807421in}}%
\pgfpathlineto{\pgfqpoint{1.189374in}{3.647829in}}%
\pgfpathlineto{\pgfqpoint{1.242743in}{3.456666in}}%
\pgfpathlineto{\pgfqpoint{1.296714in}{3.212174in}}%
\pgfpathlineto{\pgfqpoint{1.349583in}{2.964114in}}%
\pgfpathlineto{\pgfqpoint{1.403811in}{2.688894in}}%
\pgfpathlineto{\pgfqpoint{1.457962in}{2.442628in}}%
\pgfpathlineto{\pgfqpoint{1.512440in}{2.273909in}}%
\pgfpathlineto{\pgfqpoint{1.567542in}{2.162971in}}%
\pgfpathlineto{\pgfqpoint{1.620345in}{2.029357in}}%
\pgfpathlineto{\pgfqpoint{1.674588in}{1.879644in}}%
\pgfpathlineto{\pgfqpoint{1.728974in}{1.744091in}}%
\pgfpathlineto{\pgfqpoint{1.785298in}{1.628763in}}%
\pgfpathlineto{\pgfqpoint{1.839487in}{1.530527in}}%
\pgfpathlineto{\pgfqpoint{1.893584in}{1.468632in}}%
\pgfpathlineto{\pgfqpoint{1.948544in}{1.409131in}}%
\pgfpathlineto{\pgfqpoint{2.002700in}{1.360809in}}%
\pgfpathlineto{\pgfqpoint{2.057645in}{1.344668in}}%
\pgfpathlineto{\pgfqpoint{2.112255in}{1.351484in}}%
\pgfpathlineto{\pgfqpoint{2.166504in}{1.361864in}}%
\pgfpathlineto{\pgfqpoint{2.220837in}{1.386507in}}%
\pgfpathlineto{\pgfqpoint{2.275243in}{1.447490in}}%
\pgfpathlineto{\pgfqpoint{2.329860in}{1.526558in}}%
\pgfpathlineto{\pgfqpoint{2.385577in}{1.624021in}}%
\pgfpathlineto{\pgfqpoint{2.438682in}{1.743330in}}%
\pgfpathlineto{\pgfqpoint{2.492730in}{1.861335in}}%
\pgfpathlineto{\pgfqpoint{2.546735in}{1.959368in}}%
\pgfpathlineto{\pgfqpoint{2.600886in}{2.068822in}}%
\pgfpathlineto{\pgfqpoint{2.655099in}{2.188640in}}%
\pgfpathlineto{\pgfqpoint{2.709182in}{2.302562in}}%
\pgfpathlineto{\pgfqpoint{2.763396in}{2.416913in}}%
\pgfpathlineto{\pgfqpoint{2.817692in}{2.471006in}}%
\pgfpathlineto{\pgfqpoint{2.872026in}{2.448092in}}%
\pgfpathlineto{\pgfqpoint{2.926537in}{2.420265in}}%
\pgfpathlineto{\pgfqpoint{2.980773in}{2.432015in}}%
\pgfpathlineto{\pgfqpoint{3.036168in}{2.452287in}}%
\pgfpathlineto{\pgfqpoint{3.090615in}{2.480008in}}%
\pgfpathlineto{\pgfqpoint{3.145054in}{2.508985in}}%
\pgfpathlineto{\pgfqpoint{3.198523in}{2.552568in}}%
\pgfpathlineto{\pgfqpoint{3.252893in}{2.576659in}}%
\pgfpathlineto{\pgfqpoint{3.306756in}{2.568688in}}%
\pgfpathlineto{\pgfqpoint{3.360896in}{2.534135in}}%
\pgfpathlineto{\pgfqpoint{3.415278in}{2.500967in}}%
\pgfpathlineto{\pgfqpoint{3.473519in}{2.489792in}}%
\pgfpathlineto{\pgfqpoint{3.524205in}{2.476908in}}%
\pgfpathlineto{\pgfqpoint{3.579253in}{2.468571in}}%
\pgfpathlineto{\pgfqpoint{3.632797in}{2.476069in}}%
\pgfpathlineto{\pgfqpoint{3.686704in}{2.462463in}}%
\pgfpathlineto{\pgfqpoint{3.740830in}{2.429157in}}%
\pgfpathlineto{\pgfqpoint{3.794927in}{2.405678in}}%
\pgfpathlineto{\pgfqpoint{3.849035in}{2.379118in}}%
\pgfpathlineto{\pgfqpoint{3.903240in}{2.363969in}}%
\pgfpathlineto{\pgfqpoint{3.957832in}{2.353674in}}%
\pgfpathlineto{\pgfqpoint{4.012036in}{2.348488in}}%
\pgfpathlineto{\pgfqpoint{4.066126in}{2.352436in}}%
\pgfpathlineto{\pgfqpoint{4.122806in}{2.357360in}}%
\pgfpathlineto{\pgfqpoint{4.176011in}{2.360585in}}%
\pgfpathlineto{\pgfqpoint{4.229495in}{2.361281in}}%
\pgfpathlineto{\pgfqpoint{4.284122in}{2.360763in}}%
\pgfpathlineto{\pgfqpoint{4.338318in}{2.362235in}}%
\pgfpathlineto{\pgfqpoint{4.392933in}{2.363732in}}%
\pgfpathlineto{\pgfqpoint{4.447362in}{2.364678in}}%
\pgfpathlineto{\pgfqpoint{4.501399in}{2.368060in}}%
\pgfpathlineto{\pgfqpoint{4.555385in}{2.375293in}}%
\pgfpathlineto{\pgfqpoint{4.609811in}{2.377847in}}%
\pgfpathlineto{\pgfqpoint{4.664101in}{2.381879in}}%
\pgfpathlineto{\pgfqpoint{4.720547in}{2.386650in}}%
\pgfpathlineto{\pgfqpoint{4.773658in}{2.394154in}}%
\pgfpathlineto{\pgfqpoint{4.827538in}{2.404636in}}%
\pgfpathlineto{\pgfqpoint{4.881133in}{2.417134in}}%
\pgfpathlineto{\pgfqpoint{4.935726in}{2.432387in}}%
\pgfpathlineto{\pgfqpoint{4.989441in}{2.451062in}}%
\pgfpathlineto{\pgfqpoint{5.043694in}{2.475929in}}%
\pgfpathlineto{\pgfqpoint{5.097926in}{2.505504in}}%
\pgfpathlineto{\pgfqpoint{5.153140in}{2.530601in}}%
\pgfpathlineto{\pgfqpoint{5.207015in}{2.538767in}}%
\pgfpathlineto{\pgfqpoint{5.261250in}{2.531235in}}%
\pgfpathlineto{\pgfqpoint{5.315997in}{2.511984in}}%
\pgfpathlineto{\pgfqpoint{5.372967in}{2.483015in}}%
\pgfpathlineto{\pgfqpoint{5.426319in}{2.443175in}}%
\pgfpathlineto{\pgfqpoint{5.480504in}{2.409411in}}%
\pgfpathlineto{\pgfqpoint{5.534545in}{2.389410in}}%
\pgfusepath{stroke}%
\end{pgfscope}%
\begin{pgfscope}%
\pgfpathrectangle{\pgfqpoint{0.800000in}{0.528000in}}{\pgfqpoint{4.960000in}{3.696000in}}%
\pgfusepath{clip}%
\pgfsetrectcap%
\pgfsetroundjoin%
\pgfsetlinewidth{1.505625pt}%
\definecolor{currentstroke}{rgb}{0.580392,0.403922,0.741176}%
\pgfsetstrokecolor{currentstroke}%
\pgfsetdash{}{0pt}%
\pgfpathmoveto{\pgfqpoint{1.025455in}{4.049434in}}%
\pgfpathlineto{\pgfqpoint{1.079678in}{4.030526in}}%
\pgfpathlineto{\pgfqpoint{1.133870in}{3.938561in}}%
\pgfpathlineto{\pgfqpoint{1.187734in}{3.801054in}}%
\pgfpathlineto{\pgfqpoint{1.241889in}{3.600007in}}%
\pgfpathlineto{\pgfqpoint{1.296423in}{3.358039in}}%
\pgfpathlineto{\pgfqpoint{1.350697in}{3.089539in}}%
\pgfpathlineto{\pgfqpoint{1.405557in}{2.830696in}}%
\pgfpathlineto{\pgfqpoint{1.460418in}{2.572614in}}%
\pgfpathlineto{\pgfqpoint{1.513987in}{2.365178in}}%
\pgfpathlineto{\pgfqpoint{1.567487in}{2.183935in}}%
\pgfpathlineto{\pgfqpoint{1.621247in}{2.015536in}}%
\pgfpathlineto{\pgfqpoint{1.675531in}{1.888999in}}%
\pgfpathlineto{\pgfqpoint{1.729613in}{1.732181in}}%
\pgfpathlineto{\pgfqpoint{1.783628in}{1.578105in}}%
\pgfpathlineto{\pgfqpoint{1.837870in}{1.475518in}}%
\pgfpathlineto{\pgfqpoint{1.892205in}{1.386576in}}%
\pgfpathlineto{\pgfqpoint{1.946340in}{1.313120in}}%
\pgfpathlineto{\pgfqpoint{2.000653in}{1.257623in}}%
\pgfpathlineto{\pgfqpoint{2.056394in}{1.226931in}}%
\pgfpathlineto{\pgfqpoint{2.109624in}{1.247338in}}%
\pgfpathlineto{\pgfqpoint{2.163468in}{1.301427in}}%
\pgfpathlineto{\pgfqpoint{2.217770in}{1.361063in}}%
\pgfpathlineto{\pgfqpoint{2.271725in}{1.418698in}}%
\pgfpathlineto{\pgfqpoint{2.326053in}{1.526372in}}%
\pgfpathlineto{\pgfqpoint{2.379841in}{1.626617in}}%
\pgfpathlineto{\pgfqpoint{2.434883in}{1.758549in}}%
\pgfpathlineto{\pgfqpoint{2.489075in}{1.854179in}}%
\pgfpathlineto{\pgfqpoint{2.543467in}{1.972002in}}%
\pgfpathlineto{\pgfqpoint{2.597829in}{2.072128in}}%
\pgfpathlineto{\pgfqpoint{2.651872in}{2.201660in}}%
\pgfpathlineto{\pgfqpoint{2.707213in}{2.321901in}}%
\pgfpathlineto{\pgfqpoint{2.760777in}{2.447953in}}%
\pgfpathlineto{\pgfqpoint{2.814858in}{2.562781in}}%
\pgfpathlineto{\pgfqpoint{2.869391in}{2.683865in}}%
\pgfpathlineto{\pgfqpoint{2.923610in}{2.803188in}}%
\pgfpathlineto{\pgfqpoint{2.978034in}{2.864304in}}%
\pgfpathlineto{\pgfqpoint{3.032428in}{2.829617in}}%
\pgfpathlineto{\pgfqpoint{3.086957in}{2.731576in}}%
\pgfpathlineto{\pgfqpoint{3.141180in}{2.642608in}}%
\pgfpathlineto{\pgfqpoint{3.196689in}{2.583146in}}%
\pgfpathlineto{\pgfqpoint{3.250042in}{2.555936in}}%
\pgfpathlineto{\pgfqpoint{3.303736in}{2.525060in}}%
\pgfpathlineto{\pgfqpoint{3.357780in}{2.522815in}}%
\pgfpathlineto{\pgfqpoint{3.412059in}{2.511925in}}%
\pgfpathlineto{\pgfqpoint{3.466320in}{2.479297in}}%
\pgfpathlineto{\pgfqpoint{3.520817in}{2.411242in}}%
\pgfpathlineto{\pgfqpoint{3.575235in}{2.338042in}}%
\pgfpathlineto{\pgfqpoint{3.629571in}{2.249452in}}%
\pgfpathlineto{\pgfqpoint{3.683862in}{2.190301in}}%
\pgfpathlineto{\pgfqpoint{3.738111in}{2.142988in}}%
\pgfpathlineto{\pgfqpoint{3.792593in}{2.117544in}}%
\pgfpathlineto{\pgfqpoint{3.847539in}{2.119601in}}%
\pgfpathlineto{\pgfqpoint{3.900746in}{2.119270in}}%
\pgfpathlineto{\pgfqpoint{3.954880in}{2.132764in}}%
\pgfpathlineto{\pgfqpoint{4.009521in}{2.151388in}}%
\pgfpathlineto{\pgfqpoint{4.063965in}{2.184117in}}%
\pgfpathlineto{\pgfqpoint{4.118325in}{2.209837in}}%
\pgfpathlineto{\pgfqpoint{4.172375in}{2.235853in}}%
\pgfpathlineto{\pgfqpoint{4.226630in}{2.264318in}}%
\pgfpathlineto{\pgfqpoint{4.280745in}{2.298316in}}%
\pgfpathlineto{\pgfqpoint{4.335030in}{2.335581in}}%
\pgfpathlineto{\pgfqpoint{4.389502in}{2.366944in}}%
\pgfpathlineto{\pgfqpoint{4.445307in}{2.399437in}}%
\pgfpathlineto{\pgfqpoint{4.498495in}{2.439256in}}%
\pgfpathlineto{\pgfqpoint{4.552551in}{2.481341in}}%
\pgfpathlineto{\pgfqpoint{4.606617in}{2.526112in}}%
\pgfpathlineto{\pgfqpoint{4.660704in}{2.564117in}}%
\pgfpathlineto{\pgfqpoint{4.714940in}{2.593150in}}%
\pgfpathlineto{\pgfqpoint{4.768919in}{2.601571in}}%
\pgfpathlineto{\pgfqpoint{4.823066in}{2.541636in}}%
\pgfpathlineto{\pgfqpoint{4.877425in}{2.443858in}}%
\pgfpathlineto{\pgfqpoint{4.931670in}{2.360697in}}%
\pgfpathlineto{\pgfqpoint{4.986005in}{2.312550in}}%
\pgfpathlineto{\pgfqpoint{5.040119in}{2.235211in}}%
\pgfpathlineto{\pgfqpoint{5.095467in}{2.178844in}}%
\pgfpathlineto{\pgfqpoint{5.149117in}{2.160578in}}%
\pgfpathlineto{\pgfqpoint{5.203347in}{2.156318in}}%
\pgfpathlineto{\pgfqpoint{5.257147in}{2.196319in}}%
\pgfpathlineto{\pgfqpoint{5.311097in}{2.202119in}}%
\pgfpathlineto{\pgfqpoint{5.365808in}{2.245491in}}%
\pgfpathlineto{\pgfqpoint{5.419776in}{2.275600in}}%
\pgfpathlineto{\pgfqpoint{5.473789in}{2.309083in}}%
\pgfpathlineto{\pgfqpoint{5.528091in}{2.355878in}}%
\pgfusepath{stroke}%
\end{pgfscope}%
\begin{pgfscope}%
\pgfpathrectangle{\pgfqpoint{0.800000in}{0.528000in}}{\pgfqpoint{4.960000in}{3.696000in}}%
\pgfusepath{clip}%
\pgfsetrectcap%
\pgfsetroundjoin%
\pgfsetlinewidth{1.505625pt}%
\definecolor{currentstroke}{rgb}{0.549020,0.337255,0.294118}%
\pgfsetstrokecolor{currentstroke}%
\pgfsetdash{}{0pt}%
\pgfpathmoveto{\pgfqpoint{1.025455in}{2.282012in}}%
\pgfpathlineto{\pgfqpoint{5.526086in}{2.282012in}}%
\pgfusepath{stroke}%
\end{pgfscope}%
\begin{pgfscope}%
\pgfsetrectcap%
\pgfsetmiterjoin%
\pgfsetlinewidth{0.803000pt}%
\definecolor{currentstroke}{rgb}{0.000000,0.000000,0.000000}%
\pgfsetstrokecolor{currentstroke}%
\pgfsetdash{}{0pt}%
\pgfpathmoveto{\pgfqpoint{0.800000in}{0.528000in}}%
\pgfpathlineto{\pgfqpoint{0.800000in}{4.224000in}}%
\pgfusepath{stroke}%
\end{pgfscope}%
\begin{pgfscope}%
\pgfsetrectcap%
\pgfsetmiterjoin%
\pgfsetlinewidth{0.803000pt}%
\definecolor{currentstroke}{rgb}{0.000000,0.000000,0.000000}%
\pgfsetstrokecolor{currentstroke}%
\pgfsetdash{}{0pt}%
\pgfpathmoveto{\pgfqpoint{5.760000in}{0.528000in}}%
\pgfpathlineto{\pgfqpoint{5.760000in}{4.224000in}}%
\pgfusepath{stroke}%
\end{pgfscope}%
\begin{pgfscope}%
\pgfsetrectcap%
\pgfsetmiterjoin%
\pgfsetlinewidth{0.803000pt}%
\definecolor{currentstroke}{rgb}{0.000000,0.000000,0.000000}%
\pgfsetstrokecolor{currentstroke}%
\pgfsetdash{}{0pt}%
\pgfpathmoveto{\pgfqpoint{0.800000in}{0.528000in}}%
\pgfpathlineto{\pgfqpoint{5.760000in}{0.528000in}}%
\pgfusepath{stroke}%
\end{pgfscope}%
\begin{pgfscope}%
\pgfsetrectcap%
\pgfsetmiterjoin%
\pgfsetlinewidth{0.803000pt}%
\definecolor{currentstroke}{rgb}{0.000000,0.000000,0.000000}%
\pgfsetstrokecolor{currentstroke}%
\pgfsetdash{}{0pt}%
\pgfpathmoveto{\pgfqpoint{0.800000in}{4.224000in}}%
\pgfpathlineto{\pgfqpoint{5.760000in}{4.224000in}}%
\pgfusepath{stroke}%
\end{pgfscope}%
\begin{pgfscope}%
\definecolor{textcolor}{rgb}{0.000000,0.000000,0.000000}%
\pgfsetstrokecolor{textcolor}%
\pgfsetfillcolor{textcolor}%
\pgftext[x=3.280000in,y=4.307333in,,base]{\color{textcolor}\sffamily\fontsize{12.000000}{14.400000}\selectfont Measured forward position}%
\end{pgfscope}%
\begin{pgfscope}%
\pgfsetbuttcap%
\pgfsetmiterjoin%
\definecolor{currentfill}{rgb}{1.000000,1.000000,1.000000}%
\pgfsetfillcolor{currentfill}%
\pgfsetfillopacity{0.800000}%
\pgfsetlinewidth{1.003750pt}%
\definecolor{currentstroke}{rgb}{0.800000,0.800000,0.800000}%
\pgfsetstrokecolor{currentstroke}%
\pgfsetstrokeopacity{0.800000}%
\pgfsetdash{}{0pt}%
\pgfpathmoveto{\pgfqpoint{4.788646in}{2.889746in}}%
\pgfpathlineto{\pgfqpoint{5.662778in}{2.889746in}}%
\pgfpathquadraticcurveto{\pgfqpoint{5.690556in}{2.889746in}}{\pgfqpoint{5.690556in}{2.917523in}}%
\pgfpathlineto{\pgfqpoint{5.690556in}{4.126778in}}%
\pgfpathquadraticcurveto{\pgfqpoint{5.690556in}{4.154556in}}{\pgfqpoint{5.662778in}{4.154556in}}%
\pgfpathlineto{\pgfqpoint{4.788646in}{4.154556in}}%
\pgfpathquadraticcurveto{\pgfqpoint{4.760868in}{4.154556in}}{\pgfqpoint{4.760868in}{4.126778in}}%
\pgfpathlineto{\pgfqpoint{4.760868in}{2.917523in}}%
\pgfpathquadraticcurveto{\pgfqpoint{4.760868in}{2.889746in}}{\pgfqpoint{4.788646in}{2.889746in}}%
\pgfpathlineto{\pgfqpoint{4.788646in}{2.889746in}}%
\pgfpathclose%
\pgfusepath{stroke,fill}%
\end{pgfscope}%
\begin{pgfscope}%
\pgfsetrectcap%
\pgfsetroundjoin%
\pgfsetlinewidth{1.505625pt}%
\definecolor{currentstroke}{rgb}{0.121569,0.466667,0.705882}%
\pgfsetstrokecolor{currentstroke}%
\pgfsetdash{}{0pt}%
\pgfpathmoveto{\pgfqpoint{4.816424in}{4.042088in}}%
\pgfpathlineto{\pgfqpoint{4.955312in}{4.042088in}}%
\pgfpathlineto{\pgfqpoint{5.094201in}{4.042088in}}%
\pgfusepath{stroke}%
\end{pgfscope}%
\begin{pgfscope}%
\definecolor{textcolor}{rgb}{0.000000,0.000000,0.000000}%
\pgfsetstrokecolor{textcolor}%
\pgfsetfillcolor{textcolor}%
\pgftext[x=5.205312in,y=3.993477in,left,base]{\color{textcolor}\sffamily\fontsize{10.000000}{12.000000}\selectfont 0}%
\end{pgfscope}%
\begin{pgfscope}%
\pgfsetrectcap%
\pgfsetroundjoin%
\pgfsetlinewidth{1.505625pt}%
\definecolor{currentstroke}{rgb}{1.000000,0.498039,0.054902}%
\pgfsetstrokecolor{currentstroke}%
\pgfsetdash{}{0pt}%
\pgfpathmoveto{\pgfqpoint{4.816424in}{3.838231in}}%
\pgfpathlineto{\pgfqpoint{4.955312in}{3.838231in}}%
\pgfpathlineto{\pgfqpoint{5.094201in}{3.838231in}}%
\pgfusepath{stroke}%
\end{pgfscope}%
\begin{pgfscope}%
\definecolor{textcolor}{rgb}{0.000000,0.000000,0.000000}%
\pgfsetstrokecolor{textcolor}%
\pgfsetfillcolor{textcolor}%
\pgftext[x=5.205312in,y=3.789620in,left,base]{\color{textcolor}\sffamily\fontsize{10.000000}{12.000000}\selectfont 1}%
\end{pgfscope}%
\begin{pgfscope}%
\pgfsetrectcap%
\pgfsetroundjoin%
\pgfsetlinewidth{1.505625pt}%
\definecolor{currentstroke}{rgb}{0.172549,0.627451,0.172549}%
\pgfsetstrokecolor{currentstroke}%
\pgfsetdash{}{0pt}%
\pgfpathmoveto{\pgfqpoint{4.816424in}{3.634374in}}%
\pgfpathlineto{\pgfqpoint{4.955312in}{3.634374in}}%
\pgfpathlineto{\pgfqpoint{5.094201in}{3.634374in}}%
\pgfusepath{stroke}%
\end{pgfscope}%
\begin{pgfscope}%
\definecolor{textcolor}{rgb}{0.000000,0.000000,0.000000}%
\pgfsetstrokecolor{textcolor}%
\pgfsetfillcolor{textcolor}%
\pgftext[x=5.205312in,y=3.585762in,left,base]{\color{textcolor}\sffamily\fontsize{10.000000}{12.000000}\selectfont 2}%
\end{pgfscope}%
\begin{pgfscope}%
\pgfsetrectcap%
\pgfsetroundjoin%
\pgfsetlinewidth{1.505625pt}%
\definecolor{currentstroke}{rgb}{0.839216,0.152941,0.156863}%
\pgfsetstrokecolor{currentstroke}%
\pgfsetdash{}{0pt}%
\pgfpathmoveto{\pgfqpoint{4.816424in}{3.430516in}}%
\pgfpathlineto{\pgfqpoint{4.955312in}{3.430516in}}%
\pgfpathlineto{\pgfqpoint{5.094201in}{3.430516in}}%
\pgfusepath{stroke}%
\end{pgfscope}%
\begin{pgfscope}%
\definecolor{textcolor}{rgb}{0.000000,0.000000,0.000000}%
\pgfsetstrokecolor{textcolor}%
\pgfsetfillcolor{textcolor}%
\pgftext[x=5.205312in,y=3.381905in,left,base]{\color{textcolor}\sffamily\fontsize{10.000000}{12.000000}\selectfont 3}%
\end{pgfscope}%
\begin{pgfscope}%
\pgfsetrectcap%
\pgfsetroundjoin%
\pgfsetlinewidth{1.505625pt}%
\definecolor{currentstroke}{rgb}{0.580392,0.403922,0.741176}%
\pgfsetstrokecolor{currentstroke}%
\pgfsetdash{}{0pt}%
\pgfpathmoveto{\pgfqpoint{4.816424in}{3.226659in}}%
\pgfpathlineto{\pgfqpoint{4.955312in}{3.226659in}}%
\pgfpathlineto{\pgfqpoint{5.094201in}{3.226659in}}%
\pgfusepath{stroke}%
\end{pgfscope}%
\begin{pgfscope}%
\definecolor{textcolor}{rgb}{0.000000,0.000000,0.000000}%
\pgfsetstrokecolor{textcolor}%
\pgfsetfillcolor{textcolor}%
\pgftext[x=5.205312in,y=3.178048in,left,base]{\color{textcolor}\sffamily\fontsize{10.000000}{12.000000}\selectfont 4}%
\end{pgfscope}%
\begin{pgfscope}%
\pgfsetrectcap%
\pgfsetroundjoin%
\pgfsetlinewidth{1.505625pt}%
\definecolor{currentstroke}{rgb}{0.549020,0.337255,0.294118}%
\pgfsetstrokecolor{currentstroke}%
\pgfsetdash{}{0pt}%
\pgfpathmoveto{\pgfqpoint{4.816424in}{3.022802in}}%
\pgfpathlineto{\pgfqpoint{4.955312in}{3.022802in}}%
\pgfpathlineto{\pgfqpoint{5.094201in}{3.022802in}}%
\pgfusepath{stroke}%
\end{pgfscope}%
\begin{pgfscope}%
\definecolor{textcolor}{rgb}{0.000000,0.000000,0.000000}%
\pgfsetstrokecolor{textcolor}%
\pgfsetfillcolor{textcolor}%
\pgftext[x=5.205312in,y=2.974191in,left,base]{\color{textcolor}\sffamily\fontsize{10.000000}{12.000000}\selectfont Target}%
\end{pgfscope}%
\end{pgfpicture}%
\makeatother%
\endgroup%
}
    \end{minipage}
    \begin{minipage}[t]{0.5\linewidth}
        \centering
        \scalebox{0.55}{%% Creator: Matplotlib, PGF backend
%%
%% To include the figure in your LaTeX document, write
%%   \input{<filename>.pgf}
%%
%% Make sure the required packages are loaded in your preamble
%%   \usepackage{pgf}
%%
%% Also ensure that all the required font packages are loaded; for instance,
%% the lmodern package is sometimes necessary when using math font.
%%   \usepackage{lmodern}
%%
%% Figures using additional raster images can only be included by \input if
%% they are in the same directory as the main LaTeX file. For loading figures
%% from other directories you can use the `import` package
%%   \usepackage{import}
%%
%% and then include the figures with
%%   \import{<path to file>}{<filename>.pgf}
%%
%% Matplotlib used the following preamble
%%   \usepackage{fontspec}
%%   \setmainfont{DejaVuSerif.ttf}[Path=\detokenize{/home/lgonz/tfg-aero/tfg-giaa-dronecontrol/venv/lib/python3.8/site-packages/matplotlib/mpl-data/fonts/ttf/}]
%%   \setsansfont{DejaVuSans.ttf}[Path=\detokenize{/home/lgonz/tfg-aero/tfg-giaa-dronecontrol/venv/lib/python3.8/site-packages/matplotlib/mpl-data/fonts/ttf/}]
%%   \setmonofont{DejaVuSansMono.ttf}[Path=\detokenize{/home/lgonz/tfg-aero/tfg-giaa-dronecontrol/venv/lib/python3.8/site-packages/matplotlib/mpl-data/fonts/ttf/}]
%%
\begingroup%
\makeatletter%
\begin{pgfpicture}%
\pgfpathrectangle{\pgfpointorigin}{\pgfqpoint{6.400000in}{4.800000in}}%
\pgfusepath{use as bounding box, clip}%
\begin{pgfscope}%
\pgfsetbuttcap%
\pgfsetmiterjoin%
\definecolor{currentfill}{rgb}{1.000000,1.000000,1.000000}%
\pgfsetfillcolor{currentfill}%
\pgfsetlinewidth{0.000000pt}%
\definecolor{currentstroke}{rgb}{1.000000,1.000000,1.000000}%
\pgfsetstrokecolor{currentstroke}%
\pgfsetdash{}{0pt}%
\pgfpathmoveto{\pgfqpoint{0.000000in}{0.000000in}}%
\pgfpathlineto{\pgfqpoint{6.400000in}{0.000000in}}%
\pgfpathlineto{\pgfqpoint{6.400000in}{4.800000in}}%
\pgfpathlineto{\pgfqpoint{0.000000in}{4.800000in}}%
\pgfpathlineto{\pgfqpoint{0.000000in}{0.000000in}}%
\pgfpathclose%
\pgfusepath{fill}%
\end{pgfscope}%
\begin{pgfscope}%
\pgfsetbuttcap%
\pgfsetmiterjoin%
\definecolor{currentfill}{rgb}{1.000000,1.000000,1.000000}%
\pgfsetfillcolor{currentfill}%
\pgfsetlinewidth{0.000000pt}%
\definecolor{currentstroke}{rgb}{0.000000,0.000000,0.000000}%
\pgfsetstrokecolor{currentstroke}%
\pgfsetstrokeopacity{0.000000}%
\pgfsetdash{}{0pt}%
\pgfpathmoveto{\pgfqpoint{0.800000in}{0.528000in}}%
\pgfpathlineto{\pgfqpoint{5.760000in}{0.528000in}}%
\pgfpathlineto{\pgfqpoint{5.760000in}{4.224000in}}%
\pgfpathlineto{\pgfqpoint{0.800000in}{4.224000in}}%
\pgfpathlineto{\pgfqpoint{0.800000in}{0.528000in}}%
\pgfpathclose%
\pgfusepath{fill}%
\end{pgfscope}%
\begin{pgfscope}%
\pgfpathrectangle{\pgfqpoint{0.800000in}{0.528000in}}{\pgfqpoint{4.960000in}{3.696000in}}%
\pgfusepath{clip}%
\pgfsetrectcap%
\pgfsetroundjoin%
\pgfsetlinewidth{0.803000pt}%
\definecolor{currentstroke}{rgb}{0.690196,0.690196,0.690196}%
\pgfsetstrokecolor{currentstroke}%
\pgfsetdash{}{0pt}%
\pgfpathmoveto{\pgfqpoint{1.025455in}{0.528000in}}%
\pgfpathlineto{\pgfqpoint{1.025455in}{4.224000in}}%
\pgfusepath{stroke}%
\end{pgfscope}%
\begin{pgfscope}%
\pgfsetbuttcap%
\pgfsetroundjoin%
\definecolor{currentfill}{rgb}{0.000000,0.000000,0.000000}%
\pgfsetfillcolor{currentfill}%
\pgfsetlinewidth{0.803000pt}%
\definecolor{currentstroke}{rgb}{0.000000,0.000000,0.000000}%
\pgfsetstrokecolor{currentstroke}%
\pgfsetdash{}{0pt}%
\pgfsys@defobject{currentmarker}{\pgfqpoint{0.000000in}{-0.048611in}}{\pgfqpoint{0.000000in}{0.000000in}}{%
\pgfpathmoveto{\pgfqpoint{0.000000in}{0.000000in}}%
\pgfpathlineto{\pgfqpoint{0.000000in}{-0.048611in}}%
\pgfusepath{stroke,fill}%
}%
\begin{pgfscope}%
\pgfsys@transformshift{1.025455in}{0.528000in}%
\pgfsys@useobject{currentmarker}{}%
\end{pgfscope}%
\end{pgfscope}%
\begin{pgfscope}%
\definecolor{textcolor}{rgb}{0.000000,0.000000,0.000000}%
\pgfsetstrokecolor{textcolor}%
\pgfsetfillcolor{textcolor}%
\pgftext[x=1.025455in,y=0.430778in,,top]{\color{textcolor}\sffamily\fontsize{10.000000}{12.000000}\selectfont 0}%
\end{pgfscope}%
\begin{pgfscope}%
\pgfpathrectangle{\pgfqpoint{0.800000in}{0.528000in}}{\pgfqpoint{4.960000in}{3.696000in}}%
\pgfusepath{clip}%
\pgfsetrectcap%
\pgfsetroundjoin%
\pgfsetlinewidth{0.803000pt}%
\definecolor{currentstroke}{rgb}{0.690196,0.690196,0.690196}%
\pgfsetstrokecolor{currentstroke}%
\pgfsetdash{}{0pt}%
\pgfpathmoveto{\pgfqpoint{1.775560in}{0.528000in}}%
\pgfpathlineto{\pgfqpoint{1.775560in}{4.224000in}}%
\pgfusepath{stroke}%
\end{pgfscope}%
\begin{pgfscope}%
\pgfsetbuttcap%
\pgfsetroundjoin%
\definecolor{currentfill}{rgb}{0.000000,0.000000,0.000000}%
\pgfsetfillcolor{currentfill}%
\pgfsetlinewidth{0.803000pt}%
\definecolor{currentstroke}{rgb}{0.000000,0.000000,0.000000}%
\pgfsetstrokecolor{currentstroke}%
\pgfsetdash{}{0pt}%
\pgfsys@defobject{currentmarker}{\pgfqpoint{0.000000in}{-0.048611in}}{\pgfqpoint{0.000000in}{0.000000in}}{%
\pgfpathmoveto{\pgfqpoint{0.000000in}{0.000000in}}%
\pgfpathlineto{\pgfqpoint{0.000000in}{-0.048611in}}%
\pgfusepath{stroke,fill}%
}%
\begin{pgfscope}%
\pgfsys@transformshift{1.775560in}{0.528000in}%
\pgfsys@useobject{currentmarker}{}%
\end{pgfscope}%
\end{pgfscope}%
\begin{pgfscope}%
\definecolor{textcolor}{rgb}{0.000000,0.000000,0.000000}%
\pgfsetstrokecolor{textcolor}%
\pgfsetfillcolor{textcolor}%
\pgftext[x=1.775560in,y=0.430778in,,top]{\color{textcolor}\sffamily\fontsize{10.000000}{12.000000}\selectfont 5}%
\end{pgfscope}%
\begin{pgfscope}%
\pgfpathrectangle{\pgfqpoint{0.800000in}{0.528000in}}{\pgfqpoint{4.960000in}{3.696000in}}%
\pgfusepath{clip}%
\pgfsetrectcap%
\pgfsetroundjoin%
\pgfsetlinewidth{0.803000pt}%
\definecolor{currentstroke}{rgb}{0.690196,0.690196,0.690196}%
\pgfsetstrokecolor{currentstroke}%
\pgfsetdash{}{0pt}%
\pgfpathmoveto{\pgfqpoint{2.525665in}{0.528000in}}%
\pgfpathlineto{\pgfqpoint{2.525665in}{4.224000in}}%
\pgfusepath{stroke}%
\end{pgfscope}%
\begin{pgfscope}%
\pgfsetbuttcap%
\pgfsetroundjoin%
\definecolor{currentfill}{rgb}{0.000000,0.000000,0.000000}%
\pgfsetfillcolor{currentfill}%
\pgfsetlinewidth{0.803000pt}%
\definecolor{currentstroke}{rgb}{0.000000,0.000000,0.000000}%
\pgfsetstrokecolor{currentstroke}%
\pgfsetdash{}{0pt}%
\pgfsys@defobject{currentmarker}{\pgfqpoint{0.000000in}{-0.048611in}}{\pgfqpoint{0.000000in}{0.000000in}}{%
\pgfpathmoveto{\pgfqpoint{0.000000in}{0.000000in}}%
\pgfpathlineto{\pgfqpoint{0.000000in}{-0.048611in}}%
\pgfusepath{stroke,fill}%
}%
\begin{pgfscope}%
\pgfsys@transformshift{2.525665in}{0.528000in}%
\pgfsys@useobject{currentmarker}{}%
\end{pgfscope}%
\end{pgfscope}%
\begin{pgfscope}%
\definecolor{textcolor}{rgb}{0.000000,0.000000,0.000000}%
\pgfsetstrokecolor{textcolor}%
\pgfsetfillcolor{textcolor}%
\pgftext[x=2.525665in,y=0.430778in,,top]{\color{textcolor}\sffamily\fontsize{10.000000}{12.000000}\selectfont 10}%
\end{pgfscope}%
\begin{pgfscope}%
\pgfpathrectangle{\pgfqpoint{0.800000in}{0.528000in}}{\pgfqpoint{4.960000in}{3.696000in}}%
\pgfusepath{clip}%
\pgfsetrectcap%
\pgfsetroundjoin%
\pgfsetlinewidth{0.803000pt}%
\definecolor{currentstroke}{rgb}{0.690196,0.690196,0.690196}%
\pgfsetstrokecolor{currentstroke}%
\pgfsetdash{}{0pt}%
\pgfpathmoveto{\pgfqpoint{3.275770in}{0.528000in}}%
\pgfpathlineto{\pgfqpoint{3.275770in}{4.224000in}}%
\pgfusepath{stroke}%
\end{pgfscope}%
\begin{pgfscope}%
\pgfsetbuttcap%
\pgfsetroundjoin%
\definecolor{currentfill}{rgb}{0.000000,0.000000,0.000000}%
\pgfsetfillcolor{currentfill}%
\pgfsetlinewidth{0.803000pt}%
\definecolor{currentstroke}{rgb}{0.000000,0.000000,0.000000}%
\pgfsetstrokecolor{currentstroke}%
\pgfsetdash{}{0pt}%
\pgfsys@defobject{currentmarker}{\pgfqpoint{0.000000in}{-0.048611in}}{\pgfqpoint{0.000000in}{0.000000in}}{%
\pgfpathmoveto{\pgfqpoint{0.000000in}{0.000000in}}%
\pgfpathlineto{\pgfqpoint{0.000000in}{-0.048611in}}%
\pgfusepath{stroke,fill}%
}%
\begin{pgfscope}%
\pgfsys@transformshift{3.275770in}{0.528000in}%
\pgfsys@useobject{currentmarker}{}%
\end{pgfscope}%
\end{pgfscope}%
\begin{pgfscope}%
\definecolor{textcolor}{rgb}{0.000000,0.000000,0.000000}%
\pgfsetstrokecolor{textcolor}%
\pgfsetfillcolor{textcolor}%
\pgftext[x=3.275770in,y=0.430778in,,top]{\color{textcolor}\sffamily\fontsize{10.000000}{12.000000}\selectfont 15}%
\end{pgfscope}%
\begin{pgfscope}%
\pgfpathrectangle{\pgfqpoint{0.800000in}{0.528000in}}{\pgfqpoint{4.960000in}{3.696000in}}%
\pgfusepath{clip}%
\pgfsetrectcap%
\pgfsetroundjoin%
\pgfsetlinewidth{0.803000pt}%
\definecolor{currentstroke}{rgb}{0.690196,0.690196,0.690196}%
\pgfsetstrokecolor{currentstroke}%
\pgfsetdash{}{0pt}%
\pgfpathmoveto{\pgfqpoint{4.025875in}{0.528000in}}%
\pgfpathlineto{\pgfqpoint{4.025875in}{4.224000in}}%
\pgfusepath{stroke}%
\end{pgfscope}%
\begin{pgfscope}%
\pgfsetbuttcap%
\pgfsetroundjoin%
\definecolor{currentfill}{rgb}{0.000000,0.000000,0.000000}%
\pgfsetfillcolor{currentfill}%
\pgfsetlinewidth{0.803000pt}%
\definecolor{currentstroke}{rgb}{0.000000,0.000000,0.000000}%
\pgfsetstrokecolor{currentstroke}%
\pgfsetdash{}{0pt}%
\pgfsys@defobject{currentmarker}{\pgfqpoint{0.000000in}{-0.048611in}}{\pgfqpoint{0.000000in}{0.000000in}}{%
\pgfpathmoveto{\pgfqpoint{0.000000in}{0.000000in}}%
\pgfpathlineto{\pgfqpoint{0.000000in}{-0.048611in}}%
\pgfusepath{stroke,fill}%
}%
\begin{pgfscope}%
\pgfsys@transformshift{4.025875in}{0.528000in}%
\pgfsys@useobject{currentmarker}{}%
\end{pgfscope}%
\end{pgfscope}%
\begin{pgfscope}%
\definecolor{textcolor}{rgb}{0.000000,0.000000,0.000000}%
\pgfsetstrokecolor{textcolor}%
\pgfsetfillcolor{textcolor}%
\pgftext[x=4.025875in,y=0.430778in,,top]{\color{textcolor}\sffamily\fontsize{10.000000}{12.000000}\selectfont 20}%
\end{pgfscope}%
\begin{pgfscope}%
\pgfpathrectangle{\pgfqpoint{0.800000in}{0.528000in}}{\pgfqpoint{4.960000in}{3.696000in}}%
\pgfusepath{clip}%
\pgfsetrectcap%
\pgfsetroundjoin%
\pgfsetlinewidth{0.803000pt}%
\definecolor{currentstroke}{rgb}{0.690196,0.690196,0.690196}%
\pgfsetstrokecolor{currentstroke}%
\pgfsetdash{}{0pt}%
\pgfpathmoveto{\pgfqpoint{4.775981in}{0.528000in}}%
\pgfpathlineto{\pgfqpoint{4.775981in}{4.224000in}}%
\pgfusepath{stroke}%
\end{pgfscope}%
\begin{pgfscope}%
\pgfsetbuttcap%
\pgfsetroundjoin%
\definecolor{currentfill}{rgb}{0.000000,0.000000,0.000000}%
\pgfsetfillcolor{currentfill}%
\pgfsetlinewidth{0.803000pt}%
\definecolor{currentstroke}{rgb}{0.000000,0.000000,0.000000}%
\pgfsetstrokecolor{currentstroke}%
\pgfsetdash{}{0pt}%
\pgfsys@defobject{currentmarker}{\pgfqpoint{0.000000in}{-0.048611in}}{\pgfqpoint{0.000000in}{0.000000in}}{%
\pgfpathmoveto{\pgfqpoint{0.000000in}{0.000000in}}%
\pgfpathlineto{\pgfqpoint{0.000000in}{-0.048611in}}%
\pgfusepath{stroke,fill}%
}%
\begin{pgfscope}%
\pgfsys@transformshift{4.775981in}{0.528000in}%
\pgfsys@useobject{currentmarker}{}%
\end{pgfscope}%
\end{pgfscope}%
\begin{pgfscope}%
\definecolor{textcolor}{rgb}{0.000000,0.000000,0.000000}%
\pgfsetstrokecolor{textcolor}%
\pgfsetfillcolor{textcolor}%
\pgftext[x=4.775981in,y=0.430778in,,top]{\color{textcolor}\sffamily\fontsize{10.000000}{12.000000}\selectfont 25}%
\end{pgfscope}%
\begin{pgfscope}%
\pgfpathrectangle{\pgfqpoint{0.800000in}{0.528000in}}{\pgfqpoint{4.960000in}{3.696000in}}%
\pgfusepath{clip}%
\pgfsetrectcap%
\pgfsetroundjoin%
\pgfsetlinewidth{0.803000pt}%
\definecolor{currentstroke}{rgb}{0.690196,0.690196,0.690196}%
\pgfsetstrokecolor{currentstroke}%
\pgfsetdash{}{0pt}%
\pgfpathmoveto{\pgfqpoint{5.526086in}{0.528000in}}%
\pgfpathlineto{\pgfqpoint{5.526086in}{4.224000in}}%
\pgfusepath{stroke}%
\end{pgfscope}%
\begin{pgfscope}%
\pgfsetbuttcap%
\pgfsetroundjoin%
\definecolor{currentfill}{rgb}{0.000000,0.000000,0.000000}%
\pgfsetfillcolor{currentfill}%
\pgfsetlinewidth{0.803000pt}%
\definecolor{currentstroke}{rgb}{0.000000,0.000000,0.000000}%
\pgfsetstrokecolor{currentstroke}%
\pgfsetdash{}{0pt}%
\pgfsys@defobject{currentmarker}{\pgfqpoint{0.000000in}{-0.048611in}}{\pgfqpoint{0.000000in}{0.000000in}}{%
\pgfpathmoveto{\pgfqpoint{0.000000in}{0.000000in}}%
\pgfpathlineto{\pgfqpoint{0.000000in}{-0.048611in}}%
\pgfusepath{stroke,fill}%
}%
\begin{pgfscope}%
\pgfsys@transformshift{5.526086in}{0.528000in}%
\pgfsys@useobject{currentmarker}{}%
\end{pgfscope}%
\end{pgfscope}%
\begin{pgfscope}%
\definecolor{textcolor}{rgb}{0.000000,0.000000,0.000000}%
\pgfsetstrokecolor{textcolor}%
\pgfsetfillcolor{textcolor}%
\pgftext[x=5.526086in,y=0.430778in,,top]{\color{textcolor}\sffamily\fontsize{10.000000}{12.000000}\selectfont 30}%
\end{pgfscope}%
\begin{pgfscope}%
\definecolor{textcolor}{rgb}{0.000000,0.000000,0.000000}%
\pgfsetstrokecolor{textcolor}%
\pgfsetfillcolor{textcolor}%
\pgftext[x=3.280000in,y=0.240809in,,top]{\color{textcolor}\sffamily\fontsize{10.000000}{12.000000}\selectfont time [s]}%
\end{pgfscope}%
\begin{pgfscope}%
\pgfpathrectangle{\pgfqpoint{0.800000in}{0.528000in}}{\pgfqpoint{4.960000in}{3.696000in}}%
\pgfusepath{clip}%
\pgfsetrectcap%
\pgfsetroundjoin%
\pgfsetlinewidth{0.803000pt}%
\definecolor{currentstroke}{rgb}{0.690196,0.690196,0.690196}%
\pgfsetstrokecolor{currentstroke}%
\pgfsetdash{}{0pt}%
\pgfpathmoveto{\pgfqpoint{0.800000in}{0.696000in}}%
\pgfpathlineto{\pgfqpoint{5.760000in}{0.696000in}}%
\pgfusepath{stroke}%
\end{pgfscope}%
\begin{pgfscope}%
\pgfsetbuttcap%
\pgfsetroundjoin%
\definecolor{currentfill}{rgb}{0.000000,0.000000,0.000000}%
\pgfsetfillcolor{currentfill}%
\pgfsetlinewidth{0.803000pt}%
\definecolor{currentstroke}{rgb}{0.000000,0.000000,0.000000}%
\pgfsetstrokecolor{currentstroke}%
\pgfsetdash{}{0pt}%
\pgfsys@defobject{currentmarker}{\pgfqpoint{-0.048611in}{0.000000in}}{\pgfqpoint{-0.000000in}{0.000000in}}{%
\pgfpathmoveto{\pgfqpoint{-0.000000in}{0.000000in}}%
\pgfpathlineto{\pgfqpoint{-0.048611in}{0.000000in}}%
\pgfusepath{stroke,fill}%
}%
\begin{pgfscope}%
\pgfsys@transformshift{0.800000in}{0.696000in}%
\pgfsys@useobject{currentmarker}{}%
\end{pgfscope}%
\end{pgfscope}%
\begin{pgfscope}%
\definecolor{textcolor}{rgb}{0.000000,0.000000,0.000000}%
\pgfsetstrokecolor{textcolor}%
\pgfsetfillcolor{textcolor}%
\pgftext[x=0.481898in, y=0.643238in, left, base]{\color{textcolor}\sffamily\fontsize{10.000000}{12.000000}\selectfont 0.0}%
\end{pgfscope}%
\begin{pgfscope}%
\pgfpathrectangle{\pgfqpoint{0.800000in}{0.528000in}}{\pgfqpoint{4.960000in}{3.696000in}}%
\pgfusepath{clip}%
\pgfsetrectcap%
\pgfsetroundjoin%
\pgfsetlinewidth{0.803000pt}%
\definecolor{currentstroke}{rgb}{0.690196,0.690196,0.690196}%
\pgfsetstrokecolor{currentstroke}%
\pgfsetdash{}{0pt}%
\pgfpathmoveto{\pgfqpoint{0.800000in}{1.438100in}}%
\pgfpathlineto{\pgfqpoint{5.760000in}{1.438100in}}%
\pgfusepath{stroke}%
\end{pgfscope}%
\begin{pgfscope}%
\pgfsetbuttcap%
\pgfsetroundjoin%
\definecolor{currentfill}{rgb}{0.000000,0.000000,0.000000}%
\pgfsetfillcolor{currentfill}%
\pgfsetlinewidth{0.803000pt}%
\definecolor{currentstroke}{rgb}{0.000000,0.000000,0.000000}%
\pgfsetstrokecolor{currentstroke}%
\pgfsetdash{}{0pt}%
\pgfsys@defobject{currentmarker}{\pgfqpoint{-0.048611in}{0.000000in}}{\pgfqpoint{-0.000000in}{0.000000in}}{%
\pgfpathmoveto{\pgfqpoint{-0.000000in}{0.000000in}}%
\pgfpathlineto{\pgfqpoint{-0.048611in}{0.000000in}}%
\pgfusepath{stroke,fill}%
}%
\begin{pgfscope}%
\pgfsys@transformshift{0.800000in}{1.438100in}%
\pgfsys@useobject{currentmarker}{}%
\end{pgfscope}%
\end{pgfscope}%
\begin{pgfscope}%
\definecolor{textcolor}{rgb}{0.000000,0.000000,0.000000}%
\pgfsetstrokecolor{textcolor}%
\pgfsetfillcolor{textcolor}%
\pgftext[x=0.481898in, y=1.385338in, left, base]{\color{textcolor}\sffamily\fontsize{10.000000}{12.000000}\selectfont 0.1}%
\end{pgfscope}%
\begin{pgfscope}%
\pgfpathrectangle{\pgfqpoint{0.800000in}{0.528000in}}{\pgfqpoint{4.960000in}{3.696000in}}%
\pgfusepath{clip}%
\pgfsetrectcap%
\pgfsetroundjoin%
\pgfsetlinewidth{0.803000pt}%
\definecolor{currentstroke}{rgb}{0.690196,0.690196,0.690196}%
\pgfsetstrokecolor{currentstroke}%
\pgfsetdash{}{0pt}%
\pgfpathmoveto{\pgfqpoint{0.800000in}{2.180200in}}%
\pgfpathlineto{\pgfqpoint{5.760000in}{2.180200in}}%
\pgfusepath{stroke}%
\end{pgfscope}%
\begin{pgfscope}%
\pgfsetbuttcap%
\pgfsetroundjoin%
\definecolor{currentfill}{rgb}{0.000000,0.000000,0.000000}%
\pgfsetfillcolor{currentfill}%
\pgfsetlinewidth{0.803000pt}%
\definecolor{currentstroke}{rgb}{0.000000,0.000000,0.000000}%
\pgfsetstrokecolor{currentstroke}%
\pgfsetdash{}{0pt}%
\pgfsys@defobject{currentmarker}{\pgfqpoint{-0.048611in}{0.000000in}}{\pgfqpoint{-0.000000in}{0.000000in}}{%
\pgfpathmoveto{\pgfqpoint{-0.000000in}{0.000000in}}%
\pgfpathlineto{\pgfqpoint{-0.048611in}{0.000000in}}%
\pgfusepath{stroke,fill}%
}%
\begin{pgfscope}%
\pgfsys@transformshift{0.800000in}{2.180200in}%
\pgfsys@useobject{currentmarker}{}%
\end{pgfscope}%
\end{pgfscope}%
\begin{pgfscope}%
\definecolor{textcolor}{rgb}{0.000000,0.000000,0.000000}%
\pgfsetstrokecolor{textcolor}%
\pgfsetfillcolor{textcolor}%
\pgftext[x=0.481898in, y=2.127438in, left, base]{\color{textcolor}\sffamily\fontsize{10.000000}{12.000000}\selectfont 0.2}%
\end{pgfscope}%
\begin{pgfscope}%
\pgfpathrectangle{\pgfqpoint{0.800000in}{0.528000in}}{\pgfqpoint{4.960000in}{3.696000in}}%
\pgfusepath{clip}%
\pgfsetrectcap%
\pgfsetroundjoin%
\pgfsetlinewidth{0.803000pt}%
\definecolor{currentstroke}{rgb}{0.690196,0.690196,0.690196}%
\pgfsetstrokecolor{currentstroke}%
\pgfsetdash{}{0pt}%
\pgfpathmoveto{\pgfqpoint{0.800000in}{2.922300in}}%
\pgfpathlineto{\pgfqpoint{5.760000in}{2.922300in}}%
\pgfusepath{stroke}%
\end{pgfscope}%
\begin{pgfscope}%
\pgfsetbuttcap%
\pgfsetroundjoin%
\definecolor{currentfill}{rgb}{0.000000,0.000000,0.000000}%
\pgfsetfillcolor{currentfill}%
\pgfsetlinewidth{0.803000pt}%
\definecolor{currentstroke}{rgb}{0.000000,0.000000,0.000000}%
\pgfsetstrokecolor{currentstroke}%
\pgfsetdash{}{0pt}%
\pgfsys@defobject{currentmarker}{\pgfqpoint{-0.048611in}{0.000000in}}{\pgfqpoint{-0.000000in}{0.000000in}}{%
\pgfpathmoveto{\pgfqpoint{-0.000000in}{0.000000in}}%
\pgfpathlineto{\pgfqpoint{-0.048611in}{0.000000in}}%
\pgfusepath{stroke,fill}%
}%
\begin{pgfscope}%
\pgfsys@transformshift{0.800000in}{2.922300in}%
\pgfsys@useobject{currentmarker}{}%
\end{pgfscope}%
\end{pgfscope}%
\begin{pgfscope}%
\definecolor{textcolor}{rgb}{0.000000,0.000000,0.000000}%
\pgfsetstrokecolor{textcolor}%
\pgfsetfillcolor{textcolor}%
\pgftext[x=0.481898in, y=2.869538in, left, base]{\color{textcolor}\sffamily\fontsize{10.000000}{12.000000}\selectfont 0.3}%
\end{pgfscope}%
\begin{pgfscope}%
\pgfpathrectangle{\pgfqpoint{0.800000in}{0.528000in}}{\pgfqpoint{4.960000in}{3.696000in}}%
\pgfusepath{clip}%
\pgfsetrectcap%
\pgfsetroundjoin%
\pgfsetlinewidth{0.803000pt}%
\definecolor{currentstroke}{rgb}{0.690196,0.690196,0.690196}%
\pgfsetstrokecolor{currentstroke}%
\pgfsetdash{}{0pt}%
\pgfpathmoveto{\pgfqpoint{0.800000in}{3.664400in}}%
\pgfpathlineto{\pgfqpoint{5.760000in}{3.664400in}}%
\pgfusepath{stroke}%
\end{pgfscope}%
\begin{pgfscope}%
\pgfsetbuttcap%
\pgfsetroundjoin%
\definecolor{currentfill}{rgb}{0.000000,0.000000,0.000000}%
\pgfsetfillcolor{currentfill}%
\pgfsetlinewidth{0.803000pt}%
\definecolor{currentstroke}{rgb}{0.000000,0.000000,0.000000}%
\pgfsetstrokecolor{currentstroke}%
\pgfsetdash{}{0pt}%
\pgfsys@defobject{currentmarker}{\pgfqpoint{-0.048611in}{0.000000in}}{\pgfqpoint{-0.000000in}{0.000000in}}{%
\pgfpathmoveto{\pgfqpoint{-0.000000in}{0.000000in}}%
\pgfpathlineto{\pgfqpoint{-0.048611in}{0.000000in}}%
\pgfusepath{stroke,fill}%
}%
\begin{pgfscope}%
\pgfsys@transformshift{0.800000in}{3.664400in}%
\pgfsys@useobject{currentmarker}{}%
\end{pgfscope}%
\end{pgfscope}%
\begin{pgfscope}%
\definecolor{textcolor}{rgb}{0.000000,0.000000,0.000000}%
\pgfsetstrokecolor{textcolor}%
\pgfsetfillcolor{textcolor}%
\pgftext[x=0.481898in, y=3.611638in, left, base]{\color{textcolor}\sffamily\fontsize{10.000000}{12.000000}\selectfont 0.4}%
\end{pgfscope}%
\begin{pgfscope}%
\definecolor{textcolor}{rgb}{0.000000,0.000000,0.000000}%
\pgfsetstrokecolor{textcolor}%
\pgfsetfillcolor{textcolor}%
\pgftext[x=0.426343in,y=2.376000in,,bottom,rotate=90.000000]{\color{textcolor}\sffamily\fontsize{10.000000}{12.000000}\selectfont Velocity [m/s]}%
\end{pgfscope}%
\begin{pgfscope}%
\pgfpathrectangle{\pgfqpoint{0.800000in}{0.528000in}}{\pgfqpoint{4.960000in}{3.696000in}}%
\pgfusepath{clip}%
\pgfsetrectcap%
\pgfsetroundjoin%
\pgfsetlinewidth{1.505625pt}%
\definecolor{currentstroke}{rgb}{0.121569,0.466667,0.705882}%
\pgfsetstrokecolor{currentstroke}%
\pgfsetdash{}{0pt}%
\pgfpathmoveto{\pgfqpoint{1.025455in}{0.696000in}}%
\pgfpathlineto{\pgfqpoint{1.079621in}{1.737587in}}%
\pgfpathlineto{\pgfqpoint{1.134067in}{2.638261in}}%
\pgfpathlineto{\pgfqpoint{1.188105in}{3.228942in}}%
\pgfpathlineto{\pgfqpoint{1.242094in}{3.531560in}}%
\pgfpathlineto{\pgfqpoint{1.296178in}{3.826923in}}%
\pgfpathlineto{\pgfqpoint{1.350915in}{3.974704in}}%
\pgfpathlineto{\pgfqpoint{1.403595in}{4.056000in}}%
\pgfpathlineto{\pgfqpoint{1.457824in}{4.056000in}}%
\pgfpathlineto{\pgfqpoint{1.512339in}{4.056000in}}%
\pgfpathlineto{\pgfqpoint{1.567679in}{3.991458in}}%
\pgfpathlineto{\pgfqpoint{1.621568in}{3.834828in}}%
\pgfpathlineto{\pgfqpoint{1.674970in}{3.540286in}}%
\pgfpathlineto{\pgfqpoint{1.729154in}{3.099533in}}%
\pgfpathlineto{\pgfqpoint{1.783989in}{2.721538in}}%
\pgfpathlineto{\pgfqpoint{1.837894in}{2.209593in}}%
\pgfpathlineto{\pgfqpoint{1.891785in}{1.686085in}}%
\pgfpathlineto{\pgfqpoint{1.946107in}{1.165345in}}%
\pgfpathlineto{\pgfqpoint{2.000204in}{0.800949in}}%
\pgfpathlineto{\pgfqpoint{2.054649in}{0.844420in}}%
\pgfpathlineto{\pgfqpoint{2.108960in}{1.067050in}}%
\pgfpathlineto{\pgfqpoint{2.164485in}{1.215470in}}%
\pgfpathlineto{\pgfqpoint{2.217989in}{1.363890in}}%
\pgfpathlineto{\pgfqpoint{2.271947in}{1.589607in}}%
\pgfpathlineto{\pgfqpoint{2.325870in}{1.811621in}}%
\pgfpathlineto{\pgfqpoint{2.380043in}{1.966270in}}%
\pgfpathlineto{\pgfqpoint{2.434222in}{2.040000in}}%
\pgfpathlineto{\pgfqpoint{2.488415in}{1.966270in}}%
\pgfpathlineto{\pgfqpoint{2.543838in}{1.819001in}}%
\pgfpathlineto{\pgfqpoint{2.597142in}{1.598803in}}%
\pgfpathlineto{\pgfqpoint{2.651160in}{1.452796in}}%
\pgfpathlineto{\pgfqpoint{2.707046in}{1.307951in}}%
\pgfpathlineto{\pgfqpoint{2.760613in}{1.236257in}}%
\pgfpathlineto{\pgfqpoint{2.814827in}{1.236257in}}%
\pgfpathlineto{\pgfqpoint{2.869156in}{1.147402in}}%
\pgfpathlineto{\pgfqpoint{2.923530in}{1.147402in}}%
\pgfpathlineto{\pgfqpoint{2.977612in}{1.220744in}}%
\pgfpathlineto{\pgfqpoint{3.032110in}{1.294300in}}%
\pgfpathlineto{\pgfqpoint{3.086546in}{1.368000in}}%
\pgfpathlineto{\pgfqpoint{3.141581in}{1.368000in}}%
\pgfpathlineto{\pgfqpoint{3.195204in}{1.368000in}}%
\pgfpathlineto{\pgfqpoint{3.250008in}{1.368000in}}%
\pgfpathlineto{\pgfqpoint{3.304642in}{1.363890in}}%
\pgfpathlineto{\pgfqpoint{3.359094in}{1.363890in}}%
\pgfpathlineto{\pgfqpoint{3.414749in}{1.067050in}}%
\pgfpathlineto{\pgfqpoint{3.466566in}{0.844420in}}%
\pgfpathlineto{\pgfqpoint{3.522217in}{0.696000in}}%
\pgfpathlineto{\pgfqpoint{3.577031in}{0.770210in}}%
\pgfpathlineto{\pgfqpoint{3.632084in}{0.770210in}}%
\pgfpathlineto{\pgfqpoint{3.689284in}{0.770210in}}%
\pgfpathlineto{\pgfqpoint{3.741898in}{0.844420in}}%
\pgfpathlineto{\pgfqpoint{3.795342in}{0.844420in}}%
\pgfpathlineto{\pgfqpoint{3.849219in}{0.844420in}}%
\pgfpathlineto{\pgfqpoint{3.902878in}{0.696000in}}%
\pgfpathlineto{\pgfqpoint{3.957147in}{0.770210in}}%
\pgfpathlineto{\pgfqpoint{4.011641in}{0.844420in}}%
\pgfpathlineto{\pgfqpoint{4.065985in}{0.918630in}}%
\pgfpathlineto{\pgfqpoint{4.119943in}{0.992840in}}%
\pgfpathlineto{\pgfqpoint{4.174014in}{0.992840in}}%
\pgfpathlineto{\pgfqpoint{4.228240in}{0.992840in}}%
\pgfpathlineto{\pgfqpoint{4.282506in}{0.696000in}}%
\pgfpathlineto{\pgfqpoint{4.336707in}{0.844420in}}%
\pgfpathlineto{\pgfqpoint{4.391324in}{0.918630in}}%
\pgfpathlineto{\pgfqpoint{4.445418in}{0.992840in}}%
\pgfpathlineto{\pgfqpoint{4.499016in}{1.067050in}}%
\pgfpathlineto{\pgfqpoint{4.553792in}{1.141260in}}%
\pgfpathlineto{\pgfqpoint{4.608095in}{1.215470in}}%
\pgfpathlineto{\pgfqpoint{4.663392in}{0.992840in}}%
\pgfpathlineto{\pgfqpoint{4.717249in}{0.918630in}}%
\pgfpathlineto{\pgfqpoint{4.770790in}{0.918630in}}%
\pgfpathlineto{\pgfqpoint{4.824973in}{0.918630in}}%
\pgfpathlineto{\pgfqpoint{4.879363in}{0.844420in}}%
\pgfpathlineto{\pgfqpoint{4.933609in}{0.770210in}}%
\pgfpathlineto{\pgfqpoint{4.987553in}{0.696000in}}%
\pgfpathlineto{\pgfqpoint{5.041976in}{0.696000in}}%
\pgfpathlineto{\pgfqpoint{5.095720in}{0.696000in}}%
\pgfpathlineto{\pgfqpoint{5.150542in}{0.696000in}}%
\pgfpathlineto{\pgfqpoint{5.204867in}{0.696000in}}%
\pgfpathlineto{\pgfqpoint{5.260652in}{0.696000in}}%
\pgfpathlineto{\pgfqpoint{5.313598in}{0.696000in}}%
\pgfpathlineto{\pgfqpoint{5.367731in}{0.770210in}}%
\pgfpathlineto{\pgfqpoint{5.421938in}{0.770210in}}%
\pgfpathlineto{\pgfqpoint{5.475898in}{0.844420in}}%
\pgfpathlineto{\pgfqpoint{5.530179in}{0.844420in}}%
\pgfusepath{stroke}%
\end{pgfscope}%
\begin{pgfscope}%
\pgfpathrectangle{\pgfqpoint{0.800000in}{0.528000in}}{\pgfqpoint{4.960000in}{3.696000in}}%
\pgfusepath{clip}%
\pgfsetrectcap%
\pgfsetroundjoin%
\pgfsetlinewidth{1.505625pt}%
\definecolor{currentstroke}{rgb}{1.000000,0.498039,0.054902}%
\pgfsetstrokecolor{currentstroke}%
\pgfsetdash{}{0pt}%
\pgfpathmoveto{\pgfqpoint{1.025455in}{0.696000in}}%
\pgfpathlineto{\pgfqpoint{1.078219in}{1.663580in}}%
\pgfpathlineto{\pgfqpoint{1.132936in}{2.638261in}}%
\pgfpathlineto{\pgfqpoint{1.187392in}{3.155028in}}%
\pgfpathlineto{\pgfqpoint{1.242205in}{3.531560in}}%
\pgfpathlineto{\pgfqpoint{1.296628in}{3.753055in}}%
\pgfpathlineto{\pgfqpoint{1.350891in}{3.900806in}}%
\pgfpathlineto{\pgfqpoint{1.403861in}{3.450780in}}%
\pgfpathlineto{\pgfqpoint{1.459530in}{2.483213in}}%
\pgfpathlineto{\pgfqpoint{1.513111in}{1.737587in}}%
\pgfpathlineto{\pgfqpoint{1.567568in}{0.861939in}}%
\pgfpathlineto{\pgfqpoint{1.621564in}{0.861939in}}%
\pgfpathlineto{\pgfqpoint{1.675632in}{0.844420in}}%
\pgfpathlineto{\pgfqpoint{1.729840in}{0.844420in}}%
\pgfpathlineto{\pgfqpoint{1.784194in}{0.800949in}}%
\pgfpathlineto{\pgfqpoint{1.838904in}{0.918630in}}%
\pgfpathlineto{\pgfqpoint{1.893050in}{1.067050in}}%
\pgfpathlineto{\pgfqpoint{1.947643in}{0.992840in}}%
\pgfpathlineto{\pgfqpoint{2.003377in}{1.067050in}}%
\pgfpathlineto{\pgfqpoint{2.056559in}{0.992840in}}%
\pgfpathlineto{\pgfqpoint{2.110632in}{1.001976in}}%
\pgfpathlineto{\pgfqpoint{2.164642in}{0.930673in}}%
\pgfpathlineto{\pgfqpoint{2.218548in}{0.770210in}}%
\pgfpathlineto{\pgfqpoint{2.272691in}{0.844420in}}%
\pgfpathlineto{\pgfqpoint{2.327036in}{0.992840in}}%
\pgfpathlineto{\pgfqpoint{2.381198in}{1.067050in}}%
\pgfpathlineto{\pgfqpoint{2.435498in}{1.141260in}}%
\pgfpathlineto{\pgfqpoint{2.489813in}{1.067050in}}%
\pgfpathlineto{\pgfqpoint{2.544985in}{1.067050in}}%
\pgfpathlineto{\pgfqpoint{2.600407in}{0.918630in}}%
\pgfpathlineto{\pgfqpoint{2.653635in}{0.844420in}}%
\pgfpathlineto{\pgfqpoint{2.707681in}{0.844420in}}%
\pgfpathlineto{\pgfqpoint{2.761906in}{0.844420in}}%
\pgfpathlineto{\pgfqpoint{2.816153in}{0.770210in}}%
\pgfpathlineto{\pgfqpoint{2.870336in}{0.696000in}}%
\pgfpathlineto{\pgfqpoint{2.924320in}{0.770210in}}%
\pgfpathlineto{\pgfqpoint{2.978679in}{0.844420in}}%
\pgfpathlineto{\pgfqpoint{3.033237in}{0.844420in}}%
\pgfpathlineto{\pgfqpoint{3.087250in}{0.844420in}}%
\pgfpathlineto{\pgfqpoint{3.142788in}{0.844420in}}%
\pgfpathlineto{\pgfqpoint{3.196597in}{0.696000in}}%
\pgfpathlineto{\pgfqpoint{3.250471in}{0.770210in}}%
\pgfpathlineto{\pgfqpoint{3.304787in}{0.770210in}}%
\pgfpathlineto{\pgfqpoint{3.359088in}{0.770210in}}%
\pgfpathlineto{\pgfqpoint{3.412967in}{0.770210in}}%
\pgfpathlineto{\pgfqpoint{3.467023in}{0.696000in}}%
\pgfpathlineto{\pgfqpoint{3.521323in}{0.696000in}}%
\pgfpathlineto{\pgfqpoint{3.575366in}{0.696000in}}%
\pgfpathlineto{\pgfqpoint{3.630264in}{0.770210in}}%
\pgfpathlineto{\pgfqpoint{3.686099in}{0.770210in}}%
\pgfpathlineto{\pgfqpoint{3.739827in}{0.770210in}}%
\pgfpathlineto{\pgfqpoint{3.794152in}{0.770210in}}%
\pgfpathlineto{\pgfqpoint{3.848946in}{0.770210in}}%
\pgfpathlineto{\pgfqpoint{3.903098in}{0.770210in}}%
\pgfpathlineto{\pgfqpoint{3.957325in}{0.696000in}}%
\pgfpathlineto{\pgfqpoint{4.011195in}{0.696000in}}%
\pgfpathlineto{\pgfqpoint{4.065648in}{0.696000in}}%
\pgfpathlineto{\pgfqpoint{4.120089in}{0.696000in}}%
\pgfpathlineto{\pgfqpoint{4.174735in}{0.696000in}}%
\pgfpathlineto{\pgfqpoint{4.230250in}{0.696000in}}%
\pgfpathlineto{\pgfqpoint{4.283857in}{0.696000in}}%
\pgfpathlineto{\pgfqpoint{4.337616in}{0.696000in}}%
\pgfpathlineto{\pgfqpoint{4.391557in}{0.770210in}}%
\pgfpathlineto{\pgfqpoint{4.445715in}{0.770210in}}%
\pgfpathlineto{\pgfqpoint{4.500165in}{0.770210in}}%
\pgfpathlineto{\pgfqpoint{4.554023in}{0.844420in}}%
\pgfpathlineto{\pgfqpoint{4.608007in}{0.770210in}}%
\pgfpathlineto{\pgfqpoint{4.662517in}{0.770210in}}%
\pgfpathlineto{\pgfqpoint{4.716704in}{0.696000in}}%
\pgfpathlineto{\pgfqpoint{4.770918in}{0.696000in}}%
\pgfpathlineto{\pgfqpoint{4.826939in}{0.696000in}}%
\pgfpathlineto{\pgfqpoint{4.880405in}{0.696000in}}%
\pgfpathlineto{\pgfqpoint{4.933976in}{0.696000in}}%
\pgfpathlineto{\pgfqpoint{4.987795in}{0.696000in}}%
\pgfpathlineto{\pgfqpoint{5.042006in}{0.696000in}}%
\pgfpathlineto{\pgfqpoint{5.096184in}{0.696000in}}%
\pgfpathlineto{\pgfqpoint{5.150144in}{0.696000in}}%
\pgfpathlineto{\pgfqpoint{5.204225in}{0.696000in}}%
\pgfpathlineto{\pgfqpoint{5.258772in}{0.696000in}}%
\pgfpathlineto{\pgfqpoint{5.312795in}{0.770210in}}%
\pgfpathlineto{\pgfqpoint{5.367331in}{0.770210in}}%
\pgfpathlineto{\pgfqpoint{5.421537in}{0.770210in}}%
\pgfpathlineto{\pgfqpoint{5.477599in}{0.696000in}}%
\pgfpathlineto{\pgfqpoint{5.530378in}{0.770210in}}%
\pgfusepath{stroke}%
\end{pgfscope}%
\begin{pgfscope}%
\pgfpathrectangle{\pgfqpoint{0.800000in}{0.528000in}}{\pgfqpoint{4.960000in}{3.696000in}}%
\pgfusepath{clip}%
\pgfsetrectcap%
\pgfsetroundjoin%
\pgfsetlinewidth{1.505625pt}%
\definecolor{currentstroke}{rgb}{0.172549,0.627451,0.172549}%
\pgfsetstrokecolor{currentstroke}%
\pgfsetdash{}{0pt}%
\pgfpathmoveto{\pgfqpoint{1.025455in}{0.696000in}}%
\pgfpathlineto{\pgfqpoint{1.079547in}{1.745488in}}%
\pgfpathlineto{\pgfqpoint{1.133794in}{2.564560in}}%
\pgfpathlineto{\pgfqpoint{1.187710in}{3.162854in}}%
\pgfpathlineto{\pgfqpoint{1.241823in}{3.531560in}}%
\pgfpathlineto{\pgfqpoint{1.296054in}{3.753055in}}%
\pgfpathlineto{\pgfqpoint{1.350774in}{3.753055in}}%
\pgfpathlineto{\pgfqpoint{1.404251in}{3.531560in}}%
\pgfpathlineto{\pgfqpoint{1.458677in}{3.236541in}}%
\pgfpathlineto{\pgfqpoint{1.512039in}{3.310257in}}%
\pgfpathlineto{\pgfqpoint{1.565969in}{3.089200in}}%
\pgfpathlineto{\pgfqpoint{1.620412in}{2.648160in}}%
\pgfpathlineto{\pgfqpoint{1.674518in}{2.282428in}}%
\pgfpathlineto{\pgfqpoint{1.729185in}{2.209593in}}%
\pgfpathlineto{\pgfqpoint{1.783493in}{2.050205in}}%
\pgfpathlineto{\pgfqpoint{1.838449in}{1.672080in}}%
\pgfpathlineto{\pgfqpoint{1.892271in}{1.380182in}}%
\pgfpathlineto{\pgfqpoint{1.948822in}{1.307951in}}%
\pgfpathlineto{\pgfqpoint{2.001892in}{1.147402in}}%
\pgfpathlineto{\pgfqpoint{2.056770in}{1.001976in}}%
\pgfpathlineto{\pgfqpoint{2.109781in}{0.770210in}}%
\pgfpathlineto{\pgfqpoint{2.163546in}{0.696000in}}%
\pgfpathlineto{\pgfqpoint{2.217499in}{0.696000in}}%
\pgfpathlineto{\pgfqpoint{2.271750in}{0.770210in}}%
\pgfpathlineto{\pgfqpoint{2.326221in}{0.844420in}}%
\pgfpathlineto{\pgfqpoint{2.380311in}{0.992840in}}%
\pgfpathlineto{\pgfqpoint{2.434635in}{1.074398in}}%
\pgfpathlineto{\pgfqpoint{2.488716in}{1.147402in}}%
\pgfpathlineto{\pgfqpoint{2.544057in}{1.294300in}}%
\pgfpathlineto{\pgfqpoint{2.597737in}{1.368000in}}%
\pgfpathlineto{\pgfqpoint{2.651697in}{1.441801in}}%
\pgfpathlineto{\pgfqpoint{2.706178in}{1.368000in}}%
\pgfpathlineto{\pgfqpoint{2.760045in}{1.220744in}}%
\pgfpathlineto{\pgfqpoint{2.814157in}{1.452796in}}%
\pgfpathlineto{\pgfqpoint{2.868425in}{1.307951in}}%
\pgfpathlineto{\pgfqpoint{2.923075in}{1.525693in}}%
\pgfpathlineto{\pgfqpoint{2.976936in}{1.441801in}}%
\pgfpathlineto{\pgfqpoint{3.031307in}{1.441801in}}%
\pgfpathlineto{\pgfqpoint{3.085412in}{1.515676in}}%
\pgfpathlineto{\pgfqpoint{3.140703in}{1.589607in}}%
\pgfpathlineto{\pgfqpoint{3.194425in}{1.672080in}}%
\pgfpathlineto{\pgfqpoint{3.249562in}{1.663580in}}%
\pgfpathlineto{\pgfqpoint{3.304606in}{1.737587in}}%
\pgfpathlineto{\pgfqpoint{3.357992in}{1.811621in}}%
\pgfpathlineto{\pgfqpoint{3.413253in}{1.215470in}}%
\pgfpathlineto{\pgfqpoint{3.468038in}{1.067050in}}%
\pgfpathlineto{\pgfqpoint{3.522678in}{1.147402in}}%
\pgfpathlineto{\pgfqpoint{3.575616in}{1.147402in}}%
\pgfpathlineto{\pgfqpoint{3.629520in}{0.918630in}}%
\pgfpathlineto{\pgfqpoint{3.684969in}{0.696000in}}%
\pgfpathlineto{\pgfqpoint{3.737150in}{0.770210in}}%
\pgfpathlineto{\pgfqpoint{3.791816in}{0.844420in}}%
\pgfpathlineto{\pgfqpoint{3.845655in}{0.696000in}}%
\pgfpathlineto{\pgfqpoint{3.899891in}{0.918630in}}%
\pgfpathlineto{\pgfqpoint{3.953880in}{0.918630in}}%
\pgfpathlineto{\pgfqpoint{4.008212in}{0.918630in}}%
\pgfpathlineto{\pgfqpoint{4.062359in}{0.918630in}}%
\pgfpathlineto{\pgfqpoint{4.118065in}{0.918630in}}%
\pgfpathlineto{\pgfqpoint{4.171607in}{0.918630in}}%
\pgfpathlineto{\pgfqpoint{4.225472in}{0.918630in}}%
\pgfpathlineto{\pgfqpoint{4.279618in}{0.918630in}}%
\pgfpathlineto{\pgfqpoint{4.333765in}{1.001976in}}%
\pgfpathlineto{\pgfqpoint{4.387769in}{1.001976in}}%
\pgfpathlineto{\pgfqpoint{4.442344in}{1.001976in}}%
\pgfpathlineto{\pgfqpoint{4.496438in}{0.844420in}}%
\pgfpathlineto{\pgfqpoint{4.550833in}{1.141260in}}%
\pgfpathlineto{\pgfqpoint{4.605202in}{0.770210in}}%
\pgfpathlineto{\pgfqpoint{4.659356in}{1.067050in}}%
\pgfpathlineto{\pgfqpoint{4.713153in}{1.363890in}}%
\pgfpathlineto{\pgfqpoint{4.768252in}{1.289680in}}%
\pgfpathlineto{\pgfqpoint{4.822749in}{1.141260in}}%
\pgfpathlineto{\pgfqpoint{4.876558in}{0.992840in}}%
\pgfpathlineto{\pgfqpoint{4.931230in}{0.844420in}}%
\pgfpathlineto{\pgfqpoint{4.985009in}{0.696000in}}%
\pgfpathlineto{\pgfqpoint{5.039275in}{0.696000in}}%
\pgfpathlineto{\pgfqpoint{5.093785in}{0.696000in}}%
\pgfpathlineto{\pgfqpoint{5.147994in}{0.844420in}}%
\pgfpathlineto{\pgfqpoint{5.202131in}{1.067050in}}%
\pgfpathlineto{\pgfqpoint{5.257035in}{1.141260in}}%
\pgfpathlineto{\pgfqpoint{5.311110in}{1.141260in}}%
\pgfpathlineto{\pgfqpoint{5.367126in}{1.067050in}}%
\pgfpathlineto{\pgfqpoint{5.420063in}{1.141260in}}%
\pgfpathlineto{\pgfqpoint{5.473871in}{1.141260in}}%
\pgfpathlineto{\pgfqpoint{5.527798in}{1.141260in}}%
\pgfusepath{stroke}%
\end{pgfscope}%
\begin{pgfscope}%
\pgfpathrectangle{\pgfqpoint{0.800000in}{0.528000in}}{\pgfqpoint{4.960000in}{3.696000in}}%
\pgfusepath{clip}%
\pgfsetrectcap%
\pgfsetroundjoin%
\pgfsetlinewidth{1.505625pt}%
\definecolor{currentstroke}{rgb}{0.839216,0.152941,0.156863}%
\pgfsetstrokecolor{currentstroke}%
\pgfsetdash{}{0pt}%
\pgfpathmoveto{\pgfqpoint{1.025455in}{0.696000in}}%
\pgfpathlineto{\pgfqpoint{1.079566in}{1.745488in}}%
\pgfpathlineto{\pgfqpoint{1.133651in}{2.638261in}}%
\pgfpathlineto{\pgfqpoint{1.189374in}{3.162854in}}%
\pgfpathlineto{\pgfqpoint{1.242743in}{3.531560in}}%
\pgfpathlineto{\pgfqpoint{1.296714in}{3.753055in}}%
\pgfpathlineto{\pgfqpoint{1.349583in}{3.908530in}}%
\pgfpathlineto{\pgfqpoint{1.403811in}{3.679204in}}%
\pgfpathlineto{\pgfqpoint{1.457962in}{2.933403in}}%
\pgfpathlineto{\pgfqpoint{1.512440in}{1.589607in}}%
\pgfpathlineto{\pgfqpoint{1.567542in}{2.123458in}}%
\pgfpathlineto{\pgfqpoint{1.620345in}{2.806749in}}%
\pgfpathlineto{\pgfqpoint{1.674588in}{2.733736in}}%
\pgfpathlineto{\pgfqpoint{1.728974in}{2.196804in}}%
\pgfpathlineto{\pgfqpoint{1.785298in}{2.050205in}}%
\pgfpathlineto{\pgfqpoint{1.839487in}{1.525693in}}%
\pgfpathlineto{\pgfqpoint{1.893584in}{1.441801in}}%
\pgfpathlineto{\pgfqpoint{1.948544in}{1.441801in}}%
\pgfpathlineto{\pgfqpoint{2.002700in}{1.074398in}}%
\pgfpathlineto{\pgfqpoint{2.057645in}{0.770210in}}%
\pgfpathlineto{\pgfqpoint{2.112255in}{0.844420in}}%
\pgfpathlineto{\pgfqpoint{2.166504in}{0.770210in}}%
\pgfpathlineto{\pgfqpoint{2.220837in}{1.294300in}}%
\pgfpathlineto{\pgfqpoint{2.275243in}{1.663580in}}%
\pgfpathlineto{\pgfqpoint{2.329860in}{1.737587in}}%
\pgfpathlineto{\pgfqpoint{2.385577in}{2.187602in}}%
\pgfpathlineto{\pgfqpoint{2.438682in}{2.187602in}}%
\pgfpathlineto{\pgfqpoint{2.492730in}{1.819001in}}%
\pgfpathlineto{\pgfqpoint{2.546735in}{1.819001in}}%
\pgfpathlineto{\pgfqpoint{2.600886in}{2.050205in}}%
\pgfpathlineto{\pgfqpoint{2.655099in}{2.196804in}}%
\pgfpathlineto{\pgfqpoint{2.709182in}{2.270232in}}%
\pgfpathlineto{\pgfqpoint{2.763396in}{1.598803in}}%
\pgfpathlineto{\pgfqpoint{2.817692in}{0.992840in}}%
\pgfpathlineto{\pgfqpoint{2.872026in}{1.512310in}}%
\pgfpathlineto{\pgfqpoint{2.926537in}{0.905898in}}%
\pgfpathlineto{\pgfqpoint{2.980773in}{1.307951in}}%
\pgfpathlineto{\pgfqpoint{3.036168in}{1.307951in}}%
\pgfpathlineto{\pgfqpoint{3.090615in}{1.074398in}}%
\pgfpathlineto{\pgfqpoint{3.145054in}{1.294300in}}%
\pgfpathlineto{\pgfqpoint{3.198523in}{1.141260in}}%
\pgfpathlineto{\pgfqpoint{3.252893in}{0.770210in}}%
\pgfpathlineto{\pgfqpoint{3.306756in}{1.220744in}}%
\pgfpathlineto{\pgfqpoint{3.360896in}{1.368000in}}%
\pgfpathlineto{\pgfqpoint{3.415278in}{0.770210in}}%
\pgfpathlineto{\pgfqpoint{3.473519in}{0.696000in}}%
\pgfpathlineto{\pgfqpoint{3.524205in}{0.844420in}}%
\pgfpathlineto{\pgfqpoint{3.579253in}{0.770210in}}%
\pgfpathlineto{\pgfqpoint{3.632797in}{0.696000in}}%
\pgfpathlineto{\pgfqpoint{3.686704in}{1.220744in}}%
\pgfpathlineto{\pgfqpoint{3.740830in}{1.001976in}}%
\pgfpathlineto{\pgfqpoint{3.794927in}{0.992840in}}%
\pgfpathlineto{\pgfqpoint{3.849035in}{0.770210in}}%
\pgfpathlineto{\pgfqpoint{3.903240in}{0.696000in}}%
\pgfpathlineto{\pgfqpoint{3.957832in}{0.696000in}}%
\pgfpathlineto{\pgfqpoint{4.012036in}{0.696000in}}%
\pgfpathlineto{\pgfqpoint{4.066126in}{0.844420in}}%
\pgfpathlineto{\pgfqpoint{4.122806in}{0.770210in}}%
\pgfpathlineto{\pgfqpoint{4.176011in}{0.696000in}}%
\pgfpathlineto{\pgfqpoint{4.229495in}{0.696000in}}%
\pgfpathlineto{\pgfqpoint{4.284122in}{0.696000in}}%
\pgfpathlineto{\pgfqpoint{4.338318in}{0.696000in}}%
\pgfpathlineto{\pgfqpoint{4.392933in}{0.696000in}}%
\pgfpathlineto{\pgfqpoint{4.447362in}{0.696000in}}%
\pgfpathlineto{\pgfqpoint{4.501399in}{0.696000in}}%
\pgfpathlineto{\pgfqpoint{4.555385in}{0.696000in}}%
\pgfpathlineto{\pgfqpoint{4.609811in}{0.770210in}}%
\pgfpathlineto{\pgfqpoint{4.664101in}{0.770210in}}%
\pgfpathlineto{\pgfqpoint{4.720547in}{0.770210in}}%
\pgfpathlineto{\pgfqpoint{4.773658in}{0.770210in}}%
\pgfpathlineto{\pgfqpoint{4.827538in}{0.844420in}}%
\pgfpathlineto{\pgfqpoint{4.881133in}{0.844420in}}%
\pgfpathlineto{\pgfqpoint{4.935726in}{0.844420in}}%
\pgfpathlineto{\pgfqpoint{4.989441in}{0.844420in}}%
\pgfpathlineto{\pgfqpoint{5.043694in}{0.992840in}}%
\pgfpathlineto{\pgfqpoint{5.097926in}{0.992840in}}%
\pgfpathlineto{\pgfqpoint{5.153140in}{0.696000in}}%
\pgfpathlineto{\pgfqpoint{5.207015in}{0.844420in}}%
\pgfpathlineto{\pgfqpoint{5.261250in}{1.067050in}}%
\pgfpathlineto{\pgfqpoint{5.315997in}{0.992840in}}%
\pgfpathlineto{\pgfqpoint{5.372967in}{1.215470in}}%
\pgfpathlineto{\pgfqpoint{5.426319in}{1.215470in}}%
\pgfpathlineto{\pgfqpoint{5.480504in}{0.770210in}}%
\pgfpathlineto{\pgfqpoint{5.534545in}{0.918630in}}%
\pgfusepath{stroke}%
\end{pgfscope}%
\begin{pgfscope}%
\pgfpathrectangle{\pgfqpoint{0.800000in}{0.528000in}}{\pgfqpoint{4.960000in}{3.696000in}}%
\pgfusepath{clip}%
\pgfsetrectcap%
\pgfsetroundjoin%
\pgfsetlinewidth{1.505625pt}%
\definecolor{currentstroke}{rgb}{0.580392,0.403922,0.741176}%
\pgfsetstrokecolor{currentstroke}%
\pgfsetdash{}{0pt}%
\pgfpathmoveto{\pgfqpoint{1.025455in}{0.696000in}}%
\pgfpathlineto{\pgfqpoint{1.079678in}{1.663580in}}%
\pgfpathlineto{\pgfqpoint{1.133870in}{2.638261in}}%
\pgfpathlineto{\pgfqpoint{1.187734in}{3.155028in}}%
\pgfpathlineto{\pgfqpoint{1.241889in}{3.531560in}}%
\pgfpathlineto{\pgfqpoint{1.296423in}{3.753055in}}%
\pgfpathlineto{\pgfqpoint{1.350697in}{3.900806in}}%
\pgfpathlineto{\pgfqpoint{1.405557in}{3.826923in}}%
\pgfpathlineto{\pgfqpoint{1.460418in}{2.933403in}}%
\pgfpathlineto{\pgfqpoint{1.513987in}{2.785772in}}%
\pgfpathlineto{\pgfqpoint{1.567487in}{2.942002in}}%
\pgfpathlineto{\pgfqpoint{1.621247in}{2.196804in}}%
\pgfpathlineto{\pgfqpoint{1.675531in}{2.574847in}}%
\pgfpathlineto{\pgfqpoint{1.729613in}{3.026241in}}%
\pgfpathlineto{\pgfqpoint{1.783628in}{2.209593in}}%
\pgfpathlineto{\pgfqpoint{1.837870in}{1.758525in}}%
\pgfpathlineto{\pgfqpoint{1.892205in}{1.672080in}}%
\pgfpathlineto{\pgfqpoint{1.946340in}{1.598803in}}%
\pgfpathlineto{\pgfqpoint{2.000653in}{1.294300in}}%
\pgfpathlineto{\pgfqpoint{2.056394in}{0.770210in}}%
\pgfpathlineto{\pgfqpoint{2.109624in}{1.438100in}}%
\pgfpathlineto{\pgfqpoint{2.163468in}{1.589607in}}%
\pgfpathlineto{\pgfqpoint{2.217770in}{0.992840in}}%
\pgfpathlineto{\pgfqpoint{2.271725in}{1.959751in}}%
\pgfpathlineto{\pgfqpoint{2.326053in}{1.737587in}}%
\pgfpathlineto{\pgfqpoint{2.379841in}{2.417288in}}%
\pgfpathlineto{\pgfqpoint{2.434883in}{1.589607in}}%
\pgfpathlineto{\pgfqpoint{2.489075in}{2.123458in}}%
\pgfpathlineto{\pgfqpoint{2.543467in}{1.745488in}}%
\pgfpathlineto{\pgfqpoint{2.597829in}{2.270232in}}%
\pgfpathlineto{\pgfqpoint{2.651872in}{2.123458in}}%
\pgfpathlineto{\pgfqpoint{2.707213in}{2.343729in}}%
\pgfpathlineto{\pgfqpoint{2.760777in}{2.123458in}}%
\pgfpathlineto{\pgfqpoint{2.814858in}{2.050205in}}%
\pgfpathlineto{\pgfqpoint{2.869391in}{2.490900in}}%
\pgfpathlineto{\pgfqpoint{2.923610in}{1.892600in}}%
\pgfpathlineto{\pgfqpoint{2.978034in}{0.918630in}}%
\pgfpathlineto{\pgfqpoint{3.032428in}{2.187602in}}%
\pgfpathlineto{\pgfqpoint{3.086957in}{1.959751in}}%
\pgfpathlineto{\pgfqpoint{3.141180in}{1.289680in}}%
\pgfpathlineto{\pgfqpoint{3.196689in}{0.861939in}}%
\pgfpathlineto{\pgfqpoint{3.250042in}{0.844420in}}%
\pgfpathlineto{\pgfqpoint{3.303736in}{0.696000in}}%
\pgfpathlineto{\pgfqpoint{3.357780in}{0.696000in}}%
\pgfpathlineto{\pgfqpoint{3.412059in}{0.918630in}}%
\pgfpathlineto{\pgfqpoint{3.466320in}{1.672080in}}%
\pgfpathlineto{\pgfqpoint{3.520817in}{1.598803in}}%
\pgfpathlineto{\pgfqpoint{3.575235in}{1.892600in}}%
\pgfpathlineto{\pgfqpoint{3.629571in}{1.441801in}}%
\pgfpathlineto{\pgfqpoint{3.683862in}{1.147402in}}%
\pgfpathlineto{\pgfqpoint{3.738111in}{1.074398in}}%
\pgfpathlineto{\pgfqpoint{3.792593in}{0.770210in}}%
\pgfpathlineto{\pgfqpoint{3.847539in}{0.696000in}}%
\pgfpathlineto{\pgfqpoint{3.900746in}{0.770210in}}%
\pgfpathlineto{\pgfqpoint{3.954880in}{0.844420in}}%
\pgfpathlineto{\pgfqpoint{4.009521in}{1.067050in}}%
\pgfpathlineto{\pgfqpoint{4.063965in}{0.918630in}}%
\pgfpathlineto{\pgfqpoint{4.118325in}{0.918630in}}%
\pgfpathlineto{\pgfqpoint{4.172375in}{0.918630in}}%
\pgfpathlineto{\pgfqpoint{4.226630in}{0.992840in}}%
\pgfpathlineto{\pgfqpoint{4.280745in}{1.067050in}}%
\pgfpathlineto{\pgfqpoint{4.335030in}{1.067050in}}%
\pgfpathlineto{\pgfqpoint{4.389502in}{0.918630in}}%
\pgfpathlineto{\pgfqpoint{4.445307in}{1.067050in}}%
\pgfpathlineto{\pgfqpoint{4.498495in}{1.215470in}}%
\pgfpathlineto{\pgfqpoint{4.552551in}{1.220744in}}%
\pgfpathlineto{\pgfqpoint{4.606617in}{1.067050in}}%
\pgfpathlineto{\pgfqpoint{4.660704in}{1.001976in}}%
\pgfpathlineto{\pgfqpoint{4.714940in}{0.918630in}}%
\pgfpathlineto{\pgfqpoint{4.768919in}{1.215470in}}%
\pgfpathlineto{\pgfqpoint{4.823066in}{2.261461in}}%
\pgfpathlineto{\pgfqpoint{4.877425in}{2.107941in}}%
\pgfpathlineto{\pgfqpoint{4.931670in}{0.844420in}}%
\pgfpathlineto{\pgfqpoint{4.986005in}{1.363890in}}%
\pgfpathlineto{\pgfqpoint{5.040119in}{1.663580in}}%
\pgfpathlineto{\pgfqpoint{5.095467in}{0.696000in}}%
\pgfpathlineto{\pgfqpoint{5.149117in}{0.918630in}}%
\pgfpathlineto{\pgfqpoint{5.203347in}{1.515676in}}%
\pgfpathlineto{\pgfqpoint{5.257147in}{0.696000in}}%
\pgfpathlineto{\pgfqpoint{5.311097in}{0.992840in}}%
\pgfpathlineto{\pgfqpoint{5.365808in}{1.067050in}}%
\pgfpathlineto{\pgfqpoint{5.419776in}{0.918630in}}%
\pgfpathlineto{\pgfqpoint{5.473789in}{1.289680in}}%
\pgfpathlineto{\pgfqpoint{5.528091in}{1.141260in}}%
\pgfusepath{stroke}%
\end{pgfscope}%
\begin{pgfscope}%
\pgfsetrectcap%
\pgfsetmiterjoin%
\pgfsetlinewidth{0.803000pt}%
\definecolor{currentstroke}{rgb}{0.000000,0.000000,0.000000}%
\pgfsetstrokecolor{currentstroke}%
\pgfsetdash{}{0pt}%
\pgfpathmoveto{\pgfqpoint{0.800000in}{0.528000in}}%
\pgfpathlineto{\pgfqpoint{0.800000in}{4.224000in}}%
\pgfusepath{stroke}%
\end{pgfscope}%
\begin{pgfscope}%
\pgfsetrectcap%
\pgfsetmiterjoin%
\pgfsetlinewidth{0.803000pt}%
\definecolor{currentstroke}{rgb}{0.000000,0.000000,0.000000}%
\pgfsetstrokecolor{currentstroke}%
\pgfsetdash{}{0pt}%
\pgfpathmoveto{\pgfqpoint{5.760000in}{0.528000in}}%
\pgfpathlineto{\pgfqpoint{5.760000in}{4.224000in}}%
\pgfusepath{stroke}%
\end{pgfscope}%
\begin{pgfscope}%
\pgfsetrectcap%
\pgfsetmiterjoin%
\pgfsetlinewidth{0.803000pt}%
\definecolor{currentstroke}{rgb}{0.000000,0.000000,0.000000}%
\pgfsetstrokecolor{currentstroke}%
\pgfsetdash{}{0pt}%
\pgfpathmoveto{\pgfqpoint{0.800000in}{0.528000in}}%
\pgfpathlineto{\pgfqpoint{5.760000in}{0.528000in}}%
\pgfusepath{stroke}%
\end{pgfscope}%
\begin{pgfscope}%
\pgfsetrectcap%
\pgfsetmiterjoin%
\pgfsetlinewidth{0.803000pt}%
\definecolor{currentstroke}{rgb}{0.000000,0.000000,0.000000}%
\pgfsetstrokecolor{currentstroke}%
\pgfsetdash{}{0pt}%
\pgfpathmoveto{\pgfqpoint{0.800000in}{4.224000in}}%
\pgfpathlineto{\pgfqpoint{5.760000in}{4.224000in}}%
\pgfusepath{stroke}%
\end{pgfscope}%
\begin{pgfscope}%
\definecolor{textcolor}{rgb}{0.000000,0.000000,0.000000}%
\pgfsetstrokecolor{textcolor}%
\pgfsetfillcolor{textcolor}%
\pgftext[x=3.280000in,y=4.307333in,,base]{\color{textcolor}\sffamily\fontsize{12.000000}{14.400000}\selectfont Measured ground speed}%
\end{pgfscope}%
\begin{pgfscope}%
\pgfsetbuttcap%
\pgfsetmiterjoin%
\definecolor{currentfill}{rgb}{1.000000,1.000000,1.000000}%
\pgfsetfillcolor{currentfill}%
\pgfsetfillopacity{0.800000}%
\pgfsetlinewidth{1.003750pt}%
\definecolor{currentstroke}{rgb}{0.800000,0.800000,0.800000}%
\pgfsetstrokecolor{currentstroke}%
\pgfsetstrokeopacity{0.800000}%
\pgfsetdash{}{0pt}%
\pgfpathmoveto{\pgfqpoint{5.129968in}{3.093603in}}%
\pgfpathlineto{\pgfqpoint{5.662778in}{3.093603in}}%
\pgfpathquadraticcurveto{\pgfqpoint{5.690556in}{3.093603in}}{\pgfqpoint{5.690556in}{3.121381in}}%
\pgfpathlineto{\pgfqpoint{5.690556in}{4.126778in}}%
\pgfpathquadraticcurveto{\pgfqpoint{5.690556in}{4.154556in}}{\pgfqpoint{5.662778in}{4.154556in}}%
\pgfpathlineto{\pgfqpoint{5.129968in}{4.154556in}}%
\pgfpathquadraticcurveto{\pgfqpoint{5.102190in}{4.154556in}}{\pgfqpoint{5.102190in}{4.126778in}}%
\pgfpathlineto{\pgfqpoint{5.102190in}{3.121381in}}%
\pgfpathquadraticcurveto{\pgfqpoint{5.102190in}{3.093603in}}{\pgfqpoint{5.129968in}{3.093603in}}%
\pgfpathlineto{\pgfqpoint{5.129968in}{3.093603in}}%
\pgfpathclose%
\pgfusepath{stroke,fill}%
\end{pgfscope}%
\begin{pgfscope}%
\pgfsetrectcap%
\pgfsetroundjoin%
\pgfsetlinewidth{1.505625pt}%
\definecolor{currentstroke}{rgb}{0.121569,0.466667,0.705882}%
\pgfsetstrokecolor{currentstroke}%
\pgfsetdash{}{0pt}%
\pgfpathmoveto{\pgfqpoint{5.157746in}{4.042088in}}%
\pgfpathlineto{\pgfqpoint{5.296635in}{4.042088in}}%
\pgfpathlineto{\pgfqpoint{5.435524in}{4.042088in}}%
\pgfusepath{stroke}%
\end{pgfscope}%
\begin{pgfscope}%
\definecolor{textcolor}{rgb}{0.000000,0.000000,0.000000}%
\pgfsetstrokecolor{textcolor}%
\pgfsetfillcolor{textcolor}%
\pgftext[x=5.546635in,y=3.993477in,left,base]{\color{textcolor}\sffamily\fontsize{10.000000}{12.000000}\selectfont 0}%
\end{pgfscope}%
\begin{pgfscope}%
\pgfsetrectcap%
\pgfsetroundjoin%
\pgfsetlinewidth{1.505625pt}%
\definecolor{currentstroke}{rgb}{1.000000,0.498039,0.054902}%
\pgfsetstrokecolor{currentstroke}%
\pgfsetdash{}{0pt}%
\pgfpathmoveto{\pgfqpoint{5.157746in}{3.838231in}}%
\pgfpathlineto{\pgfqpoint{5.296635in}{3.838231in}}%
\pgfpathlineto{\pgfqpoint{5.435524in}{3.838231in}}%
\pgfusepath{stroke}%
\end{pgfscope}%
\begin{pgfscope}%
\definecolor{textcolor}{rgb}{0.000000,0.000000,0.000000}%
\pgfsetstrokecolor{textcolor}%
\pgfsetfillcolor{textcolor}%
\pgftext[x=5.546635in,y=3.789620in,left,base]{\color{textcolor}\sffamily\fontsize{10.000000}{12.000000}\selectfont 1}%
\end{pgfscope}%
\begin{pgfscope}%
\pgfsetrectcap%
\pgfsetroundjoin%
\pgfsetlinewidth{1.505625pt}%
\definecolor{currentstroke}{rgb}{0.172549,0.627451,0.172549}%
\pgfsetstrokecolor{currentstroke}%
\pgfsetdash{}{0pt}%
\pgfpathmoveto{\pgfqpoint{5.157746in}{3.634374in}}%
\pgfpathlineto{\pgfqpoint{5.296635in}{3.634374in}}%
\pgfpathlineto{\pgfqpoint{5.435524in}{3.634374in}}%
\pgfusepath{stroke}%
\end{pgfscope}%
\begin{pgfscope}%
\definecolor{textcolor}{rgb}{0.000000,0.000000,0.000000}%
\pgfsetstrokecolor{textcolor}%
\pgfsetfillcolor{textcolor}%
\pgftext[x=5.546635in,y=3.585762in,left,base]{\color{textcolor}\sffamily\fontsize{10.000000}{12.000000}\selectfont 2}%
\end{pgfscope}%
\begin{pgfscope}%
\pgfsetrectcap%
\pgfsetroundjoin%
\pgfsetlinewidth{1.505625pt}%
\definecolor{currentstroke}{rgb}{0.839216,0.152941,0.156863}%
\pgfsetstrokecolor{currentstroke}%
\pgfsetdash{}{0pt}%
\pgfpathmoveto{\pgfqpoint{5.157746in}{3.430516in}}%
\pgfpathlineto{\pgfqpoint{5.296635in}{3.430516in}}%
\pgfpathlineto{\pgfqpoint{5.435524in}{3.430516in}}%
\pgfusepath{stroke}%
\end{pgfscope}%
\begin{pgfscope}%
\definecolor{textcolor}{rgb}{0.000000,0.000000,0.000000}%
\pgfsetstrokecolor{textcolor}%
\pgfsetfillcolor{textcolor}%
\pgftext[x=5.546635in,y=3.381905in,left,base]{\color{textcolor}\sffamily\fontsize{10.000000}{12.000000}\selectfont 3}%
\end{pgfscope}%
\begin{pgfscope}%
\pgfsetrectcap%
\pgfsetroundjoin%
\pgfsetlinewidth{1.505625pt}%
\definecolor{currentstroke}{rgb}{0.580392,0.403922,0.741176}%
\pgfsetstrokecolor{currentstroke}%
\pgfsetdash{}{0pt}%
\pgfpathmoveto{\pgfqpoint{5.157746in}{3.226659in}}%
\pgfpathlineto{\pgfqpoint{5.296635in}{3.226659in}}%
\pgfpathlineto{\pgfqpoint{5.435524in}{3.226659in}}%
\pgfusepath{stroke}%
\end{pgfscope}%
\begin{pgfscope}%
\definecolor{textcolor}{rgb}{0.000000,0.000000,0.000000}%
\pgfsetstrokecolor{textcolor}%
\pgfsetfillcolor{textcolor}%
\pgftext[x=5.546635in,y=3.178048in,left,base]{\color{textcolor}\sffamily\fontsize{10.000000}{12.000000}\selectfont 4}%
\end{pgfscope}%
\end{pgfpicture}%
\makeatother%
\endgroup%
}
    \end{minipage}
    \caption{Variation of (a) measured forward position and (b) measured absolute ground velocity for different values of $K_{I}$ and $K_P=4$, $K_D=0$ while the forward controller is engaged.}
    \label{fig:tune-fwd-int-measures}
\end{figure}

\begin{figure}[H]
    \begin{minipage}[t]{0.5\linewidth}
        \centering
        \scalebox{0.55}{%% Creator: Matplotlib, PGF backend
%%
%% To include the figure in your LaTeX document, write
%%   \input{<filename>.pgf}
%%
%% Make sure the required packages are loaded in your preamble
%%   \usepackage{pgf}
%%
%% Also ensure that all the required font packages are loaded; for instance,
%% the lmodern package is sometimes necessary when using math font.
%%   \usepackage{lmodern}
%%
%% Figures using additional raster images can only be included by \input if
%% they are in the same directory as the main LaTeX file. For loading figures
%% from other directories you can use the `import` package
%%   \usepackage{import}
%%
%% and then include the figures with
%%   \import{<path to file>}{<filename>.pgf}
%%
%% Matplotlib used the following preamble
%%   \usepackage{fontspec}
%%   \setmainfont{DejaVuSerif.ttf}[Path=\detokenize{/home/lgonz/tfg-aero/tfg-giaa-dronecontrol/venv/lib/python3.8/site-packages/matplotlib/mpl-data/fonts/ttf/}]
%%   \setsansfont{DejaVuSans.ttf}[Path=\detokenize{/home/lgonz/tfg-aero/tfg-giaa-dronecontrol/venv/lib/python3.8/site-packages/matplotlib/mpl-data/fonts/ttf/}]
%%   \setmonofont{DejaVuSansMono.ttf}[Path=\detokenize{/home/lgonz/tfg-aero/tfg-giaa-dronecontrol/venv/lib/python3.8/site-packages/matplotlib/mpl-data/fonts/ttf/}]
%%
\begingroup%
\makeatletter%
\begin{pgfpicture}%
\pgfpathrectangle{\pgfpointorigin}{\pgfqpoint{6.400000in}{4.800000in}}%
\pgfusepath{use as bounding box, clip}%
\begin{pgfscope}%
\pgfsetbuttcap%
\pgfsetmiterjoin%
\definecolor{currentfill}{rgb}{1.000000,1.000000,1.000000}%
\pgfsetfillcolor{currentfill}%
\pgfsetlinewidth{0.000000pt}%
\definecolor{currentstroke}{rgb}{1.000000,1.000000,1.000000}%
\pgfsetstrokecolor{currentstroke}%
\pgfsetdash{}{0pt}%
\pgfpathmoveto{\pgfqpoint{0.000000in}{0.000000in}}%
\pgfpathlineto{\pgfqpoint{6.400000in}{0.000000in}}%
\pgfpathlineto{\pgfqpoint{6.400000in}{4.800000in}}%
\pgfpathlineto{\pgfqpoint{0.000000in}{4.800000in}}%
\pgfpathlineto{\pgfqpoint{0.000000in}{0.000000in}}%
\pgfpathclose%
\pgfusepath{fill}%
\end{pgfscope}%
\begin{pgfscope}%
\pgfsetbuttcap%
\pgfsetmiterjoin%
\definecolor{currentfill}{rgb}{1.000000,1.000000,1.000000}%
\pgfsetfillcolor{currentfill}%
\pgfsetlinewidth{0.000000pt}%
\definecolor{currentstroke}{rgb}{0.000000,0.000000,0.000000}%
\pgfsetstrokecolor{currentstroke}%
\pgfsetstrokeopacity{0.000000}%
\pgfsetdash{}{0pt}%
\pgfpathmoveto{\pgfqpoint{0.800000in}{0.528000in}}%
\pgfpathlineto{\pgfqpoint{5.760000in}{0.528000in}}%
\pgfpathlineto{\pgfqpoint{5.760000in}{4.224000in}}%
\pgfpathlineto{\pgfqpoint{0.800000in}{4.224000in}}%
\pgfpathlineto{\pgfqpoint{0.800000in}{0.528000in}}%
\pgfpathclose%
\pgfusepath{fill}%
\end{pgfscope}%
\begin{pgfscope}%
\pgfpathrectangle{\pgfqpoint{0.800000in}{0.528000in}}{\pgfqpoint{4.960000in}{3.696000in}}%
\pgfusepath{clip}%
\pgfsetrectcap%
\pgfsetroundjoin%
\pgfsetlinewidth{0.803000pt}%
\definecolor{currentstroke}{rgb}{0.690196,0.690196,0.690196}%
\pgfsetstrokecolor{currentstroke}%
\pgfsetdash{}{0pt}%
\pgfpathmoveto{\pgfqpoint{1.025455in}{0.528000in}}%
\pgfpathlineto{\pgfqpoint{1.025455in}{4.224000in}}%
\pgfusepath{stroke}%
\end{pgfscope}%
\begin{pgfscope}%
\pgfsetbuttcap%
\pgfsetroundjoin%
\definecolor{currentfill}{rgb}{0.000000,0.000000,0.000000}%
\pgfsetfillcolor{currentfill}%
\pgfsetlinewidth{0.803000pt}%
\definecolor{currentstroke}{rgb}{0.000000,0.000000,0.000000}%
\pgfsetstrokecolor{currentstroke}%
\pgfsetdash{}{0pt}%
\pgfsys@defobject{currentmarker}{\pgfqpoint{0.000000in}{-0.048611in}}{\pgfqpoint{0.000000in}{0.000000in}}{%
\pgfpathmoveto{\pgfqpoint{0.000000in}{0.000000in}}%
\pgfpathlineto{\pgfqpoint{0.000000in}{-0.048611in}}%
\pgfusepath{stroke,fill}%
}%
\begin{pgfscope}%
\pgfsys@transformshift{1.025455in}{0.528000in}%
\pgfsys@useobject{currentmarker}{}%
\end{pgfscope}%
\end{pgfscope}%
\begin{pgfscope}%
\definecolor{textcolor}{rgb}{0.000000,0.000000,0.000000}%
\pgfsetstrokecolor{textcolor}%
\pgfsetfillcolor{textcolor}%
\pgftext[x=1.025455in,y=0.430778in,,top]{\color{textcolor}\sffamily\fontsize{10.000000}{12.000000}\selectfont 0}%
\end{pgfscope}%
\begin{pgfscope}%
\pgfpathrectangle{\pgfqpoint{0.800000in}{0.528000in}}{\pgfqpoint{4.960000in}{3.696000in}}%
\pgfusepath{clip}%
\pgfsetrectcap%
\pgfsetroundjoin%
\pgfsetlinewidth{0.803000pt}%
\definecolor{currentstroke}{rgb}{0.690196,0.690196,0.690196}%
\pgfsetstrokecolor{currentstroke}%
\pgfsetdash{}{0pt}%
\pgfpathmoveto{\pgfqpoint{1.776345in}{0.528000in}}%
\pgfpathlineto{\pgfqpoint{1.776345in}{4.224000in}}%
\pgfusepath{stroke}%
\end{pgfscope}%
\begin{pgfscope}%
\pgfsetbuttcap%
\pgfsetroundjoin%
\definecolor{currentfill}{rgb}{0.000000,0.000000,0.000000}%
\pgfsetfillcolor{currentfill}%
\pgfsetlinewidth{0.803000pt}%
\definecolor{currentstroke}{rgb}{0.000000,0.000000,0.000000}%
\pgfsetstrokecolor{currentstroke}%
\pgfsetdash{}{0pt}%
\pgfsys@defobject{currentmarker}{\pgfqpoint{0.000000in}{-0.048611in}}{\pgfqpoint{0.000000in}{0.000000in}}{%
\pgfpathmoveto{\pgfqpoint{0.000000in}{0.000000in}}%
\pgfpathlineto{\pgfqpoint{0.000000in}{-0.048611in}}%
\pgfusepath{stroke,fill}%
}%
\begin{pgfscope}%
\pgfsys@transformshift{1.776345in}{0.528000in}%
\pgfsys@useobject{currentmarker}{}%
\end{pgfscope}%
\end{pgfscope}%
\begin{pgfscope}%
\definecolor{textcolor}{rgb}{0.000000,0.000000,0.000000}%
\pgfsetstrokecolor{textcolor}%
\pgfsetfillcolor{textcolor}%
\pgftext[x=1.776345in,y=0.430778in,,top]{\color{textcolor}\sffamily\fontsize{10.000000}{12.000000}\selectfont 5}%
\end{pgfscope}%
\begin{pgfscope}%
\pgfpathrectangle{\pgfqpoint{0.800000in}{0.528000in}}{\pgfqpoint{4.960000in}{3.696000in}}%
\pgfusepath{clip}%
\pgfsetrectcap%
\pgfsetroundjoin%
\pgfsetlinewidth{0.803000pt}%
\definecolor{currentstroke}{rgb}{0.690196,0.690196,0.690196}%
\pgfsetstrokecolor{currentstroke}%
\pgfsetdash{}{0pt}%
\pgfpathmoveto{\pgfqpoint{2.527235in}{0.528000in}}%
\pgfpathlineto{\pgfqpoint{2.527235in}{4.224000in}}%
\pgfusepath{stroke}%
\end{pgfscope}%
\begin{pgfscope}%
\pgfsetbuttcap%
\pgfsetroundjoin%
\definecolor{currentfill}{rgb}{0.000000,0.000000,0.000000}%
\pgfsetfillcolor{currentfill}%
\pgfsetlinewidth{0.803000pt}%
\definecolor{currentstroke}{rgb}{0.000000,0.000000,0.000000}%
\pgfsetstrokecolor{currentstroke}%
\pgfsetdash{}{0pt}%
\pgfsys@defobject{currentmarker}{\pgfqpoint{0.000000in}{-0.048611in}}{\pgfqpoint{0.000000in}{0.000000in}}{%
\pgfpathmoveto{\pgfqpoint{0.000000in}{0.000000in}}%
\pgfpathlineto{\pgfqpoint{0.000000in}{-0.048611in}}%
\pgfusepath{stroke,fill}%
}%
\begin{pgfscope}%
\pgfsys@transformshift{2.527235in}{0.528000in}%
\pgfsys@useobject{currentmarker}{}%
\end{pgfscope}%
\end{pgfscope}%
\begin{pgfscope}%
\definecolor{textcolor}{rgb}{0.000000,0.000000,0.000000}%
\pgfsetstrokecolor{textcolor}%
\pgfsetfillcolor{textcolor}%
\pgftext[x=2.527235in,y=0.430778in,,top]{\color{textcolor}\sffamily\fontsize{10.000000}{12.000000}\selectfont 10}%
\end{pgfscope}%
\begin{pgfscope}%
\pgfpathrectangle{\pgfqpoint{0.800000in}{0.528000in}}{\pgfqpoint{4.960000in}{3.696000in}}%
\pgfusepath{clip}%
\pgfsetrectcap%
\pgfsetroundjoin%
\pgfsetlinewidth{0.803000pt}%
\definecolor{currentstroke}{rgb}{0.690196,0.690196,0.690196}%
\pgfsetstrokecolor{currentstroke}%
\pgfsetdash{}{0pt}%
\pgfpathmoveto{\pgfqpoint{3.278125in}{0.528000in}}%
\pgfpathlineto{\pgfqpoint{3.278125in}{4.224000in}}%
\pgfusepath{stroke}%
\end{pgfscope}%
\begin{pgfscope}%
\pgfsetbuttcap%
\pgfsetroundjoin%
\definecolor{currentfill}{rgb}{0.000000,0.000000,0.000000}%
\pgfsetfillcolor{currentfill}%
\pgfsetlinewidth{0.803000pt}%
\definecolor{currentstroke}{rgb}{0.000000,0.000000,0.000000}%
\pgfsetstrokecolor{currentstroke}%
\pgfsetdash{}{0pt}%
\pgfsys@defobject{currentmarker}{\pgfqpoint{0.000000in}{-0.048611in}}{\pgfqpoint{0.000000in}{0.000000in}}{%
\pgfpathmoveto{\pgfqpoint{0.000000in}{0.000000in}}%
\pgfpathlineto{\pgfqpoint{0.000000in}{-0.048611in}}%
\pgfusepath{stroke,fill}%
}%
\begin{pgfscope}%
\pgfsys@transformshift{3.278125in}{0.528000in}%
\pgfsys@useobject{currentmarker}{}%
\end{pgfscope}%
\end{pgfscope}%
\begin{pgfscope}%
\definecolor{textcolor}{rgb}{0.000000,0.000000,0.000000}%
\pgfsetstrokecolor{textcolor}%
\pgfsetfillcolor{textcolor}%
\pgftext[x=3.278125in,y=0.430778in,,top]{\color{textcolor}\sffamily\fontsize{10.000000}{12.000000}\selectfont 15}%
\end{pgfscope}%
\begin{pgfscope}%
\pgfpathrectangle{\pgfqpoint{0.800000in}{0.528000in}}{\pgfqpoint{4.960000in}{3.696000in}}%
\pgfusepath{clip}%
\pgfsetrectcap%
\pgfsetroundjoin%
\pgfsetlinewidth{0.803000pt}%
\definecolor{currentstroke}{rgb}{0.690196,0.690196,0.690196}%
\pgfsetstrokecolor{currentstroke}%
\pgfsetdash{}{0pt}%
\pgfpathmoveto{\pgfqpoint{4.029015in}{0.528000in}}%
\pgfpathlineto{\pgfqpoint{4.029015in}{4.224000in}}%
\pgfusepath{stroke}%
\end{pgfscope}%
\begin{pgfscope}%
\pgfsetbuttcap%
\pgfsetroundjoin%
\definecolor{currentfill}{rgb}{0.000000,0.000000,0.000000}%
\pgfsetfillcolor{currentfill}%
\pgfsetlinewidth{0.803000pt}%
\definecolor{currentstroke}{rgb}{0.000000,0.000000,0.000000}%
\pgfsetstrokecolor{currentstroke}%
\pgfsetdash{}{0pt}%
\pgfsys@defobject{currentmarker}{\pgfqpoint{0.000000in}{-0.048611in}}{\pgfqpoint{0.000000in}{0.000000in}}{%
\pgfpathmoveto{\pgfqpoint{0.000000in}{0.000000in}}%
\pgfpathlineto{\pgfqpoint{0.000000in}{-0.048611in}}%
\pgfusepath{stroke,fill}%
}%
\begin{pgfscope}%
\pgfsys@transformshift{4.029015in}{0.528000in}%
\pgfsys@useobject{currentmarker}{}%
\end{pgfscope}%
\end{pgfscope}%
\begin{pgfscope}%
\definecolor{textcolor}{rgb}{0.000000,0.000000,0.000000}%
\pgfsetstrokecolor{textcolor}%
\pgfsetfillcolor{textcolor}%
\pgftext[x=4.029015in,y=0.430778in,,top]{\color{textcolor}\sffamily\fontsize{10.000000}{12.000000}\selectfont 20}%
\end{pgfscope}%
\begin{pgfscope}%
\pgfpathrectangle{\pgfqpoint{0.800000in}{0.528000in}}{\pgfqpoint{4.960000in}{3.696000in}}%
\pgfusepath{clip}%
\pgfsetrectcap%
\pgfsetroundjoin%
\pgfsetlinewidth{0.803000pt}%
\definecolor{currentstroke}{rgb}{0.690196,0.690196,0.690196}%
\pgfsetstrokecolor{currentstroke}%
\pgfsetdash{}{0pt}%
\pgfpathmoveto{\pgfqpoint{4.779905in}{0.528000in}}%
\pgfpathlineto{\pgfqpoint{4.779905in}{4.224000in}}%
\pgfusepath{stroke}%
\end{pgfscope}%
\begin{pgfscope}%
\pgfsetbuttcap%
\pgfsetroundjoin%
\definecolor{currentfill}{rgb}{0.000000,0.000000,0.000000}%
\pgfsetfillcolor{currentfill}%
\pgfsetlinewidth{0.803000pt}%
\definecolor{currentstroke}{rgb}{0.000000,0.000000,0.000000}%
\pgfsetstrokecolor{currentstroke}%
\pgfsetdash{}{0pt}%
\pgfsys@defobject{currentmarker}{\pgfqpoint{0.000000in}{-0.048611in}}{\pgfqpoint{0.000000in}{0.000000in}}{%
\pgfpathmoveto{\pgfqpoint{0.000000in}{0.000000in}}%
\pgfpathlineto{\pgfqpoint{0.000000in}{-0.048611in}}%
\pgfusepath{stroke,fill}%
}%
\begin{pgfscope}%
\pgfsys@transformshift{4.779905in}{0.528000in}%
\pgfsys@useobject{currentmarker}{}%
\end{pgfscope}%
\end{pgfscope}%
\begin{pgfscope}%
\definecolor{textcolor}{rgb}{0.000000,0.000000,0.000000}%
\pgfsetstrokecolor{textcolor}%
\pgfsetfillcolor{textcolor}%
\pgftext[x=4.779905in,y=0.430778in,,top]{\color{textcolor}\sffamily\fontsize{10.000000}{12.000000}\selectfont 25}%
\end{pgfscope}%
\begin{pgfscope}%
\pgfpathrectangle{\pgfqpoint{0.800000in}{0.528000in}}{\pgfqpoint{4.960000in}{3.696000in}}%
\pgfusepath{clip}%
\pgfsetrectcap%
\pgfsetroundjoin%
\pgfsetlinewidth{0.803000pt}%
\definecolor{currentstroke}{rgb}{0.690196,0.690196,0.690196}%
\pgfsetstrokecolor{currentstroke}%
\pgfsetdash{}{0pt}%
\pgfpathmoveto{\pgfqpoint{5.530795in}{0.528000in}}%
\pgfpathlineto{\pgfqpoint{5.530795in}{4.224000in}}%
\pgfusepath{stroke}%
\end{pgfscope}%
\begin{pgfscope}%
\pgfsetbuttcap%
\pgfsetroundjoin%
\definecolor{currentfill}{rgb}{0.000000,0.000000,0.000000}%
\pgfsetfillcolor{currentfill}%
\pgfsetlinewidth{0.803000pt}%
\definecolor{currentstroke}{rgb}{0.000000,0.000000,0.000000}%
\pgfsetstrokecolor{currentstroke}%
\pgfsetdash{}{0pt}%
\pgfsys@defobject{currentmarker}{\pgfqpoint{0.000000in}{-0.048611in}}{\pgfqpoint{0.000000in}{0.000000in}}{%
\pgfpathmoveto{\pgfqpoint{0.000000in}{0.000000in}}%
\pgfpathlineto{\pgfqpoint{0.000000in}{-0.048611in}}%
\pgfusepath{stroke,fill}%
}%
\begin{pgfscope}%
\pgfsys@transformshift{5.530795in}{0.528000in}%
\pgfsys@useobject{currentmarker}{}%
\end{pgfscope}%
\end{pgfscope}%
\begin{pgfscope}%
\definecolor{textcolor}{rgb}{0.000000,0.000000,0.000000}%
\pgfsetstrokecolor{textcolor}%
\pgfsetfillcolor{textcolor}%
\pgftext[x=5.530795in,y=0.430778in,,top]{\color{textcolor}\sffamily\fontsize{10.000000}{12.000000}\selectfont 30}%
\end{pgfscope}%
\begin{pgfscope}%
\definecolor{textcolor}{rgb}{0.000000,0.000000,0.000000}%
\pgfsetstrokecolor{textcolor}%
\pgfsetfillcolor{textcolor}%
\pgftext[x=3.280000in,y=0.240809in,,top]{\color{textcolor}\sffamily\fontsize{10.000000}{12.000000}\selectfont time [s]}%
\end{pgfscope}%
\begin{pgfscope}%
\pgfpathrectangle{\pgfqpoint{0.800000in}{0.528000in}}{\pgfqpoint{4.960000in}{3.696000in}}%
\pgfusepath{clip}%
\pgfsetrectcap%
\pgfsetroundjoin%
\pgfsetlinewidth{0.803000pt}%
\definecolor{currentstroke}{rgb}{0.690196,0.690196,0.690196}%
\pgfsetstrokecolor{currentstroke}%
\pgfsetdash{}{0pt}%
\pgfpathmoveto{\pgfqpoint{0.800000in}{0.735542in}}%
\pgfpathlineto{\pgfqpoint{5.760000in}{0.735542in}}%
\pgfusepath{stroke}%
\end{pgfscope}%
\begin{pgfscope}%
\pgfsetbuttcap%
\pgfsetroundjoin%
\definecolor{currentfill}{rgb}{0.000000,0.000000,0.000000}%
\pgfsetfillcolor{currentfill}%
\pgfsetlinewidth{0.803000pt}%
\definecolor{currentstroke}{rgb}{0.000000,0.000000,0.000000}%
\pgfsetstrokecolor{currentstroke}%
\pgfsetdash{}{0pt}%
\pgfsys@defobject{currentmarker}{\pgfqpoint{-0.048611in}{0.000000in}}{\pgfqpoint{-0.000000in}{0.000000in}}{%
\pgfpathmoveto{\pgfqpoint{-0.000000in}{0.000000in}}%
\pgfpathlineto{\pgfqpoint{-0.048611in}{0.000000in}}%
\pgfusepath{stroke,fill}%
}%
\begin{pgfscope}%
\pgfsys@transformshift{0.800000in}{0.735542in}%
\pgfsys@useobject{currentmarker}{}%
\end{pgfscope}%
\end{pgfscope}%
\begin{pgfscope}%
\definecolor{textcolor}{rgb}{0.000000,0.000000,0.000000}%
\pgfsetstrokecolor{textcolor}%
\pgfsetfillcolor{textcolor}%
\pgftext[x=0.285508in, y=0.682780in, left, base]{\color{textcolor}\sffamily\fontsize{10.000000}{12.000000}\selectfont \ensuremath{-}0.20}%
\end{pgfscope}%
\begin{pgfscope}%
\pgfpathrectangle{\pgfqpoint{0.800000in}{0.528000in}}{\pgfqpoint{4.960000in}{3.696000in}}%
\pgfusepath{clip}%
\pgfsetrectcap%
\pgfsetroundjoin%
\pgfsetlinewidth{0.803000pt}%
\definecolor{currentstroke}{rgb}{0.690196,0.690196,0.690196}%
\pgfsetstrokecolor{currentstroke}%
\pgfsetdash{}{0pt}%
\pgfpathmoveto{\pgfqpoint{0.800000in}{1.376777in}}%
\pgfpathlineto{\pgfqpoint{5.760000in}{1.376777in}}%
\pgfusepath{stroke}%
\end{pgfscope}%
\begin{pgfscope}%
\pgfsetbuttcap%
\pgfsetroundjoin%
\definecolor{currentfill}{rgb}{0.000000,0.000000,0.000000}%
\pgfsetfillcolor{currentfill}%
\pgfsetlinewidth{0.803000pt}%
\definecolor{currentstroke}{rgb}{0.000000,0.000000,0.000000}%
\pgfsetstrokecolor{currentstroke}%
\pgfsetdash{}{0pt}%
\pgfsys@defobject{currentmarker}{\pgfqpoint{-0.048611in}{0.000000in}}{\pgfqpoint{-0.000000in}{0.000000in}}{%
\pgfpathmoveto{\pgfqpoint{-0.000000in}{0.000000in}}%
\pgfpathlineto{\pgfqpoint{-0.048611in}{0.000000in}}%
\pgfusepath{stroke,fill}%
}%
\begin{pgfscope}%
\pgfsys@transformshift{0.800000in}{1.376777in}%
\pgfsys@useobject{currentmarker}{}%
\end{pgfscope}%
\end{pgfscope}%
\begin{pgfscope}%
\definecolor{textcolor}{rgb}{0.000000,0.000000,0.000000}%
\pgfsetstrokecolor{textcolor}%
\pgfsetfillcolor{textcolor}%
\pgftext[x=0.285508in, y=1.324016in, left, base]{\color{textcolor}\sffamily\fontsize{10.000000}{12.000000}\selectfont \ensuremath{-}0.15}%
\end{pgfscope}%
\begin{pgfscope}%
\pgfpathrectangle{\pgfqpoint{0.800000in}{0.528000in}}{\pgfqpoint{4.960000in}{3.696000in}}%
\pgfusepath{clip}%
\pgfsetrectcap%
\pgfsetroundjoin%
\pgfsetlinewidth{0.803000pt}%
\definecolor{currentstroke}{rgb}{0.690196,0.690196,0.690196}%
\pgfsetstrokecolor{currentstroke}%
\pgfsetdash{}{0pt}%
\pgfpathmoveto{\pgfqpoint{0.800000in}{2.018013in}}%
\pgfpathlineto{\pgfqpoint{5.760000in}{2.018013in}}%
\pgfusepath{stroke}%
\end{pgfscope}%
\begin{pgfscope}%
\pgfsetbuttcap%
\pgfsetroundjoin%
\definecolor{currentfill}{rgb}{0.000000,0.000000,0.000000}%
\pgfsetfillcolor{currentfill}%
\pgfsetlinewidth{0.803000pt}%
\definecolor{currentstroke}{rgb}{0.000000,0.000000,0.000000}%
\pgfsetstrokecolor{currentstroke}%
\pgfsetdash{}{0pt}%
\pgfsys@defobject{currentmarker}{\pgfqpoint{-0.048611in}{0.000000in}}{\pgfqpoint{-0.000000in}{0.000000in}}{%
\pgfpathmoveto{\pgfqpoint{-0.000000in}{0.000000in}}%
\pgfpathlineto{\pgfqpoint{-0.048611in}{0.000000in}}%
\pgfusepath{stroke,fill}%
}%
\begin{pgfscope}%
\pgfsys@transformshift{0.800000in}{2.018013in}%
\pgfsys@useobject{currentmarker}{}%
\end{pgfscope}%
\end{pgfscope}%
\begin{pgfscope}%
\definecolor{textcolor}{rgb}{0.000000,0.000000,0.000000}%
\pgfsetstrokecolor{textcolor}%
\pgfsetfillcolor{textcolor}%
\pgftext[x=0.285508in, y=1.965251in, left, base]{\color{textcolor}\sffamily\fontsize{10.000000}{12.000000}\selectfont \ensuremath{-}0.10}%
\end{pgfscope}%
\begin{pgfscope}%
\pgfpathrectangle{\pgfqpoint{0.800000in}{0.528000in}}{\pgfqpoint{4.960000in}{3.696000in}}%
\pgfusepath{clip}%
\pgfsetrectcap%
\pgfsetroundjoin%
\pgfsetlinewidth{0.803000pt}%
\definecolor{currentstroke}{rgb}{0.690196,0.690196,0.690196}%
\pgfsetstrokecolor{currentstroke}%
\pgfsetdash{}{0pt}%
\pgfpathmoveto{\pgfqpoint{0.800000in}{2.659248in}}%
\pgfpathlineto{\pgfqpoint{5.760000in}{2.659248in}}%
\pgfusepath{stroke}%
\end{pgfscope}%
\begin{pgfscope}%
\pgfsetbuttcap%
\pgfsetroundjoin%
\definecolor{currentfill}{rgb}{0.000000,0.000000,0.000000}%
\pgfsetfillcolor{currentfill}%
\pgfsetlinewidth{0.803000pt}%
\definecolor{currentstroke}{rgb}{0.000000,0.000000,0.000000}%
\pgfsetstrokecolor{currentstroke}%
\pgfsetdash{}{0pt}%
\pgfsys@defobject{currentmarker}{\pgfqpoint{-0.048611in}{0.000000in}}{\pgfqpoint{-0.000000in}{0.000000in}}{%
\pgfpathmoveto{\pgfqpoint{-0.000000in}{0.000000in}}%
\pgfpathlineto{\pgfqpoint{-0.048611in}{0.000000in}}%
\pgfusepath{stroke,fill}%
}%
\begin{pgfscope}%
\pgfsys@transformshift{0.800000in}{2.659248in}%
\pgfsys@useobject{currentmarker}{}%
\end{pgfscope}%
\end{pgfscope}%
\begin{pgfscope}%
\definecolor{textcolor}{rgb}{0.000000,0.000000,0.000000}%
\pgfsetstrokecolor{textcolor}%
\pgfsetfillcolor{textcolor}%
\pgftext[x=0.285508in, y=2.606487in, left, base]{\color{textcolor}\sffamily\fontsize{10.000000}{12.000000}\selectfont \ensuremath{-}0.05}%
\end{pgfscope}%
\begin{pgfscope}%
\pgfpathrectangle{\pgfqpoint{0.800000in}{0.528000in}}{\pgfqpoint{4.960000in}{3.696000in}}%
\pgfusepath{clip}%
\pgfsetrectcap%
\pgfsetroundjoin%
\pgfsetlinewidth{0.803000pt}%
\definecolor{currentstroke}{rgb}{0.690196,0.690196,0.690196}%
\pgfsetstrokecolor{currentstroke}%
\pgfsetdash{}{0pt}%
\pgfpathmoveto{\pgfqpoint{0.800000in}{3.300484in}}%
\pgfpathlineto{\pgfqpoint{5.760000in}{3.300484in}}%
\pgfusepath{stroke}%
\end{pgfscope}%
\begin{pgfscope}%
\pgfsetbuttcap%
\pgfsetroundjoin%
\definecolor{currentfill}{rgb}{0.000000,0.000000,0.000000}%
\pgfsetfillcolor{currentfill}%
\pgfsetlinewidth{0.803000pt}%
\definecolor{currentstroke}{rgb}{0.000000,0.000000,0.000000}%
\pgfsetstrokecolor{currentstroke}%
\pgfsetdash{}{0pt}%
\pgfsys@defobject{currentmarker}{\pgfqpoint{-0.048611in}{0.000000in}}{\pgfqpoint{-0.000000in}{0.000000in}}{%
\pgfpathmoveto{\pgfqpoint{-0.000000in}{0.000000in}}%
\pgfpathlineto{\pgfqpoint{-0.048611in}{0.000000in}}%
\pgfusepath{stroke,fill}%
}%
\begin{pgfscope}%
\pgfsys@transformshift{0.800000in}{3.300484in}%
\pgfsys@useobject{currentmarker}{}%
\end{pgfscope}%
\end{pgfscope}%
\begin{pgfscope}%
\definecolor{textcolor}{rgb}{0.000000,0.000000,0.000000}%
\pgfsetstrokecolor{textcolor}%
\pgfsetfillcolor{textcolor}%
\pgftext[x=0.393533in, y=3.247723in, left, base]{\color{textcolor}\sffamily\fontsize{10.000000}{12.000000}\selectfont 0.00}%
\end{pgfscope}%
\begin{pgfscope}%
\pgfpathrectangle{\pgfqpoint{0.800000in}{0.528000in}}{\pgfqpoint{4.960000in}{3.696000in}}%
\pgfusepath{clip}%
\pgfsetrectcap%
\pgfsetroundjoin%
\pgfsetlinewidth{0.803000pt}%
\definecolor{currentstroke}{rgb}{0.690196,0.690196,0.690196}%
\pgfsetstrokecolor{currentstroke}%
\pgfsetdash{}{0pt}%
\pgfpathmoveto{\pgfqpoint{0.800000in}{3.941720in}}%
\pgfpathlineto{\pgfqpoint{5.760000in}{3.941720in}}%
\pgfusepath{stroke}%
\end{pgfscope}%
\begin{pgfscope}%
\pgfsetbuttcap%
\pgfsetroundjoin%
\definecolor{currentfill}{rgb}{0.000000,0.000000,0.000000}%
\pgfsetfillcolor{currentfill}%
\pgfsetlinewidth{0.803000pt}%
\definecolor{currentstroke}{rgb}{0.000000,0.000000,0.000000}%
\pgfsetstrokecolor{currentstroke}%
\pgfsetdash{}{0pt}%
\pgfsys@defobject{currentmarker}{\pgfqpoint{-0.048611in}{0.000000in}}{\pgfqpoint{-0.000000in}{0.000000in}}{%
\pgfpathmoveto{\pgfqpoint{-0.000000in}{0.000000in}}%
\pgfpathlineto{\pgfqpoint{-0.048611in}{0.000000in}}%
\pgfusepath{stroke,fill}%
}%
\begin{pgfscope}%
\pgfsys@transformshift{0.800000in}{3.941720in}%
\pgfsys@useobject{currentmarker}{}%
\end{pgfscope}%
\end{pgfscope}%
\begin{pgfscope}%
\definecolor{textcolor}{rgb}{0.000000,0.000000,0.000000}%
\pgfsetstrokecolor{textcolor}%
\pgfsetfillcolor{textcolor}%
\pgftext[x=0.393533in, y=3.888958in, left, base]{\color{textcolor}\sffamily\fontsize{10.000000}{12.000000}\selectfont 0.05}%
\end{pgfscope}%
\begin{pgfscope}%
\definecolor{textcolor}{rgb}{0.000000,0.000000,0.000000}%
\pgfsetstrokecolor{textcolor}%
\pgfsetfillcolor{textcolor}%
\pgftext[x=0.229952in,y=2.376000in,,bottom,rotate=90.000000]{\color{textcolor}\sffamily\fontsize{10.000000}{12.000000}\selectfont Computed error [-]}%
\end{pgfscope}%
\begin{pgfscope}%
\pgfpathrectangle{\pgfqpoint{0.800000in}{0.528000in}}{\pgfqpoint{4.960000in}{3.696000in}}%
\pgfusepath{clip}%
\pgfsetrectcap%
\pgfsetroundjoin%
\pgfsetlinewidth{1.505625pt}%
\definecolor{currentstroke}{rgb}{0.121569,0.466667,0.705882}%
\pgfsetstrokecolor{currentstroke}%
\pgfsetdash{}{0pt}%
\pgfpathmoveto{\pgfqpoint{1.025455in}{0.789673in}}%
\pgfpathlineto{\pgfqpoint{1.079789in}{0.950567in}}%
\pgfpathlineto{\pgfqpoint{1.135150in}{0.845159in}}%
\pgfpathlineto{\pgfqpoint{1.189142in}{0.820096in}}%
\pgfpathlineto{\pgfqpoint{1.243683in}{1.067883in}}%
\pgfpathlineto{\pgfqpoint{1.299666in}{1.483182in}}%
\pgfpathlineto{\pgfqpoint{1.351799in}{2.325746in}}%
\pgfpathlineto{\pgfqpoint{1.405541in}{2.805047in}}%
\pgfpathlineto{\pgfqpoint{1.459709in}{3.413207in}}%
\pgfpathlineto{\pgfqpoint{1.514300in}{3.225394in}}%
\pgfpathlineto{\pgfqpoint{1.568274in}{2.977620in}}%
\pgfpathlineto{\pgfqpoint{1.624677in}{3.066249in}}%
\pgfpathlineto{\pgfqpoint{1.679007in}{3.221147in}}%
\pgfpathlineto{\pgfqpoint{1.733292in}{3.262181in}}%
\pgfpathlineto{\pgfqpoint{1.787480in}{3.266514in}}%
\pgfpathlineto{\pgfqpoint{1.844118in}{3.256729in}}%
\pgfpathlineto{\pgfqpoint{1.896952in}{3.232663in}}%
\pgfpathlineto{\pgfqpoint{1.952236in}{3.219953in}}%
\pgfpathlineto{\pgfqpoint{2.005498in}{3.217143in}}%
\pgfpathlineto{\pgfqpoint{2.060533in}{3.248926in}}%
\pgfpathlineto{\pgfqpoint{2.114225in}{3.271072in}}%
\pgfpathlineto{\pgfqpoint{2.167964in}{3.305619in}}%
\pgfpathlineto{\pgfqpoint{2.223511in}{3.312494in}}%
\pgfpathlineto{\pgfqpoint{2.276562in}{3.289065in}}%
\pgfpathlineto{\pgfqpoint{2.331563in}{3.288455in}}%
\pgfpathlineto{\pgfqpoint{2.386619in}{3.282400in}}%
\pgfpathlineto{\pgfqpoint{2.439989in}{3.272196in}}%
\pgfpathlineto{\pgfqpoint{2.494745in}{3.266637in}}%
\pgfpathlineto{\pgfqpoint{2.549359in}{3.275576in}}%
\pgfpathlineto{\pgfqpoint{2.602881in}{3.251761in}}%
\pgfpathlineto{\pgfqpoint{2.657497in}{3.245606in}}%
\pgfpathlineto{\pgfqpoint{2.711859in}{3.248692in}}%
\pgfpathlineto{\pgfqpoint{2.767421in}{3.243391in}}%
\pgfpathlineto{\pgfqpoint{2.822550in}{3.239646in}}%
\pgfpathlineto{\pgfqpoint{2.875066in}{3.237770in}}%
\pgfpathlineto{\pgfqpoint{2.928644in}{3.227534in}}%
\pgfpathlineto{\pgfqpoint{2.982160in}{3.241572in}}%
\pgfpathlineto{\pgfqpoint{3.035876in}{3.297209in}}%
\pgfpathlineto{\pgfqpoint{3.090532in}{3.329484in}}%
\pgfpathlineto{\pgfqpoint{3.144546in}{3.353738in}}%
\pgfpathlineto{\pgfqpoint{3.200570in}{3.375311in}}%
\pgfpathlineto{\pgfqpoint{3.253349in}{3.360207in}}%
\pgfpathlineto{\pgfqpoint{3.307357in}{3.345455in}}%
\pgfpathlineto{\pgfqpoint{3.362385in}{3.335889in}}%
\pgfpathlineto{\pgfqpoint{3.416454in}{3.321767in}}%
\pgfpathlineto{\pgfqpoint{3.470642in}{3.283522in}}%
\pgfpathlineto{\pgfqpoint{3.524714in}{3.229213in}}%
\pgfpathlineto{\pgfqpoint{3.579145in}{3.175734in}}%
\pgfpathlineto{\pgfqpoint{3.633040in}{3.181520in}}%
\pgfpathlineto{\pgfqpoint{3.687614in}{3.198075in}}%
\pgfpathlineto{\pgfqpoint{3.741880in}{3.176202in}}%
\pgfpathlineto{\pgfqpoint{3.796084in}{3.185624in}}%
\pgfpathlineto{\pgfqpoint{3.851834in}{3.190127in}}%
\pgfpathlineto{\pgfqpoint{3.905705in}{3.219350in}}%
\pgfpathlineto{\pgfqpoint{3.959606in}{3.339962in}}%
\pgfpathlineto{\pgfqpoint{4.015050in}{3.378235in}}%
\pgfpathlineto{\pgfqpoint{4.068056in}{3.369481in}}%
\pgfpathlineto{\pgfqpoint{4.122407in}{3.506423in}}%
\pgfpathlineto{\pgfqpoint{4.176246in}{3.442111in}}%
\pgfpathlineto{\pgfqpoint{4.230670in}{3.216211in}}%
\pgfpathlineto{\pgfqpoint{4.284415in}{3.267618in}}%
\pgfpathlineto{\pgfqpoint{4.339474in}{3.269464in}}%
\pgfpathlineto{\pgfqpoint{4.395105in}{3.288863in}}%
\pgfpathlineto{\pgfqpoint{4.448722in}{3.274381in}}%
\pgfpathlineto{\pgfqpoint{4.502698in}{3.280456in}}%
\pgfpathlineto{\pgfqpoint{4.556582in}{3.261171in}}%
\pgfpathlineto{\pgfqpoint{4.610820in}{3.282273in}}%
\pgfpathlineto{\pgfqpoint{4.665054in}{3.278877in}}%
\pgfpathlineto{\pgfqpoint{4.719361in}{3.279440in}}%
\pgfpathlineto{\pgfqpoint{4.773076in}{3.269990in}}%
\pgfpathlineto{\pgfqpoint{4.827357in}{3.289705in}}%
\pgfpathlineto{\pgfqpoint{4.883545in}{3.342001in}}%
\pgfpathlineto{\pgfqpoint{4.936847in}{3.516190in}}%
\pgfpathlineto{\pgfqpoint{4.990680in}{3.352123in}}%
\pgfpathlineto{\pgfqpoint{5.044635in}{3.323972in}}%
\pgfpathlineto{\pgfqpoint{5.098891in}{3.291947in}}%
\pgfpathlineto{\pgfqpoint{5.153310in}{3.278248in}}%
\pgfpathlineto{\pgfqpoint{5.207160in}{3.256566in}}%
\pgfpathlineto{\pgfqpoint{5.261262in}{3.245910in}}%
\pgfpathlineto{\pgfqpoint{5.315747in}{3.243489in}}%
\pgfpathlineto{\pgfqpoint{5.370090in}{3.246052in}}%
\pgfpathlineto{\pgfqpoint{5.424768in}{3.237258in}}%
\pgfpathlineto{\pgfqpoint{5.481781in}{3.284848in}}%
\pgfpathlineto{\pgfqpoint{5.534539in}{3.373071in}}%
\pgfusepath{stroke}%
\end{pgfscope}%
\begin{pgfscope}%
\pgfpathrectangle{\pgfqpoint{0.800000in}{0.528000in}}{\pgfqpoint{4.960000in}{3.696000in}}%
\pgfusepath{clip}%
\pgfsetrectcap%
\pgfsetroundjoin%
\pgfsetlinewidth{1.505625pt}%
\definecolor{currentstroke}{rgb}{1.000000,0.498039,0.054902}%
\pgfsetstrokecolor{currentstroke}%
\pgfsetdash{}{0pt}%
\pgfpathmoveto{\pgfqpoint{1.025455in}{0.971742in}}%
\pgfpathlineto{\pgfqpoint{1.080472in}{0.968963in}}%
\pgfpathlineto{\pgfqpoint{1.134899in}{0.935158in}}%
\pgfpathlineto{\pgfqpoint{1.188629in}{1.054694in}}%
\pgfpathlineto{\pgfqpoint{1.243261in}{1.363399in}}%
\pgfpathlineto{\pgfqpoint{1.299328in}{2.165936in}}%
\pgfpathlineto{\pgfqpoint{1.351473in}{2.517895in}}%
\pgfpathlineto{\pgfqpoint{1.405475in}{2.789073in}}%
\pgfpathlineto{\pgfqpoint{1.459672in}{3.211604in}}%
\pgfpathlineto{\pgfqpoint{1.513726in}{3.364809in}}%
\pgfpathlineto{\pgfqpoint{1.567885in}{3.453788in}}%
\pgfpathlineto{\pgfqpoint{1.622461in}{3.540322in}}%
\pgfpathlineto{\pgfqpoint{1.676777in}{3.633968in}}%
\pgfpathlineto{\pgfqpoint{1.732580in}{3.654861in}}%
\pgfpathlineto{\pgfqpoint{1.785968in}{3.683937in}}%
\pgfpathlineto{\pgfqpoint{1.839810in}{3.696532in}}%
\pgfpathlineto{\pgfqpoint{1.895430in}{3.670759in}}%
\pgfpathlineto{\pgfqpoint{1.949144in}{3.633259in}}%
\pgfpathlineto{\pgfqpoint{2.003118in}{3.596730in}}%
\pgfpathlineto{\pgfqpoint{2.057395in}{3.631691in}}%
\pgfpathlineto{\pgfqpoint{2.111905in}{3.632253in}}%
\pgfpathlineto{\pgfqpoint{2.166086in}{3.623209in}}%
\pgfpathlineto{\pgfqpoint{2.220441in}{3.596853in}}%
\pgfpathlineto{\pgfqpoint{2.274857in}{3.587251in}}%
\pgfpathlineto{\pgfqpoint{2.329238in}{3.582080in}}%
\pgfpathlineto{\pgfqpoint{2.383534in}{3.570390in}}%
\pgfpathlineto{\pgfqpoint{2.437778in}{3.566775in}}%
\pgfpathlineto{\pgfqpoint{2.493618in}{3.569294in}}%
\pgfpathlineto{\pgfqpoint{2.546529in}{3.564592in}}%
\pgfpathlineto{\pgfqpoint{2.600779in}{3.544797in}}%
\pgfpathlineto{\pgfqpoint{2.654856in}{3.549039in}}%
\pgfpathlineto{\pgfqpoint{2.708956in}{3.542758in}}%
\pgfpathlineto{\pgfqpoint{2.763130in}{3.530934in}}%
\pgfpathlineto{\pgfqpoint{2.818699in}{3.520968in}}%
\pgfpathlineto{\pgfqpoint{2.873475in}{3.513559in}}%
\pgfpathlineto{\pgfqpoint{2.927831in}{3.504186in}}%
\pgfpathlineto{\pgfqpoint{2.982487in}{3.496911in}}%
\pgfpathlineto{\pgfqpoint{3.036270in}{3.478136in}}%
\pgfpathlineto{\pgfqpoint{3.091407in}{3.468678in}}%
\pgfpathlineto{\pgfqpoint{3.144716in}{3.465575in}}%
\pgfpathlineto{\pgfqpoint{3.198950in}{3.464548in}}%
\pgfpathlineto{\pgfqpoint{3.252844in}{3.465137in}}%
\pgfpathlineto{\pgfqpoint{3.307327in}{3.468039in}}%
\pgfpathlineto{\pgfqpoint{3.361373in}{3.454820in}}%
\pgfpathlineto{\pgfqpoint{3.415995in}{3.436215in}}%
\pgfpathlineto{\pgfqpoint{3.470101in}{3.420665in}}%
\pgfpathlineto{\pgfqpoint{3.524627in}{3.414010in}}%
\pgfpathlineto{\pgfqpoint{3.578793in}{3.409269in}}%
\pgfpathlineto{\pgfqpoint{3.633143in}{3.401585in}}%
\pgfpathlineto{\pgfqpoint{3.688462in}{3.399286in}}%
\pgfpathlineto{\pgfqpoint{3.742550in}{3.392526in}}%
\pgfpathlineto{\pgfqpoint{3.796250in}{3.374910in}}%
\pgfpathlineto{\pgfqpoint{3.850639in}{3.371080in}}%
\pgfpathlineto{\pgfqpoint{3.905218in}{3.357029in}}%
\pgfpathlineto{\pgfqpoint{3.959371in}{3.338785in}}%
\pgfpathlineto{\pgfqpoint{4.013749in}{3.328558in}}%
\pgfpathlineto{\pgfqpoint{4.068332in}{3.330927in}}%
\pgfpathlineto{\pgfqpoint{4.122103in}{3.326837in}}%
\pgfpathlineto{\pgfqpoint{4.177258in}{3.321638in}}%
\pgfpathlineto{\pgfqpoint{4.231580in}{3.325202in}}%
\pgfpathlineto{\pgfqpoint{4.285587in}{3.331861in}}%
\pgfpathlineto{\pgfqpoint{4.339574in}{3.340104in}}%
\pgfpathlineto{\pgfqpoint{4.393761in}{3.345788in}}%
\pgfpathlineto{\pgfqpoint{4.447979in}{3.366590in}}%
\pgfpathlineto{\pgfqpoint{4.502243in}{3.401025in}}%
\pgfpathlineto{\pgfqpoint{4.557236in}{3.407371in}}%
\pgfpathlineto{\pgfqpoint{4.611235in}{3.425161in}}%
\pgfpathlineto{\pgfqpoint{4.666785in}{3.423488in}}%
\pgfpathlineto{\pgfqpoint{4.719813in}{3.421668in}}%
\pgfpathlineto{\pgfqpoint{4.774165in}{3.406374in}}%
\pgfpathlineto{\pgfqpoint{4.828253in}{3.390050in}}%
\pgfpathlineto{\pgfqpoint{4.882612in}{3.365528in}}%
\pgfpathlineto{\pgfqpoint{4.936862in}{3.327640in}}%
\pgfpathlineto{\pgfqpoint{4.991280in}{3.338590in}}%
\pgfpathlineto{\pgfqpoint{5.045331in}{3.304867in}}%
\pgfpathlineto{\pgfqpoint{5.100650in}{3.297622in}}%
\pgfpathlineto{\pgfqpoint{5.155064in}{3.303159in}}%
\pgfpathlineto{\pgfqpoint{5.209813in}{3.292365in}}%
\pgfpathlineto{\pgfqpoint{5.265011in}{3.286738in}}%
\pgfpathlineto{\pgfqpoint{5.318305in}{3.282453in}}%
\pgfpathlineto{\pgfqpoint{5.372376in}{3.283784in}}%
\pgfpathlineto{\pgfqpoint{5.426214in}{3.286545in}}%
\pgfpathlineto{\pgfqpoint{5.480431in}{3.288139in}}%
\pgfpathlineto{\pgfqpoint{5.534545in}{3.289826in}}%
\pgfusepath{stroke}%
\end{pgfscope}%
\begin{pgfscope}%
\pgfpathrectangle{\pgfqpoint{0.800000in}{0.528000in}}{\pgfqpoint{4.960000in}{3.696000in}}%
\pgfusepath{clip}%
\pgfsetrectcap%
\pgfsetroundjoin%
\pgfsetlinewidth{1.505625pt}%
\definecolor{currentstroke}{rgb}{0.172549,0.627451,0.172549}%
\pgfsetstrokecolor{currentstroke}%
\pgfsetdash{}{0pt}%
\pgfpathmoveto{\pgfqpoint{1.025455in}{0.696000in}}%
\pgfpathlineto{\pgfqpoint{1.079338in}{0.816743in}}%
\pgfpathlineto{\pgfqpoint{1.134373in}{0.804336in}}%
\pgfpathlineto{\pgfqpoint{1.188140in}{0.997650in}}%
\pgfpathlineto{\pgfqpoint{1.242541in}{1.096552in}}%
\pgfpathlineto{\pgfqpoint{1.296558in}{1.614435in}}%
\pgfpathlineto{\pgfqpoint{1.351086in}{2.352667in}}%
\pgfpathlineto{\pgfqpoint{1.403899in}{2.817437in}}%
\pgfpathlineto{\pgfqpoint{1.457826in}{3.016953in}}%
\pgfpathlineto{\pgfqpoint{1.513266in}{3.598308in}}%
\pgfpathlineto{\pgfqpoint{1.566904in}{3.478220in}}%
\pgfpathlineto{\pgfqpoint{1.620778in}{3.539856in}}%
\pgfpathlineto{\pgfqpoint{1.674915in}{3.740487in}}%
\pgfpathlineto{\pgfqpoint{1.729322in}{3.853295in}}%
\pgfpathlineto{\pgfqpoint{1.783620in}{3.925874in}}%
\pgfpathlineto{\pgfqpoint{1.837657in}{3.991807in}}%
\pgfpathlineto{\pgfqpoint{1.891863in}{4.056000in}}%
\pgfpathlineto{\pgfqpoint{1.946086in}{4.013145in}}%
\pgfpathlineto{\pgfqpoint{2.000445in}{3.965658in}}%
\pgfpathlineto{\pgfqpoint{2.054578in}{3.904682in}}%
\pgfpathlineto{\pgfqpoint{2.110349in}{3.897496in}}%
\pgfpathlineto{\pgfqpoint{2.164018in}{3.888925in}}%
\pgfpathlineto{\pgfqpoint{2.217545in}{3.896474in}}%
\pgfpathlineto{\pgfqpoint{2.271883in}{3.921055in}}%
\pgfpathlineto{\pgfqpoint{2.325902in}{3.903380in}}%
\pgfpathlineto{\pgfqpoint{2.380461in}{3.860110in}}%
\pgfpathlineto{\pgfqpoint{2.435017in}{3.844291in}}%
\pgfpathlineto{\pgfqpoint{2.489119in}{3.845812in}}%
\pgfpathlineto{\pgfqpoint{2.544546in}{3.800994in}}%
\pgfpathlineto{\pgfqpoint{2.600533in}{3.749685in}}%
\pgfpathlineto{\pgfqpoint{2.654479in}{3.675440in}}%
\pgfpathlineto{\pgfqpoint{2.707662in}{3.635435in}}%
\pgfpathlineto{\pgfqpoint{2.762151in}{3.573329in}}%
\pgfpathlineto{\pgfqpoint{2.816214in}{3.564322in}}%
\pgfpathlineto{\pgfqpoint{2.870221in}{3.543801in}}%
\pgfpathlineto{\pgfqpoint{2.924444in}{3.606672in}}%
\pgfpathlineto{\pgfqpoint{2.978898in}{3.593597in}}%
\pgfpathlineto{\pgfqpoint{3.033114in}{3.572174in}}%
\pgfpathlineto{\pgfqpoint{3.087068in}{3.565503in}}%
\pgfpathlineto{\pgfqpoint{3.141341in}{3.561899in}}%
\pgfpathlineto{\pgfqpoint{3.195661in}{3.571735in}}%
\pgfpathlineto{\pgfqpoint{3.250331in}{3.487004in}}%
\pgfpathlineto{\pgfqpoint{3.304263in}{3.296200in}}%
\pgfpathlineto{\pgfqpoint{3.358421in}{3.104642in}}%
\pgfpathlineto{\pgfqpoint{3.412925in}{3.216272in}}%
\pgfpathlineto{\pgfqpoint{3.468686in}{3.247320in}}%
\pgfpathlineto{\pgfqpoint{3.521890in}{3.198513in}}%
\pgfpathlineto{\pgfqpoint{3.575593in}{3.198746in}}%
\pgfpathlineto{\pgfqpoint{3.630199in}{3.125944in}}%
\pgfpathlineto{\pgfqpoint{3.684605in}{3.170322in}}%
\pgfpathlineto{\pgfqpoint{3.738863in}{3.174892in}}%
\pgfpathlineto{\pgfqpoint{3.793823in}{3.174691in}}%
\pgfpathlineto{\pgfqpoint{3.847600in}{3.175180in}}%
\pgfpathlineto{\pgfqpoint{3.901804in}{3.181715in}}%
\pgfpathlineto{\pgfqpoint{3.956167in}{3.183997in}}%
\pgfpathlineto{\pgfqpoint{4.012325in}{3.198692in}}%
\pgfpathlineto{\pgfqpoint{4.065420in}{3.266989in}}%
\pgfpathlineto{\pgfqpoint{4.119116in}{3.269787in}}%
\pgfpathlineto{\pgfqpoint{4.173526in}{3.239530in}}%
\pgfpathlineto{\pgfqpoint{4.228117in}{3.233853in}}%
\pgfpathlineto{\pgfqpoint{4.284293in}{3.244787in}}%
\pgfpathlineto{\pgfqpoint{4.336210in}{3.266524in}}%
\pgfpathlineto{\pgfqpoint{4.390143in}{3.317152in}}%
\pgfpathlineto{\pgfqpoint{4.445557in}{3.364927in}}%
\pgfpathlineto{\pgfqpoint{4.499281in}{3.358034in}}%
\pgfpathlineto{\pgfqpoint{4.554693in}{3.398912in}}%
\pgfpathlineto{\pgfqpoint{4.608759in}{3.420129in}}%
\pgfpathlineto{\pgfqpoint{4.663266in}{3.430952in}}%
\pgfpathlineto{\pgfqpoint{4.716945in}{3.411474in}}%
\pgfpathlineto{\pgfqpoint{4.771111in}{3.313895in}}%
\pgfpathlineto{\pgfqpoint{4.825139in}{3.279835in}}%
\pgfpathlineto{\pgfqpoint{4.879672in}{3.287275in}}%
\pgfpathlineto{\pgfqpoint{4.937695in}{3.265639in}}%
\pgfpathlineto{\pgfqpoint{4.989674in}{3.242180in}}%
\pgfpathlineto{\pgfqpoint{5.043158in}{3.254520in}}%
\pgfpathlineto{\pgfqpoint{5.097306in}{3.235460in}}%
\pgfpathlineto{\pgfqpoint{5.151255in}{3.239525in}}%
\pgfpathlineto{\pgfqpoint{5.205714in}{3.235778in}}%
\pgfpathlineto{\pgfqpoint{5.259536in}{3.238505in}}%
\pgfpathlineto{\pgfqpoint{5.314236in}{3.241857in}}%
\pgfpathlineto{\pgfqpoint{5.368442in}{3.234434in}}%
\pgfpathlineto{\pgfqpoint{5.422763in}{3.221594in}}%
\pgfpathlineto{\pgfqpoint{5.477325in}{3.218135in}}%
\pgfpathlineto{\pgfqpoint{5.531315in}{3.217967in}}%
\pgfusepath{stroke}%
\end{pgfscope}%
\begin{pgfscope}%
\pgfpathrectangle{\pgfqpoint{0.800000in}{0.528000in}}{\pgfqpoint{4.960000in}{3.696000in}}%
\pgfusepath{clip}%
\pgfsetrectcap%
\pgfsetroundjoin%
\pgfsetlinewidth{1.505625pt}%
\definecolor{currentstroke}{rgb}{0.839216,0.152941,0.156863}%
\pgfsetstrokecolor{currentstroke}%
\pgfsetdash{}{0pt}%
\pgfpathmoveto{\pgfqpoint{1.025455in}{0.825961in}}%
\pgfpathlineto{\pgfqpoint{1.079563in}{0.937985in}}%
\pgfpathlineto{\pgfqpoint{1.133745in}{1.036719in}}%
\pgfpathlineto{\pgfqpoint{1.188369in}{0.960604in}}%
\pgfpathlineto{\pgfqpoint{1.242836in}{1.114223in}}%
\pgfpathlineto{\pgfqpoint{1.296352in}{1.412077in}}%
\pgfpathlineto{\pgfqpoint{1.350170in}{2.044298in}}%
\pgfpathlineto{\pgfqpoint{1.405568in}{2.848771in}}%
\pgfpathlineto{\pgfqpoint{1.459476in}{3.120644in}}%
\pgfpathlineto{\pgfqpoint{1.513382in}{3.349057in}}%
\pgfpathlineto{\pgfqpoint{1.567290in}{3.816830in}}%
\pgfpathlineto{\pgfqpoint{1.622195in}{3.703242in}}%
\pgfpathlineto{\pgfqpoint{1.675724in}{3.774446in}}%
\pgfpathlineto{\pgfqpoint{1.730171in}{3.879816in}}%
\pgfpathlineto{\pgfqpoint{1.784643in}{3.896633in}}%
\pgfpathlineto{\pgfqpoint{1.838886in}{3.916416in}}%
\pgfpathlineto{\pgfqpoint{1.894019in}{3.948761in}}%
\pgfpathlineto{\pgfqpoint{1.948113in}{3.943035in}}%
\pgfpathlineto{\pgfqpoint{2.003267in}{3.871849in}}%
\pgfpathlineto{\pgfqpoint{2.057163in}{3.918920in}}%
\pgfpathlineto{\pgfqpoint{2.111040in}{3.873175in}}%
\pgfpathlineto{\pgfqpoint{2.165065in}{3.820665in}}%
\pgfpathlineto{\pgfqpoint{2.219893in}{3.856585in}}%
\pgfpathlineto{\pgfqpoint{2.273806in}{3.832695in}}%
\pgfpathlineto{\pgfqpoint{2.328151in}{3.840154in}}%
\pgfpathlineto{\pgfqpoint{2.382662in}{3.838931in}}%
\pgfpathlineto{\pgfqpoint{2.436713in}{3.596645in}}%
\pgfpathlineto{\pgfqpoint{2.491306in}{3.518179in}}%
\pgfpathlineto{\pgfqpoint{2.545562in}{3.472902in}}%
\pgfpathlineto{\pgfqpoint{2.600972in}{3.439095in}}%
\pgfpathlineto{\pgfqpoint{2.654416in}{3.346317in}}%
\pgfpathlineto{\pgfqpoint{2.708411in}{3.224119in}}%
\pgfpathlineto{\pgfqpoint{2.762550in}{3.031805in}}%
\pgfpathlineto{\pgfqpoint{2.816965in}{2.867867in}}%
\pgfpathlineto{\pgfqpoint{2.871193in}{2.639952in}}%
\pgfpathlineto{\pgfqpoint{2.925747in}{2.468048in}}%
\pgfpathlineto{\pgfqpoint{2.980559in}{2.453351in}}%
\pgfpathlineto{\pgfqpoint{3.034734in}{2.705573in}}%
\pgfpathlineto{\pgfqpoint{3.088855in}{2.937388in}}%
\pgfpathlineto{\pgfqpoint{3.144324in}{3.125234in}}%
\pgfpathlineto{\pgfqpoint{3.199059in}{3.274103in}}%
\pgfpathlineto{\pgfqpoint{3.252618in}{3.687225in}}%
\pgfpathlineto{\pgfqpoint{3.306665in}{3.460040in}}%
\pgfpathlineto{\pgfqpoint{3.360795in}{3.141490in}}%
\pgfpathlineto{\pgfqpoint{3.414952in}{3.426717in}}%
\pgfpathlineto{\pgfqpoint{3.469285in}{3.563945in}}%
\pgfpathlineto{\pgfqpoint{3.523559in}{3.630757in}}%
\pgfpathlineto{\pgfqpoint{3.578142in}{3.609872in}}%
\pgfpathlineto{\pgfqpoint{3.631962in}{3.661310in}}%
\pgfpathlineto{\pgfqpoint{3.686230in}{3.624721in}}%
\pgfpathlineto{\pgfqpoint{3.742106in}{3.634469in}}%
\pgfpathlineto{\pgfqpoint{3.795478in}{3.594682in}}%
\pgfpathlineto{\pgfqpoint{3.849263in}{3.285399in}}%
\pgfpathlineto{\pgfqpoint{3.903426in}{3.338275in}}%
\pgfpathlineto{\pgfqpoint{3.957571in}{3.331321in}}%
\pgfpathlineto{\pgfqpoint{4.011819in}{3.221096in}}%
\pgfpathlineto{\pgfqpoint{4.066240in}{3.089267in}}%
\pgfpathlineto{\pgfqpoint{4.120524in}{3.053702in}}%
\pgfpathlineto{\pgfqpoint{4.174846in}{3.132058in}}%
\pgfpathlineto{\pgfqpoint{4.230263in}{3.175712in}}%
\pgfpathlineto{\pgfqpoint{4.283570in}{3.191651in}}%
\pgfpathlineto{\pgfqpoint{4.337704in}{3.197481in}}%
\pgfpathlineto{\pgfqpoint{4.392219in}{3.147631in}}%
\pgfpathlineto{\pgfqpoint{4.446270in}{3.133473in}}%
\pgfpathlineto{\pgfqpoint{4.500787in}{3.084655in}}%
\pgfpathlineto{\pgfqpoint{4.555204in}{3.053487in}}%
\pgfpathlineto{\pgfqpoint{4.609805in}{3.063717in}}%
\pgfpathlineto{\pgfqpoint{4.664059in}{3.262017in}}%
\pgfpathlineto{\pgfqpoint{4.718797in}{3.289421in}}%
\pgfpathlineto{\pgfqpoint{4.772885in}{3.349909in}}%
\pgfpathlineto{\pgfqpoint{4.828486in}{3.335121in}}%
\pgfpathlineto{\pgfqpoint{4.882605in}{3.350100in}}%
\pgfpathlineto{\pgfqpoint{4.936386in}{3.394788in}}%
\pgfpathlineto{\pgfqpoint{4.990649in}{3.418827in}}%
\pgfpathlineto{\pgfqpoint{5.045036in}{3.382618in}}%
\pgfpathlineto{\pgfqpoint{5.099396in}{3.380312in}}%
\pgfpathlineto{\pgfqpoint{5.153858in}{3.371127in}}%
\pgfpathlineto{\pgfqpoint{5.207915in}{3.387479in}}%
\pgfpathlineto{\pgfqpoint{5.262469in}{3.368995in}}%
\pgfpathlineto{\pgfqpoint{5.316492in}{3.382263in}}%
\pgfpathlineto{\pgfqpoint{5.370799in}{3.367753in}}%
\pgfpathlineto{\pgfqpoint{5.426493in}{3.371726in}}%
\pgfpathlineto{\pgfqpoint{5.480191in}{3.383535in}}%
\pgfpathlineto{\pgfqpoint{5.533861in}{3.380775in}}%
\pgfusepath{stroke}%
\end{pgfscope}%
\begin{pgfscope}%
\pgfpathrectangle{\pgfqpoint{0.800000in}{0.528000in}}{\pgfqpoint{4.960000in}{3.696000in}}%
\pgfusepath{clip}%
\pgfsetrectcap%
\pgfsetroundjoin%
\pgfsetlinewidth{1.505625pt}%
\definecolor{currentstroke}{rgb}{0.580392,0.403922,0.741176}%
\pgfsetstrokecolor{currentstroke}%
\pgfsetdash{}{0pt}%
\pgfpathmoveto{\pgfqpoint{1.025455in}{0.796338in}}%
\pgfpathlineto{\pgfqpoint{1.079158in}{0.892366in}}%
\pgfpathlineto{\pgfqpoint{1.133625in}{0.707252in}}%
\pgfpathlineto{\pgfqpoint{1.187998in}{0.749208in}}%
\pgfpathlineto{\pgfqpoint{1.242592in}{1.076041in}}%
\pgfpathlineto{\pgfqpoint{1.297030in}{1.794030in}}%
\pgfpathlineto{\pgfqpoint{1.351098in}{2.288495in}}%
\pgfpathlineto{\pgfqpoint{1.404136in}{2.671711in}}%
\pgfpathlineto{\pgfqpoint{1.458220in}{2.985294in}}%
\pgfpathlineto{\pgfqpoint{1.512633in}{3.226770in}}%
\pgfpathlineto{\pgfqpoint{1.567117in}{3.427106in}}%
\pgfpathlineto{\pgfqpoint{1.622859in}{3.623368in}}%
\pgfpathlineto{\pgfqpoint{1.676422in}{3.722084in}}%
\pgfpathlineto{\pgfqpoint{1.730222in}{3.833662in}}%
\pgfpathlineto{\pgfqpoint{1.784392in}{3.927077in}}%
\pgfpathlineto{\pgfqpoint{1.838589in}{3.983329in}}%
\pgfpathlineto{\pgfqpoint{1.892759in}{3.976164in}}%
\pgfpathlineto{\pgfqpoint{1.946836in}{3.954978in}}%
\pgfpathlineto{\pgfqpoint{2.001194in}{3.859727in}}%
\pgfpathlineto{\pgfqpoint{2.055235in}{3.855226in}}%
\pgfpathlineto{\pgfqpoint{2.109634in}{3.821862in}}%
\pgfpathlineto{\pgfqpoint{2.163983in}{3.817838in}}%
\pgfpathlineto{\pgfqpoint{2.218154in}{3.657843in}}%
\pgfpathlineto{\pgfqpoint{2.273665in}{3.556769in}}%
\pgfpathlineto{\pgfqpoint{2.327050in}{3.403866in}}%
\pgfpathlineto{\pgfqpoint{2.381143in}{3.202208in}}%
\pgfpathlineto{\pgfqpoint{2.435175in}{3.303599in}}%
\pgfpathlineto{\pgfqpoint{2.489389in}{3.275107in}}%
\pgfpathlineto{\pgfqpoint{2.543680in}{3.197163in}}%
\pgfpathlineto{\pgfqpoint{2.598429in}{3.185825in}}%
\pgfpathlineto{\pgfqpoint{2.652754in}{3.131539in}}%
\pgfpathlineto{\pgfqpoint{2.706983in}{2.967604in}}%
\pgfpathlineto{\pgfqpoint{2.761159in}{2.992330in}}%
\pgfpathlineto{\pgfqpoint{2.815621in}{2.996831in}}%
\pgfpathlineto{\pgfqpoint{2.870024in}{2.940640in}}%
\pgfpathlineto{\pgfqpoint{2.925271in}{2.897037in}}%
\pgfpathlineto{\pgfqpoint{2.978818in}{2.874099in}}%
\pgfpathlineto{\pgfqpoint{3.035845in}{2.921823in}}%
\pgfpathlineto{\pgfqpoint{3.087551in}{2.927009in}}%
\pgfpathlineto{\pgfqpoint{3.142088in}{2.951939in}}%
\pgfpathlineto{\pgfqpoint{3.196778in}{3.015940in}}%
\pgfpathlineto{\pgfqpoint{3.250727in}{3.173121in}}%
\pgfpathlineto{\pgfqpoint{3.305079in}{3.312161in}}%
\pgfpathlineto{\pgfqpoint{3.359364in}{3.358624in}}%
\pgfpathlineto{\pgfqpoint{3.414288in}{3.490660in}}%
\pgfpathlineto{\pgfqpoint{3.468158in}{3.565789in}}%
\pgfpathlineto{\pgfqpoint{3.523353in}{3.568292in}}%
\pgfpathlineto{\pgfqpoint{3.577287in}{3.540692in}}%
\pgfpathlineto{\pgfqpoint{3.631162in}{3.556205in}}%
\pgfpathlineto{\pgfqpoint{3.685684in}{3.586136in}}%
\pgfpathlineto{\pgfqpoint{3.739819in}{3.593842in}}%
\pgfpathlineto{\pgfqpoint{3.793889in}{3.548748in}}%
\pgfpathlineto{\pgfqpoint{3.849315in}{3.503794in}}%
\pgfpathlineto{\pgfqpoint{3.903591in}{3.522961in}}%
\pgfpathlineto{\pgfqpoint{3.957883in}{3.419142in}}%
\pgfpathlineto{\pgfqpoint{4.012780in}{3.406195in}}%
\pgfpathlineto{\pgfqpoint{4.066745in}{3.416333in}}%
\pgfpathlineto{\pgfqpoint{4.120934in}{3.436356in}}%
\pgfpathlineto{\pgfqpoint{4.176212in}{3.399010in}}%
\pgfpathlineto{\pgfqpoint{4.229959in}{3.453859in}}%
\pgfpathlineto{\pgfqpoint{4.283907in}{3.409335in}}%
\pgfpathlineto{\pgfqpoint{4.337780in}{3.153699in}}%
\pgfpathlineto{\pgfqpoint{4.392392in}{3.084749in}}%
\pgfpathlineto{\pgfqpoint{4.446739in}{3.066577in}}%
\pgfpathlineto{\pgfqpoint{4.501050in}{3.043134in}}%
\pgfpathlineto{\pgfqpoint{4.555178in}{2.994759in}}%
\pgfpathlineto{\pgfqpoint{4.611013in}{3.046532in}}%
\pgfpathlineto{\pgfqpoint{4.664073in}{3.075207in}}%
\pgfpathlineto{\pgfqpoint{4.718174in}{3.101483in}}%
\pgfpathlineto{\pgfqpoint{4.772397in}{3.096143in}}%
\pgfpathlineto{\pgfqpoint{4.826437in}{3.096509in}}%
\pgfpathlineto{\pgfqpoint{4.880571in}{3.117578in}}%
\pgfpathlineto{\pgfqpoint{4.935390in}{3.324135in}}%
\pgfpathlineto{\pgfqpoint{4.989268in}{3.298656in}}%
\pgfpathlineto{\pgfqpoint{5.044055in}{3.227457in}}%
\pgfpathlineto{\pgfqpoint{5.098236in}{3.326751in}}%
\pgfpathlineto{\pgfqpoint{5.152541in}{3.492943in}}%
\pgfpathlineto{\pgfqpoint{5.206757in}{3.516904in}}%
\pgfpathlineto{\pgfqpoint{5.262378in}{3.438672in}}%
\pgfpathlineto{\pgfqpoint{5.316255in}{3.435843in}}%
\pgfpathlineto{\pgfqpoint{5.370230in}{3.428148in}}%
\pgfpathlineto{\pgfqpoint{5.424284in}{3.441038in}}%
\pgfpathlineto{\pgfqpoint{5.478769in}{3.437268in}}%
\pgfpathlineto{\pgfqpoint{5.532911in}{3.444029in}}%
\pgfusepath{stroke}%
\end{pgfscope}%
\begin{pgfscope}%
\pgfsetrectcap%
\pgfsetmiterjoin%
\pgfsetlinewidth{0.803000pt}%
\definecolor{currentstroke}{rgb}{0.000000,0.000000,0.000000}%
\pgfsetstrokecolor{currentstroke}%
\pgfsetdash{}{0pt}%
\pgfpathmoveto{\pgfqpoint{0.800000in}{0.528000in}}%
\pgfpathlineto{\pgfqpoint{0.800000in}{4.224000in}}%
\pgfusepath{stroke}%
\end{pgfscope}%
\begin{pgfscope}%
\pgfsetrectcap%
\pgfsetmiterjoin%
\pgfsetlinewidth{0.803000pt}%
\definecolor{currentstroke}{rgb}{0.000000,0.000000,0.000000}%
\pgfsetstrokecolor{currentstroke}%
\pgfsetdash{}{0pt}%
\pgfpathmoveto{\pgfqpoint{5.760000in}{0.528000in}}%
\pgfpathlineto{\pgfqpoint{5.760000in}{4.224000in}}%
\pgfusepath{stroke}%
\end{pgfscope}%
\begin{pgfscope}%
\pgfsetrectcap%
\pgfsetmiterjoin%
\pgfsetlinewidth{0.803000pt}%
\definecolor{currentstroke}{rgb}{0.000000,0.000000,0.000000}%
\pgfsetstrokecolor{currentstroke}%
\pgfsetdash{}{0pt}%
\pgfpathmoveto{\pgfqpoint{0.800000in}{0.528000in}}%
\pgfpathlineto{\pgfqpoint{5.760000in}{0.528000in}}%
\pgfusepath{stroke}%
\end{pgfscope}%
\begin{pgfscope}%
\pgfsetrectcap%
\pgfsetmiterjoin%
\pgfsetlinewidth{0.803000pt}%
\definecolor{currentstroke}{rgb}{0.000000,0.000000,0.000000}%
\pgfsetstrokecolor{currentstroke}%
\pgfsetdash{}{0pt}%
\pgfpathmoveto{\pgfqpoint{0.800000in}{4.224000in}}%
\pgfpathlineto{\pgfqpoint{5.760000in}{4.224000in}}%
\pgfusepath{stroke}%
\end{pgfscope}%
\begin{pgfscope}%
\definecolor{textcolor}{rgb}{0.000000,0.000000,0.000000}%
\pgfsetstrokecolor{textcolor}%
\pgfsetfillcolor{textcolor}%
\pgftext[x=3.280000in,y=4.307333in,,base]{\color{textcolor}\sffamily\fontsize{12.000000}{14.400000}\selectfont Forward controller input}%
\end{pgfscope}%
\begin{pgfscope}%
\pgfsetbuttcap%
\pgfsetmiterjoin%
\definecolor{currentfill}{rgb}{1.000000,1.000000,1.000000}%
\pgfsetfillcolor{currentfill}%
\pgfsetfillopacity{0.800000}%
\pgfsetlinewidth{1.003750pt}%
\definecolor{currentstroke}{rgb}{0.800000,0.800000,0.800000}%
\pgfsetstrokecolor{currentstroke}%
\pgfsetstrokeopacity{0.800000}%
\pgfsetdash{}{0pt}%
\pgfpathmoveto{\pgfqpoint{4.997454in}{0.597444in}}%
\pgfpathlineto{\pgfqpoint{5.662778in}{0.597444in}}%
\pgfpathquadraticcurveto{\pgfqpoint{5.690556in}{0.597444in}}{\pgfqpoint{5.690556in}{0.625222in}}%
\pgfpathlineto{\pgfqpoint{5.690556in}{1.630619in}}%
\pgfpathquadraticcurveto{\pgfqpoint{5.690556in}{1.658397in}}{\pgfqpoint{5.662778in}{1.658397in}}%
\pgfpathlineto{\pgfqpoint{4.997454in}{1.658397in}}%
\pgfpathquadraticcurveto{\pgfqpoint{4.969676in}{1.658397in}}{\pgfqpoint{4.969676in}{1.630619in}}%
\pgfpathlineto{\pgfqpoint{4.969676in}{0.625222in}}%
\pgfpathquadraticcurveto{\pgfqpoint{4.969676in}{0.597444in}}{\pgfqpoint{4.997454in}{0.597444in}}%
\pgfpathlineto{\pgfqpoint{4.997454in}{0.597444in}}%
\pgfpathclose%
\pgfusepath{stroke,fill}%
\end{pgfscope}%
\begin{pgfscope}%
\pgfsetrectcap%
\pgfsetroundjoin%
\pgfsetlinewidth{1.505625pt}%
\definecolor{currentstroke}{rgb}{0.121569,0.466667,0.705882}%
\pgfsetstrokecolor{currentstroke}%
\pgfsetdash{}{0pt}%
\pgfpathmoveto{\pgfqpoint{5.025232in}{1.545930in}}%
\pgfpathlineto{\pgfqpoint{5.164121in}{1.545930in}}%
\pgfpathlineto{\pgfqpoint{5.303009in}{1.545930in}}%
\pgfusepath{stroke}%
\end{pgfscope}%
\begin{pgfscope}%
\definecolor{textcolor}{rgb}{0.000000,0.000000,0.000000}%
\pgfsetstrokecolor{textcolor}%
\pgfsetfillcolor{textcolor}%
\pgftext[x=5.414121in,y=1.497319in,left,base]{\color{textcolor}\sffamily\fontsize{10.000000}{12.000000}\selectfont 0}%
\end{pgfscope}%
\begin{pgfscope}%
\pgfsetrectcap%
\pgfsetroundjoin%
\pgfsetlinewidth{1.505625pt}%
\definecolor{currentstroke}{rgb}{1.000000,0.498039,0.054902}%
\pgfsetstrokecolor{currentstroke}%
\pgfsetdash{}{0pt}%
\pgfpathmoveto{\pgfqpoint{5.025232in}{1.342073in}}%
\pgfpathlineto{\pgfqpoint{5.164121in}{1.342073in}}%
\pgfpathlineto{\pgfqpoint{5.303009in}{1.342073in}}%
\pgfusepath{stroke}%
\end{pgfscope}%
\begin{pgfscope}%
\definecolor{textcolor}{rgb}{0.000000,0.000000,0.000000}%
\pgfsetstrokecolor{textcolor}%
\pgfsetfillcolor{textcolor}%
\pgftext[x=5.414121in,y=1.293461in,left,base]{\color{textcolor}\sffamily\fontsize{10.000000}{12.000000}\selectfont 0.5}%
\end{pgfscope}%
\begin{pgfscope}%
\pgfsetrectcap%
\pgfsetroundjoin%
\pgfsetlinewidth{1.505625pt}%
\definecolor{currentstroke}{rgb}{0.172549,0.627451,0.172549}%
\pgfsetstrokecolor{currentstroke}%
\pgfsetdash{}{0pt}%
\pgfpathmoveto{\pgfqpoint{5.025232in}{1.138215in}}%
\pgfpathlineto{\pgfqpoint{5.164121in}{1.138215in}}%
\pgfpathlineto{\pgfqpoint{5.303009in}{1.138215in}}%
\pgfusepath{stroke}%
\end{pgfscope}%
\begin{pgfscope}%
\definecolor{textcolor}{rgb}{0.000000,0.000000,0.000000}%
\pgfsetstrokecolor{textcolor}%
\pgfsetfillcolor{textcolor}%
\pgftext[x=5.414121in,y=1.089604in,left,base]{\color{textcolor}\sffamily\fontsize{10.000000}{12.000000}\selectfont 1}%
\end{pgfscope}%
\begin{pgfscope}%
\pgfsetrectcap%
\pgfsetroundjoin%
\pgfsetlinewidth{1.505625pt}%
\definecolor{currentstroke}{rgb}{0.839216,0.152941,0.156863}%
\pgfsetstrokecolor{currentstroke}%
\pgfsetdash{}{0pt}%
\pgfpathmoveto{\pgfqpoint{5.025232in}{0.934358in}}%
\pgfpathlineto{\pgfqpoint{5.164121in}{0.934358in}}%
\pgfpathlineto{\pgfqpoint{5.303009in}{0.934358in}}%
\pgfusepath{stroke}%
\end{pgfscope}%
\begin{pgfscope}%
\definecolor{textcolor}{rgb}{0.000000,0.000000,0.000000}%
\pgfsetstrokecolor{textcolor}%
\pgfsetfillcolor{textcolor}%
\pgftext[x=5.414121in,y=0.885747in,left,base]{\color{textcolor}\sffamily\fontsize{10.000000}{12.000000}\selectfont 2}%
\end{pgfscope}%
\begin{pgfscope}%
\pgfsetrectcap%
\pgfsetroundjoin%
\pgfsetlinewidth{1.505625pt}%
\definecolor{currentstroke}{rgb}{0.580392,0.403922,0.741176}%
\pgfsetstrokecolor{currentstroke}%
\pgfsetdash{}{0pt}%
\pgfpathmoveto{\pgfqpoint{5.025232in}{0.730501in}}%
\pgfpathlineto{\pgfqpoint{5.164121in}{0.730501in}}%
\pgfpathlineto{\pgfqpoint{5.303009in}{0.730501in}}%
\pgfusepath{stroke}%
\end{pgfscope}%
\begin{pgfscope}%
\definecolor{textcolor}{rgb}{0.000000,0.000000,0.000000}%
\pgfsetstrokecolor{textcolor}%
\pgfsetfillcolor{textcolor}%
\pgftext[x=5.414121in,y=0.681890in,left,base]{\color{textcolor}\sffamily\fontsize{10.000000}{12.000000}\selectfont 3}%
\end{pgfscope}%
\end{pgfpicture}%
\makeatother%
\endgroup%
}
    \end{minipage}
    \begin{minipage}[t]{0.5\linewidth}
        \centering
        \scalebox{0.55}{%% Creator: Matplotlib, PGF backend
%%
%% To include the figure in your LaTeX document, write
%%   \input{<filename>.pgf}
%%
%% Make sure the required packages are loaded in your preamble
%%   \usepackage{pgf}
%%
%% Also ensure that all the required font packages are loaded; for instance,
%% the lmodern package is sometimes necessary when using math font.
%%   \usepackage{lmodern}
%%
%% Figures using additional raster images can only be included by \input if
%% they are in the same directory as the main LaTeX file. For loading figures
%% from other directories you can use the `import` package
%%   \usepackage{import}
%%
%% and then include the figures with
%%   \import{<path to file>}{<filename>.pgf}
%%
%% Matplotlib used the following preamble
%%   \usepackage{fontspec}
%%   \setmainfont{DejaVuSerif.ttf}[Path=\detokenize{/home/lgonz/tfg-aero/tfg-giaa-dronecontrol/venv/lib/python3.8/site-packages/matplotlib/mpl-data/fonts/ttf/}]
%%   \setsansfont{DejaVuSans.ttf}[Path=\detokenize{/home/lgonz/tfg-aero/tfg-giaa-dronecontrol/venv/lib/python3.8/site-packages/matplotlib/mpl-data/fonts/ttf/}]
%%   \setmonofont{DejaVuSansMono.ttf}[Path=\detokenize{/home/lgonz/tfg-aero/tfg-giaa-dronecontrol/venv/lib/python3.8/site-packages/matplotlib/mpl-data/fonts/ttf/}]
%%
\begingroup%
\makeatletter%
\begin{pgfpicture}%
\pgfpathrectangle{\pgfpointorigin}{\pgfqpoint{6.400000in}{4.800000in}}%
\pgfusepath{use as bounding box, clip}%
\begin{pgfscope}%
\pgfsetbuttcap%
\pgfsetmiterjoin%
\definecolor{currentfill}{rgb}{1.000000,1.000000,1.000000}%
\pgfsetfillcolor{currentfill}%
\pgfsetlinewidth{0.000000pt}%
\definecolor{currentstroke}{rgb}{1.000000,1.000000,1.000000}%
\pgfsetstrokecolor{currentstroke}%
\pgfsetdash{}{0pt}%
\pgfpathmoveto{\pgfqpoint{0.000000in}{0.000000in}}%
\pgfpathlineto{\pgfqpoint{6.400000in}{0.000000in}}%
\pgfpathlineto{\pgfqpoint{6.400000in}{4.800000in}}%
\pgfpathlineto{\pgfqpoint{0.000000in}{4.800000in}}%
\pgfpathlineto{\pgfqpoint{0.000000in}{0.000000in}}%
\pgfpathclose%
\pgfusepath{fill}%
\end{pgfscope}%
\begin{pgfscope}%
\pgfsetbuttcap%
\pgfsetmiterjoin%
\definecolor{currentfill}{rgb}{1.000000,1.000000,1.000000}%
\pgfsetfillcolor{currentfill}%
\pgfsetlinewidth{0.000000pt}%
\definecolor{currentstroke}{rgb}{0.000000,0.000000,0.000000}%
\pgfsetstrokecolor{currentstroke}%
\pgfsetstrokeopacity{0.000000}%
\pgfsetdash{}{0pt}%
\pgfpathmoveto{\pgfqpoint{0.800000in}{0.528000in}}%
\pgfpathlineto{\pgfqpoint{5.760000in}{0.528000in}}%
\pgfpathlineto{\pgfqpoint{5.760000in}{4.224000in}}%
\pgfpathlineto{\pgfqpoint{0.800000in}{4.224000in}}%
\pgfpathlineto{\pgfqpoint{0.800000in}{0.528000in}}%
\pgfpathclose%
\pgfusepath{fill}%
\end{pgfscope}%
\begin{pgfscope}%
\pgfpathrectangle{\pgfqpoint{0.800000in}{0.528000in}}{\pgfqpoint{4.960000in}{3.696000in}}%
\pgfusepath{clip}%
\pgfsetrectcap%
\pgfsetroundjoin%
\pgfsetlinewidth{0.803000pt}%
\definecolor{currentstroke}{rgb}{0.690196,0.690196,0.690196}%
\pgfsetstrokecolor{currentstroke}%
\pgfsetdash{}{0pt}%
\pgfpathmoveto{\pgfqpoint{1.025455in}{0.528000in}}%
\pgfpathlineto{\pgfqpoint{1.025455in}{4.224000in}}%
\pgfusepath{stroke}%
\end{pgfscope}%
\begin{pgfscope}%
\pgfsetbuttcap%
\pgfsetroundjoin%
\definecolor{currentfill}{rgb}{0.000000,0.000000,0.000000}%
\pgfsetfillcolor{currentfill}%
\pgfsetlinewidth{0.803000pt}%
\definecolor{currentstroke}{rgb}{0.000000,0.000000,0.000000}%
\pgfsetstrokecolor{currentstroke}%
\pgfsetdash{}{0pt}%
\pgfsys@defobject{currentmarker}{\pgfqpoint{0.000000in}{-0.048611in}}{\pgfqpoint{0.000000in}{0.000000in}}{%
\pgfpathmoveto{\pgfqpoint{0.000000in}{0.000000in}}%
\pgfpathlineto{\pgfqpoint{0.000000in}{-0.048611in}}%
\pgfusepath{stroke,fill}%
}%
\begin{pgfscope}%
\pgfsys@transformshift{1.025455in}{0.528000in}%
\pgfsys@useobject{currentmarker}{}%
\end{pgfscope}%
\end{pgfscope}%
\begin{pgfscope}%
\definecolor{textcolor}{rgb}{0.000000,0.000000,0.000000}%
\pgfsetstrokecolor{textcolor}%
\pgfsetfillcolor{textcolor}%
\pgftext[x=1.025455in,y=0.430778in,,top]{\color{textcolor}\sffamily\fontsize{10.000000}{12.000000}\selectfont 0}%
\end{pgfscope}%
\begin{pgfscope}%
\pgfpathrectangle{\pgfqpoint{0.800000in}{0.528000in}}{\pgfqpoint{4.960000in}{3.696000in}}%
\pgfusepath{clip}%
\pgfsetrectcap%
\pgfsetroundjoin%
\pgfsetlinewidth{0.803000pt}%
\definecolor{currentstroke}{rgb}{0.690196,0.690196,0.690196}%
\pgfsetstrokecolor{currentstroke}%
\pgfsetdash{}{0pt}%
\pgfpathmoveto{\pgfqpoint{1.776345in}{0.528000in}}%
\pgfpathlineto{\pgfqpoint{1.776345in}{4.224000in}}%
\pgfusepath{stroke}%
\end{pgfscope}%
\begin{pgfscope}%
\pgfsetbuttcap%
\pgfsetroundjoin%
\definecolor{currentfill}{rgb}{0.000000,0.000000,0.000000}%
\pgfsetfillcolor{currentfill}%
\pgfsetlinewidth{0.803000pt}%
\definecolor{currentstroke}{rgb}{0.000000,0.000000,0.000000}%
\pgfsetstrokecolor{currentstroke}%
\pgfsetdash{}{0pt}%
\pgfsys@defobject{currentmarker}{\pgfqpoint{0.000000in}{-0.048611in}}{\pgfqpoint{0.000000in}{0.000000in}}{%
\pgfpathmoveto{\pgfqpoint{0.000000in}{0.000000in}}%
\pgfpathlineto{\pgfqpoint{0.000000in}{-0.048611in}}%
\pgfusepath{stroke,fill}%
}%
\begin{pgfscope}%
\pgfsys@transformshift{1.776345in}{0.528000in}%
\pgfsys@useobject{currentmarker}{}%
\end{pgfscope}%
\end{pgfscope}%
\begin{pgfscope}%
\definecolor{textcolor}{rgb}{0.000000,0.000000,0.000000}%
\pgfsetstrokecolor{textcolor}%
\pgfsetfillcolor{textcolor}%
\pgftext[x=1.776345in,y=0.430778in,,top]{\color{textcolor}\sffamily\fontsize{10.000000}{12.000000}\selectfont 5}%
\end{pgfscope}%
\begin{pgfscope}%
\pgfpathrectangle{\pgfqpoint{0.800000in}{0.528000in}}{\pgfqpoint{4.960000in}{3.696000in}}%
\pgfusepath{clip}%
\pgfsetrectcap%
\pgfsetroundjoin%
\pgfsetlinewidth{0.803000pt}%
\definecolor{currentstroke}{rgb}{0.690196,0.690196,0.690196}%
\pgfsetstrokecolor{currentstroke}%
\pgfsetdash{}{0pt}%
\pgfpathmoveto{\pgfqpoint{2.527235in}{0.528000in}}%
\pgfpathlineto{\pgfqpoint{2.527235in}{4.224000in}}%
\pgfusepath{stroke}%
\end{pgfscope}%
\begin{pgfscope}%
\pgfsetbuttcap%
\pgfsetroundjoin%
\definecolor{currentfill}{rgb}{0.000000,0.000000,0.000000}%
\pgfsetfillcolor{currentfill}%
\pgfsetlinewidth{0.803000pt}%
\definecolor{currentstroke}{rgb}{0.000000,0.000000,0.000000}%
\pgfsetstrokecolor{currentstroke}%
\pgfsetdash{}{0pt}%
\pgfsys@defobject{currentmarker}{\pgfqpoint{0.000000in}{-0.048611in}}{\pgfqpoint{0.000000in}{0.000000in}}{%
\pgfpathmoveto{\pgfqpoint{0.000000in}{0.000000in}}%
\pgfpathlineto{\pgfqpoint{0.000000in}{-0.048611in}}%
\pgfusepath{stroke,fill}%
}%
\begin{pgfscope}%
\pgfsys@transformshift{2.527235in}{0.528000in}%
\pgfsys@useobject{currentmarker}{}%
\end{pgfscope}%
\end{pgfscope}%
\begin{pgfscope}%
\definecolor{textcolor}{rgb}{0.000000,0.000000,0.000000}%
\pgfsetstrokecolor{textcolor}%
\pgfsetfillcolor{textcolor}%
\pgftext[x=2.527235in,y=0.430778in,,top]{\color{textcolor}\sffamily\fontsize{10.000000}{12.000000}\selectfont 10}%
\end{pgfscope}%
\begin{pgfscope}%
\pgfpathrectangle{\pgfqpoint{0.800000in}{0.528000in}}{\pgfqpoint{4.960000in}{3.696000in}}%
\pgfusepath{clip}%
\pgfsetrectcap%
\pgfsetroundjoin%
\pgfsetlinewidth{0.803000pt}%
\definecolor{currentstroke}{rgb}{0.690196,0.690196,0.690196}%
\pgfsetstrokecolor{currentstroke}%
\pgfsetdash{}{0pt}%
\pgfpathmoveto{\pgfqpoint{3.278125in}{0.528000in}}%
\pgfpathlineto{\pgfqpoint{3.278125in}{4.224000in}}%
\pgfusepath{stroke}%
\end{pgfscope}%
\begin{pgfscope}%
\pgfsetbuttcap%
\pgfsetroundjoin%
\definecolor{currentfill}{rgb}{0.000000,0.000000,0.000000}%
\pgfsetfillcolor{currentfill}%
\pgfsetlinewidth{0.803000pt}%
\definecolor{currentstroke}{rgb}{0.000000,0.000000,0.000000}%
\pgfsetstrokecolor{currentstroke}%
\pgfsetdash{}{0pt}%
\pgfsys@defobject{currentmarker}{\pgfqpoint{0.000000in}{-0.048611in}}{\pgfqpoint{0.000000in}{0.000000in}}{%
\pgfpathmoveto{\pgfqpoint{0.000000in}{0.000000in}}%
\pgfpathlineto{\pgfqpoint{0.000000in}{-0.048611in}}%
\pgfusepath{stroke,fill}%
}%
\begin{pgfscope}%
\pgfsys@transformshift{3.278125in}{0.528000in}%
\pgfsys@useobject{currentmarker}{}%
\end{pgfscope}%
\end{pgfscope}%
\begin{pgfscope}%
\definecolor{textcolor}{rgb}{0.000000,0.000000,0.000000}%
\pgfsetstrokecolor{textcolor}%
\pgfsetfillcolor{textcolor}%
\pgftext[x=3.278125in,y=0.430778in,,top]{\color{textcolor}\sffamily\fontsize{10.000000}{12.000000}\selectfont 15}%
\end{pgfscope}%
\begin{pgfscope}%
\pgfpathrectangle{\pgfqpoint{0.800000in}{0.528000in}}{\pgfqpoint{4.960000in}{3.696000in}}%
\pgfusepath{clip}%
\pgfsetrectcap%
\pgfsetroundjoin%
\pgfsetlinewidth{0.803000pt}%
\definecolor{currentstroke}{rgb}{0.690196,0.690196,0.690196}%
\pgfsetstrokecolor{currentstroke}%
\pgfsetdash{}{0pt}%
\pgfpathmoveto{\pgfqpoint{4.029015in}{0.528000in}}%
\pgfpathlineto{\pgfqpoint{4.029015in}{4.224000in}}%
\pgfusepath{stroke}%
\end{pgfscope}%
\begin{pgfscope}%
\pgfsetbuttcap%
\pgfsetroundjoin%
\definecolor{currentfill}{rgb}{0.000000,0.000000,0.000000}%
\pgfsetfillcolor{currentfill}%
\pgfsetlinewidth{0.803000pt}%
\definecolor{currentstroke}{rgb}{0.000000,0.000000,0.000000}%
\pgfsetstrokecolor{currentstroke}%
\pgfsetdash{}{0pt}%
\pgfsys@defobject{currentmarker}{\pgfqpoint{0.000000in}{-0.048611in}}{\pgfqpoint{0.000000in}{0.000000in}}{%
\pgfpathmoveto{\pgfqpoint{0.000000in}{0.000000in}}%
\pgfpathlineto{\pgfqpoint{0.000000in}{-0.048611in}}%
\pgfusepath{stroke,fill}%
}%
\begin{pgfscope}%
\pgfsys@transformshift{4.029015in}{0.528000in}%
\pgfsys@useobject{currentmarker}{}%
\end{pgfscope}%
\end{pgfscope}%
\begin{pgfscope}%
\definecolor{textcolor}{rgb}{0.000000,0.000000,0.000000}%
\pgfsetstrokecolor{textcolor}%
\pgfsetfillcolor{textcolor}%
\pgftext[x=4.029015in,y=0.430778in,,top]{\color{textcolor}\sffamily\fontsize{10.000000}{12.000000}\selectfont 20}%
\end{pgfscope}%
\begin{pgfscope}%
\pgfpathrectangle{\pgfqpoint{0.800000in}{0.528000in}}{\pgfqpoint{4.960000in}{3.696000in}}%
\pgfusepath{clip}%
\pgfsetrectcap%
\pgfsetroundjoin%
\pgfsetlinewidth{0.803000pt}%
\definecolor{currentstroke}{rgb}{0.690196,0.690196,0.690196}%
\pgfsetstrokecolor{currentstroke}%
\pgfsetdash{}{0pt}%
\pgfpathmoveto{\pgfqpoint{4.779905in}{0.528000in}}%
\pgfpathlineto{\pgfqpoint{4.779905in}{4.224000in}}%
\pgfusepath{stroke}%
\end{pgfscope}%
\begin{pgfscope}%
\pgfsetbuttcap%
\pgfsetroundjoin%
\definecolor{currentfill}{rgb}{0.000000,0.000000,0.000000}%
\pgfsetfillcolor{currentfill}%
\pgfsetlinewidth{0.803000pt}%
\definecolor{currentstroke}{rgb}{0.000000,0.000000,0.000000}%
\pgfsetstrokecolor{currentstroke}%
\pgfsetdash{}{0pt}%
\pgfsys@defobject{currentmarker}{\pgfqpoint{0.000000in}{-0.048611in}}{\pgfqpoint{0.000000in}{0.000000in}}{%
\pgfpathmoveto{\pgfqpoint{0.000000in}{0.000000in}}%
\pgfpathlineto{\pgfqpoint{0.000000in}{-0.048611in}}%
\pgfusepath{stroke,fill}%
}%
\begin{pgfscope}%
\pgfsys@transformshift{4.779905in}{0.528000in}%
\pgfsys@useobject{currentmarker}{}%
\end{pgfscope}%
\end{pgfscope}%
\begin{pgfscope}%
\definecolor{textcolor}{rgb}{0.000000,0.000000,0.000000}%
\pgfsetstrokecolor{textcolor}%
\pgfsetfillcolor{textcolor}%
\pgftext[x=4.779905in,y=0.430778in,,top]{\color{textcolor}\sffamily\fontsize{10.000000}{12.000000}\selectfont 25}%
\end{pgfscope}%
\begin{pgfscope}%
\pgfpathrectangle{\pgfqpoint{0.800000in}{0.528000in}}{\pgfqpoint{4.960000in}{3.696000in}}%
\pgfusepath{clip}%
\pgfsetrectcap%
\pgfsetroundjoin%
\pgfsetlinewidth{0.803000pt}%
\definecolor{currentstroke}{rgb}{0.690196,0.690196,0.690196}%
\pgfsetstrokecolor{currentstroke}%
\pgfsetdash{}{0pt}%
\pgfpathmoveto{\pgfqpoint{5.530795in}{0.528000in}}%
\pgfpathlineto{\pgfqpoint{5.530795in}{4.224000in}}%
\pgfusepath{stroke}%
\end{pgfscope}%
\begin{pgfscope}%
\pgfsetbuttcap%
\pgfsetroundjoin%
\definecolor{currentfill}{rgb}{0.000000,0.000000,0.000000}%
\pgfsetfillcolor{currentfill}%
\pgfsetlinewidth{0.803000pt}%
\definecolor{currentstroke}{rgb}{0.000000,0.000000,0.000000}%
\pgfsetstrokecolor{currentstroke}%
\pgfsetdash{}{0pt}%
\pgfsys@defobject{currentmarker}{\pgfqpoint{0.000000in}{-0.048611in}}{\pgfqpoint{0.000000in}{0.000000in}}{%
\pgfpathmoveto{\pgfqpoint{0.000000in}{0.000000in}}%
\pgfpathlineto{\pgfqpoint{0.000000in}{-0.048611in}}%
\pgfusepath{stroke,fill}%
}%
\begin{pgfscope}%
\pgfsys@transformshift{5.530795in}{0.528000in}%
\pgfsys@useobject{currentmarker}{}%
\end{pgfscope}%
\end{pgfscope}%
\begin{pgfscope}%
\definecolor{textcolor}{rgb}{0.000000,0.000000,0.000000}%
\pgfsetstrokecolor{textcolor}%
\pgfsetfillcolor{textcolor}%
\pgftext[x=5.530795in,y=0.430778in,,top]{\color{textcolor}\sffamily\fontsize{10.000000}{12.000000}\selectfont 30}%
\end{pgfscope}%
\begin{pgfscope}%
\definecolor{textcolor}{rgb}{0.000000,0.000000,0.000000}%
\pgfsetstrokecolor{textcolor}%
\pgfsetfillcolor{textcolor}%
\pgftext[x=3.280000in,y=0.240809in,,top]{\color{textcolor}\sffamily\fontsize{10.000000}{12.000000}\selectfont time [s]}%
\end{pgfscope}%
\begin{pgfscope}%
\pgfpathrectangle{\pgfqpoint{0.800000in}{0.528000in}}{\pgfqpoint{4.960000in}{3.696000in}}%
\pgfusepath{clip}%
\pgfsetrectcap%
\pgfsetroundjoin%
\pgfsetlinewidth{0.803000pt}%
\definecolor{currentstroke}{rgb}{0.690196,0.690196,0.690196}%
\pgfsetstrokecolor{currentstroke}%
\pgfsetdash{}{0pt}%
\pgfpathmoveto{\pgfqpoint{0.800000in}{0.696000in}}%
\pgfpathlineto{\pgfqpoint{5.760000in}{0.696000in}}%
\pgfusepath{stroke}%
\end{pgfscope}%
\begin{pgfscope}%
\pgfsetbuttcap%
\pgfsetroundjoin%
\definecolor{currentfill}{rgb}{0.000000,0.000000,0.000000}%
\pgfsetfillcolor{currentfill}%
\pgfsetlinewidth{0.803000pt}%
\definecolor{currentstroke}{rgb}{0.000000,0.000000,0.000000}%
\pgfsetstrokecolor{currentstroke}%
\pgfsetdash{}{0pt}%
\pgfsys@defobject{currentmarker}{\pgfqpoint{-0.048611in}{0.000000in}}{\pgfqpoint{-0.000000in}{0.000000in}}{%
\pgfpathmoveto{\pgfqpoint{-0.000000in}{0.000000in}}%
\pgfpathlineto{\pgfqpoint{-0.048611in}{0.000000in}}%
\pgfusepath{stroke,fill}%
}%
\begin{pgfscope}%
\pgfsys@transformshift{0.800000in}{0.696000in}%
\pgfsys@useobject{currentmarker}{}%
\end{pgfscope}%
\end{pgfscope}%
\begin{pgfscope}%
\definecolor{textcolor}{rgb}{0.000000,0.000000,0.000000}%
\pgfsetstrokecolor{textcolor}%
\pgfsetfillcolor{textcolor}%
\pgftext[x=0.373873in, y=0.643238in, left, base]{\color{textcolor}\sffamily\fontsize{10.000000}{12.000000}\selectfont \ensuremath{-}0.4}%
\end{pgfscope}%
\begin{pgfscope}%
\pgfpathrectangle{\pgfqpoint{0.800000in}{0.528000in}}{\pgfqpoint{4.960000in}{3.696000in}}%
\pgfusepath{clip}%
\pgfsetrectcap%
\pgfsetroundjoin%
\pgfsetlinewidth{0.803000pt}%
\definecolor{currentstroke}{rgb}{0.690196,0.690196,0.690196}%
\pgfsetstrokecolor{currentstroke}%
\pgfsetdash{}{0pt}%
\pgfpathmoveto{\pgfqpoint{0.800000in}{1.187987in}}%
\pgfpathlineto{\pgfqpoint{5.760000in}{1.187987in}}%
\pgfusepath{stroke}%
\end{pgfscope}%
\begin{pgfscope}%
\pgfsetbuttcap%
\pgfsetroundjoin%
\definecolor{currentfill}{rgb}{0.000000,0.000000,0.000000}%
\pgfsetfillcolor{currentfill}%
\pgfsetlinewidth{0.803000pt}%
\definecolor{currentstroke}{rgb}{0.000000,0.000000,0.000000}%
\pgfsetstrokecolor{currentstroke}%
\pgfsetdash{}{0pt}%
\pgfsys@defobject{currentmarker}{\pgfqpoint{-0.048611in}{0.000000in}}{\pgfqpoint{-0.000000in}{0.000000in}}{%
\pgfpathmoveto{\pgfqpoint{-0.000000in}{0.000000in}}%
\pgfpathlineto{\pgfqpoint{-0.048611in}{0.000000in}}%
\pgfusepath{stroke,fill}%
}%
\begin{pgfscope}%
\pgfsys@transformshift{0.800000in}{1.187987in}%
\pgfsys@useobject{currentmarker}{}%
\end{pgfscope}%
\end{pgfscope}%
\begin{pgfscope}%
\definecolor{textcolor}{rgb}{0.000000,0.000000,0.000000}%
\pgfsetstrokecolor{textcolor}%
\pgfsetfillcolor{textcolor}%
\pgftext[x=0.373873in, y=1.135225in, left, base]{\color{textcolor}\sffamily\fontsize{10.000000}{12.000000}\selectfont \ensuremath{-}0.3}%
\end{pgfscope}%
\begin{pgfscope}%
\pgfpathrectangle{\pgfqpoint{0.800000in}{0.528000in}}{\pgfqpoint{4.960000in}{3.696000in}}%
\pgfusepath{clip}%
\pgfsetrectcap%
\pgfsetroundjoin%
\pgfsetlinewidth{0.803000pt}%
\definecolor{currentstroke}{rgb}{0.690196,0.690196,0.690196}%
\pgfsetstrokecolor{currentstroke}%
\pgfsetdash{}{0pt}%
\pgfpathmoveto{\pgfqpoint{0.800000in}{1.679974in}}%
\pgfpathlineto{\pgfqpoint{5.760000in}{1.679974in}}%
\pgfusepath{stroke}%
\end{pgfscope}%
\begin{pgfscope}%
\pgfsetbuttcap%
\pgfsetroundjoin%
\definecolor{currentfill}{rgb}{0.000000,0.000000,0.000000}%
\pgfsetfillcolor{currentfill}%
\pgfsetlinewidth{0.803000pt}%
\definecolor{currentstroke}{rgb}{0.000000,0.000000,0.000000}%
\pgfsetstrokecolor{currentstroke}%
\pgfsetdash{}{0pt}%
\pgfsys@defobject{currentmarker}{\pgfqpoint{-0.048611in}{0.000000in}}{\pgfqpoint{-0.000000in}{0.000000in}}{%
\pgfpathmoveto{\pgfqpoint{-0.000000in}{0.000000in}}%
\pgfpathlineto{\pgfqpoint{-0.048611in}{0.000000in}}%
\pgfusepath{stroke,fill}%
}%
\begin{pgfscope}%
\pgfsys@transformshift{0.800000in}{1.679974in}%
\pgfsys@useobject{currentmarker}{}%
\end{pgfscope}%
\end{pgfscope}%
\begin{pgfscope}%
\definecolor{textcolor}{rgb}{0.000000,0.000000,0.000000}%
\pgfsetstrokecolor{textcolor}%
\pgfsetfillcolor{textcolor}%
\pgftext[x=0.373873in, y=1.627212in, left, base]{\color{textcolor}\sffamily\fontsize{10.000000}{12.000000}\selectfont \ensuremath{-}0.2}%
\end{pgfscope}%
\begin{pgfscope}%
\pgfpathrectangle{\pgfqpoint{0.800000in}{0.528000in}}{\pgfqpoint{4.960000in}{3.696000in}}%
\pgfusepath{clip}%
\pgfsetrectcap%
\pgfsetroundjoin%
\pgfsetlinewidth{0.803000pt}%
\definecolor{currentstroke}{rgb}{0.690196,0.690196,0.690196}%
\pgfsetstrokecolor{currentstroke}%
\pgfsetdash{}{0pt}%
\pgfpathmoveto{\pgfqpoint{0.800000in}{2.171961in}}%
\pgfpathlineto{\pgfqpoint{5.760000in}{2.171961in}}%
\pgfusepath{stroke}%
\end{pgfscope}%
\begin{pgfscope}%
\pgfsetbuttcap%
\pgfsetroundjoin%
\definecolor{currentfill}{rgb}{0.000000,0.000000,0.000000}%
\pgfsetfillcolor{currentfill}%
\pgfsetlinewidth{0.803000pt}%
\definecolor{currentstroke}{rgb}{0.000000,0.000000,0.000000}%
\pgfsetstrokecolor{currentstroke}%
\pgfsetdash{}{0pt}%
\pgfsys@defobject{currentmarker}{\pgfqpoint{-0.048611in}{0.000000in}}{\pgfqpoint{-0.000000in}{0.000000in}}{%
\pgfpathmoveto{\pgfqpoint{-0.000000in}{0.000000in}}%
\pgfpathlineto{\pgfqpoint{-0.048611in}{0.000000in}}%
\pgfusepath{stroke,fill}%
}%
\begin{pgfscope}%
\pgfsys@transformshift{0.800000in}{2.171961in}%
\pgfsys@useobject{currentmarker}{}%
\end{pgfscope}%
\end{pgfscope}%
\begin{pgfscope}%
\definecolor{textcolor}{rgb}{0.000000,0.000000,0.000000}%
\pgfsetstrokecolor{textcolor}%
\pgfsetfillcolor{textcolor}%
\pgftext[x=0.373873in, y=2.119199in, left, base]{\color{textcolor}\sffamily\fontsize{10.000000}{12.000000}\selectfont \ensuremath{-}0.1}%
\end{pgfscope}%
\begin{pgfscope}%
\pgfpathrectangle{\pgfqpoint{0.800000in}{0.528000in}}{\pgfqpoint{4.960000in}{3.696000in}}%
\pgfusepath{clip}%
\pgfsetrectcap%
\pgfsetroundjoin%
\pgfsetlinewidth{0.803000pt}%
\definecolor{currentstroke}{rgb}{0.690196,0.690196,0.690196}%
\pgfsetstrokecolor{currentstroke}%
\pgfsetdash{}{0pt}%
\pgfpathmoveto{\pgfqpoint{0.800000in}{2.663948in}}%
\pgfpathlineto{\pgfqpoint{5.760000in}{2.663948in}}%
\pgfusepath{stroke}%
\end{pgfscope}%
\begin{pgfscope}%
\pgfsetbuttcap%
\pgfsetroundjoin%
\definecolor{currentfill}{rgb}{0.000000,0.000000,0.000000}%
\pgfsetfillcolor{currentfill}%
\pgfsetlinewidth{0.803000pt}%
\definecolor{currentstroke}{rgb}{0.000000,0.000000,0.000000}%
\pgfsetstrokecolor{currentstroke}%
\pgfsetdash{}{0pt}%
\pgfsys@defobject{currentmarker}{\pgfqpoint{-0.048611in}{0.000000in}}{\pgfqpoint{-0.000000in}{0.000000in}}{%
\pgfpathmoveto{\pgfqpoint{-0.000000in}{0.000000in}}%
\pgfpathlineto{\pgfqpoint{-0.048611in}{0.000000in}}%
\pgfusepath{stroke,fill}%
}%
\begin{pgfscope}%
\pgfsys@transformshift{0.800000in}{2.663948in}%
\pgfsys@useobject{currentmarker}{}%
\end{pgfscope}%
\end{pgfscope}%
\begin{pgfscope}%
\definecolor{textcolor}{rgb}{0.000000,0.000000,0.000000}%
\pgfsetstrokecolor{textcolor}%
\pgfsetfillcolor{textcolor}%
\pgftext[x=0.481898in, y=2.611186in, left, base]{\color{textcolor}\sffamily\fontsize{10.000000}{12.000000}\selectfont 0.0}%
\end{pgfscope}%
\begin{pgfscope}%
\pgfpathrectangle{\pgfqpoint{0.800000in}{0.528000in}}{\pgfqpoint{4.960000in}{3.696000in}}%
\pgfusepath{clip}%
\pgfsetrectcap%
\pgfsetroundjoin%
\pgfsetlinewidth{0.803000pt}%
\definecolor{currentstroke}{rgb}{0.690196,0.690196,0.690196}%
\pgfsetstrokecolor{currentstroke}%
\pgfsetdash{}{0pt}%
\pgfpathmoveto{\pgfqpoint{0.800000in}{3.155935in}}%
\pgfpathlineto{\pgfqpoint{5.760000in}{3.155935in}}%
\pgfusepath{stroke}%
\end{pgfscope}%
\begin{pgfscope}%
\pgfsetbuttcap%
\pgfsetroundjoin%
\definecolor{currentfill}{rgb}{0.000000,0.000000,0.000000}%
\pgfsetfillcolor{currentfill}%
\pgfsetlinewidth{0.803000pt}%
\definecolor{currentstroke}{rgb}{0.000000,0.000000,0.000000}%
\pgfsetstrokecolor{currentstroke}%
\pgfsetdash{}{0pt}%
\pgfsys@defobject{currentmarker}{\pgfqpoint{-0.048611in}{0.000000in}}{\pgfqpoint{-0.000000in}{0.000000in}}{%
\pgfpathmoveto{\pgfqpoint{-0.000000in}{0.000000in}}%
\pgfpathlineto{\pgfqpoint{-0.048611in}{0.000000in}}%
\pgfusepath{stroke,fill}%
}%
\begin{pgfscope}%
\pgfsys@transformshift{0.800000in}{3.155935in}%
\pgfsys@useobject{currentmarker}{}%
\end{pgfscope}%
\end{pgfscope}%
\begin{pgfscope}%
\definecolor{textcolor}{rgb}{0.000000,0.000000,0.000000}%
\pgfsetstrokecolor{textcolor}%
\pgfsetfillcolor{textcolor}%
\pgftext[x=0.481898in, y=3.103173in, left, base]{\color{textcolor}\sffamily\fontsize{10.000000}{12.000000}\selectfont 0.1}%
\end{pgfscope}%
\begin{pgfscope}%
\pgfpathrectangle{\pgfqpoint{0.800000in}{0.528000in}}{\pgfqpoint{4.960000in}{3.696000in}}%
\pgfusepath{clip}%
\pgfsetrectcap%
\pgfsetroundjoin%
\pgfsetlinewidth{0.803000pt}%
\definecolor{currentstroke}{rgb}{0.690196,0.690196,0.690196}%
\pgfsetstrokecolor{currentstroke}%
\pgfsetdash{}{0pt}%
\pgfpathmoveto{\pgfqpoint{0.800000in}{3.647922in}}%
\pgfpathlineto{\pgfqpoint{5.760000in}{3.647922in}}%
\pgfusepath{stroke}%
\end{pgfscope}%
\begin{pgfscope}%
\pgfsetbuttcap%
\pgfsetroundjoin%
\definecolor{currentfill}{rgb}{0.000000,0.000000,0.000000}%
\pgfsetfillcolor{currentfill}%
\pgfsetlinewidth{0.803000pt}%
\definecolor{currentstroke}{rgb}{0.000000,0.000000,0.000000}%
\pgfsetstrokecolor{currentstroke}%
\pgfsetdash{}{0pt}%
\pgfsys@defobject{currentmarker}{\pgfqpoint{-0.048611in}{0.000000in}}{\pgfqpoint{-0.000000in}{0.000000in}}{%
\pgfpathmoveto{\pgfqpoint{-0.000000in}{0.000000in}}%
\pgfpathlineto{\pgfqpoint{-0.048611in}{0.000000in}}%
\pgfusepath{stroke,fill}%
}%
\begin{pgfscope}%
\pgfsys@transformshift{0.800000in}{3.647922in}%
\pgfsys@useobject{currentmarker}{}%
\end{pgfscope}%
\end{pgfscope}%
\begin{pgfscope}%
\definecolor{textcolor}{rgb}{0.000000,0.000000,0.000000}%
\pgfsetstrokecolor{textcolor}%
\pgfsetfillcolor{textcolor}%
\pgftext[x=0.481898in, y=3.595160in, left, base]{\color{textcolor}\sffamily\fontsize{10.000000}{12.000000}\selectfont 0.2}%
\end{pgfscope}%
\begin{pgfscope}%
\pgfpathrectangle{\pgfqpoint{0.800000in}{0.528000in}}{\pgfqpoint{4.960000in}{3.696000in}}%
\pgfusepath{clip}%
\pgfsetrectcap%
\pgfsetroundjoin%
\pgfsetlinewidth{0.803000pt}%
\definecolor{currentstroke}{rgb}{0.690196,0.690196,0.690196}%
\pgfsetstrokecolor{currentstroke}%
\pgfsetdash{}{0pt}%
\pgfpathmoveto{\pgfqpoint{0.800000in}{4.139909in}}%
\pgfpathlineto{\pgfqpoint{5.760000in}{4.139909in}}%
\pgfusepath{stroke}%
\end{pgfscope}%
\begin{pgfscope}%
\pgfsetbuttcap%
\pgfsetroundjoin%
\definecolor{currentfill}{rgb}{0.000000,0.000000,0.000000}%
\pgfsetfillcolor{currentfill}%
\pgfsetlinewidth{0.803000pt}%
\definecolor{currentstroke}{rgb}{0.000000,0.000000,0.000000}%
\pgfsetstrokecolor{currentstroke}%
\pgfsetdash{}{0pt}%
\pgfsys@defobject{currentmarker}{\pgfqpoint{-0.048611in}{0.000000in}}{\pgfqpoint{-0.000000in}{0.000000in}}{%
\pgfpathmoveto{\pgfqpoint{-0.000000in}{0.000000in}}%
\pgfpathlineto{\pgfqpoint{-0.048611in}{0.000000in}}%
\pgfusepath{stroke,fill}%
}%
\begin{pgfscope}%
\pgfsys@transformshift{0.800000in}{4.139909in}%
\pgfsys@useobject{currentmarker}{}%
\end{pgfscope}%
\end{pgfscope}%
\begin{pgfscope}%
\definecolor{textcolor}{rgb}{0.000000,0.000000,0.000000}%
\pgfsetstrokecolor{textcolor}%
\pgfsetfillcolor{textcolor}%
\pgftext[x=0.481898in, y=4.087147in, left, base]{\color{textcolor}\sffamily\fontsize{10.000000}{12.000000}\selectfont 0.3}%
\end{pgfscope}%
\begin{pgfscope}%
\definecolor{textcolor}{rgb}{0.000000,0.000000,0.000000}%
\pgfsetstrokecolor{textcolor}%
\pgfsetfillcolor{textcolor}%
\pgftext[x=0.318318in,y=2.376000in,,bottom,rotate=90.000000]{\color{textcolor}\sffamily\fontsize{10.000000}{12.000000}\selectfont  [m/s]}%
\end{pgfscope}%
\begin{pgfscope}%
\pgfpathrectangle{\pgfqpoint{0.800000in}{0.528000in}}{\pgfqpoint{4.960000in}{3.696000in}}%
\pgfusepath{clip}%
\pgfsetrectcap%
\pgfsetroundjoin%
\pgfsetlinewidth{1.505625pt}%
\definecolor{currentstroke}{rgb}{0.121569,0.466667,0.705882}%
\pgfsetstrokecolor{currentstroke}%
\pgfsetdash{}{0pt}%
\pgfpathmoveto{\pgfqpoint{1.025455in}{0.696000in}}%
\pgfpathlineto{\pgfqpoint{1.079789in}{0.696000in}}%
\pgfpathlineto{\pgfqpoint{1.135150in}{0.696000in}}%
\pgfpathlineto{\pgfqpoint{1.189142in}{0.696000in}}%
\pgfpathlineto{\pgfqpoint{1.243683in}{0.696000in}}%
\pgfpathlineto{\pgfqpoint{1.299666in}{0.696000in}}%
\pgfpathlineto{\pgfqpoint{1.351799in}{2.099322in}}%
\pgfpathlineto{\pgfqpoint{1.405541in}{2.417515in}}%
\pgfpathlineto{\pgfqpoint{1.459709in}{3.483754in}}%
\pgfpathlineto{\pgfqpoint{1.514300in}{2.350517in}}%
\pgfpathlineto{\pgfqpoint{1.568274in}{1.904040in}}%
\pgfpathlineto{\pgfqpoint{1.624677in}{2.395045in}}%
\pgfpathlineto{\pgfqpoint{1.679007in}{2.706461in}}%
\pgfpathlineto{\pgfqpoint{1.733292in}{2.648720in}}%
\pgfpathlineto{\pgfqpoint{1.787480in}{2.616426in}}%
\pgfpathlineto{\pgfqpoint{1.844118in}{2.586853in}}%
\pgfpathlineto{\pgfqpoint{1.896952in}{2.533633in}}%
\pgfpathlineto{\pgfqpoint{1.952236in}{2.527128in}}%
\pgfpathlineto{\pgfqpoint{2.005498in}{2.533022in}}%
\pgfpathlineto{\pgfqpoint{2.060533in}{2.618105in}}%
\pgfpathlineto{\pgfqpoint{2.114225in}{2.642576in}}%
\pgfpathlineto{\pgfqpoint{2.167964in}{2.708864in}}%
\pgfpathlineto{\pgfqpoint{2.223511in}{2.689508in}}%
\pgfpathlineto{\pgfqpoint{2.276562in}{2.620983in}}%
\pgfpathlineto{\pgfqpoint{2.331563in}{2.644851in}}%
\pgfpathlineto{\pgfqpoint{2.386619in}{2.629862in}}%
\pgfpathlineto{\pgfqpoint{2.439989in}{2.609525in}}%
\pgfpathlineto{\pgfqpoint{2.494745in}{2.606162in}}%
\pgfpathlineto{\pgfqpoint{2.549359in}{2.635155in}}%
\pgfpathlineto{\pgfqpoint{2.602881in}{2.563548in}}%
\pgfpathlineto{\pgfqpoint{2.657497in}{2.573245in}}%
\pgfpathlineto{\pgfqpoint{2.711859in}{2.587745in}}%
\pgfpathlineto{\pgfqpoint{2.767421in}{2.570843in}}%
\pgfpathlineto{\pgfqpoint{2.822550in}{2.566678in}}%
\pgfpathlineto{\pgfqpoint{2.875066in}{2.565656in}}%
\pgfpathlineto{\pgfqpoint{2.928644in}{2.541000in}}%
\pgfpathlineto{\pgfqpoint{2.982160in}{2.588661in}}%
\pgfpathlineto{\pgfqpoint{3.035876in}{2.718594in}}%
\pgfpathlineto{\pgfqpoint{3.090532in}{2.742469in}}%
\pgfpathlineto{\pgfqpoint{3.144546in}{2.771536in}}%
\pgfpathlineto{\pgfqpoint{3.200570in}{2.800954in}}%
\pgfpathlineto{\pgfqpoint{3.253349in}{2.739105in}}%
\pgfpathlineto{\pgfqpoint{3.307357in}{2.717219in}}%
\pgfpathlineto{\pgfqpoint{3.362385in}{2.708260in}}%
\pgfpathlineto{\pgfqpoint{3.416454in}{2.681559in}}%
\pgfpathlineto{\pgfqpoint{3.470642in}{2.597258in}}%
\pgfpathlineto{\pgfqpoint{3.524714in}{2.496718in}}%
\pgfpathlineto{\pgfqpoint{3.579145in}{2.415914in}}%
\pgfpathlineto{\pgfqpoint{3.633040in}{2.487583in}}%
\pgfpathlineto{\pgfqpoint{3.687614in}{2.524279in}}%
\pgfpathlineto{\pgfqpoint{3.741880in}{2.450015in}}%
\pgfpathlineto{\pgfqpoint{3.796084in}{2.497708in}}%
\pgfpathlineto{\pgfqpoint{3.851834in}{2.499258in}}%
\pgfpathlineto{\pgfqpoint{3.905705in}{2.570700in}}%
\pgfpathlineto{\pgfqpoint{3.959606in}{2.853445in}}%
\pgfpathlineto{\pgfqpoint{4.015050in}{2.823024in}}%
\pgfpathlineto{\pgfqpoint{4.068056in}{2.760309in}}%
\pgfpathlineto{\pgfqpoint{4.122407in}{3.125118in}}%
\pgfpathlineto{\pgfqpoint{4.176246in}{2.812456in}}%
\pgfpathlineto{\pgfqpoint{4.230670in}{2.295501in}}%
\pgfpathlineto{\pgfqpoint{4.284415in}{2.668621in}}%
\pgfpathlineto{\pgfqpoint{4.339474in}{2.618278in}}%
\pgfpathlineto{\pgfqpoint{4.395105in}{2.666206in}}%
\pgfpathlineto{\pgfqpoint{4.448722in}{2.608334in}}%
\pgfpathlineto{\pgfqpoint{4.502698in}{2.639698in}}%
\pgfpathlineto{\pgfqpoint{4.556582in}{2.583004in}}%
\pgfpathlineto{\pgfqpoint{4.610820in}{2.658417in}}%
\pgfpathlineto{\pgfqpoint{4.665054in}{2.627184in}}%
\pgfpathlineto{\pgfqpoint{4.719361in}{2.632253in}}%
\pgfpathlineto{\pgfqpoint{4.773076in}{2.607019in}}%
\pgfpathlineto{\pgfqpoint{4.827357in}{2.668331in}}%
\pgfpathlineto{\pgfqpoint{4.883545in}{2.781279in}}%
\pgfpathlineto{\pgfqpoint{4.936847in}{3.183224in}}%
\pgfpathlineto{\pgfqpoint{4.990680in}{2.567604in}}%
\pgfpathlineto{\pgfqpoint{5.044635in}{2.669931in}}%
\pgfpathlineto{\pgfqpoint{5.098891in}{2.616843in}}%
\pgfpathlineto{\pgfqpoint{5.153310in}{2.615323in}}%
\pgfpathlineto{\pgfqpoint{5.207160in}{2.573361in}}%
\pgfpathlineto{\pgfqpoint{5.261262in}{2.568857in}}%
\pgfpathlineto{\pgfqpoint{5.315747in}{2.573928in}}%
\pgfpathlineto{\pgfqpoint{5.370090in}{2.583141in}}%
\pgfpathlineto{\pgfqpoint{5.424768in}{2.557662in}}%
\pgfpathlineto{\pgfqpoint{5.481781in}{2.688044in}}%
\pgfpathlineto{\pgfqpoint{5.534539in}{2.871673in}}%
\pgfusepath{stroke}%
\end{pgfscope}%
\begin{pgfscope}%
\pgfpathrectangle{\pgfqpoint{0.800000in}{0.528000in}}{\pgfqpoint{4.960000in}{3.696000in}}%
\pgfusepath{clip}%
\pgfsetrectcap%
\pgfsetroundjoin%
\pgfsetlinewidth{1.505625pt}%
\definecolor{currentstroke}{rgb}{1.000000,0.498039,0.054902}%
\pgfsetstrokecolor{currentstroke}%
\pgfsetdash{}{0pt}%
\pgfpathmoveto{\pgfqpoint{1.025455in}{0.696000in}}%
\pgfpathlineto{\pgfqpoint{1.080472in}{0.696000in}}%
\pgfpathlineto{\pgfqpoint{1.134899in}{0.696000in}}%
\pgfpathlineto{\pgfqpoint{1.188629in}{0.696000in}}%
\pgfpathlineto{\pgfqpoint{1.243261in}{0.696000in}}%
\pgfpathlineto{\pgfqpoint{1.299328in}{0.696000in}}%
\pgfpathlineto{\pgfqpoint{1.351473in}{0.712006in}}%
\pgfpathlineto{\pgfqpoint{1.405475in}{1.092854in}}%
\pgfpathlineto{\pgfqpoint{1.459672in}{1.735075in}}%
\pgfpathlineto{\pgfqpoint{1.513726in}{1.974608in}}%
\pgfpathlineto{\pgfqpoint{1.567885in}{2.121750in}}%
\pgfpathlineto{\pgfqpoint{1.622461in}{2.271254in}}%
\pgfpathlineto{\pgfqpoint{1.676777in}{2.438090in}}%
\pgfpathlineto{\pgfqpoint{1.732580in}{2.495407in}}%
\pgfpathlineto{\pgfqpoint{1.785968in}{2.566171in}}%
\pgfpathlineto{\pgfqpoint{1.839810in}{2.612735in}}%
\pgfpathlineto{\pgfqpoint{1.895430in}{2.599490in}}%
\pgfpathlineto{\pgfqpoint{1.949144in}{2.564776in}}%
\pgfpathlineto{\pgfqpoint{2.003118in}{2.529145in}}%
\pgfpathlineto{\pgfqpoint{2.057395in}{2.605754in}}%
\pgfpathlineto{\pgfqpoint{2.111905in}{2.629714in}}%
\pgfpathlineto{\pgfqpoint{2.166086in}{2.638170in}}%
\pgfpathlineto{\pgfqpoint{2.220441in}{2.618301in}}%
\pgfpathlineto{\pgfqpoint{2.274857in}{2.623498in}}%
\pgfpathlineto{\pgfqpoint{2.329238in}{2.635121in}}%
\pgfpathlineto{\pgfqpoint{2.383534in}{2.635902in}}%
\pgfpathlineto{\pgfqpoint{2.437778in}{2.648804in}}%
\pgfpathlineto{\pgfqpoint{2.493618in}{2.671841in}}%
\pgfpathlineto{\pgfqpoint{2.546529in}{2.682473in}}%
\pgfpathlineto{\pgfqpoint{2.600779in}{2.669026in}}%
\pgfpathlineto{\pgfqpoint{2.654856in}{2.692702in}}%
\pgfpathlineto{\pgfqpoint{2.708956in}{2.699806in}}%
\pgfpathlineto{\pgfqpoint{2.763130in}{2.697607in}}%
\pgfpathlineto{\pgfqpoint{2.818699in}{2.697962in}}%
\pgfpathlineto{\pgfqpoint{2.873475in}{2.701502in}}%
\pgfpathlineto{\pgfqpoint{2.927831in}{2.701260in}}%
\pgfpathlineto{\pgfqpoint{2.982487in}{2.703808in}}%
\pgfpathlineto{\pgfqpoint{3.036270in}{2.687203in}}%
\pgfpathlineto{\pgfqpoint{3.091407in}{2.684534in}}%
\pgfpathlineto{\pgfqpoint{3.144716in}{2.691013in}}%
\pgfpathlineto{\pgfqpoint{3.198950in}{2.700802in}}%
\pgfpathlineto{\pgfqpoint{3.252844in}{2.713040in}}%
\pgfpathlineto{\pgfqpoint{3.307327in}{2.729152in}}%
\pgfpathlineto{\pgfqpoint{3.361373in}{2.719522in}}%
\pgfpathlineto{\pgfqpoint{3.415995in}{2.700441in}}%
\pgfpathlineto{\pgfqpoint{3.470101in}{2.684885in}}%
\pgfpathlineto{\pgfqpoint{3.524627in}{2.682579in}}%
\pgfpathlineto{\pgfqpoint{3.578793in}{2.682831in}}%
\pgfpathlineto{\pgfqpoint{3.633143in}{2.678058in}}%
\pgfpathlineto{\pgfqpoint{3.688462in}{2.681510in}}%
\pgfpathlineto{\pgfqpoint{3.742550in}{2.677496in}}%
\pgfpathlineto{\pgfqpoint{3.796250in}{2.655569in}}%
\pgfpathlineto{\pgfqpoint{3.850639in}{2.654595in}}%
\pgfpathlineto{\pgfqpoint{3.905218in}{2.636976in}}%
\pgfpathlineto{\pgfqpoint{3.959371in}{2.611631in}}%
\pgfpathlineto{\pgfqpoint{4.013749in}{2.597886in}}%
\pgfpathlineto{\pgfqpoint{4.068332in}{2.603644in}}%
\pgfpathlineto{\pgfqpoint{4.122103in}{2.599177in}}%
\pgfpathlineto{\pgfqpoint{4.177258in}{2.592689in}}%
\pgfpathlineto{\pgfqpoint{4.231580in}{2.599874in}}%
\pgfpathlineto{\pgfqpoint{4.285587in}{2.612256in}}%
\pgfpathlineto{\pgfqpoint{4.339574in}{2.627637in}}%
\pgfpathlineto{\pgfqpoint{4.393761in}{2.639495in}}%
\pgfpathlineto{\pgfqpoint{4.447979in}{2.675993in}}%
\pgfpathlineto{\pgfqpoint{4.502243in}{2.735802in}}%
\pgfpathlineto{\pgfqpoint{4.557236in}{2.753047in}}%
\pgfpathlineto{\pgfqpoint{4.611235in}{2.788945in}}%
\pgfpathlineto{\pgfqpoint{4.666785in}{2.795106in}}%
\pgfpathlineto{\pgfqpoint{4.719813in}{2.800519in}}%
\pgfpathlineto{\pgfqpoint{4.774165in}{2.784403in}}%
\pgfpathlineto{\pgfqpoint{4.828253in}{2.765541in}}%
\pgfpathlineto{\pgfqpoint{4.882612in}{2.732428in}}%
\pgfpathlineto{\pgfqpoint{4.936862in}{2.676171in}}%
\pgfpathlineto{\pgfqpoint{4.991280in}{2.695622in}}%
\pgfpathlineto{\pgfqpoint{5.045331in}{2.644176in}}%
\pgfpathlineto{\pgfqpoint{5.100650in}{2.632856in}}%
\pgfpathlineto{\pgfqpoint{5.155064in}{2.641538in}}%
\pgfpathlineto{\pgfqpoint{5.209813in}{2.624407in}}%
\pgfpathlineto{\pgfqpoint{5.265011in}{2.614804in}}%
\pgfpathlineto{\pgfqpoint{5.318305in}{2.607001in}}%
\pgfpathlineto{\pgfqpoint{5.372376in}{2.607891in}}%
\pgfpathlineto{\pgfqpoint{5.426214in}{2.611168in}}%
\pgfpathlineto{\pgfqpoint{5.480431in}{2.612760in}}%
\pgfpathlineto{\pgfqpoint{5.534545in}{2.614611in}}%
\pgfusepath{stroke}%
\end{pgfscope}%
\begin{pgfscope}%
\pgfpathrectangle{\pgfqpoint{0.800000in}{0.528000in}}{\pgfqpoint{4.960000in}{3.696000in}}%
\pgfusepath{clip}%
\pgfsetrectcap%
\pgfsetroundjoin%
\pgfsetlinewidth{1.505625pt}%
\definecolor{currentstroke}{rgb}{0.172549,0.627451,0.172549}%
\pgfsetstrokecolor{currentstroke}%
\pgfsetdash{}{0pt}%
\pgfpathmoveto{\pgfqpoint{1.025455in}{0.696000in}}%
\pgfpathlineto{\pgfqpoint{1.079338in}{0.696000in}}%
\pgfpathlineto{\pgfqpoint{1.134373in}{0.696000in}}%
\pgfpathlineto{\pgfqpoint{1.188140in}{0.696000in}}%
\pgfpathlineto{\pgfqpoint{1.242541in}{0.696000in}}%
\pgfpathlineto{\pgfqpoint{1.296558in}{0.696000in}}%
\pgfpathlineto{\pgfqpoint{1.351086in}{0.696000in}}%
\pgfpathlineto{\pgfqpoint{1.403899in}{0.696000in}}%
\pgfpathlineto{\pgfqpoint{1.457826in}{0.696000in}}%
\pgfpathlineto{\pgfqpoint{1.513266in}{1.680625in}}%
\pgfpathlineto{\pgfqpoint{1.566904in}{1.154142in}}%
\pgfpathlineto{\pgfqpoint{1.620778in}{1.379113in}}%
\pgfpathlineto{\pgfqpoint{1.674915in}{1.821629in}}%
\pgfpathlineto{\pgfqpoint{1.729322in}{2.024534in}}%
\pgfpathlineto{\pgfqpoint{1.783620in}{2.201427in}}%
\pgfpathlineto{\pgfqpoint{1.837657in}{2.394670in}}%
\pgfpathlineto{\pgfqpoint{1.891863in}{2.596750in}}%
\pgfpathlineto{\pgfqpoint{1.946086in}{2.572826in}}%
\pgfpathlineto{\pgfqpoint{2.000445in}{2.589924in}}%
\pgfpathlineto{\pgfqpoint{2.054578in}{2.572622in}}%
\pgfpathlineto{\pgfqpoint{2.110349in}{2.675383in}}%
\pgfpathlineto{\pgfqpoint{2.164018in}{2.742016in}}%
\pgfpathlineto{\pgfqpoint{2.217545in}{2.843754in}}%
\pgfpathlineto{\pgfqpoint{2.271883in}{2.976579in}}%
\pgfpathlineto{\pgfqpoint{2.325902in}{3.010195in}}%
\pgfpathlineto{\pgfqpoint{2.380461in}{3.008370in}}%
\pgfpathlineto{\pgfqpoint{2.435017in}{3.074375in}}%
\pgfpathlineto{\pgfqpoint{2.489119in}{3.161236in}}%
\pgfpathlineto{\pgfqpoint{2.544546in}{3.139227in}}%
\pgfpathlineto{\pgfqpoint{2.600533in}{3.121628in}}%
\pgfpathlineto{\pgfqpoint{2.654479in}{3.046125in}}%
\pgfpathlineto{\pgfqpoint{2.707662in}{3.048219in}}%
\pgfpathlineto{\pgfqpoint{2.762151in}{2.979731in}}%
\pgfpathlineto{\pgfqpoint{2.816214in}{3.030378in}}%
\pgfpathlineto{\pgfqpoint{2.870221in}{3.026311in}}%
\pgfpathlineto{\pgfqpoint{2.924444in}{3.209543in}}%
\pgfpathlineto{\pgfqpoint{2.978898in}{3.189934in}}%
\pgfpathlineto{\pgfqpoint{3.033114in}{3.190221in}}%
\pgfpathlineto{\pgfqpoint{3.087068in}{3.224333in}}%
\pgfpathlineto{\pgfqpoint{3.141341in}{3.256693in}}%
\pgfpathlineto{\pgfqpoint{3.195661in}{3.316553in}}%
\pgfpathlineto{\pgfqpoint{3.250331in}{3.162720in}}%
\pgfpathlineto{\pgfqpoint{3.304263in}{2.812077in}}%
\pgfpathlineto{\pgfqpoint{3.358421in}{2.491062in}}%
\pgfpathlineto{\pgfqpoint{3.412925in}{2.811518in}}%
\pgfpathlineto{\pgfqpoint{3.468686in}{2.808630in}}%
\pgfpathlineto{\pgfqpoint{3.521890in}{2.677411in}}%
\pgfpathlineto{\pgfqpoint{3.575593in}{2.690362in}}%
\pgfpathlineto{\pgfqpoint{3.630199in}{2.515772in}}%
\pgfpathlineto{\pgfqpoint{3.684605in}{2.627683in}}%
\pgfpathlineto{\pgfqpoint{3.738863in}{2.596218in}}%
\pgfpathlineto{\pgfqpoint{3.793823in}{2.575716in}}%
\pgfpathlineto{\pgfqpoint{3.847600in}{2.559622in}}%
\pgfpathlineto{\pgfqpoint{3.901804in}{2.556415in}}%
\pgfpathlineto{\pgfqpoint{3.956167in}{2.541477in}}%
\pgfpathlineto{\pgfqpoint{4.012325in}{2.555751in}}%
\pgfpathlineto{\pgfqpoint{4.065420in}{2.685526in}}%
\pgfpathlineto{\pgfqpoint{4.119116in}{2.650058in}}%
\pgfpathlineto{\pgfqpoint{4.173526in}{2.577637in}}%
\pgfpathlineto{\pgfqpoint{4.228117in}{2.572657in}}%
\pgfpathlineto{\pgfqpoint{4.284293in}{2.590043in}}%
\pgfpathlineto{\pgfqpoint{4.336210in}{2.625349in}}%
\pgfpathlineto{\pgfqpoint{4.390143in}{2.720315in}}%
\pgfpathlineto{\pgfqpoint{4.445557in}{2.800543in}}%
\pgfpathlineto{\pgfqpoint{4.499281in}{2.769331in}}%
\pgfpathlineto{\pgfqpoint{4.554693in}{2.870938in}}%
\pgfpathlineto{\pgfqpoint{4.608759in}{2.910073in}}%
\pgfpathlineto{\pgfqpoint{4.663266in}{2.939263in}}%
\pgfpathlineto{\pgfqpoint{4.716945in}{2.908419in}}%
\pgfpathlineto{\pgfqpoint{4.771111in}{2.719099in}}%
\pgfpathlineto{\pgfqpoint{4.825139in}{2.697720in}}%
\pgfpathlineto{\pgfqpoint{4.879672in}{2.729384in}}%
\pgfpathlineto{\pgfqpoint{4.937695in}{2.676350in}}%
\pgfpathlineto{\pgfqpoint{4.989674in}{2.630349in}}%
\pgfpathlineto{\pgfqpoint{5.043158in}{2.662653in}}%
\pgfpathlineto{\pgfqpoint{5.097306in}{2.607626in}}%
\pgfpathlineto{\pgfqpoint{5.151255in}{2.617772in}}%
\pgfpathlineto{\pgfqpoint{5.205714in}{2.598868in}}%
\pgfpathlineto{\pgfqpoint{5.259536in}{2.597974in}}%
\pgfpathlineto{\pgfqpoint{5.314236in}{2.595231in}}%
\pgfpathlineto{\pgfqpoint{5.368442in}{2.568985in}}%
\pgfpathlineto{\pgfqpoint{5.422763in}{2.535470in}}%
\pgfpathlineto{\pgfqpoint{5.477325in}{2.523669in}}%
\pgfpathlineto{\pgfqpoint{5.531315in}{2.513765in}}%
\pgfusepath{stroke}%
\end{pgfscope}%
\begin{pgfscope}%
\pgfpathrectangle{\pgfqpoint{0.800000in}{0.528000in}}{\pgfqpoint{4.960000in}{3.696000in}}%
\pgfusepath{clip}%
\pgfsetrectcap%
\pgfsetroundjoin%
\pgfsetlinewidth{1.505625pt}%
\definecolor{currentstroke}{rgb}{0.839216,0.152941,0.156863}%
\pgfsetstrokecolor{currentstroke}%
\pgfsetdash{}{0pt}%
\pgfpathmoveto{\pgfqpoint{1.025455in}{0.696000in}}%
\pgfpathlineto{\pgfqpoint{1.079563in}{0.696000in}}%
\pgfpathlineto{\pgfqpoint{1.133745in}{0.696000in}}%
\pgfpathlineto{\pgfqpoint{1.188369in}{0.696000in}}%
\pgfpathlineto{\pgfqpoint{1.242836in}{0.696000in}}%
\pgfpathlineto{\pgfqpoint{1.296352in}{0.696000in}}%
\pgfpathlineto{\pgfqpoint{1.350170in}{0.696000in}}%
\pgfpathlineto{\pgfqpoint{1.405568in}{0.839464in}}%
\pgfpathlineto{\pgfqpoint{1.459476in}{0.710592in}}%
\pgfpathlineto{\pgfqpoint{1.513382in}{1.028026in}}%
\pgfpathlineto{\pgfqpoint{1.567290in}{2.143834in}}%
\pgfpathlineto{\pgfqpoint{1.622195in}{1.463407in}}%
\pgfpathlineto{\pgfqpoint{1.675724in}{1.898106in}}%
\pgfpathlineto{\pgfqpoint{1.730171in}{2.255800in}}%
\pgfpathlineto{\pgfqpoint{1.784643in}{2.353807in}}%
\pgfpathlineto{\pgfqpoint{1.838886in}{2.558080in}}%
\pgfpathlineto{\pgfqpoint{1.894019in}{2.803100in}}%
\pgfpathlineto{\pgfqpoint{1.948113in}{2.931994in}}%
\pgfpathlineto{\pgfqpoint{2.003267in}{2.915494in}}%
\pgfpathlineto{\pgfqpoint{2.057163in}{3.282687in}}%
\pgfpathlineto{\pgfqpoint{2.111040in}{3.270895in}}%
\pgfpathlineto{\pgfqpoint{2.165065in}{3.326813in}}%
\pgfpathlineto{\pgfqpoint{2.219893in}{3.631442in}}%
\pgfpathlineto{\pgfqpoint{2.273806in}{3.678101in}}%
\pgfpathlineto{\pgfqpoint{2.328151in}{3.872822in}}%
\pgfpathlineto{\pgfqpoint{2.382662in}{4.011699in}}%
\pgfpathlineto{\pgfqpoint{2.436713in}{3.464741in}}%
\pgfpathlineto{\pgfqpoint{2.491306in}{3.580491in}}%
\pgfpathlineto{\pgfqpoint{2.545562in}{3.593537in}}%
\pgfpathlineto{\pgfqpoint{2.600972in}{3.593824in}}%
\pgfpathlineto{\pgfqpoint{2.654416in}{3.399111in}}%
\pgfpathlineto{\pgfqpoint{2.708411in}{3.160160in}}%
\pgfpathlineto{\pgfqpoint{2.762550in}{2.716473in}}%
\pgfpathlineto{\pgfqpoint{2.816965in}{2.375725in}}%
\pgfpathlineto{\pgfqpoint{2.871193in}{1.774421in}}%
\pgfpathlineto{\pgfqpoint{2.925747in}{1.339225in}}%
\pgfpathlineto{\pgfqpoint{2.980559in}{1.245539in}}%
\pgfpathlineto{\pgfqpoint{3.034734in}{1.751591in}}%
\pgfpathlineto{\pgfqpoint{3.088855in}{1.985455in}}%
\pgfpathlineto{\pgfqpoint{3.144324in}{2.172376in}}%
\pgfpathlineto{\pgfqpoint{3.199059in}{2.355031in}}%
\pgfpathlineto{\pgfqpoint{3.252618in}{3.382478in}}%
\pgfpathlineto{\pgfqpoint{3.306665in}{2.391368in}}%
\pgfpathlineto{\pgfqpoint{3.360795in}{1.761713in}}%
\pgfpathlineto{\pgfqpoint{3.414952in}{2.876786in}}%
\pgfpathlineto{\pgfqpoint{3.469285in}{3.002575in}}%
\pgfpathlineto{\pgfqpoint{3.523559in}{3.122090in}}%
\pgfpathlineto{\pgfqpoint{3.578142in}{3.083353in}}%
\pgfpathlineto{\pgfqpoint{3.631962in}{3.338605in}}%
\pgfpathlineto{\pgfqpoint{3.686230in}{3.278447in}}%
\pgfpathlineto{\pgfqpoint{3.742106in}{3.437641in}}%
\pgfpathlineto{\pgfqpoint{3.795478in}{3.403811in}}%
\pgfpathlineto{\pgfqpoint{3.849263in}{2.636735in}}%
\pgfpathlineto{\pgfqpoint{3.903426in}{3.115859in}}%
\pgfpathlineto{\pgfqpoint{3.957571in}{3.050075in}}%
\pgfpathlineto{\pgfqpoint{4.011819in}{2.749272in}}%
\pgfpathlineto{\pgfqpoint{4.066240in}{2.465754in}}%
\pgfpathlineto{\pgfqpoint{4.120524in}{2.444555in}}%
\pgfpathlineto{\pgfqpoint{4.174846in}{2.638897in}}%
\pgfpathlineto{\pgfqpoint{4.230263in}{2.632839in}}%
\pgfpathlineto{\pgfqpoint{4.283570in}{2.599501in}}%
\pgfpathlineto{\pgfqpoint{4.337704in}{2.568936in}}%
\pgfpathlineto{\pgfqpoint{4.392219in}{2.390986in}}%
\pgfpathlineto{\pgfqpoint{4.446270in}{2.360731in}}%
\pgfpathlineto{\pgfqpoint{4.500787in}{2.189210in}}%
\pgfpathlineto{\pgfqpoint{4.555204in}{2.091303in}}%
\pgfpathlineto{\pgfqpoint{4.609805in}{2.084749in}}%
\pgfpathlineto{\pgfqpoint{4.664059in}{2.578155in}}%
\pgfpathlineto{\pgfqpoint{4.718797in}{2.435383in}}%
\pgfpathlineto{\pgfqpoint{4.772885in}{2.577447in}}%
\pgfpathlineto{\pgfqpoint{4.828486in}{2.484839in}}%
\pgfpathlineto{\pgfqpoint{4.882605in}{2.552813in}}%
\pgfpathlineto{\pgfqpoint{4.936386in}{2.679223in}}%
\pgfpathlineto{\pgfqpoint{4.990649in}{2.726569in}}%
\pgfpathlineto{\pgfqpoint{5.045036in}{2.629952in}}%
\pgfpathlineto{\pgfqpoint{5.099396in}{2.684494in}}%
\pgfpathlineto{\pgfqpoint{5.153858in}{2.682784in}}%
\pgfpathlineto{\pgfqpoint{5.207915in}{2.759045in}}%
\pgfpathlineto{\pgfqpoint{5.262469in}{2.712827in}}%
\pgfpathlineto{\pgfqpoint{5.316492in}{2.789430in}}%
\pgfpathlineto{\pgfqpoint{5.370799in}{2.756283in}}%
\pgfpathlineto{\pgfqpoint{5.426493in}{2.802156in}}%
\pgfpathlineto{\pgfqpoint{5.480191in}{2.851618in}}%
\pgfpathlineto{\pgfqpoint{5.533861in}{2.853767in}}%
\pgfusepath{stroke}%
\end{pgfscope}%
\begin{pgfscope}%
\pgfpathrectangle{\pgfqpoint{0.800000in}{0.528000in}}{\pgfqpoint{4.960000in}{3.696000in}}%
\pgfusepath{clip}%
\pgfsetrectcap%
\pgfsetroundjoin%
\pgfsetlinewidth{1.505625pt}%
\definecolor{currentstroke}{rgb}{0.580392,0.403922,0.741176}%
\pgfsetstrokecolor{currentstroke}%
\pgfsetdash{}{0pt}%
\pgfpathmoveto{\pgfqpoint{1.025455in}{0.696000in}}%
\pgfpathlineto{\pgfqpoint{1.079158in}{0.696000in}}%
\pgfpathlineto{\pgfqpoint{1.133625in}{0.696000in}}%
\pgfpathlineto{\pgfqpoint{1.187998in}{0.696000in}}%
\pgfpathlineto{\pgfqpoint{1.242592in}{0.696000in}}%
\pgfpathlineto{\pgfqpoint{1.297030in}{0.696000in}}%
\pgfpathlineto{\pgfqpoint{1.351098in}{0.696000in}}%
\pgfpathlineto{\pgfqpoint{1.404136in}{0.696000in}}%
\pgfpathlineto{\pgfqpoint{1.458220in}{0.880427in}}%
\pgfpathlineto{\pgfqpoint{1.512633in}{1.094228in}}%
\pgfpathlineto{\pgfqpoint{1.567117in}{1.366844in}}%
\pgfpathlineto{\pgfqpoint{1.622859in}{1.787953in}}%
\pgfpathlineto{\pgfqpoint{1.676422in}{1.919152in}}%
\pgfpathlineto{\pgfqpoint{1.730222in}{2.336805in}}%
\pgfpathlineto{\pgfqpoint{1.784392in}{2.699996in}}%
\pgfpathlineto{\pgfqpoint{1.838589in}{2.990818in}}%
\pgfpathlineto{\pgfqpoint{1.892759in}{3.125481in}}%
\pgfpathlineto{\pgfqpoint{1.946836in}{3.334299in}}%
\pgfpathlineto{\pgfqpoint{2.001194in}{3.264340in}}%
\pgfpathlineto{\pgfqpoint{2.055235in}{3.679475in}}%
\pgfpathlineto{\pgfqpoint{2.109634in}{3.784561in}}%
\pgfpathlineto{\pgfqpoint{2.163983in}{4.056000in}}%
\pgfpathlineto{\pgfqpoint{2.218154in}{3.627054in}}%
\pgfpathlineto{\pgfqpoint{2.273665in}{3.711496in}}%
\pgfpathlineto{\pgfqpoint{2.327050in}{3.398941in}}%
\pgfpathlineto{\pgfqpoint{2.381143in}{2.949234in}}%
\pgfpathlineto{\pgfqpoint{2.435175in}{3.751869in}}%
\pgfpathlineto{\pgfqpoint{2.489389in}{3.420832in}}%
\pgfpathlineto{\pgfqpoint{2.543680in}{3.153376in}}%
\pgfpathlineto{\pgfqpoint{2.598429in}{3.229428in}}%
\pgfpathlineto{\pgfqpoint{2.652754in}{2.984517in}}%
\pgfpathlineto{\pgfqpoint{2.706983in}{2.361432in}}%
\pgfpathlineto{\pgfqpoint{2.761159in}{2.672353in}}%
\pgfpathlineto{\pgfqpoint{2.815621in}{2.509461in}}%
\pgfpathlineto{\pgfqpoint{2.870024in}{2.144681in}}%
\pgfpathlineto{\pgfqpoint{2.925271in}{1.935035in}}%
\pgfpathlineto{\pgfqpoint{2.978818in}{1.766450in}}%
\pgfpathlineto{\pgfqpoint{3.035845in}{1.819980in}}%
\pgfpathlineto{\pgfqpoint{3.087551in}{1.595083in}}%
\pgfpathlineto{\pgfqpoint{3.142088in}{1.528783in}}%
\pgfpathlineto{\pgfqpoint{3.196778in}{1.589907in}}%
\pgfpathlineto{\pgfqpoint{3.250727in}{1.979305in}}%
\pgfpathlineto{\pgfqpoint{3.305079in}{2.156584in}}%
\pgfpathlineto{\pgfqpoint{3.359364in}{2.055926in}}%
\pgfpathlineto{\pgfqpoint{3.414288in}{2.516961in}}%
\pgfpathlineto{\pgfqpoint{3.468158in}{2.625467in}}%
\pgfpathlineto{\pgfqpoint{3.523353in}{2.587113in}}%
\pgfpathlineto{\pgfqpoint{3.577287in}{2.579856in}}%
\pgfpathlineto{\pgfqpoint{3.631162in}{2.801379in}}%
\pgfpathlineto{\pgfqpoint{3.685684in}{2.996738in}}%
\pgfpathlineto{\pgfqpoint{3.739819in}{3.083412in}}%
\pgfpathlineto{\pgfqpoint{3.793889in}{3.004589in}}%
\pgfpathlineto{\pgfqpoint{3.849315in}{3.024605in}}%
\pgfpathlineto{\pgfqpoint{3.903591in}{3.280696in}}%
\pgfpathlineto{\pgfqpoint{3.957883in}{2.909734in}}%
\pgfpathlineto{\pgfqpoint{4.012780in}{3.127494in}}%
\pgfpathlineto{\pgfqpoint{4.066745in}{3.239783in}}%
\pgfpathlineto{\pgfqpoint{4.120934in}{3.347864in}}%
\pgfpathlineto{\pgfqpoint{4.176212in}{3.211871in}}%
\pgfpathlineto{\pgfqpoint{4.229959in}{3.554640in}}%
\pgfpathlineto{\pgfqpoint{4.283907in}{3.318638in}}%
\pgfpathlineto{\pgfqpoint{4.337780in}{2.414106in}}%
\pgfpathlineto{\pgfqpoint{4.392392in}{2.619295in}}%
\pgfpathlineto{\pgfqpoint{4.446739in}{2.600942in}}%
\pgfpathlineto{\pgfqpoint{4.501050in}{2.446648in}}%
\pgfpathlineto{\pgfqpoint{4.555178in}{2.192358in}}%
\pgfpathlineto{\pgfqpoint{4.611013in}{2.372959in}}%
\pgfpathlineto{\pgfqpoint{4.664073in}{2.280791in}}%
\pgfpathlineto{\pgfqpoint{4.718174in}{2.232296in}}%
\pgfpathlineto{\pgfqpoint{4.772397in}{2.071880in}}%
\pgfpathlineto{\pgfqpoint{4.826437in}{2.000102in}}%
\pgfpathlineto{\pgfqpoint{4.880571in}{2.000615in}}%
\pgfpathlineto{\pgfqpoint{4.935390in}{2.716838in}}%
\pgfpathlineto{\pgfqpoint{4.989268in}{2.188324in}}%
\pgfpathlineto{\pgfqpoint{5.044055in}{1.953160in}}%
\pgfpathlineto{\pgfqpoint{5.098236in}{2.477334in}}%
\pgfpathlineto{\pgfqpoint{5.152541in}{2.953913in}}%
\pgfpathlineto{\pgfqpoint{5.206757in}{2.778896in}}%
\pgfpathlineto{\pgfqpoint{5.262378in}{2.504760in}}%
\pgfpathlineto{\pgfqpoint{5.316255in}{2.712324in}}%
\pgfpathlineto{\pgfqpoint{5.370230in}{2.742947in}}%
\pgfpathlineto{\pgfqpoint{5.424284in}{2.864851in}}%
\pgfpathlineto{\pgfqpoint{5.478769in}{2.880730in}}%
\pgfpathlineto{\pgfqpoint{5.532911in}{2.973025in}}%
\pgfusepath{stroke}%
\end{pgfscope}%
\begin{pgfscope}%
\pgfsetrectcap%
\pgfsetmiterjoin%
\pgfsetlinewidth{0.803000pt}%
\definecolor{currentstroke}{rgb}{0.000000,0.000000,0.000000}%
\pgfsetstrokecolor{currentstroke}%
\pgfsetdash{}{0pt}%
\pgfpathmoveto{\pgfqpoint{0.800000in}{0.528000in}}%
\pgfpathlineto{\pgfqpoint{0.800000in}{4.224000in}}%
\pgfusepath{stroke}%
\end{pgfscope}%
\begin{pgfscope}%
\pgfsetrectcap%
\pgfsetmiterjoin%
\pgfsetlinewidth{0.803000pt}%
\definecolor{currentstroke}{rgb}{0.000000,0.000000,0.000000}%
\pgfsetstrokecolor{currentstroke}%
\pgfsetdash{}{0pt}%
\pgfpathmoveto{\pgfqpoint{5.760000in}{0.528000in}}%
\pgfpathlineto{\pgfqpoint{5.760000in}{4.224000in}}%
\pgfusepath{stroke}%
\end{pgfscope}%
\begin{pgfscope}%
\pgfsetrectcap%
\pgfsetmiterjoin%
\pgfsetlinewidth{0.803000pt}%
\definecolor{currentstroke}{rgb}{0.000000,0.000000,0.000000}%
\pgfsetstrokecolor{currentstroke}%
\pgfsetdash{}{0pt}%
\pgfpathmoveto{\pgfqpoint{0.800000in}{0.528000in}}%
\pgfpathlineto{\pgfqpoint{5.760000in}{0.528000in}}%
\pgfusepath{stroke}%
\end{pgfscope}%
\begin{pgfscope}%
\pgfsetrectcap%
\pgfsetmiterjoin%
\pgfsetlinewidth{0.803000pt}%
\definecolor{currentstroke}{rgb}{0.000000,0.000000,0.000000}%
\pgfsetstrokecolor{currentstroke}%
\pgfsetdash{}{0pt}%
\pgfpathmoveto{\pgfqpoint{0.800000in}{4.224000in}}%
\pgfpathlineto{\pgfqpoint{5.760000in}{4.224000in}}%
\pgfusepath{stroke}%
\end{pgfscope}%
\begin{pgfscope}%
\definecolor{textcolor}{rgb}{0.000000,0.000000,0.000000}%
\pgfsetstrokecolor{textcolor}%
\pgfsetfillcolor{textcolor}%
\pgftext[x=3.280000in,y=4.307333in,,base]{\color{textcolor}\sffamily\fontsize{12.000000}{14.400000}\selectfont Forward controller output}%
\end{pgfscope}%
\begin{pgfscope}%
\pgfsetbuttcap%
\pgfsetmiterjoin%
\definecolor{currentfill}{rgb}{1.000000,1.000000,1.000000}%
\pgfsetfillcolor{currentfill}%
\pgfsetfillopacity{0.800000}%
\pgfsetlinewidth{1.003750pt}%
\definecolor{currentstroke}{rgb}{0.800000,0.800000,0.800000}%
\pgfsetstrokecolor{currentstroke}%
\pgfsetstrokeopacity{0.800000}%
\pgfsetdash{}{0pt}%
\pgfpathmoveto{\pgfqpoint{4.997454in}{3.093603in}}%
\pgfpathlineto{\pgfqpoint{5.662778in}{3.093603in}}%
\pgfpathquadraticcurveto{\pgfqpoint{5.690556in}{3.093603in}}{\pgfqpoint{5.690556in}{3.121381in}}%
\pgfpathlineto{\pgfqpoint{5.690556in}{4.126778in}}%
\pgfpathquadraticcurveto{\pgfqpoint{5.690556in}{4.154556in}}{\pgfqpoint{5.662778in}{4.154556in}}%
\pgfpathlineto{\pgfqpoint{4.997454in}{4.154556in}}%
\pgfpathquadraticcurveto{\pgfqpoint{4.969676in}{4.154556in}}{\pgfqpoint{4.969676in}{4.126778in}}%
\pgfpathlineto{\pgfqpoint{4.969676in}{3.121381in}}%
\pgfpathquadraticcurveto{\pgfqpoint{4.969676in}{3.093603in}}{\pgfqpoint{4.997454in}{3.093603in}}%
\pgfpathlineto{\pgfqpoint{4.997454in}{3.093603in}}%
\pgfpathclose%
\pgfusepath{stroke,fill}%
\end{pgfscope}%
\begin{pgfscope}%
\pgfsetrectcap%
\pgfsetroundjoin%
\pgfsetlinewidth{1.505625pt}%
\definecolor{currentstroke}{rgb}{0.121569,0.466667,0.705882}%
\pgfsetstrokecolor{currentstroke}%
\pgfsetdash{}{0pt}%
\pgfpathmoveto{\pgfqpoint{5.025232in}{4.042088in}}%
\pgfpathlineto{\pgfqpoint{5.164121in}{4.042088in}}%
\pgfpathlineto{\pgfqpoint{5.303009in}{4.042088in}}%
\pgfusepath{stroke}%
\end{pgfscope}%
\begin{pgfscope}%
\definecolor{textcolor}{rgb}{0.000000,0.000000,0.000000}%
\pgfsetstrokecolor{textcolor}%
\pgfsetfillcolor{textcolor}%
\pgftext[x=5.414121in,y=3.993477in,left,base]{\color{textcolor}\sffamily\fontsize{10.000000}{12.000000}\selectfont 0}%
\end{pgfscope}%
\begin{pgfscope}%
\pgfsetrectcap%
\pgfsetroundjoin%
\pgfsetlinewidth{1.505625pt}%
\definecolor{currentstroke}{rgb}{1.000000,0.498039,0.054902}%
\pgfsetstrokecolor{currentstroke}%
\pgfsetdash{}{0pt}%
\pgfpathmoveto{\pgfqpoint{5.025232in}{3.838231in}}%
\pgfpathlineto{\pgfqpoint{5.164121in}{3.838231in}}%
\pgfpathlineto{\pgfqpoint{5.303009in}{3.838231in}}%
\pgfusepath{stroke}%
\end{pgfscope}%
\begin{pgfscope}%
\definecolor{textcolor}{rgb}{0.000000,0.000000,0.000000}%
\pgfsetstrokecolor{textcolor}%
\pgfsetfillcolor{textcolor}%
\pgftext[x=5.414121in,y=3.789620in,left,base]{\color{textcolor}\sffamily\fontsize{10.000000}{12.000000}\selectfont 0.5}%
\end{pgfscope}%
\begin{pgfscope}%
\pgfsetrectcap%
\pgfsetroundjoin%
\pgfsetlinewidth{1.505625pt}%
\definecolor{currentstroke}{rgb}{0.172549,0.627451,0.172549}%
\pgfsetstrokecolor{currentstroke}%
\pgfsetdash{}{0pt}%
\pgfpathmoveto{\pgfqpoint{5.025232in}{3.634374in}}%
\pgfpathlineto{\pgfqpoint{5.164121in}{3.634374in}}%
\pgfpathlineto{\pgfqpoint{5.303009in}{3.634374in}}%
\pgfusepath{stroke}%
\end{pgfscope}%
\begin{pgfscope}%
\definecolor{textcolor}{rgb}{0.000000,0.000000,0.000000}%
\pgfsetstrokecolor{textcolor}%
\pgfsetfillcolor{textcolor}%
\pgftext[x=5.414121in,y=3.585762in,left,base]{\color{textcolor}\sffamily\fontsize{10.000000}{12.000000}\selectfont 1}%
\end{pgfscope}%
\begin{pgfscope}%
\pgfsetrectcap%
\pgfsetroundjoin%
\pgfsetlinewidth{1.505625pt}%
\definecolor{currentstroke}{rgb}{0.839216,0.152941,0.156863}%
\pgfsetstrokecolor{currentstroke}%
\pgfsetdash{}{0pt}%
\pgfpathmoveto{\pgfqpoint{5.025232in}{3.430516in}}%
\pgfpathlineto{\pgfqpoint{5.164121in}{3.430516in}}%
\pgfpathlineto{\pgfqpoint{5.303009in}{3.430516in}}%
\pgfusepath{stroke}%
\end{pgfscope}%
\begin{pgfscope}%
\definecolor{textcolor}{rgb}{0.000000,0.000000,0.000000}%
\pgfsetstrokecolor{textcolor}%
\pgfsetfillcolor{textcolor}%
\pgftext[x=5.414121in,y=3.381905in,left,base]{\color{textcolor}\sffamily\fontsize{10.000000}{12.000000}\selectfont 2}%
\end{pgfscope}%
\begin{pgfscope}%
\pgfsetrectcap%
\pgfsetroundjoin%
\pgfsetlinewidth{1.505625pt}%
\definecolor{currentstroke}{rgb}{0.580392,0.403922,0.741176}%
\pgfsetstrokecolor{currentstroke}%
\pgfsetdash{}{0pt}%
\pgfpathmoveto{\pgfqpoint{5.025232in}{3.226659in}}%
\pgfpathlineto{\pgfqpoint{5.164121in}{3.226659in}}%
\pgfpathlineto{\pgfqpoint{5.303009in}{3.226659in}}%
\pgfusepath{stroke}%
\end{pgfscope}%
\begin{pgfscope}%
\definecolor{textcolor}{rgb}{0.000000,0.000000,0.000000}%
\pgfsetstrokecolor{textcolor}%
\pgfsetfillcolor{textcolor}%
\pgftext[x=5.414121in,y=3.178048in,left,base]{\color{textcolor}\sffamily\fontsize{10.000000}{12.000000}\selectfont 3}%
\end{pgfscope}%
\end{pgfpicture}%
\makeatother%
\endgroup%
}
    \end{minipage}
    \caption{Variation of (a) computed error and (b) output velocity for different values of $K_{D}$ and $K_P=4$, $K_I=1$ while the forward controller is engaged.}
    \label{fig:tune-fwd-der-io}
\end{figure}
\begin{figure}[H]
    \begin{minipage}[t]{0.5\linewidth}
        \centering
        \scalebox{0.55}{%% Creator: Matplotlib, PGF backend
%%
%% To include the figure in your LaTeX document, write
%%   \input{<filename>.pgf}
%%
%% Make sure the required packages are loaded in your preamble
%%   \usepackage{pgf}
%%
%% Also ensure that all the required font packages are loaded; for instance,
%% the lmodern package is sometimes necessary when using math font.
%%   \usepackage{lmodern}
%%
%% Figures using additional raster images can only be included by \input if
%% they are in the same directory as the main LaTeX file. For loading figures
%% from other directories you can use the `import` package
%%   \usepackage{import}
%%
%% and then include the figures with
%%   \import{<path to file>}{<filename>.pgf}
%%
%% Matplotlib used the following preamble
%%   \usepackage{fontspec}
%%   \setmainfont{DejaVuSerif.ttf}[Path=\detokenize{/home/lgonz/tfg-aero/tfg-giaa-dronecontrol/venv/lib/python3.8/site-packages/matplotlib/mpl-data/fonts/ttf/}]
%%   \setsansfont{DejaVuSans.ttf}[Path=\detokenize{/home/lgonz/tfg-aero/tfg-giaa-dronecontrol/venv/lib/python3.8/site-packages/matplotlib/mpl-data/fonts/ttf/}]
%%   \setmonofont{DejaVuSansMono.ttf}[Path=\detokenize{/home/lgonz/tfg-aero/tfg-giaa-dronecontrol/venv/lib/python3.8/site-packages/matplotlib/mpl-data/fonts/ttf/}]
%%
\begingroup%
\makeatletter%
\begin{pgfpicture}%
\pgfpathrectangle{\pgfpointorigin}{\pgfqpoint{6.400000in}{4.800000in}}%
\pgfusepath{use as bounding box, clip}%
\begin{pgfscope}%
\pgfsetbuttcap%
\pgfsetmiterjoin%
\definecolor{currentfill}{rgb}{1.000000,1.000000,1.000000}%
\pgfsetfillcolor{currentfill}%
\pgfsetlinewidth{0.000000pt}%
\definecolor{currentstroke}{rgb}{1.000000,1.000000,1.000000}%
\pgfsetstrokecolor{currentstroke}%
\pgfsetdash{}{0pt}%
\pgfpathmoveto{\pgfqpoint{0.000000in}{0.000000in}}%
\pgfpathlineto{\pgfqpoint{6.400000in}{0.000000in}}%
\pgfpathlineto{\pgfqpoint{6.400000in}{4.800000in}}%
\pgfpathlineto{\pgfqpoint{0.000000in}{4.800000in}}%
\pgfpathlineto{\pgfqpoint{0.000000in}{0.000000in}}%
\pgfpathclose%
\pgfusepath{fill}%
\end{pgfscope}%
\begin{pgfscope}%
\pgfsetbuttcap%
\pgfsetmiterjoin%
\definecolor{currentfill}{rgb}{1.000000,1.000000,1.000000}%
\pgfsetfillcolor{currentfill}%
\pgfsetlinewidth{0.000000pt}%
\definecolor{currentstroke}{rgb}{0.000000,0.000000,0.000000}%
\pgfsetstrokecolor{currentstroke}%
\pgfsetstrokeopacity{0.000000}%
\pgfsetdash{}{0pt}%
\pgfpathmoveto{\pgfqpoint{0.800000in}{0.528000in}}%
\pgfpathlineto{\pgfqpoint{5.760000in}{0.528000in}}%
\pgfpathlineto{\pgfqpoint{5.760000in}{4.224000in}}%
\pgfpathlineto{\pgfqpoint{0.800000in}{4.224000in}}%
\pgfpathlineto{\pgfqpoint{0.800000in}{0.528000in}}%
\pgfpathclose%
\pgfusepath{fill}%
\end{pgfscope}%
\begin{pgfscope}%
\pgfpathrectangle{\pgfqpoint{0.800000in}{0.528000in}}{\pgfqpoint{4.960000in}{3.696000in}}%
\pgfusepath{clip}%
\pgfsetrectcap%
\pgfsetroundjoin%
\pgfsetlinewidth{0.803000pt}%
\definecolor{currentstroke}{rgb}{0.690196,0.690196,0.690196}%
\pgfsetstrokecolor{currentstroke}%
\pgfsetdash{}{0pt}%
\pgfpathmoveto{\pgfqpoint{1.025455in}{0.528000in}}%
\pgfpathlineto{\pgfqpoint{1.025455in}{4.224000in}}%
\pgfusepath{stroke}%
\end{pgfscope}%
\begin{pgfscope}%
\pgfsetbuttcap%
\pgfsetroundjoin%
\definecolor{currentfill}{rgb}{0.000000,0.000000,0.000000}%
\pgfsetfillcolor{currentfill}%
\pgfsetlinewidth{0.803000pt}%
\definecolor{currentstroke}{rgb}{0.000000,0.000000,0.000000}%
\pgfsetstrokecolor{currentstroke}%
\pgfsetdash{}{0pt}%
\pgfsys@defobject{currentmarker}{\pgfqpoint{0.000000in}{-0.048611in}}{\pgfqpoint{0.000000in}{0.000000in}}{%
\pgfpathmoveto{\pgfqpoint{0.000000in}{0.000000in}}%
\pgfpathlineto{\pgfqpoint{0.000000in}{-0.048611in}}%
\pgfusepath{stroke,fill}%
}%
\begin{pgfscope}%
\pgfsys@transformshift{1.025455in}{0.528000in}%
\pgfsys@useobject{currentmarker}{}%
\end{pgfscope}%
\end{pgfscope}%
\begin{pgfscope}%
\definecolor{textcolor}{rgb}{0.000000,0.000000,0.000000}%
\pgfsetstrokecolor{textcolor}%
\pgfsetfillcolor{textcolor}%
\pgftext[x=1.025455in,y=0.430778in,,top]{\color{textcolor}\sffamily\fontsize{10.000000}{12.000000}\selectfont 0}%
\end{pgfscope}%
\begin{pgfscope}%
\pgfpathrectangle{\pgfqpoint{0.800000in}{0.528000in}}{\pgfqpoint{4.960000in}{3.696000in}}%
\pgfusepath{clip}%
\pgfsetrectcap%
\pgfsetroundjoin%
\pgfsetlinewidth{0.803000pt}%
\definecolor{currentstroke}{rgb}{0.690196,0.690196,0.690196}%
\pgfsetstrokecolor{currentstroke}%
\pgfsetdash{}{0pt}%
\pgfpathmoveto{\pgfqpoint{1.776343in}{0.528000in}}%
\pgfpathlineto{\pgfqpoint{1.776343in}{4.224000in}}%
\pgfusepath{stroke}%
\end{pgfscope}%
\begin{pgfscope}%
\pgfsetbuttcap%
\pgfsetroundjoin%
\definecolor{currentfill}{rgb}{0.000000,0.000000,0.000000}%
\pgfsetfillcolor{currentfill}%
\pgfsetlinewidth{0.803000pt}%
\definecolor{currentstroke}{rgb}{0.000000,0.000000,0.000000}%
\pgfsetstrokecolor{currentstroke}%
\pgfsetdash{}{0pt}%
\pgfsys@defobject{currentmarker}{\pgfqpoint{0.000000in}{-0.048611in}}{\pgfqpoint{0.000000in}{0.000000in}}{%
\pgfpathmoveto{\pgfqpoint{0.000000in}{0.000000in}}%
\pgfpathlineto{\pgfqpoint{0.000000in}{-0.048611in}}%
\pgfusepath{stroke,fill}%
}%
\begin{pgfscope}%
\pgfsys@transformshift{1.776343in}{0.528000in}%
\pgfsys@useobject{currentmarker}{}%
\end{pgfscope}%
\end{pgfscope}%
\begin{pgfscope}%
\definecolor{textcolor}{rgb}{0.000000,0.000000,0.000000}%
\pgfsetstrokecolor{textcolor}%
\pgfsetfillcolor{textcolor}%
\pgftext[x=1.776343in,y=0.430778in,,top]{\color{textcolor}\sffamily\fontsize{10.000000}{12.000000}\selectfont 5}%
\end{pgfscope}%
\begin{pgfscope}%
\pgfpathrectangle{\pgfqpoint{0.800000in}{0.528000in}}{\pgfqpoint{4.960000in}{3.696000in}}%
\pgfusepath{clip}%
\pgfsetrectcap%
\pgfsetroundjoin%
\pgfsetlinewidth{0.803000pt}%
\definecolor{currentstroke}{rgb}{0.690196,0.690196,0.690196}%
\pgfsetstrokecolor{currentstroke}%
\pgfsetdash{}{0pt}%
\pgfpathmoveto{\pgfqpoint{2.527231in}{0.528000in}}%
\pgfpathlineto{\pgfqpoint{2.527231in}{4.224000in}}%
\pgfusepath{stroke}%
\end{pgfscope}%
\begin{pgfscope}%
\pgfsetbuttcap%
\pgfsetroundjoin%
\definecolor{currentfill}{rgb}{0.000000,0.000000,0.000000}%
\pgfsetfillcolor{currentfill}%
\pgfsetlinewidth{0.803000pt}%
\definecolor{currentstroke}{rgb}{0.000000,0.000000,0.000000}%
\pgfsetstrokecolor{currentstroke}%
\pgfsetdash{}{0pt}%
\pgfsys@defobject{currentmarker}{\pgfqpoint{0.000000in}{-0.048611in}}{\pgfqpoint{0.000000in}{0.000000in}}{%
\pgfpathmoveto{\pgfqpoint{0.000000in}{0.000000in}}%
\pgfpathlineto{\pgfqpoint{0.000000in}{-0.048611in}}%
\pgfusepath{stroke,fill}%
}%
\begin{pgfscope}%
\pgfsys@transformshift{2.527231in}{0.528000in}%
\pgfsys@useobject{currentmarker}{}%
\end{pgfscope}%
\end{pgfscope}%
\begin{pgfscope}%
\definecolor{textcolor}{rgb}{0.000000,0.000000,0.000000}%
\pgfsetstrokecolor{textcolor}%
\pgfsetfillcolor{textcolor}%
\pgftext[x=2.527231in,y=0.430778in,,top]{\color{textcolor}\sffamily\fontsize{10.000000}{12.000000}\selectfont 10}%
\end{pgfscope}%
\begin{pgfscope}%
\pgfpathrectangle{\pgfqpoint{0.800000in}{0.528000in}}{\pgfqpoint{4.960000in}{3.696000in}}%
\pgfusepath{clip}%
\pgfsetrectcap%
\pgfsetroundjoin%
\pgfsetlinewidth{0.803000pt}%
\definecolor{currentstroke}{rgb}{0.690196,0.690196,0.690196}%
\pgfsetstrokecolor{currentstroke}%
\pgfsetdash{}{0pt}%
\pgfpathmoveto{\pgfqpoint{3.278119in}{0.528000in}}%
\pgfpathlineto{\pgfqpoint{3.278119in}{4.224000in}}%
\pgfusepath{stroke}%
\end{pgfscope}%
\begin{pgfscope}%
\pgfsetbuttcap%
\pgfsetroundjoin%
\definecolor{currentfill}{rgb}{0.000000,0.000000,0.000000}%
\pgfsetfillcolor{currentfill}%
\pgfsetlinewidth{0.803000pt}%
\definecolor{currentstroke}{rgb}{0.000000,0.000000,0.000000}%
\pgfsetstrokecolor{currentstroke}%
\pgfsetdash{}{0pt}%
\pgfsys@defobject{currentmarker}{\pgfqpoint{0.000000in}{-0.048611in}}{\pgfqpoint{0.000000in}{0.000000in}}{%
\pgfpathmoveto{\pgfqpoint{0.000000in}{0.000000in}}%
\pgfpathlineto{\pgfqpoint{0.000000in}{-0.048611in}}%
\pgfusepath{stroke,fill}%
}%
\begin{pgfscope}%
\pgfsys@transformshift{3.278119in}{0.528000in}%
\pgfsys@useobject{currentmarker}{}%
\end{pgfscope}%
\end{pgfscope}%
\begin{pgfscope}%
\definecolor{textcolor}{rgb}{0.000000,0.000000,0.000000}%
\pgfsetstrokecolor{textcolor}%
\pgfsetfillcolor{textcolor}%
\pgftext[x=3.278119in,y=0.430778in,,top]{\color{textcolor}\sffamily\fontsize{10.000000}{12.000000}\selectfont 15}%
\end{pgfscope}%
\begin{pgfscope}%
\pgfpathrectangle{\pgfqpoint{0.800000in}{0.528000in}}{\pgfqpoint{4.960000in}{3.696000in}}%
\pgfusepath{clip}%
\pgfsetrectcap%
\pgfsetroundjoin%
\pgfsetlinewidth{0.803000pt}%
\definecolor{currentstroke}{rgb}{0.690196,0.690196,0.690196}%
\pgfsetstrokecolor{currentstroke}%
\pgfsetdash{}{0pt}%
\pgfpathmoveto{\pgfqpoint{4.029007in}{0.528000in}}%
\pgfpathlineto{\pgfqpoint{4.029007in}{4.224000in}}%
\pgfusepath{stroke}%
\end{pgfscope}%
\begin{pgfscope}%
\pgfsetbuttcap%
\pgfsetroundjoin%
\definecolor{currentfill}{rgb}{0.000000,0.000000,0.000000}%
\pgfsetfillcolor{currentfill}%
\pgfsetlinewidth{0.803000pt}%
\definecolor{currentstroke}{rgb}{0.000000,0.000000,0.000000}%
\pgfsetstrokecolor{currentstroke}%
\pgfsetdash{}{0pt}%
\pgfsys@defobject{currentmarker}{\pgfqpoint{0.000000in}{-0.048611in}}{\pgfqpoint{0.000000in}{0.000000in}}{%
\pgfpathmoveto{\pgfqpoint{0.000000in}{0.000000in}}%
\pgfpathlineto{\pgfqpoint{0.000000in}{-0.048611in}}%
\pgfusepath{stroke,fill}%
}%
\begin{pgfscope}%
\pgfsys@transformshift{4.029007in}{0.528000in}%
\pgfsys@useobject{currentmarker}{}%
\end{pgfscope}%
\end{pgfscope}%
\begin{pgfscope}%
\definecolor{textcolor}{rgb}{0.000000,0.000000,0.000000}%
\pgfsetstrokecolor{textcolor}%
\pgfsetfillcolor{textcolor}%
\pgftext[x=4.029007in,y=0.430778in,,top]{\color{textcolor}\sffamily\fontsize{10.000000}{12.000000}\selectfont 20}%
\end{pgfscope}%
\begin{pgfscope}%
\pgfpathrectangle{\pgfqpoint{0.800000in}{0.528000in}}{\pgfqpoint{4.960000in}{3.696000in}}%
\pgfusepath{clip}%
\pgfsetrectcap%
\pgfsetroundjoin%
\pgfsetlinewidth{0.803000pt}%
\definecolor{currentstroke}{rgb}{0.690196,0.690196,0.690196}%
\pgfsetstrokecolor{currentstroke}%
\pgfsetdash{}{0pt}%
\pgfpathmoveto{\pgfqpoint{4.779895in}{0.528000in}}%
\pgfpathlineto{\pgfqpoint{4.779895in}{4.224000in}}%
\pgfusepath{stroke}%
\end{pgfscope}%
\begin{pgfscope}%
\pgfsetbuttcap%
\pgfsetroundjoin%
\definecolor{currentfill}{rgb}{0.000000,0.000000,0.000000}%
\pgfsetfillcolor{currentfill}%
\pgfsetlinewidth{0.803000pt}%
\definecolor{currentstroke}{rgb}{0.000000,0.000000,0.000000}%
\pgfsetstrokecolor{currentstroke}%
\pgfsetdash{}{0pt}%
\pgfsys@defobject{currentmarker}{\pgfqpoint{0.000000in}{-0.048611in}}{\pgfqpoint{0.000000in}{0.000000in}}{%
\pgfpathmoveto{\pgfqpoint{0.000000in}{0.000000in}}%
\pgfpathlineto{\pgfqpoint{0.000000in}{-0.048611in}}%
\pgfusepath{stroke,fill}%
}%
\begin{pgfscope}%
\pgfsys@transformshift{4.779895in}{0.528000in}%
\pgfsys@useobject{currentmarker}{}%
\end{pgfscope}%
\end{pgfscope}%
\begin{pgfscope}%
\definecolor{textcolor}{rgb}{0.000000,0.000000,0.000000}%
\pgfsetstrokecolor{textcolor}%
\pgfsetfillcolor{textcolor}%
\pgftext[x=4.779895in,y=0.430778in,,top]{\color{textcolor}\sffamily\fontsize{10.000000}{12.000000}\selectfont 25}%
\end{pgfscope}%
\begin{pgfscope}%
\pgfpathrectangle{\pgfqpoint{0.800000in}{0.528000in}}{\pgfqpoint{4.960000in}{3.696000in}}%
\pgfusepath{clip}%
\pgfsetrectcap%
\pgfsetroundjoin%
\pgfsetlinewidth{0.803000pt}%
\definecolor{currentstroke}{rgb}{0.690196,0.690196,0.690196}%
\pgfsetstrokecolor{currentstroke}%
\pgfsetdash{}{0pt}%
\pgfpathmoveto{\pgfqpoint{5.530783in}{0.528000in}}%
\pgfpathlineto{\pgfqpoint{5.530783in}{4.224000in}}%
\pgfusepath{stroke}%
\end{pgfscope}%
\begin{pgfscope}%
\pgfsetbuttcap%
\pgfsetroundjoin%
\definecolor{currentfill}{rgb}{0.000000,0.000000,0.000000}%
\pgfsetfillcolor{currentfill}%
\pgfsetlinewidth{0.803000pt}%
\definecolor{currentstroke}{rgb}{0.000000,0.000000,0.000000}%
\pgfsetstrokecolor{currentstroke}%
\pgfsetdash{}{0pt}%
\pgfsys@defobject{currentmarker}{\pgfqpoint{0.000000in}{-0.048611in}}{\pgfqpoint{0.000000in}{0.000000in}}{%
\pgfpathmoveto{\pgfqpoint{0.000000in}{0.000000in}}%
\pgfpathlineto{\pgfqpoint{0.000000in}{-0.048611in}}%
\pgfusepath{stroke,fill}%
}%
\begin{pgfscope}%
\pgfsys@transformshift{5.530783in}{0.528000in}%
\pgfsys@useobject{currentmarker}{}%
\end{pgfscope}%
\end{pgfscope}%
\begin{pgfscope}%
\definecolor{textcolor}{rgb}{0.000000,0.000000,0.000000}%
\pgfsetstrokecolor{textcolor}%
\pgfsetfillcolor{textcolor}%
\pgftext[x=5.530783in,y=0.430778in,,top]{\color{textcolor}\sffamily\fontsize{10.000000}{12.000000}\selectfont 30}%
\end{pgfscope}%
\begin{pgfscope}%
\definecolor{textcolor}{rgb}{0.000000,0.000000,0.000000}%
\pgfsetstrokecolor{textcolor}%
\pgfsetfillcolor{textcolor}%
\pgftext[x=3.280000in,y=0.240809in,,top]{\color{textcolor}\sffamily\fontsize{10.000000}{12.000000}\selectfont time [s]}%
\end{pgfscope}%
\begin{pgfscope}%
\pgfpathrectangle{\pgfqpoint{0.800000in}{0.528000in}}{\pgfqpoint{4.960000in}{3.696000in}}%
\pgfusepath{clip}%
\pgfsetrectcap%
\pgfsetroundjoin%
\pgfsetlinewidth{0.803000pt}%
\definecolor{currentstroke}{rgb}{0.690196,0.690196,0.690196}%
\pgfsetstrokecolor{currentstroke}%
\pgfsetdash{}{0pt}%
\pgfpathmoveto{\pgfqpoint{0.800000in}{0.738988in}}%
\pgfpathlineto{\pgfqpoint{5.760000in}{0.738988in}}%
\pgfusepath{stroke}%
\end{pgfscope}%
\begin{pgfscope}%
\pgfsetbuttcap%
\pgfsetroundjoin%
\definecolor{currentfill}{rgb}{0.000000,0.000000,0.000000}%
\pgfsetfillcolor{currentfill}%
\pgfsetlinewidth{0.803000pt}%
\definecolor{currentstroke}{rgb}{0.000000,0.000000,0.000000}%
\pgfsetstrokecolor{currentstroke}%
\pgfsetdash{}{0pt}%
\pgfsys@defobject{currentmarker}{\pgfqpoint{-0.048611in}{0.000000in}}{\pgfqpoint{-0.000000in}{0.000000in}}{%
\pgfpathmoveto{\pgfqpoint{-0.000000in}{0.000000in}}%
\pgfpathlineto{\pgfqpoint{-0.048611in}{0.000000in}}%
\pgfusepath{stroke,fill}%
}%
\begin{pgfscope}%
\pgfsys@transformshift{0.800000in}{0.738988in}%
\pgfsys@useobject{currentmarker}{}%
\end{pgfscope}%
\end{pgfscope}%
\begin{pgfscope}%
\definecolor{textcolor}{rgb}{0.000000,0.000000,0.000000}%
\pgfsetstrokecolor{textcolor}%
\pgfsetfillcolor{textcolor}%
\pgftext[x=0.285508in, y=0.686226in, left, base]{\color{textcolor}\sffamily\fontsize{10.000000}{12.000000}\selectfont \ensuremath{-}1.75}%
\end{pgfscope}%
\begin{pgfscope}%
\pgfpathrectangle{\pgfqpoint{0.800000in}{0.528000in}}{\pgfqpoint{4.960000in}{3.696000in}}%
\pgfusepath{clip}%
\pgfsetrectcap%
\pgfsetroundjoin%
\pgfsetlinewidth{0.803000pt}%
\definecolor{currentstroke}{rgb}{0.690196,0.690196,0.690196}%
\pgfsetstrokecolor{currentstroke}%
\pgfsetdash{}{0pt}%
\pgfpathmoveto{\pgfqpoint{0.800000in}{1.209462in}}%
\pgfpathlineto{\pgfqpoint{5.760000in}{1.209462in}}%
\pgfusepath{stroke}%
\end{pgfscope}%
\begin{pgfscope}%
\pgfsetbuttcap%
\pgfsetroundjoin%
\definecolor{currentfill}{rgb}{0.000000,0.000000,0.000000}%
\pgfsetfillcolor{currentfill}%
\pgfsetlinewidth{0.803000pt}%
\definecolor{currentstroke}{rgb}{0.000000,0.000000,0.000000}%
\pgfsetstrokecolor{currentstroke}%
\pgfsetdash{}{0pt}%
\pgfsys@defobject{currentmarker}{\pgfqpoint{-0.048611in}{0.000000in}}{\pgfqpoint{-0.000000in}{0.000000in}}{%
\pgfpathmoveto{\pgfqpoint{-0.000000in}{0.000000in}}%
\pgfpathlineto{\pgfqpoint{-0.048611in}{0.000000in}}%
\pgfusepath{stroke,fill}%
}%
\begin{pgfscope}%
\pgfsys@transformshift{0.800000in}{1.209462in}%
\pgfsys@useobject{currentmarker}{}%
\end{pgfscope}%
\end{pgfscope}%
\begin{pgfscope}%
\definecolor{textcolor}{rgb}{0.000000,0.000000,0.000000}%
\pgfsetstrokecolor{textcolor}%
\pgfsetfillcolor{textcolor}%
\pgftext[x=0.285508in, y=1.156700in, left, base]{\color{textcolor}\sffamily\fontsize{10.000000}{12.000000}\selectfont \ensuremath{-}1.50}%
\end{pgfscope}%
\begin{pgfscope}%
\pgfpathrectangle{\pgfqpoint{0.800000in}{0.528000in}}{\pgfqpoint{4.960000in}{3.696000in}}%
\pgfusepath{clip}%
\pgfsetrectcap%
\pgfsetroundjoin%
\pgfsetlinewidth{0.803000pt}%
\definecolor{currentstroke}{rgb}{0.690196,0.690196,0.690196}%
\pgfsetstrokecolor{currentstroke}%
\pgfsetdash{}{0pt}%
\pgfpathmoveto{\pgfqpoint{0.800000in}{1.679936in}}%
\pgfpathlineto{\pgfqpoint{5.760000in}{1.679936in}}%
\pgfusepath{stroke}%
\end{pgfscope}%
\begin{pgfscope}%
\pgfsetbuttcap%
\pgfsetroundjoin%
\definecolor{currentfill}{rgb}{0.000000,0.000000,0.000000}%
\pgfsetfillcolor{currentfill}%
\pgfsetlinewidth{0.803000pt}%
\definecolor{currentstroke}{rgb}{0.000000,0.000000,0.000000}%
\pgfsetstrokecolor{currentstroke}%
\pgfsetdash{}{0pt}%
\pgfsys@defobject{currentmarker}{\pgfqpoint{-0.048611in}{0.000000in}}{\pgfqpoint{-0.000000in}{0.000000in}}{%
\pgfpathmoveto{\pgfqpoint{-0.000000in}{0.000000in}}%
\pgfpathlineto{\pgfqpoint{-0.048611in}{0.000000in}}%
\pgfusepath{stroke,fill}%
}%
\begin{pgfscope}%
\pgfsys@transformshift{0.800000in}{1.679936in}%
\pgfsys@useobject{currentmarker}{}%
\end{pgfscope}%
\end{pgfscope}%
\begin{pgfscope}%
\definecolor{textcolor}{rgb}{0.000000,0.000000,0.000000}%
\pgfsetstrokecolor{textcolor}%
\pgfsetfillcolor{textcolor}%
\pgftext[x=0.285508in, y=1.627175in, left, base]{\color{textcolor}\sffamily\fontsize{10.000000}{12.000000}\selectfont \ensuremath{-}1.25}%
\end{pgfscope}%
\begin{pgfscope}%
\pgfpathrectangle{\pgfqpoint{0.800000in}{0.528000in}}{\pgfqpoint{4.960000in}{3.696000in}}%
\pgfusepath{clip}%
\pgfsetrectcap%
\pgfsetroundjoin%
\pgfsetlinewidth{0.803000pt}%
\definecolor{currentstroke}{rgb}{0.690196,0.690196,0.690196}%
\pgfsetstrokecolor{currentstroke}%
\pgfsetdash{}{0pt}%
\pgfpathmoveto{\pgfqpoint{0.800000in}{2.150411in}}%
\pgfpathlineto{\pgfqpoint{5.760000in}{2.150411in}}%
\pgfusepath{stroke}%
\end{pgfscope}%
\begin{pgfscope}%
\pgfsetbuttcap%
\pgfsetroundjoin%
\definecolor{currentfill}{rgb}{0.000000,0.000000,0.000000}%
\pgfsetfillcolor{currentfill}%
\pgfsetlinewidth{0.803000pt}%
\definecolor{currentstroke}{rgb}{0.000000,0.000000,0.000000}%
\pgfsetstrokecolor{currentstroke}%
\pgfsetdash{}{0pt}%
\pgfsys@defobject{currentmarker}{\pgfqpoint{-0.048611in}{0.000000in}}{\pgfqpoint{-0.000000in}{0.000000in}}{%
\pgfpathmoveto{\pgfqpoint{-0.000000in}{0.000000in}}%
\pgfpathlineto{\pgfqpoint{-0.048611in}{0.000000in}}%
\pgfusepath{stroke,fill}%
}%
\begin{pgfscope}%
\pgfsys@transformshift{0.800000in}{2.150411in}%
\pgfsys@useobject{currentmarker}{}%
\end{pgfscope}%
\end{pgfscope}%
\begin{pgfscope}%
\definecolor{textcolor}{rgb}{0.000000,0.000000,0.000000}%
\pgfsetstrokecolor{textcolor}%
\pgfsetfillcolor{textcolor}%
\pgftext[x=0.285508in, y=2.097649in, left, base]{\color{textcolor}\sffamily\fontsize{10.000000}{12.000000}\selectfont \ensuremath{-}1.00}%
\end{pgfscope}%
\begin{pgfscope}%
\pgfpathrectangle{\pgfqpoint{0.800000in}{0.528000in}}{\pgfqpoint{4.960000in}{3.696000in}}%
\pgfusepath{clip}%
\pgfsetrectcap%
\pgfsetroundjoin%
\pgfsetlinewidth{0.803000pt}%
\definecolor{currentstroke}{rgb}{0.690196,0.690196,0.690196}%
\pgfsetstrokecolor{currentstroke}%
\pgfsetdash{}{0pt}%
\pgfpathmoveto{\pgfqpoint{0.800000in}{2.620885in}}%
\pgfpathlineto{\pgfqpoint{5.760000in}{2.620885in}}%
\pgfusepath{stroke}%
\end{pgfscope}%
\begin{pgfscope}%
\pgfsetbuttcap%
\pgfsetroundjoin%
\definecolor{currentfill}{rgb}{0.000000,0.000000,0.000000}%
\pgfsetfillcolor{currentfill}%
\pgfsetlinewidth{0.803000pt}%
\definecolor{currentstroke}{rgb}{0.000000,0.000000,0.000000}%
\pgfsetstrokecolor{currentstroke}%
\pgfsetdash{}{0pt}%
\pgfsys@defobject{currentmarker}{\pgfqpoint{-0.048611in}{0.000000in}}{\pgfqpoint{-0.000000in}{0.000000in}}{%
\pgfpathmoveto{\pgfqpoint{-0.000000in}{0.000000in}}%
\pgfpathlineto{\pgfqpoint{-0.048611in}{0.000000in}}%
\pgfusepath{stroke,fill}%
}%
\begin{pgfscope}%
\pgfsys@transformshift{0.800000in}{2.620885in}%
\pgfsys@useobject{currentmarker}{}%
\end{pgfscope}%
\end{pgfscope}%
\begin{pgfscope}%
\definecolor{textcolor}{rgb}{0.000000,0.000000,0.000000}%
\pgfsetstrokecolor{textcolor}%
\pgfsetfillcolor{textcolor}%
\pgftext[x=0.285508in, y=2.568124in, left, base]{\color{textcolor}\sffamily\fontsize{10.000000}{12.000000}\selectfont \ensuremath{-}0.75}%
\end{pgfscope}%
\begin{pgfscope}%
\pgfpathrectangle{\pgfqpoint{0.800000in}{0.528000in}}{\pgfqpoint{4.960000in}{3.696000in}}%
\pgfusepath{clip}%
\pgfsetrectcap%
\pgfsetroundjoin%
\pgfsetlinewidth{0.803000pt}%
\definecolor{currentstroke}{rgb}{0.690196,0.690196,0.690196}%
\pgfsetstrokecolor{currentstroke}%
\pgfsetdash{}{0pt}%
\pgfpathmoveto{\pgfqpoint{0.800000in}{3.091360in}}%
\pgfpathlineto{\pgfqpoint{5.760000in}{3.091360in}}%
\pgfusepath{stroke}%
\end{pgfscope}%
\begin{pgfscope}%
\pgfsetbuttcap%
\pgfsetroundjoin%
\definecolor{currentfill}{rgb}{0.000000,0.000000,0.000000}%
\pgfsetfillcolor{currentfill}%
\pgfsetlinewidth{0.803000pt}%
\definecolor{currentstroke}{rgb}{0.000000,0.000000,0.000000}%
\pgfsetstrokecolor{currentstroke}%
\pgfsetdash{}{0pt}%
\pgfsys@defobject{currentmarker}{\pgfqpoint{-0.048611in}{0.000000in}}{\pgfqpoint{-0.000000in}{0.000000in}}{%
\pgfpathmoveto{\pgfqpoint{-0.000000in}{0.000000in}}%
\pgfpathlineto{\pgfqpoint{-0.048611in}{0.000000in}}%
\pgfusepath{stroke,fill}%
}%
\begin{pgfscope}%
\pgfsys@transformshift{0.800000in}{3.091360in}%
\pgfsys@useobject{currentmarker}{}%
\end{pgfscope}%
\end{pgfscope}%
\begin{pgfscope}%
\definecolor{textcolor}{rgb}{0.000000,0.000000,0.000000}%
\pgfsetstrokecolor{textcolor}%
\pgfsetfillcolor{textcolor}%
\pgftext[x=0.285508in, y=3.038598in, left, base]{\color{textcolor}\sffamily\fontsize{10.000000}{12.000000}\selectfont \ensuremath{-}0.50}%
\end{pgfscope}%
\begin{pgfscope}%
\pgfpathrectangle{\pgfqpoint{0.800000in}{0.528000in}}{\pgfqpoint{4.960000in}{3.696000in}}%
\pgfusepath{clip}%
\pgfsetrectcap%
\pgfsetroundjoin%
\pgfsetlinewidth{0.803000pt}%
\definecolor{currentstroke}{rgb}{0.690196,0.690196,0.690196}%
\pgfsetstrokecolor{currentstroke}%
\pgfsetdash{}{0pt}%
\pgfpathmoveto{\pgfqpoint{0.800000in}{3.561834in}}%
\pgfpathlineto{\pgfqpoint{5.760000in}{3.561834in}}%
\pgfusepath{stroke}%
\end{pgfscope}%
\begin{pgfscope}%
\pgfsetbuttcap%
\pgfsetroundjoin%
\definecolor{currentfill}{rgb}{0.000000,0.000000,0.000000}%
\pgfsetfillcolor{currentfill}%
\pgfsetlinewidth{0.803000pt}%
\definecolor{currentstroke}{rgb}{0.000000,0.000000,0.000000}%
\pgfsetstrokecolor{currentstroke}%
\pgfsetdash{}{0pt}%
\pgfsys@defobject{currentmarker}{\pgfqpoint{-0.048611in}{0.000000in}}{\pgfqpoint{-0.000000in}{0.000000in}}{%
\pgfpathmoveto{\pgfqpoint{-0.000000in}{0.000000in}}%
\pgfpathlineto{\pgfqpoint{-0.048611in}{0.000000in}}%
\pgfusepath{stroke,fill}%
}%
\begin{pgfscope}%
\pgfsys@transformshift{0.800000in}{3.561834in}%
\pgfsys@useobject{currentmarker}{}%
\end{pgfscope}%
\end{pgfscope}%
\begin{pgfscope}%
\definecolor{textcolor}{rgb}{0.000000,0.000000,0.000000}%
\pgfsetstrokecolor{textcolor}%
\pgfsetfillcolor{textcolor}%
\pgftext[x=0.285508in, y=3.509073in, left, base]{\color{textcolor}\sffamily\fontsize{10.000000}{12.000000}\selectfont \ensuremath{-}0.25}%
\end{pgfscope}%
\begin{pgfscope}%
\pgfpathrectangle{\pgfqpoint{0.800000in}{0.528000in}}{\pgfqpoint{4.960000in}{3.696000in}}%
\pgfusepath{clip}%
\pgfsetrectcap%
\pgfsetroundjoin%
\pgfsetlinewidth{0.803000pt}%
\definecolor{currentstroke}{rgb}{0.690196,0.690196,0.690196}%
\pgfsetstrokecolor{currentstroke}%
\pgfsetdash{}{0pt}%
\pgfpathmoveto{\pgfqpoint{0.800000in}{4.032309in}}%
\pgfpathlineto{\pgfqpoint{5.760000in}{4.032309in}}%
\pgfusepath{stroke}%
\end{pgfscope}%
\begin{pgfscope}%
\pgfsetbuttcap%
\pgfsetroundjoin%
\definecolor{currentfill}{rgb}{0.000000,0.000000,0.000000}%
\pgfsetfillcolor{currentfill}%
\pgfsetlinewidth{0.803000pt}%
\definecolor{currentstroke}{rgb}{0.000000,0.000000,0.000000}%
\pgfsetstrokecolor{currentstroke}%
\pgfsetdash{}{0pt}%
\pgfsys@defobject{currentmarker}{\pgfqpoint{-0.048611in}{0.000000in}}{\pgfqpoint{-0.000000in}{0.000000in}}{%
\pgfpathmoveto{\pgfqpoint{-0.000000in}{0.000000in}}%
\pgfpathlineto{\pgfqpoint{-0.048611in}{0.000000in}}%
\pgfusepath{stroke,fill}%
}%
\begin{pgfscope}%
\pgfsys@transformshift{0.800000in}{4.032309in}%
\pgfsys@useobject{currentmarker}{}%
\end{pgfscope}%
\end{pgfscope}%
\begin{pgfscope}%
\definecolor{textcolor}{rgb}{0.000000,0.000000,0.000000}%
\pgfsetstrokecolor{textcolor}%
\pgfsetfillcolor{textcolor}%
\pgftext[x=0.393533in, y=3.979547in, left, base]{\color{textcolor}\sffamily\fontsize{10.000000}{12.000000}\selectfont 0.00}%
\end{pgfscope}%
\begin{pgfscope}%
\definecolor{textcolor}{rgb}{0.000000,0.000000,0.000000}%
\pgfsetstrokecolor{textcolor}%
\pgfsetfillcolor{textcolor}%
\pgftext[x=0.229952in,y=2.376000in,,bottom,rotate=90.000000]{\color{textcolor}\sffamily\fontsize{10.000000}{12.000000}\selectfont Forward movement [m]}%
\end{pgfscope}%
\begin{pgfscope}%
\pgfpathrectangle{\pgfqpoint{0.800000in}{0.528000in}}{\pgfqpoint{4.960000in}{3.696000in}}%
\pgfusepath{clip}%
\pgfsetrectcap%
\pgfsetroundjoin%
\pgfsetlinewidth{1.505625pt}%
\definecolor{currentstroke}{rgb}{0.121569,0.466667,0.705882}%
\pgfsetstrokecolor{currentstroke}%
\pgfsetdash{}{0pt}%
\pgfpathmoveto{\pgfqpoint{1.025455in}{4.035050in}}%
\pgfpathlineto{\pgfqpoint{1.079735in}{4.016102in}}%
\pgfpathlineto{\pgfqpoint{1.135063in}{3.917025in}}%
\pgfpathlineto{\pgfqpoint{1.189075in}{3.757262in}}%
\pgfpathlineto{\pgfqpoint{1.243658in}{3.558286in}}%
\pgfpathlineto{\pgfqpoint{1.299689in}{3.284300in}}%
\pgfpathlineto{\pgfqpoint{1.351745in}{3.011038in}}%
\pgfpathlineto{\pgfqpoint{1.405496in}{2.747210in}}%
\pgfpathlineto{\pgfqpoint{1.459653in}{2.514778in}}%
\pgfpathlineto{\pgfqpoint{1.514227in}{2.354532in}}%
\pgfpathlineto{\pgfqpoint{1.568240in}{2.280528in}}%
\pgfpathlineto{\pgfqpoint{1.624651in}{2.222179in}}%
\pgfpathlineto{\pgfqpoint{1.679031in}{2.182044in}}%
\pgfpathlineto{\pgfqpoint{1.733275in}{2.162111in}}%
\pgfpathlineto{\pgfqpoint{1.787527in}{2.160883in}}%
\pgfpathlineto{\pgfqpoint{1.844160in}{2.166366in}}%
\pgfpathlineto{\pgfqpoint{1.896960in}{2.175125in}}%
\pgfpathlineto{\pgfqpoint{1.952216in}{2.186343in}}%
\pgfpathlineto{\pgfqpoint{2.005474in}{2.195043in}}%
\pgfpathlineto{\pgfqpoint{2.060575in}{2.201610in}}%
\pgfpathlineto{\pgfqpoint{2.114239in}{2.206570in}}%
\pgfpathlineto{\pgfqpoint{2.167985in}{2.212789in}}%
\pgfpathlineto{\pgfqpoint{2.223457in}{2.219207in}}%
\pgfpathlineto{\pgfqpoint{2.276526in}{2.228132in}}%
\pgfpathlineto{\pgfqpoint{2.331554in}{2.237077in}}%
\pgfpathlineto{\pgfqpoint{2.386592in}{2.243557in}}%
\pgfpathlineto{\pgfqpoint{2.439992in}{2.246984in}}%
\pgfpathlineto{\pgfqpoint{2.494740in}{2.248387in}}%
\pgfpathlineto{\pgfqpoint{2.549385in}{2.247934in}}%
\pgfpathlineto{\pgfqpoint{2.602877in}{2.248336in}}%
\pgfpathlineto{\pgfqpoint{2.657535in}{2.248817in}}%
\pgfpathlineto{\pgfqpoint{2.711868in}{2.244433in}}%
\pgfpathlineto{\pgfqpoint{2.767412in}{2.236104in}}%
\pgfpathlineto{\pgfqpoint{2.822529in}{2.223791in}}%
\pgfpathlineto{\pgfqpoint{2.875151in}{2.208728in}}%
\pgfpathlineto{\pgfqpoint{2.928591in}{2.194073in}}%
\pgfpathlineto{\pgfqpoint{2.982157in}{2.178672in}}%
\pgfpathlineto{\pgfqpoint{3.035815in}{2.164533in}}%
\pgfpathlineto{\pgfqpoint{3.090489in}{2.151626in}}%
\pgfpathlineto{\pgfqpoint{3.144495in}{2.143260in}}%
\pgfpathlineto{\pgfqpoint{3.200514in}{2.139830in}}%
\pgfpathlineto{\pgfqpoint{3.253298in}{2.139971in}}%
\pgfpathlineto{\pgfqpoint{3.307342in}{2.143599in}}%
\pgfpathlineto{\pgfqpoint{3.362302in}{2.146662in}}%
\pgfpathlineto{\pgfqpoint{3.416387in}{2.153897in}}%
\pgfpathlineto{\pgfqpoint{3.470648in}{2.164696in}}%
\pgfpathlineto{\pgfqpoint{3.524668in}{2.174968in}}%
\pgfpathlineto{\pgfqpoint{3.579123in}{2.181339in}}%
\pgfpathlineto{\pgfqpoint{3.632992in}{2.178409in}}%
\pgfpathlineto{\pgfqpoint{3.687547in}{2.169152in}}%
\pgfpathlineto{\pgfqpoint{3.741880in}{2.156267in}}%
\pgfpathlineto{\pgfqpoint{3.796081in}{2.139020in}}%
\pgfpathlineto{\pgfqpoint{3.851858in}{2.116697in}}%
\pgfpathlineto{\pgfqpoint{3.905695in}{2.091957in}}%
\pgfpathlineto{\pgfqpoint{3.959560in}{2.065950in}}%
\pgfpathlineto{\pgfqpoint{4.014993in}{2.048397in}}%
\pgfpathlineto{\pgfqpoint{4.068019in}{2.044607in}}%
\pgfpathlineto{\pgfqpoint{4.122358in}{2.046740in}}%
\pgfpathlineto{\pgfqpoint{4.176154in}{2.055990in}}%
\pgfpathlineto{\pgfqpoint{4.230593in}{2.080512in}}%
\pgfpathlineto{\pgfqpoint{4.284345in}{2.100719in}}%
\pgfpathlineto{\pgfqpoint{4.339380in}{2.109182in}}%
\pgfpathlineto{\pgfqpoint{4.395112in}{2.116697in}}%
\pgfpathlineto{\pgfqpoint{4.448699in}{2.119515in}}%
\pgfpathlineto{\pgfqpoint{4.502654in}{2.120602in}}%
\pgfpathlineto{\pgfqpoint{4.556517in}{2.120888in}}%
\pgfpathlineto{\pgfqpoint{4.610716in}{2.121029in}}%
\pgfpathlineto{\pgfqpoint{4.665011in}{2.118447in}}%
\pgfpathlineto{\pgfqpoint{4.719273in}{2.114562in}}%
\pgfpathlineto{\pgfqpoint{4.772997in}{2.107879in}}%
\pgfpathlineto{\pgfqpoint{4.827311in}{2.101207in}}%
\pgfpathlineto{\pgfqpoint{4.883565in}{2.093297in}}%
\pgfpathlineto{\pgfqpoint{4.936795in}{2.089012in}}%
\pgfpathlineto{\pgfqpoint{4.990638in}{2.093748in}}%
\pgfpathlineto{\pgfqpoint{5.044624in}{2.107357in}}%
\pgfpathlineto{\pgfqpoint{5.098870in}{2.114070in}}%
\pgfpathlineto{\pgfqpoint{5.153236in}{2.121237in}}%
\pgfpathlineto{\pgfqpoint{5.207127in}{2.125555in}}%
\pgfpathlineto{\pgfqpoint{5.261203in}{2.125241in}}%
\pgfpathlineto{\pgfqpoint{5.315735in}{2.122153in}}%
\pgfpathlineto{\pgfqpoint{5.370037in}{2.119623in}}%
\pgfpathlineto{\pgfqpoint{5.424754in}{2.114909in}}%
\pgfpathlineto{\pgfqpoint{5.481843in}{2.106408in}}%
\pgfpathlineto{\pgfqpoint{5.534545in}{2.098463in}}%
\pgfusepath{stroke}%
\end{pgfscope}%
\begin{pgfscope}%
\pgfpathrectangle{\pgfqpoint{0.800000in}{0.528000in}}{\pgfqpoint{4.960000in}{3.696000in}}%
\pgfusepath{clip}%
\pgfsetrectcap%
\pgfsetroundjoin%
\pgfsetlinewidth{1.505625pt}%
\definecolor{currentstroke}{rgb}{1.000000,0.498039,0.054902}%
\pgfsetstrokecolor{currentstroke}%
\pgfsetdash{}{0pt}%
\pgfpathmoveto{\pgfqpoint{1.025455in}{3.910056in}}%
\pgfpathlineto{\pgfqpoint{1.080463in}{3.890544in}}%
\pgfpathlineto{\pgfqpoint{1.134886in}{3.801040in}}%
\pgfpathlineto{\pgfqpoint{1.188573in}{3.643569in}}%
\pgfpathlineto{\pgfqpoint{1.243222in}{3.427563in}}%
\pgfpathlineto{\pgfqpoint{1.299341in}{3.166324in}}%
\pgfpathlineto{\pgfqpoint{1.351484in}{2.899063in}}%
\pgfpathlineto{\pgfqpoint{1.405458in}{2.591263in}}%
\pgfpathlineto{\pgfqpoint{1.459678in}{2.298563in}}%
\pgfpathlineto{\pgfqpoint{1.513697in}{2.020998in}}%
\pgfpathlineto{\pgfqpoint{1.567828in}{1.793799in}}%
\pgfpathlineto{\pgfqpoint{1.622402in}{1.603818in}}%
\pgfpathlineto{\pgfqpoint{1.676700in}{1.465932in}}%
\pgfpathlineto{\pgfqpoint{1.732519in}{1.369005in}}%
\pgfpathlineto{\pgfqpoint{1.785919in}{1.314143in}}%
\pgfpathlineto{\pgfqpoint{1.839733in}{1.286565in}}%
\pgfpathlineto{\pgfqpoint{1.895406in}{1.284609in}}%
\pgfpathlineto{\pgfqpoint{1.949133in}{1.295540in}}%
\pgfpathlineto{\pgfqpoint{2.003134in}{1.313543in}}%
\pgfpathlineto{\pgfqpoint{2.057367in}{1.332010in}}%
\pgfpathlineto{\pgfqpoint{2.111853in}{1.350267in}}%
\pgfpathlineto{\pgfqpoint{2.166055in}{1.367100in}}%
\pgfpathlineto{\pgfqpoint{2.220433in}{1.383874in}}%
\pgfpathlineto{\pgfqpoint{2.274913in}{1.397560in}}%
\pgfpathlineto{\pgfqpoint{2.329307in}{1.410689in}}%
\pgfpathlineto{\pgfqpoint{2.383517in}{1.421261in}}%
\pgfpathlineto{\pgfqpoint{2.437746in}{1.429690in}}%
\pgfpathlineto{\pgfqpoint{2.493624in}{1.437862in}}%
\pgfpathlineto{\pgfqpoint{2.546508in}{1.446337in}}%
\pgfpathlineto{\pgfqpoint{2.600736in}{1.454150in}}%
\pgfpathlineto{\pgfqpoint{2.654834in}{1.461451in}}%
\pgfpathlineto{\pgfqpoint{2.708919in}{1.468465in}}%
\pgfpathlineto{\pgfqpoint{2.763065in}{1.472600in}}%
\pgfpathlineto{\pgfqpoint{2.818646in}{1.475242in}}%
\pgfpathlineto{\pgfqpoint{2.873454in}{1.482905in}}%
\pgfpathlineto{\pgfqpoint{2.927786in}{1.492524in}}%
\pgfpathlineto{\pgfqpoint{2.982519in}{1.502453in}}%
\pgfpathlineto{\pgfqpoint{3.036262in}{1.512041in}}%
\pgfpathlineto{\pgfqpoint{3.091444in}{1.522089in}}%
\pgfpathlineto{\pgfqpoint{3.144682in}{1.531713in}}%
\pgfpathlineto{\pgfqpoint{3.198899in}{1.541636in}}%
\pgfpathlineto{\pgfqpoint{3.252778in}{1.552598in}}%
\pgfpathlineto{\pgfqpoint{3.307308in}{1.565620in}}%
\pgfpathlineto{\pgfqpoint{3.361350in}{1.577474in}}%
\pgfpathlineto{\pgfqpoint{3.415973in}{1.594214in}}%
\pgfpathlineto{\pgfqpoint{3.470005in}{1.610672in}}%
\pgfpathlineto{\pgfqpoint{3.524641in}{1.624836in}}%
\pgfpathlineto{\pgfqpoint{3.578737in}{1.633697in}}%
\pgfpathlineto{\pgfqpoint{3.633092in}{1.642075in}}%
\pgfpathlineto{\pgfqpoint{3.688418in}{1.652537in}}%
\pgfpathlineto{\pgfqpoint{3.742517in}{1.661392in}}%
\pgfpathlineto{\pgfqpoint{3.796225in}{1.670219in}}%
\pgfpathlineto{\pgfqpoint{3.850584in}{1.677929in}}%
\pgfpathlineto{\pgfqpoint{3.905198in}{1.680390in}}%
\pgfpathlineto{\pgfqpoint{3.959297in}{1.680364in}}%
\pgfpathlineto{\pgfqpoint{4.013722in}{1.679225in}}%
\pgfpathlineto{\pgfqpoint{4.068313in}{1.678707in}}%
\pgfpathlineto{\pgfqpoint{4.122026in}{1.677566in}}%
\pgfpathlineto{\pgfqpoint{4.177259in}{1.678195in}}%
\pgfpathlineto{\pgfqpoint{4.231514in}{1.678307in}}%
\pgfpathlineto{\pgfqpoint{4.285536in}{1.672595in}}%
\pgfpathlineto{\pgfqpoint{4.339524in}{1.664719in}}%
\pgfpathlineto{\pgfqpoint{4.393726in}{1.653139in}}%
\pgfpathlineto{\pgfqpoint{4.447919in}{1.642018in}}%
\pgfpathlineto{\pgfqpoint{4.502149in}{1.633137in}}%
\pgfpathlineto{\pgfqpoint{4.557238in}{1.627387in}}%
\pgfpathlineto{\pgfqpoint{4.611211in}{1.623094in}}%
\pgfpathlineto{\pgfqpoint{4.666759in}{1.622362in}}%
\pgfpathlineto{\pgfqpoint{4.719773in}{1.623356in}}%
\pgfpathlineto{\pgfqpoint{4.774092in}{1.628371in}}%
\pgfpathlineto{\pgfqpoint{4.828199in}{1.638991in}}%
\pgfpathlineto{\pgfqpoint{4.882527in}{1.652959in}}%
\pgfpathlineto{\pgfqpoint{4.936776in}{1.669036in}}%
\pgfpathlineto{\pgfqpoint{4.991258in}{1.683750in}}%
\pgfpathlineto{\pgfqpoint{5.045399in}{1.696015in}}%
\pgfpathlineto{\pgfqpoint{5.100600in}{1.704363in}}%
\pgfpathlineto{\pgfqpoint{5.155003in}{1.708234in}}%
\pgfpathlineto{\pgfqpoint{5.209765in}{1.709292in}}%
\pgfpathlineto{\pgfqpoint{5.264978in}{1.710457in}}%
\pgfpathlineto{\pgfqpoint{5.318254in}{1.712203in}}%
\pgfpathlineto{\pgfqpoint{5.372363in}{1.714391in}}%
\pgfpathlineto{\pgfqpoint{5.426133in}{1.715511in}}%
\pgfpathlineto{\pgfqpoint{5.480334in}{1.716083in}}%
\pgfpathlineto{\pgfqpoint{5.534460in}{1.714110in}}%
\pgfusepath{stroke}%
\end{pgfscope}%
\begin{pgfscope}%
\pgfpathrectangle{\pgfqpoint{0.800000in}{0.528000in}}{\pgfqpoint{4.960000in}{3.696000in}}%
\pgfusepath{clip}%
\pgfsetrectcap%
\pgfsetroundjoin%
\pgfsetlinewidth{1.505625pt}%
\definecolor{currentstroke}{rgb}{0.172549,0.627451,0.172549}%
\pgfsetstrokecolor{currentstroke}%
\pgfsetdash{}{0pt}%
\pgfpathmoveto{\pgfqpoint{1.025455in}{3.995358in}}%
\pgfpathlineto{\pgfqpoint{1.079276in}{3.975766in}}%
\pgfpathlineto{\pgfqpoint{1.134337in}{3.877125in}}%
\pgfpathlineto{\pgfqpoint{1.188053in}{3.716713in}}%
\pgfpathlineto{\pgfqpoint{1.242507in}{3.500181in}}%
\pgfpathlineto{\pgfqpoint{1.296487in}{3.262368in}}%
\pgfpathlineto{\pgfqpoint{1.351007in}{2.977278in}}%
\pgfpathlineto{\pgfqpoint{1.403815in}{2.691812in}}%
\pgfpathlineto{\pgfqpoint{1.457773in}{2.390392in}}%
\pgfpathlineto{\pgfqpoint{1.513226in}{2.079185in}}%
\pgfpathlineto{\pgfqpoint{1.566874in}{1.798502in}}%
\pgfpathlineto{\pgfqpoint{1.620724in}{1.548350in}}%
\pgfpathlineto{\pgfqpoint{1.674856in}{1.322598in}}%
\pgfpathlineto{\pgfqpoint{1.729280in}{1.116871in}}%
\pgfpathlineto{\pgfqpoint{1.783579in}{0.953056in}}%
\pgfpathlineto{\pgfqpoint{1.837576in}{0.838447in}}%
\pgfpathlineto{\pgfqpoint{1.891857in}{0.753674in}}%
\pgfpathlineto{\pgfqpoint{1.946067in}{0.707546in}}%
\pgfpathlineto{\pgfqpoint{2.000385in}{0.696000in}}%
\pgfpathlineto{\pgfqpoint{2.054520in}{0.703137in}}%
\pgfpathlineto{\pgfqpoint{2.110379in}{0.719028in}}%
\pgfpathlineto{\pgfqpoint{2.164011in}{0.740738in}}%
\pgfpathlineto{\pgfqpoint{2.217480in}{0.768481in}}%
\pgfpathlineto{\pgfqpoint{2.271834in}{0.805170in}}%
\pgfpathlineto{\pgfqpoint{2.325843in}{0.845417in}}%
\pgfpathlineto{\pgfqpoint{2.380409in}{0.897780in}}%
\pgfpathlineto{\pgfqpoint{2.434965in}{0.955379in}}%
\pgfpathlineto{\pgfqpoint{2.489042in}{1.013654in}}%
\pgfpathlineto{\pgfqpoint{2.544500in}{1.083471in}}%
\pgfpathlineto{\pgfqpoint{2.600597in}{1.153724in}}%
\pgfpathlineto{\pgfqpoint{2.654488in}{1.223967in}}%
\pgfpathlineto{\pgfqpoint{2.707657in}{1.296984in}}%
\pgfpathlineto{\pgfqpoint{2.762110in}{1.362965in}}%
\pgfpathlineto{\pgfqpoint{2.816152in}{1.430717in}}%
\pgfpathlineto{\pgfqpoint{2.870177in}{1.492400in}}%
\pgfpathlineto{\pgfqpoint{2.924385in}{1.548604in}}%
\pgfpathlineto{\pgfqpoint{2.978863in}{1.607188in}}%
\pgfpathlineto{\pgfqpoint{3.033033in}{1.670855in}}%
\pgfpathlineto{\pgfqpoint{3.087035in}{1.733354in}}%
\pgfpathlineto{\pgfqpoint{3.141243in}{1.803477in}}%
\pgfpathlineto{\pgfqpoint{3.195658in}{1.878710in}}%
\pgfpathlineto{\pgfqpoint{3.250274in}{1.954560in}}%
\pgfpathlineto{\pgfqpoint{3.304241in}{2.036313in}}%
\pgfpathlineto{\pgfqpoint{3.358406in}{2.113508in}}%
\pgfpathlineto{\pgfqpoint{3.412858in}{2.168891in}}%
\pgfpathlineto{\pgfqpoint{3.468680in}{2.206415in}}%
\pgfpathlineto{\pgfqpoint{3.521891in}{2.233477in}}%
\pgfpathlineto{\pgfqpoint{3.575560in}{2.251917in}}%
\pgfpathlineto{\pgfqpoint{3.630153in}{2.261654in}}%
\pgfpathlineto{\pgfqpoint{3.684531in}{2.263083in}}%
\pgfpathlineto{\pgfqpoint{3.738832in}{2.257378in}}%
\pgfpathlineto{\pgfqpoint{3.793789in}{2.251385in}}%
\pgfpathlineto{\pgfqpoint{3.847587in}{2.239936in}}%
\pgfpathlineto{\pgfqpoint{3.901790in}{2.225431in}}%
\pgfpathlineto{\pgfqpoint{3.956116in}{2.207711in}}%
\pgfpathlineto{\pgfqpoint{4.012377in}{2.186423in}}%
\pgfpathlineto{\pgfqpoint{4.065383in}{2.164189in}}%
\pgfpathlineto{\pgfqpoint{4.119094in}{2.141982in}}%
\pgfpathlineto{\pgfqpoint{4.173489in}{2.123571in}}%
\pgfpathlineto{\pgfqpoint{4.228086in}{2.108955in}}%
\pgfpathlineto{\pgfqpoint{4.284288in}{2.095336in}}%
\pgfpathlineto{\pgfqpoint{4.336115in}{2.080888in}}%
\pgfpathlineto{\pgfqpoint{4.390091in}{2.066367in}}%
\pgfpathlineto{\pgfqpoint{4.445591in}{2.054889in}}%
\pgfpathlineto{\pgfqpoint{4.499268in}{2.050385in}}%
\pgfpathlineto{\pgfqpoint{4.554652in}{2.053944in}}%
\pgfpathlineto{\pgfqpoint{4.608705in}{2.059826in}}%
\pgfpathlineto{\pgfqpoint{4.663270in}{2.071897in}}%
\pgfpathlineto{\pgfqpoint{4.716879in}{2.090374in}}%
\pgfpathlineto{\pgfqpoint{4.771042in}{2.118758in}}%
\pgfpathlineto{\pgfqpoint{4.825033in}{2.149386in}}%
\pgfpathlineto{\pgfqpoint{4.879624in}{2.179388in}}%
\pgfpathlineto{\pgfqpoint{4.937747in}{2.207711in}}%
\pgfpathlineto{\pgfqpoint{4.989660in}{2.229315in}}%
\pgfpathlineto{\pgfqpoint{5.043100in}{2.244608in}}%
\pgfpathlineto{\pgfqpoint{5.097222in}{2.253867in}}%
\pgfpathlineto{\pgfqpoint{5.151195in}{2.259047in}}%
\pgfpathlineto{\pgfqpoint{5.205645in}{2.260272in}}%
\pgfpathlineto{\pgfqpoint{5.259429in}{2.260556in}}%
\pgfpathlineto{\pgfqpoint{5.314230in}{2.259908in}}%
\pgfpathlineto{\pgfqpoint{5.368365in}{2.254107in}}%
\pgfpathlineto{\pgfqpoint{5.422685in}{2.246485in}}%
\pgfpathlineto{\pgfqpoint{5.477278in}{2.234945in}}%
\pgfpathlineto{\pgfqpoint{5.531250in}{2.219538in}}%
\pgfusepath{stroke}%
\end{pgfscope}%
\begin{pgfscope}%
\pgfpathrectangle{\pgfqpoint{0.800000in}{0.528000in}}{\pgfqpoint{4.960000in}{3.696000in}}%
\pgfusepath{clip}%
\pgfsetrectcap%
\pgfsetroundjoin%
\pgfsetlinewidth{1.505625pt}%
\definecolor{currentstroke}{rgb}{0.839216,0.152941,0.156863}%
\pgfsetstrokecolor{currentstroke}%
\pgfsetdash{}{0pt}%
\pgfpathmoveto{\pgfqpoint{1.025455in}{4.010108in}}%
\pgfpathlineto{\pgfqpoint{1.079574in}{3.985713in}}%
\pgfpathlineto{\pgfqpoint{1.133752in}{3.887985in}}%
\pgfpathlineto{\pgfqpoint{1.188403in}{3.725910in}}%
\pgfpathlineto{\pgfqpoint{1.242916in}{3.520385in}}%
\pgfpathlineto{\pgfqpoint{1.296283in}{3.262954in}}%
\pgfpathlineto{\pgfqpoint{1.350175in}{3.000422in}}%
\pgfpathlineto{\pgfqpoint{1.405571in}{2.700141in}}%
\pgfpathlineto{\pgfqpoint{1.459450in}{2.392131in}}%
\pgfpathlineto{\pgfqpoint{1.513421in}{2.110043in}}%
\pgfpathlineto{\pgfqpoint{1.567305in}{1.811482in}}%
\pgfpathlineto{\pgfqpoint{1.622216in}{1.548843in}}%
\pgfpathlineto{\pgfqpoint{1.675720in}{1.355632in}}%
\pgfpathlineto{\pgfqpoint{1.730249in}{1.172759in}}%
\pgfpathlineto{\pgfqpoint{1.784621in}{1.036275in}}%
\pgfpathlineto{\pgfqpoint{1.838862in}{0.942486in}}%
\pgfpathlineto{\pgfqpoint{1.893984in}{0.887182in}}%
\pgfpathlineto{\pgfqpoint{1.948088in}{0.868545in}}%
\pgfpathlineto{\pgfqpoint{2.003314in}{0.882714in}}%
\pgfpathlineto{\pgfqpoint{2.057154in}{0.919243in}}%
\pgfpathlineto{\pgfqpoint{2.111057in}{0.974886in}}%
\pgfpathlineto{\pgfqpoint{2.165041in}{1.058462in}}%
\pgfpathlineto{\pgfqpoint{2.219870in}{1.160762in}}%
\pgfpathlineto{\pgfqpoint{2.273776in}{1.270115in}}%
\pgfpathlineto{\pgfqpoint{2.328187in}{1.407198in}}%
\pgfpathlineto{\pgfqpoint{2.382694in}{1.560779in}}%
\pgfpathlineto{\pgfqpoint{2.436719in}{1.718490in}}%
\pgfpathlineto{\pgfqpoint{2.491312in}{1.899092in}}%
\pgfpathlineto{\pgfqpoint{2.545586in}{2.068382in}}%
\pgfpathlineto{\pgfqpoint{2.600987in}{2.231420in}}%
\pgfpathlineto{\pgfqpoint{2.654417in}{2.372476in}}%
\pgfpathlineto{\pgfqpoint{2.708363in}{2.513629in}}%
\pgfpathlineto{\pgfqpoint{2.762531in}{2.626960in}}%
\pgfpathlineto{\pgfqpoint{2.816924in}{2.724062in}}%
\pgfpathlineto{\pgfqpoint{2.871160in}{2.782230in}}%
\pgfpathlineto{\pgfqpoint{2.925809in}{2.789369in}}%
\pgfpathlineto{\pgfqpoint{2.980556in}{2.740981in}}%
\pgfpathlineto{\pgfqpoint{3.034718in}{2.629221in}}%
\pgfpathlineto{\pgfqpoint{3.088847in}{2.473033in}}%
\pgfpathlineto{\pgfqpoint{3.144344in}{2.303301in}}%
\pgfpathlineto{\pgfqpoint{3.199084in}{2.150429in}}%
\pgfpathlineto{\pgfqpoint{3.252586in}{2.004371in}}%
\pgfpathlineto{\pgfqpoint{3.306657in}{1.897695in}}%
\pgfpathlineto{\pgfqpoint{3.360776in}{1.846991in}}%
\pgfpathlineto{\pgfqpoint{3.414926in}{1.797307in}}%
\pgfpathlineto{\pgfqpoint{3.469275in}{1.748786in}}%
\pgfpathlineto{\pgfqpoint{3.523542in}{1.732883in}}%
\pgfpathlineto{\pgfqpoint{3.578135in}{1.742016in}}%
\pgfpathlineto{\pgfqpoint{3.631967in}{1.775804in}}%
\pgfpathlineto{\pgfqpoint{3.686225in}{1.829648in}}%
\pgfpathlineto{\pgfqpoint{3.742116in}{1.907453in}}%
\pgfpathlineto{\pgfqpoint{3.795501in}{2.004107in}}%
\pgfpathlineto{\pgfqpoint{3.849236in}{2.109678in}}%
\pgfpathlineto{\pgfqpoint{3.903432in}{2.215910in}}%
\pgfpathlineto{\pgfqpoint{3.957555in}{2.304676in}}%
\pgfpathlineto{\pgfqpoint{4.011821in}{2.384916in}}%
\pgfpathlineto{\pgfqpoint{4.066215in}{2.456681in}}%
\pgfpathlineto{\pgfqpoint{4.120523in}{2.505576in}}%
\pgfpathlineto{\pgfqpoint{4.174850in}{2.528564in}}%
\pgfpathlineto{\pgfqpoint{4.230312in}{2.532024in}}%
\pgfpathlineto{\pgfqpoint{4.283605in}{2.528878in}}%
\pgfpathlineto{\pgfqpoint{4.337688in}{2.519983in}}%
\pgfpathlineto{\pgfqpoint{4.392175in}{2.504959in}}%
\pgfpathlineto{\pgfqpoint{4.446261in}{2.482814in}}%
\pgfpathlineto{\pgfqpoint{4.500733in}{2.450489in}}%
\pgfpathlineto{\pgfqpoint{4.555215in}{2.408951in}}%
\pgfpathlineto{\pgfqpoint{4.609863in}{2.349472in}}%
\pgfpathlineto{\pgfqpoint{4.664009in}{2.278136in}}%
\pgfpathlineto{\pgfqpoint{4.718818in}{2.205761in}}%
\pgfpathlineto{\pgfqpoint{4.772848in}{2.140210in}}%
\pgfpathlineto{\pgfqpoint{4.828500in}{2.083371in}}%
\pgfpathlineto{\pgfqpoint{4.882571in}{2.040426in}}%
\pgfpathlineto{\pgfqpoint{4.936339in}{2.005445in}}%
\pgfpathlineto{\pgfqpoint{4.990646in}{1.977834in}}%
\pgfpathlineto{\pgfqpoint{5.045021in}{1.961788in}}%
\pgfpathlineto{\pgfqpoint{5.099433in}{1.955366in}}%
\pgfpathlineto{\pgfqpoint{5.153905in}{1.952424in}}%
\pgfpathlineto{\pgfqpoint{5.207925in}{1.952077in}}%
\pgfpathlineto{\pgfqpoint{5.262413in}{1.952688in}}%
\pgfpathlineto{\pgfqpoint{5.316485in}{1.955714in}}%
\pgfpathlineto{\pgfqpoint{5.370794in}{1.965108in}}%
\pgfpathlineto{\pgfqpoint{5.426554in}{1.980563in}}%
\pgfpathlineto{\pgfqpoint{5.480194in}{1.998907in}}%
\pgfpathlineto{\pgfqpoint{5.533860in}{2.021630in}}%
\pgfusepath{stroke}%
\end{pgfscope}%
\begin{pgfscope}%
\pgfpathrectangle{\pgfqpoint{0.800000in}{0.528000in}}{\pgfqpoint{4.960000in}{3.696000in}}%
\pgfusepath{clip}%
\pgfsetrectcap%
\pgfsetroundjoin%
\pgfsetlinewidth{1.505625pt}%
\definecolor{currentstroke}{rgb}{0.580392,0.403922,0.741176}%
\pgfsetstrokecolor{currentstroke}%
\pgfsetdash{}{0pt}%
\pgfpathmoveto{\pgfqpoint{1.025455in}{4.056000in}}%
\pgfpathlineto{\pgfqpoint{1.079196in}{4.036592in}}%
\pgfpathlineto{\pgfqpoint{1.133647in}{3.939319in}}%
\pgfpathlineto{\pgfqpoint{1.187976in}{3.780231in}}%
\pgfpathlineto{\pgfqpoint{1.242551in}{3.583449in}}%
\pgfpathlineto{\pgfqpoint{1.296968in}{3.326698in}}%
\pgfpathlineto{\pgfqpoint{1.351086in}{3.038582in}}%
\pgfpathlineto{\pgfqpoint{1.404085in}{2.749295in}}%
\pgfpathlineto{\pgfqpoint{1.458169in}{2.445076in}}%
\pgfpathlineto{\pgfqpoint{1.512631in}{2.149877in}}%
\pgfpathlineto{\pgfqpoint{1.567104in}{1.852578in}}%
\pgfpathlineto{\pgfqpoint{1.622903in}{1.585344in}}%
\pgfpathlineto{\pgfqpoint{1.676449in}{1.347211in}}%
\pgfpathlineto{\pgfqpoint{1.730170in}{1.162814in}}%
\pgfpathlineto{\pgfqpoint{1.784346in}{1.014387in}}%
\pgfpathlineto{\pgfqpoint{1.838594in}{0.920819in}}%
\pgfpathlineto{\pgfqpoint{1.892722in}{0.881462in}}%
\pgfpathlineto{\pgfqpoint{1.946804in}{0.890971in}}%
\pgfpathlineto{\pgfqpoint{2.001161in}{0.940871in}}%
\pgfpathlineto{\pgfqpoint{2.055203in}{1.023449in}}%
\pgfpathlineto{\pgfqpoint{2.109614in}{1.139462in}}%
\pgfpathlineto{\pgfqpoint{2.163964in}{1.288923in}}%
\pgfpathlineto{\pgfqpoint{2.218178in}{1.450148in}}%
\pgfpathlineto{\pgfqpoint{2.273666in}{1.645088in}}%
\pgfpathlineto{\pgfqpoint{2.327048in}{1.838360in}}%
\pgfpathlineto{\pgfqpoint{2.381133in}{2.011624in}}%
\pgfpathlineto{\pgfqpoint{2.435138in}{2.177786in}}%
\pgfpathlineto{\pgfqpoint{2.489333in}{2.310991in}}%
\pgfpathlineto{\pgfqpoint{2.543660in}{2.446861in}}%
\pgfpathlineto{\pgfqpoint{2.598384in}{2.567462in}}%
\pgfpathlineto{\pgfqpoint{2.652710in}{2.667311in}}%
\pgfpathlineto{\pgfqpoint{2.706972in}{2.745037in}}%
\pgfpathlineto{\pgfqpoint{2.761165in}{2.800865in}}%
\pgfpathlineto{\pgfqpoint{2.815619in}{2.818100in}}%
\pgfpathlineto{\pgfqpoint{2.870097in}{2.818547in}}%
\pgfpathlineto{\pgfqpoint{2.925260in}{2.796863in}}%
\pgfpathlineto{\pgfqpoint{2.978807in}{2.745612in}}%
\pgfpathlineto{\pgfqpoint{3.035832in}{2.662638in}}%
\pgfpathlineto{\pgfqpoint{3.087526in}{2.561557in}}%
\pgfpathlineto{\pgfqpoint{3.142088in}{2.431258in}}%
\pgfpathlineto{\pgfqpoint{3.196771in}{2.281072in}}%
\pgfpathlineto{\pgfqpoint{3.250659in}{2.125249in}}%
\pgfpathlineto{\pgfqpoint{3.305076in}{1.963976in}}%
\pgfpathlineto{\pgfqpoint{3.359345in}{1.814773in}}%
\pgfpathlineto{\pgfqpoint{3.414317in}{1.684286in}}%
\pgfpathlineto{\pgfqpoint{3.468160in}{1.582419in}}%
\pgfpathlineto{\pgfqpoint{3.523339in}{1.505702in}}%
\pgfpathlineto{\pgfqpoint{3.577269in}{1.460931in}}%
\pgfpathlineto{\pgfqpoint{3.631162in}{1.440547in}}%
\pgfpathlineto{\pgfqpoint{3.685660in}{1.437789in}}%
\pgfpathlineto{\pgfqpoint{3.739808in}{1.454419in}}%
\pgfpathlineto{\pgfqpoint{3.793901in}{1.488484in}}%
\pgfpathlineto{\pgfqpoint{3.849309in}{1.540644in}}%
\pgfpathlineto{\pgfqpoint{3.903570in}{1.606883in}}%
\pgfpathlineto{\pgfqpoint{3.957903in}{1.683289in}}%
\pgfpathlineto{\pgfqpoint{4.012789in}{1.768234in}}%
\pgfpathlineto{\pgfqpoint{4.066735in}{1.843049in}}%
\pgfpathlineto{\pgfqpoint{4.120902in}{1.928073in}}%
\pgfpathlineto{\pgfqpoint{4.176220in}{2.018631in}}%
\pgfpathlineto{\pgfqpoint{4.229915in}{2.105205in}}%
\pgfpathlineto{\pgfqpoint{4.283837in}{2.201538in}}%
\pgfpathlineto{\pgfqpoint{4.337771in}{2.307979in}}%
\pgfpathlineto{\pgfqpoint{4.392424in}{2.399075in}}%
\pgfpathlineto{\pgfqpoint{4.446706in}{2.456713in}}%
\pgfpathlineto{\pgfqpoint{4.500988in}{2.492011in}}%
\pgfpathlineto{\pgfqpoint{4.555181in}{2.500393in}}%
\pgfpathlineto{\pgfqpoint{4.611088in}{2.484957in}}%
\pgfpathlineto{\pgfqpoint{4.664110in}{2.450898in}}%
\pgfpathlineto{\pgfqpoint{4.718185in}{2.404101in}}%
\pgfpathlineto{\pgfqpoint{4.772520in}{2.339668in}}%
\pgfpathlineto{\pgfqpoint{4.826399in}{2.266692in}}%
\pgfpathlineto{\pgfqpoint{4.880521in}{2.183936in}}%
\pgfpathlineto{\pgfqpoint{4.935345in}{2.084984in}}%
\pgfpathlineto{\pgfqpoint{4.989196in}{1.991201in}}%
\pgfpathlineto{\pgfqpoint{5.043972in}{1.916772in}}%
\pgfpathlineto{\pgfqpoint{5.098176in}{1.835545in}}%
\pgfpathlineto{\pgfqpoint{5.152554in}{1.755586in}}%
\pgfpathlineto{\pgfqpoint{5.206726in}{1.699792in}}%
\pgfpathlineto{\pgfqpoint{5.262396in}{1.673203in}}%
\pgfpathlineto{\pgfqpoint{5.316235in}{1.662567in}}%
\pgfpathlineto{\pgfqpoint{5.370220in}{1.655437in}}%
\pgfpathlineto{\pgfqpoint{5.424245in}{1.658396in}}%
\pgfpathlineto{\pgfqpoint{5.478748in}{1.674389in}}%
\pgfpathlineto{\pgfqpoint{5.532896in}{1.702744in}}%
\pgfusepath{stroke}%
\end{pgfscope}%
\begin{pgfscope}%
\pgfpathrectangle{\pgfqpoint{0.800000in}{0.528000in}}{\pgfqpoint{4.960000in}{3.696000in}}%
\pgfusepath{clip}%
\pgfsetrectcap%
\pgfsetroundjoin%
\pgfsetlinewidth{1.505625pt}%
\definecolor{currentstroke}{rgb}{0.549020,0.337255,0.294118}%
\pgfsetstrokecolor{currentstroke}%
\pgfsetdash{}{0pt}%
\pgfpathmoveto{\pgfqpoint{1.025455in}{2.150411in}}%
\pgfpathlineto{\pgfqpoint{5.530783in}{2.150411in}}%
\pgfusepath{stroke}%
\end{pgfscope}%
\begin{pgfscope}%
\pgfsetrectcap%
\pgfsetmiterjoin%
\pgfsetlinewidth{0.803000pt}%
\definecolor{currentstroke}{rgb}{0.000000,0.000000,0.000000}%
\pgfsetstrokecolor{currentstroke}%
\pgfsetdash{}{0pt}%
\pgfpathmoveto{\pgfqpoint{0.800000in}{0.528000in}}%
\pgfpathlineto{\pgfqpoint{0.800000in}{4.224000in}}%
\pgfusepath{stroke}%
\end{pgfscope}%
\begin{pgfscope}%
\pgfsetrectcap%
\pgfsetmiterjoin%
\pgfsetlinewidth{0.803000pt}%
\definecolor{currentstroke}{rgb}{0.000000,0.000000,0.000000}%
\pgfsetstrokecolor{currentstroke}%
\pgfsetdash{}{0pt}%
\pgfpathmoveto{\pgfqpoint{5.760000in}{0.528000in}}%
\pgfpathlineto{\pgfqpoint{5.760000in}{4.224000in}}%
\pgfusepath{stroke}%
\end{pgfscope}%
\begin{pgfscope}%
\pgfsetrectcap%
\pgfsetmiterjoin%
\pgfsetlinewidth{0.803000pt}%
\definecolor{currentstroke}{rgb}{0.000000,0.000000,0.000000}%
\pgfsetstrokecolor{currentstroke}%
\pgfsetdash{}{0pt}%
\pgfpathmoveto{\pgfqpoint{0.800000in}{0.528000in}}%
\pgfpathlineto{\pgfqpoint{5.760000in}{0.528000in}}%
\pgfusepath{stroke}%
\end{pgfscope}%
\begin{pgfscope}%
\pgfsetrectcap%
\pgfsetmiterjoin%
\pgfsetlinewidth{0.803000pt}%
\definecolor{currentstroke}{rgb}{0.000000,0.000000,0.000000}%
\pgfsetstrokecolor{currentstroke}%
\pgfsetdash{}{0pt}%
\pgfpathmoveto{\pgfqpoint{0.800000in}{4.224000in}}%
\pgfpathlineto{\pgfqpoint{5.760000in}{4.224000in}}%
\pgfusepath{stroke}%
\end{pgfscope}%
\begin{pgfscope}%
\definecolor{textcolor}{rgb}{0.000000,0.000000,0.000000}%
\pgfsetstrokecolor{textcolor}%
\pgfsetfillcolor{textcolor}%
\pgftext[x=3.280000in,y=4.307333in,,base]{\color{textcolor}\sffamily\fontsize{12.000000}{14.400000}\selectfont Measured forward position}%
\end{pgfscope}%
\begin{pgfscope}%
\pgfsetbuttcap%
\pgfsetmiterjoin%
\definecolor{currentfill}{rgb}{1.000000,1.000000,1.000000}%
\pgfsetfillcolor{currentfill}%
\pgfsetfillopacity{0.800000}%
\pgfsetlinewidth{1.003750pt}%
\definecolor{currentstroke}{rgb}{0.800000,0.800000,0.800000}%
\pgfsetstrokecolor{currentstroke}%
\pgfsetstrokeopacity{0.800000}%
\pgfsetdash{}{0pt}%
\pgfpathmoveto{\pgfqpoint{4.788646in}{2.889746in}}%
\pgfpathlineto{\pgfqpoint{5.662778in}{2.889746in}}%
\pgfpathquadraticcurveto{\pgfqpoint{5.690556in}{2.889746in}}{\pgfqpoint{5.690556in}{2.917523in}}%
\pgfpathlineto{\pgfqpoint{5.690556in}{4.126778in}}%
\pgfpathquadraticcurveto{\pgfqpoint{5.690556in}{4.154556in}}{\pgfqpoint{5.662778in}{4.154556in}}%
\pgfpathlineto{\pgfqpoint{4.788646in}{4.154556in}}%
\pgfpathquadraticcurveto{\pgfqpoint{4.760868in}{4.154556in}}{\pgfqpoint{4.760868in}{4.126778in}}%
\pgfpathlineto{\pgfqpoint{4.760868in}{2.917523in}}%
\pgfpathquadraticcurveto{\pgfqpoint{4.760868in}{2.889746in}}{\pgfqpoint{4.788646in}{2.889746in}}%
\pgfpathlineto{\pgfqpoint{4.788646in}{2.889746in}}%
\pgfpathclose%
\pgfusepath{stroke,fill}%
\end{pgfscope}%
\begin{pgfscope}%
\pgfsetrectcap%
\pgfsetroundjoin%
\pgfsetlinewidth{1.505625pt}%
\definecolor{currentstroke}{rgb}{0.121569,0.466667,0.705882}%
\pgfsetstrokecolor{currentstroke}%
\pgfsetdash{}{0pt}%
\pgfpathmoveto{\pgfqpoint{4.816424in}{4.042088in}}%
\pgfpathlineto{\pgfqpoint{4.955312in}{4.042088in}}%
\pgfpathlineto{\pgfqpoint{5.094201in}{4.042088in}}%
\pgfusepath{stroke}%
\end{pgfscope}%
\begin{pgfscope}%
\definecolor{textcolor}{rgb}{0.000000,0.000000,0.000000}%
\pgfsetstrokecolor{textcolor}%
\pgfsetfillcolor{textcolor}%
\pgftext[x=5.205312in,y=3.993477in,left,base]{\color{textcolor}\sffamily\fontsize{10.000000}{12.000000}\selectfont 0}%
\end{pgfscope}%
\begin{pgfscope}%
\pgfsetrectcap%
\pgfsetroundjoin%
\pgfsetlinewidth{1.505625pt}%
\definecolor{currentstroke}{rgb}{1.000000,0.498039,0.054902}%
\pgfsetstrokecolor{currentstroke}%
\pgfsetdash{}{0pt}%
\pgfpathmoveto{\pgfqpoint{4.816424in}{3.838231in}}%
\pgfpathlineto{\pgfqpoint{4.955312in}{3.838231in}}%
\pgfpathlineto{\pgfqpoint{5.094201in}{3.838231in}}%
\pgfusepath{stroke}%
\end{pgfscope}%
\begin{pgfscope}%
\definecolor{textcolor}{rgb}{0.000000,0.000000,0.000000}%
\pgfsetstrokecolor{textcolor}%
\pgfsetfillcolor{textcolor}%
\pgftext[x=5.205312in,y=3.789620in,left,base]{\color{textcolor}\sffamily\fontsize{10.000000}{12.000000}\selectfont 0.5}%
\end{pgfscope}%
\begin{pgfscope}%
\pgfsetrectcap%
\pgfsetroundjoin%
\pgfsetlinewidth{1.505625pt}%
\definecolor{currentstroke}{rgb}{0.172549,0.627451,0.172549}%
\pgfsetstrokecolor{currentstroke}%
\pgfsetdash{}{0pt}%
\pgfpathmoveto{\pgfqpoint{4.816424in}{3.634374in}}%
\pgfpathlineto{\pgfqpoint{4.955312in}{3.634374in}}%
\pgfpathlineto{\pgfqpoint{5.094201in}{3.634374in}}%
\pgfusepath{stroke}%
\end{pgfscope}%
\begin{pgfscope}%
\definecolor{textcolor}{rgb}{0.000000,0.000000,0.000000}%
\pgfsetstrokecolor{textcolor}%
\pgfsetfillcolor{textcolor}%
\pgftext[x=5.205312in,y=3.585762in,left,base]{\color{textcolor}\sffamily\fontsize{10.000000}{12.000000}\selectfont 1}%
\end{pgfscope}%
\begin{pgfscope}%
\pgfsetrectcap%
\pgfsetroundjoin%
\pgfsetlinewidth{1.505625pt}%
\definecolor{currentstroke}{rgb}{0.839216,0.152941,0.156863}%
\pgfsetstrokecolor{currentstroke}%
\pgfsetdash{}{0pt}%
\pgfpathmoveto{\pgfqpoint{4.816424in}{3.430516in}}%
\pgfpathlineto{\pgfqpoint{4.955312in}{3.430516in}}%
\pgfpathlineto{\pgfqpoint{5.094201in}{3.430516in}}%
\pgfusepath{stroke}%
\end{pgfscope}%
\begin{pgfscope}%
\definecolor{textcolor}{rgb}{0.000000,0.000000,0.000000}%
\pgfsetstrokecolor{textcolor}%
\pgfsetfillcolor{textcolor}%
\pgftext[x=5.205312in,y=3.381905in,left,base]{\color{textcolor}\sffamily\fontsize{10.000000}{12.000000}\selectfont 2}%
\end{pgfscope}%
\begin{pgfscope}%
\pgfsetrectcap%
\pgfsetroundjoin%
\pgfsetlinewidth{1.505625pt}%
\definecolor{currentstroke}{rgb}{0.580392,0.403922,0.741176}%
\pgfsetstrokecolor{currentstroke}%
\pgfsetdash{}{0pt}%
\pgfpathmoveto{\pgfqpoint{4.816424in}{3.226659in}}%
\pgfpathlineto{\pgfqpoint{4.955312in}{3.226659in}}%
\pgfpathlineto{\pgfqpoint{5.094201in}{3.226659in}}%
\pgfusepath{stroke}%
\end{pgfscope}%
\begin{pgfscope}%
\definecolor{textcolor}{rgb}{0.000000,0.000000,0.000000}%
\pgfsetstrokecolor{textcolor}%
\pgfsetfillcolor{textcolor}%
\pgftext[x=5.205312in,y=3.178048in,left,base]{\color{textcolor}\sffamily\fontsize{10.000000}{12.000000}\selectfont 3}%
\end{pgfscope}%
\begin{pgfscope}%
\pgfsetrectcap%
\pgfsetroundjoin%
\pgfsetlinewidth{1.505625pt}%
\definecolor{currentstroke}{rgb}{0.549020,0.337255,0.294118}%
\pgfsetstrokecolor{currentstroke}%
\pgfsetdash{}{0pt}%
\pgfpathmoveto{\pgfqpoint{4.816424in}{3.022802in}}%
\pgfpathlineto{\pgfqpoint{4.955312in}{3.022802in}}%
\pgfpathlineto{\pgfqpoint{5.094201in}{3.022802in}}%
\pgfusepath{stroke}%
\end{pgfscope}%
\begin{pgfscope}%
\definecolor{textcolor}{rgb}{0.000000,0.000000,0.000000}%
\pgfsetstrokecolor{textcolor}%
\pgfsetfillcolor{textcolor}%
\pgftext[x=5.205312in,y=2.974191in,left,base]{\color{textcolor}\sffamily\fontsize{10.000000}{12.000000}\selectfont Target}%
\end{pgfscope}%
\end{pgfpicture}%
\makeatother%
\endgroup%
}
    \end{minipage}
    \begin{minipage}[t]{0.5\linewidth}
        \centering
        \scalebox{0.55}{%% Creator: Matplotlib, PGF backend
%%
%% To include the figure in your LaTeX document, write
%%   \input{<filename>.pgf}
%%
%% Make sure the required packages are loaded in your preamble
%%   \usepackage{pgf}
%%
%% Also ensure that all the required font packages are loaded; for instance,
%% the lmodern package is sometimes necessary when using math font.
%%   \usepackage{lmodern}
%%
%% Figures using additional raster images can only be included by \input if
%% they are in the same directory as the main LaTeX file. For loading figures
%% from other directories you can use the `import` package
%%   \usepackage{import}
%%
%% and then include the figures with
%%   \import{<path to file>}{<filename>.pgf}
%%
%% Matplotlib used the following preamble
%%   \usepackage{fontspec}
%%   \setmainfont{DejaVuSerif.ttf}[Path=\detokenize{/home/lgonz/tfg-aero/tfg-giaa-dronecontrol/venv/lib/python3.8/site-packages/matplotlib/mpl-data/fonts/ttf/}]
%%   \setsansfont{DejaVuSans.ttf}[Path=\detokenize{/home/lgonz/tfg-aero/tfg-giaa-dronecontrol/venv/lib/python3.8/site-packages/matplotlib/mpl-data/fonts/ttf/}]
%%   \setmonofont{DejaVuSansMono.ttf}[Path=\detokenize{/home/lgonz/tfg-aero/tfg-giaa-dronecontrol/venv/lib/python3.8/site-packages/matplotlib/mpl-data/fonts/ttf/}]
%%
\begingroup%
\makeatletter%
\begin{pgfpicture}%
\pgfpathrectangle{\pgfpointorigin}{\pgfqpoint{6.400000in}{4.800000in}}%
\pgfusepath{use as bounding box, clip}%
\begin{pgfscope}%
\pgfsetbuttcap%
\pgfsetmiterjoin%
\definecolor{currentfill}{rgb}{1.000000,1.000000,1.000000}%
\pgfsetfillcolor{currentfill}%
\pgfsetlinewidth{0.000000pt}%
\definecolor{currentstroke}{rgb}{1.000000,1.000000,1.000000}%
\pgfsetstrokecolor{currentstroke}%
\pgfsetdash{}{0pt}%
\pgfpathmoveto{\pgfqpoint{0.000000in}{0.000000in}}%
\pgfpathlineto{\pgfqpoint{6.400000in}{0.000000in}}%
\pgfpathlineto{\pgfqpoint{6.400000in}{4.800000in}}%
\pgfpathlineto{\pgfqpoint{0.000000in}{4.800000in}}%
\pgfpathlineto{\pgfqpoint{0.000000in}{0.000000in}}%
\pgfpathclose%
\pgfusepath{fill}%
\end{pgfscope}%
\begin{pgfscope}%
\pgfsetbuttcap%
\pgfsetmiterjoin%
\definecolor{currentfill}{rgb}{1.000000,1.000000,1.000000}%
\pgfsetfillcolor{currentfill}%
\pgfsetlinewidth{0.000000pt}%
\definecolor{currentstroke}{rgb}{0.000000,0.000000,0.000000}%
\pgfsetstrokecolor{currentstroke}%
\pgfsetstrokeopacity{0.000000}%
\pgfsetdash{}{0pt}%
\pgfpathmoveto{\pgfqpoint{0.800000in}{0.528000in}}%
\pgfpathlineto{\pgfqpoint{5.760000in}{0.528000in}}%
\pgfpathlineto{\pgfqpoint{5.760000in}{4.224000in}}%
\pgfpathlineto{\pgfqpoint{0.800000in}{4.224000in}}%
\pgfpathlineto{\pgfqpoint{0.800000in}{0.528000in}}%
\pgfpathclose%
\pgfusepath{fill}%
\end{pgfscope}%
\begin{pgfscope}%
\pgfpathrectangle{\pgfqpoint{0.800000in}{0.528000in}}{\pgfqpoint{4.960000in}{3.696000in}}%
\pgfusepath{clip}%
\pgfsetrectcap%
\pgfsetroundjoin%
\pgfsetlinewidth{0.803000pt}%
\definecolor{currentstroke}{rgb}{0.690196,0.690196,0.690196}%
\pgfsetstrokecolor{currentstroke}%
\pgfsetdash{}{0pt}%
\pgfpathmoveto{\pgfqpoint{1.025455in}{0.528000in}}%
\pgfpathlineto{\pgfqpoint{1.025455in}{4.224000in}}%
\pgfusepath{stroke}%
\end{pgfscope}%
\begin{pgfscope}%
\pgfsetbuttcap%
\pgfsetroundjoin%
\definecolor{currentfill}{rgb}{0.000000,0.000000,0.000000}%
\pgfsetfillcolor{currentfill}%
\pgfsetlinewidth{0.803000pt}%
\definecolor{currentstroke}{rgb}{0.000000,0.000000,0.000000}%
\pgfsetstrokecolor{currentstroke}%
\pgfsetdash{}{0pt}%
\pgfsys@defobject{currentmarker}{\pgfqpoint{0.000000in}{-0.048611in}}{\pgfqpoint{0.000000in}{0.000000in}}{%
\pgfpathmoveto{\pgfqpoint{0.000000in}{0.000000in}}%
\pgfpathlineto{\pgfqpoint{0.000000in}{-0.048611in}}%
\pgfusepath{stroke,fill}%
}%
\begin{pgfscope}%
\pgfsys@transformshift{1.025455in}{0.528000in}%
\pgfsys@useobject{currentmarker}{}%
\end{pgfscope}%
\end{pgfscope}%
\begin{pgfscope}%
\definecolor{textcolor}{rgb}{0.000000,0.000000,0.000000}%
\pgfsetstrokecolor{textcolor}%
\pgfsetfillcolor{textcolor}%
\pgftext[x=1.025455in,y=0.430778in,,top]{\color{textcolor}\sffamily\fontsize{10.000000}{12.000000}\selectfont 0}%
\end{pgfscope}%
\begin{pgfscope}%
\pgfpathrectangle{\pgfqpoint{0.800000in}{0.528000in}}{\pgfqpoint{4.960000in}{3.696000in}}%
\pgfusepath{clip}%
\pgfsetrectcap%
\pgfsetroundjoin%
\pgfsetlinewidth{0.803000pt}%
\definecolor{currentstroke}{rgb}{0.690196,0.690196,0.690196}%
\pgfsetstrokecolor{currentstroke}%
\pgfsetdash{}{0pt}%
\pgfpathmoveto{\pgfqpoint{1.776343in}{0.528000in}}%
\pgfpathlineto{\pgfqpoint{1.776343in}{4.224000in}}%
\pgfusepath{stroke}%
\end{pgfscope}%
\begin{pgfscope}%
\pgfsetbuttcap%
\pgfsetroundjoin%
\definecolor{currentfill}{rgb}{0.000000,0.000000,0.000000}%
\pgfsetfillcolor{currentfill}%
\pgfsetlinewidth{0.803000pt}%
\definecolor{currentstroke}{rgb}{0.000000,0.000000,0.000000}%
\pgfsetstrokecolor{currentstroke}%
\pgfsetdash{}{0pt}%
\pgfsys@defobject{currentmarker}{\pgfqpoint{0.000000in}{-0.048611in}}{\pgfqpoint{0.000000in}{0.000000in}}{%
\pgfpathmoveto{\pgfqpoint{0.000000in}{0.000000in}}%
\pgfpathlineto{\pgfqpoint{0.000000in}{-0.048611in}}%
\pgfusepath{stroke,fill}%
}%
\begin{pgfscope}%
\pgfsys@transformshift{1.776343in}{0.528000in}%
\pgfsys@useobject{currentmarker}{}%
\end{pgfscope}%
\end{pgfscope}%
\begin{pgfscope}%
\definecolor{textcolor}{rgb}{0.000000,0.000000,0.000000}%
\pgfsetstrokecolor{textcolor}%
\pgfsetfillcolor{textcolor}%
\pgftext[x=1.776343in,y=0.430778in,,top]{\color{textcolor}\sffamily\fontsize{10.000000}{12.000000}\selectfont 5}%
\end{pgfscope}%
\begin{pgfscope}%
\pgfpathrectangle{\pgfqpoint{0.800000in}{0.528000in}}{\pgfqpoint{4.960000in}{3.696000in}}%
\pgfusepath{clip}%
\pgfsetrectcap%
\pgfsetroundjoin%
\pgfsetlinewidth{0.803000pt}%
\definecolor{currentstroke}{rgb}{0.690196,0.690196,0.690196}%
\pgfsetstrokecolor{currentstroke}%
\pgfsetdash{}{0pt}%
\pgfpathmoveto{\pgfqpoint{2.527231in}{0.528000in}}%
\pgfpathlineto{\pgfqpoint{2.527231in}{4.224000in}}%
\pgfusepath{stroke}%
\end{pgfscope}%
\begin{pgfscope}%
\pgfsetbuttcap%
\pgfsetroundjoin%
\definecolor{currentfill}{rgb}{0.000000,0.000000,0.000000}%
\pgfsetfillcolor{currentfill}%
\pgfsetlinewidth{0.803000pt}%
\definecolor{currentstroke}{rgb}{0.000000,0.000000,0.000000}%
\pgfsetstrokecolor{currentstroke}%
\pgfsetdash{}{0pt}%
\pgfsys@defobject{currentmarker}{\pgfqpoint{0.000000in}{-0.048611in}}{\pgfqpoint{0.000000in}{0.000000in}}{%
\pgfpathmoveto{\pgfqpoint{0.000000in}{0.000000in}}%
\pgfpathlineto{\pgfqpoint{0.000000in}{-0.048611in}}%
\pgfusepath{stroke,fill}%
}%
\begin{pgfscope}%
\pgfsys@transformshift{2.527231in}{0.528000in}%
\pgfsys@useobject{currentmarker}{}%
\end{pgfscope}%
\end{pgfscope}%
\begin{pgfscope}%
\definecolor{textcolor}{rgb}{0.000000,0.000000,0.000000}%
\pgfsetstrokecolor{textcolor}%
\pgfsetfillcolor{textcolor}%
\pgftext[x=2.527231in,y=0.430778in,,top]{\color{textcolor}\sffamily\fontsize{10.000000}{12.000000}\selectfont 10}%
\end{pgfscope}%
\begin{pgfscope}%
\pgfpathrectangle{\pgfqpoint{0.800000in}{0.528000in}}{\pgfqpoint{4.960000in}{3.696000in}}%
\pgfusepath{clip}%
\pgfsetrectcap%
\pgfsetroundjoin%
\pgfsetlinewidth{0.803000pt}%
\definecolor{currentstroke}{rgb}{0.690196,0.690196,0.690196}%
\pgfsetstrokecolor{currentstroke}%
\pgfsetdash{}{0pt}%
\pgfpathmoveto{\pgfqpoint{3.278119in}{0.528000in}}%
\pgfpathlineto{\pgfqpoint{3.278119in}{4.224000in}}%
\pgfusepath{stroke}%
\end{pgfscope}%
\begin{pgfscope}%
\pgfsetbuttcap%
\pgfsetroundjoin%
\definecolor{currentfill}{rgb}{0.000000,0.000000,0.000000}%
\pgfsetfillcolor{currentfill}%
\pgfsetlinewidth{0.803000pt}%
\definecolor{currentstroke}{rgb}{0.000000,0.000000,0.000000}%
\pgfsetstrokecolor{currentstroke}%
\pgfsetdash{}{0pt}%
\pgfsys@defobject{currentmarker}{\pgfqpoint{0.000000in}{-0.048611in}}{\pgfqpoint{0.000000in}{0.000000in}}{%
\pgfpathmoveto{\pgfqpoint{0.000000in}{0.000000in}}%
\pgfpathlineto{\pgfqpoint{0.000000in}{-0.048611in}}%
\pgfusepath{stroke,fill}%
}%
\begin{pgfscope}%
\pgfsys@transformshift{3.278119in}{0.528000in}%
\pgfsys@useobject{currentmarker}{}%
\end{pgfscope}%
\end{pgfscope}%
\begin{pgfscope}%
\definecolor{textcolor}{rgb}{0.000000,0.000000,0.000000}%
\pgfsetstrokecolor{textcolor}%
\pgfsetfillcolor{textcolor}%
\pgftext[x=3.278119in,y=0.430778in,,top]{\color{textcolor}\sffamily\fontsize{10.000000}{12.000000}\selectfont 15}%
\end{pgfscope}%
\begin{pgfscope}%
\pgfpathrectangle{\pgfqpoint{0.800000in}{0.528000in}}{\pgfqpoint{4.960000in}{3.696000in}}%
\pgfusepath{clip}%
\pgfsetrectcap%
\pgfsetroundjoin%
\pgfsetlinewidth{0.803000pt}%
\definecolor{currentstroke}{rgb}{0.690196,0.690196,0.690196}%
\pgfsetstrokecolor{currentstroke}%
\pgfsetdash{}{0pt}%
\pgfpathmoveto{\pgfqpoint{4.029007in}{0.528000in}}%
\pgfpathlineto{\pgfqpoint{4.029007in}{4.224000in}}%
\pgfusepath{stroke}%
\end{pgfscope}%
\begin{pgfscope}%
\pgfsetbuttcap%
\pgfsetroundjoin%
\definecolor{currentfill}{rgb}{0.000000,0.000000,0.000000}%
\pgfsetfillcolor{currentfill}%
\pgfsetlinewidth{0.803000pt}%
\definecolor{currentstroke}{rgb}{0.000000,0.000000,0.000000}%
\pgfsetstrokecolor{currentstroke}%
\pgfsetdash{}{0pt}%
\pgfsys@defobject{currentmarker}{\pgfqpoint{0.000000in}{-0.048611in}}{\pgfqpoint{0.000000in}{0.000000in}}{%
\pgfpathmoveto{\pgfqpoint{0.000000in}{0.000000in}}%
\pgfpathlineto{\pgfqpoint{0.000000in}{-0.048611in}}%
\pgfusepath{stroke,fill}%
}%
\begin{pgfscope}%
\pgfsys@transformshift{4.029007in}{0.528000in}%
\pgfsys@useobject{currentmarker}{}%
\end{pgfscope}%
\end{pgfscope}%
\begin{pgfscope}%
\definecolor{textcolor}{rgb}{0.000000,0.000000,0.000000}%
\pgfsetstrokecolor{textcolor}%
\pgfsetfillcolor{textcolor}%
\pgftext[x=4.029007in,y=0.430778in,,top]{\color{textcolor}\sffamily\fontsize{10.000000}{12.000000}\selectfont 20}%
\end{pgfscope}%
\begin{pgfscope}%
\pgfpathrectangle{\pgfqpoint{0.800000in}{0.528000in}}{\pgfqpoint{4.960000in}{3.696000in}}%
\pgfusepath{clip}%
\pgfsetrectcap%
\pgfsetroundjoin%
\pgfsetlinewidth{0.803000pt}%
\definecolor{currentstroke}{rgb}{0.690196,0.690196,0.690196}%
\pgfsetstrokecolor{currentstroke}%
\pgfsetdash{}{0pt}%
\pgfpathmoveto{\pgfqpoint{4.779895in}{0.528000in}}%
\pgfpathlineto{\pgfqpoint{4.779895in}{4.224000in}}%
\pgfusepath{stroke}%
\end{pgfscope}%
\begin{pgfscope}%
\pgfsetbuttcap%
\pgfsetroundjoin%
\definecolor{currentfill}{rgb}{0.000000,0.000000,0.000000}%
\pgfsetfillcolor{currentfill}%
\pgfsetlinewidth{0.803000pt}%
\definecolor{currentstroke}{rgb}{0.000000,0.000000,0.000000}%
\pgfsetstrokecolor{currentstroke}%
\pgfsetdash{}{0pt}%
\pgfsys@defobject{currentmarker}{\pgfqpoint{0.000000in}{-0.048611in}}{\pgfqpoint{0.000000in}{0.000000in}}{%
\pgfpathmoveto{\pgfqpoint{0.000000in}{0.000000in}}%
\pgfpathlineto{\pgfqpoint{0.000000in}{-0.048611in}}%
\pgfusepath{stroke,fill}%
}%
\begin{pgfscope}%
\pgfsys@transformshift{4.779895in}{0.528000in}%
\pgfsys@useobject{currentmarker}{}%
\end{pgfscope}%
\end{pgfscope}%
\begin{pgfscope}%
\definecolor{textcolor}{rgb}{0.000000,0.000000,0.000000}%
\pgfsetstrokecolor{textcolor}%
\pgfsetfillcolor{textcolor}%
\pgftext[x=4.779895in,y=0.430778in,,top]{\color{textcolor}\sffamily\fontsize{10.000000}{12.000000}\selectfont 25}%
\end{pgfscope}%
\begin{pgfscope}%
\pgfpathrectangle{\pgfqpoint{0.800000in}{0.528000in}}{\pgfqpoint{4.960000in}{3.696000in}}%
\pgfusepath{clip}%
\pgfsetrectcap%
\pgfsetroundjoin%
\pgfsetlinewidth{0.803000pt}%
\definecolor{currentstroke}{rgb}{0.690196,0.690196,0.690196}%
\pgfsetstrokecolor{currentstroke}%
\pgfsetdash{}{0pt}%
\pgfpathmoveto{\pgfqpoint{5.530783in}{0.528000in}}%
\pgfpathlineto{\pgfqpoint{5.530783in}{4.224000in}}%
\pgfusepath{stroke}%
\end{pgfscope}%
\begin{pgfscope}%
\pgfsetbuttcap%
\pgfsetroundjoin%
\definecolor{currentfill}{rgb}{0.000000,0.000000,0.000000}%
\pgfsetfillcolor{currentfill}%
\pgfsetlinewidth{0.803000pt}%
\definecolor{currentstroke}{rgb}{0.000000,0.000000,0.000000}%
\pgfsetstrokecolor{currentstroke}%
\pgfsetdash{}{0pt}%
\pgfsys@defobject{currentmarker}{\pgfqpoint{0.000000in}{-0.048611in}}{\pgfqpoint{0.000000in}{0.000000in}}{%
\pgfpathmoveto{\pgfqpoint{0.000000in}{0.000000in}}%
\pgfpathlineto{\pgfqpoint{0.000000in}{-0.048611in}}%
\pgfusepath{stroke,fill}%
}%
\begin{pgfscope}%
\pgfsys@transformshift{5.530783in}{0.528000in}%
\pgfsys@useobject{currentmarker}{}%
\end{pgfscope}%
\end{pgfscope}%
\begin{pgfscope}%
\definecolor{textcolor}{rgb}{0.000000,0.000000,0.000000}%
\pgfsetstrokecolor{textcolor}%
\pgfsetfillcolor{textcolor}%
\pgftext[x=5.530783in,y=0.430778in,,top]{\color{textcolor}\sffamily\fontsize{10.000000}{12.000000}\selectfont 30}%
\end{pgfscope}%
\begin{pgfscope}%
\definecolor{textcolor}{rgb}{0.000000,0.000000,0.000000}%
\pgfsetstrokecolor{textcolor}%
\pgfsetfillcolor{textcolor}%
\pgftext[x=3.280000in,y=0.240809in,,top]{\color{textcolor}\sffamily\fontsize{10.000000}{12.000000}\selectfont time [s]}%
\end{pgfscope}%
\begin{pgfscope}%
\pgfpathrectangle{\pgfqpoint{0.800000in}{0.528000in}}{\pgfqpoint{4.960000in}{3.696000in}}%
\pgfusepath{clip}%
\pgfsetrectcap%
\pgfsetroundjoin%
\pgfsetlinewidth{0.803000pt}%
\definecolor{currentstroke}{rgb}{0.690196,0.690196,0.690196}%
\pgfsetstrokecolor{currentstroke}%
\pgfsetdash{}{0pt}%
\pgfpathmoveto{\pgfqpoint{0.800000in}{0.696000in}}%
\pgfpathlineto{\pgfqpoint{5.760000in}{0.696000in}}%
\pgfusepath{stroke}%
\end{pgfscope}%
\begin{pgfscope}%
\pgfsetbuttcap%
\pgfsetroundjoin%
\definecolor{currentfill}{rgb}{0.000000,0.000000,0.000000}%
\pgfsetfillcolor{currentfill}%
\pgfsetlinewidth{0.803000pt}%
\definecolor{currentstroke}{rgb}{0.000000,0.000000,0.000000}%
\pgfsetstrokecolor{currentstroke}%
\pgfsetdash{}{0pt}%
\pgfsys@defobject{currentmarker}{\pgfqpoint{-0.048611in}{0.000000in}}{\pgfqpoint{-0.000000in}{0.000000in}}{%
\pgfpathmoveto{\pgfqpoint{-0.000000in}{0.000000in}}%
\pgfpathlineto{\pgfqpoint{-0.048611in}{0.000000in}}%
\pgfusepath{stroke,fill}%
}%
\begin{pgfscope}%
\pgfsys@transformshift{0.800000in}{0.696000in}%
\pgfsys@useobject{currentmarker}{}%
\end{pgfscope}%
\end{pgfscope}%
\begin{pgfscope}%
\definecolor{textcolor}{rgb}{0.000000,0.000000,0.000000}%
\pgfsetstrokecolor{textcolor}%
\pgfsetfillcolor{textcolor}%
\pgftext[x=0.481898in, y=0.643238in, left, base]{\color{textcolor}\sffamily\fontsize{10.000000}{12.000000}\selectfont 0.0}%
\end{pgfscope}%
\begin{pgfscope}%
\pgfpathrectangle{\pgfqpoint{0.800000in}{0.528000in}}{\pgfqpoint{4.960000in}{3.696000in}}%
\pgfusepath{clip}%
\pgfsetrectcap%
\pgfsetroundjoin%
\pgfsetlinewidth{0.803000pt}%
\definecolor{currentstroke}{rgb}{0.690196,0.690196,0.690196}%
\pgfsetstrokecolor{currentstroke}%
\pgfsetdash{}{0pt}%
\pgfpathmoveto{\pgfqpoint{0.800000in}{1.438100in}}%
\pgfpathlineto{\pgfqpoint{5.760000in}{1.438100in}}%
\pgfusepath{stroke}%
\end{pgfscope}%
\begin{pgfscope}%
\pgfsetbuttcap%
\pgfsetroundjoin%
\definecolor{currentfill}{rgb}{0.000000,0.000000,0.000000}%
\pgfsetfillcolor{currentfill}%
\pgfsetlinewidth{0.803000pt}%
\definecolor{currentstroke}{rgb}{0.000000,0.000000,0.000000}%
\pgfsetstrokecolor{currentstroke}%
\pgfsetdash{}{0pt}%
\pgfsys@defobject{currentmarker}{\pgfqpoint{-0.048611in}{0.000000in}}{\pgfqpoint{-0.000000in}{0.000000in}}{%
\pgfpathmoveto{\pgfqpoint{-0.000000in}{0.000000in}}%
\pgfpathlineto{\pgfqpoint{-0.048611in}{0.000000in}}%
\pgfusepath{stroke,fill}%
}%
\begin{pgfscope}%
\pgfsys@transformshift{0.800000in}{1.438100in}%
\pgfsys@useobject{currentmarker}{}%
\end{pgfscope}%
\end{pgfscope}%
\begin{pgfscope}%
\definecolor{textcolor}{rgb}{0.000000,0.000000,0.000000}%
\pgfsetstrokecolor{textcolor}%
\pgfsetfillcolor{textcolor}%
\pgftext[x=0.481898in, y=1.385338in, left, base]{\color{textcolor}\sffamily\fontsize{10.000000}{12.000000}\selectfont 0.1}%
\end{pgfscope}%
\begin{pgfscope}%
\pgfpathrectangle{\pgfqpoint{0.800000in}{0.528000in}}{\pgfqpoint{4.960000in}{3.696000in}}%
\pgfusepath{clip}%
\pgfsetrectcap%
\pgfsetroundjoin%
\pgfsetlinewidth{0.803000pt}%
\definecolor{currentstroke}{rgb}{0.690196,0.690196,0.690196}%
\pgfsetstrokecolor{currentstroke}%
\pgfsetdash{}{0pt}%
\pgfpathmoveto{\pgfqpoint{0.800000in}{2.180200in}}%
\pgfpathlineto{\pgfqpoint{5.760000in}{2.180200in}}%
\pgfusepath{stroke}%
\end{pgfscope}%
\begin{pgfscope}%
\pgfsetbuttcap%
\pgfsetroundjoin%
\definecolor{currentfill}{rgb}{0.000000,0.000000,0.000000}%
\pgfsetfillcolor{currentfill}%
\pgfsetlinewidth{0.803000pt}%
\definecolor{currentstroke}{rgb}{0.000000,0.000000,0.000000}%
\pgfsetstrokecolor{currentstroke}%
\pgfsetdash{}{0pt}%
\pgfsys@defobject{currentmarker}{\pgfqpoint{-0.048611in}{0.000000in}}{\pgfqpoint{-0.000000in}{0.000000in}}{%
\pgfpathmoveto{\pgfqpoint{-0.000000in}{0.000000in}}%
\pgfpathlineto{\pgfqpoint{-0.048611in}{0.000000in}}%
\pgfusepath{stroke,fill}%
}%
\begin{pgfscope}%
\pgfsys@transformshift{0.800000in}{2.180200in}%
\pgfsys@useobject{currentmarker}{}%
\end{pgfscope}%
\end{pgfscope}%
\begin{pgfscope}%
\definecolor{textcolor}{rgb}{0.000000,0.000000,0.000000}%
\pgfsetstrokecolor{textcolor}%
\pgfsetfillcolor{textcolor}%
\pgftext[x=0.481898in, y=2.127438in, left, base]{\color{textcolor}\sffamily\fontsize{10.000000}{12.000000}\selectfont 0.2}%
\end{pgfscope}%
\begin{pgfscope}%
\pgfpathrectangle{\pgfqpoint{0.800000in}{0.528000in}}{\pgfqpoint{4.960000in}{3.696000in}}%
\pgfusepath{clip}%
\pgfsetrectcap%
\pgfsetroundjoin%
\pgfsetlinewidth{0.803000pt}%
\definecolor{currentstroke}{rgb}{0.690196,0.690196,0.690196}%
\pgfsetstrokecolor{currentstroke}%
\pgfsetdash{}{0pt}%
\pgfpathmoveto{\pgfqpoint{0.800000in}{2.922300in}}%
\pgfpathlineto{\pgfqpoint{5.760000in}{2.922300in}}%
\pgfusepath{stroke}%
\end{pgfscope}%
\begin{pgfscope}%
\pgfsetbuttcap%
\pgfsetroundjoin%
\definecolor{currentfill}{rgb}{0.000000,0.000000,0.000000}%
\pgfsetfillcolor{currentfill}%
\pgfsetlinewidth{0.803000pt}%
\definecolor{currentstroke}{rgb}{0.000000,0.000000,0.000000}%
\pgfsetstrokecolor{currentstroke}%
\pgfsetdash{}{0pt}%
\pgfsys@defobject{currentmarker}{\pgfqpoint{-0.048611in}{0.000000in}}{\pgfqpoint{-0.000000in}{0.000000in}}{%
\pgfpathmoveto{\pgfqpoint{-0.000000in}{0.000000in}}%
\pgfpathlineto{\pgfqpoint{-0.048611in}{0.000000in}}%
\pgfusepath{stroke,fill}%
}%
\begin{pgfscope}%
\pgfsys@transformshift{0.800000in}{2.922300in}%
\pgfsys@useobject{currentmarker}{}%
\end{pgfscope}%
\end{pgfscope}%
\begin{pgfscope}%
\definecolor{textcolor}{rgb}{0.000000,0.000000,0.000000}%
\pgfsetstrokecolor{textcolor}%
\pgfsetfillcolor{textcolor}%
\pgftext[x=0.481898in, y=2.869538in, left, base]{\color{textcolor}\sffamily\fontsize{10.000000}{12.000000}\selectfont 0.3}%
\end{pgfscope}%
\begin{pgfscope}%
\pgfpathrectangle{\pgfqpoint{0.800000in}{0.528000in}}{\pgfqpoint{4.960000in}{3.696000in}}%
\pgfusepath{clip}%
\pgfsetrectcap%
\pgfsetroundjoin%
\pgfsetlinewidth{0.803000pt}%
\definecolor{currentstroke}{rgb}{0.690196,0.690196,0.690196}%
\pgfsetstrokecolor{currentstroke}%
\pgfsetdash{}{0pt}%
\pgfpathmoveto{\pgfqpoint{0.800000in}{3.664400in}}%
\pgfpathlineto{\pgfqpoint{5.760000in}{3.664400in}}%
\pgfusepath{stroke}%
\end{pgfscope}%
\begin{pgfscope}%
\pgfsetbuttcap%
\pgfsetroundjoin%
\definecolor{currentfill}{rgb}{0.000000,0.000000,0.000000}%
\pgfsetfillcolor{currentfill}%
\pgfsetlinewidth{0.803000pt}%
\definecolor{currentstroke}{rgb}{0.000000,0.000000,0.000000}%
\pgfsetstrokecolor{currentstroke}%
\pgfsetdash{}{0pt}%
\pgfsys@defobject{currentmarker}{\pgfqpoint{-0.048611in}{0.000000in}}{\pgfqpoint{-0.000000in}{0.000000in}}{%
\pgfpathmoveto{\pgfqpoint{-0.000000in}{0.000000in}}%
\pgfpathlineto{\pgfqpoint{-0.048611in}{0.000000in}}%
\pgfusepath{stroke,fill}%
}%
\begin{pgfscope}%
\pgfsys@transformshift{0.800000in}{3.664400in}%
\pgfsys@useobject{currentmarker}{}%
\end{pgfscope}%
\end{pgfscope}%
\begin{pgfscope}%
\definecolor{textcolor}{rgb}{0.000000,0.000000,0.000000}%
\pgfsetstrokecolor{textcolor}%
\pgfsetfillcolor{textcolor}%
\pgftext[x=0.481898in, y=3.611638in, left, base]{\color{textcolor}\sffamily\fontsize{10.000000}{12.000000}\selectfont 0.4}%
\end{pgfscope}%
\begin{pgfscope}%
\definecolor{textcolor}{rgb}{0.000000,0.000000,0.000000}%
\pgfsetstrokecolor{textcolor}%
\pgfsetfillcolor{textcolor}%
\pgftext[x=0.426343in,y=2.376000in,,bottom,rotate=90.000000]{\color{textcolor}\sffamily\fontsize{10.000000}{12.000000}\selectfont Velocity [m/s]}%
\end{pgfscope}%
\begin{pgfscope}%
\pgfpathrectangle{\pgfqpoint{0.800000in}{0.528000in}}{\pgfqpoint{4.960000in}{3.696000in}}%
\pgfusepath{clip}%
\pgfsetrectcap%
\pgfsetroundjoin%
\pgfsetlinewidth{1.505625pt}%
\definecolor{currentstroke}{rgb}{0.121569,0.466667,0.705882}%
\pgfsetstrokecolor{currentstroke}%
\pgfsetdash{}{0pt}%
\pgfpathmoveto{\pgfqpoint{1.025455in}{0.696000in}}%
\pgfpathlineto{\pgfqpoint{1.079735in}{1.737587in}}%
\pgfpathlineto{\pgfqpoint{1.135063in}{2.638261in}}%
\pgfpathlineto{\pgfqpoint{1.189075in}{3.162854in}}%
\pgfpathlineto{\pgfqpoint{1.243658in}{3.531560in}}%
\pgfpathlineto{\pgfqpoint{1.299689in}{3.826923in}}%
\pgfpathlineto{\pgfqpoint{1.351745in}{3.974704in}}%
\pgfpathlineto{\pgfqpoint{1.405496in}{3.302873in}}%
\pgfpathlineto{\pgfqpoint{1.459653in}{2.483213in}}%
\pgfpathlineto{\pgfqpoint{1.514227in}{1.368000in}}%
\pgfpathlineto{\pgfqpoint{1.568240in}{1.074398in}}%
\pgfpathlineto{\pgfqpoint{1.624651in}{1.330050in}}%
\pgfpathlineto{\pgfqpoint{1.679031in}{1.165345in}}%
\pgfpathlineto{\pgfqpoint{1.733275in}{0.861939in}}%
\pgfpathlineto{\pgfqpoint{1.787527in}{0.770210in}}%
\pgfpathlineto{\pgfqpoint{1.844160in}{0.696000in}}%
\pgfpathlineto{\pgfqpoint{1.896960in}{0.696000in}}%
\pgfpathlineto{\pgfqpoint{1.952216in}{0.696000in}}%
\pgfpathlineto{\pgfqpoint{2.005474in}{0.696000in}}%
\pgfpathlineto{\pgfqpoint{2.060575in}{0.696000in}}%
\pgfpathlineto{\pgfqpoint{2.114239in}{0.696000in}}%
\pgfpathlineto{\pgfqpoint{2.167985in}{0.696000in}}%
\pgfpathlineto{\pgfqpoint{2.223457in}{0.696000in}}%
\pgfpathlineto{\pgfqpoint{2.276526in}{0.800949in}}%
\pgfpathlineto{\pgfqpoint{2.331554in}{0.696000in}}%
\pgfpathlineto{\pgfqpoint{2.386592in}{0.770210in}}%
\pgfpathlineto{\pgfqpoint{2.439992in}{0.696000in}}%
\pgfpathlineto{\pgfqpoint{2.494740in}{0.696000in}}%
\pgfpathlineto{\pgfqpoint{2.549385in}{0.696000in}}%
\pgfpathlineto{\pgfqpoint{2.602877in}{0.696000in}}%
\pgfpathlineto{\pgfqpoint{2.657535in}{0.696000in}}%
\pgfpathlineto{\pgfqpoint{2.711868in}{0.770210in}}%
\pgfpathlineto{\pgfqpoint{2.767412in}{0.770210in}}%
\pgfpathlineto{\pgfqpoint{2.822529in}{0.770210in}}%
\pgfpathlineto{\pgfqpoint{2.875151in}{0.770210in}}%
\pgfpathlineto{\pgfqpoint{2.928591in}{0.770210in}}%
\pgfpathlineto{\pgfqpoint{2.982157in}{0.844420in}}%
\pgfpathlineto{\pgfqpoint{3.035815in}{0.770210in}}%
\pgfpathlineto{\pgfqpoint{3.090489in}{0.696000in}}%
\pgfpathlineto{\pgfqpoint{3.144495in}{0.696000in}}%
\pgfpathlineto{\pgfqpoint{3.200514in}{0.770210in}}%
\pgfpathlineto{\pgfqpoint{3.253298in}{0.770210in}}%
\pgfpathlineto{\pgfqpoint{3.307342in}{0.770210in}}%
\pgfpathlineto{\pgfqpoint{3.362302in}{0.770210in}}%
\pgfpathlineto{\pgfqpoint{3.416387in}{0.770210in}}%
\pgfpathlineto{\pgfqpoint{3.470648in}{0.770210in}}%
\pgfpathlineto{\pgfqpoint{3.524668in}{0.696000in}}%
\pgfpathlineto{\pgfqpoint{3.579123in}{0.696000in}}%
\pgfpathlineto{\pgfqpoint{3.632992in}{0.844420in}}%
\pgfpathlineto{\pgfqpoint{3.687547in}{0.844420in}}%
\pgfpathlineto{\pgfqpoint{3.741880in}{0.844420in}}%
\pgfpathlineto{\pgfqpoint{3.796081in}{0.918630in}}%
\pgfpathlineto{\pgfqpoint{3.851858in}{0.918630in}}%
\pgfpathlineto{\pgfqpoint{3.905695in}{0.930673in}}%
\pgfpathlineto{\pgfqpoint{3.959560in}{0.918630in}}%
\pgfpathlineto{\pgfqpoint{4.014993in}{0.696000in}}%
\pgfpathlineto{\pgfqpoint{4.068019in}{0.770210in}}%
\pgfpathlineto{\pgfqpoint{4.122358in}{0.770210in}}%
\pgfpathlineto{\pgfqpoint{4.176154in}{0.992840in}}%
\pgfpathlineto{\pgfqpoint{4.230593in}{0.992840in}}%
\pgfpathlineto{\pgfqpoint{4.284345in}{0.696000in}}%
\pgfpathlineto{\pgfqpoint{4.339380in}{0.696000in}}%
\pgfpathlineto{\pgfqpoint{4.395112in}{0.696000in}}%
\pgfpathlineto{\pgfqpoint{4.448699in}{0.696000in}}%
\pgfpathlineto{\pgfqpoint{4.502654in}{0.696000in}}%
\pgfpathlineto{\pgfqpoint{4.556517in}{0.696000in}}%
\pgfpathlineto{\pgfqpoint{4.610716in}{0.696000in}}%
\pgfpathlineto{\pgfqpoint{4.665011in}{0.696000in}}%
\pgfpathlineto{\pgfqpoint{4.719273in}{0.696000in}}%
\pgfpathlineto{\pgfqpoint{4.772997in}{0.696000in}}%
\pgfpathlineto{\pgfqpoint{4.827311in}{0.696000in}}%
\pgfpathlineto{\pgfqpoint{4.883565in}{0.696000in}}%
\pgfpathlineto{\pgfqpoint{4.936795in}{0.696000in}}%
\pgfpathlineto{\pgfqpoint{4.990638in}{0.992840in}}%
\pgfpathlineto{\pgfqpoint{5.044624in}{0.844420in}}%
\pgfpathlineto{\pgfqpoint{5.098870in}{0.770210in}}%
\pgfpathlineto{\pgfqpoint{5.153236in}{0.696000in}}%
\pgfpathlineto{\pgfqpoint{5.207127in}{0.696000in}}%
\pgfpathlineto{\pgfqpoint{5.261203in}{0.696000in}}%
\pgfpathlineto{\pgfqpoint{5.315735in}{0.770210in}}%
\pgfpathlineto{\pgfqpoint{5.370037in}{0.770210in}}%
\pgfpathlineto{\pgfqpoint{5.424754in}{0.770210in}}%
\pgfpathlineto{\pgfqpoint{5.481843in}{0.770210in}}%
\pgfpathlineto{\pgfqpoint{5.534545in}{0.770210in}}%
\pgfusepath{stroke}%
\end{pgfscope}%
\begin{pgfscope}%
\pgfpathrectangle{\pgfqpoint{0.800000in}{0.528000in}}{\pgfqpoint{4.960000in}{3.696000in}}%
\pgfusepath{clip}%
\pgfsetrectcap%
\pgfsetroundjoin%
\pgfsetlinewidth{1.505625pt}%
\definecolor{currentstroke}{rgb}{1.000000,0.498039,0.054902}%
\pgfsetstrokecolor{currentstroke}%
\pgfsetdash{}{0pt}%
\pgfpathmoveto{\pgfqpoint{1.025455in}{0.696000in}}%
\pgfpathlineto{\pgfqpoint{1.080463in}{1.745488in}}%
\pgfpathlineto{\pgfqpoint{1.134886in}{2.564560in}}%
\pgfpathlineto{\pgfqpoint{1.188573in}{3.162854in}}%
\pgfpathlineto{\pgfqpoint{1.243222in}{3.531560in}}%
\pgfpathlineto{\pgfqpoint{1.299341in}{3.753055in}}%
\pgfpathlineto{\pgfqpoint{1.351484in}{3.974704in}}%
\pgfpathlineto{\pgfqpoint{1.405458in}{4.056000in}}%
\pgfpathlineto{\pgfqpoint{1.459678in}{3.834828in}}%
\pgfpathlineto{\pgfqpoint{1.513697in}{3.310257in}}%
\pgfpathlineto{\pgfqpoint{1.567828in}{2.794975in}}%
\pgfpathlineto{\pgfqpoint{1.622402in}{2.270232in}}%
\pgfpathlineto{\pgfqpoint{1.676700in}{1.831194in}}%
\pgfpathlineto{\pgfqpoint{1.732519in}{1.470775in}}%
\pgfpathlineto{\pgfqpoint{1.785919in}{1.165345in}}%
\pgfpathlineto{\pgfqpoint{1.839733in}{0.930673in}}%
\pgfpathlineto{\pgfqpoint{1.895406in}{0.770210in}}%
\pgfpathlineto{\pgfqpoint{1.949133in}{0.696000in}}%
\pgfpathlineto{\pgfqpoint{2.003134in}{0.770210in}}%
\pgfpathlineto{\pgfqpoint{2.057367in}{0.696000in}}%
\pgfpathlineto{\pgfqpoint{2.111853in}{0.770210in}}%
\pgfpathlineto{\pgfqpoint{2.166055in}{0.770210in}}%
\pgfpathlineto{\pgfqpoint{2.220433in}{0.770210in}}%
\pgfpathlineto{\pgfqpoint{2.274913in}{0.696000in}}%
\pgfpathlineto{\pgfqpoint{2.329307in}{0.770210in}}%
\pgfpathlineto{\pgfqpoint{2.383517in}{0.770210in}}%
\pgfpathlineto{\pgfqpoint{2.437746in}{0.770210in}}%
\pgfpathlineto{\pgfqpoint{2.493624in}{0.770210in}}%
\pgfpathlineto{\pgfqpoint{2.546508in}{0.696000in}}%
\pgfpathlineto{\pgfqpoint{2.600736in}{0.770210in}}%
\pgfpathlineto{\pgfqpoint{2.654834in}{0.770210in}}%
\pgfpathlineto{\pgfqpoint{2.708919in}{0.696000in}}%
\pgfpathlineto{\pgfqpoint{2.763065in}{0.696000in}}%
\pgfpathlineto{\pgfqpoint{2.818646in}{0.696000in}}%
\pgfpathlineto{\pgfqpoint{2.873454in}{0.770210in}}%
\pgfpathlineto{\pgfqpoint{2.927786in}{0.770210in}}%
\pgfpathlineto{\pgfqpoint{2.982519in}{0.770210in}}%
\pgfpathlineto{\pgfqpoint{3.036262in}{0.770210in}}%
\pgfpathlineto{\pgfqpoint{3.091444in}{0.770210in}}%
\pgfpathlineto{\pgfqpoint{3.144682in}{0.696000in}}%
\pgfpathlineto{\pgfqpoint{3.198899in}{0.770210in}}%
\pgfpathlineto{\pgfqpoint{3.252778in}{0.770210in}}%
\pgfpathlineto{\pgfqpoint{3.307308in}{0.770210in}}%
\pgfpathlineto{\pgfqpoint{3.361350in}{0.770210in}}%
\pgfpathlineto{\pgfqpoint{3.415973in}{0.770210in}}%
\pgfpathlineto{\pgfqpoint{3.470005in}{0.770210in}}%
\pgfpathlineto{\pgfqpoint{3.524641in}{0.696000in}}%
\pgfpathlineto{\pgfqpoint{3.578737in}{0.696000in}}%
\pgfpathlineto{\pgfqpoint{3.633092in}{0.696000in}}%
\pgfpathlineto{\pgfqpoint{3.688418in}{0.696000in}}%
\pgfpathlineto{\pgfqpoint{3.742517in}{0.696000in}}%
\pgfpathlineto{\pgfqpoint{3.796225in}{0.696000in}}%
\pgfpathlineto{\pgfqpoint{3.850584in}{0.696000in}}%
\pgfpathlineto{\pgfqpoint{3.905198in}{0.696000in}}%
\pgfpathlineto{\pgfqpoint{3.959297in}{0.696000in}}%
\pgfpathlineto{\pgfqpoint{4.013722in}{0.696000in}}%
\pgfpathlineto{\pgfqpoint{4.068313in}{0.770210in}}%
\pgfpathlineto{\pgfqpoint{4.122026in}{0.770210in}}%
\pgfpathlineto{\pgfqpoint{4.177259in}{0.696000in}}%
\pgfpathlineto{\pgfqpoint{4.231514in}{0.696000in}}%
\pgfpathlineto{\pgfqpoint{4.285536in}{0.770210in}}%
\pgfpathlineto{\pgfqpoint{4.339524in}{0.770210in}}%
\pgfpathlineto{\pgfqpoint{4.393726in}{0.770210in}}%
\pgfpathlineto{\pgfqpoint{4.447919in}{0.696000in}}%
\pgfpathlineto{\pgfqpoint{4.502149in}{0.696000in}}%
\pgfpathlineto{\pgfqpoint{4.557238in}{0.696000in}}%
\pgfpathlineto{\pgfqpoint{4.611211in}{0.696000in}}%
\pgfpathlineto{\pgfqpoint{4.666759in}{0.770210in}}%
\pgfpathlineto{\pgfqpoint{4.719773in}{0.770210in}}%
\pgfpathlineto{\pgfqpoint{4.774092in}{0.844420in}}%
\pgfpathlineto{\pgfqpoint{4.828199in}{0.844420in}}%
\pgfpathlineto{\pgfqpoint{4.882527in}{0.844420in}}%
\pgfpathlineto{\pgfqpoint{4.936776in}{0.844420in}}%
\pgfpathlineto{\pgfqpoint{4.991258in}{0.770210in}}%
\pgfpathlineto{\pgfqpoint{5.045399in}{0.770210in}}%
\pgfpathlineto{\pgfqpoint{5.100600in}{0.696000in}}%
\pgfpathlineto{\pgfqpoint{5.155003in}{0.696000in}}%
\pgfpathlineto{\pgfqpoint{5.209765in}{0.696000in}}%
\pgfpathlineto{\pgfqpoint{5.264978in}{0.696000in}}%
\pgfpathlineto{\pgfqpoint{5.318254in}{0.696000in}}%
\pgfpathlineto{\pgfqpoint{5.372363in}{0.696000in}}%
\pgfpathlineto{\pgfqpoint{5.426133in}{0.696000in}}%
\pgfpathlineto{\pgfqpoint{5.480334in}{0.696000in}}%
\pgfpathlineto{\pgfqpoint{5.534460in}{0.696000in}}%
\pgfusepath{stroke}%
\end{pgfscope}%
\begin{pgfscope}%
\pgfpathrectangle{\pgfqpoint{0.800000in}{0.528000in}}{\pgfqpoint{4.960000in}{3.696000in}}%
\pgfusepath{clip}%
\pgfsetrectcap%
\pgfsetroundjoin%
\pgfsetlinewidth{1.505625pt}%
\definecolor{currentstroke}{rgb}{0.172549,0.627451,0.172549}%
\pgfsetstrokecolor{currentstroke}%
\pgfsetdash{}{0pt}%
\pgfpathmoveto{\pgfqpoint{1.025455in}{0.696000in}}%
\pgfpathlineto{\pgfqpoint{1.079276in}{1.672080in}}%
\pgfpathlineto{\pgfqpoint{1.134337in}{2.638261in}}%
\pgfpathlineto{\pgfqpoint{1.188053in}{3.162854in}}%
\pgfpathlineto{\pgfqpoint{1.242507in}{3.531560in}}%
\pgfpathlineto{\pgfqpoint{1.296487in}{3.753055in}}%
\pgfpathlineto{\pgfqpoint{1.351007in}{3.900806in}}%
\pgfpathlineto{\pgfqpoint{1.403815in}{3.982254in}}%
\pgfpathlineto{\pgfqpoint{1.457773in}{4.056000in}}%
\pgfpathlineto{\pgfqpoint{1.513226in}{4.056000in}}%
\pgfpathlineto{\pgfqpoint{1.566874in}{3.466727in}}%
\pgfpathlineto{\pgfqpoint{1.620724in}{3.319719in}}%
\pgfpathlineto{\pgfqpoint{1.674856in}{3.099533in}}%
\pgfpathlineto{\pgfqpoint{1.729280in}{2.721538in}}%
\pgfpathlineto{\pgfqpoint{1.783579in}{2.282428in}}%
\pgfpathlineto{\pgfqpoint{1.837576in}{1.831194in}}%
\pgfpathlineto{\pgfqpoint{1.891857in}{1.452796in}}%
\pgfpathlineto{\pgfqpoint{1.946067in}{1.095633in}}%
\pgfpathlineto{\pgfqpoint{2.000385in}{0.800949in}}%
\pgfpathlineto{\pgfqpoint{2.054520in}{0.770210in}}%
\pgfpathlineto{\pgfqpoint{2.110379in}{0.770210in}}%
\pgfpathlineto{\pgfqpoint{2.164011in}{0.770210in}}%
\pgfpathlineto{\pgfqpoint{2.217480in}{0.918630in}}%
\pgfpathlineto{\pgfqpoint{2.271834in}{0.992840in}}%
\pgfpathlineto{\pgfqpoint{2.325843in}{1.147402in}}%
\pgfpathlineto{\pgfqpoint{2.380409in}{1.294300in}}%
\pgfpathlineto{\pgfqpoint{2.434965in}{1.294300in}}%
\pgfpathlineto{\pgfqpoint{2.489042in}{1.368000in}}%
\pgfpathlineto{\pgfqpoint{2.544500in}{1.452796in}}%
\pgfpathlineto{\pgfqpoint{2.600597in}{1.525693in}}%
\pgfpathlineto{\pgfqpoint{2.654488in}{1.525693in}}%
\pgfpathlineto{\pgfqpoint{2.707657in}{1.452796in}}%
\pgfpathlineto{\pgfqpoint{2.762110in}{1.441801in}}%
\pgfpathlineto{\pgfqpoint{2.816152in}{1.368000in}}%
\pgfpathlineto{\pgfqpoint{2.870177in}{1.294300in}}%
\pgfpathlineto{\pgfqpoint{2.924385in}{1.294300in}}%
\pgfpathlineto{\pgfqpoint{2.978863in}{1.368000in}}%
\pgfpathlineto{\pgfqpoint{3.033033in}{1.441801in}}%
\pgfpathlineto{\pgfqpoint{3.087035in}{1.515676in}}%
\pgfpathlineto{\pgfqpoint{3.141243in}{1.515676in}}%
\pgfpathlineto{\pgfqpoint{3.195658in}{1.589607in}}%
\pgfpathlineto{\pgfqpoint{3.250274in}{1.589607in}}%
\pgfpathlineto{\pgfqpoint{3.304241in}{1.589607in}}%
\pgfpathlineto{\pgfqpoint{3.358406in}{1.368000in}}%
\pgfpathlineto{\pgfqpoint{3.412858in}{0.992840in}}%
\pgfpathlineto{\pgfqpoint{3.468680in}{0.918630in}}%
\pgfpathlineto{\pgfqpoint{3.521891in}{0.918630in}}%
\pgfpathlineto{\pgfqpoint{3.575560in}{0.770210in}}%
\pgfpathlineto{\pgfqpoint{3.630153in}{0.696000in}}%
\pgfpathlineto{\pgfqpoint{3.684531in}{0.770210in}}%
\pgfpathlineto{\pgfqpoint{3.738832in}{0.770210in}}%
\pgfpathlineto{\pgfqpoint{3.793789in}{0.770210in}}%
\pgfpathlineto{\pgfqpoint{3.847587in}{0.844420in}}%
\pgfpathlineto{\pgfqpoint{3.901790in}{0.844420in}}%
\pgfpathlineto{\pgfqpoint{3.956116in}{0.844420in}}%
\pgfpathlineto{\pgfqpoint{4.012377in}{0.918630in}}%
\pgfpathlineto{\pgfqpoint{4.065383in}{0.918630in}}%
\pgfpathlineto{\pgfqpoint{4.119094in}{0.844420in}}%
\pgfpathlineto{\pgfqpoint{4.173489in}{0.770210in}}%
\pgfpathlineto{\pgfqpoint{4.228086in}{0.770210in}}%
\pgfpathlineto{\pgfqpoint{4.284288in}{0.770210in}}%
\pgfpathlineto{\pgfqpoint{4.336115in}{0.844420in}}%
\pgfpathlineto{\pgfqpoint{4.390091in}{0.770210in}}%
\pgfpathlineto{\pgfqpoint{4.445591in}{0.696000in}}%
\pgfpathlineto{\pgfqpoint{4.499268in}{0.696000in}}%
\pgfpathlineto{\pgfqpoint{4.554652in}{0.770210in}}%
\pgfpathlineto{\pgfqpoint{4.608705in}{0.844420in}}%
\pgfpathlineto{\pgfqpoint{4.663270in}{0.918630in}}%
\pgfpathlineto{\pgfqpoint{4.716879in}{0.992840in}}%
\pgfpathlineto{\pgfqpoint{4.771042in}{1.067050in}}%
\pgfpathlineto{\pgfqpoint{4.825033in}{0.992840in}}%
\pgfpathlineto{\pgfqpoint{4.879624in}{0.844420in}}%
\pgfpathlineto{\pgfqpoint{4.937747in}{0.844420in}}%
\pgfpathlineto{\pgfqpoint{4.989660in}{0.770210in}}%
\pgfpathlineto{\pgfqpoint{5.043100in}{0.696000in}}%
\pgfpathlineto{\pgfqpoint{5.097222in}{0.696000in}}%
\pgfpathlineto{\pgfqpoint{5.151195in}{0.696000in}}%
\pgfpathlineto{\pgfqpoint{5.205645in}{0.696000in}}%
\pgfpathlineto{\pgfqpoint{5.259429in}{0.696000in}}%
\pgfpathlineto{\pgfqpoint{5.314230in}{0.770210in}}%
\pgfpathlineto{\pgfqpoint{5.368365in}{0.770210in}}%
\pgfpathlineto{\pgfqpoint{5.422685in}{0.770210in}}%
\pgfpathlineto{\pgfqpoint{5.477278in}{0.844420in}}%
\pgfpathlineto{\pgfqpoint{5.531250in}{0.844420in}}%
\pgfusepath{stroke}%
\end{pgfscope}%
\begin{pgfscope}%
\pgfpathrectangle{\pgfqpoint{0.800000in}{0.528000in}}{\pgfqpoint{4.960000in}{3.696000in}}%
\pgfusepath{clip}%
\pgfsetrectcap%
\pgfsetroundjoin%
\pgfsetlinewidth{1.505625pt}%
\definecolor{currentstroke}{rgb}{0.839216,0.152941,0.156863}%
\pgfsetstrokecolor{currentstroke}%
\pgfsetdash{}{0pt}%
\pgfpathmoveto{\pgfqpoint{1.025455in}{0.696000in}}%
\pgfpathlineto{\pgfqpoint{1.079574in}{1.737587in}}%
\pgfpathlineto{\pgfqpoint{1.133752in}{2.638261in}}%
\pgfpathlineto{\pgfqpoint{1.188403in}{3.155028in}}%
\pgfpathlineto{\pgfqpoint{1.242916in}{3.524754in}}%
\pgfpathlineto{\pgfqpoint{1.296283in}{3.753055in}}%
\pgfpathlineto{\pgfqpoint{1.350175in}{3.900806in}}%
\pgfpathlineto{\pgfqpoint{1.405571in}{3.974704in}}%
\pgfpathlineto{\pgfqpoint{1.459450in}{3.982254in}}%
\pgfpathlineto{\pgfqpoint{1.513421in}{3.982254in}}%
\pgfpathlineto{\pgfqpoint{1.567305in}{3.761151in}}%
\pgfpathlineto{\pgfqpoint{1.622216in}{3.015582in}}%
\pgfpathlineto{\pgfqpoint{1.675720in}{2.721538in}}%
\pgfpathlineto{\pgfqpoint{1.730249in}{2.428450in}}%
\pgfpathlineto{\pgfqpoint{1.784621in}{1.904051in}}%
\pgfpathlineto{\pgfqpoint{1.838862in}{1.525693in}}%
\pgfpathlineto{\pgfqpoint{1.893984in}{1.165345in}}%
\pgfpathlineto{\pgfqpoint{1.948088in}{0.770210in}}%
\pgfpathlineto{\pgfqpoint{2.003314in}{0.918630in}}%
\pgfpathlineto{\pgfqpoint{2.057154in}{1.141260in}}%
\pgfpathlineto{\pgfqpoint{2.111057in}{1.438100in}}%
\pgfpathlineto{\pgfqpoint{2.165041in}{1.663580in}}%
\pgfpathlineto{\pgfqpoint{2.219870in}{1.811621in}}%
\pgfpathlineto{\pgfqpoint{2.273776in}{2.040000in}}%
\pgfpathlineto{\pgfqpoint{2.328187in}{2.261461in}}%
\pgfpathlineto{\pgfqpoint{2.382694in}{2.490900in}}%
\pgfpathlineto{\pgfqpoint{2.436719in}{2.721538in}}%
\pgfpathlineto{\pgfqpoint{2.491312in}{2.574847in}}%
\pgfpathlineto{\pgfqpoint{2.545586in}{2.501607in}}%
\pgfpathlineto{\pgfqpoint{2.600987in}{2.428450in}}%
\pgfpathlineto{\pgfqpoint{2.654417in}{2.355386in}}%
\pgfpathlineto{\pgfqpoint{2.708363in}{2.196804in}}%
\pgfpathlineto{\pgfqpoint{2.762531in}{1.904051in}}%
\pgfpathlineto{\pgfqpoint{2.816924in}{1.452796in}}%
\pgfpathlineto{\pgfqpoint{2.871160in}{0.930673in}}%
\pgfpathlineto{\pgfqpoint{2.925809in}{1.067050in}}%
\pgfpathlineto{\pgfqpoint{2.980556in}{1.809150in}}%
\pgfpathlineto{\pgfqpoint{3.034718in}{2.404442in}}%
\pgfpathlineto{\pgfqpoint{3.088847in}{2.483213in}}%
\pgfpathlineto{\pgfqpoint{3.144344in}{2.330305in}}%
\pgfpathlineto{\pgfqpoint{3.199084in}{2.107941in}}%
\pgfpathlineto{\pgfqpoint{3.252586in}{1.811621in}}%
\pgfpathlineto{\pgfqpoint{3.306657in}{1.067050in}}%
\pgfpathlineto{\pgfqpoint{3.360776in}{1.001976in}}%
\pgfpathlineto{\pgfqpoint{3.414926in}{1.380182in}}%
\pgfpathlineto{\pgfqpoint{3.469275in}{1.095633in}}%
\pgfpathlineto{\pgfqpoint{3.523542in}{0.770210in}}%
\pgfpathlineto{\pgfqpoint{3.578135in}{1.001976in}}%
\pgfpathlineto{\pgfqpoint{3.631967in}{1.215470in}}%
\pgfpathlineto{\pgfqpoint{3.686225in}{1.438100in}}%
\pgfpathlineto{\pgfqpoint{3.742116in}{1.660730in}}%
\pgfpathlineto{\pgfqpoint{3.795501in}{1.811621in}}%
\pgfpathlineto{\pgfqpoint{3.849236in}{1.885677in}}%
\pgfpathlineto{\pgfqpoint{3.903432in}{1.512310in}}%
\pgfpathlineto{\pgfqpoint{3.957555in}{1.438100in}}%
\pgfpathlineto{\pgfqpoint{4.011821in}{1.441801in}}%
\pgfpathlineto{\pgfqpoint{4.066215in}{1.220744in}}%
\pgfpathlineto{\pgfqpoint{4.120523in}{0.930673in}}%
\pgfpathlineto{\pgfqpoint{4.174850in}{0.696000in}}%
\pgfpathlineto{\pgfqpoint{4.230312in}{0.770210in}}%
\pgfpathlineto{\pgfqpoint{4.283605in}{0.770210in}}%
\pgfpathlineto{\pgfqpoint{4.337688in}{0.770210in}}%
\pgfpathlineto{\pgfqpoint{4.392175in}{0.844420in}}%
\pgfpathlineto{\pgfqpoint{4.446261in}{0.992840in}}%
\pgfpathlineto{\pgfqpoint{4.500733in}{1.067050in}}%
\pgfpathlineto{\pgfqpoint{4.555215in}{1.215470in}}%
\pgfpathlineto{\pgfqpoint{4.609863in}{1.438100in}}%
\pgfpathlineto{\pgfqpoint{4.664009in}{1.515676in}}%
\pgfpathlineto{\pgfqpoint{4.718818in}{1.363890in}}%
\pgfpathlineto{\pgfqpoint{4.772848in}{1.220744in}}%
\pgfpathlineto{\pgfqpoint{4.828500in}{1.074398in}}%
\pgfpathlineto{\pgfqpoint{4.882571in}{1.001976in}}%
\pgfpathlineto{\pgfqpoint{4.936339in}{1.001976in}}%
\pgfpathlineto{\pgfqpoint{4.990646in}{0.844420in}}%
\pgfpathlineto{\pgfqpoint{5.045021in}{0.696000in}}%
\pgfpathlineto{\pgfqpoint{5.099433in}{0.696000in}}%
\pgfpathlineto{\pgfqpoint{5.153905in}{0.696000in}}%
\pgfpathlineto{\pgfqpoint{5.207925in}{0.696000in}}%
\pgfpathlineto{\pgfqpoint{5.262413in}{0.770210in}}%
\pgfpathlineto{\pgfqpoint{5.316485in}{0.770210in}}%
\pgfpathlineto{\pgfqpoint{5.370794in}{0.770210in}}%
\pgfpathlineto{\pgfqpoint{5.426554in}{0.844420in}}%
\pgfpathlineto{\pgfqpoint{5.480194in}{0.844420in}}%
\pgfpathlineto{\pgfqpoint{5.533860in}{0.918630in}}%
\pgfusepath{stroke}%
\end{pgfscope}%
\begin{pgfscope}%
\pgfpathrectangle{\pgfqpoint{0.800000in}{0.528000in}}{\pgfqpoint{4.960000in}{3.696000in}}%
\pgfusepath{clip}%
\pgfsetrectcap%
\pgfsetroundjoin%
\pgfsetlinewidth{1.505625pt}%
\definecolor{currentstroke}{rgb}{0.580392,0.403922,0.741176}%
\pgfsetstrokecolor{currentstroke}%
\pgfsetdash{}{0pt}%
\pgfpathmoveto{\pgfqpoint{1.025455in}{0.696000in}}%
\pgfpathlineto{\pgfqpoint{1.079196in}{1.737587in}}%
\pgfpathlineto{\pgfqpoint{1.133647in}{2.638261in}}%
\pgfpathlineto{\pgfqpoint{1.187976in}{3.155028in}}%
\pgfpathlineto{\pgfqpoint{1.242551in}{3.524754in}}%
\pgfpathlineto{\pgfqpoint{1.296968in}{3.746744in}}%
\pgfpathlineto{\pgfqpoint{1.351086in}{3.900806in}}%
\pgfpathlineto{\pgfqpoint{1.404085in}{4.048616in}}%
\pgfpathlineto{\pgfqpoint{1.458169in}{4.056000in}}%
\pgfpathlineto{\pgfqpoint{1.512631in}{3.982254in}}%
\pgfpathlineto{\pgfqpoint{1.567104in}{3.771017in}}%
\pgfpathlineto{\pgfqpoint{1.622903in}{3.393204in}}%
\pgfpathlineto{\pgfqpoint{1.676449in}{2.879842in}}%
\pgfpathlineto{\pgfqpoint{1.730170in}{2.501607in}}%
\pgfpathlineto{\pgfqpoint{1.784346in}{1.904051in}}%
\pgfpathlineto{\pgfqpoint{1.838594in}{1.307951in}}%
\pgfpathlineto{\pgfqpoint{1.892722in}{0.861939in}}%
\pgfpathlineto{\pgfqpoint{1.946804in}{1.074398in}}%
\pgfpathlineto{\pgfqpoint{2.001161in}{1.512310in}}%
\pgfpathlineto{\pgfqpoint{2.055203in}{1.734940in}}%
\pgfpathlineto{\pgfqpoint{2.109614in}{2.107941in}}%
\pgfpathlineto{\pgfqpoint{2.163964in}{2.409271in}}%
\pgfpathlineto{\pgfqpoint{2.218178in}{2.712000in}}%
\pgfpathlineto{\pgfqpoint{2.273666in}{2.712000in}}%
\pgfpathlineto{\pgfqpoint{2.327048in}{2.638261in}}%
\pgfpathlineto{\pgfqpoint{2.381133in}{2.490900in}}%
\pgfpathlineto{\pgfqpoint{2.435138in}{2.050205in}}%
\pgfpathlineto{\pgfqpoint{2.489333in}{2.209593in}}%
\pgfpathlineto{\pgfqpoint{2.543660in}{2.136897in}}%
\pgfpathlineto{\pgfqpoint{2.598384in}{1.919902in}}%
\pgfpathlineto{\pgfqpoint{2.652710in}{1.758525in}}%
\pgfpathlineto{\pgfqpoint{2.706972in}{1.525693in}}%
\pgfpathlineto{\pgfqpoint{2.761165in}{1.001976in}}%
\pgfpathlineto{\pgfqpoint{2.815619in}{0.800949in}}%
\pgfpathlineto{\pgfqpoint{2.870097in}{0.696000in}}%
\pgfpathlineto{\pgfqpoint{2.925260in}{1.067050in}}%
\pgfpathlineto{\pgfqpoint{2.978807in}{1.438100in}}%
\pgfpathlineto{\pgfqpoint{3.035832in}{1.737587in}}%
\pgfpathlineto{\pgfqpoint{3.087526in}{1.959751in}}%
\pgfpathlineto{\pgfqpoint{3.142088in}{2.187602in}}%
\pgfpathlineto{\pgfqpoint{3.196771in}{2.409271in}}%
\pgfpathlineto{\pgfqpoint{3.250659in}{2.490900in}}%
\pgfpathlineto{\pgfqpoint{3.305076in}{2.270232in}}%
\pgfpathlineto{\pgfqpoint{3.359345in}{2.050205in}}%
\pgfpathlineto{\pgfqpoint{3.414317in}{1.904051in}}%
\pgfpathlineto{\pgfqpoint{3.468160in}{1.598803in}}%
\pgfpathlineto{\pgfqpoint{3.523339in}{1.236257in}}%
\pgfpathlineto{\pgfqpoint{3.577269in}{1.001976in}}%
\pgfpathlineto{\pgfqpoint{3.631162in}{0.861939in}}%
\pgfpathlineto{\pgfqpoint{3.685660in}{0.770210in}}%
\pgfpathlineto{\pgfqpoint{3.739808in}{0.918630in}}%
\pgfpathlineto{\pgfqpoint{3.793901in}{1.141260in}}%
\pgfpathlineto{\pgfqpoint{3.849309in}{1.289680in}}%
\pgfpathlineto{\pgfqpoint{3.903570in}{1.363890in}}%
\pgfpathlineto{\pgfqpoint{3.957903in}{1.512310in}}%
\pgfpathlineto{\pgfqpoint{4.012789in}{1.438100in}}%
\pgfpathlineto{\pgfqpoint{4.066735in}{1.441801in}}%
\pgfpathlineto{\pgfqpoint{4.120902in}{1.589607in}}%
\pgfpathlineto{\pgfqpoint{4.176220in}{1.672080in}}%
\pgfpathlineto{\pgfqpoint{4.229915in}{1.672080in}}%
\pgfpathlineto{\pgfqpoint{4.283837in}{1.819001in}}%
\pgfpathlineto{\pgfqpoint{4.337771in}{1.819001in}}%
\pgfpathlineto{\pgfqpoint{4.392424in}{1.368000in}}%
\pgfpathlineto{\pgfqpoint{4.446706in}{1.001976in}}%
\pgfpathlineto{\pgfqpoint{4.500988in}{0.800949in}}%
\pgfpathlineto{\pgfqpoint{4.555181in}{0.770210in}}%
\pgfpathlineto{\pgfqpoint{4.611088in}{1.067050in}}%
\pgfpathlineto{\pgfqpoint{4.664110in}{1.141260in}}%
\pgfpathlineto{\pgfqpoint{4.718185in}{1.289680in}}%
\pgfpathlineto{\pgfqpoint{4.772520in}{1.363890in}}%
\pgfpathlineto{\pgfqpoint{4.826399in}{1.515676in}}%
\pgfpathlineto{\pgfqpoint{4.880521in}{1.663580in}}%
\pgfpathlineto{\pgfqpoint{4.935345in}{1.737587in}}%
\pgfpathlineto{\pgfqpoint{4.989196in}{1.441801in}}%
\pgfpathlineto{\pgfqpoint{5.043972in}{1.368000in}}%
\pgfpathlineto{\pgfqpoint{5.098176in}{1.598803in}}%
\pgfpathlineto{\pgfqpoint{5.152554in}{1.452796in}}%
\pgfpathlineto{\pgfqpoint{5.206726in}{0.930673in}}%
\pgfpathlineto{\pgfqpoint{5.262396in}{0.696000in}}%
\pgfpathlineto{\pgfqpoint{5.316235in}{0.800949in}}%
\pgfpathlineto{\pgfqpoint{5.370220in}{0.770210in}}%
\pgfpathlineto{\pgfqpoint{5.424245in}{0.696000in}}%
\pgfpathlineto{\pgfqpoint{5.478748in}{0.844420in}}%
\pgfpathlineto{\pgfqpoint{5.532896in}{0.992840in}}%
\pgfusepath{stroke}%
\end{pgfscope}%
\begin{pgfscope}%
\pgfsetrectcap%
\pgfsetmiterjoin%
\pgfsetlinewidth{0.803000pt}%
\definecolor{currentstroke}{rgb}{0.000000,0.000000,0.000000}%
\pgfsetstrokecolor{currentstroke}%
\pgfsetdash{}{0pt}%
\pgfpathmoveto{\pgfqpoint{0.800000in}{0.528000in}}%
\pgfpathlineto{\pgfqpoint{0.800000in}{4.224000in}}%
\pgfusepath{stroke}%
\end{pgfscope}%
\begin{pgfscope}%
\pgfsetrectcap%
\pgfsetmiterjoin%
\pgfsetlinewidth{0.803000pt}%
\definecolor{currentstroke}{rgb}{0.000000,0.000000,0.000000}%
\pgfsetstrokecolor{currentstroke}%
\pgfsetdash{}{0pt}%
\pgfpathmoveto{\pgfqpoint{5.760000in}{0.528000in}}%
\pgfpathlineto{\pgfqpoint{5.760000in}{4.224000in}}%
\pgfusepath{stroke}%
\end{pgfscope}%
\begin{pgfscope}%
\pgfsetrectcap%
\pgfsetmiterjoin%
\pgfsetlinewidth{0.803000pt}%
\definecolor{currentstroke}{rgb}{0.000000,0.000000,0.000000}%
\pgfsetstrokecolor{currentstroke}%
\pgfsetdash{}{0pt}%
\pgfpathmoveto{\pgfqpoint{0.800000in}{0.528000in}}%
\pgfpathlineto{\pgfqpoint{5.760000in}{0.528000in}}%
\pgfusepath{stroke}%
\end{pgfscope}%
\begin{pgfscope}%
\pgfsetrectcap%
\pgfsetmiterjoin%
\pgfsetlinewidth{0.803000pt}%
\definecolor{currentstroke}{rgb}{0.000000,0.000000,0.000000}%
\pgfsetstrokecolor{currentstroke}%
\pgfsetdash{}{0pt}%
\pgfpathmoveto{\pgfqpoint{0.800000in}{4.224000in}}%
\pgfpathlineto{\pgfqpoint{5.760000in}{4.224000in}}%
\pgfusepath{stroke}%
\end{pgfscope}%
\begin{pgfscope}%
\definecolor{textcolor}{rgb}{0.000000,0.000000,0.000000}%
\pgfsetstrokecolor{textcolor}%
\pgfsetfillcolor{textcolor}%
\pgftext[x=3.280000in,y=4.307333in,,base]{\color{textcolor}\sffamily\fontsize{12.000000}{14.400000}\selectfont Measured ground speed}%
\end{pgfscope}%
\begin{pgfscope}%
\pgfsetbuttcap%
\pgfsetmiterjoin%
\definecolor{currentfill}{rgb}{1.000000,1.000000,1.000000}%
\pgfsetfillcolor{currentfill}%
\pgfsetfillopacity{0.800000}%
\pgfsetlinewidth{1.003750pt}%
\definecolor{currentstroke}{rgb}{0.800000,0.800000,0.800000}%
\pgfsetstrokecolor{currentstroke}%
\pgfsetstrokeopacity{0.800000}%
\pgfsetdash{}{0pt}%
\pgfpathmoveto{\pgfqpoint{4.997454in}{3.093603in}}%
\pgfpathlineto{\pgfqpoint{5.662778in}{3.093603in}}%
\pgfpathquadraticcurveto{\pgfqpoint{5.690556in}{3.093603in}}{\pgfqpoint{5.690556in}{3.121381in}}%
\pgfpathlineto{\pgfqpoint{5.690556in}{4.126778in}}%
\pgfpathquadraticcurveto{\pgfqpoint{5.690556in}{4.154556in}}{\pgfqpoint{5.662778in}{4.154556in}}%
\pgfpathlineto{\pgfqpoint{4.997454in}{4.154556in}}%
\pgfpathquadraticcurveto{\pgfqpoint{4.969676in}{4.154556in}}{\pgfqpoint{4.969676in}{4.126778in}}%
\pgfpathlineto{\pgfqpoint{4.969676in}{3.121381in}}%
\pgfpathquadraticcurveto{\pgfqpoint{4.969676in}{3.093603in}}{\pgfqpoint{4.997454in}{3.093603in}}%
\pgfpathlineto{\pgfqpoint{4.997454in}{3.093603in}}%
\pgfpathclose%
\pgfusepath{stroke,fill}%
\end{pgfscope}%
\begin{pgfscope}%
\pgfsetrectcap%
\pgfsetroundjoin%
\pgfsetlinewidth{1.505625pt}%
\definecolor{currentstroke}{rgb}{0.121569,0.466667,0.705882}%
\pgfsetstrokecolor{currentstroke}%
\pgfsetdash{}{0pt}%
\pgfpathmoveto{\pgfqpoint{5.025232in}{4.042088in}}%
\pgfpathlineto{\pgfqpoint{5.164121in}{4.042088in}}%
\pgfpathlineto{\pgfqpoint{5.303009in}{4.042088in}}%
\pgfusepath{stroke}%
\end{pgfscope}%
\begin{pgfscope}%
\definecolor{textcolor}{rgb}{0.000000,0.000000,0.000000}%
\pgfsetstrokecolor{textcolor}%
\pgfsetfillcolor{textcolor}%
\pgftext[x=5.414121in,y=3.993477in,left,base]{\color{textcolor}\sffamily\fontsize{10.000000}{12.000000}\selectfont 0}%
\end{pgfscope}%
\begin{pgfscope}%
\pgfsetrectcap%
\pgfsetroundjoin%
\pgfsetlinewidth{1.505625pt}%
\definecolor{currentstroke}{rgb}{1.000000,0.498039,0.054902}%
\pgfsetstrokecolor{currentstroke}%
\pgfsetdash{}{0pt}%
\pgfpathmoveto{\pgfqpoint{5.025232in}{3.838231in}}%
\pgfpathlineto{\pgfqpoint{5.164121in}{3.838231in}}%
\pgfpathlineto{\pgfqpoint{5.303009in}{3.838231in}}%
\pgfusepath{stroke}%
\end{pgfscope}%
\begin{pgfscope}%
\definecolor{textcolor}{rgb}{0.000000,0.000000,0.000000}%
\pgfsetstrokecolor{textcolor}%
\pgfsetfillcolor{textcolor}%
\pgftext[x=5.414121in,y=3.789620in,left,base]{\color{textcolor}\sffamily\fontsize{10.000000}{12.000000}\selectfont 0.5}%
\end{pgfscope}%
\begin{pgfscope}%
\pgfsetrectcap%
\pgfsetroundjoin%
\pgfsetlinewidth{1.505625pt}%
\definecolor{currentstroke}{rgb}{0.172549,0.627451,0.172549}%
\pgfsetstrokecolor{currentstroke}%
\pgfsetdash{}{0pt}%
\pgfpathmoveto{\pgfqpoint{5.025232in}{3.634374in}}%
\pgfpathlineto{\pgfqpoint{5.164121in}{3.634374in}}%
\pgfpathlineto{\pgfqpoint{5.303009in}{3.634374in}}%
\pgfusepath{stroke}%
\end{pgfscope}%
\begin{pgfscope}%
\definecolor{textcolor}{rgb}{0.000000,0.000000,0.000000}%
\pgfsetstrokecolor{textcolor}%
\pgfsetfillcolor{textcolor}%
\pgftext[x=5.414121in,y=3.585762in,left,base]{\color{textcolor}\sffamily\fontsize{10.000000}{12.000000}\selectfont 1}%
\end{pgfscope}%
\begin{pgfscope}%
\pgfsetrectcap%
\pgfsetroundjoin%
\pgfsetlinewidth{1.505625pt}%
\definecolor{currentstroke}{rgb}{0.839216,0.152941,0.156863}%
\pgfsetstrokecolor{currentstroke}%
\pgfsetdash{}{0pt}%
\pgfpathmoveto{\pgfqpoint{5.025232in}{3.430516in}}%
\pgfpathlineto{\pgfqpoint{5.164121in}{3.430516in}}%
\pgfpathlineto{\pgfqpoint{5.303009in}{3.430516in}}%
\pgfusepath{stroke}%
\end{pgfscope}%
\begin{pgfscope}%
\definecolor{textcolor}{rgb}{0.000000,0.000000,0.000000}%
\pgfsetstrokecolor{textcolor}%
\pgfsetfillcolor{textcolor}%
\pgftext[x=5.414121in,y=3.381905in,left,base]{\color{textcolor}\sffamily\fontsize{10.000000}{12.000000}\selectfont 2}%
\end{pgfscope}%
\begin{pgfscope}%
\pgfsetrectcap%
\pgfsetroundjoin%
\pgfsetlinewidth{1.505625pt}%
\definecolor{currentstroke}{rgb}{0.580392,0.403922,0.741176}%
\pgfsetstrokecolor{currentstroke}%
\pgfsetdash{}{0pt}%
\pgfpathmoveto{\pgfqpoint{5.025232in}{3.226659in}}%
\pgfpathlineto{\pgfqpoint{5.164121in}{3.226659in}}%
\pgfpathlineto{\pgfqpoint{5.303009in}{3.226659in}}%
\pgfusepath{stroke}%
\end{pgfscope}%
\begin{pgfscope}%
\definecolor{textcolor}{rgb}{0.000000,0.000000,0.000000}%
\pgfsetstrokecolor{textcolor}%
\pgfsetfillcolor{textcolor}%
\pgftext[x=5.414121in,y=3.178048in,left,base]{\color{textcolor}\sffamily\fontsize{10.000000}{12.000000}\selectfont 3}%
\end{pgfscope}%
\end{pgfpicture}%
\makeatother%
\endgroup%
}
    \end{minipage}
    \caption{Variation of (a) measured forward position and (b) measured absolute ground velocity for different values of $K_{D}$ and $K_P=4$, $K_I=1$ while the forward controller is engaged.}
    \label{fig:tune-fwd-der-measures}
\end{figure}