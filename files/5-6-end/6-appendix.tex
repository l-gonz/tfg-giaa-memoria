\chapter{Installation manuals}

This appendix 
\url{https://github.com/l-gonz/tfg-giaa-dronecontrol}

\section{SITL: Development environment}
\label{app:install-dev-env}

This section describes the process of installing all the necessary applications to set up a development environment to run PX4's SITL simulation in a system similar to the one described in section \ref{sec:devenv}.
The instructions assume a computer running Windows 10/11 as the operating system and Windows Subsystem for Linux installed.
PX4 details the installation steps of their source code for several platforms in their documentation \cite{install-px4}, where Ubuntu is the recommended platform.
To make it easier, the DroneVisionControl repository contains a small shell script that aggregates all the steps and installs all the dependencies with the folder structure that the project expects.
This includes installing and setting up PX4 and QGroundControl and creating a virtual environment for the project, installing the Python packages and the DroneVisionControl application.

To run the script, simply clone the repository, navigate to the project folder and execute:
\begin{minted}[breaklines, fontsize=\footnotesize, baselinestretch=1]{bash}
./install.sh
\end{minted}

To test the installation of PX4, execute the following line (requires a graphic interface for WSL):
\begin{minted}[breaklines, fontsize=\footnotesize, baselinestretch=1]{bash}
./simulator.sh --gazebo
\end{minted}

To test the installation of DroneVisionControl, execute:
\begin{minted}[breaklines, fontsize=\footnotesize, baselinestretch=1]{bash}
dronevisioncontrol tools test-camera -s -c
\end{minted}
Camera input is not supported from within WSL, but it should be possible to control the simulated drone with the keyboard.

The next step is to install DroneVisionControl in the Windows machine to be able to access an integrated or USB camera.
It requires having Python already installed.
First, clone the project repository, navigate to the folder and set up the virtual environment:
\begin{minted}[breaklines, fontsize=\footnotesize, baselinestretch=1]{bash}
pip install virtualenv
virtualenv venv
venv\Scripts\activate
\end{minted}

Then install DroneVisionControl:
\begin{minted}[breaklines, fontsize=\footnotesize, baselinestretch=1]{bash}
pip install -r requirements.txt
pip install -e .
\end{minted}

Additionally, to make the simulated PX4 application broadcast to the Windows machine on port 14550, it is necessary to edit the file \texttt{px4-rc.mavlink} located in \texttt{Firmware/PX4-Autopilot/etc/init.d-posix} on the project folder in WSL.
To the mavlink start command for the ground control link on line 14, append -p to enable broadcasting:
\begin{minted}[breaklines, fontsize=\footnotesize, baselinestretch=1]{bash}
mavlink start -x -u $udp_gcs_port_local -r 4000000 -f
\end{minted}

To test the installation, with the PX4 simulator still running in WSL, execute:
\begin{minted}[breaklines, fontsize=\footnotesize, baselinestretch=1]{bash}
dronevisioncontrol tools test-camera -s -c
\end{minted}
It should now be possible to both control the drone with the keyboard and obtain images from a camera attached to the computer, as well as run the hand control solution in the simulator.

\subsection{Installation of AirSim}

To run the PX4 software-in-the-loop simulator in AirSim and test the follow solution,
first, install Unreal Engine at version 4.27 at least from the Epic Games Launcher\footnote{\url{https://www.unrealengine.com/download}} and open the environment found in the \texttt{data} folder of the repository.
To use AirSim with a different Unreal environment, follow the guide in the AirSim documentation\footnote{\url{https://microsoft.github.io/AirSim/unreal_custenv/}}
After starting play mode in Unreal for the first time, a \texttt{settings.json} file will appear in an AirSim folder in the user's Documents folder.
The contents of this file need to be replaced with the configuration file found in section \ref{app:airsim-config} to be able to interact with PX4 running inside WSL, selecting the correct value for \texttt{UseSerial} and exchanging the \texttt{LocalHostIp} in the file for the IP of the Windows machine in the virtual WSL network.
This IP can be obtained by typing \texttt{ipconfig} in the Windows command prompt and looking for the IPv4 address under "Ethernet Adapter vEthernet (WSL)".

After the settings have been set, restart play mode in Unreal; the output log should show a message saying "Waiting for TCP connection on port 4560, local IP <Windows-IP>".
It is now possible to build and start PX4 by executing in WSL:
\begin{minted}[breaklines, fontsize=\footnotesize, baselinestretch=1]{bash}
./simulator.sh --airsim
\end{minted}

If the IP has been set correctly in the AirSim settings, 
PX4 and the simulator will find each other correctly and connect.
The test-camera tool can be run either from Linux within WSL or from Windows by executing:
\begin{minted}[breaklines, fontsize=\footnotesize, baselinestretch=1]{bash}
dronevisioncontrol tools test-camera -s -w
\end{minted}

The follow control solution can be run from Linux with:
\begin{minted}[breaklines, fontsize=\footnotesize, baselinestretch=1]{bash}
dronevisioncontrol follow --sim 172.19.112.1
\end{minted}
and from Windows with:
\begin{minted}[breaklines, fontsize=\footnotesize, baselinestretch=1]{bash}
dronevisioncontrol follow --sim -p 14550
\end{minted}

\section{HITL: Installation on a Raspberry Pi 4}
\label{app:install-dronecontrol-rpi}

\todo[inline]{Write this part}

Install DroneVisionControl
        
Set up XRDP to connect from Windows PC
\url{https://linuxize.com/post/how-to-install-xrdp-on-raspberry-pi/}

Set up UART serial connection in RPi: 
\url{https://discuss.px4.io/t/talking-to-a-px4-fmu-with-a-rpi-via-serial-nousb/14119?page=2}


\section{AirSim configuration file}
\label{app:airsim-config}

Listing \ref{lst:airsim-settings} provides the configuration file used for AirSim. The file includes various settings and parameters tailored to the specific requirements of the project.

The \texttt{PX4} section contains the configuration settings for the PX4-type flight controller. It tells the simulator to connect to the specific external flight stack. The \texttt{LockStep} parameter is set to true to indicate synchronized step execution between the simulator and the flight controller and the rest of the connection settings replicate the network configuration described in Figure \ref{fig:px4-ports}.

The \texttt{Parameters} section includes specific PX4 parameters relevant only for simulation, such as disabling safety checks for RC control and the start world location of the vehicle.
These parameters can be customized based on the desired behaviour and performance of the flight controller.

The only settings that need to be edited depending on the individual machine and current simulation mode are the \texttt{LocalHostIp} and \texttt{UseSerial}.
The \texttt{LocalHostIp} should be replaced with the IP of the Windows host for the WSL virtual network interface to allow communication with the Linux subsystem. This is only necessary for SITL mode.
The \texttt{UseSerial} parameter determines the communication channel with the flight controller and is set to either \texttt{true} for HITL simulation mode through USB or \texttt{false} for SITL simulation mode to use UDP instead.

The \texttt{CameraDefaults} section defines the settings for the camera images that can be retrieved through the \texttt{airlib} library from external code. 

\begin{listing}[h]
    \caption{AirSim's \texttt{settings.json} file, located in the computer's Documents folder, with settings required for configuring this project.}
    \label{lst:airsim-settings}
    \begin{minted}[breaklines, fontsize=\footnotesize, baselinestretch=1]{json}
{
    "SettingsVersion": 1.2,
    "SimMode": "Multirotor",
    "ClockType": "SteppableClock",
    "Vehicles": {
        "PX4": {
            "VehicleType": "PX4Multirotor",
            "LockStep": true,
            "UseSerial": "<SITL: false, HITL: true>"
            "UseTcp": true,
            "TcpPort": 4560,
            "ControlIp": "remote",
            "ControlPortLocal": 14550,
            "ControlPortRemote": 18570,
            "LocalHostIp": "<Windows-IP>",
            "Sensors":{
                "Barometer":{
                    "SensorType": 1,
                    "Enabled": true
                }
            },
            "Parameters": {
                "NAV_RCL_ACT": 1,
                "NAV_DLL_ACT": 0,
                "COM_RCL_EXCEPT": 7,
                "LPE_LAT": 47.641468,
                "LPE_LON": -122.140165
            }
        }
    },
    "CameraDefaults": {
        "CaptureSettings": [
            {
                "ImageType": 0,
                "Width": 640,
                "Height": 400
            }
        ]
    }
}
    \end{minted}
\end{listing}

\chapter{Command-line interface of the application}
\label{app:cli}

\begin{minted}[breaklines, fontsize=\footnotesize, baselinestretch=1]{text}
Usage: dronevisioncontrol tools test-camera [OPTIONS]

Options:
  -s, --sim TEXT        attach to a simulator through UDP, optionally provide
                        the IP the simulator listens at
  -r, --hardware TEXT   attach to a hardware drone through serial, optionally
                        provide the address of the device that connects to PX4
  -w, --wsl             expects the program to run on a Linux WSL OS
  -c, --camera          use a physical camera as source
  -h, --hand-detection  use hand detection for image processing
  -p, --pose-detection  use pose detection for image processing
  -f, --file TEXT       file name to use as video source
  --help                Show this message and exit.


Usage: dronevisioncontrol tools tune [OPTIONS]

Options:
  --yaw / --forward     test the controller yaw or forward movement
  --manual              manual tuning
  -t, --time INTEGER    sample time for each of the values to test
  -p, --kp-values TEXT  values to test for Kp parameter
  -i, --ki-values TEXT  values to test for Ki parameter
  -d, --kd-values TEXT  values to test for Kd parameter
  -h, --help            Show this message and exit.


Usage: dronevisioncontrol tools test-controller [OPTIONS]

Options:
  --yaw / --forward  test the controller yaw or forward movement
  -f, --file TEXT    file name to use as data source
  -h, --help         Show this message and exit.


Usage: dronevisioncontrol hand [OPTIONS]

Options:
  -i, --ip TEXT       pilot IP address, ignored if serial is provided
  -p, --port INTEGER  port for UDP connections
  -s, --serial TEXT   connect to drone system through serial, default device
                      is /dev/ttyUSB0
  -f, --file PATH     file to use as source instead of the camera
  -l, --log           log important info and save video
  -h, --help          Show this message and exit.


Usage: dronevisioncontrol follow [OPTIONS]

Options:
  --ip TEXT          pilot IP address, ignored if serial is provided
  -p, --port TEXT    pilot UDP port, ignored if serial is provided, default is
                     14540
  --sim TEXT         run with AirSim as flight engine, optionally provide ip
                     the sim listens to
  -l, --log          log important info and save video
  -s, --serial TEXT  use serial to connect to PX4 (HITL), optionally provide
                     the address of the serial port
  -h, --help         Show this message and exit.
\end{minted}

% \chapter{PX4 parameters}
%%% Maybe %%%
% All parameters changed from their default values
% Connect Pixhawk to QGC, then check config for changed values
