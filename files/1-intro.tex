\cleardoublepage
\chapter{Introduction}
\label{sec:intro}
\pagenumbering{arabic} % para empezar la numeración de página con números



%%-- Objetivos del  proyecto
%%-- Si la sección anterior ha quedado muy extensa, se puede considerar convertir
%%-- Las siguientes tres secciones en un capítulo independiente de la memoria

\section{Goal of the project}
\label{sec:objetivos}

\subsection{Objetivo general} % título de subsección (se muestra)
\label{sec:objetivo-general} % identificador de subsección (no se muestra, es para poder referenciarla)


Aquí vendría el objetivo general en una frase:
Mi Trabajo Fin de Grado/Master consiste en crear de una herramienta de análisis de los comentarios jocosos en repositorios de software libre alojados en la plataforma GitHub.

Recuerda que los objetivos siempre vienen en infinitivo.


\subsection{Objetivos específicos}
\label{sec:objetivos-especificos}

Los objetivos específicos se pueden entender como las tareas en las que se ha desglosado el objetivo general. Y, sí, también vienen en infinitivo.

Lo mejor suele ser utilizar una lista no numerada, como sigue:

    \begin{itemize}
        \item Un objetivo específico.
        \item Otro objetivo específico.
        \item Tercer objetivo específico.
        \item \ldots
    \end{itemize}

\section{Planificación temporal}
\label{sec:planificacion-temporal}

Es conveniente que incluyas una descripción de lo que te ha llevado realizar el trabajo.
Hay gente que añade un diagrama de GANTT.
Lo importante es que quede claro cuánto tiempo has consumido en realizar el TFG/TFM 
(tiempo natural, p.ej., 6 meses) y a qué nivel de esfuerzo (p.ej., principalmente los 
fines de semana).

\section{Estructura de la memoria}
\label{sec:estructura}

Por último, en esta sección se introduce a alto nivel la organización del resto del documento
y qué contenidos se van a encontrar en cada capítulo.

    \begin{itemize}
      \item En el primer capítulo se hace una breve introducción al proyecto, se describen los objetivos del mismo y se refleja la planificación temporal.
      \item En el siguiente capítulo se describen las tecnologías utilizadas en el desarrollo de este TFM/TFG (Capítulo~\ref{chap:tecnologias}).
      \item En el capítulo~\ref{chap:desing} Se describe el proceso de desarrollo
      de la herramienta \ldots
      \item En el capítulo~\ref{chap:experimentos} Se presentan las principales pruebas realizadas
      para validación de la plataforma/herramienta\ldots (o resultados de los experimentos
      efectuados).
      \item Por último, se presentan las conclusiones del proyecto así como los trabajos futuros que podrían derivarse de éste (Capítulo~\ref{chap:conclusiones}).
    \end{itemize}

\cleardoublepage