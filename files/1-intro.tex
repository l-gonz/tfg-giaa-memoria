\cleardoublepage
\chapter{Introduction}
\label{sec:intro}
\pagenumbering{arabic} % para empezar la numeración de página con números

\section{The Dronecontrol project}



%%-- Objetivos del  proyecto
%%-- Si la sección anterior ha quedado muy extensa, se puede considerar convertir
%%-- Las siguientes tres secciones en un capítulo independiente de la memoria

\section{Objectives}
\label{sec:objetives}
\todo[inline]{Write: objectives}

%%%%%%%%%%%%%%%%%%%%%%%%%%%%%
% •	Main Objectives:  
% o	detect + track + follow with vision control techiniques
% o	explore software and hardware possibilities for vision control
% o	machine learning techniques for vision control
% o	SIL/HIL validation procedure
%%%%%%%%%%%%%%%%%%%%%%%%%%%%%


% \subsection{General goal} % título de subsección (se muestra)
% \label{sec:objetivo-general} % identificador de subsección (no se muestra, es para poder referenciarla)
% Aquí vendría el objetivo general en una frase:
% Mi Trabajo Fin de Grado/Master consiste en crear de una herramienta de análisis de los comentarios jocosos en repositorios de software libre alojados en la plataforma GitHub.
% Recuerda que los objetivos siempre vienen en infinitivo.
% \subsection{Specific goals}
% \label{sec:objetivos-especificos}
% Los objetivos específicos se pueden entender como las tareas en las que se ha desglosado el objetivo general. Y, sí, también vienen en infinitivo.
% Lo mejor suele ser utilizar una lista no numerada, como sigue:
%     \begin{itemize}
%         \item Un objetivo específico.
%         \item Otro objetivo específico.
%         \item Tercer objetivo específico.
%         \item \ldots
%     \end{itemize}

\section{Time planning}
\label{sec:time-planning}
\todo[inline]{Polish: time planning}

% Es conveniente que incluyas una descripción de lo que te ha llevado realizar el trabajo.
% Hay gente que añade un diagrama de GANTT.
% Lo importante es que quede claro cuánto tiempo has consumido en realizar el TFG/TFM 
% (tiempo natural, p.ej., 6 meses) y a qué nivel de esfuerzo (p.ej., principalmente los 
% fines de semana).

\section{Thesis layout}
\label{sec:layout}
\todo[inline]{Write: layout}

% Por último, en esta sección se introduce a alto nivel la organización del resto del documento
% y qué contenidos se van a encontrar en cada capítulo.
%     \begin{itemize}
%       \item En el primer capítulo se hace una breve introducción al proyecto, se describen los objetivos del mismo y se refleja la planificación temporal.
%       \item En el siguiente capítulo se describen las tecnologías utilizadas en el desarrollo de este TFM/TFG (Capítulo~\ref{chap:tecnologias}).
%       \item En el capítulo~\ref{chap:desing} Se describe el proceso de desarrollo
%       de la herramienta \ldots
%       \item En el capítulo~\ref{chap:experimentos} Se presentan las principales pruebas realizadas
%       para validación de la plataforma/herramienta\ldots (o resultados de los experimentos
%       efectuados).
%       \item Por último, se presentan las conclusiones del proyecto así como los trabajos futuros que podrían derivarse de éste (Capítulo~\ref{chap:conclusiones}).
%     \end{itemize}

\cleardoublepage
\chapter{State of the art}
\label{chap:sota}
\section{Literature review}
\begin{enumerate}

\item Ref~\cite{Mao2017345} Indoor Follow Me Drone
Tracking through acoustic signals in indoor environments produced by speakers on the drone and received by a mobile device. Discards CV for stability and processing power. Controller on mobile device with three degrees of freedom, pitch and yaw through MPC and roll through PID, with prediction of target's movement.
\begin{itemize}
\item Identification and path following control of an AR.Drone quadrotor \href{https://www.scopus.com/record/display.uri?eid=2-s2.0-84893212045&origin=reflist}{--Link--} \\ 
Path following application based on IMC position controllers, external webcam video stream.

\item Tracking a ground moving target with a quadrotor using switching control: Nonlinear modeling and control \href{https://www.scopus.com/record/display.uri?eid=2-s2.0-84871633622&origin=reflist}{--Link--} \\ 
Tracking of a moving target on ground, embedded camera, use of 2-dimensional images to compute the relative 3-dimensional position and translational velocity of the UAV with respect to the target, switching controllers.

\item A computer vision and control algorithm to follow a human target in a generic environment using a drone \href{https://www.scopus.com/record/display.uri?eid=2-s2.0-84978870802&origin=reflist}{--Link--} \\ 
Tracking and following a generic human target, using the HOG classifier, and on local brightness information, using the optical flow algorithm.

\item Framework for autonomous on-board navigation with the AR.Drone \href{https://www.scopus.com/record/display.uri?eid=2-s2.0-84899426060&origin=reflist}{--Link--} \\ 
All sensing and computations on-board, three systems to autonomously following several trajectory patterns, visually estimate its position and detecting and following a person.

\item A robust real-time embedded vision system on an unmanned rotorcraft for ground target following \href{https://www.scopus.com/record/display.uri?eid=2-s2.0-80054803865&origin=reflist}{--Link--} \\

\item UAV path following for constant line-of-sight \href{https://www.scopus.com/record/display.uri?eid=2-s2.0-85088181184&origin=reflist}{--Link--} \\ 
Flight path guidance and synchronous camera angles to observe a target, analytic expressions are derived for trajectories required for constant line-of-sight orientation relative to the aircraft.

\item Reactive control of autonomous drones \href{https://www.scopus.com/record/display.uri?eid=2-s2.0-84979920485&origin=reflist}{--Link--} \\ 
Reactive control that supersedes the time-triggered approach, control decisions are taken only upon recognizing the need to, based on observed changes in the navigation sensors, rate of execution dynamically adapts to the circumstances.

\item Correlation filter based visual trackers for person pursuit using a low-cost Quadrotor \href{https://www.scopus.com/record/display.uri?eid=2-s2.0-84954554999&origin=reflist}{--Link--} \\ 
Correlation filters for short-term tracking and a redetection strategy based on tracking-learning-detection (TLD), flight experiments in unconstrained environments using human targets and an existing visual servoing controller.

\item The Navigation and Control technology inside the AR.Drone micro UAV \href{https://www.scopus.com/record/display.uri?eid=2-s2.0-84863704626&origin=reflist}{--Link--} \\ 
Navigation and Control technology embedded in a recently commercialized micro Unmanned Aerial Vehicle (UAV), the AR.Drone.

\item Vector field path following for miniature air vehicles \href{https://www.scopus.com/record/display.uri?eid=2-s2.0-34447327237&origin=reflist}{--Link--} \\ 
Method for accurate path following for miniature air vehicles is developed, vector-field path-following control laws are developed for straight-line paths and circular arcs and orbits.

\item Trajectory tracking for autonomous vehicles: An integrated approach to guidance and control \href{https://www.scopus.com/record/display.uri?eid=2-s2.0-0031673631&origin=reflist}{--Link--} \\

\item Trajectory-tracking and path-following of underactuated autonomous vehicles with parametric modeling uncertainty \href{https://www.scopus.com/record/display.uri?eid=2-s2.0-34548237452&origin=reflist}{--Link--} \\

\item Combined trajectory tracking and path following: An application to the coordinated control of autonomous marine craft \href{https://www.scopus.com/record/display.uri?eid=2-s2.0-0035713115&origin=reflist}{--Link--} \\ 
Good trajectory tracking performance while keeping some of the desired properties normally associated with path following

\item Understanding the basis of the kalman filter via a simple and intuitive derivation [lecture notes] \href{https://www.scopus.com/record/display.uri?eid=2-s2.0-85032780920&origin=reflist}{--Link--} \\
\end{itemize}

\item Ref~\cite{Pestana20141886} Computer Vision Based General Object Following for GPS-denied Multirotor Unmanned Vehicles, Ref~\cite{Pestana2013} (6) Vision based GPS-denied Object Tracking and Following for Unmanned Aerial Vehicles
Object tracker + IBVS controllers

\item Ref~\cite{Chakrabarty201625} Autonomous Indoor Object Tracking with the Parrot AR.Drone
Deformable Object
Tracking (CMT) tracker with image-based visual servoing (IBVS), relies directly on image features to compute control values as opposed to position-based servoing

\item Ref~\cite{Bartak201635} Any Object Tracking and Following by a Flying Drone
 Offboard control solution though WiFi on Parrot AR Drone with computer vision (Training-Learning-Detection), with two PID controllers for forward/backward (scale estimation) and yaw (center position).

\end{enumerate}
\newpage
\section{Technologies employed}

\subsection{Software libraries}

\subsubsection{PX4 autopilot}
\label{subsec:px4}
PX4 \footnote{\url{https://docs.px4.io/main/en/}} is a professional open-source autopilot flight stack developed in C++ by developers from industry and academia, and supported by an active world-wide community,
it powers all kinds of vehicles from racing and cargo drones through to ground vehicles and submersibles.
The flight stack software runs on a vehicle controller or flight controller hardware. It supports both Ready To Fly vehicles and custom builds made from scratch,
as well as many additional kinds of sensors and peripherals, such as distance and obstacle sensors, GPS, camera payloads and onboard computers.

PX4 is a core part of a broader drone platform, the Dronecode Project \footnote{\url{https://www.dronecode.org/}}, that includes the QGroundControl ground station \todo{cite}, Pixhawk hardware,
and MAVSDK for integration with companion computers, cameras and other hardware using the MAVLink protocol.
PX4 was initially designed to run on Pixhawk Series controllers, but can now run on Linux computers and other hardware.
The software controls the vehicles through flight modes. 
Flight modes define how the autopilot responds to remote control input, and how it manages vehicle movement during fully autonomous flight.
The modes provide different types or levels of autopilot assistance to the user, ranging from automation of common tasks like takeoff and landing, 
to mechanisms that make it easier to regain level flight or hold the vehicle to a fixed path or position.

\subsubsection{MAVLink and MavSDK}
\label{subsec:mavlink}
MAVLink \footnote{\url{https://mavlink.io/en/}} is a very lightweight messaging protocol for communicating with drones and between onboard drone components. 
It follows a modern hybrid publish-subscribe and point-to-point design pattern 
where data streams are published as topics while configuration sub-protocols 
such as the mission protocol or parameter protocol are sent as point-to-point with retransmission. 
Messages are defined within XML files. 
Each XML file defines the message set supported by a particular MAVLink system.

MAVSDK \footnote{\url{https://mavsdk.mavlink.io/main/en/index.html}} is a collection of libraries for various programming languages,
to interface with MAVLink systems such as drones, cameras or ground systems.
It is primarly written in C++ with wrappers available for,
among others, Swift, Python and Java.
The libraries provides a simple API for managing one or more vehicles, 
providing programmatic access to vehicle information and telemetry, 
and control over missions, movement and other operations.
The libraries can be used onboard a drone on a companion computer
or on the ground for a ground station or mobile device.
MAVSDK is cross-platform: Linux, macOS, Windows, Android and iOS.


\subsubsection{AirSim}
\label{subsec:airsim}
AirSim \footnote{\url{https://microsoft.github.io/AirSim/}} is a simulator for drones, cars and more, built on Unreal Engine and developed by Microsoft. It is open-source, cross platform, and supports software-in-the-loop \gls{sitl} simulation with popular flight controllers such as PX4 and ArduPilot and hardware-in-loop \gls{hitl} with PX4 for physically and visually realistic simulations. It is developed as an Unreal plugin that can simply be dropped into any Unreal environment.

Its goal is to develop a platform for AI research to experiment with deep learning, computer vision and reinforcement learning algorithms for autonomous vehicles. For this purpose, AirSim also exposes APIs to retrieve data and control vehicles in a platform independent way.

\subsubsection{MediaPipe}
\label{subsec:mediapipe}
Mediapipe \footnote{\url{https://google.github.io/mediapipe/}} is an open-source project developed by Google that offers cross-platform, customizable machine learning solutions for live and streaming media.
It supports End-to-End acceleration with built-in fast ML inference and processing accelerated even on common hardware and a unified solution that works across Android, iOS, desktop/cloud, web and IoT.
It offers a framework designed specifically for complex perception pipelines, like real-time perception of human pose, face landmarks and hand tracking that can enable a variety of impactful applications, such as fitness and sport analysis, gesture control and sign language recognition, augmented reality effects and more. 


\subsection{Hardware employed}
\subsubsection{Holybro X500 + Pixhawk 4}
\label{subsec:pixhawk}
The Holybro X500 \footnote{\url{https://docs.px4.io/main/en/frames_multicopter/holybro_x500_pixhawk4.html}} kit is composed of a full carbon-fiber twill frame and the Pixhawk 4 flight controller, 
an advanced autopilot designed and made in collaboration with Holybro and the PX4 team.
It is based on the Pixhawk-project \footnote{\url{https://pixhawk.org/}} FMUv5 open hardware design and it is optimized to run PX4 on the NuttX OS \footnote{\url{https://nuttx.apache.org/}}.
The Pixhawk 4 has an integrated accelerometer/gyroscope, a magnetomer and a barometer.
The kit also includes a power management board, motors an GPS module, an RC receiver and a telemetry radio.
\todo[inline]{More about this in validation (complete build)}


\subsubsection{Raspberry Pi 4}
\label{subsec:rpi}
The Raspberry Pi \footnote{\url{https://www.raspberrypi.com/products/raspberry-pi-4-model-b/}} is a line of single-board computers that stands out thanks to its affordable price,
compact size and maker-friendly design. The model 4B is an improved version of its predecessors,
getting a major increase in processing power, enhanced video output and peripheral connectivity,
while maintaining the same low price and tiny size offered on past models.
This small computer comes as a bare circuit board,
without any sort of housing or add-ons such as a cooling fan or a power button,
but it includes USB, HDMI and Ethernet ports and both Wi-Fi and Bluetooth connectivity,
as well as a 40-pin GPIO \gls{gpio} header, a row of input/output pins that provides direct access for connecting external devices.
The Raspberry Pi runs natively Raspbian OS, a free operating system based on Debian optimized for the Pi hardware, but it is compatible with other standard flavours of Linux.

\subsubsection{Real Sense T265}
\url{https://www.intelrealsense.com/tracking-camera-t265/}

\cleardoublepage