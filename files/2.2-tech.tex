\subsection{Methodology}

\todo[inline]{Cite pictures}

\subsubsection{Software}

\subsubsection{PX4 autopilot}
\label{subsec:px4}
PX4 \footnote{\url{https://docs.px4.io/main/en/}} is a professional open-source autopilot flight stack developed in C++ by developers from industry and academia, and supported by an active world-wide community,
it powers all kinds of vehicles from racing and cargo drones through to ground vehicles and submersibles.
The flight stack software runs on a vehicle controller or flight controller hardware. It supports both Ready To Fly vehicles and custom builds made from scratch,
as well as many additional kinds of sensors and peripherals, such as distance and obstacle sensors, GPS, camera payloads and onboard computers.

PX4 is a core part of a broader drone platform, the Dronecode Project \footnote{\url{https://www.dronecode.org/}}, that includes the QGroundControl ground station \todo{cite}, Pixhawk hardware,
and MAVSDK for integration with companion computers, cameras and other hardware using the MAVLink protocol.
PX4 was initially designed to run on Pixhawk Series controllers, but can now run on Linux computers and other hardware.
The software controls the vehicles through flight modes. 
Flight modes define how the autopilot responds to remote control input, and how it manages vehicle movement during fully autonomous flight.
The modes provide different types or levels of autopilot assistance to the user, ranging from automation of common tasks like takeoff and landing, 
to mechanisms that make it easier to regain level flight or hold the vehicle to a fixed path or position.

\subsubsection{MAVLink and MavSDK}
\label{subsec:mavlink}
MAVLink \footnote{\url{https://mavlink.io/en/}} is a very lightweight messaging protocol for communicating with drones and between onboard drone components. 
It follows a modern hybrid publish-subscribe and point-to-point design pattern 
where data streams are published as topics while configuration sub-protocols 
such as the mission protocol or parameter protocol are sent as point-to-point with retransmission. 
Messages are defined within XML files. 
Each XML file defines the message set supported by a particular MAVLink system.

MAVSDK \footnote{\url{https://mavsdk.mavlink.io/main/en/index.html}} is a collection of libraries for various programming languages,
to interface with MAVLink systems such as drones, cameras or ground systems.
It is primarly written in C++ with wrappers available for,
among others, Swift, Python and Java.
The Python wrapper is based on a gRPC (Google Remote Procedure Call) client communicating with the gRPC server written in C++.
The libraries provides a simple API for managing one or more vehicles, 
providing programmatic access to vehicle information and telemetry, 
and control over missions, movement and other operations.
The libraries can be used onboard a drone on a companion computer
or on the ground for a ground station or mobile device.
MAVSDK is cross-platform: Linux, macOS, Windows, Android and iOS.
\todo[inline]{why not ROS}


\subsubsection{QGroundControl}
\label{subsec:qgc}


\subsubsection{Unreal Engine}
\label{subsec:unreal}


\subsubsection{AirSim}
\label{subsec:airsim}
AirSim \footnote{\url{https://microsoft.github.io/AirSim/}} is a simulator for drones, cars and more, built on Unreal Engine and developed by Microsoft. It is open-source, cross platform, and supports software-in-the-loop \gls{sitl} simulation with popular flight controllers such as PX4 and ArduPilot and hardware-in-loop \gls{hitl} with PX4 for physically and visually realistic simulations. It is developed as an Unreal plugin that can simply be dropped into any Unreal environment.

Its goal is to develop a platform for AI research to experiment with deep learning, computer vision and reinforcement learning algorithms for autonomous vehicles. For this purpose, AirSim also exposes APIs to retrieve data and control vehicles in a platform independent way.

\subsubsection{MediaPipe}
\label{subsec:mediapipe}
Mediapipe \footnote{\url{https://google.github.io/mediapipe/}} is an open-source project developed by Google that offers cross-platform, customizable machine learning solutions for live and streaming media.
It supports End-to-End acceleration with built-in fast ML inference and processing accelerated even on common hardware and a unified solution that works across Android, iOS, desktop/cloud, web and IoT.
It offers a framework designed specifically for complex perception pipelines, like real-time perception of human pose, face landmarks and hand tracking that can enable a variety of impactful applications, such as fitness and sport analysis, gesture control and sign language recognition, augmented reality effects and more. 
\todo[inline]{Explain hand and pose solutions}


\subsubsection{GitHub}
\label{subsec:github}


\subsubsection{PID controller}
\label{subsec:pid-theory}



\subsubsection{Hardware}
\subsubsection{Holybro X500}
\label{subsec:x500}
The Holybro X500 \footnote{\url{https://docs.px4.io/main/en/frames_multicopter/holybro_x500_pixhawk4.html}} kit is composed of a full carbon-fiber twill frame and the Pixhawk 4 flight controller.
The kit also includes a power management board, motors an GPS module, an RC receiver and a telemetry radio.
\todo[inline]{More about this in validation (complete build)}


\subsubsection{Pixhawk 4}
\label{subsec:pixhawk}
an advanced autopilot designed and made in collaboration with Holybro and the PX4 team.
It is based on the Pixhawk-project \footnote{\url{https://pixhawk.org/}} FMUv5 open hardware design and it is optimized to run PX4 on the NuttX OS \footnote{\url{https://nuttx.apache.org/}}.
The Pixhawk 4 has an integrated accelerometer/gyroscope, a magnetomer and a barometer.


\subsubsection{Raspberry Pi 4}
\label{subsec:rpi}
The Raspberry Pi \footnote{\url{https://www.raspberrypi.com/products/raspberry-pi-4-model-b/}} is a line of single-board computers that stands out thanks to its affordable price,
compact size and maker-friendly design. The model 4B is an improved version of its predecessors,
getting a major increase in processing power, enhanced video output and peripheral connectivity,
while maintaining the same low price and tiny size offered on past models.
This small computer comes as a bare circuit board,
without any sort of housing or add-ons such as a cooling fan or a power button,
but it includes USB, HDMI and Ethernet ports and both Wi-Fi and Bluetooth connectivity,
as well as a 40-pin GPIO \gls{gpio} header, a row of input/output pins that provides direct access for connecting external devices.
The Raspberry Pi runs natively Raspbian OS, a free operating system based on Debian optimized for the Pi hardware, but it is compatible with other standard flavours of Linux.
