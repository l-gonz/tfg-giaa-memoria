\section{Technologies}

\subsection{Software libraries}
\subsubsection{PX4 autopilot}
PX4 \todo{cite} is the a professional open-source autopilot flight stack developed in C++ by developers from industry and academia, and supported by an active world-wide community,
it powers all kinds of vehicles from racing and cargo drones through to ground vehicles and submersibles.
The flight stack software runs on a vehicle controller or flight controller hardware. It supports both Ready To Fly vehicles and custom builds made from scratch,
as well as many additional kinds of sensors and peripherals, such as distance and obstacle sensors, GPS, camera payloads and onboard computers.
PX4 is a core part of a broader drone platform, the Dronecode Project \todo{cite}, that includes the QGroundControl ground station \todo{cite}, Pixhawk hardware,
and MAVSDK for integration with companion computers, cameras and other hardware using the MAVLink protocol.
PX4 was initially designed to run on Pixhawk Series controllers, but can now run on Linux computers and other hardware.
The software controls the vehicles through flight modes. 
Flight modes define how the autopilot responds to remote control input, and how it manages vehicle movement during fully autonomous flight.
The modes provide different types or levels of autopilot assistance to the user, ranging from automation of common tasks like takeoff and landing, 
to mechanisms that make it easier to regain level flight or hold the vehicle to a fixed path or position.

\subsubsection{MavSDK}


\subsubsection{AirSim}


\subsubsection{MediaPipe}



\subsection{Hardware employed}
\subsubsection{Holybro X500 + Pixhawk 4}
\url{https://docs.px4.io/main/en/frames_multicopter/holybro_x500_pixhawk4.html}

PX4 uses sensors to determine vehicle state (needed for stabilization and to enable autonomous control).
It minimally requires a gyroscope, accelerometer, magnetometer (compass) and barometer.
A GPS or other positioning system is needed to enable all automatic flight modes, and some assisted ones.
PX4 uses outputs to control motor speed, flight surfaces like ailerons and flaps, camera triggers, parachutes, grippers, and many other types of payloads.
Many PX4 drones use brushless motors that are driven by the flight controller via an Electronic Speed Controller (ESC) 
(the ESC converts a signal from the flight controller to an appropriate level of power delivered to the motor).
PX4 drones are mostly commonly powered from Lithium-Polymer (LiPo) batteries. 
The battery is typically connected to the system using a Power Module or Power Management Board, 
which provide separate power for the flight controller and to the ESCs (for the motors).
A Radio Control (RC) system is used to manually control the vehicle. 
It consists of a remote control unit that uses a transmitter to communicate stick/control positions with a receiver based on the vehicle. 
Some RC systems can additionally receive telemetry information back from the autopilot.
Telemetry Radios can provide a wireless MAVLink connection between a ground control station like QGroundControl and a vehicle running PX4. 
This makes it possible to tune parameters while a vehicle is in flight, inspect telemetry in real-time, change a mission on the fly, etc


\subsubsection{Raspberry Pi 4}
\subsubsection{Real Sense T265}