\section{Flight tests}

1. Recreate hand solution flight test and record
2. Configure setup Pihawx + Raspberry Pi + Camera
3. Test setup with HITL
4. Test setup with very simple real flight: takeoff, take picture, land

Stretch goals

1. Test tracking solution during flight: takeoff, record video, land, then run detection on recorded video
2. Test follow algorithm (velocity controller)
%3. Test ground station: receive images from drone (on Android ????)
4. Test follow algorithm (position controller) -- very unlikely



Development needed for that:

1.3 - Add test option to program to run on HITL

2.1 - 200 OK
2.2 - Stabilize PID controller (use tool to get better parameters ???)
    - Fix problems when losing sight (graph errors)
    - Make better recovery when landmarks too different from previous
2.3 - (Send video real-time to ground control ?????)
2.4 - Make 2D to 3D calculation
    - Calculate target drone position from followed target position
    - Add new PID position controller based on target drone position
    
    
    
    
    ---------------------------
%%%%%%%%%%%%%%%%%%%%%%%%%%%%%%%%%%%%%%%%%%%%%%%%
%%%%%%%%%% Onsite meeting %%%%%%%%%%%%%%%%%%%%%%
%%%%%%%%%%%%%%%%%%%%%%%%%%%%%%%%%%%%%%%%%%%%%%%%
    
1. Connect setup Pihawx + Raspberry Pi + Camera
    a. Config rasperry pi
        
        1. Build Raspberry Pi \url{https://www.okdo.com/getstarted/}
        2. Install OS on SD card with Raspberry Pi Imager - Raspberry Pi OS 64 bits + Desktop 
        3. (Optional) Use Raspberry Pi Imager to setup login user, ssh and network settings
        4. Clone repo
        5. Create venv (need to install at least python3.9-dev & python3.9-venv)
        6. Install requirements (pip install -r requirements.txt) 
        7. Install own package (pip install . -e)
        
        OPTIONAL: Setup XRDP to connect from windows PC (no need for screen, keyboard and mouse)
        \url{https://linuxize.com/post/how-to-install-xrdp-on-raspberry-pi/}
        
    b. connect cam to pi -> get images / video (Setup realsense camera to work on RPi)
        
        Modified from:
        \url{https://github.com/IntelRealSense/librealsense/blob/master/doc/installation_raspbian.md}
        
        Step changes:
        1. Check versions: install cmake and gcc from apt repo (cmake/stable 3.18.4-2+rpt1+rpi1 arm64, gcc/stable, 4:10.2.1-1 arm64)
        2. Add swap
        3. Install packages: ignore not found packages (*-dbgsym libglui-dev libglui2c2
        4. Update udev rule
        5. Update cmake version --> SKIP
        6. Set path: set on ~/.bashrc
        7. Install protobuf: install from apt repo (protobuf-compiler/stable, 3.12.4-1 arm64)
        8. Install TBB: install from apt repo (libtbb-dev/stable, 2020.3-1 arm64)
        9. Install OpenCV --> SKIP (if already setup environment)
        10, 11. Install RealSense SDK/librealsense & install pyrealsense2: keep step 10 only but use this command (cmake ../ -DFORCE_RSUSB_BACKEND=true -DBUILD_PYTHON_BINDINGS=bool:true -DCMAKE_BUILD_TYPE=release -DBUILD_EXAMPLES=true -DBUILD_GRAPHICAL_EXAMPLES=true  -DBUILD_WITH_TM2=true), then (export PYTHONPATH=\$PYTHONPATH:/usr/local/lib) on ~/.bashrc
        12. Change pi settings (enable OpenGL) --> SKIP (pyopengl and pyopengl_accelerate should be on pip requirements, GL driver is enabled by default)
        13. EXTRA STEP: Copy librealsense.so and wrappers/python/pyrealsense2.so from build folder to project folder
        13. REBOOT
        
        NOTES: Using adapted stereo algorithm from example code from RealSense to transform fisheye image to flat undistorted.
        
        TEST: Use realsense-viewer and other utils located at /usr/local/bin

    c-1. Test HITL simulation with Airsim    
        1. Build drone (\url{https://docs.px4.io/master/en/frames_multicopter/holybro_x500_pixhawk4.html})
        2. Configure and calibrate autopilot (\url{https://docs.px4.io/master/en/config/})
            a. Install PX4 Pro Stable Release v1.12.3
            b. RC config according to PDF -> AUX Passthrough channels for controlling camera??
            c. HITL config for PX4 \url{https://docs.px4.io/master/en/simulation/hitl.html#hardware-in-the-loop-simulation-hitl}
                - Activate HIL on QGroundControl and deactivate autoconnect (optional)
                - Set up RC with arm switch, can't directly arm from AirSim
            d. HITL config for AirSim \url{https://microsoft.github.io/AirSim/px4_setup/}
            -----> Works by default with direct USB connection to PX4, telemetry radio needs serial port config)
            -----> add standard tcp and port config to get airsim images from inside wsl
                ("UseTcp": true,
                "TcpPort": 4560,
                "ControlIp": "remote",
                "ControlPortLocal": 14540,
                "ControlPortRemote": 14580,
                "LocalHostIp": "<WSL network Windows host IP>",
                "SerialPort": "<COM10>",
                "SerialBaudrate": <baudrate>)
                
            Connection over cable to PX4 -> baudrate 115200
            Connection over telemetry radio -> baudrate 57600
            Connection over serial -> baudrate 921600
            
            e. Final configuration:
                - Pixhawk <-> AirSim:  connect preferably over USB cable, with telemetry it hangs and AirSim freezes
                - AirSim <-> Dronecontrol: python on Windows, connects through localhost (no ip on airsim source constructor, providing WSL ip works too)
                - Dronecontrol <-> Pixhawk: connect over telemetry radio, set pilot.system address to telemetry port (COM10) and appropriate baudrate
                
            NOTES: 
            Starting RC controller with receiver plugged to PX4 automatically connect to simulator, and simulated drone can be flown with controller.
            If no offboard API attached trough telemetry radio (MavSDK, dronecontrol) can use QGroundControl to see GPS location, speeds, logs and send mode changes.
        
    c. connect px4 to pi -> test simple commands work (take-off/land) - HITL (processing in the loop)
    
        1. Three way connection: \url{https://www.etechnophiles.com/raspberry-pi-4-gpio-pinout-specifications-and-schematic/#raspberry-pi-4-gpio-pinout} \url{https://docs.px4.io/master/en/companion_computer/pixhawk_companion.html#companion-computer-for-pixhawk-series}
            - Pixhawk <-> AirSim: over USB cable, auto-connect with no config needed
            - AirSim <-> Dronecontrol: connect through Wi-Fi on local network (WLAN), need to                                 specify host computer (Windows) IP on AirSim constructor
            - Dronecontrol <-> Pixhawk: for on flight, connect through TELEM2 port on PX4 to UART pins on RPi (TELEM2 left to right cables to RPi pins, cable 2 TX to pin 10 RX, cable 3 RX to pin 8 TX, cable 6 GND to pin 6 GND ((if powered through pins at the same time, more GND on pin 9))) Activate second MAV channel through parameters
            
        2. Power the RPi while in flight: \url{http://www.holybro.com/manual/PM07-Quick-Start-Guide.pdf} \url{https://github.com/raspberrypi/hats/blob/master/designguide.md#back-powering-the-pi-via-the-gpio-header}
            - 5v pin on TELEM2 port not enough current to power RPi consistently
            - 5v pins on power module not activated by default
            - POWER2 port provides good supply through custom adapter to RPi pins 4 (5V) and 6 (GND)
            
        A guide to set up UART serial connection in RPi: \url{https://discuss.px4.io/t/talking-to-a-px4-fmu-with-a-rpi-via-serial-nousb/14119?page=2}
    
        ???? (verify that sensors can be simulated for HITL, flight mechanics on px4, sensors from airsim) -> Tested with first HITL simulation without RPi - OK
        
        ???? (check if groundcontrol can get information from rp4, can logs be sent, just get info if it's possible) -> If serial connection between RPi and PX4, telemetry radio can be use on an offboard computer with QGroundControl
        
        ???? (check if groundcontrol can tell px4 to not listen to rpi port,
        maybe: program listens for ground control to change to offboard mode, it doesn't start control until then, if offboard is stopped then control loop stops, detection continues)
        -> modify code to handle mode changes manually
        
2. Simulate dronecontrol follow in raspberry pi
    -> Verify powering of raspberry pi from telemetry connection to Piwhakx - NOT OK, see 1.c.2.
    -> Goal is to check if RP4 can handle the computations required for detection and control with decent frame rate
    Resolution 1280x800:
        - 0-1 FPS on dronecontrol running on RPi with AirSim for image and flight engine (NO CONTROL)
3. Introduce real camera to detection algorithm
    -> Check how real sense image performs, dronecontrol running on RPi with RealSense for image and AirSim for flight engine
    -------- write up table with average loop time for different combinations  ---------
    ????? Tool in code to auto calculate ?????
    Resolution 300x300:
        - 18-23 FPS no control, no detection
        - 4-7 FPS no control, pose detection
        - 1-2 FPS pose detection, follow control
    Resolution 600x600:
        - 14-16 FPS no control, no detection
        - 3-4 FPS no control, pose detection
        - 1-2 FPS pose detection, follow control
    Resolution 800x800:
        - 10-13 FPS no control, no detection
        - 3-4 FPS no control, pose detection
        - 1 FPS pose detection, follow control
    -> FLY THE DRONE to record images only, no control
4. Safety measures: 
    -> react to loss of inputs from detections (FIX stop moving if loss)
    -> check for coherence of inputs to prevent recognizing wrong figure
    -> add config (parameters) and minimal checks to prevent bad inputs from dronecontrol controller to px4 (angular velocity max, hold on lost connection) 
            - COM_OBL_ACT/COM_OBL_RC_ACT -> Hold on lost offboard link while offboard
            - COM_RCL_EXCEPT -> Hold on offboard active without RC (disabled in sim)
            - NAV_RCL_ACT -> Return on RC loss (disabled in sim)
    -> detect if failsafe engages in-code (add tests to check that config safety checks are engaged)
    \url{https://docs.px4.io/master/en/config/safety.html}
    -----------------------------
    
FLIGHT TEST PLAN
1. Test assembly with RC only
2. Test config with QGroundControl on computer (telemetry)
3. Test data link with hand solution (dronecontrol on Windows, telemetry and usb webcam to pc)
4. Add RPi + RS camera onboard recording, control with RC only, start dronecontrol through ssh
5. Same setup, add QGroundControl through telemetry on pc
6. Run follow, control with offboard switch on RC
EXTRA: Test hand solution from onboard camera
------------------------    TEST 1   ----------------------------------
Test RC and ground control flight OK
Camera connection fails every time the RPi is restarted, but works afterwards
Drone will not connect to RPi through serial, telemetry radio works but too slow (is it the connection or the computer that is slow ????)
RPi seems to run slower with battery and often warns low voltage
!!! FIX !!! : After changing Airframe (from HITL to normal and back) telemetry port parameters reset, need to be: 
MAV_1_CONFIG = 102 (TELEM2)
MAV_1_MODE = Onboard (Companion computer)
SER_TEL1_BAUD = 921600 (Or use different on connecting through mavsdk, maybe)
------------------------    TEST 2   ----------------------------------
Next flight test:
1. Run test camera without drone connection, fly with ground control, record video
2. Try hand solution
- dronecontrol tools test-camera -r /dev/serial0:921600 -rs -p   ---->>> TEST
- dronecontrol tools test-camera -r /dev/serial0:921600 -rs      ---->>> RECORD
- dronecontrol hand --serial COM7:57600 (Windows)                ---->>> TEST FLIGHT
- dronecontrol follow -s /dev/serial0:921600 -rs                 ---->>> TEST FLIGHT
----
Camera draws all the power, low voltage triggers when the camera connection starts
If the camera is  not transmiting images  (black screen) everything is super fast everything works
----
Image showing starts low voltage warning:
- Is processing faster with RPi own power supply
- Is there low voltage warning with smaller image?
Video recorded by Real Sense doesn't detect person
- Is it too far?
- Does it improve with a different camera?
- 


Next steps:
 - Write email with latest test results. offboard control, resolutions and control fps
 XXXXXX Fix safety measures (step 4)
 XXXXXX Write flight plan with tests
 XXXXXX Add pose coherence detection
 XXXXXX Calibrate yaw PID
 - Execute flight plan
 - Fix documentation
 - Add tests for all code
 - Fix CLI and standardize 
 !- Try decouple process_image and offboard_control in different threads
 
Later:
 XXXXXX Fix PID
 - Move to position control
 
 Smaller thing:
  - Get stereo algorithm out of video source and document
  - Write headers on files ???
  
  
  
  
  
MEETING 06.09.22



  