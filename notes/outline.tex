OUTLINE
1. Introduction
 	- Project goals
	- Timeline -> (Solution development, simulation testing, hardware setup, writing) how it's distributed in time
	- Structure
2. State-of-the-art
	- Drone + object-tracking research
	- Software, libraries used on the project
		- PX4
		- AirSim
		- MediaPipe
		- MavSDK, OpenCV, …
	- Hardware
	    - Pixhawk
	    - Raspberry Pi
	    - Real Sense T265
3. Design and implementation
	- Diagram of architecture (high-level view of all systems)
	- Development environment (Ubuntu, WSL - Windows, RaspberryPi)
	- Hardware-software interaction (hardware setup, connections between flight controller, camera and companion computer, OS)
	- Hand Solution
	- Follow Solution
	     - 1. Translate 2D coordinates to 3D world position:
	           %https://stackoverflow.com/questions/12299870/computing-x-y-coordinate-3d-from-image-point
	           %https://docs.opencv.org/2.4/modules/calib3d/doc/camera_calibration_and_3d_reconstruction.html
	     - 2. Path following
	     - 3. Control loop with PID
4. Experiments and validation
	- Individual module testing (how every part of the system is validated separately)
	- Simulation setup and results from simulation (SITL, HITL, Jmavsim vs. Airsim, Airsim environments)
	- Flight tests and results on real-time deployment onboard (hand solution and follow solution)
5. Conclusions and future work
	- Achievement of proposed goals (from 1a)
	- Application of bachelor curriculum
	- Lessons learned
	- Future work

APPENDIX
	- CLI interface
	- Full installation process (reproduction steps)